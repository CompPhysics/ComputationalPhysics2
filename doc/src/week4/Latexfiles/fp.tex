\documentclass{beamer}

\usetheme{Madrid}
\usepackage{amsmath,amssymb,physics,bm}

\title{Green's Function Solution of the Fokker--Planck Equation}
\subtitle{Forward Equation in Higher Dimensions}
\author{MHJ}
\date{}

\begin{document}

%-------------------------------------------------
\begin{frame}
\titlepage
\end{frame}

%=================================================
\section{Fokker--Planck Equation}

\begin{frame}{Forward Fokker--Planck Equation}
Let $p(\bm{x},t)$ be the probability density of a diffusion process
$\bm{x}\in\mathbb{R}^d$.

\vspace{0.3cm}

The forward Fokker--Planck equation reads
\[
\partial_t p(\bm{x},t)
=
-\sum_{i=1}^d \partial_{x_i}
\big[A_i(\bm{x},t)\,p(\bm{x},t)\big]
+
\sum_{i,j=1}^d
\partial_{x_i}\partial_{x_j}
\big[D_{ij}(\bm{x},t)\,p(\bm{x},t)\big],
\]
where
\begin{itemize}
  \item $\bm{A}(\bm{x},t)$ is the drift vector,
  \item $D_{ij}(\bm{x},t)$ is the diffusion tensor.
\end{itemize}
\end{frame}

%-------------------------------------------------
\begin{frame}{Operator form}
Define the Fokker--Planck operator
\[
L^\dagger
=
-\sum_i \partial_{x_i} A_i(\bm{x},t)
+
\sum_{i,j} \partial_{x_i}\partial_{x_j} D_{ij}(\bm{x},t).
\]

Then the equation becomes
\[
\partial_t p(\bm{x},t) = L^\dagger p(\bm{x},t).
\]

\vspace{0.3cm}

This is a linear parabolic partial differential equation.
\end{frame}

%=================================================
\section{Green's function}

\begin{frame}{Definition of the Green's function}
The Green's function (fundamental solution)
$G(\bm{x},t\mid\bm{x}_0,t_0)$ is defined by
\[
\partial_t G = L^\dagger_{\bm{x}}\,G,
\qquad
G(\bm{x},t_0\mid\bm{x}_0,t_0)
=
\delta(\bm{x}-\bm{x}_0).
\]

\vspace{0.3cm}

Physical interpretation:
\begin{itemize}
  \item $G$ is the transition probability density,
  \item probability to go from $\bm{x}_0$ at $t_0$ to $\bm{x}$ at $t$.
\end{itemize}
\end{frame}

%-------------------------------------------------
\begin{frame}{Superposition principle}
Because the Fokker--Planck equation is linear,
the solution for a general initial condition
$p(\bm{x},t_0)$ is
\[
p(\bm{x},t)
=
\int d^d\bm{x}_0\;
G(\bm{x},t\mid\bm{x}_0,t_0)\,
p(\bm{x}_0,t_0).
\]

\vspace{0.3cm}

Thus the Green's function completely characterizes
the time evolution.
\end{frame}

%=================================================
\section{Derivation via Chapman--Kolmogorov}

\begin{frame}{Chapman--Kolmogorov equation}
Markov processes satisfy
\[
G(\bm{x},t+\Delta t\mid\bm{x}_0,t_0)
=
\int d^d\bm{y}\;
G(\bm{x},t+\Delta t\mid\bm{y},t)\,
G(\bm{y},t\mid\bm{x}_0,t_0).
\]

This identity encodes the semigroup property
of the propagator.
\end{frame}

%-------------------------------------------------
\begin{frame}{Short-time expansion}
For small $\Delta t$,
\[
G(\bm{x},t+\Delta t\mid\bm{y},t)
=
\delta(\bm{x}-\bm{y})
+
\Delta t\,L^\dagger_{\bm{y}}
\delta(\bm{x}-\bm{y})
+
\mathcal{O}(\Delta t^2).
\]

Insert into the Chapman--Kolmogorov equation
and expand to first order in $\Delta t$.
\end{frame}

%-------------------------------------------------
\begin{frame}{Emergence of the Fokker--Planck equation}
After integration by parts and taking
$\Delta t \to 0$, one finds
\[
\partial_t G(\bm{x},t\mid\bm{x}_0,t_0)
=
L^\dagger_{\bm{x}}\,G(\bm{x},t\mid\bm{x}_0,t_0),
\]
which is precisely the forward Fokker--Planck equation
with a delta-function initial condition.

\vspace{0.3cm}

This establishes $G$ as the fundamental solution.
\end{frame}

%=================================================
\section{Formal solution and properties}

\begin{frame}{Formal operator solution}
Formally, the Green's function can be written as
\[
G(\bm{x},t\mid\bm{x}_0,t_0)
=
\exp\!\big[(t-t_0)L^\dagger\big]\,
\delta(\bm{x}-\bm{x}_0).
\]

\vspace{0.3cm}

For constant drift and diffusion,
this reduces to a Gaussian kernel;
for general coefficients, it defines a semigroup.
\end{frame}

%-------------------------------------------------
\begin{frame}{Path-integral / stochastic interpretation}
Using the underlying stochastic differential equation,
\[
d\bm{X}_t = \bm{A}(\bm{X}_t,t)\,dt
+ \bm{\sigma}(\bm{X}_t,t)\,d\bm{W}_t,
\quad
D = \tfrac{1}{2}\bm{\sigma}\bm{\sigma}^T,
\]
the Green's function satisfies
\[
G(\bm{x},t\mid\bm{x}_0,t_0)
=
\left\langle
\delta\!\big(\bm{x}-\bm{X}_t\big)
\right\rangle_{\bm{X}_{t_0}=\bm{x}_0}.
\]

This links the Fokker--Planck equation to stochastic paths.
\end{frame}

%-------------------------------------------------
\begin{frame}{Key properties of the Green's function}
\begin{itemize}
  \item \textbf{Normalization:}
  \[
  \int d^d\bm{x}\,G(\bm{x},t\mid\bm{x}_0,t_0)=1.
  \]
  \item \textbf{Positivity:} $G \ge 0$.
  \item \textbf{Semigroup property:}
  \[
  \int d^d\bm{y}\;
  G(\bm{x},t\mid\bm{y},s)\,
  G(\bm{y},s\mid\bm{x}_0,t_0)
  =
  G(\bm{x},t\mid\bm{x}_0,t_0).
  \]
\end{itemize}
\end{frame}

%=================================================
\section{Summary}

\begin{frame}{Summary}
\begin{itemize}
  \item The forward Fokker--Planck equation governs probability evolution.
  \item The Green's function is the fundamental solution with delta initial data.
  \item It is derived via the Chapman--Kolmogorov equation and short-time expansion.
  \item General solutions follow by convolution with the Green's function.
  \item The propagator has a clear stochastic and physical interpretation.
\end{itemize}
\end{frame}


%=================================================
\section{Recap: Forward Fokker--Planck and Green's function}

\begin{frame}{Forward Fokker--Planck equation (recap)}
For $\bm{x}\in\mathbb{R}^d$, drift $\bm{A}(\bm{x},t)$ and diffusion tensor $D_{ij}(\bm{x},t)$:
\[
\partial_t p(\bm{x},t)
=
-\sum_{i=1}^d \partial_{x_i}\big[A_i(\bm{x},t)\,p(\bm{x},t)\big]
+
\sum_{i,j=1}^d \partial_{x_i}\partial_{x_j}\big[D_{ij}(\bm{x},t)\,p(\bm{x},t)\big].
\]

Green's function (propagator) $G(\bm{x},t\mid \bm{x}_0,t_0)$ satisfies
\[
\partial_t G = L^\dagger_{\bm{x}} G,\qquad
G(\bm{x},t_0\mid \bm{x}_0,t_0)=\delta(\bm{x}-\bm{x}_0),
\]
and general solutions follow from
\[
p(\bm{x},t) = \int d^d\bm{x}_0\;G(\bm{x},t\mid \bm{x}_0,t_0)\,p(\bm{x}_0,t_0).
\]
\end{frame}

%=================================================
\section{Worked Gaussian example: constant drift and diffusion}

\begin{frame}{Constant-coefficient FPE (Ornstein-free drift)}
Take constant drift $\bm{A}(\bm{x},t)=\bm{a}$ and constant diffusion $D_{ij}(\bm{x},t)=D_{ij}$ (symmetric, positive definite).
Then
\[
\partial_t p(\bm{x},t)= -\bm{\nabla}\cdot(\bm{a}\,p) + \sum_{i,j}\partial_{x_i}\partial_{x_j}\big(D_{ij}p\big)
= -\bm{a}\cdot \bm{\nabla}p + \bm{\nabla}\cdot\!\big(D\,\bm{\nabla}p\big),
\]
where $D$ is the diffusion matrix.

\vspace{0.3cm}
Goal: find the Green's function $G(\bm{x},t\mid\bm{x}_0,t_0)$.
\end{frame}

\begin{frame}{Shift to a comoving frame}
Define the comoving coordinate
\[
\bm{y} = \bm{x}-\bm{a}(t-t_0),
\qquad
\tilde{G}(\bm{y},t) \equiv G(\bm{x},t\mid\bm{x}_0,t_0).
\]

Then using $\partial_t \bm{y}=-\bm{a}$ and $\bm{\nabla}_{\bm{x}}=\bm{\nabla}_{\bm{y}}$,
\[
\partial_t G = \partial_t \tilde{G} - \bm{a}\cdot \bm{\nabla}_{\bm{y}}\tilde{G}.
\]
Plugging into the constant-coefficient FPE gives cancellation of the drift term:
\[
\partial_t \tilde{G} = \bm{\nabla}_{\bm{y}}\cdot\!\big(D\,\bm{\nabla}_{\bm{y}}\tilde{G}\big).
\]

So drift converts the problem into pure diffusion in $\bm{y}$.
\end{frame}

\begin{frame}{Fourier-space solution (anisotropic diffusion)}
Solve
\[
\partial_t \tilde{G}(\bm{y},t)=\bm{\nabla}\cdot(D\bm{\nabla}\tilde{G}),
\qquad
\tilde{G}(\bm{y},t_0)=\delta(\bm{y}-\bm{x}_0).
\]

Fourier transform in $\bm{y}$:
\[
\hat{G}(\bm{k},t)=\int d^d\bm{y}\;e^{-i\bm{k}\cdot\bm{y}}\,\tilde{G}(\bm{y},t).
\]
Then $\bm{\nabla}\mapsto i\bm{k}$ implies
\[
\partial_t \hat{G}(\bm{k},t)=-(\bm{k}^T D \bm{k})\,\hat{G}(\bm{k},t),
\qquad
\hat{G}(\bm{k},t_0)=e^{-i\bm{k}\cdot\bm{x}_0}.
\]
Hence
\[
\hat{G}(\bm{k},t)=\exp\!\Big(-i\bm{k}\cdot\bm{x}_0 - (t-t_0)\,\bm{k}^T D \bm{k}\Big).
\]
\end{frame}

\begin{frame}{Inverse Fourier transform: Gaussian propagator}
Invert the transform (Gaussian integral):
\[
\tilde{G}(\bm{y},t)=
\frac{1}{(4\pi (t-t_0))^{d/2}\sqrt{\det D}}
\exp\!\left(
-\frac{1}{4(t-t_0)}(\bm{y}-\bm{x}_0)^T D^{-1}(\bm{y}-\bm{x}_0)
\right).
\]

Transform back to $\bm{x}$ using $\bm{y}=\bm{x}-\bm{a}(t-t_0)$:
\[
G(\bm{x},t\mid\bm{x}_0,t_0)=
\frac{1}{(4\pi \Delta t)^{d/2}\sqrt{\det D}}
\exp\!\left(
-\frac{1}{4\Delta t}\Big(\bm{x}-\bm{x}_0-\bm{a}\Delta t\Big)^T D^{-1}\Big(\bm{x}-\bm{x}_0-\bm{a}\Delta t\Big)
\right),
\]
with $\Delta t=t-t_0>0$.

\vspace{0.2cm}
Mean: $\mathbb{E}[\bm{X}_t]=\bm{x}_0+\bm{a}\Delta t$, \quad Covariance: $\mathrm{Cov}(\bm{X}_t)=2D\,\Delta t$.
\end{frame}

%=================================================
\section{Boundary conditions via probability current}

\begin{frame}{Continuity equation and probability current}
Write the FPE as a continuity equation
\[
\partial_t p + \bm{\nabla}\cdot \bm{J} = 0,
\]
with probability current
\[
J_i(\bm{x},t)=A_i(\bm{x},t)\,p(\bm{x},t)-\sum_{j=1}^d \partial_{x_j}\!\big(D_{ij}(\bm{x},t)\,p(\bm{x},t)\big).
\]

Integrating over a domain $\Omega$ gives
\[
\frac{d}{dt}\int_\Omega p\,d^d\bm{x}
= -\int_{\partial\Omega} \bm{J}\cdot \bm{n}\,dS,
\]
so boundary conditions control probability loss/gain through $\partial\Omega$.
\end{frame}

\begin{frame}{Reflecting vs absorbing boundaries}
\textbf{Reflecting (no-flux) boundary:}
\[
\bm{J}\cdot\bm{n}\big|_{\partial\Omega}=0.
\]
Interpretation: probability is conserved inside $\Omega$; trajectories reflect at the boundary.

\vspace{0.4cm}
\textbf{Absorbing boundary (killing):}
\[
p(\bm{x},t)\big|_{\partial\Omega}=0.
\]
Interpretation: probability reaching the boundary is removed; total probability in $\Omega$ decays.

\vspace{0.2cm}
In both cases, $G$ must satisfy the same boundary condition in $\bm{x}$.
\end{frame}

%=================================================
\section{Absorbing states and killed propagators}

\begin{frame}{Absorbing states and survival probability}
If $\partial\Omega$ is absorbing, the total probability in $\Omega$ is the \emph{survival probability}
\[
S(t\mid \bm{x}_0,t_0)=\int_\Omega d^d\bm{x}\;G(\bm{x},t\mid \bm{x}_0,t_0).
\]
Its decay rate is the boundary flux:
\[
\frac{dS}{dt} = -\int_{\partial\Omega} \bm{J}\cdot\bm{n}\,dS.
\]

The corresponding \textbf{first-passage-time density} is
\[
f(t\mid \bm{x}_0,t_0) = -\frac{dS}{dt}.
\]
\end{frame}

\begin{frame}{Method of images: 1D absorbing wall (worked example)}
To illustrate absorbing boundaries explicitly, consider 1D drift--diffusion on $x>0$:
\[
\partial_t p = -a\,\partial_x p + D\,\partial_x^2 p,
\qquad p(0,t)=0,
\qquad p(x,t_0)=\delta(x-x_0),\;x_0>0.
\]

Free-space Green's function:
\[
G_0(x,t\mid x_0,t_0)=\frac{1}{\sqrt{4\pi D\Delta t}}
\exp\!\left[-\frac{(x-x_0-a\Delta t)^2}{4D\Delta t}\right].
\]

Absorbing boundary enforced by images:
\[
G(x,t\mid x_0,t_0)=G_0(x,t\mid x_0,t_0)-G_0(x,t\mid -x_0,t_0).
\]

Then $G(0,t\mid x_0,t_0)=0$ holds identically.
\end{frame}

\begin{frame}{Remarks on higher-dimensional absorbing boundaries}
In $d>1$, explicit closed forms are generally available only for special geometries:
\begin{itemize}
  \item half-space (planar boundary) $\rightarrow$ image methods (with modifications under drift/aniso diffusion),
  \item sphere/ball $\rightarrow$ eigenfunction expansions,
  \item general domains $\rightarrow$ spectral methods / numerical PDE / Monte Carlo.
\end{itemize}

\vspace{0.3cm}
Conceptually, the Green's function in a domain $\Omega$ is the \textbf{killed transition density}:
\[
G_\Omega(\bm{x},t\mid \bm{x}_0,t_0)
=
\mathbb{E}\!\left[\delta(\bm{x}-\bm{X}_t)\,\mathbf{1}_{\{\tau_{\partial\Omega}>t\}}\;\middle|\;\bm{X}_{t_0}=\bm{x}_0\right],
\]
where $\tau_{\partial\Omega}$ is the first hitting time of the absorbing boundary.
\end{frame}

%=================================================
\section{Summary}

\begin{frame}{Summary}
\begin{itemize}
  \item Constant drift and diffusion yield an explicit Gaussian propagator:
  \[
  G(\bm{x},t\mid\bm{x}_0,t_0)\propto
  \exp\!\left(-\frac{1}{4\Delta t}(\bm{x}-\bm{x}_0-\bm{a}\Delta t)^T D^{-1}(\bm{x}-\bm{x}_0-\bm{a}\Delta t)\right).
  \]
  \item Boundary conditions are imposed through the probability current $\bm{J}$:
  \begin{itemize}
    \item reflecting: $\bm{J}\cdot\bm{n}=0$,
    \item absorbing: $p=0$ on $\partial\Omega$.
  \end{itemize}
  \item Absorbing boundaries define survival and first-passage-time distributions via boundary flux.
\end{itemize}
\end{frame}

\end{document}
