\documentclass{beamer}

\usetheme{Madrid}
\usepackage{amsmath,amssymb,physics,bm}
\usepackage{mathtools}

\title{From Langevin Dynamics to the Fokker--Planck Equation}
\subtitle{Derivation and Physical Interpretation}
\author{MHJ}
\date{}

\begin{document}

%-------------------------------------------------
\begin{frame}
\titlepage
\end{frame}

%=================================================
\section{Motivation and setup}

\begin{frame}{Goal and roadmap}
\textbf{Goal:} Derive the forward Fokker--Planck equation (FPE) for the probability
density of a stochastic process and link it to the Langevin equation.

\vspace{0.3cm}
We proceed in three steps:
\begin{enumerate}
  \item Write the (Ito) Langevin / stochastic differential equation (SDE).
  \item Use the Chapman--Kolmogorov relation + Kramers--Moyal expansion.
  \item Obtain the Fokker--Planck equation and identify drift/diffusion.
\end{enumerate}

\vspace{0.3cm}
We emphasize the $d$-dimensional case.
\end{frame}

%-------------------------------------------------
\begin{frame}{Langevin equation: physical picture}
A Langevin equation models the dynamics of a coarse-grained variable $\bm{X}_t$:
\[
\text{systematic forces} \;+\; \text{random kicks from the environment}.
\]

Example (overdamped Brownian motion in a medium):
\[
\gamma \dot{\bm{X}}_t = \bm{F}(\bm{X}_t,t) + \bm{\xi}(t),
\qquad
\langle \xi_i(t)\rangle=0,
\qquad
\langle \xi_i(t)\xi_j(t')\rangle = 2\gamma^2 D_{ij}\,\delta(t-t').
\]

After dividing by $\gamma$, this becomes a drift + noise equation.
\end{frame}

%=================================================
\section{Ito Langevin / SDE formulation}

\begin{frame}{Ito SDE in $d$ dimensions}
We use the Ito SDE
\[
d\bm{X}_t = \bm{A}(\bm{X}_t,t)\,dt + \bm{\sigma}(\bm{X}_t,t)\,d\bm{W}_t,
\]
where
\begin{itemize}
  \item $\bm{A}(\bm{x},t)$ is the drift vector,
  \item $\bm{W}_t$ is $m$-dimensional Wiener process,
  \item $\bm{\sigma}(\bm{x},t)\in\mathbb{R}^{d\times m}$ sets noise amplitudes.
\end{itemize}

The diffusion tensor is
\[
D(\bm{x},t) \equiv \frac{1}{2}\,\bm{\sigma}(\bm{x},t)\bm{\sigma}(\bm{x},t)^{T},
\qquad
D_{ij} = \frac{1}{2}\sum_{k=1}^{m}\sigma_{ik}\sigma_{jk}.
\]
\end{frame}

%-------------------------------------------------
\begin{frame}{Transition density and what we want}
Define the transition probability density (propagator)
\[
G(\bm{x},t\mid \bm{x}_0,t_0)
=
\mathbb{P}\big(\bm{X}_t\in d^d\bm{x}\,\big|\,\bm{X}_{t_0}=\bm{x}_0\big)/d^d\bm{x}.
\]

For an initial density $p(\bm{x},t_0)$, the density at later times is
\[
p(\bm{x},t)=\int d^d\bm{x}_0\;G(\bm{x},t\mid\bm{x}_0,t_0)\,p(\bm{x}_0,t_0).
\]

\textbf{Goal:} derive a PDE for $p(\bm{x},t)$ (the Fokker--Planck equation).
\end{frame}

%=================================================
\section{Short-time expansion and Kramers--Moyal}

\begin{frame}{Chapman--Kolmogorov (Markov property)}
Markovity implies (for $t_0<s<t$)
\[
G(\bm{x},t\mid \bm{x}_0,t_0)
=
\int d^d\bm{y}\;
G(\bm{x},t\mid \bm{y},s)\,
G(\bm{y},s\mid \bm{x}_0,t_0).
\]

This semigroup property is the starting point for a short-time expansion.
\end{frame}

%-------------------------------------------------
\begin{frame}{Infinitesimal increment from the SDE}
Over a short interval $\Delta t$,
\[
\Delta \bm{X} \equiv \bm{X}_{t+\Delta t}-\bm{X}_t
=
\bm{A}(\bm{X}_t,t)\,\Delta t
+
\bm{\sigma}(\bm{X}_t,t)\,\Delta \bm{W},
\]
with $\Delta \bm{W}\sim \mathcal{N}(\bm{0},\Delta t\,I)$.

Conditional moments (Ito):
\[
\mathbb{E}[\Delta X_i \mid \bm{X}_t=\bm{x}] = A_i(\bm{x},t)\,\Delta t + o(\Delta t),
\]
\[
\mathbb{E}[\Delta X_i\Delta X_j\mid \bm{X}_t=\bm{x}]
=
(\bm{\sigma}\bm{\sigma}^T)_{ij}(\bm{x},t)\,\Delta t + o(\Delta t)
= 2D_{ij}(\bm{x},t)\,\Delta t + o(\Delta t).
\]
Higher conditional moments are $o(\Delta t)$ for a diffusion process.
\end{frame}

%-------------------------------------------------
\begin{frame}{Kramers--Moyal expansion (idea)}
For a Markov process with smooth transition density,
expand $p(\bm{x},t+\Delta t)$ in terms of jump moments:
\[
p(\bm{x},t+\Delta t)
=
\int d^d\bm{y}\;p(\bm{y},t)\,G(\bm{x},t+\Delta t\mid \bm{y},t).
\]

Let $\bm{\xi}=\bm{x}-\bm{y}$ and expand around $\bm{\xi}=0$:
\[
p(\bm{y},t)=p(\bm{x}-\bm{\xi},t)
=
p(\bm{x},t)-\xi_i\partial_{x_i}p+\frac{1}{2}\xi_i\xi_j\partial_{x_i}\partial_{x_j}p+\cdots
\]
and average powers of $\bm{\xi}$ using the conditional moments from the SDE.
\end{frame}

%=================================================
\section{Derivation of the Fokker--Planck equation}

\begin{frame}{Derivation: keep terms up to $\mathcal{O}(\Delta t)$}
Using the conditional moments:
\[
\langle \xi_i \rangle = A_i(\bm{x},t)\,\Delta t,
\qquad
\langle \xi_i\xi_j \rangle = 2D_{ij}(\bm{x},t)\,\Delta t,
\]
and neglecting higher moments ($o(\Delta t)$), we obtain
\[
p(\bm{x},t+\Delta t)-p(\bm{x},t)
=
-\partial_{x_i}\!\big(A_i(\bm{x},t)p(\bm{x},t)\big)\Delta t
+\partial_{x_i}\partial_{x_j}\!\big(D_{ij}(\bm{x},t)p(\bm{x},t)\big)\Delta t
+o(\Delta t).
\]

Divide by $\Delta t$ and take $\Delta t\to 0$.
\end{frame}

%-------------------------------------------------
\begin{frame}{Forward Fokker--Planck equation (result)}
We arrive at the forward Fokker--Planck equation:
\[
\boxed{
\partial_t p(\bm{x},t)
=
-\sum_{i=1}^d \partial_{x_i}\Big(A_i(\bm{x},t)\,p(\bm{x},t)\Big)
+
\sum_{i,j=1}^d \partial_{x_i}\partial_{x_j}\Big(D_{ij}(\bm{x},t)\,p(\bm{x},t)\Big)
}
\]

\vspace{0.3cm}
Identification:
\begin{itemize}
  \item Drift $A_i$ comes from the mean increment of $\Delta X_i$.
  \item Diffusion $D_{ij}$ comes from the covariance of increments.
\end{itemize}
\end{frame}

%=================================================
\section{Generator and Ito's formula connection}

\begin{frame}{Backward generator and Ito's formula}
For a smooth test function $f(\bm{x},t)$, Ito's formula gives
\[
df(\bm{X}_t,t)
=
\left(\partial_t f + A_i\partial_{x_i}f + D_{ij}\partial_{x_i}\partial_{x_j}f\right)dt
+
(\partial_{x_i}f)\,\sigma_{ik}\,dW_k.
\]

Define the (backward) generator
\[
(Lf)(\bm{x},t)=A_i(\bm{x},t)\partial_{x_i}f(\bm{x},t)
+
D_{ij}(\bm{x},t)\partial_{x_i}\partial_{x_j}f(\bm{x},t).
\]

Then $\mathbb{E}[f(\bm{X}_t,t)]$ evolves according to $L$.
\end{frame}

%-------------------------------------------------
\begin{frame}{Adjoint relation: generator vs Fokker--Planck}
The Fokker--Planck operator $L^\dagger$ is the adjoint of $L$:
\[
\int d^d\bm{x}\; (Lf)\,p
=
\int d^d\bm{x}\; f\,(L^\dagger p)
\quad \text{(up to boundary terms)}.
\]

Explicitly,
\[
(L^\dagger p)(\bm{x},t)
=
-\partial_{x_i}\big(A_i p\big)
+
\partial_{x_i}\partial_{x_j}\big(D_{ij}p\big),
\]
so that $\partial_t p=L^\dagger p$.

\vspace{0.3cm}
This is the clean operator link between Langevin (SDE) and FPE (PDE).
\end{frame}

%=================================================
\section{Continuity equation and current}

\begin{frame}{Probability current and boundary conditions}
Write the FPE as a continuity equation:
\[
\partial_t p + \bm{\nabla}\cdot \bm{J}=0,
\]
with current
\[
J_i(\bm{x},t)=A_i(\bm{x},t)\,p(\bm{x},t)
-\partial_{x_j}\big(D_{ij}(\bm{x},t)\,p(\bm{x},t)\big).
\]

Boundary conditions are formulated in terms of $\bm{J}$:
\begin{itemize}
  \item Reflecting: $\bm{J}\cdot \bm{n}=0$ on $\partial\Omega$,
  \item Absorbing: $p=0$ on $\partial\Omega$.
\end{itemize}
\end{frame}

%=================================================
\section{Worked example: overdamped Brownian motion}

\begin{frame}{Example: overdamped Brownian motion in a potential}
Overdamped Langevin equation:
\[
\gamma \dot{\bm{X}}_t = -\bm{\nabla}U(\bm{X}_t) + \bm{\xi}(t),
\qquad
\langle \xi_i(t)\xi_j(t')\rangle = 2\gamma k_BT\,\delta_{ij}\delta(t-t').
\]

Equivalently (Ito SDE),
\[
d\bm{X}_t = -\frac{1}{\gamma}\bm{\nabla}U(\bm{X}_t)\,dt + \sqrt{2D}\,d\bm{W}_t,
\qquad D=\frac{k_BT}{\gamma}.
\]

Thus
\[
\bm{A}(\bm{x})=-\frac{1}{\gamma}\bm{\nabla}U(\bm{x}),
\qquad
D_{ij}=D\,\delta_{ij}.
\]
\end{frame}

\begin{frame}{Corresponding Fokker--Planck equation}
Insert $\bm{A},D$:
\[
\partial_t p(\bm{x},t)
=
\bm{\nabla}\cdot\!\left(\frac{1}{\gamma}\bm{\nabla}U(\bm{x})\,p(\bm{x},t)\right)
+
D\,\nabla^2 p(\bm{x},t).
\]

In equilibrium (detailed balance), the stationary solution is the Gibbs density
\[
p_{\mathrm{eq}}(\bm{x})
\propto e^{-\beta U(\bm{x})},\qquad \beta=(k_BT)^{-1},
\]
which satisfies $\bm{J}=0$.
\end{frame}

%=================================================
\section{Summary}

\begin{frame}{Summary}
\begin{itemize}
  \item Start from Ito Langevin dynamics:
  \[
  d\bm{X}_t=\bm{A}(\bm{X}_t,t)\,dt+\bm{\sigma}(\bm{X}_t,t)\,d\bm{W}_t,
  \quad
  D=\tfrac12 \bm{\sigma}\bm{\sigma}^T.
  \]
  \item Short-time increment statistics imply:
  \[
  \mathbb{E}[\Delta X_i]=A_i\Delta t,\qquad
  \mathbb{E}[\Delta X_i\Delta X_j]=2D_{ij}\Delta t.
  \]
  \item Keeping terms up to $\mathcal{O}(\Delta t)$ yields the forward FPE:
  \[
  \partial_t p = -\partial_i(A_i p)+\partial_i\partial_j(D_{ij}p).
  \]
  \item Operator viewpoint: $L$ (Ito generator) and $L^\dagger$ (Fokker--Planck) are adjoints.
\end{itemize}
\end{frame}

\end{document}
