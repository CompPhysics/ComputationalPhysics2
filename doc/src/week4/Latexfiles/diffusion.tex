\documentclass{beamer}

\usetheme{Madrid}
\usepackage{amsmath,amssymb,bm}
\usepackage{physics}

\title{From Markov Chains to the 1D Diffusion Equation}
\subtitle{Master Equation $\rightarrow$ Continuum Limit $\rightarrow$ Fick's Law}
\author{Morten Hjorth-Jensen}
\date{Spring 2026}

\begin{document}

%-------------------------------------------------
\begin{frame}
\titlepage
\end{frame}

%-------------------------------------------------
\begin{frame}{Goal and roadmap}
\textbf{Goal:} derive the 1D diffusion equation
\[
\partial_t \rho(x,t)=D\,\partial_x^2 \rho(x,t)
\]
starting from a discrete-time/discrete-space \textbf{Markov chain}.

\vspace{0.3cm}
\textbf{Roadmap:}
\begin{enumerate}
  \item Random walk as a Markov chain on a lattice
  \item Chapman--Kolmogorov / master equation
  \item Taylor expansion and scaling limit
  \item Identify diffusion coefficient $D$
  \item Continuity equation and Fick's law
\end{enumerate}
\end{frame}

%=================================================
\section{Markov chain on a 1D lattice}

\begin{frame}{Random walk as a Markov chain}
Consider a particle on a 1D lattice with spacing $a$:
\[
x_n = n a,\qquad n\in\mathbb{Z}.
\]
Let $P_n(k)$ be the probability to be at site $n$ after $k$ time steps.

\vspace{0.3cm}
Assume a nearest-neighbor Markov chain:
\[
\mathbb{P}(n\to n+1)=p,\qquad \mathbb{P}(n\to n-1)=q,\qquad p+q=1.
\]

\textbf{Markov property:} the next state depends only on the current state.
\end{frame}

%-------------------------------------------------
\begin{frame}{Transition matrix and Chapman--Kolmogorov}
The one-step transition probabilities define a stochastic matrix $T$:
\[
P_n(k+1)=\sum_m T_{nm}\,P_m(k),
\]
with (nearest-neighbor)
\[
T_{n,n-1}=p,\qquad T_{n,n+1}=q,\qquad T_{n,n}=0.
\]

The Chapman--Kolmogorov property for $r$ steps:
\[
T^{(r+s)} = T^{(r)}T^{(s)}.
\]
This semigroup property underlies the continuum limit.
\end{frame}

%=================================================
\section{Master equation (discrete)}

\begin{frame}{Discrete master equation}
From the transition rule:
\[
P_n(k+1)=p\,P_{n-1}(k)+q\,P_{n+1}(k).
\]
Rewrite as an increment equation:
\[
P_n(k+1)-P_n(k)
=
p\big(P_{n-1}(k)-P_n(k)\big)
+
q\big(P_{n+1}(k)-P_n(k)\big).
\]

Interpretation:
\begin{itemize}
  \item gain from neighbors $n\pm 1$,
  \item loss from leaving site $n$.
\end{itemize}
\end{frame}

%-------------------------------------------------
\begin{frame}{Continuous-time version (optional but standard)}
Let steps occur in continuous time with rate $\gamma$.
Define $P_n(t)$ and assume exponential waiting times (continuous-time Markov chain).
Then
\[
\frac{dP_n(t)}{dt}
=
\gamma\Big[pP_{n-1}(t)+qP_{n+1}(t)-(p+q)P_n(t)\Big].
\]
Since $p+q=1$:
\[
\boxed{
\frac{dP_n}{dt}
=
\gamma\Big[pP_{n-1}+qP_{n+1}-P_n\Big].
}
\]
This is the \textbf{master equation} for a nearest-neighbor jump process.
\end{frame}

%=================================================
\section{Continuum limit and diffusion}

\begin{frame}{From lattice probabilities to a density}
Define a smooth probability density $\rho(x,t)$ such that
\[
P_n(t)\approx a\,\rho(x,t)\Big|_{x=na}.
\]
Assume $\rho$ varies slowly on the scale of $a$.

\vspace{0.3cm}
We expand
\[
\rho(x\pm a,t)
=
\rho(x,t)\pm a\,\partial_x\rho(x,t)
+\frac{a^2}{2}\partial_x^2\rho(x,t)
+\mathcal{O}(a^3).
\]
\end{frame}

%-------------------------------------------------
\begin{frame}{Taylor expansion of the master equation}
Start from
\[
\frac{dP_n}{dt}=\gamma\Big[pP_{n-1}+qP_{n+1}-P_n\Big].
\]
Divide by $a$ and use $P_n\approx a\rho(x,t)$:
\[
\partial_t \rho(x,t)
=
\gamma\Big[p\rho(x-a,t)+q\rho(x+a,t)-\rho(x,t)\Big].
\]

Insert Taylor expansions:
\begin{align*}
p\rho(x-a)+q\rho(x+a)
&=
(p+q)\rho
+(q-p)a\,\partial_x\rho
+\frac{a^2}{2}(p+q)\partial_x^2\rho
+\mathcal{O}(a^3)\\
&=
\rho +(q-p)a\,\partial_x\rho+\frac{a^2}{2}\partial_x^2\rho+\mathcal{O}(a^3).
\end{align*}

Hence
\[
\partial_t\rho
=
\gamma\Big[(q-p)a\,\partial_x\rho+\frac{a^2}{2}\partial_x^2\rho\Big]
+\mathcal{O}(a^3).
\]
\end{frame}

%-------------------------------------------------
\begin{frame}{Drift--diffusion equation and unbiased diffusion}
Define drift velocity and diffusion coefficient
\[
v \equiv \gamma (q-p)a,
\qquad
D \equiv \frac{\gamma a^2}{2}.
\]
Then, to leading order,
\[
\boxed{
\partial_t\rho(x,t)= -v\,\partial_x\rho(x,t) + D\,\partial_x^2\rho(x,t).
}
\]

\textbf{Unbiased random walk:} $p=q=\tfrac{1}{2}$ $\Rightarrow$ $v=0$:
\[
\boxed{
\partial_t\rho(x,t)= D\,\partial_x^2\rho(x,t).
}
\]
This is the 1D diffusion equation.
\end{frame}

%-------------------------------------------------
\begin{frame}{Scaling limit (diffusive scaling)}
To obtain a nontrivial continuum PDE as $a\to 0$, we keep $D$ finite:
\[
D=\frac{\gamma a^2}{2}=\text{fixed}.
\]
This is achieved by letting the jump rate scale as
\[
\gamma \sim \frac{2D}{a^2}.
\]

Interpretation:
\begin{itemize}
  \item steps become smaller ($a\to 0$),
  \item jumps become more frequent ($\gamma\to\infty$),
  \item their combined effect yields finite diffusion.
\end{itemize}
\end{frame}

%=================================================
\section{Continuity equation and Fick's law}

\begin{frame}{Continuity equation form}
Write diffusion as conservation of probability:
\[
\partial_t \rho + \partial_x J = 0.
\]
For the diffusion equation $\partial_t\rho=D\partial_x^2\rho$,
choose
\[
J(x,t) = -D\,\partial_x\rho(x,t).
\]
Then
\[
\partial_t\rho + \partial_x(-D\partial_x\rho)=0
\;\Rightarrow\;
\partial_t\rho=D\partial_x^2\rho.
\]

\textbf{This is Fick's first law:} $J=-D\nabla\rho$ in 1D.
\end{frame}

%-------------------------------------------------
\begin{frame}{Physical interpretation and mean-square displacement}
For unbiased diffusion, the second moment grows linearly:
\[
\langle x^2(t)\rangle - \langle x(t)\rangle^2 = 2Dt.
\]

From the Markov chain viewpoint:
after $k$ steps with step length $a$ and time step $\Delta t$,
\[
\mathrm{Var}(x_k)=k a^2,
\qquad t=k\Delta t,
\qquad D=\frac{a^2}{2\Delta t}.
\]
This matches $2Dt$ and provides an empirical route to estimate $D$.
\end{frame}

%=================================================
\section{Summary}

\begin{frame}{Summary}
\begin{itemize}
  \item A nearest-neighbor random walk is a Markov chain with transition probabilities $p,q$.
  \item The master equation reads
  \[
  \frac{dP_n}{dt}=\gamma\big[pP_{n-1}+qP_{n+1}-P_n\big].
  \]
  \item With $P_n\approx a\rho(na,t)$ and Taylor expansion,
  \[
  \partial_t\rho=-v\partial_x\rho+D\partial_x^2\rho,
  \quad
  v=\gamma(q-p)a,
  \quad
  D=\frac{\gamma a^2}{2}.
  \]
  \item For $p=q$, one obtains the 1D diffusion equation
  \[
  \partial_t\rho=D\,\partial_x^2\rho,
  \]
  equivalent to $\partial_t\rho+\partial_x J=0$ with $J=-D\partial_x\rho$ (Fick's law).
\end{itemize}
\end{frame}

\end{document}
