%%
%% Automatically generated file from DocOnce source
%% (https://github.com/doconce/doconce/)
%% doconce format latex week5.do.txt --minted_latex_style=trac --latex_admon=paragraph --no_mako
%%


%-------------------- begin preamble ----------------------

\documentclass[%
oneside,                 % oneside: electronic viewing, twoside: printing
final,                   % draft: marks overfull hboxes, figures with paths
10pt]{article}

\listfiles               %  print all files needed to compile this document

\usepackage{relsize,makeidx,color,setspace,amsmath,amsfonts,amssymb}
\usepackage[table]{xcolor}
\usepackage{bm,ltablex,microtype}

\usepackage[pdftex]{graphicx}

\usepackage{fancyvrb} % packages needed for verbatim environments
\usepackage{minted}
\usemintedstyle{default}

\usepackage[T1]{fontenc}
%\usepackage[latin1]{inputenc}
\usepackage{ucs}
\usepackage[utf8x]{inputenc}

\usepackage{lmodern}         % Latin Modern fonts derived from Computer Modern

% Hyperlinks in PDF:
\definecolor{linkcolor}{rgb}{0,0,0.4}
\usepackage{hyperref}
\hypersetup{
    breaklinks=true,
    colorlinks=true,
    linkcolor=linkcolor,
    urlcolor=linkcolor,
    citecolor=black,
    filecolor=black,
    %filecolor=blue,
    pdfmenubar=true,
    pdftoolbar=true,
    bookmarksdepth=3   % Uncomment (and tweak) for PDF bookmarks with more levels than the TOC
    }
%\hyperbaseurl{}   % hyperlinks are relative to this root

\setcounter{tocdepth}{2}  % levels in table of contents

% --- fancyhdr package for fancy headers ---
\usepackage{fancyhdr}
\fancyhf{} % sets both header and footer to nothing
\renewcommand{\headrulewidth}{0pt}
\fancyfoot[LE,RO]{\thepage}
% Ensure copyright on titlepage (article style) and chapter pages (book style)
\fancypagestyle{plain}{
  \fancyhf{}
  \fancyfoot[C]{{\footnotesize \copyright\ 1999-2023, Morten Hjorth-Jensen  Email morten.hjorth-jensen@fys.uio.no. Released under CC Attribution-NonCommercial 4.0 license}}
%  \renewcommand{\footrulewidth}{0mm}
  \renewcommand{\headrulewidth}{0mm}
}
% Ensure copyright on titlepages with \thispagestyle{empty}
\fancypagestyle{empty}{
  \fancyhf{}
  \fancyfoot[C]{{\footnotesize \copyright\ 1999-2023, Morten Hjorth-Jensen  Email morten.hjorth-jensen@fys.uio.no. Released under CC Attribution-NonCommercial 4.0 license}}
  \renewcommand{\footrulewidth}{0mm}
  \renewcommand{\headrulewidth}{0mm}
}

\pagestyle{fancy}


\usepackage[framemethod=TikZ]{mdframed}

% --- begin definitions of admonition environments ---

% --- end of definitions of admonition environments ---

% prevent orhpans and widows
\clubpenalty = 10000
\widowpenalty = 10000

\newenvironment{doconceexercise}{}{}
\newcounter{doconceexercisecounter}


% ------ header in subexercises ------
%\newcommand{\subex}[1]{\paragraph{#1}}
%\newcommand{\subex}[1]{\par\vspace{1.7mm}\noindent{\bf #1}\ \ }
\makeatletter
% 1.5ex is the spacing above the header, 0.5em the spacing after subex title
\newcommand\subex{\@startsection*{paragraph}{4}{\z@}%
                  {1.5ex\@plus1ex \@minus.2ex}%
                  {-0.5em}%
                  {\normalfont\normalsize\bfseries}}
\makeatother


% --- end of standard preamble for documents ---


% insert custom LaTeX commands...

\raggedbottom
\makeindex
\usepackage[totoc]{idxlayout}   % for index in the toc
\usepackage[nottoc]{tocbibind}  % for references/bibliography in the toc

%-------------------- end preamble ----------------------

\begin{document}

% matching end for #ifdef PREAMBLE

\newcommand{\exercisesection}[1]{\subsection*{#1}}


% ------------------- main content ----------------------



% ----------------- title -------------------------

\thispagestyle{empty}

\begin{center}
{\LARGE\bf
\begin{spacing}{1.25}
Week 8 February 20-24: Optimization and gradient methods
\end{spacing}
}
\end{center}

% ----------------- author(s) -------------------------

\begin{center}
{\bf Morten Hjorth-Jensen  Email morten.hjorth-jensen@fys.uio.no${}^{1, 2}$} \\ [0mm]
\end{center}

\begin{center}
% List of all institutions:
\centerline{{\small ${}^1$Department of Physics and Center fo Computing in Science Education, University of Oslo, Oslo, Norway}}
\centerline{{\small ${}^2$Department of Physics and Astronomy and Facility for Rare Isotope Beams, Michigan State University, East Lansing, Michigan, USA}}
\end{center}
    
% ----------------- end author(s) -------------------------

% --- begin date ---
\begin{center}
February 20-24
\end{center}
% --- end date ---

\vspace{1cm}


% stochastic reconfiguration (see Sorella, PRB 2011).

% !split
\subsection*{Overview of week 8, February 20-24}

% --- begin paragraph admon ---
\paragraph{Topics.}
\begin{itemize}
\item Expressions for derivatives

\item Gradient Descent methods, Steepest descent and Conjugate Gradient Descent 

\item \href{{https://youtu.be/}}{Video of Lecture TBA}

\item \href{{https://github.com/CompPhysics/ComputationalPhysics2/blob/gh-pages/doc/HandWrittenNotes/2022/NotesFebruary23.pdf}}{Handwritten notes}
\end{itemize}

\noindent
% --- end paragraph admon ---




% --- begin paragraph admon ---
\paragraph{Teaching Material, videos and written material.}
\begin{itemize}
\item These lecture notes

\item \href{{https://www.youtube.com/watch?v=eAYohMUpPMA&ab_channel=TomCarlone}}{Video on the Conjugate Gradient methods}

\item Recommended background literature, \href{{https://web.stanford.edu/~boyd/cvxbook/}}{Convex Optimization} by Boyd and Vandenberghe. Their \href{{https://web.stanford.edu/~boyd/cvxbook/bv_cvxslides.pdf}}{lecture slides} are very useful (warning, these are some 300 pages).
\end{itemize}

\noindent
% --- end paragraph admon ---



% !split
\subsection*{Top-down start}

\begin{itemize}
\item We will start with a top-down view, with a simple harmonic oscillator problem in one dimension as case.

\item Thereafter we continue with implementing the simplest possible steepest descent approach to our two-electron problem with an electrostatic (Coulomb) interaction. Our code includes also importance sampling. The simple Python code here illustrates the basic elements which need to be included in our own code.

\item Then we move on to the mathematical description of various gradient methods.
\end{itemize}

\noindent
% !split
\subsection*{Motivation}

% --- begin paragraph admon ---
\paragraph{}
Our aim with this part of the project is to be able to
\begin{itemize}
\item find an optimal value for the variational parameters using only some few Monte Carlo cycles

\item use these optimal values for the variational parameters to perform a large-scale Monte Carlo calculation
\end{itemize}

\noindent
To achieve this will look at methods like \emph{Steepest descent} and the \emph{conjugate gradient method}. Both these methods allow us to find
the minima of a multivariable  function like our energy (function of several variational parameters). 
Alternatively, you can always use Newton's method. In particular, since we will normally have one variational parameter,
Newton's method can be easily used in finding the minimum of the local energy.
% --- end paragraph admon ---



% !split
\subsection*{Simple example and demonstration}

Let us illustrate what is needed in our calculations using a simple example, the harmonic oscillator in one dimension.
For the harmonic oscillator in one-dimension we have a  trial wave function and probability
\[
\psi_T(x;\alpha) = \exp{-(\frac{1}{2}\alpha^2x^2)},
\]
which results in a local energy 
\[
\frac{1}{2}\left(\alpha^2+x^2(1-\alpha^4)\right).
\]
We can compare our numerically calculated energies with the exact energy as function of $\alpha$
\[
\overline{E}[\alpha] = \frac{1}{4}\left(\alpha^2+\frac{1}{\alpha^2}\right).
\]

% !split
\subsection*{Simple example and demonstration}

% --- begin paragraph admon ---
\paragraph{}
The derivative of the energy with respect to $\alpha$ gives
\begin{equation*}
\frac{d\langle  E_L[\alpha]\rangle}{d\alpha} = \frac{1}{2}\alpha-\frac{1}{2\alpha^3}
\end{equation*}
and a second derivative which is always positive (meaning that we find a minimum)
\begin{equation*}
\frac{d^2\langle  E_L[\alpha]\rangle}{d\alpha^2} = \frac{1}{2}+\frac{3}{2\alpha^4}
\end{equation*}
The condition
\begin{equation*}
\frac{d\langle  E_L[\alpha]\rangle}{d\alpha} = 0,
\end{equation*}
gives the optimal $\alpha=1$, as expected.
% --- end paragraph admon ---



% !split

% --- begin exercise ---
\begin{doconceexercise}
\refstepcounter{doconceexercisecounter}

\exercisesection*{Exercise \thedoconceexercisecounter: Find the local energy for the harmonic oscillator}
                             

% --- begin subexercise ---
\subex{a)}
Derive the local energy for the harmonic oscillator in one dimension and find its expectation value.

% --- end subexercise ---

% --- begin subexercise ---
\subex{b)}
Show also that the optimal value of optimal $\alpha=1$

% --- end subexercise ---

% --- begin subexercise ---
\subex{c)}
Repeat the above steps in two dimensions for $N$ bosons or electrons. What is the optimal value of $\alpha$?

% --- end subexercise ---

\end{doconceexercise}
% --- end exercise ---

% !split
\subsection*{Variance in the simple model}

% --- begin paragraph admon ---
\paragraph{}
We can also minimize the variance. In our simple model the variance is

\[
\sigma^2[\alpha]=\frac{1}{4}\left(1+(1-\alpha^4)^2\frac{3}{4\alpha^4}\right)-\overline{E}^2.
\]
which yields a second derivative which is always positive.
% --- end paragraph admon ---



% !split
\subsection*{Computing the derivatives}

% --- begin paragraph admon ---
\paragraph{}

In general we end up computing the expectation value of the energy in terms 
of some parameters $\alpha_0,\alpha_1,\dots,\alpha_n$
and we search for a minimum in this multi-variable parameter space.  
This leads to an energy minimization problem \emph{where we need the derivative of the energy as a function of the variational parameters}.

In the above example this was easy and we were able to find the expression for the derivative by simple derivations. 
However, in our actual calculations the energy is represented by a multi-dimensional integral with several variational parameters.
How can we can then obtain the derivatives of the energy with respect to the variational parameters without having 
to resort to expensive numerical derivations?
% --- end paragraph admon ---



% !split
\subsection*{Expressions for finding the derivatives of the local energy}

% --- begin paragraph admon ---
\paragraph{}

To find the derivatives of the local energy expectation value as function of the variational parameters, we can use the chain rule and the hermiticity of the Hamiltonian.  

Let us define 
\[
\bar{E}_{\alpha}=\frac{d\langle  E_L[\alpha]\rangle}{d\alpha}.
\]
as the derivative of the energy with respect to the variational parameter $\alpha$ (we limit ourselves to one parameter only).
In the above example this was easy and we obtain a simple expression for the derivative.
We define also the derivative of the trial function (skipping the subindex $T$) as 
\[
\bar{\psi}_{\alpha}=\frac{d\psi[\alpha]\rangle}{d\alpha}.
\]
% --- end paragraph admon ---



% !split
\subsection*{Derivatives of the local energy}

% --- begin paragraph admon ---
\paragraph{}
The elements of the gradient of the local energy are then (using the chain rule and the hermiticity of the Hamiltonian)
\[
\bar{E}_{\alpha} = 2\left( \langle \frac{\bar{\psi}_{\alpha}}{\psi[\alpha]}E_L[\alpha]\rangle -\langle \frac{\bar{\psi}_{\alpha}}{\psi[\alpha]}\rangle\langle E_L[\alpha] \rangle\right).
\]
From a computational point of view it means that you need to compute the expectation values of 
\[
\langle \frac{\bar{\psi}_{\alpha}}{\psi[\alpha]}E_L[\alpha]\rangle,
\]
and
\[
\langle \frac{\bar{\psi}_{\alpha}}{\psi[\alpha]}\rangle\langle E_L[\alpha]\rangle
\]
% --- end paragraph admon ---



% !split

% --- begin exercise ---
\begin{doconceexercise}
\refstepcounter{doconceexercisecounter}

\exercisesection*{Exercise \thedoconceexercisecounter: General expression for the derivative of the energy}
                             

% --- begin subexercise ---
\subex{a)}
Show that 
\[
\bar{E}_{\alpha} = 2\left( \langle \frac{\bar{\psi}_{\alpha}}{\psi[\alpha]}E_L[\alpha]\rangle -\langle \frac{\bar{\psi}_{\alpha}}{\psi[\alpha]}\rangle\langle E_L[\alpha] \rangle\right).
\]

% --- end subexercise ---

% --- begin subexercise ---
\subex{b)}
Find the corresponding expression for the variance.

% --- end subexercise ---

\end{doconceexercise}
% --- end exercise ---

% !split
\subsection*{Python program for 2-electrons in 2 dimensions}



















































































































































\begin{minted}[fontsize=\fontsize{9pt}{9pt},linenos=false,mathescape,baselinestretch=1.0,fontfamily=tt,xleftmargin=7mm]{python}
# 2-electron VMC code for 2dim quantum dot with importance sampling
# Using gaussian rng for new positions and Metropolis- Hastings 
# Added energy minimization with gradient descent using fixed step size
# To do: replace with optimization codes from scipy and/or use stochastic gradient descent
from math import exp, sqrt
from random import random, seed, normalvariate
import numpy as np
import matplotlib.pyplot as plt
from mpl_toolkits.mplot3d import Axes3D
from matplotlib import cm
from matplotlib.ticker import LinearLocator, FormatStrFormatter
import sys



# Trial wave function for the 2-electron quantum dot in two dims
def WaveFunction(r,alpha,beta):
    r1 = r[0,0]**2 + r[0,1]**2
    r2 = r[1,0]**2 + r[1,1]**2
    r12 = sqrt((r[0,0]-r[1,0])**2 + (r[0,1]-r[1,1])**2)
    deno = r12/(1+beta*r12)
    return exp(-0.5*alpha*(r1+r2)+deno)

# Local energy  for the 2-electron quantum dot in two dims, using analytical local energy
def LocalEnergy(r,alpha,beta):
    
    r1 = (r[0,0]**2 + r[0,1]**2)
    r2 = (r[1,0]**2 + r[1,1]**2)
    r12 = sqrt((r[0,0]-r[1,0])**2 + (r[0,1]-r[1,1])**2)
    deno = 1.0/(1+beta*r12)
    deno2 = deno*deno
    return 0.5*(1-alpha*alpha)*(r1 + r2) +2.0*alpha + 1.0/r12+deno2*(alpha*r12-deno2+2*beta*deno-1.0/r12)

# Derivate of wave function ansatz as function of variational parameters
def DerivativeWFansatz(r,alpha,beta):
    
    WfDer  = np.zeros((2), np.double)
    r1 = (r[0,0]**2 + r[0,1]**2)
    r2 = (r[1,0]**2 + r[1,1]**2)
    r12 = sqrt((r[0,0]-r[1,0])**2 + (r[0,1]-r[1,1])**2)
    deno = 1.0/(1+beta*r12)
    deno2 = deno*deno
    WfDer[0] = -0.5*(r1+r2)
    WfDer[1] = -r12*r12*deno2
    return  WfDer

# Setting up the quantum force for the two-electron quantum dot, recall that it is a vector
def QuantumForce(r,alpha,beta):

    qforce = np.zeros((NumberParticles,Dimension), np.double)
    r12 = sqrt((r[0,0]-r[1,0])**2 + (r[0,1]-r[1,1])**2)
    deno = 1.0/(1+beta*r12)
    qforce[0,:] = -2*r[0,:]*alpha*(r[0,:]-r[1,:])*deno*deno/r12
    qforce[1,:] = -2*r[1,:]*alpha*(r[1,:]-r[0,:])*deno*deno/r12
    return qforce
    

# Computing the derivative of the energy and the energy 
def EnergyMinimization(alpha, beta):

    NumberMCcycles= 10000
    # Parameters in the Fokker-Planck simulation of the quantum force
    D = 0.5
    TimeStep = 0.05
    # positions
    PositionOld = np.zeros((NumberParticles,Dimension), np.double)
    PositionNew = np.zeros((NumberParticles,Dimension), np.double)
    # Quantum force
    QuantumForceOld = np.zeros((NumberParticles,Dimension), np.double)
    QuantumForceNew = np.zeros((NumberParticles,Dimension), np.double)

    # seed for rng generator 
    seed()
    energy = 0.0
    DeltaE = 0.0
    EnergyDer = np.zeros((2), np.double)
    DeltaPsi = np.zeros((2), np.double)
    DerivativePsiE = np.zeros((2), np.double)
    #Initial position
    for i in range(NumberParticles):
        for j in range(Dimension):
            PositionOld[i,j] = normalvariate(0.0,1.0)*sqrt(TimeStep)
    wfold = WaveFunction(PositionOld,alpha,beta)
    QuantumForceOld = QuantumForce(PositionOld,alpha, beta)

    #Loop over MC MCcycles
    for MCcycle in range(NumberMCcycles):
        #Trial position moving one particle at the time
        for i in range(NumberParticles):
            for j in range(Dimension):
                PositionNew[i,j] = PositionOld[i,j]+normalvariate(0.0,1.0)*sqrt(TimeStep)+\
                                       QuantumForceOld[i,j]*TimeStep*D
            wfnew = WaveFunction(PositionNew,alpha,beta)
            QuantumForceNew = QuantumForce(PositionNew,alpha, beta)
            GreensFunction = 0.0
            for j in range(Dimension):
                GreensFunction += 0.5*(QuantumForceOld[i,j]+QuantumForceNew[i,j])*\
	                              (D*TimeStep*0.5*(QuantumForceOld[i,j]-QuantumForceNew[i,j])-\
                                      PositionNew[i,j]+PositionOld[i,j])
      
            GreensFunction = exp(GreensFunction)
            ProbabilityRatio = GreensFunction*wfnew**2/wfold**2
            #Metropolis-Hastings test to see whether we accept the move
            if random() <= ProbabilityRatio:
                for j in range(Dimension):
                    PositionOld[i,j] = PositionNew[i,j]
                    QuantumForceOld[i,j] = QuantumForceNew[i,j]
                wfold = wfnew
        DeltaE = LocalEnergy(PositionOld,alpha,beta)
        DerPsi = DerivativeWFansatz(PositionOld,alpha,beta)
        DeltaPsi += DerPsi
        energy += DeltaE
        DerivativePsiE += DerPsi*DeltaE
            
    # We calculate mean values
    energy /= NumberMCcycles
    DerivativePsiE /= NumberMCcycles
    DeltaPsi /= NumberMCcycles
    EnergyDer  = 2*(DerivativePsiE-DeltaPsi*energy)
    return energy, EnergyDer


#Here starts the main program with variable declarations
NumberParticles = 2
Dimension = 2
# guess for variational parameters
alpha = 0.9
beta = 0.2
# Set up iteration using gradient descent method
Energy = 0
EDerivative = np.zeros((2), np.double)
eta = 0.01
Niterations = 50
# 
for iter in range(Niterations):
    Energy, EDerivative = EnergyMinimization(alpha,beta)
    alphagradient = EDerivative[0]
    betagradient = EDerivative[1]
    alpha -= eta*alphagradient
    beta -= eta*betagradient 

print(alpha, beta)
print(Energy, EDerivative[0], EDerivative[1])




\end{minted}


% !split
\subsection*{Using Broyden's algorithm in scipy}
The following function uses the above described BFGS algorithm. Here we have defined a function which calculates the energy and a function which computes the first derivative.






























































































































































































\begin{minted}[fontsize=\fontsize{9pt}{9pt},linenos=false,mathescape,baselinestretch=1.0,fontfamily=tt,xleftmargin=7mm]{python}
# 2-electron VMC code for 2dim quantum dot with importance sampling
# Using gaussian rng for new positions and Metropolis- Hastings 
# Added energy minimization using the BFGS algorithm, see p. 136 of https://www.springer.com/it/book/9780387303031
from math import exp, sqrt
from random import random, seed, normalvariate
import numpy as np
import matplotlib.pyplot as plt
from mpl_toolkits.mplot3d import Axes3D
from matplotlib import cm
from matplotlib.ticker import LinearLocator, FormatStrFormatter
from scipy.optimize import minimize
import sys


# Trial wave function for the 2-electron quantum dot in two dims
def WaveFunction(r,alpha,beta):
    r1 = r[0,0]**2 + r[0,1]**2
    r2 = r[1,0]**2 + r[1,1]**2
    r12 = sqrt((r[0,0]-r[1,0])**2 + (r[0,1]-r[1,1])**2)
    deno = r12/(1+beta*r12)
    return exp(-0.5*alpha*(r1+r2)+deno)

# Local energy  for the 2-electron quantum dot in two dims, using analytical local energy
def LocalEnergy(r,alpha,beta):
    
    r1 = (r[0,0]**2 + r[0,1]**2)
    r2 = (r[1,0]**2 + r[1,1]**2)
    r12 = sqrt((r[0,0]-r[1,0])**2 + (r[0,1]-r[1,1])**2)
    deno = 1.0/(1+beta*r12)
    deno2 = deno*deno
    return 0.5*(1-alpha*alpha)*(r1 + r2) +2.0*alpha + 1.0/r12+deno2*(alpha*r12-deno2+2*beta*deno-1.0/r12)

# Derivate of wave function ansatz as function of variational parameters
def DerivativeWFansatz(r,alpha,beta):
    
    WfDer  = np.zeros((2), np.double)
    r1 = (r[0,0]**2 + r[0,1]**2)
    r2 = (r[1,0]**2 + r[1,1]**2)
    r12 = sqrt((r[0,0]-r[1,0])**2 + (r[0,1]-r[1,1])**2)
    deno = 1.0/(1+beta*r12)
    deno2 = deno*deno
    WfDer[0] = -0.5*(r1+r2)
    WfDer[1] = -r12*r12*deno2
    return  WfDer

# Setting up the quantum force for the two-electron quantum dot, recall that it is a vector
def QuantumForce(r,alpha,beta):

    qforce = np.zeros((NumberParticles,Dimension), np.double)
    r12 = sqrt((r[0,0]-r[1,0])**2 + (r[0,1]-r[1,1])**2)
    deno = 1.0/(1+beta*r12)
    qforce[0,:] = -2*r[0,:]*alpha*(r[0,:]-r[1,:])*deno*deno/r12
    qforce[1,:] = -2*r[1,:]*alpha*(r[1,:]-r[0,:])*deno*deno/r12
    return qforce
    

# Computing the derivative of the energy and the energy 
def EnergyDerivative(x0):

    
    # Parameters in the Fokker-Planck simulation of the quantum force
    D = 0.5
    TimeStep = 0.05
    NumberMCcycles= 10000
    # positions
    PositionOld = np.zeros((NumberParticles,Dimension), np.double)
    PositionNew = np.zeros((NumberParticles,Dimension), np.double)
    # Quantum force
    QuantumForceOld = np.zeros((NumberParticles,Dimension), np.double)
    QuantumForceNew = np.zeros((NumberParticles,Dimension), np.double)

    energy = 0.0
    DeltaE = 0.0
    alpha = x0[0]
    beta = x0[1]
    EnergyDer = 0.0
    DeltaPsi = 0.0
    DerivativePsiE = 0.0 
    #Initial position
    for i in range(NumberParticles):
        for j in range(Dimension):
            PositionOld[i,j] = normalvariate(0.0,1.0)*sqrt(TimeStep)
    wfold = WaveFunction(PositionOld,alpha,beta)
    QuantumForceOld = QuantumForce(PositionOld,alpha, beta)

    #Loop over MC MCcycles
    for MCcycle in range(NumberMCcycles):
        #Trial position moving one particle at the time
        for i in range(NumberParticles):
            for j in range(Dimension):
                PositionNew[i,j] = PositionOld[i,j]+normalvariate(0.0,1.0)*sqrt(TimeStep)+\
                                       QuantumForceOld[i,j]*TimeStep*D
            wfnew = WaveFunction(PositionNew,alpha,beta)
            QuantumForceNew = QuantumForce(PositionNew,alpha, beta)
            GreensFunction = 0.0
            for j in range(Dimension):
                GreensFunction += 0.5*(QuantumForceOld[i,j]+QuantumForceNew[i,j])*\
	                              (D*TimeStep*0.5*(QuantumForceOld[i,j]-QuantumForceNew[i,j])-\
                                      PositionNew[i,j]+PositionOld[i,j])
      
            GreensFunction = exp(GreensFunction)
            ProbabilityRatio = GreensFunction*wfnew**2/wfold**2
            #Metropolis-Hastings test to see whether we accept the move
            if random() <= ProbabilityRatio:
                for j in range(Dimension):
                    PositionOld[i,j] = PositionNew[i,j]
                    QuantumForceOld[i,j] = QuantumForceNew[i,j]
                wfold = wfnew
        DeltaE = LocalEnergy(PositionOld,alpha,beta)
        DerPsi = DerivativeWFansatz(PositionOld,alpha,beta)
        DeltaPsi += DerPsi
        energy += DeltaE
        DerivativePsiE += DerPsi*DeltaE
            
    # We calculate mean values
    energy /= NumberMCcycles
    DerivativePsiE /= NumberMCcycles
    DeltaPsi /= NumberMCcycles
    EnergyDer  = 2*(DerivativePsiE-DeltaPsi*energy)
    return EnergyDer


# Computing the expectation value of the local energy 
def Energy(x0):
    # Parameters in the Fokker-Planck simulation of the quantum force
    D = 0.5
    TimeStep = 0.05
    # positions
    PositionOld = np.zeros((NumberParticles,Dimension), np.double)
    PositionNew = np.zeros((NumberParticles,Dimension), np.double)
    # Quantum force
    QuantumForceOld = np.zeros((NumberParticles,Dimension), np.double)
    QuantumForceNew = np.zeros((NumberParticles,Dimension), np.double)

    energy = 0.0
    DeltaE = 0.0
    alpha = x0[0]
    beta = x0[1]
    NumberMCcycles= 10000
    #Initial position
    for i in range(NumberParticles):
        for j in range(Dimension):
            PositionOld[i,j] = normalvariate(0.0,1.0)*sqrt(TimeStep)
    wfold = WaveFunction(PositionOld,alpha,beta)
    QuantumForceOld = QuantumForce(PositionOld,alpha, beta)

    #Loop over MC MCcycles
    for MCcycle in range(NumberMCcycles):
        #Trial position moving one particle at the time
        for i in range(NumberParticles):
            for j in range(Dimension):
                PositionNew[i,j] = PositionOld[i,j]+normalvariate(0.0,1.0)*sqrt(TimeStep)+\
                                       QuantumForceOld[i,j]*TimeStep*D
            wfnew = WaveFunction(PositionNew,alpha,beta)
            QuantumForceNew = QuantumForce(PositionNew,alpha, beta)
            GreensFunction = 0.0
            for j in range(Dimension):
                GreensFunction += 0.5*(QuantumForceOld[i,j]+QuantumForceNew[i,j])*\
	                              (D*TimeStep*0.5*(QuantumForceOld[i,j]-QuantumForceNew[i,j])-\
                                      PositionNew[i,j]+PositionOld[i,j])
      
            GreensFunction = exp(GreensFunction)
            ProbabilityRatio = GreensFunction*wfnew**2/wfold**2
            #Metropolis-Hastings test to see whether we accept the move
            if random() <= ProbabilityRatio:
                for j in range(Dimension):
                    PositionOld[i,j] = PositionNew[i,j]
                    QuantumForceOld[i,j] = QuantumForceNew[i,j]
                wfold = wfnew
        DeltaE = LocalEnergy(PositionOld,alpha,beta)
        energy += DeltaE
            
    # We calculate mean values
    energy /= NumberMCcycles
    return energy




#Here starts the main program with variable declarations
NumberParticles = 2
Dimension = 2
# seed for rng generator 
seed()
# guess for variational parameters
x0 = np.array([0.9,0.2])
# Using Broydens method
res = minimize(Energy, x0, method='BFGS', jac=EnergyDerivative, options={'gtol': 1e-4,'disp': True})
print(res.x)

\end{minted}

Note that the \textbf{minimize} function returns the finale values for the variable $\alpha=x0[0]$ and $\beta=x0[1]$ in the array $x$. 

% !split
\subsection*{Brief reminder on Newton-Raphson's method}

Let us quickly remind ourselves how we derive the above method.

Perhaps the most celebrated of all one-dimensional root-finding
routines is Newton's method, also called the Newton-Raphson
method. This method  requires the evaluation of both the
function $f$ and its derivative $f'$ at arbitrary points. 
If you can only calculate the derivative
numerically and/or your function is not of the smooth type, we
normally discourage the use of this method.

% !split
\subsection*{The equations}

The Newton-Raphson formula consists geometrically of extending the
tangent line at a current point until it crosses zero, then setting
the next guess to the abscissa of that zero-crossing.  The mathematics
behind this method is rather simple. Employing a Taylor expansion for
$x$ sufficiently close to the solution $s$, we have

\[
    f(s)=0=f(x)+(s-x)f'(x)+\frac{(s-x)^2}{2}f''(x) +\dots.
    \label{eq:taylornr}
\]

For small enough values of the function and for well-behaved
functions, the terms beyond linear are unimportant, hence we obtain

\[
   f(x)+(s-x)f'(x)\approx 0,
\]
yielding
\[
   s\approx x-\frac{f(x)}{f'(x)}.
\]

Having in mind an iterative procedure, it is natural to start iterating with
\[
   x_{n+1}=x_n-\frac{f(x_n)}{f'(x_n)}.
\]

% !split
\subsection*{Simple geometric interpretation}

The above is Newton-Raphson's method. It has a simple geometric
interpretation, namely $x_{n+1}$ is the point where the tangent from
$(x_n,f(x_n))$ crosses the $x$-axis.  Close to the solution,
Newton-Raphson converges fast to the desired result. However, if we
are far from a root, where the higher-order terms in the series are
important, the Newton-Raphson formula can give grossly inaccurate
results. For instance, the initial guess for the root might be so far
from the true root as to let the search interval include a local
maximum or minimum of the function.  If an iteration places a trial
guess near such a local extremum, so that the first derivative nearly
vanishes, then Newton-Raphson may fail totally

% !split
\subsection*{Extending to more than one variable}

Newton's method can be generalized to systems of several non-linear equations
and variables. Consider the case with two equations
\[
   \begin{array}{cc} f_1(x_1,x_2) &=0\\
                     f_2(x_1,x_2) &=0,\end{array}
\]
which we Taylor expand to obtain

\[
   \begin{array}{cc} 0=f_1(x_1+h_1,x_2+h_2)=&f_1(x_1,x_2)+h_1
                     \partial f_1/\partial x_1+h_2
                     \partial f_1/\partial x_2+\dots\\
                     0=f_2(x_1+h_1,x_2+h_2)=&f_2(x_1,x_2)+h_1
                     \partial f_2/\partial x_1+h_2
                     \partial f_2/\partial x_2+\dots
                       \end{array}.
\]
Defining the Jacobian matrix $\hat{J}$ we have
\[
 \hat{J}=\left( \begin{array}{cc}
                         \partial f_1/\partial x_1  & \partial f_1/\partial x_2 \\
                          \partial f_2/\partial x_1     &\partial f_2/\partial x_2
             \end{array} \right),
\]
we can rephrase Newton's method as
\[
\left(\begin{array}{c} x_1^{n+1} \\ x_2^{n+1} \end{array} \right)=
\left(\begin{array}{c} x_1^{n} \\ x_2^{n} \end{array} \right)+
\left(\begin{array}{c} h_1^{n} \\ h_2^{n} \end{array} \right),
\]
where we have defined
\[
   \left(\begin{array}{c} h_1^{n} \\ h_2^{n} \end{array} \right)=
   -{\bf \hat{J}}^{-1}
   \left(\begin{array}{c} f_1(x_1^{n},x_2^{n}) \\ f_2(x_1^{n},x_2^{n}) \end{array} \right).
\]
We need thus to compute the inverse of the Jacobian matrix and it
is to understand that difficulties  may
arise in case $\hat{J}$ is nearly singular.

It is rather straightforward to extend the above scheme to systems of
more than two non-linear equations. In our case, the Jacobian matrix is given by the Hessian that represents the second derivative of cost function. 

% !split
\subsection*{Steepest descent}

The basic idea of gradient descent is
that a function $F(\mathbf{x})$, 
$\mathbf{x} \equiv (x_1,\cdots,x_n)$, decreases fastest if one goes from $\bf {x}$ in the
direction of the negative gradient $-\nabla F(\mathbf{x})$.

It can be shown that if 
\[
\mathbf{x}_{k+1} = \mathbf{x}_k - \gamma_k \nabla F(\mathbf{x}_k),
\]
with $\gamma_k > 0$.

For $\gamma_k$ small enough, then $F(\mathbf{x}_{k+1}) \leq
F(\mathbf{x}_k)$. This means that for a sufficiently small $\gamma_k$
we are always moving towards smaller function values, i.e a minimum.

% !split 
\subsection*{More on Steepest descent}

The previous observation is the basis of the method of steepest
descent, which is also referred to as just gradient descent (GD). One
starts with an initial guess $\mathbf{x}_0$ for a minimum of $F$ and
computes new approximations according to

\[
\mathbf{x}_{k+1} = \mathbf{x}_k - \gamma_k \nabla F(\mathbf{x}_k), \ \ k \geq 0.
\]

The parameter $\gamma_k$ is often referred to as the step length or
the learning rate within the context of Machine Learning.

% !split 
\subsection*{The ideal}

Ideally the sequence $\{\mathbf{x}_k \}_{k=0}$ converges to a global
minimum of the function $F$. In general we do not know if we are in a
global or local minimum. In the special case when $F$ is a convex
function, all local minima are also global minima, so in this case
gradient descent can converge to the global solution. The advantage of
this scheme is that it is conceptually simple and straightforward to
implement. However the method in this form has some severe
limitations:

In machine learing we are often faced with non-convex high dimensional
cost functions with many local minima. Since GD is deterministic we
will get stuck in a local minimum, if the method converges, unless we
have a very good intial guess. This also implies that the scheme is
sensitive to the chosen initial condition.

Note that the gradient is a function of $\mathbf{x} =
(x_1,\cdots,x_n)$ which makes it expensive to compute numerically.

% !split 
\subsection*{The sensitiveness of the gradient descent}

The gradient descent method 
is sensitive to the choice of learning rate $\gamma_k$. This is due
to the fact that we are only guaranteed that $F(\mathbf{x}_{k+1}) \leq
F(\mathbf{x}_k)$ for sufficiently small $\gamma_k$. The problem is to
determine an optimal learning rate. If the learning rate is chosen too
small the method will take a long time to converge and if it is too
large we can experience erratic behavior.

Many of these shortcomings can be alleviated by introducing
randomness. One such method is that of Stochastic Gradient Descent
(SGD), see below.

% !split 
\subsection*{Convex functions}

Ideally we want our cost/loss function to be convex(concave).

First we give the definition of a convex set: A set $C$ in
$\mathbb{R}^n$ is said to be convex if, for all $x$ and $y$ in $C$ and
all $t \in (0,1)$ , the point $(1 − t)x + ty$ also belongs to
C. Geometrically this means that every point on the line segment
connecting $x$ and $y$ is in $C$ as discussed below.

The convex subsets of $\mathbb{R}$ are the intervals of
$\mathbb{R}$. Examples of convex sets of $\mathbb{R}^2$ are the
regular polygons (triangles, rectangles, pentagons, etc...).

% !split
\subsection*{Convex function}

\textbf{Convex function}: Let $X \subset \mathbb{R}^n$ be a convex set. Assume that the function $f: X \rightarrow \mathbb{R}$ is continuous, then $f$ is said to be convex if $$f(tx_1 + (1-t)x_2) \leq tf(x_1) + (1-t)f(x_2) $$ for all $x_1, x_2 \in X$ and for all $t \in [0,1]$. If $\leq$ is replaced with a strict inequaltiy in the definition, we demand $x_1 \neq x_2$ and $t\in(0,1)$ then $f$ is said to be strictly convex. For a single variable function, convexity means that if you draw a straight line connecting $f(x_1)$ and $f(x_2)$, the value of the function on the interval $[x_1,x_2]$ is always below the line as illustrated below.

% !split
\subsection*{Conditions on convex functions}

In the following we state first and second-order conditions which
ensures convexity of a function $f$. We write $D_f$ to denote the
domain of $f$, i.e the subset of $R^n$ where $f$ is defined. For more
details and proofs we refer to: \href{{http://stanford.edu/boyd/cvxbook/, 2004}}{S. Boyd and L. Vandenberghe. Convex Optimization. Cambridge University Press}.


% --- begin paragraph admon ---
\paragraph{First order condition.}
Suppose $f$ is differentiable (i.e $\nabla f(x)$ is well defined for
all $x$ in the domain of $f$). Then $f$ is convex if and only if $D_f$
is a convex set and $$f(y) \geq f(x) + \nabla f(x)^T (y-x) $$ holds
for all $x,y \in D_f$. This condition means that for a convex function
the first order Taylor expansion (right hand side above) at any point
a global under estimator of the function. To convince yourself you can
make a drawing of $f(x) = x^2+1$ and draw the tangent line to $f(x)$ and
note that it is always below the graph.
% --- end paragraph admon ---




% --- begin paragraph admon ---
\paragraph{Second order condition.}
Assume that $f$ is twice
differentiable, i.e the Hessian matrix exists at each point in
$D_f$. Then $f$ is convex if and only if $D_f$ is a convex set and its
Hessian is positive semi-definite for all $x\in D_f$. For a
single-variable function this reduces to $f''(x) \geq 0$. Geometrically this means that $f$ has nonnegative curvature
everywhere.
% --- end paragraph admon ---



This condition is particularly useful since it gives us an procedure for determining if the function under consideration is convex, apart from using the definition.

% !split
\subsection*{More on convex functions}

The next result is of great importance to us and the reason why we are
going on about convex functions. In machine learning we frequently
have to minimize a loss/cost function in order to find the best
parameters for the model we are considering. 

Ideally we want the
global minimum (for high-dimensional models it is hard to know
if we have local or global minimum). However, if the cost/loss function
is convex the following result provides invaluable information:


% --- begin paragraph admon ---
\paragraph{Any minimum is global for convex functions.}
Consider the problem of finding $x \in \mathbb{R}^n$ such that $f(x)$
is minimal, where $f$ is convex and differentiable. Then, any point
$x^*$ that satisfies $\nabla f(x^*) = 0$ is a global minimum.
% --- end paragraph admon ---



This result means that if we know that the cost/loss function is convex and we are able to find a minimum, we are guaranteed that it is a global minimum.

% !split
\subsection*{Some simple problems}

\begin{enumerate}
\item Show that $f(x)=x^2$ is convex for $x \in \mathbb{R}$ using the definition of convexity. Hint: If you re-write the definition, $f$ is convex if the following holds for all $x,y \in D_f$ and any $\lambda \in [0,1]$ $\lambda f(x)+(1-\lambda)f(y)-f(\lambda x + (1-\lambda) y ) \geq 0$.

\item Using the second order condition show that the following functions are convex on the specified domain.
\begin{itemize}

 \item $f(x) = e^x$ is convex for $x \in \mathbb{R}$.

 \item $g(x) = -\ln(x)$ is convex for $x \in (0,\infty)$.

\end{itemize}

\noindent
\item Let $f(x) = x^2$ and $g(x) = e^x$. Show that $f(g(x))$ and $g(f(x))$ is convex for $x \in \mathbb{R}$. Also show that if $f(x)$ is any convex function than $h(x) = e^{f(x)}$ is convex.

\item A norm is any function that satisfy the following properties
\begin{itemize}

 \item $f(\alpha x) = |\alpha| f(x)$ for all $\alpha \in \mathbb{R}$.

 \item $f(x+y) \leq f(x) + f(y)$

 \item $f(x) \leq 0$ for all $x \in \mathbb{R}^n$ with equality if and only if $x = 0$
\end{itemize}

\noindent
\end{enumerate}

\noindent
Using the definition of convexity, try to show that a function satisfying the properties above is convex (the third condition is not needed to show this).

% !split
\subsection*{Standard steepest descent}

Before we proceed, we would like to discuss the approach called the
\textbf{standard Steepest descent}, which again leads to us having to be able
to compute a matrix. It belongs to the class of Conjugate Gradient methods (CG).

\href{{https://www.cs.cmu.edu/~quake-papers/painless-conjugate-gradient.pdf}}{The success of the CG method}
for finding solutions of non-linear problems is based on the theory
of conjugate gradients for linear systems of equations. It belongs to
the class of iterative methods for solving problems from linear
algebra of the type 
\begin{equation*} 
\hat{A}\hat{x} = \hat{b}.
\end{equation*} 

In the iterative process we end up with a problem like

\begin{equation*}
  \hat{r}= \hat{b}-\hat{A}\hat{x},
\end{equation*}
where $\hat{r}$ is the so-called residual or error in the iterative process.

When we have found the exact solution, $\hat{r}=0$.

% !split
\subsection*{Gradient method}

The residual is zero when we reach the minimum of the quadratic equation
\begin{equation*}
  P(\hat{x})=\frac{1}{2}\hat{x}^T\hat{A}\hat{x} - \hat{x}^T\hat{b},
\end{equation*}

with the constraint that the matrix $\hat{A}$ is positive definite and
symmetric.  This defines also the Hessian and we want it to be  positive definite.  

% !split
\subsection*{Steepest descent  method}

We denote the initial guess for $\hat{x}$ as $\hat{x}_0$. 
We can assume without loss of generality that
\begin{equation*}
\hat{x}_0=0,
\end{equation*}
or consider the system
\begin{equation*}
\hat{A}\hat{z} = \hat{b}-\hat{A}\hat{x}_0,
\end{equation*}
instead.

% !split
\subsection*{Steepest descent  method}

% --- begin paragraph admon ---
\paragraph{}
One can show that the solution $\hat{x}$ is also the unique minimizer of the quadratic form
\begin{equation*}
  f(\hat{x}) = \frac{1}{2}\hat{x}^T\hat{A}\hat{x} - \hat{x}^T \hat{x} , \quad \hat{x}\in\mathbf{R}^n. 
\end{equation*}
This suggests taking the first basis vector $\hat{r}_1$ (see below for definition) 
to be the gradient of $f$ at $\hat{x}=\hat{x}_0$, 
which equals
\begin{equation*}
\hat{A}\hat{x}_0-\hat{b},
\end{equation*}
and 
$\hat{x}_0=0$ it is equal $-\hat{b}$.
% --- end paragraph admon ---



% !split
\subsection*{Final expressions}

% --- begin paragraph admon ---
\paragraph{}
We can compute the residual iteratively as
\begin{equation*}
\hat{r}_{k+1}=\hat{b}-\hat{A}\hat{x}_{k+1},
 \end{equation*}
which equals
\begin{equation*}
\hat{b}-\hat{A}(\hat{x}_k+\alpha_k\hat{r}_k),
 \end{equation*}
or
\begin{equation*}
(\hat{b}-\hat{A}\hat{x}_k)-\alpha_k\hat{A}\hat{r}_k,
 \end{equation*}
which gives

\[
\alpha_k = \frac{\hat{r}_k^T\hat{r}_k}{\hat{r}_k^T\hat{A}\hat{r}_k}
\]
leading to the iterative scheme
\begin{equation*}
\hat{x}_{k+1}=\hat{x}_k-\alpha_k\hat{r}_{k},
 \end{equation*}
% --- end paragraph admon ---



% !split
\subsection*{Conjugate gradient method}

% --- begin paragraph admon ---
\paragraph{}
In the CG method we define so-called conjugate directions and two vectors 
$\hat{s}$ and $\hat{t}$
are said to be
conjugate if
\begin{equation*}
\hat{s}^T\hat{A}\hat{t}= 0.
\end{equation*}
The philosophy of the CG method is to perform searches in various conjugate directions
of our vectors $\hat{x}_i$ obeying the above criterion, namely
\begin{equation*}
\hat{x}_i^T\hat{A}\hat{x}_j= 0.
\end{equation*}
Two vectors are conjugate if they are orthogonal with respect to 
this inner product. Being conjugate is a symmetric relation: if $\hat{s}$ is conjugate to $\hat{t}$, then $\hat{t}$ is conjugate to $\hat{s}$.
% --- end paragraph admon ---



% !split
\subsection*{Conjugate gradient method}

% --- begin paragraph admon ---
\paragraph{}
An example is given by the eigenvectors of the matrix
\begin{equation*}
\hat{v}_i^T\hat{A}\hat{v}_j= \lambda\hat{v}_i^T\hat{v}_j,
\end{equation*}
which is zero unless $i=j$.
% --- end paragraph admon ---



% !split
\subsection*{Conjugate gradient method}

% --- begin paragraph admon ---
\paragraph{}
Assume now that we have a symmetric positive-definite matrix $\hat{A}$ of size
$n\times n$. At each iteration $i+1$ we obtain the conjugate direction of a vector
\begin{equation*}
\hat{x}_{i+1}=\hat{x}_{i}+\alpha_i\hat{p}_{i}. 
\end{equation*}
We assume that $\hat{p}_{i}$ is a sequence of $n$ mutually conjugate directions. 
Then the $\hat{p}_{i}$  form a basis of $R^n$ and we can expand the solution 
$  \hat{A}\hat{x} = \hat{b}$ in this basis, namely

\begin{equation*}
  \hat{x}  = \sum^{n}_{i=1} \alpha_i \hat{p}_i.
\end{equation*}
% --- end paragraph admon ---



% !split
\subsection*{Conjugate gradient method}

% --- begin paragraph admon ---
\paragraph{}
The coefficients are given by
\begin{equation*}
    \mathbf{A}\mathbf{x} = \sum^{n}_{i=1} \alpha_i \mathbf{A} \mathbf{p}_i = \mathbf{b}.
\end{equation*}
Multiplying with $\hat{p}_k^T$  from the left gives

\begin{equation*}
  \hat{p}_k^T \hat{A}\hat{x} = \sum^{n}_{i=1} \alpha_i\hat{p}_k^T \hat{A}\hat{p}_i= \hat{p}_k^T \hat{b},
\end{equation*}
and we can define the coefficients $\alpha_k$ as

\begin{equation*}
    \alpha_k = \frac{\hat{p}_k^T \hat{b}}{\hat{p}_k^T \hat{A} \hat{p}_k}
\end{equation*}
% --- end paragraph admon ---



% !split
\subsection*{Conjugate gradient method and iterations}

% --- begin paragraph admon ---
\paragraph{}

If we choose the conjugate vectors $\hat{p}_k$ carefully, 
then we may not need all of them to obtain a good approximation to the solution 
$\hat{x}$. 
We want to regard the conjugate gradient method as an iterative method. 
This will us to solve systems where $n$ is so large that the direct 
method would take too much time.

We denote the initial guess for $\hat{x}$ as $\hat{x}_0$. 
We can assume without loss of generality that
\begin{equation*}
\hat{x}_0=0,
\end{equation*}
or consider the system
\begin{equation*}
\hat{A}\hat{z} = \hat{b}-\hat{A}\hat{x}_0,
\end{equation*}
instead.
% --- end paragraph admon ---



% !split
\subsection*{Conjugate gradient method}

% --- begin paragraph admon ---
\paragraph{}
One can show that the solution $\hat{x}$ is also the unique minimizer of the quadratic form
\begin{equation*}
  f(\hat{x}) = \frac{1}{2}\hat{x}^T\hat{A}\hat{x} - \hat{x}^T \hat{x} , \quad \hat{x}\in\mathbf{R}^n. 
\end{equation*}
This suggests taking the first basis vector $\hat{p}_1$ 
to be the gradient of $f$ at $\hat{x}=\hat{x}_0$, 
which equals
\begin{equation*}
\hat{A}\hat{x}_0-\hat{b},
\end{equation*}
and 
$\hat{x}_0=0$ it is equal $-\hat{b}$.
The other vectors in the basis will be conjugate to the gradient, 
hence the name conjugate gradient method.
% --- end paragraph admon ---



% !split
\subsection*{Conjugate gradient method}

% --- begin paragraph admon ---
\paragraph{}
Let  $\hat{r}_k$ be the residual at the $k$-th step:
\begin{equation*}
\hat{r}_k=\hat{b}-\hat{A}\hat{x}_k.
\end{equation*}
Note that $\hat{r}_k$ is the negative gradient of $f$ at 
$\hat{x}=\hat{x}_k$, 
so the gradient descent method would be to move in the direction $\hat{r}_k$. 
Here, we insist that the directions $\hat{p}_k$ are conjugate to each other, 
so we take the direction closest to the gradient $\hat{r}_k$  
under the conjugacy constraint. 
This gives the following expression
\begin{equation*}
\hat{p}_{k+1}=\hat{r}_k-\frac{\hat{p}_k^T \hat{A}\hat{r}_k}{\hat{p}_k^T\hat{A}\hat{p}_k} \hat{p}_k.
\end{equation*}
% --- end paragraph admon ---



% !split
\subsection*{Conjugate gradient method}

% --- begin paragraph admon ---
\paragraph{}
We can also  compute the residual iteratively as
\begin{equation*}
\hat{r}_{k+1}=\hat{b}-\hat{A}\hat{x}_{k+1},
 \end{equation*}
which equals
\begin{equation*}
\hat{b}-\hat{A}(\hat{x}_k+\alpha_k\hat{p}_k),
 \end{equation*}
or
\begin{equation*}
(\hat{b}-\hat{A}\hat{x}_k)-\alpha_k\hat{A}\hat{p}_k,
 \end{equation*}
which gives

\begin{equation*}
\hat{r}_{k+1}=\hat{r}_k-\hat{A}\hat{p}_{k},
 \end{equation*}
% --- end paragraph admon ---



% !split
\subsection*{Broyden–Fletcher–Goldfarb–Shanno algorithm}

% --- begin paragraph admon ---
\paragraph{}
The optimization problem is to minimize $f(\mathbf {x} )$ where $\mathbf {x}$  is a vector in $R^{n}$, and $f$ is a differentiable scalar function. There are no constraints on the values that  $\mathbf {x}$  can take.

The algorithm begins at an initial estimate for the optimal value $\mathbf {x}_{0}$ and proceeds iteratively to get a better estimate at each stage.

The search direction $p_k$ at stage $k$ is given by the solution of the analogue of the Newton equation
\[
B_{k}\mathbf {p} _{k}=-\nabla f(\mathbf {x}_{k}),
\]

where $B_{k}$ is an approximation to the Hessian matrix, which is
updated iteratively at each stage, and $\nabla f(\mathbf {x} _{k})$
is the gradient of the function
evaluated at $x_k$. 
A line search in the direction $p_k$ is then used to
find the next point $x_{k+1}$ by minimising 
\[
f(\mathbf {x}_{k}+\alpha \mathbf {p}_{k}),
\]
over the scalar $\alpha > 0$.
% --- end paragraph admon ---



% !split
\subsection*{Codes from numerical recipes}

% --- begin paragraph admon ---
\paragraph{}
You can use codes we have adapted from the text \href{{http://www.nr.com/}}{Numerical Recipes in C++}, see chapter 10.7.  
Here we present a program, which you also can find at the webpage of the course we use the functions \textbf{dfpmin} and \textbf{lnsrch}.  This is a variant of the Broyden et al algorithm discussed in the previous slide.

\begin{itemize}
\item The program uses the harmonic oscillator in one dimensions as example.

\item The program does not use armadillo to handle vectors and matrices, but employs rather my own vector-matrix class. These auxiliary functions, and the main program \emph{model.cpp} can all be found under the \href{{https://github.com/CompPhysics/ComputationalPhysics2/tree/gh-pages/doc/pub/cg/programs/c%2B%2B}}{program link here}.
\end{itemize}

\noindent
Below we show only excerpts from the main program. For the full program, see the above link.
% --- end paragraph admon ---



% !split
\subsection*{Finding the minimum of the harmonic oscillator model in one dimension}

% --- begin paragraph admon ---
\paragraph{}























\begin{minted}[fontsize=\fontsize{9pt}{9pt},linenos=false,mathescape,baselinestretch=1.0,fontfamily=tt,xleftmargin=7mm]{c++}
//   Main function begins here
int main()
{
     int n, iter;
     double gtol, fret;
     double alpha;
     n = 1;
//   reserve space in memory for vectors containing the variational
//   parameters
     Vector g(n), p(n);
     cout << "Read in guess for alpha" << endl;
     cin >> alpha;
     gtol = 1.0e-5;
//   now call dfmin and compute the minimum
     p(0) = alpha;
     dfpmin(p, n, gtol, &iter, &fret, Efunction, dEfunction);
     cout << "Value of energy minimum = " << fret << endl;
     cout << "Number of iterations = " << iter << endl;
     cout << "Value of alpha at minimum = " << p(0) << endl;
      return 0;
}  // end of main program


\end{minted}
% --- end paragraph admon ---



% !split
\subsection*{Functions to observe}

% --- begin paragraph admon ---
\paragraph{}
The functions \textbf{Efunction} and \textbf{dEfunction} compute the expectation value of the energy and its derivative.
They use the the quasi-Newton method of \href{{https://www.springer.com/it/book/9780387303031}}{Broyden, Fletcher, Goldfarb, and Shanno (BFGS)}
It uses the first derivatives only. The BFGS algorithm has proven good performance even for non-smooth optimizations. 
These functions need to be changed when you want to your own derivatives.













\begin{minted}[fontsize=\fontsize{9pt}{9pt},linenos=false,mathescape,baselinestretch=1.0,fontfamily=tt,xleftmargin=7mm]{c++}
//  this function defines the expectation value of the local energy
double Efunction(Vector  &x)
{
  double value = x(0)*x(0)*0.5+1.0/(8*x(0)*x(0));
  return value;
} // end of function to evaluate

//  this function defines the derivative of the energy 
void dEfunction(Vector &x, Vector &g)
{
  g(0) = x(0)-1.0/(4*x(0)*x(0)*x(0));
} // end of function to evaluate

\end{minted}

You need to change these functions in order to compute the local energy for your system. I used 1000
cycles per call to get a new value of $\langle E_L[\alpha]\rangle$.
When I compute the local energy I also compute its derivative.
After roughly 10-20 iterations I got a converged result in terms of $\alpha$.
% --- end paragraph admon ---




% ------------------- end of main content ---------------

\end{document}

