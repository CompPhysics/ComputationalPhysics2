%%
%% Automatically generated file from DocOnce source
%% (https://github.com/hplgit/doconce/)
%%
%%
% #ifdef PTEX2TEX_EXPLANATION
%%
%% The file follows the ptex2tex extended LaTeX format, see
%% ptex2tex: http://code.google.com/p/ptex2tex/
%%
%% Run
%%      ptex2tex myfile
%% or
%%      doconce ptex2tex myfile
%%
%% to turn myfile.p.tex into an ordinary LaTeX file myfile.tex.
%% (The ptex2tex program: http://code.google.com/p/ptex2tex)
%% Many preprocess options can be added to ptex2tex or doconce ptex2tex
%%
%%      ptex2tex -DMINTED myfile
%%      doconce ptex2tex myfile envir=minted
%%
%% ptex2tex will typeset code environments according to a global or local
%% .ptex2tex.cfg configure file. doconce ptex2tex will typeset code
%% according to options on the command line (just type doconce ptex2tex to
%% see examples). If doconce ptex2tex has envir=minted, it enables the
%% minted style without needing -DMINTED.
% #endif

% #define PREAMBLE

% #ifdef PREAMBLE
%-------------------- begin preamble ----------------------

\documentclass[%
oneside,                 % oneside: electronic viewing, twoside: printing
final,                   % draft: marks overfull hboxes, figures with paths
10pt]{article}

\listfiles               %  print all files needed to compile this document

\usepackage{relsize,makeidx,color,setspace,amsmath,amsfonts,amssymb}
\usepackage[table]{xcolor}
\usepackage{bm,ltablex,microtype}

\usepackage[pdftex]{graphicx}

\usepackage{ptex2tex}
% #ifdef MINTED
\usepackage{minted}
\usemintedstyle{default}
% #endif

\usepackage[T1]{fontenc}
%\usepackage[latin1]{inputenc}
\usepackage{ucs}
\usepackage[utf8x]{inputenc}

\usepackage{lmodern}         % Latin Modern fonts derived from Computer Modern

% Hyperlinks in PDF:
\definecolor{linkcolor}{rgb}{0,0,0.4}
\usepackage{hyperref}
\hypersetup{
    breaklinks=true,
    colorlinks=true,
    linkcolor=linkcolor,
    urlcolor=linkcolor,
    citecolor=black,
    filecolor=black,
    %filecolor=blue,
    pdfmenubar=true,
    pdftoolbar=true,
    bookmarksdepth=3   % Uncomment (and tweak) for PDF bookmarks with more levels than the TOC
    }
%\hyperbaseurl{}   % hyperlinks are relative to this root

\setcounter{tocdepth}{2}  % levels in table of contents

% --- fancyhdr package for fancy headers ---
\usepackage{fancyhdr}
\fancyhf{} % sets both header and footer to nothing
\renewcommand{\headrulewidth}{0pt}
\fancyfoot[LE,RO]{\thepage}
% Ensure copyright on titlepage (article style) and chapter pages (book style)
\fancypagestyle{plain}{
  \fancyhf{}
  \fancyfoot[C]{{\footnotesize \copyright\ 1999-2019, Morten Hjorth-Jensen. Released under CC Attribution-NonCommercial 4.0 license}}
%  \renewcommand{\footrulewidth}{0mm}
  \renewcommand{\headrulewidth}{0mm}
}
% Ensure copyright on titlepages with \thispagestyle{empty}
\fancypagestyle{empty}{
  \fancyhf{}
  \fancyfoot[C]{{\footnotesize \copyright\ 1999-2019, Morten Hjorth-Jensen. Released under CC Attribution-NonCommercial 4.0 license}}
  \renewcommand{\footrulewidth}{0mm}
  \renewcommand{\headrulewidth}{0mm}
}

\pagestyle{fancy}


\usepackage[framemethod=TikZ]{mdframed}

% --- begin definitions of admonition environments ---

% --- end of definitions of admonition environments ---

% prevent orhpans and widows
\clubpenalty = 10000
\widowpenalty = 10000

\newenvironment{doconceexercise}{}{}
\newcounter{doconceexercisecounter}


% ------ header in subexercises ------
%\newcommand{\subex}[1]{\paragraph{#1}}
%\newcommand{\subex}[1]{\par\vspace{1.7mm}\noindent{\bf #1}\ \ }
\makeatletter
% 1.5ex is the spacing above the header, 0.5em the spacing after subex title
\newcommand\subex{\@startsection{paragraph}{4}{\z@}%
                  {1.5ex\@plus1ex \@minus.2ex}%
                  {-0.5em}%
                  {\normalfont\normalsize\bfseries}}
\makeatother


% --- end of standard preamble for documents ---


% insert custom LaTeX commands...

\raggedbottom
\makeindex
\usepackage[totoc]{idxlayout}   % for index in the toc
\usepackage[nottoc]{tocbibind}  % for references/bibliography in the toc

%-------------------- end preamble ----------------------

\begin{document}

% matching end for #ifdef PREAMBLE
% #endif

\newcommand{\exercisesection}[1]{\subsection*{#1}}


% ------------------- main content ----------------------

% rewrite steepest descent with code example



% ----------------- title -------------------------

\thispagestyle{empty}

\begin{center}
{\LARGE\bf
\begin{spacing}{1.25}
Conjugate gradient methods and other optimization methods
\end{spacing}
}
\end{center}

% ----------------- author(s) -------------------------

\begin{center}
{\bf Morten Hjorth-Jensen${}^{1, 2}$} \\ [0mm]
\end{center}

\begin{center}
% List of all institutions:
\centerline{{\small ${}^1$Department of Physics, University of Oslo}}
\centerline{{\small ${}^2$Department of Physics and Astronomy and National Superconducting Cyclotron Laboratory, Michigan State University}}
\end{center}
    
% ----------------- end author(s) -------------------------

% --- begin date ---
\begin{center}
Feb 4, 2019
\end{center}
% --- end date ---

\vspace{1cm}


% !split
\subsection{Motivation}

% --- begin paragraph admon ---
\paragraph{}
Our aim with this part of the project is to be able to
\begin{itemize}
\item find an optimal value for the variational parameters using only some few Monte Carlo cycles

\item use these optimal values for the variational parameters to perform a large-scale Monte Carlo calculation
\end{itemize}

\noindent
To achieve this will look at methods like \emph{Steepest descent} and the \emph{conjugate gradient method}. Both these methods allow us to find
the minima of a multivariable  function like our energy (function of several variational parameters). 
Alternatively, you can always use Newton's method. In particular, since we will normally have one variational parameter,
Newton's method can be easily used in finding the minimum of the local energy.
% --- end paragraph admon ---



% !split
\subsection{Simple example and demonstration}

% --- begin paragraph admon ---
\paragraph{}
Let us illustrate what is needed in our calculations using a simple example, the harmonic oscillator in one dimension.
For the harmonic oscillator in one-dimension we have a  trial wave function and probability
\begin{equation*}
\psi_T(x) = e^{-\alpha^2 x^2} \qquad P_T(x)dx = \frac{e^{-2\alpha^2 x^2}dx}{\int dx e^{-2\alpha^2 x^2}}
\end{equation*}
with $\alpha$ being the variational parameter. 
We obtain then the following local energy
\begin{equation*}
E_L[\alpha] = \alpha^2+x^2\left(\frac{1}{2}-2\alpha^2\right),
\end{equation*}
which results in the expectation value for the local energy
\begin{equation*}
\langle  E_L[\alpha]\rangle = \frac{1}{2}\alpha^2+\frac{1}{8\alpha^2}
\end{equation*}
% --- end paragraph admon ---





% !split
\subsection{Simple example and demonstration}

% --- begin paragraph admon ---
\paragraph{}
The derivative of the energy with respect to $\alpha$ gives
\begin{equation*}
\frac{d\langle  E_L[\alpha]\rangle}{d\alpha} = \alpha-\frac{1}{4\alpha^3}
\end{equation*}
and a second derivative which is always positive (meaning that we find a minimum)
\begin{equation*}
\frac{d^2\langle  E_L[\alpha]\rangle}{d\alpha^2} = 1+\frac{3}{4\alpha^4}
\end{equation*}
The condition
\begin{equation*}
\frac{d\langle  E_L[\alpha]\rangle}{d\alpha} = 0,
\end{equation*}
gives the optimal $\alpha=1/\sqrt{2}$, as expected.
% --- end paragraph admon ---



% !split


% --- begin exercise ---
\begin{doconceexercise}
\refstepcounter{doconceexercisecounter}

\exercisesection{Exercise \thedoconceexercisecounter: Find the local energy for the harmonic oscillator}



\subex{a)}
Derive the local energy for the harmonic oscillator in one dimension and find its expectation value.

\subex{b)}
Show also that the optimal value of optimal $\alpha=1/\sqrt{2}$.

\subex{c)}
Repeat the above steps in two dimensions for $N$ bosons or electrons. What is the optimal value of $\alpha$?


\end{doconceexercise}
% --- end exercise ---


% !split
\subsection{Variance in the simple model}

% --- begin paragraph admon ---
\paragraph{}
We can also minimize the variance. In our simple model the variance is
\begin{equation*}
\sigma^2[\alpha] = \frac{1}{2}\alpha^4-\frac{1}{4}+\frac{1}{32\alpha^4},
\end{equation*}
with first derivative
\begin{equation*}
\frac{d \sigma^2[\alpha]}{d\alpha} = 2\alpha^3-\frac{1}{8\alpha^5}
\end{equation*}
and a second derivative which is always positive
\begin{equation*}
\frac{d^2\sigma^2[\alpha]}{d\alpha^2} = 6\alpha^2+\frac{5}{8\alpha^6}
\end{equation*}
% --- end paragraph admon ---




% !split
\subsection{Computing the derivatives}

% --- begin paragraph admon ---
\paragraph{}
In general we end up computing the expectation value of the energy in terms 
of some parameters $\alpha_0,\alpha_1,\dots,\alpha_n$
and we search for a minimum in this multi-variable parameter space.  
This leads to an energy minimization problem \emph{where we need the derivative of the energy as a function of the variational parameters}.

In the above example this was easy and we were able to find the expression for the derivative by simple derivations. 
However, in our actual calculations the energy is represented by a multi-dimensional integral with several variational parameters.
How can we can then obtain the derivatives of the energy with respect to the variational parameters without having 
to resort to expensive numerical derivations?
% --- end paragraph admon ---




% !split
\subsection{Expressions for finding the derivatives of the local energy}

% --- begin paragraph admon ---
\paragraph{}

To find the derivatives of the local energy expectation value as function of the variational parameters, we can use the chain rule and the hermiticity of the Hamiltonian.  

Let us define 
\[
\bar{E}_{\alpha}=\frac{d\langle  E_L[\alpha]\rangle}{d\alpha}.
\]
as the derivative of the energy with respect to the variational parameter $\alpha$ (we limit ourselves to one parameter only).
In the above example this was easy and we obtain a simple expression for the derivative.
We define also the derivative of the trial function (skipping the subindex $T$) as 
\[
\bar{\psi}_{\alpha}=\frac{d\psi[\alpha]\rangle}{d\alpha}.
\]
% --- end paragraph admon ---





% !split
\subsection{Derivatives of the local energy}

% --- begin paragraph admon ---
\paragraph{}
The elements of the gradient of the local energy are then (using the chain rule and the hermiticity of the Hamiltonian)
\[
\bar{E}_{\alpha} = 2\left( \langle \frac{\bar{\psi}_{\alpha}}{\psi[\alpha]}E_L[\alpha]\rangle -\langle \frac{\bar{\psi}_{\alpha}}{\psi[\alpha]}\rangle\langle E_L[\alpha] \rangle\right).
\]
From a computational point of view it means that you need to compute the expectation values of 
\[
\langle \frac{\bar{\psi}_{\alpha}}{\psi[\alpha]}E_L[\alpha]\rangle,
\]
and
\[
\langle \frac{\bar{\psi}_{\alpha}}{\psi[\alpha]}\rangle\langle E_L[\alpha]\rangle
\]
% --- end paragraph admon ---




% !split


% --- begin exercise ---
\begin{doconceexercise}
\refstepcounter{doconceexercisecounter}

\exercisesection{Exercise \thedoconceexercisecounter: General expression for the derivative of the energy}



\subex{a)}
Show that 
\[
\bar{E}_{\alpha} = 2\left( \langle \frac{\bar{\psi}_{\alpha}}{\psi[\alpha]}E_L[\alpha]\rangle -\langle \frac{\bar{\psi}_{\alpha}}{\psi[\alpha]}\rangle\langle E_L[\alpha] \rangle\right).
\]

\subex{b)}
Find the corresponding expression for the variance.




\end{doconceexercise}
% --- end exercise ---


% !split
\subsection{Brief reminder on Newton-Raphson's method}

Let us quickly remind ourselves how we derive the above method.

Perhaps the most celebrated of all one-dimensional root-finding
routines is Newton's method, also called the Newton-Raphson
method. This method  requires the evaluation of both the
function $f$ and its derivative $f'$ at arbitrary points. 
If you can only calculate the derivative
numerically and/or your function is not of the smooth type, we
normally discourage the use of this method.

% !split
\subsection{The equations}

The Newton-Raphson formula consists geometrically of extending the
tangent line at a current point until it crosses zero, then setting
the next guess to the abscissa of that zero-crossing.  The mathematics
behind this method is rather simple. Employing a Taylor expansion for
$x$ sufficiently close to the solution $s$, we have


\[
    f(s)=0=f(x)+(s-x)f'(x)+\frac{(s-x)^2}{2}f''(x) +\dots.
    \label{eq:taylornr}
\]

For small enough values of the function and for well-behaved
functions, the terms beyond linear are unimportant, hence we obtain


\[
   f(x)+(s-x)f'(x)\approx 0,
\]
yielding
\[
   s\approx x-\frac{f(x)}{f'(x)}.
\]

Having in mind an iterative procedure, it is natural to start iterating with
\[
   x_{n+1}=x_n-\frac{f(x_n)}{f'(x_n)}.
\]

% !split
\subsection{Simple geometric interpretation}

The above is Newton-Raphson's method. It has a simple geometric
interpretation, namely $x_{n+1}$ is the point where the tangent from
$(x_n,f(x_n))$ crosses the $x$-axis.  Close to the solution,
Newton-Raphson converges fast to the desired result. However, if we
are far from a root, where the higher-order terms in the series are
important, the Newton-Raphson formula can give grossly inaccurate
results. For instance, the initial guess for the root might be so far
from the true root as to let the search interval include a local
maximum or minimum of the function.  If an iteration places a trial
guess near such a local extremum, so that the first derivative nearly
vanishes, then Newton-Raphson may fail totally


% !split
\subsection{Extending to more than one variable}

Newton's method can be generalized to systems of several non-linear equations
and variables. Consider the case with two equations
\[
   \begin{array}{cc} f_1(x_1,x_2) &=0\\
                     f_2(x_1,x_2) &=0,\end{array}
\]
which we Taylor expand to obtain

\[
   \begin{array}{cc} 0=f_1(x_1+h_1,x_2+h_2)=&f_1(x_1,x_2)+h_1
                     \partial f_1/\partial x_1+h_2
                     \partial f_1/\partial x_2+\dots\\
                     0=f_2(x_1+h_1,x_2+h_2)=&f_2(x_1,x_2)+h_1
                     \partial f_2/\partial x_1+h_2
                     \partial f_2/\partial x_2+\dots
                       \end{array}.
\]
Defining the Jacobian matrix $\hat{J}$ we have
\[
 \hat{J}=\left( \begin{array}{cc}
                         \partial f_1/\partial x_1  & \partial f_1/\partial x_2 \\
                          \partial f_2/\partial x_1     &\partial f_2/\partial x_2
             \end{array} \right),
\]
we can rephrase Newton's method as
\[
\left(\begin{array}{c} x_1^{n+1} \\ x_2^{n+1} \end{array} \right)=
\left(\begin{array}{c} x_1^{n} \\ x_2^{n} \end{array} \right)+
\left(\begin{array}{c} h_1^{n} \\ h_2^{n} \end{array} \right),
\]
where we have defined
\[
   \left(\begin{array}{c} h_1^{n} \\ h_2^{n} \end{array} \right)=
   -{\bf \hat{J}}^{-1}
   \left(\begin{array}{c} f_1(x_1^{n},x_2^{n}) \\ f_2(x_1^{n},x_2^{n}) \end{array} \right).
\]
We need thus to compute the inverse of the Jacobian matrix and it
is to understand that difficulties  may
arise in case $\hat{J}$ is nearly singular.

It is rather straightforward to extend the above scheme to systems of
more than two non-linear equations. In our case, the Jacobian matrix is given by the Hessian that represents the second derivative of cost function. 



% !split
\subsection{Steepest descent}

The basic idea of gradient descent is
that a function $F(\mathbf{x})$, 
$\mathbf{x} \equiv (x_1,\cdots,x_n)$, decreases fastest if one goes from $\bf {x}$ in the
direction of the negative gradient $-\nabla F(\mathbf{x})$.

It can be shown that if 
\[
\mathbf{x}_{k+1} = \mathbf{x}_k - \gamma_k \nabla F(\mathbf{x}_k),
\]
with $\gamma_k > 0$.

For $\gamma_k$ small enough, then $F(\mathbf{x}_{k+1}) \leq
F(\mathbf{x}_k)$. This means that for a sufficiently small $\gamma_k$
we are always moving towards smaller function values, i.e a minimum.

% !split 
\subsection{More on Steepest descent}

The previous observation is the basis of the method of steepest
descent, which is also referred to as just gradient descent (GD). One
starts with an initial guess $\mathbf{x}_0$ for a minimum of $F$ and
computes new approximations according to

\[
\mathbf{x}_{k+1} = \mathbf{x}_k - \gamma_k \nabla F(\mathbf{x}_k), \ \ k \geq 0.
\]

The parameter $\gamma_k$ is often referred to as the step length or
the learning rate within the context of Machine Learning.

% !split 
\subsection{The ideal}

Ideally the sequence $\{\mathbf{x}_k \}_{k=0}$ converges to a global
minimum of the function $F$. In general we do not know if we are in a
global or local minimum. In the special case when $F$ is a convex
function, all local minima are also global minima, so in this case
gradient descent can converge to the global solution. The advantage of
this scheme is that it is conceptually simple and straightforward to
implement. However the method in this form has some severe
limitations:

In machine learing we are often faced with non-convex high dimensional
cost functions with many local minima. Since GD is deterministic we
will get stuck in a local minimum, if the method converges, unless we
have a very good intial guess. This also implies that the scheme is
sensitive to the chosen initial condition.

Note that the gradient is a function of $\mathbf{x} =
(x_1,\cdots,x_n)$ which makes it expensive to compute numerically.


% !split 
\subsection{The sensitiveness of the gradient descent}

The gradient descent method 
is sensitive to the choice of learning rate $\gamma_k$. This is due
to the fact that we are only guaranteed that $F(\mathbf{x}_{k+1}) \leq
F(\mathbf{x}_k)$ for sufficiently small $\gamma_k$. The problem is to
determine an optimal learning rate. If the learning rate is chosen too
small the method will take a long time to converge and if it is too
large we can experience erratic behavior.

Many of these shortcomings can be alleviated by introducing
randomness. One such method is that of Stochastic Gradient Descent
(SGD), see below.


% !split 
\subsection{Convex functions}

Ideally we want our cost/loss function to be convex(concave).

First we give the definition of a convex set: A set $C$ in
$\mathbb{R}^n$ is said to be convex if, for all $x$ and $y$ in $C$ and
all $t \in (0,1)$ , the point $(1 − t)x + ty$ also belongs to
C. Geometrically this means that every point on the line segment
connecting $x$ and $y$ is in $C$ as discussed below.

The convex subsets of $\mathbb{R}$ are the intervals of
$\mathbb{R}$. Examples of convex sets of $\mathbb{R}^2$ are the
regular polygons (triangles, rectangles, pentagons, etc...).

% !split
\subsection{Convex function}

\textbf{Convex function}: Let $X \subset \mathbb{R}^n$ be a convex set. Assume that the function $f: X \rightarrow \mathbb{R}$ is continuous, then $f$ is said to be convex if $$f(tx_1 + (1-t)x_2) \leq tf(x_1) + (1-t)f(x_2) $$ for all $x_1, x_2 \in X$ and for all $t \in [0,1]$. If $\leq$ is replaced with a strict inequaltiy in the definition, we demand $x_1 \neq x_2$ and $t\in(0,1)$ then $f$ is said to be strictly convex. For a single variable function, convexity means that if you draw a straight line connecting $f(x_1)$ and $f(x_2)$, the value of the function on the interval $[x_1,x_2]$ is always below the line as illustrated below.

% !split
\subsection{Conditions on convex functions}

In the following we state first and second-order conditions which
ensures convexity of a function $f$. We write $D_f$ to denote the
domain of $f$, i.e the subset of $R^n$ where $f$ is defined. For more
details and proofs we refer to: \href{{http://stanford.edu/boyd/cvxbook/, 2004}}{S. Boyd and L. Vandenberghe. Convex Optimization. Cambridge University Press}.


% --- begin paragraph admon ---
\paragraph{First order condition.}
Suppose $f$ is differentiable (i.e $\nabla f(x)$ is well defined for
all $x$ in the domain of $f$). Then $f$ is convex if and only if $D_f$
is a convex set and $$f(y) \geq f(x) + \nabla f(x)^T (y-x) $$ holds
for all $x,y \in D_f$. This condition means that for a convex function
the first order Taylor expansion (right hand side above) at any point
a global under estimator of the function. To convince yourself you can
make a drawing of $f(x) = x^2+1$ and draw the tangent line to $f(x)$ and
note that it is always below the graph.
% --- end paragraph admon ---




% --- begin paragraph admon ---
\paragraph{Second order condition.}
Assume that $f$ is twice
differentiable, i.e the Hessian matrix exists at each point in
$D_f$. Then $f$ is convex if and only if $D_f$ is a convex set and its
Hessian is positive semi-definite for all $x\in D_f$. For a
single-variable function this reduces to $f''(x) \geq 0$. Geometrically this means that $f$ has nonnegative curvature
everywhere.
% --- end paragraph admon ---



This condition is particularly useful since it gives us an procedure for determining if the function under consideration is convex, apart from using the definition.

% !split
\subsection{More on convex functions}

The next result is of great importance to us and the reason why we are
going on about convex functions. In machine learning we frequently
have to minimize a loss/cost function in order to find the best
parameters for the model we are considering. 

Ideally we want the
global minimum (for high-dimensional models it is hard to know
if we have local or global minimum). However, if the cost/loss function
is convex the following result provides invaluable information:


% --- begin paragraph admon ---
\paragraph{Any minimum is global for convex functions.}
Consider the problem of finding $x \in \mathbb{R}^n$ such that $f(x)$
is minimal, where $f$ is convex and differentiable. Then, any point
$x^*$ that satisfies $\nabla f(x^*) = 0$ is a global minimum.
% --- end paragraph admon ---



This result means that if we know that the cost/loss function is convex and we are able to find a minimum, we are guaranteed that it is a global minimum.

% !split
\subsection{Some simple problems}

\begin{enumerate}
\item Show that $f(x)=x^2$ is convex for $x \in \mathbb{R}$ using the definition of convexity. Hint: If you re-write the definition, $f$ is convex if the following holds for all $x,y \in D_f$ and any $\lambda \in [0,1]$ $\lambda f(x)+(1-\lambda)f(y)-f(\lambda x + (1-\lambda) y ) \geq 0$.

\item Using the second order condition show that the following functions are convex on the specified domain.
\begin{itemize}

 \item $f(x) = e^x$ is convex for $x \in \mathbb{R}$.

 \item $g(x) = -\ln(x)$ is convex for $x \in (0,\infty)$.

\end{itemize}

\noindent
\item Let $f(x) = x^2$ and $g(x) = e^x$. Show that $f(g(x))$ and $g(f(x))$ is convex for $x \in \mathbb{R}$. Also show that if $f(x)$ is any convex function than $h(x) = e^{f(x)}$ is convex.

\item A norm is any function that satisfy the following properties
\begin{itemize}

 \item $f(\alpha x) = |\alpha| f(x)$ for all $\alpha \in \mathbb{R}$.

 \item $f(x+y) \leq f(x) + f(y)$

 \item $f(x) \leq 0$ for all $x \in \mathbb{R}^n$ with equality if and only if $x = 0$
\end{itemize}

\noindent
\end{enumerate}

\noindent
Using the definition of convexity, try to show that a function satisfying the properties above is convex (the third condition is not needed to show this).


% !split
\subsection{Standard steepest descent}


Before we proceed, we would like to discuss the approach called the
\textbf{standard Steepest descent}, which again leads to us having to be able
to compute a matrix. It belongs to the class of Conjugate Gradient methods (CG).

\href{{https://www.cs.cmu.edu/~quake-papers/painless-conjugate-gradient.pdf}}{The success of the CG method}
for finding solutions of non-linear problems is based on the theory
of conjugate gradients for linear systems of equations. It belongs to
the class of iterative methods for solving problems from linear
algebra of the type 
\begin{equation*} 
\hat{A}\hat{x} = \hat{b}.
\end{equation*} 

In the iterative process we end up with a problem like

\begin{equation*}
  \hat{r}= \hat{b}-\hat{A}\hat{x},
\end{equation*}
where $\hat{r}$ is the so-called residual or error in the iterative process.

When we have found the exact solution, $\hat{r}=0$.

% !split
\subsection{Gradient method}

The residual is zero when we reach the minimum of the quadratic equation
\begin{equation*}
  P(\hat{x})=\frac{1}{2}\hat{x}^T\hat{A}\hat{x} - \hat{x}^T\hat{b},
\end{equation*}

with the constraint that the matrix $\hat{A}$ is positive definite and
symmetric.  This defines also the Hessian and we want it to be  positive definite.  


% !split
\subsection{Steepest descent  method}

We denote the initial guess for $\hat{x}$ as $\hat{x}_0$. 
We can assume without loss of generality that
\begin{equation*}
\hat{x}_0=0,
\end{equation*}
or consider the system
\begin{equation*}
\hat{A}\hat{z} = \hat{b}-\hat{A}\hat{x}_0,
\end{equation*}
instead.


% !split
\subsection{Steepest descent  method}

% --- begin paragraph admon ---
\paragraph{}
One can show that the solution $\hat{x}$ is also the unique minimizer of the quadratic form
\begin{equation*}
  f(\hat{x}) = \frac{1}{2}\hat{x}^T\hat{A}\hat{x} - \hat{x}^T \hat{x} , \quad \hat{x}\in\mathbf{R}^n. 
\end{equation*}
This suggests taking the first basis vector $\hat{r}_1$ (see below for definition) 
to be the gradient of $f$ at $\hat{x}=\hat{x}_0$, 
which equals
\begin{equation*}
\hat{A}\hat{x}_0-\hat{b},
\end{equation*}
and 
$\hat{x}_0=0$ it is equal $-\hat{b}$.
% --- end paragraph admon ---



% !split
\subsection{Final expressions}

% --- begin paragraph admon ---
\paragraph{}
We can compute the residual iteratively as
\begin{equation*}
\hat{r}_{k+1}=\hat{b}-\hat{A}\hat{x}_{k+1},
 \end{equation*}
which equals
\begin{equation*}
\hat{b}-\hat{A}(\hat{x}_k+\alpha_k\hat{r}_k),
 \end{equation*}
or
\begin{equation*}
(\hat{b}-\hat{A}\hat{x}_k)-\alpha_k\hat{A}\hat{r}_k,
 \end{equation*}
which gives

\[
\alpha_k = \frac{\hat{r}_k^T\hat{r}_k}{\hat{r}_k^T\hat{A}\hat{r}_k}
\]
leading to the iterative scheme
\begin{equation*}
\hat{x}_{k+1}=\hat{x}_k-\alpha_k\hat{r}_{k},
 \end{equation*}
% --- end paragraph admon ---




% !split
\subsection{Code examples for steepest descent}

% !split
\subsection{Simple codes for  steepest descent and conjugate gradient using a $2\times 2$ matrix, in c++, Python code to come}

% --- begin paragraph admon ---
\paragraph{}
\bcppcod
#include <cmath>
#include <iostream>
#include <fstream>
#include <iomanip>
#include "vectormatrixclass.h"
using namespace  std;
//   Main function begins here
int main(int  argc, char * argv[]){
  int dim = 2;
  Vector x(dim),xsd(dim), b(dim),x0(dim);
  Matrix A(dim,dim);

  // Set our initial guess
  x0(0) = x0(1) = 0;
  // Set the matrix
  A(0,0) =  3;    A(1,0) =  2;   A(0,1) =  2;   A(1,1) =  6;
  b(0) = 2; b(1) = -8;
  cout << "The Matrix A that we are using: " << endl;
  A.Print();
  cout << endl;
  xsd = SteepestDescent(A,b,x0);
  cout << "The approximate solution using Steepest Descent is: " << endl;
  xsd.Print();
  cout << endl;
}
\ecppcod
% --- end paragraph admon ---



% !split
\subsection{The routine for the steepest descent method}

% --- begin paragraph admon ---
\paragraph{}
\bcppcod
Vector SteepestDescent(Matrix A, Vector b, Vector x0){
  int IterMax, i;
  int dim = x0.Dimension();
  const double tolerance = 1.0e-14;
  Vector x(dim),f(dim),z(dim);
  double c,alpha,d;
  IterMax = 30;
  x = x0;
  r = A*x-b;
  i = 0;
  while (i <= IterMax){
    z = A*r;
    c = dot(r,r);
    alpha = c/dot(r,z);
    x = x - alpha*r;
    r =  A*x-b;
    if(sqrt(dot(r,r)) < tolerance) break;
    i++;
  }
  return x;
}
\ecppcod
% --- end paragraph admon ---




% !split
\subsection{Steepest descent example}

\bpycod
import numpy as np
import numpy.linalg as la

import scipy.optimize as sopt

import matplotlib.pyplot as pt
from mpl_toolkits.mplot3d import axes3d

def f(x):
    return 0.5*x[0]**2 + 2.5*x[1]**2

def df(x):
    return np.array([x[0], 5*x[1]])

fig = pt.figure()
ax = fig.gca(projection="3d")

xmesh, ymesh = np.mgrid[-2:2:50j,-2:2:50j]
fmesh = f(np.array([xmesh, ymesh]))
ax.plot_surface(xmesh, ymesh, fmesh)
\epycod
And then as countor plot
\bpycod
pt.axis("equal")
pt.contour(xmesh, ymesh, fmesh)
guesses = [np.array([2, 2./5])]
\epycod
Find guesses
\bpycod
x = guesses[-1]
s = -df(x)
\epycod
Run it!
\bpycod
def f1d(alpha):
    return f(x + alpha*s)

alpha_opt = sopt.golden(f1d)
next_guess = x + alpha_opt * s
guesses.append(next_guess)
print(next_guess)
\epycod
What happened?
\bpycod
pt.axis("equal")
pt.contour(xmesh, ymesh, fmesh, 50)
it_array = np.array(guesses)
pt.plot(it_array.T[0], it_array.T[1], "x-")
\epycod

% !split
\subsection{Conjugate gradient method}

% --- begin paragraph admon ---
\paragraph{}
In the CG method we define so-called conjugate directions and two vectors 
$\hat{s}$ and $\hat{t}$
are said to be
conjugate if
\begin{equation*}
\hat{s}^T\hat{A}\hat{t}= 0.
\end{equation*}
The philosophy of the CG method is to perform searches in various conjugate directions
of our vectors $\hat{x}_i$ obeying the above criterion, namely
\begin{equation*}
\hat{x}_i^T\hat{A}\hat{x}_j= 0.
\end{equation*}
Two vectors are conjugate if they are orthogonal with respect to 
this inner product. Being conjugate is a symmetric relation: if $\hat{s}$ is conjugate to $\hat{t}$, then $\hat{t}$ is conjugate to $\hat{s}$.
% --- end paragraph admon ---



% !split
\subsection{Conjugate gradient method}

% --- begin paragraph admon ---
\paragraph{}
An example is given by the eigenvectors of the matrix
\begin{equation*}
\hat{v}_i^T\hat{A}\hat{v}_j= \lambda\hat{v}_i^T\hat{v}_j,
\end{equation*}
which is zero unless $i=j$.
% --- end paragraph admon ---




% !split
\subsection{Conjugate gradient method}

% --- begin paragraph admon ---
\paragraph{}
Assume now that we have a symmetric positive-definite matrix $\hat{A}$ of size
$n\times n$. At each iteration $i+1$ we obtain the conjugate direction of a vector
\begin{equation*}
\hat{x}_{i+1}=\hat{x}_{i}+\alpha_i\hat{p}_{i}. 
\end{equation*}
We assume that $\hat{p}_{i}$ is a sequence of $n$ mutually conjugate directions. 
Then the $\hat{p}_{i}$  form a basis of $R^n$ and we can expand the solution 
$  \hat{A}\hat{x} = \hat{b}$ in this basis, namely

\begin{equation*}
  \hat{x}  = \sum^{n}_{i=1} \alpha_i \hat{p}_i.
\end{equation*}
% --- end paragraph admon ---



% !split
\subsection{Conjugate gradient method}

% --- begin paragraph admon ---
\paragraph{}
The coefficients are given by
\begin{equation*}
    \mathbf{A}\mathbf{x} = \sum^{n}_{i=1} \alpha_i \mathbf{A} \mathbf{p}_i = \mathbf{b}.
\end{equation*}
Multiplying with $\hat{p}_k^T$  from the left gives

\begin{equation*}
  \hat{p}_k^T \hat{A}\hat{x} = \sum^{n}_{i=1} \alpha_i\hat{p}_k^T \hat{A}\hat{p}_i= \hat{p}_k^T \hat{b},
\end{equation*}
and we can define the coefficients $\alpha_k$ as

\begin{equation*}
    \alpha_k = \frac{\hat{p}_k^T \hat{b}}{\hat{p}_k^T \hat{A} \hat{p}_k}
\end{equation*}
% --- end paragraph admon ---



% !split
\subsection{Conjugate gradient method and iterations}

% --- begin paragraph admon ---
\paragraph{}

If we choose the conjugate vectors $\hat{p}_k$ carefully, 
then we may not need all of them to obtain a good approximation to the solution 
$\hat{x}$. 
We want to regard the conjugate gradient method as an iterative method. 
This will us to solve systems where $n$ is so large that the direct 
method would take too much time.

We denote the initial guess for $\hat{x}$ as $\hat{x}_0$. 
We can assume without loss of generality that
\begin{equation*}
\hat{x}_0=0,
\end{equation*}
or consider the system
\begin{equation*}
\hat{A}\hat{z} = \hat{b}-\hat{A}\hat{x}_0,
\end{equation*}
instead.
% --- end paragraph admon ---




% !split
\subsection{Conjugate gradient method}

% --- begin paragraph admon ---
\paragraph{}
One can show that the solution $\hat{x}$ is also the unique minimizer of the quadratic form
\begin{equation*}
  f(\hat{x}) = \frac{1}{2}\hat{x}^T\hat{A}\hat{x} - \hat{x}^T \hat{x} , \quad \hat{x}\in\mathbf{R}^n. 
\end{equation*}
This suggests taking the first basis vector $\hat{p}_1$ 
to be the gradient of $f$ at $\hat{x}=\hat{x}_0$, 
which equals
\begin{equation*}
\hat{A}\hat{x}_0-\hat{b},
\end{equation*}
and 
$\hat{x}_0=0$ it is equal $-\hat{b}$.
The other vectors in the basis will be conjugate to the gradient, 
hence the name conjugate gradient method.
% --- end paragraph admon ---




% !split
\subsection{Conjugate gradient method}

% --- begin paragraph admon ---
\paragraph{}
Let  $\hat{r}_k$ be the residual at the $k$-th step:
\begin{equation*}
\hat{r}_k=\hat{b}-\hat{A}\hat{x}_k.
\end{equation*}
Note that $\hat{r}_k$ is the negative gradient of $f$ at 
$\hat{x}=\hat{x}_k$, 
so the gradient descent method would be to move in the direction $\hat{r}_k$. 
Here, we insist that the directions $\hat{p}_k$ are conjugate to each other, 
so we take the direction closest to the gradient $\hat{r}_k$  
under the conjugacy constraint. 
This gives the following expression
\begin{equation*}
\hat{p}_{k+1}=\hat{r}_k-\frac{\hat{p}_k^T \hat{A}\hat{r}_k}{\hat{p}_k^T\hat{A}\hat{p}_k} \hat{p}_k.
\end{equation*}
% --- end paragraph admon ---



% !split
\subsection{Conjugate gradient method}

% --- begin paragraph admon ---
\paragraph{}
We can also  compute the residual iteratively as
\begin{equation*}
\hat{r}_{k+1}=\hat{b}-\hat{A}\hat{x}_{k+1},
 \end{equation*}
which equals
\begin{equation*}
\hat{b}-\hat{A}(\hat{x}_k+\alpha_k\hat{p}_k),
 \end{equation*}
or
\begin{equation*}
(\hat{b}-\hat{A}\hat{x}_k)-\alpha_k\hat{A}\hat{p}_k,
 \end{equation*}
which gives

\begin{equation*}
\hat{r}_{k+1}=\hat{r}_k-\hat{A}\hat{p}_{k},
 \end{equation*}
% --- end paragraph admon ---





% !split
\subsection{Simple implementation of the Conjugate gradient algorithm}

% --- begin paragraph admon ---
\paragraph{}
\bcppcod
  Vector ConjugateGradient(Matrix A, Vector b, Vector x0){
  int dim = x0.Dimension();
  const double tolerance = 1.0e-14;
  Vector x(dim),r(dim),v(dim),z(dim);
  double c,t,d;

  x = x0;
  r = b - A*x;
  v = r;
  c = dot(r,r);
  int i = 0; IterMax = dim;
  while(i <= IterMax){
    z = A*v;
    t = c/dot(v,z);
    x = x + t*v;
    r = r - t*z;
    d = dot(r,r);
    if(sqrt(d) < tolerance)
      break;
    v = r + (d/c)*v;
    c = d;  i++;
  }
  return x;
} 
\ecppcod
% --- end paragraph admon ---




% !split
\subsection{Broyden–Fletcher–Goldfarb–Shanno algorithm}

% --- begin paragraph admon ---
\paragraph{}
The optimization problem is to minimize $f(\mathbf {x} )$ where $\mathbf {x}$  is a vector in $R^{n}$, and $f$ is a differentiable scalar function. There are no constraints on the values that  $\mathbf {x}$  can take.

The algorithm begins at an initial estimate for the optimal value $\mathbf {x}_{0}$ and proceeds iteratively to get a better estimate at each stage.

The search direction $p_k$ at stage $k$ is given by the solution of the analogue of the Newton equation
\[
B_{k}\mathbf {p} _{k}=-\nabla f(\mathbf {x}_{k}),
\]

where $B_{k}$ is an approximation to the Hessian matrix, which is
updated iteratively at each stage, and $\nabla f(\mathbf {x} _{k})$
is the gradient of the function
evaluated at $x_k$. 
A line search in the direction $p_k$ is then used to
find the next point $x_{k+1}$ by minimising 
\[
f(\mathbf {x}_{k}+\alpha \mathbf {p}_{k}),
\]
over the scalar $\alpha > 0$.
% --- end paragraph admon ---





% !split
\subsection{Automatic differentiation}
Python has tools for so-called \textbf{automatic differentiation}.
Consider the following example
\[
f(x) = \sin\left(2\pi x + x^2\right)
\]
which has the following derivative
\[
f'(x) = \cos\left(2\pi x + x^2\right)\left(2\pi + 2x\right) 
\]
Using \textbf{autograd} we have

\bpycod
import autograd.numpy as np

# To do elementwise differentiation:
from autograd import elementwise_grad as egrad 

# To plot:
import matplotlib.pyplot as plt 


def f(x):
    return np.sin(2*np.pi*x + x**2)

def f_grad_analytic(x):
    return np.cos(2*np.pi*x + x**2)*(2*np.pi + 2*x)

# Do the comparison:
x = np.linspace(0,1,1000)

f_grad = egrad(f)

computed = f_grad(x)
analytic = f_grad_analytic(x)

plt.title('Derivative computed from Autograd compared with the analytical derivative')
plt.plot(x,computed,label='autograd')
plt.plot(x,analytic,label='analytic')

plt.xlabel('x')
plt.ylabel('y')
plt.legend()

plt.show()

print("The max absolute difference is: %g"%(np.max(np.abs(computed - analytic))))
\epycod

% !split 
\subsection{Using autograd}

Here we
experiment with what kind of functions Autograd is capable
of finding the gradient of. The following Python functions are just
meant to illustrate what Autograd can do, but please feel free to
experiment with other, possibly more complicated, functions as well.

\bpycod
import autograd.numpy as np
from autograd import grad

def f1(x):
    return x**3 + 1

f1_grad = grad(f1)

# Remember to send in float as argument to the computed gradient from Autograd!
a = 1.0

# See the evaluated gradient at a using autograd:
print("The gradient of f1 evaluated at a = %g using autograd is: %g"%(a,f1_grad(a)))

# Compare with the analytical derivative, that is f1'(x) = 3*x**2 
grad_analytical = 3*a**2
print("The gradient of f1 evaluated at a = %g by finding the analytic expression is: %g"%(a,grad_analytical))
\epycod


% !split
\subsection{Autograd with more complicated functions}

To differentiate with respect to two (or more) arguments of a Python
function, Autograd need to know at which variable the function if
being differentiated with respect to.

\bpycod
import autograd.numpy as np
from autograd import grad
def f2(x1,x2):
    return 3*x1**3 + x2*(x1 - 5) + 1

# By sending the argument 0, Autograd will compute the derivative w.r.t the first variable, in this case x1
f2_grad_x1 = grad(f2,0)

# ... and differentiate w.r.t x2 by sending 1 as an additional arugment to grad
f2_grad_x2 = grad(f2,1)

x1 = 1.0
x2 = 3.0 

print("Evaluating at x1 = %g, x2 = %g"%(x1,x2))
print("-"*30)

# Compare with the analytical derivatives:

# Derivative of f2 w.r.t x1 is: 9*x1**2 + x2:
f2_grad_x1_analytical = 9*x1**2 + x2

# Derivative of f2 w.r.t x2 is: x1 - 5:
f2_grad_x2_analytical = x1 - 5

# See the evaluated derivations:
print("The derivative of f2 w.r.t x1: %g"%( f2_grad_x1(x1,x2) ))
print("The analytical derivative of f2 w.r.t x1: %g"%( f2_grad_x1(x1,x2) ))

print()

print("The derivative of f2 w.r.t x2: %g"%( f2_grad_x2(x1,x2) ))
print("The analytical derivative of f2 w.r.t x2: %g"%( f2_grad_x2(x1,x2) ))
\epycod

Note that the grad function will not produce the true gradient of the function. The true gradient of a function with two or more variables will produce a vector, where each element is the function differentiated w.r.t a variable.


% !split
\subsection{More complicated functions using the elements of their arguments directly}

\bpycod
import autograd.numpy as np
from autograd import grad
def f3(x): # Assumes x is an array of length 5 or higher
    return 2*x[0] + 3*x[1] + 5*x[2] + 7*x[3] + 11*x[4]**2

f3_grad = grad(f3)

x = np.linspace(0,4,5)

# Print the computed gradient:
print("The computed gradient of f3 is: ", f3_grad(x))

# The analytical gradient is: (2, 3, 5, 7, 22*x[4])
f3_grad_analytical = np.array([2, 3, 5, 7, 22*x[4]])

# Print the analytical gradient:
print("The analytical gradient of f3 is: ", f3_grad_analytical)
\epycod

Note that in this case, when sending an array as input argument, the
output from Autograd is another array. This is the true gradient of
the function, as opposed to the function in the previous example. By
using arrays to represent the variables, the output from Autograd
might be easier to work with, as the output is closer to what one
could expect form a gradient-evaluting function.

% !split 
\subsection{Functions using mathematical functions from Numpy}

\bpycod
import autograd.numpy as np
from autograd import grad
def f4(x):
    return np.sqrt(1+x**2) + np.exp(x) + np.sin(2*np.pi*x)

f4_grad = grad(f4)

x = 2.7

# Print the computed derivative:
print("The computed derivative of f4 at x = %g is: %g"%(x,f4_grad(x)))

# The analytical derivative is: x/sqrt(1 + x**2) + exp(x) + cos(2*pi*x)*2*pi
f4_grad_analytical = x/np.sqrt(1 + x**2) + np.exp(x) + np.cos(2*np.pi*x)*2*np.pi

# Print the analytical gradient:
print("The analytical gradient of f4 at x = %g is: %g"%(x,f4_grad_analytical))
\epycod


% !split
\subsection{More autograd}

\bpycod
import autograd.numpy as np
from autograd import grad
def f5(x):
    if x >= 0:
        return x**2
    else:
        return -3*x + 1

f5_grad = grad(f5)

x = 2.7

# Print the computed derivative:
print("The computed derivative of f5 at x = %g is: %g"%(x,f5_grad(x)))
\epycod


% !split
\subsection{And  with loops}

\bpycod
import autograd.numpy as np
from autograd import grad
def f6_for(x):
    val = 0
    for i in range(10):
        val = val + x**i
    return val

def f6_while(x):
    val = 0
    i = 0
    while i < 10:
        val = val + x**i
        i = i + 1
    return val

f6_for_grad = grad(f6_for)
f6_while_grad = grad(f6_while)

x = 0.5

# Print the computed derivaties of f6_for and f6_while
print("The computed derivative of f6_for at x = %g is: %g"%(x,f6_for_grad(x)))
print("The computed derivative of f6_while at x = %g is: %g"%(x,f6_while_grad(x)))
\epycod
\bpycod
import autograd.numpy as np
from autograd import grad
# Both of the functions are implementation of the sum: sum(x**i) for i = 0, ..., 9
# The analytical derivative is: sum(i*x**(i-1)) 
f6_grad_analytical = 0
for i in range(10):
    f6_grad_analytical += i*x**(i-1)

print("The analytical derivative of f6 at x = %g is: %g"%(x,f6_grad_analytical))
\epycod

% !split
\subsection{Using recursion}
\bpycod
import autograd.numpy as np
from autograd import grad

def f7(n): # Assume that n is an integer
    if n == 1 or n == 0:
        return 1
    else:
        return n*f7(n-1)

f7_grad = grad(f7)

n = 2.0

print("The computed derivative of f7 at n = %d is: %g"%(n,f7_grad(n)))

# The function f7 is an implementation of the factorial of n.
# By using the product rule, one can find that the derivative is:

f7_grad_analytical = 0
for i in range(int(n)-1):
    tmp = 1
    for k in range(int(n)-1):
        if k != i:
            tmp *= (n - k)
    f7_grad_analytical += tmp

print("The analytical derivative of f7 at n = %d is: %g"%(n,f7_grad_analytical))

\epycod
Note that if n is equal to zero or one, Autograd will give an error message. This message appears when the output is independent on input.

% !split
\subsection{Unsupported functions}
Autograd supports many features. However, there are some functions that is not supported (yet) by Autograd.

Assigning a value to the variable being differentiated with respect to
\bpycod
import autograd.numpy as np
from autograd import grad
def f8(x): # Assume x is an array
    x[2] = 3
    return x*2

f8_grad = grad(f8)

x = 8.4

print("The derivative of f8 is:",f8_grad(x))
\epycod
Here, Autograd tells us that an 'ArrayBox' does not support item assignment. The item assignment is done when the program tries to assign x[2] to the value 3. However, Autograd has implemented the computation of the derivative such that this assignment is not possible.

% !split
\subsection{The syntax a.dot(b) when finding the dot product}
\bpycod
import autograd.numpy as np
from autograd import grad
def f9(a): # Assume a is an array with 2 elements
    b = np.array([1.0,2.0])
    return a.dot(b)

f9_grad = grad(f9)

x = np.array([1.0,0.0])

print("The derivative of f9 is:",f9_grad(x))
\epycod

Here we are told that the 'dot' function does not belong to Autograd's
version of a Numpy array.  To overcome this, an alternative syntax
which also computed the dot product can be used:

\bpycod
import autograd.numpy as np
from autograd import grad
def f9_alternative(x): # Assume a is an array with 2 elements
    b = np.array([1.0,2.0])
    return np.dot(x,b) # The same as x_1*b_1 + x_2*b_2

f9_alternative_grad = grad(f9_alternative)

x = np.array([3.0,0.0])

print("The gradient of f9 is:",f9_alternative_grad(x))

# The analytical gradient of the dot product of vectors x and b with two elements (x_1,x_2) and (b_1, b_2) respectively
# w.r.t x is (b_1, b_2).
\epycod

% !split
\subsection{Recommended to avoid}
The documentation recommends to avoid inplace operations such as
\bpycod
a += b
a -= b
a*= b
a /=b
\epycod

% !split
\subsection{Stochastic Gradient Descent}

Stochastic gradient descent (SGD) and variants thereof address some of
the shortcomings of the Gradient descent method discussed above.

The underlying idea of SGD comes from the observation that a given 
function, which we want to minimize, can almost always be written as a
sum over $n$ data points $\{\mathbf{x}_i\}_{i=1}^n$,
\[
C(\mathbf{\beta}) = \sum_{i=1}^n c_i(\mathbf{x}_i,
\mathbf{\beta}). 
\]

% !split
\subsection{Computation of gradients}

This in turn means that the gradient can be
computed as a sum over $i$-gradients 
\[
\nabla_\beta C(\mathbf{\beta}) = \sum_i^n \nabla_\beta c_i(\mathbf{x}_i,
\mathbf{\beta}).
\]

Stochasticity/randomness is introduced by only taking the
gradient on a subset of the data called minibatches.  If there are $n$
data points and the size of each minibatch is $M$, there will be $n/M$
minibatches. We denote these minibatches by $B_k$ where
$k=1,\cdots,n/M$.

% !split
\subsection{SGD example}
As an example, suppose we have $10$ data points $(\mathbf{x}_1,\cdots, \mathbf{x}_{10})$ 
and we choose to have $M=5$ minibathces,
then each minibatch contains two data points. In particular we have
$B_1 = (\mathbf{x}_1,\mathbf{x}_2), \cdots, B_5 =
(\mathbf{x}_9,\mathbf{x}_{10})$. Note that if you choose $M=1$ you
have only a single batch with all data points and on the other extreme,
you may choose $M=n$ resulting in a minibatch for each datapoint, i.e
$B_k = \mathbf{x}_k$.

The idea is now to approximate the gradient by replacing the sum over
all data points with a sum over the data points in one the minibatches
picked at random in each gradient descent step 
\[
\nabla_{\beta}
C(\mathbf{\beta}) = \sum_{i=1}^n \nabla_\beta c_i(\mathbf{x}_i,
\mathbf{\beta}) \rightarrow \sum_{i \in B_k}^n \nabla_\beta
c_i(\mathbf{x}_i, \mathbf{\beta}).
\]

% !split
\subsection{The gradient step}

Thus a gradient descent step now looks like 
\[
\beta_{j+1} = \beta_j - \gamma_j \sum_{i \in B_k}^n \nabla_\beta c_i(\mathbf{x}_i,
\mathbf{\beta})
\]

where $k$ is picked at random with equal
probability from $[1,n/M]$. An iteration over the number of
minibathces (n/M) is commonly referred to as an epoch. Thus it is
typical to choose a number of epochs and for each epoch iterate over
the number of minibatches, as exemplified in the code below.

% !split
\subsection{Simple example code}

\bpycod
import numpy as np 

n = 100 #100 datapoints 
M = 5   #size of each minibatch
m = int(n/M) #number of minibatches
n_epochs = 10 #number of epochs

j = 0
for epoch in range(1,n_epochs+1):
    for i in range(m):
        k = np.random.randint(m) #Pick the k-th minibatch at random
        #Compute the gradient using the data in minibatch Bk
        #Compute new suggestion for 
        j += 1
\epycod

Taking the gradient only on a subset of the data has two important
benefits. First, it introduces randomness which decreases the chance
that our opmization scheme gets stuck in a local minima. Second, if
the size of the minibatches are small relative to the number of
datapoints ($M <  n$), the computation of the gradient is much
cheaper since we sum over the datapoints in the $k-th$ minibatch and not
all $n$ datapoints.

% !split
\subsection{When do we stop?}

A natural question is when do we stop the search for a new minimum?
One possibility is to compute the full gradient after a given number
of epochs and check if the norm of the gradient is smaller than some
threshold and stop if true. However, the condition that the gradient
is zero is valid also for local minima, so this would only tell us
that we are close to a local/global minimum. However, we could also
evaluate the cost function at this point, store the result and
continue the search. If the test kicks in at a later stage we can
compare the values of the cost function and keep the $\beta$ that
gave the lowest value.

% !split
\subsection{Slightly different approach}

Another approach is to let the step length $\gamma_j$ depend on the
number of epochs in such a way that it becomes very small after a
reasonable time such that we do not move at all.

As an example, let $e = 0,1,2,3,\cdots$ denote the current epoch and let $t_0, t_1 > 0$ be two fixed numbers. Furthermore, let $t = e \cdot m + i$ where $m$ is the number of minibatches and $i=0,\cdots,m-1$. Then the function $$\gamma_j(t; t_0, t_1) = \frac{t_0}{t+t_1} $$ goes to zero as the number of epochs gets large. I.e. we start with a step length $\gamma_j (0; t_0, t_1) = t_0/t_1$ which decays in \emph{time} $t$.

In this way we can fix the number of epochs, compute $\beta$ and
evaluate the cost function at the end. Repeating the computation will
give a different result since the scheme is random by design. Then we
pick the final $\beta$ that gives the lowest value of the cost
function.

\bpycod
import numpy as np 

def step_length(t,t0,t1):
    return t0/(t+t1)

n = 100 #100 datapoints 
M = 5   #size of each minibatch
m = int(n/M) #number of minibatches
n_epochs = 500 #number of epochs
t0 = 1.0
t1 = 10

gamma_j = t0/t1
j = 0
for epoch in range(1,n_epochs+1):
    for i in range(m):
        k = np.random.randint(m) #Pick the k-th minibatch at random
        #Compute the gradient using the data in minibatch Bk
        #Compute new suggestion for beta
        t = epoch*m+i
        gamma_j = step_length(t,t0,t1)
        j += 1

print("gamma_j after %d epochs: %g" % (n_epochs,gamma_j))
\epycod





% !split
\subsection{Program for stochastic gradient}

\bpycod
# Importing various packages
from math import exp, sqrt
from random import random, seed
import numpy as np
import matplotlib.pyplot as plt
from sklearn.linear_model import SGDRegressor

x = 2*np.random.rand(100,1)
y = 4+3*x+np.random.randn(100,1)

xb = np.c_[np.ones((100,1)), x]
theta_linreg = np.linalg.inv(xb.T.dot(xb)).dot(xb.T).dot(y)
print("Own inversion")
print(theta_linreg)
sgdreg = SGDRegressor(n_iter = 50, penalty=None, eta0=0.1)
sgdreg.fit(x,y.ravel())
print("sgdreg from scikit")
print(sgdreg.intercept_, sgdreg.coef_)


theta = np.random.randn(2,1)

eta = 0.1
Niterations = 1000
m = 100

for iter in range(Niterations):
    gradients = 2.0/m*xb.T.dot(xb.dot(theta)-y)
    theta -= eta*gradients
print("theta frm own gd")
print(theta)

xnew = np.array([[0],[2]])
xbnew = np.c_[np.ones((2,1)), xnew]
ypredict = xbnew.dot(theta)
ypredict2 = xbnew.dot(theta_linreg)


n_epochs = 50
t0, t1 = 5, 50
m = 100
def learning_schedule(t):
    return t0/(t+t1)

theta = np.random.randn(2,1)

for epoch in range(n_epochs):
    for i in range(m):
        random_index = np.random.randint(m)
        xi = xb[random_index:random_index+1]
        yi = y[random_index:random_index+1]
        gradients = 2 * xi.T.dot(xi.dot(theta)-yi)
        eta = learning_schedule(epoch*m+i)
        theta = theta - eta*gradients
print("theta from own sdg")
print(theta)






plt.plot(xnew, ypredict, "r-")
plt.plot(xnew, ypredict2, "b-")
plt.plot(x, y ,'ro')
plt.axis([0,2.0,0, 15.0])
plt.xlabel(r'$x$')
plt.ylabel(r'$y$')
plt.title(r'Random numbers ')
plt.show()

\epycod



% !split
\subsection{Using gradient descent methods, limitations}

\begin{itemize}
\item \textbf{Gradient descent (GD) finds local minima of our function}. Since the GD algorithm is deterministic, if it converges, it will converge to a local minimum of our energy function. Because in ML we are often dealing with extremely rugged landscapes with many local minima, this can lead to poor performance.

\item \textbf{GD is sensitive to initial conditions}. One consequence of the local nature of GD is that initial conditions matter. Depending on where one starts, one will end up at a different local minima. Therefore, it is very important to think about how one initializes the training process. This is true for GD as well as more complicated variants of GD.

\item \textbf{Gradients are computationally expensive to calculate for large datasets}. In many cases in statistics and ML, the energy function is a sum of terms, with one term for each data point. For example, in linear regression, $E \propto \sum_{i=1}^n (y_i - \mathbf{w}^T\cdot\mathbf{x}_i)^2$; for logistic regression, the square error is replaced by the cross entropy. To calculate the gradient we have to sum over \emph{all} $n$ data points. Doing this at every GD step becomes extremely computationally expensive. An ingenious solution to this, is to calculate the gradients using small subsets of the data called ``mini batches''. This has the added benefit of introducing stochasticity into our algorithm.

\item \textbf{GD is very sensitive to choices of learning rates}. GD is extremely sensitive to the choice of learning rates. If the learning rate is very small, the training process take an extremely long time. For larger learning rates, GD can diverge and give poor results. Furthermore, depending on what the local landscape looks like, we have to modify the learning rates to ensure convergence. Ideally, we would \emph{adaptively} choose the learning rates to match the landscape.

\item \textbf{GD treats all directions in parameter space uniformly.} Another major drawback of GD is that unlike Newton's method, the learning rate for GD is the same in all directions in parameter space. For this reason, the maximum learning rate is set by the behavior of the steepest direction and this can significantly slow down training. Ideally, we would like to take large steps in flat directions and small steps in steep directions. Since we are exploring rugged landscapes where curvatures change, this requires us to keep track of not only the gradient but second derivatives. The ideal scenario would be to calculate the Hessian but this proves to be too computationally expensive. 

\item GD can take exponential time to escape saddle points, even with random initialization. As we mentioned, GD is extremely sensitive to initial condition since it determines the particular local minimum GD would eventually reach. However, even with a good initialization scheme, through the introduction of randomness, GD can still take exponential time to escape saddle points.
\end{itemize}

\noindent
% !split
\subsection{Codes from numerical recipes}

% --- begin paragraph admon ---
\paragraph{}
You can however use codes we have adapted from the text \href{{http://www.nr.com/}}{Numerical Recipes in C++}, see chapter 10.7.  
Here we present a program, which you also can find at the webpage of the course we use the functions \textbf{dfpmin} and \textbf{lnsrch}.  This is a variant of the Broyden et al algorithm discussed in the previous slide.

\begin{itemize}
\item The program uses the harmonic oscillator in one dimensions as example.

\item The program does not use armadillo to handle vectors and matrices, but employs rather my own vector-matrix class. These auxiliary functions, and the main program \emph{model.cpp} can all be found under the \href{{https://github.com/CompPhysics/ComputationalPhysics2/tree/gh-pages/doc/pub/cg/programs/c%2B%2B}}{program link here}.
\end{itemize}

\noindent
Below we show only excerpts from the main program. For the full program, see the above link.
% --- end paragraph admon ---




% !split
\subsection{Finding the minimum of the harmonic oscillator model in one dimension}

% --- begin paragraph admon ---
\paragraph{}
\bcppcod
//   Main function begins here
int main()
{
     int n, iter;
     double gtol, fret;
     double alpha;
     n = 1;
//   reserve space in memory for vectors containing the variational
//   parameters
     Vector g(n), p(n);
     cout << "Read in guess for alpha" << endl;
     cin >> alpha;
     gtol = 1.0e-5;
//   now call dfmin and compute the minimum
     p(0) = alpha;
     dfpmin(p, n, gtol, &iter, &fret, Efunction, dEfunction);
     cout << "Value of energy minimum = " << fret << endl;
     cout << "Number of iterations = " << iter << endl;
     cout << "Value of alpha at minimum = " << p(0) << endl;
      return 0;
}  // end of main program

\ecppcod
% --- end paragraph admon ---





% !split
\subsection{Functions to observe}

% --- begin paragraph admon ---
\paragraph{}
The functions \textbf{Efunction} and \textbf{dEfunction} compute the expectation value of the energy and its derivative.
These functions need to be changed when you want to your own derivatives.
\bcppcod
//  this function defines the expectation value of the local energy
double Efunction(Vector  &x)
{
  double value = x(0)*x(0)*0.5+1.0/(8*x(0)*x(0));
  return value;
} // end of function to evaluate

//  this function defines the derivative of the energy 
void dEfunction(Vector &x, Vector &g)
{
  g(0) = x(0)-1.0/(4*x(0)*x(0)*x(0));
} // end of function to evaluate
\ecppcod
You need to change these functions in order to compute the local energy for your system. I used 1000
cycles per call to get a new value of $\langle E_L[\alpha]\rangle$.
When I compute the local energy I also compute its derivative.
After roughly 10-20 iterations I got a converged result in terms of $\alpha$.
% --- end paragraph admon ---



% ------------------- end of main content ---------------

% #ifdef PREAMBLE
\end{document}
% #endif

