




!split
===== What is blocking? =====
!bblock Blocking
    * Say that we have a set of samples from a Monte Carlo experiment
    * Assuming (wrongly) that our samples are uncorrelated our best estimate of the standard deviation of the mean $\langle {\cal M}\rangle$ is given by
   !bt   
   \[
        \sigma=\sqrt{\frac{1}{n}\left(\langle {\cal M}^2\rangle-\langle {\cal M}\rangle^2\right)} 
      \]
!et
    * If the samples are correlated we can rewrite our results to show  that
!bt      
\[
        \sigma=\sqrt{\frac{1+2\tau/\Delta
            t}{n}\left(\langle {\cal M}^2\rangle-\langle {\cal M}\rangle^2\right)}
      \]
!et
      where $\tau$ is the correlation time (the time between a sample and the next uncorrelated sample) and $\Delta t$ is time between each sample
!eblock    

!split
===== What is blocking? =====
!bblock Blocking 
    * If $\Delta t\gg\tau$ our first estimate of $\sigma$ still holds
    * Much more common that $\Delta t<\tau$
    * In the method of data blocking we divide the sequence of samples into blocks
    * We then take the mean $\langle {\cal M}_i\rangle$ of block $i=1\ldots n_{blocks}$ to calculate the total mean and variance
    * The size of each block must be so large that sample $j$ of block $i$ is not correlated with sample $j$ of block $i+1$
    * The correlation time $\tau$ would be a good choice
!eblock

!split
===== What is blocking? =====
!bblock Blocking
    * Problem: We don't know $\tau$ or it is too expensive to compute
    * Solution: Make a plot of std. dev. as a function of blocksize
    * The estimate of std. dev. of correlated data is too low $\to$ the error will increase with increasing block size until the blocks are uncorrelated, where we reach a plateau
    * When the std. dev. stops increasing the blocks are uncorrelated
!eblock

!split
===== Implementation =====
!bblock Main ideas
    * Do a Monte Carlo simulation, storing all samples to file
    * Do the statistical analysis on this file, independently of your Monte Carlo program
    * Read the file into an array
    * Loop over various block sizes
    * For each block size $n_b$, loop over the array in steps of $n_b$ taking the mean of elements $i n_b,\ldots,(i+1) n_b$
    * Take the mean and variance of the resulting array
    * Write the results for each block size to file for later
      analysis
!eblock

