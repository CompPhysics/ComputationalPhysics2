\frame
{
  \frametitle{Basic definitions and motivation}
\begin{small}
{\scriptsize
In this course the final aim is to develop a Hartree-Fock code which can be used to study properties of
both atoms and molecules.

For a molecular system, the eigenfunctions of the Hartree-Fock equations are called
\emph{molecular orbitals} (MOs). It is important to distinguish these
from the perhaps more familiar \emph{atomic orbitals}, where we can think of (as we have done till now)
that the electrons occupy some well-defined hydrogen-like states.

An example like the $H_2$-molecule may suffice.
In the ground state the electrons are not occupying the 1$s$-orbitals of atomic Hydrogen. The
molecular system is entirely different from the atomic one, with an entirely different Hamiltonian,
and the eigenstates of the Hartree-Fock equations will therefore also be different.
}
\end{small}
}




\frame
{
  \frametitle{Basic definitions and motivation}
\begin{small}
{\scriptsize
In order to solve the Hartree-Fock equations, we need to expand the molecular orbitals in a known
set of basis functions
\begin{equation}
 \phi_k(\vec r) = \sum_{\mu=1}^M\chi_{\mu k}(\vec r).
\end{equation}
The question we then need to answer is how to choose basis functions $\chi_{\mu k}$? The answer to this
is primarily dictated by the following three criteria:
\begin{enumerate}
 \item The functions should make physically sense, i.e., they should have a large probability where
  the electrons are likely to be and small elsewhere.
 \item It should be possible to integrate the functions efficiently.
 \item The solution of the Hartree-Fock equations should converge towards the Hartree-Fock limit
       as the number of basis functions increases.
\end{enumerate}
The first point suggests that we choose atomic orbitals as basis functions,  which is often
referred to as ``linear combination of atomic orbitals'' (LCAO). In the project we will use the various atomic nuclei as mass centers which the electrons orbit around. However, this is not strictly required since the
atomic orbitals are merely being used as basis functions, and they are not to be thought of
as orbitals occupied by electrons. 
}
\end{small}
}





\frame
{
  \frametitle{Basic definitions, Slater orbitals}
\begin{small}
{\scriptsize
A much set of state functions are the so-called 
Slater type orbitals (STO), defined as
\begin{equation}
 \chi^{STO}(r,\theta,\phi,n,l,m) = \frac{(2a)^{n+1/2}}{[(2n)!]^{1/2}}r^{n-1}\exp(-a r) Y^m_l(\theta,\phi),
\end{equation}
where $n$ is the principal quantum number, $l$ and $m$ are the orbital momentum quantum numbers,
$Y^m_l(\theta,\phi)$ are the spherical harmonics familiar from the solution of the Schr\"odinger
equation for the Hydrogen atom, and $a$ is an exponent which determines the radial decay of the
function. The main attractive features of the STOs are that they have the correct exponential decay with
increasing $r$ and that the orbital components are hydrogenic. 
Furthermore, they can be used to represent in a better way weakly bound states or even resonances. The hydrogen-like state functions 
represent only bound states.
STOs are mainly 
used in atomic Hartree-Fock calculations. When doing molecular calculations, however, they have the
disadvantage that the two-particle integrals $\bra{\mu\sigma}g\ket{\nu\lambda}$ need to compute 
an anti-symmetrized matrix interaction element have no known closed form expression. This is because integrals of products
of exponentials centered on different nuclei are difficult to handle. They can of course be calculated
numerically, but for large molecules this is very time consuming.

}
\end{small}
}
\frame
{
  \frametitle{Basic definitions}
\begin{small}
{\scriptsize
A clever trick which makes multiple center integrals easier to handle is to replace the exponential
term $\exp(-a r)$ with $\exp(-a r^2)$, that is, to functions proportional with Gaussians. This simplifies 
considerably the integrals to evaluate since the product of two Gaussians centered on nuclei with positions
$\vec A$ and $\vec B$ is equal to \emph{one} Gaussian centered on some point $\vec P$ on
the line between them:
\begin{equation}
\label{eq:gaussian_product}
 \exp(-a|\vec r - \vec A|^2)\cdot \exp(-b|\vec r - \vec B|^2) = K_{AB}\,\exp(-p|\vec r - \vec P|^2),
\end{equation}
where
\begin{eqnarray}
 K_{AB} & = & \exp\Big(-\frac{ab}{a + b}|\vec A - \vec B|^2\Big), \\
 \vec P & = & \frac{a \vec A + b \vec B}{a + b}, \\
 p & = & a + b.
\end{eqnarray}
This is the so-called \emph{Gaussian product theorem}.
}
\end{small}
}
\frame
{
  \frametitle{Basic definitions}
\begin{small}
{\scriptsize
The general functional form of a normalised Gaussian type orbital 
centered at $\vec A$ is given by
\begin{equation}
 G_{ijk}(a, \vec r_A) = \Big(\frac{2a}{\pi}\Big)^{3/4}\Big[\frac{(8a)^{i+j+k}\,i!\,j!\,k!}{(2i)!\,(2j)!\,(2k)!}\Big]x_A^i\,y_A^j\,z_A^k\,\exp(-a r_A^2),
\end{equation}
where $\vec r_A = \vec r - \vec A$ and the integers $i$, $j$, $k$ determine the orbital momentum quantum number $l=i+j+k$. 


}
\end{small}
}
\frame
{
  \frametitle{Disadvantages}
\begin{small}
{\scriptsize
The greatest drawback with GTOs is that they do not have the proper exponential radial decay. This can be fixed by forming linear combinations of GTOs
to resemble the STOs
\begin{equation}
 \chi^{CGTO}(\vec r_A,i,j,k) = \sum_{p=1}^L d_p G_{ijk}(a_p, \vec r_A).
\end{equation}
These are called STO-LG basis functions, where L refers to the number of Gaussians used in the linear combination.
The individual Gaussians are called \emph{primitive} basis functions and the linear combinations are called \emph{contracted} basis functions, hence the label
CGTO (Contracted Gaussian Type Orbital). These are the functions we will employ in this project.

A very common choice for the STO-LG basis sets is $L=3$. 
It is important to note that the parameters $(a_p,d_p)$ are static and that the linear combination of Gaussians constitute \emph{one single}
basis function. With the wording ``basis function'', we  will refer to a contracted basis function.
}
\end{small}
}
\frame
{
  \frametitle{Basic definitions}
\begin{small}
{\scriptsize
The STO-LG basis sets belong to the family of \emph{minimal basis sets}.
It means that there is one and only one basis function per atomic orbital.
The STO-LG basis sets for the Hydrogen and Helium atoms, for example, contain only one basis
function for the 1$s$ atomic orbital. This basis function is, as explained above, composed of
a linear combination of L primitives. For the atoms Lithium through Neon the STO-LG basis
sets contain 5 basis functions; one for each of the atomic orbitals
$1s$, $2s$, $2p_x$, $2p_y$ and $2p_z$.

\begin{table}
 \begin{center}
 \caption{Coefficients and exponents used in the STO-3G basis used to reproduce the $1s$ STO for hydrogen $\exp{(-r)}$ using
$\sum_{i=1}^3d_i\exp{(-\alpha_ir^2})$.}
 \label{tab:STO3G}
  \begin{tabular}{|c|c|c|c|}\hline
   $p$        &  1  &  2  &   3 \\ \hline
   $d_p$      &  0.1543   & 0.5353    & 0.4446  \\ \hline
   $a_p$ &  3.4252   & 0.6239    & 0.1688  \\ \hline
  \end{tabular}
 \end{center}
\end{table}


}
\end{small}
}
\frame
{
  \frametitle{Cartesian Gaussians and how to compute integrals}
\begin{small}
{\scriptsize
Using GTOs as basis functions improves the speed of the integrations which must be done when setting up 
for example the Fock matrix.
The Cartesian Gaussian functions centered at $\vec A$ are given by
\begin{equation}
 G_{ijk}(a, \vec r_A) = x^i_A\,y^j_A\,z^k_A\,\exp(-a r^2_A),
\end{equation}
where $\vec r_A = \vec r - \vec A$. These will be our so-called primitive basis functions (see also the flow chart of Hartree-Fock code). 
They factorize in the Cartesian components
\begin{equation}
 G_{ijk}(a, \vec r_A) = G_i(a, x_A)\,G_j(a, y_A)\,G_k(a, z_A),
\end{equation}
where
\begin{equation}
 G_i(a, x_A) = x^i_A\,\exp(-a x^2_A),
\end{equation}
and the other factors are defined similarly. Each of the components obey the simple recurrence relation
\begin{equation}
 x_A\,G_i = G_{i+1}.
\end{equation}
}
\end{small}
}
\frame
{
  \frametitle{Integral evaluations}
\begin{small}
{\scriptsize
We introduce  shorthand notation
\begin{align}
  G_a(\vec r) & = G_{ikm}(a, \vec r_A), \\
  G_b(\vec r) & = G_{jln}(b, \vec r_B),
\end{align}
and define the overlap distribution  as
\begin{equation}
 \Omega_{ab}(\vec r) = G_a(\vec r)\,G_b(\vec r).
\end{equation}
Using the Gaussian product theorem (\ref{eq:gaussian_product}) this can be written as
\begin{equation}
 \Omega_{ab}(\vec r) = K_{AB}\,x^i_A\,x^j_B\,y^k_A\,y^l_B\,z^m_A\,z^n_B\,\exp(-p\,r^2_P),
\end{equation}
where
\begin{equation}
 \begin{split}
  K_{AB}  = & \exp\Big(-\frac{ab}{a + b}R^2_{AB}\Big) \\
  \vec R_{AB} = &  \vec A - \vec B \\
  p = & a + b\\
  \vec r_P = & \vec r - \vec P \\
  \vec P = & \frac{a\vec A + b\vec B}{a + b}.
 \end{split}
\end{equation}
}
\end{small}
}
\frame
{
  \frametitle{Integral evaluations}
\begin{small}
{\scriptsize
Because the Gaussians $G_a$ and $G_b$ factorize in their Cartesian components, such a factorization applies to the overlap distribution
as well, namely
\begin{equation}
 \Omega_{ab}(\vec r) = \Omega_{ij}(x)\,\Omega_{kl}(y)\,\Omega_{mn}(z),
\end{equation}
where
\begin{equation}
 \Omega_{ij} = K^x_{AB}\,x^i_A\,x^j_B\,\exp(-px^2_P)
\end{equation}
and
\begin{equation}
\begin{split}
 K^x_{AB} = & \exp\Big(-\frac{ab}{a + b}X^2_{AB}\Big) \\
   X_{AB} = & A_x - B_x.
\end{split}
\end{equation}
The distributions $\Omega_{kl}(y)$ and $\Omega_{mn}(z)$ are defined similarly.
}
\end{small}
}

\frame
{
  \frametitle{Integral evaluations}
\begin{small}
{\scriptsize
A very useful representation of the Cartesian GTOs is given by the so-called Hermite Gaussians. These functions simplify 
the integrations significantly. The Hermite Gaussians centered at $\vec P$ are defined by
\begin{equation}
 \Lambda_{tuv}(p, \vec r_p) = \Big(\frac{\partial}{\partial P_x}\Big)^t \Big(\frac{\partial}{\partial P_y}\Big)^u \Big(\frac{\partial}{\partial P_z}\Big)^v \exp(-p\, r^2_P),
\end{equation}
where $\vec r_p = \vec r - \vec P$. They factorize as (like the cartesian GTOs also do)
\begin{equation}
 \Lambda_{tuv}(p, \vec r_P) = \Lambda_t(p,x_P)\,\Lambda_u(p,y_P)\,\Lambda_v(p,z_P),
\end{equation}
where
\begin{equation}
\label{eq:HermiteGaussian_x}
 \Lambda_t(p,x_P) = \Big(\frac{\partial}{\partial P_x}\Big)^t \exp(-p\,x^2_P),
\end{equation}
and the other factors are defined similarly. 
}
\end{small}
}

\frame
{
  \frametitle{Integral evaluations}
\begin{small}
{\scriptsize
However, their recurrence relation is quite different from that of the Cartesian Gaussians:
\begin{equation}
\begin{split}
 \Lambda_{t+1}(p,x_P) & = \Big(\frac{\partial}{\partial P_x}\Big)^t \frac{\partial}{\partial P_x}\exp(-px^2_P) \\
                      & = \Big(\frac{\partial}{\partial P_x}\Big)^t 2px_P \exp(-px^2_P)  \\
                      & = 2p[-t\Big(\frac{\partial}{\partial P_x}\Big)^{t-1} + x_P \Big(\frac{\partial}{\partial P_x}\Big)^t] \exp(-px^2_P) \\
                      & = 2p[-t\Lambda_{t-1} + x_P \Lambda_t],
\end{split}
\end{equation}
where we have used that
\begin{equation}
\label{eq:derivation_rule}
 \Big(\frac{\partial}{\partial x}\Big)^t x f(x) = t\Big(\frac{\partial}{\partial x}\Big)^{t-1}f(x) + x\Big(\frac{\partial}{\partial x}\Big)^t f(x).
\end{equation}
The recurrence relation reads
\begin{equation}
\label{eq:hermite_gaussian_recurrence}
 x_P \Lambda_t = \frac{1}{2p}\Lambda_{t+1} + t\Lambda_{t-1}.
\end{equation}

}
\end{small}
}


\frame
{
  \frametitle{Integral evaluations}
\begin{small}
{\scriptsize
Our first goal is to compute the overlap integral
\begin{equation}
S_{ab}  = \langle G_a|G_b\rangle = \int d\vec r \,\Omega_{ab}(\vec r)
\end{equation}
between two Gaussians centered at the points $\vec A$ and $\vec B$.
Since the overlap distribution $\Omega_{ab}$ factorizes in the Cartesian components, the integrals over $x$, $y$ and $z$ can be calculated independently of each other
\begin{equation}
\begin{split}
 S_{ab} = & \langle G_i|G_j\rangle \langle G_k|G_l\rangle \langle G_m|G_n\rangle \\
        = & S_{ij}\,S_{kl}\,S_{mn}.
\end{split}
\end{equation}
The $x$ component of the overlap integral, for example, is given by
\begin{equation}
\label{eq:intG_ij}
\begin{split}
 S_{ij} = & \int dx \,\Omega_{ij}(x) \\
        = & K_{AB}^x\int dx \,x_A^ix_B^j\exp(-px_P^2),
\end{split}
\end{equation}

}
\end{small}
}
\frame
{
  \frametitle{Integral evaluations}
\begin{small}
{\scriptsize
In equation (\ref{eq:intG_ij}) the two-center GTOs have been reduced to a one-center Gaussian.
However, the integral is still not straightforward to calculate because of the powers $x_A^i$ and $x_B^j$. A smart way to deal with this is to express the Cartesian Gaussian
in terms of the Hermite Gaussians. Note that (\ref{eq:HermiteGaussian_x}) is a polynomial of order $t$ in $x$ multiplied by the exponential function. In equation (\ref{eq:intG_ij}) the polynomial
is of order $i+j$. This means that we can express the overlap distribution $\Omega_{ij}(x)$ in equation (\ref{eq:intG_ij}) in terms of the Hermite Gaussians in (\ref{eq:HermiteGaussian_x}) in the following way:
\begin{equation}
\label{eq:LinCombOfHermGauss}
 \Omega_{ij}(x) = \sum_{t=0}^{i+j} E^{ij}_t \Lambda_t(p, x_P),
\end{equation}
where $E^{ij}_t$ are constants.
Note that the sum is over $t$ only. The indices $i$ and $j$ are static and are determined from the powers of $x$ in $G_i$ and $G_j$.
We use them as labels on the coefficients $E^{ij}_t$ because different sets of indices will lead to different sets of coefficients.
}
\end{small}
}
\frame
{
  \frametitle{Integral evaluations}
\begin{small}
{\scriptsize
To get the overlap integral in the $x$-direction we integrate (\ref{eq:LinCombOfHermGauss}) over $\mathbb{R}$, which now turns out to be extremely easy;
the only term that survives the integration is the term for $t=0$:
\begin{eqnarray}
 \int dx\,\Lambda_t(p,x_P) & = & \int dx\,\Big(\frac{\partial}{\partial P_x}\Big)^t\exp(-p\,x^2_P), \\
                           & = & \Big(\frac{\partial}{\partial P_x}\Big)^t \int dx\,\exp(-p\,x^2_P), \\
                           & = & \sqrt{\frac{\pi}{p}}\,\delta_{t0}.
\end{eqnarray}
We have used Leibniz' rule, which says that the differentiation of an integrand with respect to a variable which is not an integration variable can
be moved outside the integral. Thus the integral in (\ref{eq:intG_ij}) is simply
\begin{equation}
 S_{ij} = E^{ij}_0\,\sqrt{\frac{\pi}{p}}.
\end{equation}

}
\end{small}
}
\frame
{
  \frametitle{Integral evaluations}
\begin{small}
{\scriptsize
The same procedure can be used for the integrals with respect to $y$ and $z$, which means that the total overlap integral is
\begin{equation}
 S_{ab} = E^{ij}_0\,E^{kl}_0\,E^{mn}_0\,\Big(\frac{\pi}{p}\Big)^{3/2}.
\end{equation}
We need thereafter to determine the coefficients $E^{ij}_t$. When $i=j=0$ in equation (\ref{eq:LinCombOfHermGauss})
we have
\begin{equation}
 E^{0,0}_0 = K_{AB}^x.
\end{equation}
The other coefficients are found via the following recurrence relations
\begin{equation}
\label{eq:E_recurrence}
\begin{split}
 E^{i+1,j}_t & = \frac{1}{2p}E^{ij}_{t-1} + X_{PA}E^{ij}_t + (t+1)E^{ij}_{t+1} \\
 E^{i,j+1}_t & = \frac{1}{2p}E^{ij}_{t-1} + X_{PB}E^{ij}_t + (t+1)E^{ij}_{t+1}.
\end{split}
\end{equation}
}
\end{small}
}
\frame
{
  \frametitle{Integral evaluations}
\begin{small}
{\scriptsize
Analogous expressions hold for the coefficients $E^{kl}_u$ and $E^{mn}_v$. The first equation in (\ref{eq:E_recurrence}) 
can be derived by comparing two equivalent ways of expanding the product $G_{i+1}G_j$
in Hermite Gaussians. The first way is
\begin{equation}
 G_{i+1}\,G_j= \sum_{t=0}^{i+j+1}E^{i+1,j}_t \Lambda_t,
\end{equation}
and the second way is
\begin{equation}
\label{eq:derivation_E_coeffs}
\begin{split}
 G_{i+1}\,G_j & = x_A G_i\,G_j \\
              & = [(x - P_x) + (P_x - A_x)]\sum_{t=0}^{i+j} E^{ij}_t \Lambda_t\\
              & = \sum_{t=0}^{i+j}[x_P + X_{PA}] E^{ij}_t \Lambda_t \\
              & = \sum_{t=0}^{i+j}[\frac{1}{2p}\Lambda_{t+1} + t\Lambda_{t-1} + X_{PA}\Lambda_t]E^{ij}_t \\
              & = \sum_{t=0}^{i+j}[\frac{1}{2p}E^{ij}_{t-1} + X_{PA}E^{ij}_t + (t+1)E^{ij}_{t+1}] \Lambda_t,
\end{split}
\end{equation}
where we have used the recurrence relation (\ref{eq:hermite_gaussian_recurrence}) on the fourth line and changed the 
summation indices on the fifth line. 
}
\end{small}
}

\frame
{
  \frametitle{Integral evaluations, kinetic energy}
\begin{small}
{\scriptsize
Note that the change in summation indices in equation (\ref{eq:derivation_E_coeffs}) implies that we must define
\begin{equation}
 E^{ij}_t = 0, \qquad \text{if }t<0\text{ or }t > i + j.
\end{equation}

Next we turn to the evaluation of the kinetic integral:
\begin{equation}
\begin{split}
T_{ab} & = -\frac{1}{2}\bra{G_a}\nabla^2\ket{G_b} \\
       & = -\frac{1}{2}\bra{G_{ikm}(a, \vec r_A)}\nabla^2\ket{G_{jln}(b, \vec r_B)} \\
       & = -\frac{1}{2}(T_{ij}\,S_{kl}\,S_{mn} + S_{ij}\,T_{kl}\,S_{mn} + S_{ij}\,S_{kl}\,T_{mn}),
\end{split}
\end{equation}
where
\begin{equation}
 T_{ij} = \int dx \,G_i(a,x_A)\frac{\partial^2}{\partial x^2}G_j(b,x_B),
\end{equation}
and the other factors are defined in the same way. Performing the differentiation yields
\begin{equation}
 T_{ij} = 4b^2\,S_{i,j+2} - 2b(2j + 1)S_{i,j} + j(j-1)S_{i,j-2}.
\end{equation}
Thus we see that the kinetic integrals are calculated easily as products of the overlap integrals.
}
\end{small}
}
\frame
{
  \frametitle{Integral evaluations, one-body Coulomb interaction}
\begin{small}
{\scriptsize
We now turn to the Coulomb integral due to the interaction between the electrons and the atomic nuclei
\begin{equation}
 V_{ab} = \bra{G_a}\frac{1}{r_C}\ket{G_b},
\end{equation}
where $r_C = |\vec r - \vec C|$. As before, the overlap distribution is expanded in terms of  Hermite Gaussians
\begin{equation}
\begin{split}
 V_{ab} & = \int d\vec r \,\frac{\Omega_{ab}(\vec r)}{r_C} \\
        & = \sum_{tuv}E^{ij}_t E^{kl}_u E^{mn}_v\int d\vec r \, \frac{\Lambda_{tuv}(p,\vec r_P)}{r_C} \\
        & = \sum_{tuv}E^{ab}_{tuv}\int d\vec r \, \frac{\Lambda_{tuv}(p,\vec r_P)}{r_C}. \\
\end{split}
\end{equation}
Here we have used the shorthand notation
\begin{equation}
 E^{ab}_{tuv} = E^{ij}_t E^{kl}_u E^{mn}_v.
\end{equation}
In this integral other terms besides $\Lambda_{000}$ will survive due to the factor $1/r_C$. Let us nonetheless start by evaluating this term
\begin{equation}
 V_p = \int d\vec r \, \frac{\Lambda_{000}(p,\vec r_P)}{r_C} = \int d\vec r\, \frac{\exp(-p\,r_P^2)}{r_C}.
\end{equation}
}
\end{small}
}
\frame
{
  \frametitle{Integral evaluations, one-body Coulomb interaction}
\begin{small}
{\scriptsize
We will show that this three-dimensional integral can actually be converted to a one-dimensional one. The trick is to observe that the factor $1/r_C$ can be replaced by the integral
\begin{equation}
 \frac{1}{r_C} = \frac{1}{\sqrt{\pi}}\int_{-\infty}^\infty dt\,\exp(-r^2_C\,t^2).
\end{equation}
Inserting this into $V_p$ and using the Gaussian product theorem gives
\begin{eqnarray}
 V_p & = & \int \exp(-p\,r_P^2)\Big(\frac{1}{\sqrt{\pi}}\int_{-\infty}^\infty\exp(-r^2_C\,t^2)\,dt\Big)\,d\vec r \\
     & = & \frac{1}{\sqrt{\pi}}\int_{-\infty}^\infty\int\exp\Big(-\frac{pt^2}{p + t^2}R^2_{PC}\Big)\,\exp[-(p + t^2)r^2_S] d\vec r\, dt,
\end{eqnarray}
where $\vec R_{PC} = \vec P - \vec C$ and $\vec r_S = \vec r - \vec S$ for some point $\vec S$. Doing the integral over the spatial coordinates reveals that the specific value of $\vec S$ is not relevant
\begin{eqnarray}
 V_p & = & \frac{1}{\sqrt{\pi}}\int_{-\infty}^\infty\exp\Big(-\frac{pt^2}{p + t^2}R^2_{PC}\Big)\Big(\frac{\pi}{p + t^2}\Big)^{3/2}\,dt \\
     & = & 2\pi\int_0^\infty\exp\Big(-\frac{pt^2}{p + t^2}R^2_{PC}\Big)\frac{dt}{(p + t^2)^{3/2}}.
\end{eqnarray}
}
\end{small}
}
\frame
{
  \frametitle{Integral evaluations, one-body Coulomb interaction}
\begin{small}
{\scriptsize
Next we change integration variable from $t$ to $u$ by defining
\begin{equation}
 u^2 = \frac{t^2}{p + t^2}.
\end{equation}
This will change the range of integration from $[0,\infty\rangle$ to $[0,1]$. This is beneficial because the final integral at which we arrive will be calculated numerically.
The change of variables leads to
\begin{align}
\label{eq:V_p}
 V_p & = \frac{2\pi}{p}\int_0^1\exp(-p\,R^2_{PC}\,u^2)\,du \\
     & = \frac{2\pi}{p}F_0(p\,R^2_{PC}),
\end{align}
where $F_0(x)$ is a special instance of the Boys function $F_n(x)$ which is defined as
\begin{equation}
 F_n(x) = \int_0^1\exp(-xt^2)\,t^{2n}\,dt.
\end{equation}
We will discuss the Boys functions below
}
\end{small}
}




\frame
{
  \frametitle{Integral evaluations, one-body Coulomb interaction}
\begin{small}
{\scriptsize
We have now a  simplified way of calculating the integral of $\Lambda_{000}/r_C$. However, we need to integrate $\Lambda_{tuv}/r_C$ for general values of $t$, $u$ and $v$.
These integrals are actually not that hard to do once the Boys function is calculated:
\begin{align}
 V_{ab} & = \frac{2\pi}{p}\sum_{tuv}E^{ab}_{tuv}\int d\vec r \, \frac{\Lambda_{tuv}(p,\vec r_p)}{r_C} \\
        & = \frac{2\pi}{p}\sum_{tuv}E^{ab}_{tuv} \frac{\partial^{t+u+v} F_0(p R^2_{PC})}{\partial P_x^t \partial P_y^u \partial P_z^v} \\
        & = \frac{2\pi}{p}\sum_{tuv}E^{ab}_{tuv} R_{tuv}(p,\vec R_{PC}), \label{eq:V_ab}
\end{align}
where we have defined
\begin{equation}
 R_{tuv}(p,\vec R_{PC}) = \frac{\partial^{t+u+v} F_0(p R^2_{PC})}{\partial P_x^t \partial P_y^u \partial P_z^v}.
\end{equation}
}
\end{small}
}


\frame
{
  \frametitle{Integral evaluations, one-body Coulomb interaction}
\begin{small}
{\scriptsize
We need to calculate derivatives of the function $F_0$. Note first that
\begin{equation}
 \frac{d}{dx}F_n(x) = -F_{n+1}(x).
\end{equation}
This means that it is possible to derive analytical expressions for the Coulomb term $V_{ab}$. However, in practice they are calculated recursively in a manner similar to the way we calculate
the coefficients $E^{ij}_t$. Before presenting the recursion relations, we introduce the so-called auxiliary Hermite integrals
\begin{equation}
 R^n_{tuv}(p,\vec R_{PC}) = (-2p)^n\,\frac{\partial^{t+u+v} F_n(p R^2_{PC})}{\partial P_x^t \partial P_y^u \partial P_z^v}.
\end{equation}
}
\end{small}
}




\frame
{
  \frametitle{Integral evaluations, one-body Coulomb interaction}
\begin{small}
{\scriptsize
By starting with the source terms $R^n_{000}(p,\vec R_{PC}) = (-2p)^n\,F_n(p R^2_{PC})$ we can reach the targets $R^0_{tuv}(p,\vec R_{PC}) = R_{tuv}(p,\vec R_{PC})$ through the following
recurrence relations
\begin{equation}
\label{eq:R_recurrence}
 \begin{split}
  R^n_{t+1,u,v} & = tR^{n+1}_{t-1,u,v} + X_{PC} R^{n+1}_{tuv} \\
  R^n_{t,u+1,v} & = uR^{n+1}_{t,u-1,v} + Y_{PC} R^{n+1}_{tuv} \\
  R^n_{t,u,v+1} & = vR^{n+1}_{t,u,v-1} + Z_{PC} R^{n+1}_{tuv}.
 \end{split}
\end{equation}
The first of these are derived as follows
\begin{align}
 R^{n}_{t+1,u,v} & = (-2p)^n\frac{\partial^{t+u+v}}{\partial P_x^t \partial P_y^u \partial P_z^v} [2pX_{PC}F'_n(pR_{PC}^2)] \\
                 & = (-2p)^{n+1}\frac{\partial^{t+u+v}}{\partial P_x^t\partial P_y^u \partial P_z^v}\Big[X_{PC}F_{n+1}(pR_{PC}^2)\Big] \\
                 & = (-2p)^{n+1}\frac{\partial^{u+v}}{\partial P_y^u \partial P_z^v}\Big[t\frac{\partial^{t-1}}{\partial P_x^{t-1}} + X_{PC}\frac{\partial^t}{\partial P_x^t}\Big]F_{n+1}(pR_{PC}^2) \\
                 & = tR^{n+1}_{t-1,u,v} + X_{PC}R^{n+1}_{tuv},
\end{align}
where we have used equation (\ref{eq:derivation_rule}) and the fact that $F'_n(x) = -F_{n+1}(x)$.
}
\end{small}
}



\frame
{
  \frametitle{Integral evaluations, two-body Coulomb interaction}
\begin{small}
{\scriptsize
Finally we show how to calculate the Coulomb integral due to the interaction between the electrons. It is given by
\begin{equation}
\begin{split}
  g_{acbd} & = \bra{G_a G_c}\frac{1}{r_{12}}\ket{G_b G_d} \\
           & = \int\int \frac{\Omega_{ab}(\vec r_1)\Omega_{cd}(\vec r_2)}{r_{12}} d\vec r_1 d\vec r_2 \\
           & = \sum_{tuv}\sum_{\tau\nu\phi}E^{ab}_{tuv}E^{cd}_{\tau\nu\phi}\int\int\frac{\Lambda_{tuv}(p,\vec r_{1P})\Lambda_{\tau\nu\phi}(q,\vec r_{2Q})}{r_{12}}d \vec r_1 d\vec r_2 \\
           & = \sum_{tuv}\sum_{\tau\nu\phi}E^{ab}_{tuv}E^{cd}_{\tau\nu\phi}\frac{\partial^{t+u+v}}{\partial P_x^t \partial P_y^u \partial P_z^v}
                \frac{\partial^{\tau+\nu+\phi}}{\partial Q_x^\tau \partial Q_y^\nu \partial Q_z^\phi} \\
           &    \hspace{30mm} \int\int\frac{\exp(-pr^2_{1P})\exp(-qr^2_{2Q})}{r_{12}}d \vec r_1 d\vec r_2,
\end{split}
\end{equation}
where, similar to $p$ and $\vec r_{1P}$, we have defined 
\begin{equation}
 \begin{split}
    q = & c + d\\
  \vec r_{2Q} = & \vec r_2 - \vec Q \\
  \vec Q = & \frac{c\,\vec C + d\,\vec D}{c + d}.
 \end{split}
\end{equation}
}
\end{small}
}
\frame
{
  \frametitle{Integral evaluations, two-body Coulomb interaction}
\begin{small}
{\scriptsize
Thus we need to evaluate the integral
\begin{equation}
 V_{pq} = \int\int\frac{\exp(-pr^2_{1P})\exp(-qr^2_{2Q})}{r_{12}}d \vec r_1 d\vec r_2.
\end{equation}
By first integrating over $\vec r_1$ and using equation (\ref{eq:V_p}) this can be written as
\begin{equation}
 V_{pq} = \int\Big(\frac{2\pi}{p}\int_0^1\exp(-p\,r^2_{2P}\,u^2)\,du\Big)\exp(-qr^2_{2Q}) d \vec r_2.
\end{equation}
Next we change the order of integration and use the Gaussian product theorem to get
\begin{equation}
\begin{split}
 V_{pq} & = \frac{2\pi}{p}\int_0^1\int\exp(-\frac{pqu^2}{pu^2+q}R^2_{PQ})\exp[-(pu^2+q)r_{2S}^2] d\vec r_2 du \\
        & = \frac{2\pi}{p}\int_0^1\exp(-\frac{pqu^2}{pu^2+q}R^2_{PQ})\Big(\frac{\pi}{pu^2+q}\Big)^{3/2} du.
\end{split}
\end{equation}
}
\end{small}
}
\frame
{
  \frametitle{Integral evaluations, two-body Coulomb interaction}
\begin{small}
{\scriptsize
Again, the value of $\vec S$ in $\vec r_{2S} = \vec r_2 - \vec S$ is not relevant. If we now make the change of variables
\begin{equation}
 \frac{v^2}{p+q} = \frac{u^2}{pu^2+q},
\end{equation}
we get the result
\begin{equation}
 V_{pq} = \frac{2\pi^{5/2}}{pq\sqrt{p+q}}F_0\Big(\frac{pq}{p+q}R^2_{PQ}\Big).
\end{equation}
From this we get the final answer
\begin{equation}
\begin{split}
 g_{acbd} & = \frac{2\pi^{5/2}}{pq\sqrt{p+q}}\sum_{tuv}\sum_{\tau\nu\phi}(-1)^{\tau+\nu+\phi}E^{ab}_{tuv}E^{cd}_{\tau\nu\phi} \\
          & \hspace{40mm} \frac{\partial^{t+u+v+\tau+\nu+\phi}}{\partial P_x^{t+\tau} \partial P_y^{u+\nu} \partial P_z^{v+\phi}}F_0\Big(\frac{pq}{p+q}R^2_{PQ}\Big) \\
          & = \frac{2\pi^{5/2}}{pq\sqrt{p+q}}\sum_{tuv}\sum_{\tau\nu\phi}(-1)^{\tau+\nu+\phi}E^{ab}_{tuv}E^{cd}_{\tau\nu\phi}R_{t+\tau,u+\nu,v+\phi}(\alpha,\vec R_{PQ}),
\end{split}
\end{equation}
where $\alpha = pq/(p+q)$. The term $(-)^{\tau+\nu+\phi}$ arises due to the fact that
\begin{equation}
\frac{\partial}{\partial Q_x} F_0\Big(\frac{pq}{p+q}R^2_{PQ}\Big) = - \frac{\partial}{\partial P_x} F_0\Big(\frac{pq}{p+q}R^2_{PQ}\Big).
\end{equation}
}
\end{small}
}
\frame
{
  \frametitle{Integral evaluations, two-body Coulomb interaction and Boys function}
\begin{small}
{\scriptsize
Calculating the Coulomb integrals boils down to evaluating the Boys function
\begin{equation}
 F_n(x) = \int_0^1\exp(-xt^2)\,t^{2n}\,dt.
\end{equation}
Doing this by standard numerical procedures is computationally
expensive and should therefore be avoided. Here we describe a possible way to calculate the Boys function efficiently.

First note that if $x$ is very large, the function value will hardly be affected by changing the upper limit of the integral from $1$ to $\infty$. Doing this is useful
since the integral can be calculated exactly. Thus, we have the following approximation for the Boys function for large $x$:
\begin{equation}
 F_n(x) \approx \frac{(2n-1)!!}{2^{n+1}}\sqrt{\frac{\pi}{x^{2n+1}}}. \hspace{15mm} (x\hspace{2mm}\mathrm{large})
\end{equation}
For small values of $x$ there seems to be no escape from numerical calculation. However, instead of doing the integral real time, it can be tabulated for regular values of $x$.
For values between the tabulated ones, the function can be calculated by a Taylor expansion centered at the nearest tabulated point $x_t$:
\begin{equation}
 F_n(x_t+\Delta x) = \sum_{k=0}^\infty\frac{F_{n+k}(x_t) (-\Delta x)^k}{k!}. \hspace{15mm} (x\hspace{2mm}\mathrm{small})
\end{equation}
}
\end{small}
}
\frame
{
  \frametitle{Integral evaluations, two-body Coulomb interaction and Boys function}
\begin{small}
{\scriptsize
The computational cost can be reduced even further by calculating the Boys function according to the description above only for the highest values of $n$ needed; for lower values of $n$ the function
can be found via the recursion relation
\begin{equation}
 F_n(x) = \frac{2xF_{n+1}(x)+e^{-x}}{2n+1},
\end{equation}
which can be shown by integrating the function by parts.
}
\end{small}
}
\frame
{
  \frametitle{Integral evaluations, summarizing}
\begin{small}
{\scriptsize
The Gaussian functions are given by
\begin{equation}
 \begin{split}
  G_a(\vec r) & = G_{ikm}(a, \vec r_A) = x^i_A\,y^j_A\,z^k_A\exp(-a r^2_A), \\
  G_b(\vec r) & = G_{jln}(b, \vec r_B) = x^i_B\,y^j_B\,z^k_B\exp(-b r^2_B),
 \end{split}
\end{equation}
where $\vec r_A = \vec r - \vec A$ and $\vec r_B = \vec r - \vec B$. We further define
\begin{equation}
\begin{split}
  p & = a + b, \\
 \vec P & = \frac{a\vec A + b\vec B}{a + b}.
\end{split}
\end{equation}
}
\end{small}
}
\frame
{
  \frametitle{Integral evaluations, summarizing}
\begin{small}
{\scriptsize
The overlap integral
\begin{equation}
 S_{ab} = \langle G_a|G_b\rangle
\end{equation}
is calculated as
\begin{equation}
 S_{ab} = E^{ij}_0\,E^{kl}_0\,E^{mn}_0\,\Big(\frac{\pi}{p}\Big)^{3/2},
\end{equation}
where
\begin{equation}
 E^{i=0,j=0}_0 = \exp(-\frac{ab}{a+b}X_{AB}^2),
\end{equation}
and the desired coefficients are found via
\begin{equation}
\begin{split}
 E^{i+1,j}_t & = \frac{1}{2p}E^{ij}_{t-1} + X_{PA}E^{ij}_t + (t+1)E^{ij}_{t+1}, \\
 E^{i,j+1}_t & = \frac{1}{2p}E^{ij}_{t-1} + X_{PB}E^{ij}_t + (t+1)E^{ij}_{t+1}.
\end{split}
\end{equation}
}
\end{small}
}
\frame
{
  \frametitle{Integral evaluations, summarizing}
\begin{small}
{\scriptsize
The kinetic integral is calculated as
\begin{equation}
 T_{ab} = -\frac{1}{2}(T_{ij}\,S_{kl}\,S_{mn} + S_{ij}\,T_{kl}\,S_{mn} + S_{ij}\,S_{kl}\,T_{mn}),
\end{equation}
where
\begin{equation}
 T_{ij} = 4b^2\,S_{i,j+2} - 2b(2j + 1)S_{i,j} + j(j-1)S_{i,j-2}.
\end{equation}
}
\end{small}
}
\frame
{
  \frametitle{Integral evaluations, summarizing}
\begin{small}
{\scriptsize
The Coulomb integral
\begin{equation}
 V_{ab} = \bra{G_a}\frac{1}{r_C}\ket{G_b}
\end{equation}
is calculated as
\begin{equation}
 V_{ab} = \frac{2\pi}{p}\sum_{tuv}E^{ab}_{tuv} R_{tuv}(p,\vec R_{PC}),
\end{equation}
where
\begin{equation}
 E^{ab}_{tuv} = E^{ij}_t\,E^{kl}_u\,E^{mn}_v,
\end{equation}
and $R_{tuv}(p,\vec R_{PC})$ is found by first calculating the source term
\begin{equation}
 R^n_{000}(p,\vec R_{PC}) = (-2p)^n\,F_n(p R^2_{PC})
\end{equation}
and then iterating towards the target $R^0_{tuv}(p,\vec R_{PC}) = R_{tuv}(p,\vec R_{PC})$ via the recurrence relations
\begin{equation}
 \begin{split}
  R^n_{t+1,u,v} & = tR^{n+1}_{t-1,u,v} + X_{PC} R^{n+1}_{tuv}, \\
  R^n_{t,u+1,v} & = uR^{n+1}_{t,u-1,v} + Y_{PC} R^{n+1}_{tuv}, \\
  R^n_{t,u,v+1} & = vR^{n+1}_{t,u,v11} + Z_{PC} R^{n+1}_{tuv}.
 \end{split}
\end{equation}
}
\end{small}
}
\frame
{
  \frametitle{Integral evaluations, summarizing}
\begin{small}
{\scriptsize
The Coulomb integral
\begin{equation}
 g_{acbd} = \bra{G_a G_c}\frac{1}{r_{12}}\ket{G_b G_d}
\end{equation}
is calculated as
\begin{equation}
 g_{acbd} = \frac{2\pi^{5/2}}{pq\sqrt{p+q}}\sum_{tuv}\sum_{\tau\nu\phi}(-1)^{\tau+\nu+\phi}E^{ab}_{tuv}E^{cd}_{\tau\nu\phi}R_{t+\tau,u+\nu,v+\phi}(\alpha,\vec R_{PQ}),
\end{equation}
where
\begin{equation}
  \alpha = \frac{pq}{p + q}.
\end{equation}
}
\end{small}
}

\frame
{
  \frametitle{New tasks}
How to proceed:
\begin{itemize}
\item First, try to object orient your program, as discussed during last week's lectures
\item Thereafter, implement GTOs for the helium and beryllium atoms using three and five primitive functions, a so-called STO-3G and STO-5L basis.
\item The next step is to use an STO-5L basis and perform restricted Hartree-Fock calculations for the neon atom.
\end{itemize}
}

\section[Weeks 10 and 11]{Weeks 10, 11 and 12}

\frame
{
  \frametitle{Topics for Weeks 10-12, March 3-21}
  \begin{block}{Gaussian type orbitals}
\begin{itemize}
\item Discussion of code structure
\item Rewrite of Hartree-Fock equations by explicit inclusion of spin
\item Integrals and storage of integrals
\end{itemize}
  \end{block}
} 

\frame
{
  \frametitle{Hartree-Fock equations with spin degrees of freedom}
\begin{small}
{\scriptsize
Till now we have mainly dealt with the Hartree-Fock equations using anti-symmetrized matrix elements. Since the Hamiltonian does not 
contain operators acting on the spin states, it is common to write out the spin-degrees of freedom in an explicit way.

We wrote the Hartree-Fock equations as
\[
  \hat{h}^{HF}(x_i) \psi_{p}(x_i) = \epsilon_{p}\psi_{p}(x_i),
\]
with
\[
  \hat{h}^{HF}(x_i)= \hat{h}_0(x_i) + \sum_{j=1}^NV_{j}^{d}(x_i) -
  \sum_{j=1}^NV_{j}^{ex}(x_i),
\]
and where $\hat{h}_0(i)$ is the one-body part.
In this equations we include both spin and spatial degrees of freedom.

The direct part is defined as 
\begin{equation*}
  V_{p}^{d}(x_i) = \int \psi_{p}^*(x_j)\psi_{p}(x_j)\hat{v}(r_{ij}) dx_j
\end{equation*}
while the exchange operator (or Fock operator) is
\begin{equation*}
  V_{p}^{ex}(x_i) \psi_{q}(x_i) 
  = \left(\int \psi_{p}^*(x_j) 
  \hat{v}(r_{ij})\psi_{q}(x_j)
  dx_j\right)\psi_{p}(x_i).
\end{equation*}

}
\end{small}
}
















\frame
{
  \frametitle{Hartree-Fock equations with spin degrees of freedom}
\begin{small}
{\scriptsize
If now deal explicitely with the spin degrees of freedom, we can write the single-particle state
\[
\psi_{p}(x_i) = \phi_{p}({\bf r}_i)\xi_{\sigma_i},
\]
where $\xi_{\sigma_i}$ is a standard Pauli spinor. Since we have only two possible spin values, the direct terms reduces then to
(recall that we have defined $\int dx_j = \sum_{\sigma_j} \int d{\bf r}_j$)
\begin{equation*}
  V_{p}^{d}({\bf r}_i)\phi_{q}({\bf r}_i) = 2\int \phi_{p}^*({\bf r}_j)\phi_{p}({\bf r}_i)\hat{v}(r_{ij}) d{\bf r}_j\phi_{q}({\bf r}_i)
\end{equation*}
while the exchange operator (or Fock operator) is
\begin{equation*}
  V_{p}^{ex}({\bf r}_i) \phi_{q}({\bf r}_i) 
  = \left(\int \phi_{p}^*({\bf r}_j) 
  \hat{v}(r_{ij})\phi_{q}({\bf r}_j)
  d{\bf r}_j\right)\phi_{p}({\bf r}_i).
\end{equation*}
}
\end{small}
}



\frame
{
  \frametitle{Hartree-Fock equations with spin degrees of freedom}
\begin{small}
{\scriptsize
In our Hartree-Fock calculation we expand the single-particle functions in terms of known basis functions (hydrogen-like one, STOs, GTOs etc), namely
\[
\phi_{p}({\bf r}_i) = \sum_{k=1}^{d}C_{pk}\chi_k({\bf r}_i),
\]
$d$ is the number of basis functions $\chi_k({\bf r}_i)$. The Hartree-Fock equations become then
\[
\hat{h}^{HF}\hat{C}_p=\epsilon^{HF}\hat{S}\hat{C}_p,
\]
where $\hat{S}$ is the overlap matrix needed in case the basis functions $\chi_k({\bf r}_i)$ are not normalized (typically, GTOs are not). 
Our Hartree-Fock Hamiltonian leads then to matrix elements (in a bra-ket notation)
\[
\langle p | \hat{h}^{HF} | q \rangle = \langle p|\hat{h}_0|q\rangle +\sum_{k\le F}\sum_{rs}C_{kr}^*C_{ks}\left(2\langle pr | \hat{v}|qs\rangle-\langle pr | \hat{v}|sq\rangle\right),
\]
with 
\[
\langle p|\hat{h}_0|q\rangle = \int \chi_{p}^*({\bf r}_j)\left(-\frac{1}{2}\nabla^2-\frac{Z}{{\bf r}_j}  \right)\chi_{q}({\bf r}_j)
  d{\bf r}_j.
\]
}
\end{small}
}


\frame
{
  \frametitle{Hartree-Fock equations with spin degrees of freedom}
\begin{small}
{\scriptsize
The other integrals are
\[
\langle pr | \hat{v}|qs\rangle = \int\int \chi_{p}^*({\bf r}_i)\chi_{q}^*({\bf r}_j)\hat{v}(r_{ij})\chi_{r}({\bf r}_i)\chi_{s}({\bf r}_j)
  d{\bf r}_id{\bf r}_j.
\]
If we then introduce the density matrix defined as 
\[
D_{pq}=\sum_{k\le F}C_{kp}C_{kq}^*,
\]
we can rewrite the Hartree-Fock matrix elements as
\[
\langle p | \hat{h}^{HF} | q \rangle = \langle p|\hat{h}_0|q\rangle +\sum_{rs}D_{rs}\left(2\langle pr | \hat{v}|qs\rangle-\langle pr | \hat{v}|sq\rangle\right),
\]
meaning that the only quantity we need to calculate at every
interation is the density matrix, which is evaluated using the eigenvector 
obtained from the diagonalization of the Hartree-Fock matrix.
}
\end{small}
}


\frame
{
  \frametitle{Hartree-Fock equations with spin degrees of freedom}
\begin{small}
{\scriptsize
From an algorithmic point of view, we see now that we need, with our GTO basis, to evaluate 
\begin{enumerate}
\item The overlap matrix $\hat{S}$.
\item The kinetic energy and one-body interaction matrix elements $\langle p|hat{h}_0|q\rangle $ and finally the
\item two-body interaction matrix elements $\langle pr | \hat{v}|qs\rangle$.
\end{enumerate}
All these elements can be computed once and for all and stored in
memory. The overlap matrix $S$ and the one-body matrix elements
$\langle p|\hat{h}_0|q\rangle $ can be stored as simple
one-dimensional arrays, or alternatively as matrices of small
dimensions.  The time-consuming part in the Hartree-Fock calculations
involves the calculation of the two-body matrix. Furthermore, the
storage of these matrix elements plays also an important role, in
particular we wish to access the table of matrix elements as fast as
possible.  
}
\end{small}
}

\frame
{
  \frametitle{Hartree-Fock equations with spin degrees of freedom}
\begin{small}
{\scriptsize
In a brute force algorithm for storing the matrix elements, if we have $d$ basis functions, we end up with the need of storing 
$d^4$ matrix elements. We can reduce this considerably by the following considerations.
In the calculation of the two-body matrix elements $\langle pr | \hat{v}|qs\rangle$ we have the following symmetries
\begin{enumerate}
\item Invariance under permutations, that is
\[
\langle pr | \hat{v}|qs\rangle = \langle rp | \hat{v}|sq\rangle.
\]
\item  The functions entering the evaluation of the integrals are all real, meaning that if we interchange $p\leftrightarrow q$ or 
$r\leftrightarrow s$, we end up with the same matrix element.
\end{enumerate}
This reduces by a factor of eight the total number of matrix elements to be stored. 
}
\end{small}
}











\frame
{
  \frametitle{Hartree-Fock equations with spin degrees of freedom}
\begin{small}
{\scriptsize
Furthermore, in setting up a table for the two-body matrix elements we can convert the need of using four indices $pqrs$ of 
$\langle pr | \hat{v}|qs\rangle$, which in a brute forces way could be coded as a four-dimensional array, to 
a two-dimensional array $V_{lm}$, where $l$ and $m$ stand for all possible two-body configurations $pq$.
 
Each number $l$ and $m$ in $V_{lm}$  should then point to a set of single-particle  states $(p,q)$ and $(r,s)$.  

In our case, since we have 
symmetries which allow us to set $p\le q$, we have, with $d$ single particle states a total of $d(d+1)/2$ two-body configurations.
How do we store such a matrix? The simplest thing to do is to convert it into a one-dimensional array. How do we achieve that? 
}
\end{small}
}







\frame
{
  \frametitle{Hartree-Fock equations with spin degrees of freedom}
\begin{small}
{\scriptsize
We now have a matrix $V$ of dimension $n\times n$ and we want to store the elements $V_{lm}$ as a one-dimensional array $A$ using
$0 \le l \le m \le n-1$. For
\begin{itemize}
\item $l=0$ we have $n$ elements
\item $l=1$ we have $n-1$ elements
\item $\dots$
\item $l=\nu$ we have $n-\nu$ elements
\item $\dots$
\item $l=n-1$ we have $1$ element,
\end{itemize}  
and the total number is  
\[
\sum_{\nu =0}^{n-1}\left(n-\nu\right)=\frac{n(n+1)}{2}.
\] 
}
\end{small}
}



\frame
{
  \frametitle{Hartree-Fock equations with spin degrees of freedom}
\begin{small}
{\scriptsize
To find the number ($\mathrm{number}(l,m)$) in a one-dimensional array $A$ which corresponds to a matrix element $V_{lm}$, we note that
\[
\mathrm{number}(l,m)=\sum_{\nu =0}^{l-1}\left(n-\nu\right)+m-l=\frac{l(2n-l-1)}{2}+m.
\] 
The first matrix element $V(0,0)$ is obviously given by the element $A(0)$. 

We have thus reduced a four dimensional array to a one-dimensional array, where the given pairs $(p,q)$ and $(r,s)$ point to the matrix indices $l$
and $m$, respectively. The latter are used to find the explicit number $\mathrm{number}(l,m)$ which points to the desired matrix element stored 
in a one-dimensional array.
}
\end{small}
}



\frame
{
  \frametitle{The hydrogen molecule, our next step}
\begin{small}
{\scriptsize
The 
H$_2$ molecule consists of two protons and two electrons 
with a ground state energy $E=-31.949$ eV or $-1.175$ a.u. and the 
equilibrium distance between the two hydrogen atoms
of $r_0=1.40$ Bohr radii (recall that a Bohr radius is $0.05\times 10^{-9}$m.


We define our systems using the following variables.
You can choose origo is chosen to be halfway between the two protons (this is just one possibility). The distance from 
proton 1 is then defined as 
$-{\bf R}/2$ whereas proton 2 has a distance ${\bf R}/2$.
Calculations are performed for fixed distances ${\bf R}$ between the two protons.
We will need here GTOs with two centers.
}
\end{small}
}


\frame
{
  \frametitle{The hydrogen molecule, our next step}
\begin{small}
{\scriptsize
Electron 1 has a distance $r_1$ from the chose origo, while  electron $2$
has a distance $r_2$. 
The kinetic energy operator becomes then
\[
   -\frac{\nabla_1^2}{2}-\frac{\nabla_2^2}{2}.
\]
The distance between the two electrons is
$r_{12}=|{\bf r}_1-{\bf r}_2|$. 
The repulsion between the two electrons results in a potential energy term given by
\[
               +\frac{1}{r_{12}}.
\]
In a similar way we obtain a repulsive contribution from the interaction between the two 
protons given by
\[
               +\frac{1}{|{\bf R}|},
\]
where ${\bf R}$ is the distance between the two protons.
}
\end{small}
}


\frame
{
  \frametitle{The hydrogen molecule, our next step}
\begin{small}
{\scriptsize
To obtain the final potential energy we need to include the attraction the electrons feel from the protons.
To model this, we need to define the distance between the electrons and the two protons.
If we model this along a 
chosen $z$-akse with electron 1 placed at a distance 
${\bf r}_1$ from a chose origo, one proton at $-{\bf R}/2$
and the other at  ${\bf R}/2$, 
the distance from proton 1 to electron 1 becomes
\[
{\bf r}_{1p1}={\bf r}_1+ {\bf R}/2,
\]
and
\[
{\bf r}_{1p2}={\bf r}_1- {\bf R}/2,
\]
from proton 2.
}
\end{small}
}


\frame
{
  \frametitle{The hydrogen molecule, our next step}
\begin{small}
{\scriptsize
Similarly, for electron 2 we obtain
\[
{\bf r}_{2p1}={\bf r}_2+{\bf R}/2,
\]
and
\[
{\bf r}_{2p2}={\bf r}_2-{\bf R}/2.
\]
These four distances define the attractive contributions to the potential energy
\[
   -\frac{1}{r_{1p1}}-\frac{1}{r_{1p2}}-\frac{1}{r_{2p1}}-\frac{1}{r_{2p2}}.
\]
We can then write the total Hamiltonian as 
\[
   \OP{H}=-\frac{\nabla_1^2}{2}-\frac{\nabla_2^2}{2}
   -\frac{1}{r_{1p1}}-\frac{1}{r_{1p2}}-\frac{1}{r_{2p1}}-\frac{1}{r_{2p2}}
               +\frac{1}{r_{12}}+\frac{1}{|{\bf R}|}.
\]
}
\end{small}
}

\frame
{
  \frametitle{The hydrogen molecule, our next step}
\begin{small}
{\scriptsize
If we use standard hydrogen-like wave functions, one can make an ansatz for the 
Slater determinant by using the following single-particle states
\[
   \psi({\bf r}_1,{\bf R})=\left(\exp{(-\alpha r_{1p1})}
      +\exp{(-\alpha r_{1p2})}\right),
\]
for electron 1 and
\[
   \psi({\bf r}_2,{\bf R})=\left(\exp{(-\alpha r_{2p1})}
      +\exp{(-\alpha r_{2p2})}\right),
\]
for electron 2, with
$\alpha$ being a variational parameter to be optimized. 
We will use GTOs with two centers, corresponding to the atomic nuclei of the hydrogen molecules.
}
\end{small}
}
\frame
{
  \frametitle{The Be$_2$ molecule, the next step}
\begin{small}
{\scriptsize
The next step consists in estimating the binding energy of the Be$_2$ molecule.
Useful references are then
\begin{enumerate}
\item Moskowitz and Kalos, Int.~Journal of Quantum Chemistry {\bf XX}, 1107 (1981).
Results for He and H$_2$.
\item Filippi, Singh and Umrigar, J.~Chemical Physics {\bf 105}, 123 (1996).   Useful results on
Be$_2$.
\item J\o rgen H\o gberget, Master of Science thesis University of Oslo, 2013.
\end{enumerate}
With a functioning code for the H$_2$ and Be$_2$ molecules we will next move over to calculations
of closed shell atoms like Ne and Ar and molecules like H$_2$O and SiO$_2$. 
For the latter systems we need also to perform a so-called unrestricted Hartree-Fock calculation.
}
\end{small}
}

\frame
{
  \frametitle{Density distribution}
\begin{small}
{\scriptsize
An important quantity which can be related to the charge distribution of electrons,
is the so-called density distribution  (multiplying with the electrical charge gives the charge distribution) defined as
\[
\rho({\bf r})=\rho({\bf r}_1) = \int d{\bf r}_2\dots d{\bf r}_N \left| \Psi({\bf r}_1,{\bf r}_2\dots {\bf r}_N)\right|^2, 
\]
which in a Hartree-Fock based approach becomes 
\[
\rho({\bf r})=\sum_{i=1}^{N}\left| \psi_i({\bf r})\right|^2,
\]
with $\psi$ the single-particle wave functions (our best basis).  A useful check that the numerics works properly is 
to  integrate the density distributions since it has to give us the total number of electrons in the system, that is
\[
N=\int d{\bf r}\rho({\bf r}).
\]
This is a useful test of the numerics. 
}
\end{small}
}


