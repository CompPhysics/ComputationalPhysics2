%%
%% Automatically generated file from DocOnce source
%% (https://github.com/hplgit/doconce/)
%%
%%


%-------------------- begin preamble ----------------------

\documentclass[%
oneside,                 % oneside: electronic viewing, twoside: printing
final,                   % draft: marks overfull hboxes, figures with paths
10pt]{article}

\listfiles               %  print all files needed to compile this document

\usepackage{relsize,makeidx,color,setspace,amsmath,amsfonts,amssymb}
\usepackage[table]{xcolor}
\usepackage{bm,ltablex,microtype}

\usepackage[pdftex]{graphicx}

\usepackage{fancyvrb} % packages needed for verbatim environments

\usepackage[T1]{fontenc}
%\usepackage[latin1]{inputenc}
\usepackage{ucs}
\usepackage[utf8x]{inputenc}

\usepackage{lmodern}         % Latin Modern fonts derived from Computer Modern

% Hyperlinks in PDF:
\definecolor{linkcolor}{rgb}{0,0,0.4}
\usepackage{hyperref}
\hypersetup{
    breaklinks=true,
    colorlinks=true,
    linkcolor=linkcolor,
    urlcolor=linkcolor,
    citecolor=black,
    filecolor=black,
    %filecolor=blue,
    pdfmenubar=true,
    pdftoolbar=true,
    bookmarksdepth=3   % Uncomment (and tweak) for PDF bookmarks with more levels than the TOC
    }
%\hyperbaseurl{}   % hyperlinks are relative to this root

\setcounter{tocdepth}{2}  % levels in table of contents

% --- fancyhdr package for fancy headers ---
\usepackage{fancyhdr}
\fancyhf{} % sets both header and footer to nothing
\renewcommand{\headrulewidth}{0pt}
\fancyfoot[LE,RO]{\thepage}
% Ensure copyright on titlepage (article style) and chapter pages (book style)
\fancypagestyle{plain}{
  \fancyhf{}
  \fancyfoot[C]{{\footnotesize \copyright\ 1999-2021, "Computational Physics II FYS4411/FYS9411":"http://www.uio.no/studier/emner/matnat/fys/FYS4411/index-eng.html". Released under CC Attribution-NonCommercial 4.0 license}}
%  \renewcommand{\footrulewidth}{0mm}
  \renewcommand{\headrulewidth}{0mm}
}
% Ensure copyright on titlepages with \thispagestyle{empty}
\fancypagestyle{empty}{
  \fancyhf{}
  \fancyfoot[C]{{\footnotesize \copyright\ 1999-2021, "Computational Physics II FYS4411/FYS9411":"http://www.uio.no/studier/emner/matnat/fys/FYS4411/index-eng.html". Released under CC Attribution-NonCommercial 4.0 license}}
  \renewcommand{\footrulewidth}{0mm}
  \renewcommand{\headrulewidth}{0mm}
}

\pagestyle{fancy}


% prevent orhpans and widows
\clubpenalty = 10000
\widowpenalty = 10000

% --- end of standard preamble for documents ---


% insert custom LaTeX commands...

\raggedbottom
\makeindex
\usepackage[totoc]{idxlayout}   % for index in the toc
\usepackage[nottoc]{tocbibind}  % for references/bibliography in the toc

%-------------------- end preamble ----------------------

\begin{document}

% matching end for #ifdef PREAMBLE

\newcommand{\exercisesection}[1]{\subsection*{#1}}


% ------------------- main content ----------------------



% ----------------- title -------------------------

\thispagestyle{empty}

\begin{center}
{\LARGE\bf
\begin{spacing}{1.25}
Project 2, Quantum Machine Learning. Deadline June 1, Spring 2021
\end{spacing}
}
\end{center}

% ----------------- author(s) -------------------------

\begin{center}
{\bf \href{{http://www.uio.no/studier/emner/matnat/fys/FYS4411/index-eng.html}}{Computational Physics II FYS4411/FYS9411}}
\end{center}

    \begin{center}
% List of all institutions:
\centerline{{\small Department of Physics, University of Oslo, Norway}}
\end{center}
    
% ----------------- end author(s) -------------------------

% --- begin date ---
\begin{center}
Apr 8, 2021
\end{center}
% --- end date ---

\vspace{1cm}


\subsection*{Introduction}

For this project, you will perform quantum machine learning on the Scikit learn breast cancer data set. The data can be obtained the following way

\begin{print}
from sklearn.datasets import load_breast_cancer

data = load_breast_cancer()
x = data.data #features
y = data.target #targets

\end{print}
$x$ is the feature matrix and $y$ are the targets.

\paragraph{Project 2 a): Encoding the Data Into a Quantum State.}
For this task you will consider a simple way of encoding a randomly generated data set sample into a quantum state:

\begin{print}
import qiskit as qk
import numpy as np
np.random.seed(42)

p = 2 #number of features
data_register = qk.QuantumRegister(p)
classical_register = qk.ClassicalRegister(1)

circuit = qk.QuantumCircuit(data_register, classical_register) 

sample = np.random.uniform(size=p)
target = np.random.uniform(size=1)

for feature_idx in range(p):
    circuit.ry(2*np.pi*sample[feature_idx],data_register[feature_idx])

print(circuit)
\end{print}

The above code shows how a randomly generated data sample of $p=2$
features are encoded into a quantum state on two qubits utilizing
Qiskit. Each feature is encoded into a respective qubit utilizing a
$R_y(\theta)$ gate. The features are scaled with $2\pi$ to represent
rotation angles (the $R_y(\theta)$ gate performs a rotation). The
classical register will be used later for storing the measured value
of the circuit.  print(circuit) can be utilized at any point to see
what the circuit looks like.



Your task is to get familiar with the functionality utilized in the
above example and implement your own function to encode $p$ of the
first features in the breast cancer data set to a quantum state.


\paragraph{Project 2 b): Processing the Encoded Data with Parameterized Gates.}
After the quantum state has been encoded with the information of a data set sample, one needs extend the circuit with operations that process the state in a way that allows us to infer the target data. This can be done by introducing quantum gates that are dependant on learnable parameters $\boldsymbol{\theta}$. We will do this in a similar fashion as for the encoding of the features:

\begin{print}
n_params = 4
theta = 2*np.pi*np.random.uniform(size=n_params)

circuit.rx(theta[0],data_register[0])
circuit.ry(theta[1],data_register[1])
circuit.cx(data_register[0],data_register[1])
circuit.ry(theta[2],data_register[0])
circuit.rx(theta[3],data_register[1])

print(circuit)
\end{print}

The above parameterization of the quantum state is what we will refer
to as the 'ansatz'. Your task is again to familiarize yourself with
the functionality utilized in the above example and implement your own
ansatz to be utilized together with the $p$ first features of the
breast cancer data set. The number of learnable parameters 'theta'
should be arbitrary.



\paragraph{Project 2c):Measuring the Quantum State and Making Inference.}
The next step is to generate a prediction from our quantum machine learning model. This is done by performing a measurement on the quantum state:

\begin{print}
circuit.measure(data_register[-1],classical_register[0])
shots=1000

job = qk.execute(circuit,
                backend=qk.Aer.get_backend('qasm_simulator'),
                shots=shots,
                seed_simulator=42
                )
results = job.result()
results = results.get_counts(circuit)

prediction = 0
for key,value in results.items():
    if key == '1':
        prediction += value
prediction/=shots
print('Prediction:',prediction,'Target:',target[0])
\end{print}
\begin{print}
    Prediction: 0.285 Target: 0.7319939418114051
\end{print}

In the above example, we are first applying a measurement operation on
the final qubit in the circuit, and we are interpreting our prediction
as the probability that this qubit is in the $\ket{1}$ state. Make
sure all the steps in the example are understood.

Implement your own function that generates a prediction by measuring one of the qubits.



\paragraph{Project 2d): Putting it all together.}
Now it is time to put together all of the above steps. Ideally, you
should make a class or a function that given a feature matrix of $n$
samples and an arbitrary number of model parameters, returns a vector
of $n$ outputs. For example:

\begin{print}
n = 100 #number of samples
p = 10 #number of features
theta = np.random.uniform(size=20) #array of model parameters
X = np.random.uniform(size=(n,p)) #design matrix
y_pred = model(X,theta) #prediction, shape (n)
\end{print}



We will now deal with how to train the model:

\paragraph{Project 2e): Parameter Shift-Rule and Calculating the Analytical Gradient.}
Since the model with random initial parameters is no good for
inference, we need to optimize the parameters in order to yield good
results, as is the usual with machine learning.

Since we are dealing with classification, we will use cross-entropy as the loss function

\begin{equation*}
    L = -\sum_{i=1}^{n}{y_i \ln{f(x_i;\boldsymbol{\theta})}},
\end{equation*}

where $y_i$ are the target labels, and
$f(\boldsymbol{x}_i;\boldsymbol{\theta})$ is the output of our model
for a given sample $\boldsymbol{x}_i$ and parameterization
$\boldsymbol{\theta})$. We calculate the gradiant by taking the
derivative of the loss with respect to the parameters

\begin{equation*}
    \frac{\partial}{\partial \boldsymbol{\theta}_k}L = \sum_{i=1}^{n}{\frac{f_i - y_i}{f_i(1 - f_i)}} \frac{\partial}{\partial \boldsymbol{\theta}_k}f_i,
\end{equation*}

where $f_i = f(x_i;\boldsymbol{\theta})$ for clarity. The only term we do not know how to calculate is $\frac{\partial}{\partial \boldsymbol{\theta}_k}f(x_i;\boldsymbol{\theta})$, but it turns out there is a simple trick to do this, the so-called parameter shift-rule \\cite{ParameterShift}. To calculate the derivative of the model output, we need to evaluate the model twice with the respective parameter shifted by a value $\frac{\pi}{2}$ up and down. The two resulting outputs are then put together to yield the derivative



\begin{equation*}
    \frac{\partial f(x_i; \theta_1, \theta_2, \dots, \theta_k)}{\partial \theta_j}  = \frac{f(x_i; \theta_1, \theta_2, \dots, \theta_j + \pi /2, \dots, \theta_k) -f(x_i; \theta_1, \theta_2, \dots, \theta_j - \pi /2, \dots, \theta_k)}{2}
\end{equation*}

Train your model by utilizing the Parameter Shift-Rule and some
gradient descent algorithm. Compare your results with for example
logistic regression.


\paragraph{Project 2f): Adding Variations on the Data Encoding and Ansatz.}
Change the gates utilized for the encoding of the data samples and also make changes to the parameterized ansatz. Train these new models on the breast cancer data set. How do they compare?






\subsection*{Introduction to numerical projects}

Here follows a brief recipe and recommendation on how to write a report for each
project.

\begin{itemize}
  \item Give a short description of the nature of the problem and the eventual  numerical methods you have used.

  \item Describe the algorithm you have used and/or developed. Here you may find it convenient to use pseudocoding. In many cases you can describe the algorithm in the program itself.

  \item Include the source code of your program. Comment your program properly.

  \item If possible, try to find analytic solutions, or known limits in order to test your program when developing the code.

  \item Include your results either in figure form or in a table. Remember to        label your results. All tables and figures should have relevant captions        and labels on the axes.

  \item Try to evaluate the reliabilty and numerical stability/precision of your results. If possible, include a qualitative and/or quantitative discussion of the numerical stability, eventual loss of precision etc.

  \item Try to give an interpretation of you results in your answers to  the problems.

  \item Critique: if possible include your comments and reflections about the  exercise, whether you felt you learnt something, ideas for improvements and  other thoughts you've made when solving the exercise. We wish to keep this course at the interactive level and your comments can help us improve it.

  \item Try to establish a practice where you log your work at the  computerlab. You may find such a logbook very handy at later stages in your work, especially when you don't properly remember  what a previous test version  of your program did. Here you could also record  the time spent on solving the exercise, various algorithms you may have tested or other topics which you feel worthy of mentioning.
\end{itemize}

\noindent
\subsection*{Format for electronic delivery of report and programs}

The preferred format for the report is a PDF file. You can also use DOC or postscript formats or as an ipython notebook file.  As programming language we prefer that you choose between C/C++, Fortran2008 or Python. The following prescription should be followed when preparing the report:

\begin{itemize}
  \item Use canvas to hand in your projects, log in  at  \href{{http://canvas.uio.no}}{\nolinkurl{http://canvas.uio.no}} with your normal UiO username and password.

  \item Upload \textbf{only} the report file!  For the source code file(s) you have developed please provide us with your link to your github domain.  The report file should include all of your discussions and a list of the codes you have developed.  The full version of the codes should be in your github repository.

  \item In your github repository, please include a folder which contains selected results. These can be in the form of output from your code for a selected set of runs and input parameters.

  \item Still in your github make a folder where you place your codes. 

  \item In this and all later projects, you should include tests (for example unit tests) of your code(s).

  \item Comments  from us on your projects, approval or not, corrections to be made  etc can be found under your Devilry domain and are only visible to you and the teachers of the course.
\end{itemize}

\noindent
Finally, 
we encourage you to work two and two together. Optimal working groups consist of 
2-3 students. You can then hand in a common report. 














% ------------------- end of main content ---------------

\end{document}

