%%
%% Automatically generated file from DocOnce source
%% (https://github.com/doconce/doconce/)
%% doconce format latex Project2PI.do.txt --print_latex_style=trac --latex_admon=paragraph
%%
% #ifdef PTEX2TEX_EXPLANATION
%%
%% The file follows the ptex2tex extended LaTeX format, see
%% ptex2tex: https://code.google.com/p/ptex2tex/
%%
%% Run
%%      ptex2tex myfile
%% or
%%      doconce ptex2tex myfile
%%
%% to turn myfile.p.tex into an ordinary LaTeX file myfile.tex.
%% (The ptex2tex program: https://code.google.com/p/ptex2tex)
%% Many preprocess options can be added to ptex2tex or doconce ptex2tex
%%
%%      ptex2tex -DMINTED myfile
%%      doconce ptex2tex myfile envir=minted
%%
%% ptex2tex will typeset code environments according to a global or local
%% .ptex2tex.cfg configure file. doconce ptex2tex will typeset code
%% according to options on the command line (just type doconce ptex2tex to
%% see examples). If doconce ptex2tex has envir=minted, it enables the
%% minted style without needing -DMINTED.
% #endif

% #define PREAMBLE

% #ifdef PREAMBLE
%-------------------- begin preamble ----------------------

\documentclass[%
oneside,                 % oneside: electronic viewing, twoside: printing
final,                   % draft: marks overfull hboxes, figures with paths
10pt]{article}

\listfiles               %  print all files needed to compile this document

\usepackage{relsize,makeidx,color,setspace,amsmath,amsfonts,amssymb}
\usepackage[table]{xcolor}
\usepackage{bm,ltablex,microtype}

\usepackage[pdftex]{graphicx}

\usepackage[T1]{fontenc}
%\usepackage[latin1]{inputenc}
\usepackage{ucs}
\usepackage[utf8x]{inputenc}

\usepackage{lmodern}         % Latin Modern fonts derived from Computer Modern

% Hyperlinks in PDF:
\definecolor{linkcolor}{rgb}{0,0,0.4}
\usepackage{hyperref}
\hypersetup{
    breaklinks=true,
    colorlinks=true,
    linkcolor=linkcolor,
    urlcolor=linkcolor,
    citecolor=black,
    filecolor=black,
    %filecolor=blue,
    pdfmenubar=true,
    pdftoolbar=true,
    bookmarksdepth=3   % Uncomment (and tweak) for PDF bookmarks with more levels than the TOC
    }
%\hyperbaseurl{}   % hyperlinks are relative to this root

\setcounter{tocdepth}{2}  % levels in table of contents

% --- fancyhdr package for fancy headers ---
\usepackage{fancyhdr}
\fancyhf{} % sets both header and footer to nothing
\renewcommand{\headrulewidth}{0pt}
\fancyfoot[LE,RO]{\thepage}
% Ensure copyright on titlepage (article style) and chapter pages (book style)
\fancypagestyle{plain}{
  \fancyhf{}
  \fancyfoot[C]{{\footnotesize \copyright\ 1999-2022, "Computational Physics II FYS4411/FYS9411":"http://www.uio.no/studier/emner/matnat/fys/FYS4411/index-eng.html". Released under CC Attribution-NonCommercial 4.0 license}}
%  \renewcommand{\footrulewidth}{0mm}
  \renewcommand{\headrulewidth}{0mm}
}
% Ensure copyright on titlepages with \thispagestyle{empty}
\fancypagestyle{empty}{
  \fancyhf{}
  \fancyfoot[C]{{\footnotesize \copyright\ 1999-2022, "Computational Physics II FYS4411/FYS9411":"http://www.uio.no/studier/emner/matnat/fys/FYS4411/index-eng.html". Released under CC Attribution-NonCommercial 4.0 license}}
  \renewcommand{\footrulewidth}{0mm}
  \renewcommand{\headrulewidth}{0mm}
}

\pagestyle{fancy}


% prevent orhpans and widows
\clubpenalty = 10000
\widowpenalty = 10000

% --- end of standard preamble for documents ---


% insert custom LaTeX commands...

\raggedbottom
\makeindex
\usepackage[totoc]{idxlayout}   % for index in the toc
\usepackage[nottoc]{tocbibind}  % for references/bibliography in the toc

%-------------------- end preamble ----------------------

\begin{document}

% matching end for #ifdef PREAMBLE
% #endif

\newcommand{\exercisesection}[1]{\subsection*{#1}}


% ------------------- main content ----------------------



% ----------------- title -------------------------

\thispagestyle{empty}

\begin{center}
{\LARGE\bf
\begin{spacing}{1.25}
Project 2, Path integral formalism. Deadline June 1, Spring 2022
\end{spacing}
}
\end{center}

% ----------------- author(s) -------------------------

\begin{center}
{\bf \href{{http://www.uio.no/studier/emner/matnat/fys/FYS4411/index-eng.html}}{Computational Physics II FYS4411/FYS9411}}
\end{center}

    \begin{center}
% List of all institutions:
\centerline{{\small Department of Physics, University of Oslo, Norway}}
\end{center}
    
% ----------------- end author(s) -------------------------

% --- begin date ---
\begin{center}
May 13, 2022
\end{center}
% --- end date ---

\vspace{1cm}


\subsection{Path Integrals: Concepts and Formalism}

The aim of this project is to study the path integral formalism
applied first to a Harmonic oscillator problem and then to a spin-like Hamiltonian. This provide us with
many of the basic elements for moving to part b). Part a) starts with
a field-theoretic approach reformulating the path integral in
euclidean time, see for example the text of \href{{https://link.springer.com/book/10.1007/978-3-642-01850-3}}{Gattringer and
Lang}.

It introduces concepts like the \href{{https://arxiv.org/abs/1912.08854}}{Trotter-Suzuki
approximation} and how to evaluate
the energy for a simple Hamiltonian like the harmonic oscillator in
one dimension. It bridges thus the gap between project 1 and the
present as well as serving as an introduction to part b).  For the
second part, due to the structure of the Hamiltonian, one can
circumvent the needs for doing a Trotter expansion.

In the path integral formalism the evolution from a state $\vert x_i\rangle$
to a state $\vert  x_f\rangle$ from time $t_i$ to time $t_f$ is given by:

\[
	\langle x_f\vert e^{-\hat{H}(t_f-t_i)}\vert x_i\rangle = \int \mathcal{D} x(t) e^{-S[x]}
\]

On the left-hand side we have the standard quantum mechanical time
evolution operator with $\hat{H}$ being the Hamiltonian operator of
the system, the right-hand side represents the path integral of all possible paths
x(t) from $x_i$ to $x_f$ with $t = t_i \rightarrow t_f$ weighted with
the classical action $S[x]$:

\[
	S[x] = \int_{t_i}^{t_f} dt L(x,\dot{x}) =  \int_{t_i}^{t_f} dt \left( \frac{m\dot{x}(t)^2}{2} - V(x(t))\right),
\]

and  knowledge of the propagator gives full information about the system.

In a simple one dimensional case the path integral can be written as

\[
	 \int \mathcal{D} x(t) \rightarrow \int_{-\infty}^{\infty} dx_1dx_2\dots dx_{N-1}
\]

where time has been discretized in $N$ slices from $t_i = t_0$ to $t_f = t_N$. Consequently $x_i = x(t_i)$.  

Given some functional $\Gamma[x]$ that represents an observable at
intermediate times of the evolution of the system we can write its
expectation value as:

\[
	\langle\Gamma[x]\rangle = \frac{1}{Z} \int \mathcal{D} x(t) \Gamma[x] e^{-S[x]}
\]

where the normalization $Z$ is given by 

\[
	Z = \int \mathcal{D} x(t) e^{-S[x]}
\]

This can be computed using a set of configurations $x^{(k)} =
\{x^{(k)}_1x^{(k)}_2\dots x^{(k)}_{N-1} \}$ for $k = 1,2,\dots
N_{cf}$. The Monte Carlo estimator then becomes:

\[
	\langle\Gamma[x]\rangle \approx \bar{\Gamma} = \frac{1}{N_{cf}} \sum_{k = 1}^k \Gamma[x^{(k)}]
\]

As and aside, if we wish to  generalize our theory to a four dimensional
euclidean space time an observable that depends on a scalar field can
be computed in a very similar fashion

\[
	\langle\Gamma[\phi]\rangle = \frac{1}{Z} \int \mathcal{D} \phi \Gamma[\phi] e^{-S[\phi]}
\]

where the normalization $Z$ is given again by 

\[
	Z = \int \mathcal{D} \phi e^{-S[\phi]}
\]

But this time the path integral can be computed numerically by discretizing the 4 dimensional space in a lattice:

\[
	 \int \mathcal{D} \phi \rightarrow \int \prod_{x_i \in lattice} d\phi(x_i)
\] 

The first exercise here follows closely \href{{https://arxiv.org/pdf/hep-lat/0506036.pdf}}{LePage's article}. See also Ref.~[1] below.

\subsection{The Harmonic Oscillator: a Simple Example}

\paragraph{Project 2 a):}
A first example of the path integral solution of quantum mechanical
systems is the one-dimensional harmonic oscillator. It is particularly
well suited because of:

\begin{enumerate}
\item simple dimensional discretization

\item well-known results for expectation values

\item fast analytical analysis, implementation and runs 
\end{enumerate}

\noindent
Here we follow \href{{https://arxiv.org/pdf/hep-lat/0506036.pdf}}{LePage} in defining
the classical action for the harmonic oscillator. This action is given by

\[
	S[x] = \int_{t_i}^{t_f} \left[ \frac{m\dot{x}(t)^2}{2} + \frac{1}{2}\omega x(t)^2\right] dt.
\]

We can discretize the time domain into $N$ slices of step $a$, which
imposes us to discretize the time derivative of the position as
well. When considering the integral between time $t_j$ and $t_{j+1}$
the action becomes

\[
	S_{j} \approx a\left[ \frac{m(x_{j+1}-x_j)^2}{2a^2} + \frac{1}{4}\omega (x_{j+1} - x_j)^2 \right] dt.
\]

When summing over all time slices (and applying periodic boundary conditions) we get

\[	
	S_{latt} = \sum_{j = 0}^{N-1} \left[ \frac{m}{2a}(x_{j+1}-x_j)^2 + \frac{a}{2} x_j^2 \right]
\]

In order to extract information from the system we need to construct
and study an observable. We can look at a two point correlator of the
form $\langle x(t_2)x(t_1) \rangle$.  In the continuum this is
equivalent to the calculation of the ratio of path integrals

\[
	\langle x(t_2)x(t_1) \rangle = \frac{\int \mathcal{D} x(t) x(t_2) x(t_1) e^{-S[x]}}{\int \mathcal{D} x(t) e^{-S[x]}}
\]

In quantum mechanics the numerator (using the definition of the path integral) is equal to

\[	
v	\int dx \langle x\vert  e^{\bar{H}(t_f-t2)} \tilde{x} e^{\bar{H}(t_2-t1)} \tilde{x} e^{\bar{H}(t_1-ti)} \vert x\rangle,
\]

If we look at the full integral, let the hamiltionian operator act on
the states leaving the spectral decomposition of them, and substitute
$T = t_f-Ti$ and $t = t_2 - t_1$ we get to

\[
	\langle x(t_2)x(t_1) \rangle = \frac{\sum e^{-E_n T} \langle E_n\vert \tilde{x} e^{-(\bar{H}-E_n)t}  \tilde{x}\vert E_n\rangle }{\sum e^{-E_n T }}
\]

and for $T \gg t$ the ground state will dominate the summation:

\[
	G(t) = \langle x(t_2)x(t_1)\rangle \rightarrow  \langle E_0\vert \tilde{x} e^{-(\bar{H}-E_0)t}  \tilde{x}\vert E_0\rangle,
\]

and letting $t$ to be large as well the propagator ends up linking the two lowest energy states

\[
	G(t) \rightarrow  |\langle E_0\vert \tilde{x} \vert E_1\rangle|^2e^{-(E_1-E_0)t}.
\]

We can now extract the first excitation energy of the quantum mechanical harmonic oscillator, in the limit of $t$ large, as

\[
	\log\left( \frac{G(t)}{G(t+\Delta t)}\right) = (E_1 - E_0)\Delta t.
\]

On our discretized lattice the correlator can be obtained numerically by directly computing the average value of the operator

\[
	G(t) = \frac{1}{N}\sum_j\langle x(t_j+t)x(t_j)\rangle,
\]
for all $t$ in $0, a, 2a\dots (N-1)a$. This might look trivial at a first glance, but it is the key point of the whole algorithm.

The computational algorithm required to compute the correlator is
based on the idea of creating a Markov chain of possible paths for the
harmonic oscillator, via subsequent updates. This procedure is called
Metropolis algorithm:

\begin{enumerate}
\item initialize an array of $N$ position values for each time point (for example set everything to 0)

\item suggest a new random configuration starting from the current one

\item accept or reject the update, based on the action difference

\item compute the correlator for the current configuration

\item repeat the process $N_{cf}$ times, sufficiently large to have a statistically relevant ensemble.
\end{enumerate}

\noindent
In order to limit the auto correlation between data, the observable is
only computed once every $N_{corr}$ updates.  The rule for accepting
an update is: $\exp(-\Delta S) > \xi$. Where $\xi \in [0,1]$, a
randomly chosen number. This implies that if the classical action is
reduced the update is automatically accepted, if it is positive, a
random number $\xi$ is generated and compared with the exponential of
the action. This condition allows the system to explore all the phase
space and not get stuck in a local minimum for example.

Your task is to
run for the harmonic oscillator  using $10^5-10^7$ configurations ($N_{conf}$), skipping 20
updates between every measurement ($N_{corr}$). The time axis is
discretized into 20 nodes. Data, to improve the estimate of the error can be resampled using the bootstrap or the blocking
techniques.  The result that is physically relevant is the first
excitation energy, which we expect to be $1\hbar\omega$. For
simplicity we set $\hbar=\omega=1$ and the lattice spacing parameter
$a = 0.5$.

Implement statistical bootstrapping and blocking. 
Estimate energy and how it relates  with the correlation time. This exercise follows closely \href{{https://arxiv.org/pdf/hep-lat/0506036.pdf}}{LePage's article}. You may find it a useful read.

\paragraph{Project 2 b):}
Here the aim is to explore the Heisenberg-model in one and two
dimensions, by use of stochastic series expansion (SSE). Hopefully, we
will be able to calculate the critical temperature for the continious
phase transition. With this value the critical exponents for the
system can also be obtained.  The results should be compared with the
relevant literature as provided in the references.

Implementation of SSE for 1D Heisenberg model with periodic bound- ary conditions.
\begin{enumerate}
\item Add external field and anisotropy dependencies.

\item 2D implementation, again with periodic boundary conditions.

\item Study the temperature dependent phase transition and calculate the critical exponents for the system. By studying the system the follow- ing values should be attainable; $M$, $M^2$, $E$, $\xi$, $C_V$ , $C(r)$, where $M$ is magnetization, $E$ is the energy, $\xi$ is the magnetic suceptibility, $C_V$ is the heat capacity and $C(r)$ is the correlation length.
\end{enumerate}

\noindent
\paragraph{Literature.}
\begin{enumerate}
\item G. P. LePage, Lattice QCD for novices, \href{{https://arxiv.org/pdf/hep-lat/0506036.pdf}}{\nolinkurl{https://arxiv.org/pdf/hep-lat/0506036.pdf}}, 2005

\item A. W. Sandvik. Computational Studies of Quantum Spin Systems. \href{{https://doi.org/10.1063/1.3518900}}{\nolinkurl{https://doi.org/10.1063/1.3518900}}, Jan. 2011.

\item A. W. Sandvik. Finite-size scaling of the ground-state parameters of the two-dimensional heisenberg model. Phys. Rev. B, 56:11678–11690, Nov 1997.

\item A. W. Sandvik. Stochastic series expansion method with operator-loop update. Phys. Rev. B, 59:R14157–R14160, Jun 1999.

\item O. F. Syljuasen and A. W. Sandvik. Quantum Monte Carlo with directed loops. Phys. Rev. E, 66:046701, Oct 2002.
\end{enumerate}

\noindent
\subsection{Introduction to numerical projects}

Here follows a brief recipe and recommendation on how to write a report for each
project.

\begin{itemize}
  \item Give a short description of the nature of the problem and the eventual  numerical methods you have used.

  \item Describe the algorithm you have used and/or developed. Here you may find it convenient to use pseudocoding. In many cases you can describe the algorithm in the program itself.

  \item Include the source code of your program. Comment your program properly.

  \item If possible, try to find analytic solutions, or known limits in order to test your program when developing the code.

  \item Include your results either in figure form or in a table. Remember to        label your results. All tables and figures should have relevant captions        and labels on the axes.

  \item Try to evaluate the reliabilty and numerical stability/precision of your results. If possible, include a qualitative and/or quantitative discussion of the numerical stability, eventual loss of precision etc.

  \item Try to give an interpretation of you results in your answers to  the problems.

  \item Critique: if possible include your comments and reflections about the  exercise, whether you felt you learnt something, ideas for improvements and  other thoughts you've made when solving the exercise. We wish to keep this course at the interactive level and your comments can help us improve it.

  \item Try to establish a practice where you log your work at the  computerlab. You may find such a logbook very handy at later stages in your work, especially when you don't properly remember  what a previous test version  of your program did. Here you could also record  the time spent on solving the exercise, various algorithms you may have tested or other topics which you feel worthy of mentioning.
\end{itemize}

\noindent
\subsection{Format for electronic delivery of report and programs}

The preferred format for the report is a PDF file. You can also use DOC or postscript formats or as an ipython notebook file.  As programming language we prefer that you choose between C/C++, Fortran2008 or Python. The following prescription should be followed when preparing the report:

\begin{itemize}
  \item Use canvas to hand in your projects, log in  at  \href{{http://canvas.uio.no}}{\nolinkurl{http://canvas.uio.no}} with your normal UiO username and password.

  \item Upload \textbf{only} the report file!  For the source code file(s) you have developed please provide us with your link to your github domain.  The report file should include all of your discussions and a list of the codes you have developed.  The full version of the codes should be in your github repository.

  \item In your github repository, please include a folder which contains selected results. These can be in the form of output from your code for a selected set of runs and input parameters.

  \item Still in your github make a folder where you place your codes. 

  \item In this and all later projects, you should include tests (for example unit tests) of your code(s).

  \item Comments  from us on your projects, approval or not, corrections to be made  etc can be found under your Devilry domain and are only visible to you and the teachers of the course.
\end{itemize}

\noindent
Finally, 
we encourage you to work two and two together. Optimal working groups consist of 
2-3 students. You can then hand in a common report. 


% ------------------- end of main content ---------------

% #ifdef PREAMBLE
\end{document}
% #endif

