%%
%% Automatically generated file from DocOnce source
%% (https://github.com/hplgit/doconce/)
%%
%%


%-------------------- begin preamble ----------------------

\documentclass[%
oneside,                 % oneside: electronic viewing, twoside: printing
final,                   % draft: marks overfull hboxes, figures with paths
10pt]{article}

\listfiles               %  print all files needed to compile this document

\usepackage{relsize,makeidx,color,setspace,amsmath,amsfonts,amssymb}
\usepackage[table]{xcolor}
\usepackage{bm,ltablex,microtype}

\usepackage[pdftex]{graphicx}

\usepackage[T1]{fontenc}
%\usepackage[latin1]{inputenc}
\usepackage{ucs}
\usepackage[utf8x]{inputenc}

\usepackage{lmodern}         % Latin Modern fonts derived from Computer Modern

% Hyperlinks in PDF:
\definecolor{linkcolor}{rgb}{0,0,0.4}
\usepackage{hyperref}
\hypersetup{
    breaklinks=true,
    colorlinks=true,
    linkcolor=linkcolor,
    urlcolor=linkcolor,
    citecolor=black,
    filecolor=black,
    %filecolor=blue,
    pdfmenubar=true,
    pdftoolbar=true,
    bookmarksdepth=3   % Uncomment (and tweak) for PDF bookmarks with more levels than the TOC
    }
%\hyperbaseurl{}   % hyperlinks are relative to this root

\setcounter{tocdepth}{2}  % levels in table of contents

% --- fancyhdr package for fancy headers ---
\usepackage{fancyhdr}
\fancyhf{} % sets both header and footer to nothing
\renewcommand{\headrulewidth}{0pt}
\fancyfoot[LE,RO]{\thepage}
% Ensure copyright on titlepage (article style) and chapter pages (book style)
\fancypagestyle{plain}{
  \fancyhf{}
  \fancyfoot[C]{{\footnotesize \copyright\ 1999-2019, "Computational Physics I FYS4411/FYS9411":"http://www.uio.no/studier/emner/matnat/fys/FYS4411/index-eng.html". Released under CC Attribution-NonCommercial 4.0 license}}
%  \renewcommand{\footrulewidth}{0mm}
  \renewcommand{\headrulewidth}{0mm}
}
% Ensure copyright on titlepages with \thispagestyle{empty}
\fancypagestyle{empty}{
  \fancyhf{}
  \fancyfoot[C]{{\footnotesize \copyright\ 1999-2019, "Computational Physics I FYS4411/FYS9411":"http://www.uio.no/studier/emner/matnat/fys/FYS4411/index-eng.html". Released under CC Attribution-NonCommercial 4.0 license}}
  \renewcommand{\footrulewidth}{0mm}
  \renewcommand{\headrulewidth}{0mm}
}

\pagestyle{fancy}


% prevent orhpans and widows
\clubpenalty = 10000
\widowpenalty = 10000

% --- end of standard preamble for documents ---


% insert custom LaTeX commands...

\raggedbottom
\makeindex
\usepackage[totoc]{idxlayout}   % for index in the toc
\usepackage[nottoc]{tocbibind}  % for references/bibliography in the toc

%-------------------- end preamble ----------------------

\begin{document}

% matching end for #ifdef PREAMBLE

\newcommand{\exercisesection}[1]{\subsection*{#1}}


% ------------------- main content ----------------------



% ----------------- title -------------------------

\thispagestyle{empty}

\begin{center}
{\LARGE\bf
\begin{spacing}{1.25}
Project 1, deadline  March 22 
\end{spacing}
}
\end{center}

% ----------------- author(s) -------------------------

\begin{center}
{\bf \href{{http://www.uio.no/studier/emner/matnat/fys/FYS4411/index-eng.html}}{Computational Physics I FYS4411/FYS9411}}
\end{center}

    \begin{center}
% List of all institutions:
\centerline{{\small Department of Physics, University of Oslo, Norway}}
\end{center}
    
% ----------------- end author(s) -------------------------

% --- begin date ---
\begin{center}
Jan 24, 2019
\end{center}
% --- end date ---

\vspace{1cm}


\subsection*{Introduction}




 The spectacular demonstration of Bose-Einstein condensation (BEC) in gases of
 alkali atoms $^{87}$Rb, $^{23}$Na, $^7$Li confined in magnetic
 traps has led to an explosion of interest in
 confined Bose systems. Of interest is the fraction of condensed atoms, the
 nature of the condensate, the excitations above the condensate, the atomic
 density in the trap as a function of Temperature and the critical temperature of BEC,
 $T_c$. 

 A key feature of the trapped alkali and atomic hydrogen systems is that they are
 dilute. The characteristic dimensions of a typical trap for $^{87}$Rb is
 $a_{h0}=\left( {\hbar}/{m\omega_\perp}\right)^\frac{1}{2}=1-2 \times 10^4$
 \AA\ . The interaction between $^{87}$Rb atoms can be well represented
 by its s-wave scattering length, $a_{Rb}$. This scattering length lies in the
 range $85 < a_{Rb} < 140 a_0$ where $a_0 = 0.5292$ \AA\ is the Bohr radius.
 The definite value $a_{Rb} = 100 a_0$ is usually selected and
 for calculations the definite ratio of atom size to trap size 
 $a_{Rb}/a_{h0} = 4.33 \times 10^{-3}$ 
 is usually chosen. A typical $^{87}$Rb atom
 density in the trap is $n \simeq 10^{12}- 10^{14}$ atoms per cubic cm, giving an
 inter-atom spacing $\ell \simeq 10^4$ \AA. Thus the effective atom size is small
 compared to both the trap size and the inter-atom spacing, the condition
 for diluteness ($na^3_{Rb} \simeq 10^{-6}$ where $n = N/V$ is the number
 density). 

Many theoretical studies of Bose-Einstein condensates (BEC) in gases
of alkali atoms confined in magnetic or optical traps have been
conducted in the framework of the Gross-Pitaevskii (GP) equation.  The
key point for the validity of this description is the dilute condition
of these systems, that is, the average distance between the atoms is
much larger than the range of the inter-atomic interaction. In this
situation the physics is dominated by two-body collisions, well
described in terms of the $s$-wave scattering length $a$.  The crucial
parameter defining the condition for diluteness is the gas parameter
$x(\mathbf{r})= n(\mathbf{r}) a^3$, where $n(\mathbf{r})$ is the local density
of the system. For low values of the average gas parameter $x_{av}\le 10^{-3}$, the mean field Gross-Pitaevskii equation does an excellent
job.  However,
in recent experiments, the local gas parameter may well exceed this
value due to the possibility of tuning the scattering length in the
presence of a so-called Feshbach resonance.



Thus, improved many-body methods like Monte Carlo calculations may be
needed.  The aim of this project is to use the Variational Monte Carlo
(VMC) method and evaluate the ground state energy of a trapped, hard
sphere Bose gas for different numbers of particles with a specific
trial wave function.

 This trial wave function is used to study the sensitivity of
 condensate and non-condensate properties to the hard sphere radius
 and the number of particles.  The trap we will use is a spherical (S)
 or an elliptical (E) harmonic trap in one, two and finally three
 dimensions, with the latter given by

\begin{equation}
 V_{ext}(\mathbf{r}) = 
 \Bigg\{
 \begin{array}{ll}
	 \frac{1}{2}m\omega_{ho}^2r^2 & (S)\\
 \strut
	 \frac{1}{2}m[\omega_{ho}^2(x^2+y^2) + \omega_z^2z^2] & (E)
 \label{trap_eqn}
 \end{array}
 \end{equation}
 where (S) stands for symmetric and

\begin{equation}
     H = \sum_i^N \left(\frac{-\hbar^2}{2m}{\bigtriangledown }_{i}^2 +V_{ext}({\mathbf{r}}_i)\right)  +
	 \sum_{i<j}^{N} V_{int}({\mathbf{r}}_i,{\mathbf{r}}_j),
 \end{equation}
 as the two-body Hamiltonian of the system.  Here $\omega_{ho}^2$
 defines the trap potential strength.  In the case of the elliptical
 trap, $V_{ext}(x,y,z)$, $\omega_{ho}=\omega_{\perp}$ is the trap
 frequency in the perpendicular or $xy$ plane and $\omega_z$ the
 frequency in the $z$ direction.  The mean square vibrational
 amplitude of a single boson at $T=0K$ in the trap (\ref{trap_eqn}) is
 $\langle x^2\rangle=(\hbar/2m\omega_{ho})$ so that $a_{ho} \equiv
 (\hbar/m\omega_{ho})^{\frac{1}{2}}$ defines the characteristic length
 of the trap.  The ratio of the frequencies is denoted
 $\lambda=\omega_z/\omega_{\perp}$ leading to a ratio of the trap
 lengths $(a_{\perp}/a_z)=(\omega_z/\omega_{\perp})^{\frac{1}{2}} =
 \sqrt{\lambda}$. Note that we use the shorthand notation
\begin{align}
    \sum_{i < j}^{N} V_{ij} \equiv \sum_{i = 1}^{N}\sum_{j = i + 1}^{N} V_{ij},
\end{align}
 that is, the notation $i < j$ under the summation sign signifies a double sum
 running over all pairwise interactions once.

 We will represent the inter-boson interaction by a pairwise,
 repulsive potential

\begin{equation}
 V_{int}(|\mathbf{r}_i-\mathbf{r}_j|) =  \Bigg\{
 \begin{array}{ll}
	 \infty & {|\mathbf{r}_i-\mathbf{r}_j|} \leq {a}\\
	 0 & {|\mathbf{r}_i-\mathbf{r}_j|} > {a}
 \end{array}
 \end{equation}
 where $a$ is the so-called hard-core diameter of the bosons.
 Clearly, $V_{int}(|\mathbf{r}_i-\mathbf{r}_j|)$ is zero if the bosons are
 separated by a distance $|\mathbf{r}_i-\mathbf{r}_j|$ greater than $a$ but
 infinite if they attempt to come within a distance $|\mathbf{r}_i-\mathbf{r}_j| \leq a$.

 Our trial wave function for the ground state with $N$ atoms is given by

\begin{equation}
 \Psi_T(\mathbf{r})=\Psi_T(\mathbf{r}_1, \mathbf{r}_2, \dots \mathbf{r}_N,\alpha,\beta)
 =\left[
    \prod_i g(\alpha,\beta,\mathbf{r}_i)
 \right]
 \left[
    \prod_{j<k}f(a,|\mathbf{r}_j-\mathbf{r}_k|)
 \right],
 \label{eq:trialwf}
 \end{equation}
 where $\alpha$ and $\beta$ are variational parameters. The
 single-particle wave function is proportional to the harmonic
 oscillator function for the ground state, i.e.,

\begin{equation}
    g(\alpha,\beta,\mathbf{r}_i)= \exp{[-\alpha(x_i^2+y_i^2+\beta z_i^2)]}.
 \end{equation}
 For spherical traps we have $\beta = 1$ and for non-interacting
 bosons ($a=0$) we have $\alpha = 1/2a_{ho}^2$.  The correlation wave
 function is

\begin{equation}
    f(a,|\mathbf{r}_i-\mathbf{r}_j|)=\Bigg\{
 \begin{array}{ll}
	 0 & {|\mathbf{r}_i-\mathbf{r}_j|} \leq {a}\\
	 (1-\frac{a}{|\mathbf{r}_i-\mathbf{r}_j|}) & {|\mathbf{r}_i-\mathbf{r}_j|} > {a}.
 \end{array}
 \end{equation}


\paragraph{Project 1 a): Local energy.}
Find the analytic expressions for the local energy
\begin{equation}
    E_L(\mathbf{r})=\frac{1}{\Psi_T(\mathbf{r})}H\Psi_T(\mathbf{r}),
    \label{eq:locale}
 \end{equation}
 for the above 
 trial wave function of Eq. (\ref{eq:trialwf}). 
Find first the local energy the case with only the harmonic oscillator potential, that is we set $a=0$.
Use first that $\beta =1$ and find the relevant local energies in one, two and three dimensions for one and
$N$ particles with the same mass. 

 Compute also the analytic expression for the drift force to be used in importance sampling

\begin{equation}
   F = \frac{2\nabla \Psi_T}{\Psi_T}.
 \end{equation}
Find first the equivalent expressions for the just the harmonic oscillator part in one, two and three dimensions
with $\beta=1$. 

Next, we will find the local energy for the full problem in three dimensions.
The tricky part is to find an analytic expressions for the derivative of the trial wave function

\begin{equation*}
   \frac{1}{\Psi_T(\mathbf{r})}\sum_i^{N}\nabla_i^2\Psi_T(\mathbf{r}),
\end{equation*}
with the above 
trial wave function of Eq. (\ref{eq:trialwf}).
We rewrite (and we can use the same general expressions for projects 2 and 3)

\begin{equation*}
\Psi_T(\mathbf{r})=\Psi_T(\mathbf{r}_1, \mathbf{r}_2, \dots \mathbf{r}_N,\alpha,\beta)
=\left[
    \prod_i g(\alpha,\beta,\mathbf{r}_i)
\right]
\left[
    \prod_{j<k}f(a,|\mathbf{r}_j-\mathbf{r}_k|)
\right],
\end{equation*}
as

\begin{equation*}
\Psi_T(\mathbf{r})=\left[
    \prod_i g(\alpha,\beta,\mathbf{r}_i)
\right]
\exp{\left(\sum_{j<k}u(r_{jk})\right)}
\end{equation*}
where we have defined $r_{ij}=|\mathbf{r}_i-\mathbf{r}_j|$
and

\begin{equation*}
   f(r_{ij})= \exp{\left(\sum_{i<j}u(r_{ij})\right)},
\end{equation*}
with $u(r_{ij})=\ln{f(r_{ij})}$.
We have also

\begin{equation*}
    g(\alpha,\beta,\mathbf{r}_i) = \exp{\left[-\alpha(x_i^2+y_i^2+\beta
    z_i^2)\right]}= \phi(\mathbf{r}_i).
\end{equation*}

Show that the first  derivative for particle $k$ is

\begin{align*}
  \nabla_k\Psi_T(\mathbf{r}) &= \nabla_k\phi(\mathbf{r}_k)\left[\prod_{i\ne k}\phi(\mathbf{r}_i)\right]\exp{\left(\sum_{j<m}u(r_{jm})\right)}
  \\
  &\qquad
  +  \left[\prod_i\phi(\mathbf{r}_i)\right]
  \exp{\left(\sum_{j<m}u(r_{jm})\right)}\sum_{l\ne k}\nabla_k u(r_{kl}),
\end{align*}
and find the final expression for our specific trial function.
The expression for the second derivative is (show this)

\begin{align*}
   \frac{1}{\Psi_T(\mathbf{r})}\nabla_k^2\Psi_T(\mathbf{r})
   &= \frac{\nabla_k^2\phi(\mathbf{r}_k)}{\phi(\mathbf{r}_k)}
   + 2\frac{\nabla_k\phi(\mathbf{r}_k)}{\phi(\mathbf{r}_k)}
   \left(\sum_{j\ne k}\frac{(\mathbf{r}_k-\mathbf{r}_j)}{r_{kj}}u'(r_{ij})\right)
   \\
   &\qquad
   + \sum_{i\ne k}\sum_{j \ne k}\frac{(\mathbf{r}_k-\mathbf{r}_i)(\mathbf{r}_k-\mathbf{r}_j)}{r_{ki}r_{kj}}u'(r_{ki})u'(r_{kj})
   \\
   &\qquad
   + \sum_{j\ne k}\left( u''(r_{kj})+\frac{2}{r_{kj}}u'(r_{kj})\right).
\end{align*}
Use this expression to find the final second derivative entering the definition of the local energy. 
You need to get the analytic expression for this expression using the harmonic oscillator wave functions
and the correlation term defined in the project.


\paragraph{Project 1 b): Developing the code.}
Write a Variational Monte Carlo program which uses standard
   Metropolis sampling and compute the ground state energy of a
   spherical harmonic oscillator ($\beta = 1$) with no interaction and
   one dimension.  Use natural units and make an analysis of your
   calculations using both the analytic expression for the local
   energy and a numerical calculation of the kinetic energy using
   numerical derivation.  Compare the CPU time difference.  The only
   variational parameter is $\alpha$. Perform these calculations for
   $N=1$, $N=10$, $100$ and $500$ atoms. Compare your results with the
   exact answer.  Extend then your results to two and three dimensions
   and compare with the analytical results.

\paragraph{Project 1 c): Adding importance sampling.}
We repeat part b), but now we replace the brute force Metropolis algorithm with
importance sampling based on the Fokker-Planck and the Langevin equations. 
Discuss your results and comment on eventual differences between importance sampling and brute force sampling.
Run the calculations for the one, two and three-dimensional systems only and without the repulsive potential. 
Study the dependence of the results as a function of the time step $\delta t$.  
Compare the results with those obtained under b) and comment eventual differences.

\paragraph{Project 1 d): A better statistical analysis.}
In performing the Monte Carlo analysis we will use the 
  blocking and bootstrap techniques to make the statistical analysis of the
  numerical data. Present your results with a proper evaluation of the
  statistical errors. Repeat the calculations from exercise c) and
  include a proper error analysis. Limit yourself to the
  three-dimensional case only.

\paragraph{Project 1 e): The repulsive interaction.}
\begin{itemize}
  \item [e)] We turn now to the elliptic trap with a repulsive
\end{itemize}

\noindent
   interaction.  We fix, as in Refs. [1,2] below,
   $a/a_{ho}=0.0043$. Introduce lengths in units of $a_{ho}$,
   $r\rightarrow r/a_{ho}$ and energy in units of $\hbar\omega_{ho}$.
   Show then that the original Hamiltonian can be rewritten as

\begin{equation*} 
    H=\sum_{i=1}^N\frac{1}{2}\left(-\nabla^2_i+x_i^2+y_i^2+\gamma^2z_i^2\right)+\sum_{i<j}V_{int}(|\mathbf{r}_i-\mathbf{r}_j|).
 \end{equation*}
 What is the expression for $\gamma$?  Choose the initial value for
 $\beta=\gamma = 2.82843$ and set up a VMC program which computes the
 ground state energy using the trial wave function of
 Eq. (\ref{eq:trialwf})  using only $\alpha$ as variational
 parameter.  Use standard Metropolis sampling and vary the parameter
 $\alpha$ in order to find a minimum. Perform the calculations for
 $N=10,50$ and $N=100$ and compare your results to those from the
 ideal case in the previous exercises.  In actual calculations
 employing the Metropolis algorithm, all moves are recast into the
 chosen simulation cell with periodic boundary conditions. To carry
 out consistently the Metropolis moves, it has to be assumed that the
 correlation function has a range shorter than $L/2$. Then, to decide
 if a move of a single particle is accepted or not, only the set of
 particles contained in a sphere of radius $L/2$ centered at the
 referred particle have to be considered.  Benchmark your results with
 those of Refs. [1,2].

\paragraph{Project 1 f): Finding the best parameter.}
Repeat the previous calculations by varying the energy using the
conjugate gradient method or similar methods to obtain the best possible parameter
$\alpha$. 

\paragraph{Project 1 g): Onebody densities.}
With the optimal parameters for the ground state wave function, compute again the onebody density with and without the Jastrow
factor.  How important are the correlations induced by the Jastrow factor?




\section*{Literature}

\begin{enumerate}
 \item J. L. DuBois and H. R. Glyde, H. R., \emph{Bose-Einstein condensation in trapped bosons: A variational Monte Carlo analysis}, Phys. Rev. A \textbf{63}, 023602 (2001).

 \item J. K. Nilsen,  J. Mur-Petit, M. Guilleumas, M. Hjorth-Jensen, and A. Polls, \emph{Vortices in atomic Bose-Einstein condensates in the large-gas-parameter region}, Phys. Rev. A \textbf{71}, 053610 (2005).
\end{enumerate}

\noindent
\subsection*{Introduction to numerical projects}

Here follows a brief recipe and recommendation on how to write a report for each
project.

\begin{itemize}
  \item Give a short description of the nature of the problem and the eventual  numerical methods you have used.

  \item Describe the algorithm you have used and/or developed. Here you may find it convenient to use pseudocoding. In many cases you can describe the algorithm in the program itself.

  \item Include the source code of your program. Comment your program properly.

  \item If possible, try to find analytic solutions, or known limits in order to test your program when developing the code.

  \item Include your results either in figure form or in a table. Remember to        label your results. All tables and figures should have relevant captions        and labels on the axes.

  \item Try to evaluate the reliabilty and numerical stability/precision of your results. If possible, include a qualitative and/or quantitative discussion of the numerical stability, eventual loss of precision etc.

  \item Try to give an interpretation of you results in your answers to  the problems.

  \item Critique: if possible include your comments and reflections about the  exercise, whether you felt you learnt something, ideas for improvements and  other thoughts you've made when solving the exercise. We wish to keep this course at the interactive level and your comments can help us improve it.

  \item Try to establish a practice where you log your work at the  computerlab. You may find such a logbook very handy at later stages in your work, especially when you don't properly remember  what a previous test version  of your program did. Here you could also record  the time spent on solving the exercise, various algorithms you may have tested or other topics which you feel worthy of mentioning.
\end{itemize}

\noindent
\subsection*{Format for electronic delivery of report and programs}

The preferred format for the report is a PDF file. You can also use DOC or postscript formats or as an ipython notebook file.  As programming language we prefer that you choose between C/C++, Fortran2008 or Python. The following prescription should be followed when preparing the report:

\begin{itemize}
  \item Use Devilry to hand in your projects, log in  at  \href{{http://devilry.ifi.uio.no}}{\nolinkurl{http://devilry.ifi.uio.no}} with your normal UiO username and password.

  \item Upload \textbf{only} the report file!  For the source code file(s) you have developed please provide us with your link to your github domain.  The report file should include all of your discussions and a list of the codes you have developed.  The full version of the codes should be in your github repository.

  \item In your github repository, please include a folder which contains selected results. These can be in the form of output from your code for a selected set of runs and input parameters.

  \item Still in your github make a folder where you place your codes. 

  \item In this and all later projects, you should include tests (for example unit tests) of your code(s).

  \item Comments  from us on your projects, approval or not, corrections to be made  etc can be found under your Devilry domain and are only visible to you and the teachers of the course.
\end{itemize}

\noindent
Finally, 
we encourage you to work two and two together. Optimal working groups consist of 
2-3 students. You can then hand in a common report. 





% ------------------- end of main content ---------------

\end{document}

