%%%%%%%%%%%%%%%%%%%% book.tex %%%%%%%%%%%%%%%%%%%%%%%%%%%%%
%
% sample root file for the chapters of your "monograph"
%
% Use this file as a template for your own input.
%
%%%%%%%%%%%%%%%% Springer-Verlag %%%%%%%%%%%%%%%%%%%%%%%%%%


% RECOMMENDED %%%%%%%%%%%%%%%%%%%%%%%%%%%%%%%%%%%%%%%%%%%%%%%%%%%
\documentclass[graybox,envcountchap,sectrefs]{svmono}

% choose options for [] as required from the list
% in the Reference Guide

\usepackage{mathptmx}
\usepackage{helvet}
\usepackage{courier}
%
\usepackage{type1cm}         

\usepackage{makeidx}         % allows index generation
\usepackage{graphicx}        % standard LaTeX graphics tool
                             % when including figure files
\usepackage{multicol}        % used for the two-column index
\usepackage[bottom]{footmisc}% places footnotes at page bottom

% see the list of further useful packages
% in the Reference Guide

\makeindex             % used for the subject index
                       % please use the style svind.ist with
                       % your makeindex program
\usepackage[usenames,dvipsnames,x11names]{xcolor}
\usepackage{tikz}
\usetikzlibrary{arrows,snakes,shapes}

 \usepackage{listings}
 \usepackage{graphicx}
 \usepackage{epic}
 \usepackage{eepic}
 \usepackage{a4wide}
 \usepackage{color}
 \usepackage{amsmath}
 \usepackage{amssymb}
 \usepackage[dvips]{epsfig}
 \usepackage{psfig}
 \usepackage[T1]{fontenc}
 \usepackage{cite} % [2,3,4] --> [2--4]
 \usepackage{shadow}
 \usepackage{hyperref}
 \usepackage{bezier}
 \usepackage{pstricks}
 %\usepackage{refcheck}
 \setcounter{tocdepth}{2}
%\usepackage{gnuplot-lua-tikz}


\usepackage{textcomp,type1ec,pdfpages}
\usepackage{bera}

\definecolor{dkgreen}{rgb}{0,0.6,0}
\definecolor{gray}{rgb}{0.5,0.5,0.5}
\definecolor{mauve}{rgb}{0.58,0,0.82}

 \lstset{language=c++}
 \lstset{alsolanguage=[90]Fortran}
 \lstset{alsolanguage=python}
% \lstset{basicstyle=\small}
 \lstset{backgroundcolor=\color{white}}
 \lstset{frame=single}
 \lstset{stringstyle=\ttfamily}
 \lstset{keywordstyle=\color{red}\bfseries}
 \lstset{commentstyle=\itshape\color{blue}}
 \lstset{showspaces=false}
 \lstset{showstringspaces=false}
 \lstset{showtabs=false}
 \lstset{breaklines}
 

% Default settings for code listings
% \lstnewenvironment{Python}[1]{
\lstset{%frame=tb,
  language=c++,
  alsolanguage=python,
  %aboveskip=3mm,
 % belowskip=3mm,
  showstringspaces=false,
  columns=flexible,
  basicstyle={\footnotesize\ttfamily},
  numbers=none,
  numberstyle=\tiny\color{gray},
  commentstyle=\color{dkgreen},
  stringstyle=\color{mauve},
  frame=single,  
  breaklines=true,
  %%%% FOR PYTHON 
  otherkeywords={\ , \}, \{},
  keywordstyle=\color{blue},
  emph={void, ||, &&, break, class,continue, delete, else,
  for, if, include, return,try,while},
  emphstyle=\color{black}\bfseries,
  emph={[2]True, False, None, self},
  emphstyle=[2]\color{dkgreen},
  emphstyle=[2]\color{red},
  emph={[3]from, import, as},
  emphstyle=[3]\color{blue},
  upquote=true,
  morecomment=[s]{"""}{"""},
  commentstyle=\color{green}\slshape, %%% cambie gray por green
  emph={[4]1, 2, 3, 4, 5, 6, 7, 8, 9, 0},
  emphstyle=[4]\color{blue},
  breakatwhitespace=true,
  tabsize=2
}

\renewcommand{\lstlistlistingname}{Code Listings}
\renewcommand{\lstlistingname}{Code Listing}
\definecolor{gray}{gray}{0.5}
\definecolor{green}{rgb}{0,0.5,0}

\lstnewenvironment{Python}[1]{
\lstset{
language=python,
basicstyle=\footnotesize\setstretch{1},
stringstyle=\color{red},
showstringspaces=false,
alsoletter={1234567890},
otherkeywords={\ , \}, \{},
keywordstyle=\color{blue},
emph={access,and,break,class,continue,def,del,elif ,else,%
except,exec,finally,for,from,global,if,import,in,is,%
lambda,not,or,pass,print,raise,return,try,while},
emphstyle=\color{black}\bfseries,
emph={[2]True, False, None, self},
emphstyle=[2]\color{red},
emph={[3]from, import, as},
emphstyle=[3]\color{blue},
upquote=true,
morecomment=[s]{"""}{"""},
commentstyle=\color{dkgreen}\slshape, % el color era gray pero lo cambie a verde
emph={[4]1, 2, 3, 4, 5, 6, 7, 8, 9, 0},
emphstyle=[4]\color{blue},
framexleftmargin=1mm, framextopmargin=1mm, rulesepcolor=\color{blue},
breakatwhitespace=true,
tabsize=2
}}{}


\lstnewenvironment{C++}[1]{
\lstset{
language=c++,
% basicstyle=\ttfamily\small\setstretch{1},
basicstyle=\footnotesize\setstretch{1},
stringstyle=\color{red},
showstringspaces=false,
alsoletter={1234567890},
otherkeywords={\ , \}, \{},
keywordstyle=\color{blue},
emph={access,and,break,class,continue,def,del,elif ,else,%
except,exec,finally,for,from,global,if,import,in,is,%
lambda,not,or,pass,print,raise,return,try,while},
emphstyle=\color{black}\bfseries,
emph={[2]True, False, None, self},
emphstyle=[2]\color{red},
emph={[3]from, import, as},
emphstyle=[3]\color{blue},
upquote=true,
morecomment=[s]{"""}{"""},
commentstyle=\color{dkgreen}\slshape, % el color era gray pero lo cambie a verde
emph={[4]1, 2, 3, 4, 5, 6, 7, 8, 9, 0},
emphstyle=[4]\color{blue},
% literate=*{:}{{\textcolor{blue}:}}{1}%
% {=}{{\textcolor{blue}=}}{1}%
% {-}{{\textcolor{blue}-}}{1}%
% {+}{{\textcolor{blue}+}}{1}%
% {*}{{\textcolor{blue}*}}{1}%
% {!}{{\textcolor{blue}!}}{1}%
% {(}{{\textcolor{blue}(}}{1}%
% {)}{{\textcolor{blue})}}{1}%
% {[}{{\textcolor{blue}[}}{1}%
% {]}{{\textcolor{blue}]}}{1}%
% {<}{{\textcolor{blue}<}}{1}%
% {>}{{\textcolor{blue}>}}{1},%
framexleftmargin=1mm, framextopmargin=1mm, rulesepcolor=\color{blue},
breakatwhitespace=true,
tabsize=2
}}{}




\usepackage{tikz}
\usetikzlibrary{shapes,arrows}

% Define block styles
\tikzstyle{decision} = [diamond, draw, fill=blue!20,
    text width=3.5em, text badly centered, node distance=2.5cm, inner sep=0pt]
\tikzstyle{block} = [rectangle, draw, fill=blue!20,
    text width=8em, text centered, rounded corners, minimum height=4em]
\tikzstyle{line} = [draw, very thick, color=black!50, -latex']
\tikzstyle{cloud} = [draw, ellipse,fill=red!20, node distance=2.5cm,
    minimum height=2em]

\def\radius{.7mm} 
\tikzstyle{branch}=[fill,shape=circle,minimum size=3pt,inner sep=0pt]


\newcommand{\bfv}[1]{\boldsymbol{#1}} 
\newcommand{\Div}[1]{\nabla \bullet \vbf{#1}}           % define divergence
\newcommand{\Grad}[1]{\boldsymbol{\nabla}{#1}}
 \newcommand{\OP}[1]{{\bf\widehat{#1}}}
 \newcommand{\be}{\begin{equation}}
 \newcommand{\ee}{\end{equation}}
\newcommand{\beN}{\begin{equation*}}
\newcommand{\bea}{\begin{eqnarray}}
\newcommand{\beaN}{\begin{eqnarray*}}
\newcommand{\eeN}{\end{equation*}}
\newcommand{\eea}{\end{eqnarray}}
\newcommand{\eeaN}{\end{eqnarray*}}
\newcommand{\bdm}{\begin{displaymath}}
\newcommand{\edm}{\end{displaymath}}
\newcommand{\bsubeqs}{\begin{subequations}}
\newcommand{\esubeqs}{\end{subequations}}
\newcommand{\Obs}[1]{\langle{\Op{#1}\rangle}}             % define observable
\newcommand{\PsiT}{\bfv{\Psi_T}(\bfv{R})}                       % symbol for trial wave function
%\newcommand{\braket}[2]{\langle{#1}|\Op{#2}|{#1}\rangle}
\newcommand{\Det}[1]{{|\bfv{#1}|}}
\newcommand{\uvec}[1]{\mbox{\boldmath$\hat{#1}$\unboldmath}}
\newcommand{\Op}[1]{{\bf\widehat{#1}}}    
\newcommand{\eqbrace}[4]{\left\{
\begin{array}{ll}
#1 & #2 \\[0.5cm]
#3 & #4
\end{array}\right.}
\newcommand{\eqbraced}[4]{\left\{
\begin{array}{ll}
#1 & #2 \\[0.5cm]
#3 & #4
\end{array}\right\}}
\newcommand{\eqbracetriple}[6]{\left\{
\begin{array}{ll}
#1 & #2 \\
#3 & #4 \\
#5 & #6
\end{array}\right.}
\newcommand{\eqbracedtriple}[6]{\left\{
\begin{array}{ll}
#1 & #2 \\
#3 & #4 \\
#5 & #6
\end{array}\right\}}

\newcommand{\mybox}[3]{\mbox{\makebox[#1][#2]{$#3$}}}
\newcommand{\myframedbox}[3]{\mbox{\framebox[#1][#2]{$#3$}}}

%% Infinitesimal (and double infinitesimal), useful at end of integrals
%\newcommand{\ud}[1]{\mathrm d#1}
\newcommand{\ud}[1]{d#1}
\newcommand{\udd}[1]{d^2\!#1}

%% Operators, algebraic matrices, algebraic vectors

%% Operator (hat, bold or bold symbol, whichever you like best):
\newcommand{\op}[1]{\widehat{#1}}
%\newcommand{\op}[1]{\mathbf{#1}}
%\newcommand{\op}[1]{\boldsymbol{#1}}

%% Vector:
\renewcommand{\vec}[1]{\boldsymbol{#1}}

%% Matrix symbol:
\newcommand{\matr}[1]{\boldsymbol{#1}}
%\newcommand{\bb}[1]{\mathbb{#1}}

%% Determinant symbol:
\renewcommand{\det}[1]{|#1|}

%% Means (expectation values) of varius sizes
\newcommand{\mean}[1]{\langle #1 \rangle}
\newcommand{\meanb}[1]{\big\langle #1 \big\rangle}
\newcommand{\meanbb}[1]{\Big\langle #1 \Big\rangle}
\newcommand{\meanbbb}[1]{\bigg\langle #1 \bigg\rangle}
\newcommand{\meanbbbb}[1]{\Bigg\langle #1 \Bigg\rangle}

%% Shorthands for text set in roman font
\newcommand{\prob}[0]{\mathrm{Prob}} %probability
\newcommand{\cov}[0]{\mathrm{Cov}}   %covariance
\newcommand{\var}[0]{\mathrm{Var}}   %variancd

%% Big-O (typically for specifying the speed scaling of an algorithm)
\newcommand{\bigO}{\mathcal{O}}

%% Real value of a complex number
\newcommand{\real}[1]{\mathrm{Re}\!\left\{#1\right\}}

%% Quantum mechanical state vectors and matrix elements (of different sizes)
\newcommand{\brab}[1]{\big\langle #1 \big|}
\newcommand{\brabb}[1]{\Big\langle #1 \Big|}
\newcommand{\brabbb}[1]{\bigg\langle #1 \bigg|}
\newcommand{\brabbbb}[1]{\Bigg\langle #1 \Bigg|}
\newcommand{\ketb}[1]{\big| #1 \big\rangle}
\newcommand{\ketbb}[1]{\Big| #1 \Big\rangle}
\newcommand{\ketbbb}[1]{\bigg| #1 \bigg\rangle}
\newcommand{\ketbbbb}[1]{\Bigg| #1 \Bigg\rangle}
\newcommand{\overlap}[2]{\langle #1 | #2 \rangle}
\newcommand{\overlapb}[2]{\big\langle #1 \big| #2 \big\rangle}
\newcommand{\overlapbb}[2]{\Big\langle #1 \Big| #2 \Big\rangle}
\newcommand{\overlapbbb}[2]{\bigg\langle #1 \bigg| #2 \bigg\rangle}
\newcommand{\overlapbbbb}[2]{\Bigg\langle #1 \Bigg| #2 \Bigg\rangle}
\newcommand{\bracket}[3]{\langle #1 | #2 | #3 \rangle}
\newcommand{\bracketb}[3]{\big\langle #1 \big| #2 \big| #3 \big\rangle}
\newcommand{\bracketbb}[3]{\Big\langle #1 \Big| #2 \Big| #3 \Big\rangle}
\newcommand{\bracketbbb}[3]{\bigg\langle #1 \bigg| #2 \bigg| #3 \bigg\rangle}
\newcommand{\bracketbbbb}[3]{\Bigg\langle #1 \Bigg| #2 \Bigg| #3 \Bigg\rangle}
\newcommand{\projection}[2]
{| #1 \rangle \langle  #2 |}
\newcommand{\projectionb}[2]
{\big| #1 \big\rangle \big\langle #2 \big|}
\newcommand{\projectionbb}[2]
{ \Big| #1 \Big\rangle \Big\langle #2 \Big|}
\newcommand{\projectionbbb}[2]
{ \bigg| #1 \bigg\rangle \bigg\langle #2 \bigg|}
\newcommand{\projectionbbbb}[2]
{ \Bigg| #1 \Bigg\rangle \Bigg\langle #2 \Bigg|}


%\proton{xposition,yposition}
\newcommand{\proton}[1]{%
    \shade[ball color=red] (#1) circle (.25);\draw (#1) node{$+$};
}

%\neutron{xposition,yposition}
\newcommand{\neutron}[1]{%
    \shade[ball color=green] (#1) circle (.25);
}

%\electron{xwidth,ywidth,rotation angle}
\newcommand{\electron}[3]{%
    \draw[rotate = #3](0,0) ellipse (#1 and #2)[color=blue];
    \shade[ball color=Gold2] (0,#2)[rotate=#3] circle (.1);
}

\newcommand{\nucleus}{%
    \neutron{0.1,0.3}
    \proton{0,0}
    \neutron{0.3,0.2}
    \proton{-0.2,0.1}
    \neutron{-0.1,0.3}
    \proton{0.2,-0.15}
    \neutron{-0.05,-0.12}
    \proton{0.17,0.21}
}

%\photoelectron{xwidth,ywidth,rotation angle}
\newcommand{\photoelectron}[3]{%
    \draw[rotate = #3](0,0) ellipse (#1 and #2)[color=blue];%
    \draw[snake=coil,%
        line after snake=0pt, segment aspect=0,%
        segment length=20pt,color=red!50!blue](#3:#1)-- +(-6,0)%
        node[fill=white!70!Gold2,draw=red!80!white, above=0.2cm,pos=0.5]%
            {Incoming $\gamma$-photon};%
    \draw[-stealth,Gold2](#3:#1) -- ++ (5,0.625);%
    \shade[ball color=Gold2](#3:#1)  --  ++(4,0.5)%
        node[fill=white!70!Gold2,draw=red!80!white,%
        text width=3cm, below right=0.2cm]%
            {Photoelectron from an inner shell} circle(0.1);%
    \fill  (#1,0)[rotate=#3,color=white,opaque] circle (.1);%
    \draw  (#1,0)[rotate=#3,color=Gold2] circle (.1) ;%
}

%\comptonelectron{xwidth,ywidth,rotation angle}
\newcommand{\comptonelectron}[3]{%
    \draw[rotate = #3](0,0) ellipse (#1 and #2)[color=blue];%
    \draw[snake=coil, line after snake=0pt,%
        segment aspect=0, segment length=10pt,color=red!50!blue]%
        (#3:#1)-- +(-6,0)%
        node[fill=white!70!Gold2,draw=red!80!white, above=0.2cm,pos=0.5]%
            {Incoming $\gamma$-photon};%
    \draw[-stealth,Gold2](#3:#1) -- ++ (5,2.5);%
    \shade[ball color=Gold2](#3:#1)  --  ++(4,2.0)%
        node[fill=white!70!Gold2,draw=red!80!white, text width=3cm,%
        below right=0.2cm]{Scattered electron from an outer shell} circle(0.1);%
    \fill  (#1,0)[rotate=#3,color=white,opaque] circle (.1);%
    \draw  (#1,0)[rotate=#3,color=Gold2] circle (.1) ;%
    \draw[snake=coil, line after snake=1mm, segment aspect=0,%
        segment length=15pt,color=red!50!blue,-stealth] (#3:#1)-- ++(6,-3)%
        node[fill=white!70!Gold2,draw=red!80!white, right=1cm,pos=0.5]%
            {Scattered $\gamma$-photon};%
}

%\paircreation{impact parameter}
\newcommand{\paircreation}[1]{%
    \draw[snake=coil, line after snake=0pt, segment aspect=0,%
        segment length=5pt,color=red!50!blue] (0,#1)-- +(-6,0)%
        node[fill=white!70!Gold2,draw=red!80!white, above=0.2cm,pos=0.5]%
            {Incoming $\gamma$-photon};%
    \draw[-stealth,Gold2](0,#1) -- ++ (5,2.5);%
    \shade[ball color=Gold2](0,#1)  --  ++(4,2.0)%
        node[fill=white!70!Gold2,draw=red!80!white, below right=0.2cm]%
            {Positron} circle(0.1);%
    \draw[-stealth,Gold2](0,#1) -- ++ (4,-2.0);%
    \shade[ball color=Gold2](0,#1)  --  ++(3.2,-1.6)%
        node[fill=white!70!Gold2!,draw=red!80!white, above right=0.2cm]%
            {Electron} circle(0.1);%
}





%%%%%%%%%%%%%%%%%%%%%%%%%%%%%%%%%%%%%%%%%%%%%%%%%%%%%%%%%%%%%%%%%%%%%

\begin{document}

\author{Morten Hjorth-Jensen}
\title{Quantum mechanics for many-particle systems}
\subtitle{From standard methods to quantum computing and machine learning}
\maketitle

\frontmatter%%%%%%%%%%%%%%%%%%%%%%%%%%%%%%%%%%%%%%%%%%%%%%%%%%%%%%


%%%%%%%%%%%%%%%%%%%%%%% dedic.tex %%%%%%%%%%%%%%%%%%%%%%%%%%%%%%%%%
%
% sample dedication
%
% Use this file as a template for your own input.
%
%%%%%%%%%%%%%%%%%%%%%%%% Springer %%%%%%%%%%%%%%%%%%%%%%%%%%

\begin{dedication}
Use the template \emph{dedic.tex} together with the Springer document class SVMono for monograph-type books or SVMult for contributed volumes to style a quotation or a dedication\index{dedication} at the very beginning of your book in the Springer layout
\end{dedication}





\include{foreword}
\preface
%  last update : 24/8/2013  mhj

\begin{quotation}
So, ultimately, in order to understand nature it may be necessary to
have a deeper understanding of mathematical relationships. But the
real reason is that the subject is enjoyable, and although we humans
cut nature up in different ways, and we have different courses in
different departments, such compartmentalization is really artificial,
and we should take our intellectual pleasures where we find them. 
{\em Richard Feynman, The Laws of Thermodynamics.}
\end{quotation}

Why a preface you may ask? Isn't that just a mere exposition of a
raison d'$\mathrm{\hat{e}}$tre of an author's choice of material,
preferences, biases, teaching philosophy etc.?  To a large extent I
can answer in the affirmative to that. A preface ought to be personal.
Indeed, what you will see in the various chapters of these notes
represents how I perceive computational physics should be taught.

 This set of lecture notes serves the scope of presenting to you and
train you in an algorithmic approach to problems in the sciences,
represented here by the unity of three disciplines, physics,
mathematics and informatics. This trinity outlines the emerging field
of computational physics.

Our insight in a physical system, combined with numerical mathematics
gives us the rules for setting up an algorithm, viz.~a set of rules
for solving a particular problem.  Our understanding of the physical
system under study is obviously gauged by the natural laws at play,
the initial conditions, boundary conditions and other external
constraints which influence the given system. Having spelled out the
physics, for example in the form of a set of coupled partial
differential equations, we need efficient numerical methods in order
to set up the final algorithm.  This algorithm is in turn coded into a
computer program and executed on available computing facilities.  To
develop such an algorithmic approach, you will be exposed to several
physics cases, spanning from the classical pendulum to quantum
mechanical systems. We will also present some of the most popular
algorithms from numerical mathematics used to solve a plethora of
problems in the sciences.  Finally we will codify these algorithms
using some of the most widely used programming languages, presently C,
C++ and Fortran and its most recent standard Fortran
2008\footnote{Throughout this text we refer to Fortran 2008 as
Fortran, implying the latest standard.}. However, a high-level and fully
object-oriented language like Python is now emerging as a good
alternative although C++ and Fortran still outperform Python when it
comes to computational speed.  In this text we offer an approach where
one can write all programs in C/C++ or Fortran.  We will also show you
how to develop large programs in Python interfacing C++ and/or Fortran
functions for those parts of the program which are CPU intensive.
Such an approach allows you to structure the flow of data in a
high-level language like Python while tasks of a mere repetitive and
CPU intensive nature are left to low-level languages like C++ or
Fortran. Python allows you also to smoothly interface your program
with other software, such as plotting programs or operating system
instructions. A typical Python program you may end up writing contains
everything from compiling and running your codes to preparing the body
of a file for writing up your report.



Computer simulations are nowadays an integral part of contemporary
basic and applied research in the sciences.  Computation is becoming
as important as theory and experiment. In physics, computational
physics, theoretical physics and experimental physics are all equally
important in our daily research and studies of physical
systems. Physics is the unity of theory, experiment and
computation\footnote{We mentioned previously the trinity of physics,
mathematics and informatics. Viewing physics as the trinity of theory,
experiment and simulations is yet another example. It is obviously
tempting to go beyond the sciences. History shows that triunes,
trinities and for example triple deities permeate the Indo-European
cultures (and probably all human cultures), from the ancient Celts and
Hindus to modern days.  The ancient Celts revered many such trinues,
their world was divided into earth, sea and air, nature was divided in
animal, vegetable and mineral and the cardinal colours were red,
yellow and blue, just to mention a few.  As a curious digression, it
was a Gaulish Celt, Hilary, philosopher and bishop of Poitiers (AD
315-367) in his work {\em De Trinitate} who formulated the Holy
Trinity concept of Christianity, perhaps in order to accomodate
millenia of human divination practice.}.  Moreover, the ability "to
compute" forms part of the essential repertoire of research
scientists. Several new fields within computational science have
emerged and strengthened their positions in the last years, such as
computational materials science, bioinformatics, computational
mathematics and mechanics, computational chemistry and physics and so
forth, just to mention a few.  These fields underscore the importance
of simulations as a means to gain novel insights into physical
systems, especially for those cases where no analytical solutions can
be found or an experiment is too complicated or expensive to carry
out.  To be able to simulate large quantal systems with many degrees
of freedom such as strongly interacting electrons in a quantum dot
will be of great importance for future directions in novel fields like
nano-techonology.  This ability often combines knowledge from many
different subjects, in our case essentially from the physical
sciences, numerical mathematics, computing languages, topics from
high-performace computing and some knowledge of computers.


In 1999, when I started this course at the department of physics in
Oslo, computational physics and computational science in general were
still perceived by the majority of physicists and scientists as topics
dealing with just mere tools and number crunching, and not as subjects
of their own.  The computational background of most students enlisting
for the course on computational physics could span from dedicated
hackers and computer freaks to people who basically had never used a
PC. The majority of undergraduate and graduate students had a very
rudimentary knowledge of computational techniques and methods.
Questions like 'do you know of better methods for numerical
integration than the trapezoidal rule' were not uncommon. I do happen
to know of colleagues who applied for time at a supercomputing centre
because they needed to invert matrices of the size of $10^4\times
10^4$ since they were using the trapezoidal rule to compute
integrals. With Gaussian quadrature this dimensionality was easily
reduced to matrix problems of the size of $10^2\times 10^2$, with much
better precision.

More than a decade later most students have now been exposed to a
fairly uniform introduction to computers, basic programming skills and
use of numerical exercises.  Practically every undergraduate student
in physics has now made a Matlab or Maple simulation of for example
the pendulum, with or without chaotic motion.  Nowadays most of you
are familiar, through various undergraduate courses in physics and
mathematics, with interpreted languages such as Maple, Matlab and/or
Mathematica. In addition, the interest in scripting languages such as
Python or Perl has increased considerably in recent years.  The modern
programmer would typically combine several tools, computing
environments and programming languages. A typical example is the
following. Suppose you are working on a project which demands
extensive visualizations of the results. To obtain these results, that
is to solve a physics problems like obtaining the density profile of a
Bose-Einstein condensate, you need however a program which is fairly
fast when computational speed matters.  In this case you would most
likely write a high-performance computing program using Monte Carlo
methods in languages which are tailored for that. These are
represented by programming languages like Fortran and C++.  However,
to visualize the results you would find interpreted languages like
Matlab or scripting languages like Python extremely suitable for your
tasks.  You will therefore end up writing for example a script in
Matlab which calls a Fortran or C++ program where the number crunching
is done and then visualize the results of say a wave equation solver
via Matlab's large library of visualization tools. Alternatively, you
could organize everything into a Python or Perl script which does
everything for you, calls the Fortran and/or C++ programs and performs
the visualization in Matlab or Python. Used correctly, these tools,
spanning from scripting languages to high-performance computing
languages and vizualization programs, speed up your capability to
solve complicated problems.  Being multilingual is thus an advantage
which not only applies to our globalized modern society but to
computing environments as well.  This text shows you how to use C++
and Fortran as programming languages.

There is however more to the picture than meets the eye.  Although
interpreted languages like Matlab, Mathematica and Maple allow you
nowadays to solve very complicated problems, and high-level languages
like Python can be used to solve computational problems, computational
speed and the capability to write an efficient code are topics which
still do matter. To this end, the majority of scientists still use
languages like C++ and Fortran to solve scientific problems.  When you
embark on a master or PhD thesis, you will most likely meet these
high-performance computing languages.  This course emphasizes thus the
use of programming languages like Fortran, Python and C++ instead of
interpreted ones like Matlab or Maple. You should however note that
there are still large differences in computer time between for example
numerical Python and a corresponding C++ program for many numerical
applications in the physical sciences, with a code in C++ or Fortran
being the fastest.

Computational speed is not the only reason for this choice of
programming languages. Another important reason is that we feel that
at a certain stage one needs to have some insights into the algorithm
used, its stability conditions, possible pitfalls like loss of
precision, ranges of applicability, the possibility to improve the
algorithm and taylor it to special purposes etc etc.  One of our major
aims here is to present to you what we would dub 'the algorithmic
approach', a set of rules for doing mathematics or a precise
description of how to solve a problem. To device an algorithm and
thereafter write a code for solving physics problems is a marvelous
way of gaining insight into complicated physical systems. The
algorithm you end up writing reflects in essentially all cases your
own understanding of the physics and the mathematics (the way you
express yourself) of the problem.  We do therefore devote quite some
space to the algorithms behind various functions presented in the
text. Especially, insight into how errors propagate and how to avoid
them is a topic we would like you to pay special attention to. Only
then can you avoid problems like underflow, overflow and loss of
precision. Such a control is not always achievable with interpreted
languages and canned functions where the underlying algorithm and/or
code is not easily accesible.  Although we will at various stages
recommend the use of library routines for say linear
algebra\footnote{Such library functions are often taylored to a given
machine's architecture and should accordingly run faster than user
provided ones.}, our belief is that one should understand what the
given function does, at least to have a mere idea.  With such a
starting point, we strongly believe that it can be easier to develope
more complicated programs on your own using Fortran, C++ or Python.

We have several other aims as well, namely:
\begin{itemize}
\item We would like to give you  an opportunity to gain a 
      deeper understanding of the physics you have learned in other
      courses. In most courses one is normally confronted with simple
      systems which provide exact solutions and mimic to a certain
      extent the realistic cases. Many are however the comments like
      'why can't we do something else than the particle in a box
      potential?'.  In several of the projects we hope to present some
      more 'realistic' cases to solve by various numerical
      methods. This also means that we wish to give examples of how
      physics can be applied in a much broader context than it is
      discussed in the traditional physics undergraduate curriculum.
\item To encourage you to "discover" physics in a way similar to how 
researchers learn in the context of research.
\item Hopefully also to introduce numerical methods and new areas of physics that 
      can be studied with the methods discussed.
\item To teach   structured programming in the context of doing science. 
\item The projects we propose are meant to mimic to a certain extent 
      the situation encountered during a thesis or project work. You
      will tipically have at your disposal 2-3 weeks to solve
      numerically a given project. In so doing you may need to do a
      literature study as well. Finally, we would like you to write a
      report for every project.
\end{itemize}
Our overall goal is to encourage you to learn about science through
experience and by asking questions. Our objective is always
understanding and the purpose of computing is further insight, not
mere numbers!  Simulations can often be considered as
experiments. Rerunning a simulation need not be as costly as rerunning
an experiment.


 
Needless to say, these lecture notes are upgraded continuously, from
typos to new input.  And we do always benefit from your comments,
suggestions and ideas for making these notes better.  It's through the
scientific discourse and critics we advance.  Moreover, I have
benefitted immensely from many discussions with fellow colleagues and
students. In particular I must mention Hans Petter Langtangen, Anders
Malthe-S\o renssen, Knut M\o rken and \O yvind Ryan, whose input
during the last fifteen years has considerably improved these lecture
notes.  Furthermore, the time we have spent and keep spending together
on the Computing in Science Education project at the University, is
just marvelous. Thanks so much. Concerning the Computing in Science
Education initiative, you can read more
at \url{http://www.mn.uio.no/english/about/collaboration/cse/}.


Finally, I would like to add a petit note on referencing. These notes
have evolved over many years and the idea is that they should end up
in the format of a web-based learning environment for doing
computational science. It will be fully free and hopefully represent a
much more efficient way of conveying teaching material than
traditional textbooks.  I have not yet settled on a specific format,
so any input is welcome. At present however, it is very easy for me to
upgrade and improve the material on say a yearly basis, from simple
typos to adding new material.  When accessing the web page of the
course, you will have noticed that you can obtain all source files for
the programs discussed in the text.  Many people have thus written to
me about how they should properly reference this material and whether
they can freely use it. My answer is rather simple.  You are
encouraged to use these codes, modify them, include them in
publications, thesis work, your lectures etc.  As long as your use is
part of the dialectics of science you can use this material freely.
However, since many weekends have elapsed in writing several of these
programs, testing them, sweating over bugs, swearing in front of a
f*@?\%g code which didn't compile properly ten minutes before monday
morning's eight o'clock lecture etc etc, I would dearly appreciate in
case you find these codes of any use, to reference them properly. That
can be done in a simple way, refer to M.~Hjorth-Jensen, {\em
Computational Physics}, University of Oslo (2013). The weblink to the
course should also be included. Hope it is not too much to ask
for. Enjoy!

%%%%%%%%%%%%%%%%%%%%%%acknow.tex%%%%%%%%%%%%%%%%%%%%%%%%%%%%%%%%%%%%%%%%%
% sample acknowledgement chapter
%
% Use this file as a template for your own input.
%
%%%%%%%%%%%%%%%%%%%%%%%% Springer %%%%%%%%%%%%%%%%%%%%%%%%%%

\extrachap{Acknowledgements}

Use the template \emph{acknow.tex} together with the Springer document class SVMono (monograph-type books) or SVMult (edited books) if you prefer to set your acknowledgement section as a separate chapter instead of including it as last part of your preface.



\tableofcontents

%%%%%%%%%%%%%%%%%%%%%%acronym.tex%%%%%%%%%%%%%%%%%%%%%%%%%%%%%%%%%%%%%%%%%
% sample list of acronyms
%
% Use this file as a template for your own input.
%
%%%%%%%%%%%%%%%%%%%%%%%% Springer %%%%%%%%%%%%%%%%%%%%%%%%%%

\extrachap{Acronyms}

Use the template \emph{acronym.tex} together with the Springer document class SVMono (monograph-type books) or SVMult (edited books) to style your list(s) of abbreviations or symbols in the Springer layout.

Lists of abbreviations\index{acronyms, list of}, symbols\index{symbols, list of} and the like are easily formatted with the help of the Springer-enhanced \verb|description| environment.

\begin{description}[CABR]
\item[ABC]{Spelled-out abbreviation and definition}
\item[BABI]{Spelled-out abbreviation and definition}
\item[CABR]{Spelled-out abbreviation and definition}
\end{description}


\mainmatter%%%%%%%%%%%%%%%%%%%%%%%%%%%%%%%%%%%%%%%%%%%%%%%%%%%%%%%
 %  Introductory chapters
 \part{Linear algebra and second quantization}
         
% ------------------- main content ----------------------

\chapter{Many-body Hamiltonians, basic linear algebra and Second Quantization}

\subsection*{Definitions and notations}

Before we proceed we need some definitions.
We will assume that the interacting part of the Hamiltonian
can be approximated by a two-body interaction.
This means that our Hamiltonian is written as the sum of some onebody part and a twobody part
\begin{equation}
    \hat{H} = \hat{H}_0 + \hat{H}_I 
    = \sum_{i=1}^A \hat{h}_0(x_i) + \sum_{i < j}^A \hat{v}(r_{ij}),
\label{Hnuclei}
\end{equation}
with 
\begin{equation}
  H_0=\sum_{i=1}^A \hat{h}_0(x_i).
\label{hinuclei}
\end{equation}
The onebody part $u_{\mathrm{ext}}(x_i)$ is normally approximated by a harmonic oscillator potential or the Coulomb interaction an electron feels from the nucleus. However, other potentials are fully possible, such as 
one derived from the self-consistent solution of the Hartree-Fock equations to be discussed here.

Our Hamiltonian is invariant under the permutation (interchange) of two particles.
Since we deal with fermions however, the total wave function is antisymmetric.
Let $\hat{P}$ be an operator which interchanges two particles.
Due to the symmetries we have ascribed to our Hamiltonian, this operator commutes with the total Hamiltonian,
\[
[\hat{H},\hat{P}] = 0,
 \]
meaning that $\Psi_{\lambda}(x_1, x_2, \dots , x_A)$ is an eigenfunction of 
$\hat{P}$ as well, that is
\[
\hat{P}_{ij}\Psi_{\lambda}(x_1, x_2, \dots,x_i,\dots,x_j,\dots,x_A)=
\beta\Psi_{\lambda}(x_1, x_2, \dots,x_i,\dots,x_j,\dots,x_A),
\]
where $\beta$ is the eigenvalue of $\hat{P}$. We have introduced the suffix $ij$ in order to indicate that we permute particles $i$ and $j$.
The Pauli principle tells us that the total wave function for a system of fermions
has to be antisymmetric, resulting in the eigenvalue $\beta = -1$.   

In our case we assume that  we can approximate the exact eigenfunction with a Slater determinant
\begin{equation}
   \Phi(x_1, x_2,\dots ,x_A,\alpha,\beta,\dots, \sigma)=\frac{1}{\sqrt{A!}}
\left| \begin{array}{ccccc} \psi_{\alpha}(x_1)& \psi_{\alpha}(x_2)& \dots & \dots & \psi_{\alpha}(x_A)\\
                            \psi_{\beta}(x_1)&\psi_{\beta}(x_2)& \dots & \dots & \psi_{\beta}(x_A)\\  
                            \dots & \dots & \dots & \dots & \dots \\
                            \dots & \dots & \dots & \dots & \dots \\
                     \psi_{\sigma}(x_1)&\psi_{\sigma}(x_2)& \dots & \dots & \psi_{\sigma}(x_A)\end{array} \right|, \label{eq:HartreeFockDet}
\end{equation}
where  $x_i$  stand for the coordinates and spin values of a particle $i$ and $\alpha,\beta,\dots, \gamma$ 
are quantum numbers needed to describe remaining quantum numbers.  

\paragraph{Brief reminder on some linear algebra properties.}
Before we proceed with a more compact representation of a Slater determinant, we would like to repeat some linear algebra properties which will be useful for our derivations of the energy as function of a Slater determinant, Hartree-Fock theory and later the nuclear shell model.

The inverse of a matrix is defined by

\[
\mathbf{A}^{-1} \cdot \mathbf{A} = I
\]
A unitary matrix $\mathbf{A}$ is one whose inverse is its adjoint
\[
\mathbf{A}^{-1}=\mathbf{A}^{\dagger}
\]
A real unitary matrix is called orthogonal and its inverse is equal to its transpose.
A hermitian matrix is its own self-adjoint, that  is
\[
\mathbf{A}=\mathbf{A}^{\dagger}. 
\]


\begin{quote}
\begin{tabular}{ccc}
\hline
\multicolumn{1}{c}{ Relations } & \multicolumn{1}{c}{ Name } & \multicolumn{1}{c}{ matrix elements } \\
\hline
$A = A^{T}$                            & symmetric       & $a_{ij} = a_{ji}$                                                       \\
$A = \left (A^{T} \right )^{-1}$       & real orthogonal & $\sum_k a_{ik} a_{jk} = \sum_k a_{ki} a_{kj} = \delta_{ij}$             \\
$A = A^{ * }$                          & real matrix     & $a_{ij} = a_{ij}^{ * }$                                                 \\
$A = A^{\dagger}$                      & hermitian       & $a_{ij} = a_{ji}^{ * }$                                                 \\
$A = \left (A^{\dagger} \right )^{-1}$ & unitary         & $\sum_k a_{ik} a_{jk}^{ * } = \sum_k a_{ki}^{ * } a_{kj} = \delta_{ij}$ \\
\hline
\end{tabular}
\end{quote}

\noindent
Since we will deal with Fermions (identical and indistinguishable particles) we will 
form an ansatz for a given state in terms of so-called Slater determinants determined
by a chosen basis of single-particle functions. 

For a given $n\times n$ matrix $\mathbf{A}$ we can write its determinant
\[
   det(\mathbf{A})=|\mathbf{A}|=
\left| \begin{array}{ccccc} a_{11}& a_{12}& \dots & \dots & a_{1n}\\
                            a_{21}&a_{22}& \dots & \dots & a_{2n}\\  
                            \dots & \dots & \dots & \dots & \dots \\
                            \dots & \dots & \dots & \dots & \dots \\
                            a_{n1}& a_{n2}& \dots & \dots & a_{nn}\end{array} \right|,
\]
in a more compact form as 
\[
|\mathbf{A}|= \sum_{i=1}^{n!}(-1)^{p_i}\hat{P}_i a_{11}a_{22}\dots a_{nn},
\]
where $\hat{P}_i$ is a permutation operator which permutes the column indices $1,2,3,\dots,n$
and the sum runs over all $n!$ permutations.  The quantity $p_i$ represents the number of transpositions of column indices that are needed in order to bring a given permutation back to its initial ordering, in our case given by $a_{11}a_{22}\dots a_{nn}$ here.

A simple $2\times 2$ determinant illustrates this. We have
\[
   det(\mathbf{A})=
\left| \begin{array}{cc} a_{11}& a_{12}\\
                            a_{21}&a_{22}\end{array} \right|= (-1)^0a_{11}a_{22}+(-1)^1a_{12}a_{21},
\]
where in the last term we have interchanged the column indices $1$ and $2$. The natural ordering we have chosen is $a_{11}a_{22}$. 

\paragraph{Back to the derivation of the energy.}
The single-particle function $\psi_{\alpha}(x_i)$  are eigenfunctions of the onebody
Hamiltonian $h_i$, that is
\[
\hat{h}_0(x_i)=\hat{t}(x_i) + \hat{u}_{\mathrm{ext}}(x_i),
\]
with eigenvalues 
\[
\hat{h}_0(x_i) \psi_{\alpha}(x_i)=\left(\hat{t}(x_i) + \hat{u}_{\mathrm{ext}}(x_i)\right)\psi_{\alpha}(x_i)=\varepsilon_{\alpha}\psi_{\alpha}(x_i).
\]
The energies $\varepsilon_{\alpha}$ are the so-called non-interacting single-particle energies, or unperturbed energies. 
The total energy is in this case the sum over all  single-particle energies, if no two-body or more complicated
many-body interactions are present.

Let us denote the ground state energy by $E_0$. According to the
variational principle we have
\[
  E_0 \le E[\Phi] = \int \Phi^*\hat{H}\Phi d\mathbf{\tau}
\]
where $\Phi$ is a trial function which we assume to be normalized
\[
  \int \Phi^*\Phi d\mathbf{\tau} = 1,
\]
where we have used the shorthand $d\mathbf{\tau}=dx_1dr_2\dots dr_A$.

In the Hartree-Fock method the trial function is the Slater
determinant of Eq.~(\ref{eq:HartreeFockDet}) which can be rewritten as 
\[
  \Phi(x_1,x_2,\dots,x_A,\alpha,\beta,\dots,\nu) = \frac{1}{\sqrt{A!}}\sum_{P} (-)^P\hat{P}\psi_{\alpha}(x_1)
    \psi_{\beta}(x_2)\dots\psi_{\nu}(x_A)=\sqrt{A!}\hat{A}\Phi_H,
\]
where we have introduced the antisymmetrization operator $\hat{A}$ defined by the 
summation over all possible permutations of two particles.

It is defined as
\begin{equation}
  \hat{A} = \frac{1}{A!}\sum_{p} (-)^p\hat{P},
\label{antiSymmetryOperator}
\end{equation}
with $p$ standing for the number of permutations. We have introduced for later use the so-called
Hartree-function, defined by the simple product of all possible single-particle functions
\[
  \Phi_H(x_1,x_2,\dots,x_A,\alpha,\beta,\dots,\nu) =
  \psi_{\alpha}(x_1)
    \psi_{\beta}(x_2)\dots\psi_{\nu}(x_A).
\]

Both $\hat{H}_0$ and $\hat{H}_I$ are invariant under all possible permutations of any two particles
and hence commute with $\hat{A}$
\begin{equation}
  [H_0,\hat{A}] = [H_I,\hat{A}] = 0. \label{commutionAntiSym}
\end{equation}
Furthermore, $\hat{A}$ satisfies
\begin{equation}
  \hat{A}^2 = \hat{A},  \label{AntiSymSquared}
\end{equation}
since every permutation of the Slater
determinant reproduces it. 

The expectation value of $\hat{H}_0$ 
\[
  \int \Phi^*\hat{H}_0\Phi d\mathbf{\tau} 
  = A! \int \Phi_H^*\hat{A}\hat{H}_0\hat{A}\Phi_H d\mathbf{\tau}
\]
is readily reduced to
\[
  \int \Phi^*\hat{H}_0\Phi d\mathbf{\tau} 
  = A! \int \Phi_H^*\hat{H}_0\hat{A}\Phi_H d\mathbf{\tau},
\]
where we have used Eqs.~(\ref{commutionAntiSym}) and
(\ref{AntiSymSquared}). The next step is to replace the antisymmetrization
operator by its definition and to
replace $\hat{H}_0$ with the sum of one-body operators
\[
  \int \Phi^*\hat{H}_0\Phi  d\mathbf{\tau}
  = \sum_{i=1}^A \sum_{p} (-)^p\int 
  \Phi_H^*\hat{h}_0\hat{P}\Phi_H d\mathbf{\tau}.
\]

The integral vanishes if two or more particles are permuted in only one
of the Hartree-functions $\Phi_H$ because the individual single-particle wave functions are
orthogonal. We obtain then
\[
  \int \Phi^*\hat{H}_0\Phi  d\mathbf{\tau}= \sum_{i=1}^A \int \Phi_H^*\hat{h}_0\Phi_H  d\mathbf{\tau}.
\]
Orthogonality of the single-particle functions allows us to further simplify the integral, and we
arrive at the following expression for the expectation values of the
sum of one-body Hamiltonians 
\begin{equation}
  \int \Phi^*\hat{H}_0\Phi  d\mathbf{\tau}
  = \sum_{\mu=1}^A \int \psi_{\mu}^*(x)\hat{h}_0\psi_{\mu}(x)dx
  d\mathbf{r}.
  \label{H1Expectation}
\end{equation}

We introduce the following shorthand for the above integral
\[
\langle \mu | \hat{h}_0 | \mu \rangle = \int \psi_{\mu}^*(x)\hat{h}_0\psi_{\mu}(x)dx,
\]
and rewrite Eq.~(\ref{H1Expectation}) as
\begin{equation}
  \int \Phi^*\hat{H}_0\Phi  d\tau
  = \sum_{\mu=1}^A \langle \mu | \hat{h}_0 | \mu \rangle.
  \label{H1Expectation1}
\end{equation}

The expectation value of the two-body part of the Hamiltonian is obtained in a
similar manner. We have
\[
  \int \Phi^*\hat{H}_I\Phi d\mathbf{\tau} 
  = A! \int \Phi_H^*\hat{A}\hat{H}_I\hat{A}\Phi_H d\mathbf{\tau},
\]
which reduces to
\[
 \int \Phi^*\hat{H}_I\Phi d\mathbf{\tau} 
  = \sum_{i\le j=1}^A \sum_{p} (-)^p\int 
  \Phi_H^*\hat{v}(r_{ij})\hat{P}\Phi_H d\mathbf{\tau},
\]
by following the same arguments as for the one-body
Hamiltonian. 

Because of the dependence on the inter-particle distance $r_{ij}$,  permutations of
any two particles no longer vanish, and we get
\[
  \int \Phi^*\hat{H}_I\Phi d\mathbf{\tau} 
  = \sum_{i < j=1}^A \int  
  \Phi_H^*\hat{v}(r_{ij})(1-P_{ij})\Phi_H d\mathbf{\tau}.
\]
where $P_{ij}$ is the permutation operator that interchanges
particle $i$ and particle $j$. Again we use the assumption that the single-particle wave functions
are orthogonal. 

We obtain
\begin{align}
  \int \Phi^*\hat{H}_I\Phi d\mathbf{\tau} 
  = \frac{1}{2}\sum_{\mu=1}^A\sum_{\nu=1}^A
    &\left[ \int \psi_{\mu}^*(x_i)\psi_{\nu}^*(x_j)\hat{v}(r_{ij})\psi_{\mu}(x_i)\psi_{\nu}(x_j)
    dx_idx_j \right.\\
  &\left.
  - \int \psi_{\mu}^*(x_i)\psi_{\nu}^*(x_j)
  \hat{v}(r_{ij})\psi_{\nu}(x_i)\psi_{\mu}(x_j)
  dx_idx_j
  \right]. \label{H2Expectation}
\end{align}
The first term is the so-called direct term. It is frequently also called the  Hartree term, 
while the second is due to the Pauli principle and is called
the exchange term or just the Fock term.
The factor  $1/2$ is introduced because we now run over
all pairs twice. 

The last equation allows us to  introduce some further definitions.  
The single-particle wave functions $\psi_{\mu}(x)$, defined by the quantum numbers $\mu$ and $x$
are defined as the overlap 
\[
   \psi_{\alpha}(x)  = \langle x | \alpha \rangle .
\]

We introduce the following shorthands for the above two integrals
\[
\langle \mu\nu|\hat{v}|\mu\nu\rangle =  \int \psi_{\mu}^*(x_i)\psi_{\nu}^*(x_j)\hat{v}(r_{ij})\psi_{\mu}(x_i)\psi_{\nu}(x_j)
    dx_idx_j,
\]
and
\[
\langle \mu\nu|\hat{v}|\nu\mu\rangle = \int \psi_{\mu}^*(x_i)\psi_{\nu}^*(x_j)
  \hat{v}(r_{ij})\psi_{\nu}(x_i)\psi_{\mu}(x_j)
  dx_idx_j.  
\]

\subsection*{Preparing for later studies: varying the coefficients of a wave function expansion and orthogonal transformations}

It is common to  expand the single-particle functions in a known basis  and vary the coefficients, 
that is, the new single-particle wave function is written as a linear expansion
in terms of a fixed chosen orthogonal basis (for example the well-known harmonic oscillator functions or the hydrogen-like functions etc).
We define our new single-particle basis (this is a normal approach for Hartree-Fock theory) by performing a unitary transformation 
on our previous basis (labelled with greek indices) as
\begin{equation}
\psi_p^{new}  = \sum_{\lambda} C_{p\lambda}\phi_{\lambda}. \label{eq:newbasis}
\end{equation}
In this case we vary the coefficients $C_{p\lambda}$. If the basis has infinitely many solutions, we need
to truncate the above sum.  We assume that the basis $\phi_{\lambda}$ is orthogonal.

It is normal to choose a single-particle basis defined as the eigenfunctions
of parts of the full Hamiltonian. The typical situation consists of the solutions of the one-body part of the Hamiltonian, that is we have
\[
\hat{h}_0\phi_{\lambda}=\epsilon_{\lambda}\phi_{\lambda}.
\]
The single-particle wave functions $\phi_{\lambda}(\mathbf{r})$, defined by the quantum numbers $\lambda$ and $\mathbf{r}$
are defined as the overlap 
\[
   \phi_{\lambda}(\mathbf{r})  = \langle \mathbf{r} | \lambda \rangle .
\]

In deriving the Hartree-Fock equations, we  will expand the single-particle functions in a known basis  and vary the coefficients, 
that is, the new single-particle wave function is written as a linear expansion
in terms of a fixed chosen orthogonal basis (for example the well-known harmonic oscillator functions or the hydrogen-like functions etc).

We stated that a unitary transformation keeps the orthogonality. To see this consider first a basis of vectors $\mathbf{v}_i$,
\[
\mathbf{v}_i = \begin{bmatrix} v_{i1} \\ \dots \\ \dots \\v_{in} \end{bmatrix}
\]
We assume that the basis is orthogonal, that is 
\[
\mathbf{v}_j^T\mathbf{v}_i = \delta_{ij}.
\]
An orthogonal or unitary transformation
\[
\mathbf{w}_i=\mathbf{U}\mathbf{v}_i,
\]
preserves the dot product and orthogonality since
\[
\mathbf{w}_j^T\mathbf{w}_i=(\mathbf{U}\mathbf{v}_j)^T\mathbf{U}\mathbf{v}_i=\mathbf{v}_j^T\mathbf{U}^T\mathbf{U}\mathbf{v}_i= \mathbf{v}_j^T\mathbf{v}_i = \delta_{ij}.
\]

This means that if the coefficients $C_{p\lambda}$ belong to a unitary or orthogonal trasformation (using the Dirac bra-ket notation)
\[
\vert p\rangle  = \sum_{\lambda} C_{p\lambda}\vert\lambda\rangle,
\]
orthogonality is preserved, that is $\langle \alpha \vert \beta\rangle = \delta_{\alpha\beta}$
and $\langle p \vert q\rangle = \delta_{pq}$. 

This propertry is extremely useful when we build up a basis of many-body Stater determinant based states. 

\textbf{Note also that although a basis $\vert \alpha\rangle$ contains an infinity of states, for practical calculations we have always to make some truncations.} 

Before we develop for example the Hartree-Fock equations, there is another very useful property of determinants that we will use both in connection with Hartree-Fock calculations and later shell-model calculations.  

Consider the following determinant
\[
\left| \begin{array}{cc} \alpha_1b_{11}+\alpha_2sb_{12}& a_{12}\\
                         \alpha_1b_{21}+\alpha_2b_{22}&a_{22}\end{array} \right|=\alpha_1\left|\begin{array}{cc} b_{11}& a_{12}\\
                         b_{21}&a_{22}\end{array} \right|+\alpha_2\left| \begin{array}{cc} b_{12}& a_{12}\\b_{22}&a_{22}\end{array} \right|
\]

We can generalize this to  an $n\times n$ matrix and have 
\[
\left| \begin{array}{cccccc} a_{11}& a_{12} & \dots & \sum_{k=1}^n c_k b_{1k} &\dots & a_{1n}\\
a_{21}& a_{22} & \dots & \sum_{k=1}^n c_k b_{2k} &\dots & a_{2n}\\
\dots & \dots & \dots & \dots & \dots & \dots \\
\dots & \dots & \dots & \dots & \dots & \dots \\
a_{n1}& a_{n2} & \dots & \sum_{k=1}^n c_k b_{nk} &\dots & a_{nn}\end{array} \right|=
\sum_{k=1}^n c_k\left| \begin{array}{cccccc} a_{11}& a_{12} & \dots &  b_{1k} &\dots & a_{1n}\\
a_{21}& a_{22} & \dots &  b_{2k} &\dots & a_{2n}\\
\dots & \dots & \dots & \dots & \dots & \dots\\
\dots & \dots & \dots & \dots & \dots & \dots\\
a_{n1}& a_{n2} & \dots &  b_{nk} &\dots & a_{nn}\end{array} \right| .
\]
This is a property we will use in our Hartree-Fock discussions. 

We can generalize the previous results, now 
with all elements $a_{ij}$  being given as functions of 
linear combinations  of various coefficients $c$ and elements $b_{ij}$,
\[
\left| \begin{array}{cccccc} \sum_{k=1}^n b_{1k}c_{k1}& \sum_{k=1}^n b_{1k}c_{k2} & \dots & \sum_{k=1}^n b_{1k}c_{kj}  &\dots & \sum_{k=1}^n b_{1k}c_{kn}\\
\sum_{k=1}^n b_{2k}c_{k1}& \sum_{k=1}^n b_{2k}c_{k2} & \dots & \sum_{k=1}^n b_{2k}c_{kj} &\dots & \sum_{k=1}^n b_{2k}c_{kn}\\
\dots & \dots & \dots & \dots & \dots & \dots \\
\dots & \dots & \dots & \dots & \dots &\dots \\
\sum_{k=1}^n b_{nk}c_{k1}& \sum_{k=1}^n b_{nk}c_{k2} & \dots & \sum_{k=1}^n b_{nk}c_{kj} &\dots & \sum_{k=1}^n b_{nk}c_{kn}\end{array} \right|=det(\mathbf{C})det(\mathbf{B}),
\]
where $det(\mathbf{C})$ and $det(\mathbf{B})$ are the determinants of $n\times n$ matrices
with elements $c_{ij}$ and $b_{ij}$ respectively.  
This is a property we will use in our Hartree-Fock discussions. Convince yourself about the correctness of the above expression by setting $n=2$. 

With our definition of the new basis in terms of an orthogonal basis we have
\[
\psi_p(x)  = \sum_{\lambda} C_{p\lambda}\phi_{\lambda}(x).
\]
If the coefficients $C_{p\lambda}$ belong to an orthogonal or unitary matrix, the new basis
is also orthogonal. 
Our Slater determinant in the new basis $\psi_p(x)$ is written as
\[
\frac{1}{\sqrt{A!}}
\left| \begin{array}{ccccc} \psi_{p}(x_1)& \psi_{p}(x_2)& \dots & \dots & \psi_{p}(x_A)\\
                            \psi_{q}(x_1)&\psi_{q}(x_2)& \dots & \dots & \psi_{q}(x_A)\\  
                            \dots & \dots & \dots & \dots & \dots \\
                            \dots & \dots & \dots & \dots & \dots \\
                     \psi_{t}(x_1)&\psi_{t}(x_2)& \dots & \dots & \psi_{t}(x_A)\end{array} \right|=\frac{1}{\sqrt{A!}}
\left| \begin{array}{ccccc} \sum_{\lambda} C_{p\lambda}\phi_{\lambda}(x_1)& \sum_{\lambda} C_{p\lambda}\phi_{\lambda}(x_2)& \dots & \dots & \sum_{\lambda} C_{p\lambda}\phi_{\lambda}(x_A)\\
                            \sum_{\lambda} C_{q\lambda}\phi_{\lambda}(x_1)&\sum_{\lambda} C_{q\lambda}\phi_{\lambda}(x_2)& \dots & \dots & \sum_{\lambda} C_{q\lambda}\phi_{\lambda}(x_A)\\  
                            \dots & \dots & \dots & \dots & \dots \\
                            \dots & \dots & \dots & \dots & \dots \\
                     \sum_{\lambda} C_{t\lambda}\phi_{\lambda}(x_1)&\sum_{\lambda} C_{t\lambda}\phi_{\lambda}(x_2)& \dots & \dots & \sum_{\lambda} C_{t\lambda}\phi_{\lambda}(x_A)\end{array} \right|,
\]
which is nothing but $det(\mathbf{C})det(\Phi)$, with $det(\Phi)$ being the determinant given by the basis functions $\phi_{\lambda}(x)$. 

In our discussions hereafter we will use our definitions of single-particle states above and below the Fermi ($F$) level given by the labels
$ijkl\dots \le F$ for so-called single-hole states and $abcd\dots > F$ for so-called particle states.
For general single-particle states we employ the labels $pqrs\dots$. 

The energy functional is
\[
  E[\Phi] 
  = \sum_{\mu=1}^A \langle \mu | h | \mu \rangle +
  \frac{1}{2}\sum_{{\mu}=1}^A\sum_{{\nu}=1}^A \langle \mu\nu|\hat{v}|\mu\nu\rangle_{AS},
\]
we found the expression for the energy functional in terms of the basis function $\phi_{\lambda}(\mathbf{r})$. We then  varied the above energy functional with respect to the basis functions $|\mu \rangle$. 
Now we are interested in defining a new basis defined in terms of
a chosen basis as defined in Eq.~(\ref{eq:newbasis}). We can then rewrite the energy functional as
\begin{equation}
  E[\Phi^{New}] 
  = \sum_{i=1}^A \langle i | h | i \rangle +
  \frac{1}{2}\sum_{ij=1}^A\langle ij|\hat{v}|ij\rangle_{AS}, \label{FunctionalEPhi2}
\end{equation}
where $\Phi^{New}$ is the new Slater determinant defined by the new basis of Eq.~(\ref{eq:newbasis}). 

Using Eq.~(\ref{eq:newbasis}) we can rewrite Eq.~(\ref{FunctionalEPhi2}) as 
\begin{equation}
  E[\Psi] 
  = \sum_{i=1}^A \sum_{\alpha\beta} C^*_{i\alpha}C_{i\beta}\langle \alpha | h | \beta \rangle +
  \frac{1}{2}\sum_{ij=1}^A\sum_{{\alpha\beta\gamma\delta}} C^*_{i\alpha}C^*_{j\beta}C_{i\gamma}C_{j\delta}\langle \alpha\beta|\hat{v}|\gamma\delta\rangle_{AS}. \label{FunctionalEPhi3}
\end{equation}

\subsection*{Second quantization}

We introduce the time-independent  operators
$a_\alpha^{\dagger}$ and $a_\alpha$   which create and annihilate, respectively, a particle 
in the single-particle state 
$\varphi_\alpha$. 
We define the fermion creation operator
$a_\alpha^{\dagger}$ 
\begin{equation}
	a_\alpha^{\dagger}|0\rangle \equiv  |\alpha\rangle  \label{eq:2-1a},
\end{equation}
and
\begin{equation}
	a_\alpha^{\dagger}|\alpha_1\dots \alpha_n\rangle_{\mathrm{AS}} \equiv  |\alpha\alpha_1\dots \alpha_n\rangle_{\mathrm{AS}} \label{eq:2-1b}
\end{equation}

In Eq.~(\ref{eq:2-1a}) 
the operator  $a_\alpha^{\dagger}$  acts on the vacuum state 
$|0\rangle$, which does not contain any particles. Alternatively, we could define  a closed-shell nucleus or atom as our new vacuum, but then
we need to introduce the particle-hole  formalism, see the discussion to come. 

In Eq.~(\ref{eq:2-1b}) $a_\alpha^{\dagger}$ acts on an antisymmetric $n$-particle state and 
creates an antisymmetric $(n+1)$-particle state, where the one-body state 
$\varphi_\alpha$ is occupied, under the condition that
$\alpha \ne \alpha_1, \alpha_2, \dots, \alpha_n$. 
It follows that we can express an antisymmetric state as the product of the creation
operators acting on the vacuum state.  
\begin{equation}
	|\alpha_1\dots \alpha_n\rangle_{\mathrm{AS}} = a_{\alpha_1}^{\dagger} a_{\alpha_2}^{\dagger} \dots a_{\alpha_n}^{\dagger} |0\rangle \label{eq:2-2}
\end{equation}

It is easy to derive the commutation and anticommutation rules  for the fermionic creation operators 
$a_\alpha^{\dagger}$. Using the antisymmetry of the states 
(\ref{eq:2-2})
\begin{equation}
	|\alpha_1\dots \alpha_i\dots \alpha_k\dots \alpha_n\rangle_{\mathrm{AS}} = 
		- |\alpha_1\dots \alpha_k\dots \alpha_i\dots \alpha_n\rangle_{\mathrm{AS}} \label{eq:2-3a}
\end{equation}
we obtain
\begin{equation}
	 a_{\alpha_i}^{\dagger}  a_{\alpha_k}^{\dagger} = - a_{\alpha_k}^{\dagger} a_{\alpha_i}^{\dagger} \label{eq:2-3b}
\end{equation}

Using the Pauli principle
\begin{equation}
	|\alpha_1\dots \alpha_i\dots \alpha_i\dots \alpha_n\rangle_{\mathrm{AS}} = 0 \label{eq:2-4a}
\end{equation}
it follows that
\begin{equation}
	a_{\alpha_i}^{\dagger}  a_{\alpha_i}^{\dagger} = 0. \label{eq:2-4b}
\end{equation}
If we combine Eqs.~(\ref{eq:2-3b}) and (\ref{eq:2-4b}), we obtain the well-known anti-commutation rule
\begin{equation}
	a_{\alpha}^{\dagger}  a_{\beta}^{\dagger} + a_{\beta}^{\dagger}  a_{\alpha}^{\dagger} \equiv 
		\{a_{\alpha}^{\dagger},a_{\beta}^{\dagger}\} = 0 \label{eq:2-5}
\end{equation}

The hermitian conjugate  of $a_\alpha^{\dagger}$ is
\begin{equation}
	a_{\alpha} = ( a_{\alpha}^{\dagger} )^{\dagger} \label{eq:2-6}
\end{equation}
If we take the hermitian conjugate of Eq.~(\ref{eq:2-5}), we arrive at 
\begin{equation}
	\{a_{\alpha},a_{\beta}\} = 0 \label{eq:2-7}
\end{equation}

What is the physical interpretation of the operator $a_\alpha$ and what is the effect of 
$a_\alpha$ on a given state $|\alpha_1\alpha_2\dots\alpha_n\rangle_{\mathrm{AS}}$? 
Consider the following matrix element
\begin{equation}
	\langle\alpha_1\alpha_2 \dots \alpha_n|a_\alpha|\alpha_1'\alpha_2' \dots \alpha_m'\rangle \label{eq:2-8}
\end{equation}
where both sides are antisymmetric. We  distinguish between two cases. The first (1) is when
$\alpha \in \{\alpha_i\}$. Using the Pauli principle of Eq.~(\ref{eq:2-4a}) it follows
\begin{equation}
		\langle\alpha_1\alpha_2 \dots \alpha_n|a_\alpha = 0 \label{eq:2-9a}
\end{equation}
The second (2) case is when $\alpha \notin \{\alpha_i\}$. It follows that an hermitian conjugation
\begin{equation}
		\langle \alpha_1\alpha_2 \dots \alpha_n|a_\alpha = \langle\alpha\alpha_1\alpha_2 \dots \alpha_n|  \label{eq:2-9b}
\end{equation}

Eq.~(\ref{eq:2-9b}) holds for case (1) since the lefthand side is zero due to the Pauli principle. We write
Eq.~(\ref{eq:2-8}) as
\begin{equation}
	\langle\alpha_1\alpha_2 \dots \alpha_n|a_\alpha|\alpha_1'\alpha_2' \dots \alpha_m'\rangle = 
	\langle \alpha_1\alpha_2 \dots \alpha_n|\alpha\alpha_1'\alpha_2' \dots \alpha_m'\rangle \label{eq:2-10}
\end{equation}
Here we must have $m = n+1$ if Eq.~(\ref{eq:2-10}) has to be trivially different from zero.

For the last case, the minus and plus signs apply when the sequence 
$\alpha ,\alpha_1, \alpha_2, \dots, \alpha_n$ and 
$\alpha_1', \alpha_2', \dots, \alpha_{n+1}'$ are related to each other via even and odd permutations.
If we assume that  $\alpha \notin \{\alpha_i\}$ we obtain 
\begin{equation}
	\langle\alpha_1\alpha_2 \dots \alpha_n|a_\alpha|\alpha_1'\alpha_2' \dots \alpha_{n+1}'\rangle = 0 \label{eq:2-12}
\end{equation}
when $\alpha \in \{\alpha_i'\}$. If $\alpha \notin \{\alpha_i'\}$, we obtain
\begin{equation}
	a_\alpha\underbrace{|\alpha_1'\alpha_2' \dots \alpha_{n+1}'}\rangle_{\neq \alpha} = 0 \label{eq:2-13a}
\end{equation}
and in particular
\begin{equation}
	a_\alpha |0\rangle = 0 \label{eq:2-13b}
\end{equation}

If $\{\alpha\alpha_i\} = \{\alpha_i'\}$, performing the right permutations, the sequence
$\alpha ,\alpha_1, \alpha_2, \dots, \alpha_n$ is identical with the sequence
$\alpha_1', \alpha_2', \dots, \alpha_{n+1}'$. This results in
\begin{equation}
	\langle\alpha_1\alpha_2 \dots \alpha_n|a_\alpha|\alpha\alpha_1\alpha_2 \dots \alpha_{n}\rangle = 1 \label{eq:2-14}
\end{equation}
and thus
\begin{equation}
	a_\alpha |\alpha\alpha_1\alpha_2 \dots \alpha_{n}\rangle = |\alpha_1\alpha_2 \dots \alpha_{n}\rangle \label{eq:2-15}
\end{equation}

The action of the operator 
$a_\alpha$ from the left on a state vector  is to to remove  one particle in the state
$\alpha$. 
If the state vector does not contain the single-particle state $\alpha$, the outcome of the operation is zero.
The operator  $a_\alpha$ is normally called for a destruction or annihilation operator.

The next step is to establish the  commutator algebra of $a_\alpha^{\dagger}$ and
$a_\beta$. 

The action of the anti-commutator 
$\{a_\alpha^{\dagger}$,$a_\alpha\}$ on a given $n$-particle state is
\begin{align}
	a_\alpha^{\dagger} a_\alpha \underbrace{|\alpha_1\alpha_2 \dots \alpha_{n}\rangle}_{\neq \alpha} &= 0 \nonumber \\
	a_\alpha a_\alpha^{\dagger} \underbrace{|\alpha_1\alpha_2 \dots \alpha_{n}\rangle}_{\neq \alpha} &=
	a_\alpha \underbrace{|\alpha \alpha_1\alpha_2 \dots \alpha_{n}\rangle}_{\neq \alpha} = 
	\underbrace{|\alpha_1\alpha_2 \dots \alpha_{n}\rangle}_{\neq \alpha} \label{eq:2-16a}
\end{align}
if the single-particle state $\alpha$ is not contained in the state.

 If it is present
we arrive at
\begin{align}
	a_\alpha^{\dagger} a_\alpha |\alpha_1\alpha_2 \dots \alpha_{k}\alpha \alpha_{k+1} \dots \alpha_{n-1}\rangle &=
	a_\alpha^{\dagger} a_\alpha (-1)^k |\alpha \alpha_1\alpha_2 \dots \alpha_{n-1}\rangle \nonumber \\
	= (-1)^k |\alpha \alpha_1\alpha_2 \dots \alpha_{n-1}\rangle &=
	|\alpha_1\alpha_2 \dots \alpha_{k}\alpha \alpha_{k+1} \dots \alpha_{n-1}\rangle \nonumber \\
	a_\alpha a_\alpha^{\dagger}|\alpha_1\alpha_2 \dots \alpha_{k}\alpha \alpha_{k+1} \dots \alpha_{n-1}\rangle &= 0 \label{eq:2-16b}
\end{align}
From Eqs.~(\ref{eq:2-16a}) and  (\ref{eq:2-16b}) we arrive at 
\begin{equation}
	\{a_\alpha^{\dagger} , a_\alpha \} = a_\alpha^{\dagger} a_\alpha + a_\alpha a_\alpha^{\dagger} = 1 \label{eq:2-17}
\end{equation}

The action of $\left\{a_\alpha^{\dagger}, a_\beta\right\}$, with 
$\alpha \ne \beta$ on a given state yields three possibilities. 
The first case is a state vector which contains both $\alpha$ and $\beta$, then either 
$\alpha$ or $\beta$ and finally none of them.

The first case results in
\begin{align}
	a_\alpha^{\dagger} a_\beta |\alpha\beta\alpha_1\alpha_2 \dots \alpha_{n-2}\rangle = 0 \nonumber \\
	a_\beta a_\alpha^{\dagger} |\alpha\beta\alpha_1\alpha_2 \dots \alpha_{n-2}\rangle = 0 \label{eq:2-18a}
\end{align}
while the second case gives
\begin{align}
	 a_\alpha^{\dagger} a_\beta |\beta \underbrace{\alpha_1\alpha_2 \dots \alpha_{n-1}}_{\neq \alpha}\rangle =& 
	 	|\alpha \underbrace{\alpha_1\alpha_2 \dots \alpha_{n-1}}_{\neq  \alpha}\rangle \nonumber \\
	a_\beta a_\alpha^{\dagger} |\beta \underbrace{\alpha_1\alpha_2 \dots \alpha_{n-1}}_{\neq \alpha}\rangle =&
		a_\beta |\alpha\beta\underbrace{\beta \alpha_1\alpha_2 \dots \alpha_{n-1}}_{\neq \alpha}\rangle \nonumber \\
	=& - |\alpha\underbrace{\alpha_1\alpha_2 \dots \alpha_{n-1}}_{\neq \alpha}\rangle \label{eq:2-18b}
\end{align}

Finally if the state vector does not contain $\alpha$ and $\beta$
\begin{align}
	a_\alpha^{\dagger} a_\beta |\underbrace{\alpha_1\alpha_2 \dots \alpha_{n}}_{\neq \alpha,\beta}\rangle &=& 0 \nonumber \\
	a_\beta a_\alpha^{\dagger} |\underbrace{\alpha_1\alpha_2 \dots \alpha_{n}}_{\neq \alpha,\beta}\rangle &=& 
		a_\beta |\alpha \underbrace{\alpha_1\alpha_2 \dots \alpha_{n}}_{\neq \alpha,\beta}\rangle = 0 \label{eq:2-18c}
\end{align}
For all three cases we have
\begin{equation}
	\{a_\alpha^{\dagger},a_\beta \} = a_\alpha^{\dagger} a_\beta + a_\beta a_\alpha^{\dagger} = 0, \quad \alpha \neq \beta \label{eq:2-19}
\end{equation}

We can summarize  our findings in Eqs.~(\ref{eq:2-17}) and (\ref{eq:2-19}) as 
\begin{equation}
	\{a_\alpha^{\dagger},a_\beta \} = \delta_{\alpha\beta} \label{eq:2-20}
\end{equation}
with  $\delta_{\alpha\beta}$ is the Kroenecker $\delta$-symbol.

The properties of the creation and annihilation operators can be summarized as (for fermions)
\[
	a_\alpha^{\dagger}|0\rangle \equiv  |\alpha\rangle,
\]
and
\[
	a_\alpha^{\dagger}|\alpha_1\dots \alpha_n\rangle_{\mathrm{AS}} \equiv  |\alpha\alpha_1\dots \alpha_n\rangle_{\mathrm{AS}}. 
\]
from which follows
\[
        |\alpha_1\dots \alpha_n\rangle_{\mathrm{AS}} = a_{\alpha_1}^{\dagger} a_{\alpha_2}^{\dagger} \dots a_{\alpha_n}^{\dagger} |0\rangle.
\]

The hermitian conjugate has the folowing properties
\[
        a_{\alpha} = ( a_{\alpha}^{\dagger} )^{\dagger}.
\]
Finally we found 
\[
	a_\alpha\underbrace{|\alpha_1'\alpha_2' \dots \alpha_{n+1}'}\rangle_{\neq \alpha} = 0, \quad
		\textrm{in particular } a_\alpha |0\rangle = 0,
\]
and
\[
 a_\alpha |\alpha\alpha_1\alpha_2 \dots \alpha_{n}\rangle = |\alpha_1\alpha_2 \dots \alpha_{n}\rangle,
\]
and the corresponding commutator algebra
\[
	\{a_{\alpha}^{\dagger},a_{\beta}^{\dagger}\} = \{a_{\alpha},a_{\beta}\} = 0 \hspace{0.5cm} 
\{a_\alpha^{\dagger},a_\beta \} = \delta_{\alpha\beta}.
\]

\subsection*{One-body operators in second quantization}

A very useful operator is the so-called number-operator.  Most physics cases  we will
study in this text conserve the total number of particles.  The number operator is therefore
a useful quantity which allows us to test that our many-body formalism  conserves the number of particles.
In for example $(d,p)$ or $(p,d)$ reactions it is important to be able to describe quantum mechanical states
where particles get added or removed.
A creation operator $a_\alpha^{\dagger}$ adds one particle to the single-particle state
$\alpha$ of a give many-body state vector, while an annihilation operator $a_\alpha$ 
removes a particle from a single-particle
state $\alpha$. 

Let us consider an operator proportional with $a_\alpha^{\dagger} a_\beta$ and 
$\alpha=\beta$. It acts on an $n$-particle state 
resulting in
\begin{equation}
	a_\alpha^{\dagger} a_\alpha |\alpha_1\alpha_2 \dots \alpha_{n}\rangle = 
	\begin{cases}
		0  &\alpha \notin \{\alpha_i\} \\
		\\
		|\alpha_1\alpha_2 \dots \alpha_{n}\rangle & \alpha \in \{\alpha_i\}
	\end{cases}
\end{equation}
Summing over all possible one-particle states we arrive at
\begin{equation}
	\left( \sum_\alpha a_\alpha^{\dagger} a_\alpha \right) |\alpha_1\alpha_2 \dots \alpha_{n}\rangle = 
	n |\alpha_1\alpha_2 \dots \alpha_{n}\rangle \label{eq:2-21}
\end{equation}

The operator 
\begin{equation}
	\hat{N} = \sum_\alpha a_\alpha^{\dagger} a_\alpha \label{eq:2-22}
\end{equation}
is called the number operator since it counts the number of particles in a give state vector when it acts 
on the different single-particle states.  It acts on one single-particle state at the time and falls 
therefore under category one-body operators.
Next we look at another important one-body operator, namely $\hat{H}_0$ and study its operator form in the 
occupation number representation.

We want to obtain an expression for a one-body operator which conserves the number of particles.
Here we study the one-body operator for the kinetic energy plus an eventual external one-body potential.
The action of this operator on a particular $n$-body state with its pertinent expectation value has already
been studied in coordinate  space.
In coordinate space the operator reads
\begin{equation}
	\hat{H}_0 = \sum_i \hat{h}_0(x_i) \label{eq:2-23}
\end{equation}
and the anti-symmetric $n$-particle Slater determinant is defined as 
\[
\Phi(x_1, x_2,\dots ,x_n,\alpha_1,\alpha_2,\dots, \alpha_n)= \frac{1}{\sqrt{n!}} \sum_p (-1)^p\hat{P}\psi_{\alpha_1}(x_1)\psi_{\alpha_2}(x_2) \dots \psi_{\alpha_n}(x_n).
\]

Defining
\begin{equation}
	\hat{h}_0(x_i) \psi_{\alpha_i}(x_i) = \sum_{\alpha_k'} \psi_{\alpha_k'}(x_i) \langle\alpha_k'|\hat{h}_0|\alpha_k\rangle \label{eq:2-25}
\end{equation}
we can easily  evaluate the action of $\hat{H}_0$ on each product of one-particle functions in Slater determinant.
From Eq.~(\ref{eq:2-25})  we obtain the following result without  permuting any particle pair 
\begin{align}
	&& \left( \sum_i \hat{h}_0(x_i) \right) \psi_{\alpha_1}(x_1)\psi_{\alpha_2}(x_2) \dots \psi_{\alpha_n}(x_n) \nonumber \\
	& =&\sum_{\alpha_1'} \langle \alpha_1'|\hat{h}_0|\alpha_1\rangle 
		\psi_{\alpha_1'}(x_1)\psi_{\alpha_2}(x_2) \dots \psi_{\alpha_n}(x_n) \nonumber \\
	&+&\sum_{\alpha_2'} \langle \alpha_2'|\hat{h}_0|\alpha_2\rangle
		\psi_{\alpha_1}(x_1)\psi_{\alpha_2'}(x_2) \dots \psi_{\alpha_n}(x_n) \nonumber \\
	&+& \dots \nonumber \\
	&+&\sum_{\alpha_n'} \langle \alpha_n'|\hat{h}_0|\alpha_n\rangle
		\psi_{\alpha_1}(x_1)\psi_{\alpha_2}(x_2) \dots \psi_{\alpha_n'}(x_n) \label{eq:2-26}
\end{align}

If we interchange particles $1$ and $2$  we obtain
\begin{align}
	&& \left( \sum_i \hat{h}_0(x_i) \right) \psi_{\alpha_1}(x_2)\psi_{\alpha_1}(x_2) \dots \psi_{\alpha_n}(x_n) \nonumber \\
	& =&\sum_{\alpha_2'} \langle \alpha_2'|\hat{h}_0|\alpha_2\rangle 
		\psi_{\alpha_1}(x_2)\psi_{\alpha_2'}(x_1) \dots \psi_{\alpha_n}(x_n) \nonumber \\
	&+&\sum_{\alpha_1'} \langle \alpha_1'|\hat{h}_0|\alpha_1\rangle
		\psi_{\alpha_1'}(x_2)\psi_{\alpha_2}(x_1) \dots \psi_{\alpha_n}(x_n) \nonumber \\
	&+& \dots \nonumber \\
	&+&\sum_{\alpha_n'} \langle \alpha_n'|\hat{h}_0|\alpha_n\rangle
		\psi_{\alpha_1}(x_2)\psi_{\alpha_1}(x_2) \dots \psi_{\alpha_n'}(x_n) \label{eq:2-27}
\end{align}

We can continue by computing all possible permutations. 
We rewrite also our Slater determinant in its second quantized form and skip the dependence on the quantum numbers $x_i.$
Summing up all contributions and taking care of all phases
$(-1)^p$ we arrive at 
\begin{align}
	\hat{H}_0|\alpha_1,\alpha_2,\dots, \alpha_n\rangle &=& \sum_{\alpha_1'}\langle \alpha_1'|\hat{h}_0|\alpha_1\rangle
		|\alpha_1'\alpha_2 \dots \alpha_{n}\rangle \nonumber \\
	&+& \sum_{\alpha_2'} \langle \alpha_2'|\hat{h}_0|\alpha_2\rangle
		|\alpha_1\alpha_2' \dots \alpha_{n}\rangle \nonumber \\
	&+& \dots \nonumber \\
	&+& \sum_{\alpha_n'} \langle \alpha_n'|\hat{h}_0|\alpha_n\rangle
		|\alpha_1\alpha_2 \dots \alpha_{n}'\rangle \label{eq:2-28}
\end{align}

In Eq.~(\ref{eq:2-28}) 
we have expressed the action of the one-body operator
of Eq.~(\ref{eq:2-23}) on the  $n$-body state in its second quantized form.
This equation can be further manipulated if we use the properties of the creation and annihilation operator
on each primed quantum number, that is
\begin{equation}
	|\alpha_1\alpha_2 \dots \alpha_k' \dots \alpha_{n}\rangle = 
		a_{\alpha_k'}^{\dagger}  a_{\alpha_k} |\alpha_1\alpha_2 \dots \alpha_k \dots \alpha_{n}\rangle \label{eq:2-29}
\end{equation}
Inserting this in the right-hand side of Eq.~(\ref{eq:2-28}) results in
\begin{align}
	\hat{H}_0|\alpha_1\alpha_2 \dots \alpha_{n}\rangle &=& \sum_{\alpha_1'}\langle \alpha_1'|\hat{h}_0|\alpha_1\rangle
		a_{\alpha_1'}^{\dagger}  a_{\alpha_1} |\alpha_1\alpha_2 \dots \alpha_{n}\rangle \nonumber \\
	&+& \sum_{\alpha_2'} \langle \alpha_2'|\hat{h}_0|\alpha_2\rangle
		a_{\alpha_2'}^{\dagger}  a_{\alpha_2} |\alpha_1\alpha_2 \dots \alpha_{n}\rangle \nonumber \\
	&+& \dots \nonumber \\
	&+& \sum_{\alpha_n'} \langle \alpha_n'|\hat{h}_0|\alpha_n\rangle
		a_{\alpha_n'}^{\dagger}  a_{\alpha_n} |\alpha_1\alpha_2 \dots \alpha_{n}\rangle \nonumber \\
	&=& \sum_{\alpha, \beta} \langle \alpha|\hat{h}_0|\beta\rangle a_\alpha^{\dagger} a_\beta 
		|\alpha_1\alpha_2 \dots \alpha_{n}\rangle \label{eq:2-30a}
\end{align}

In the number occupation representation or second quantization we get the following expression for a one-body 
operator which conserves the number of particles
\begin{equation}
	\hat{H}_0 = \sum_{\alpha\beta} \langle \alpha|\hat{h}_0|\beta\rangle a_\alpha^{\dagger} a_\beta \label{eq:2-30b}
\end{equation}
Obviously, $\hat{H}_0$ can be replaced by any other one-body  operator which preserved the number
of particles. The stucture of the operator is therefore not limited to say the kinetic or single-particle energy only.

The opearator $\hat{H}_0$ takes a particle from the single-particle state $\beta$  to the single-particle state $\alpha$ 
with a probability for the transition given by the expectation value $\langle \alpha|\hat{h}_0|\beta\rangle$.

It is instructive to verify Eq.~(\ref{eq:2-30b}) by computing the expectation value of $\hat{H}_0$ 
between two single-particle states
\begin{equation}
	\langle \alpha_1|\hat{h}_0|\alpha_2\rangle = \sum_{\alpha\beta} \langle \alpha|\hat{h}_0|\beta\rangle
		\langle 0|a_{\alpha_1}a_\alpha^{\dagger} a_\beta a_{\alpha_2}^{\dagger}|0\rangle \label{eq:2-30c}
\end{equation}

Using the commutation relations for the creation and annihilation operators we have 
\begin{equation}
a_{\alpha_1}a_\alpha^{\dagger} a_\beta a_{\alpha_2}^{\dagger} = (\delta_{\alpha \alpha_1} - a_\alpha^{\dagger} a_{\alpha_1} )(\delta_{\beta \alpha_2} - a_{\alpha_2}^{\dagger} a_{\beta} ), \label{eq:2-30d}
\end{equation}
which results in
\begin{equation}
\langle 0|a_{\alpha_1}a_\alpha^{\dagger} a_\beta a_{\alpha_2}^{\dagger}|0\rangle = \delta_{\alpha \alpha_1} \delta_{\beta \alpha_2} \label{eq:2-30e}
\end{equation}
and
\begin{equation}
\langle \alpha_1|\hat{h}_0|\alpha_2\rangle = \sum_{\alpha\beta} \langle \alpha|\hat{h}_0|\beta\rangle\delta_{\alpha \alpha_1} \delta_{\beta \alpha_2} = \langle \alpha_1|\hat{h}_0|\alpha_2\rangle \label{eq:2-30f}
\end{equation}

\subsection*{Two-body operators in second quantization}

Let us now derive the expression for our two-body interaction part, which also conserves the number of particles.
We can proceed in exactly the same way as for the one-body operator. In the coordinate representation our
two-body interaction part takes the following expression
\begin{equation}
	\hat{H}_I = \sum_{i < j} V(x_i,x_j) \label{eq:2-31}
\end{equation}
where the summation runs over distinct pairs. The term $V$ can be an interaction model for the nucleon-nucleon interaction
or the interaction between two electrons. It can also include additional two-body interaction terms. 

The action of this operator on a product of 
two single-particle functions is defined as 
\begin{equation}
	V(x_i,x_j) \psi_{\alpha_k}(x_i) \psi_{\alpha_l}(x_j) = \sum_{\alpha_k'\alpha_l'} 
		\psi_{\alpha_k}'(x_i)\psi_{\alpha_l}'(x_j) 
		\langle \alpha_k'\alpha_l'|\hat{v}|\alpha_k\alpha_l\rangle \label{eq:2-32}
\end{equation}

We can now let $\hat{H}_I$ act on all terms in the linear combination for $|\alpha_1\alpha_2\dots\alpha_n\rangle$. Without any permutations we have
\begin{align}
	&& \left( \sum_{i < j} V(x_i,x_j) \right) \psi_{\alpha_1}(x_1)\psi_{\alpha_2}(x_2)\dots \psi_{\alpha_n}(x_n) \nonumber \\
	&=& \sum_{\alpha_1'\alpha_2'} \langle \alpha_1'\alpha_2'|\hat{v}|\alpha_1\alpha_2\rangle
		\psi_{\alpha_1}'(x_1)\psi_{\alpha_2}'(x_2)\dots \psi_{\alpha_n}(x_n) \nonumber \\
	& +& \dots \nonumber \\
	&+& \sum_{\alpha_1'\alpha_n'} \langle \alpha_1'\alpha_n'|\hat{v}|\alpha_1\alpha_n\rangle
		\psi_{\alpha_1}'(x_1)\psi_{\alpha_2}(x_2)\dots \psi_{\alpha_n}'(x_n) \nonumber \\
	& +& \dots \nonumber \\
	&+& \sum_{\alpha_2'\alpha_n'} \langle \alpha_2'\alpha_n'|\hat{v}|\alpha_2\alpha_n\rangle
		\psi_{\alpha_1}(x_1)\psi_{\alpha_2}'(x_2)\dots \psi_{\alpha_n}'(x_n) \nonumber \\
	 & +& \dots \label{eq:2-33}
\end{align}
where on the rhs we have a term for each distinct pairs. 

For the other terms on the rhs we obtain similar expressions  and summing over all terms we obtain
\begin{align}
	H_I |\alpha_1\alpha_2\dots\alpha_n\rangle &=& \sum_{\alpha_1', \alpha_2'} \langle \alpha_1'\alpha_2'|\hat{v}|\alpha_1\alpha_2\rangle
		|\alpha_1'\alpha_2'\dots\alpha_n\rangle \nonumber \\
	&+& \dots \nonumber \\
	&+& \sum_{\alpha_1', \alpha_n'} \langle \alpha_1'\alpha_n'|\hat{v}|\alpha_1\alpha_n\rangle
		|\alpha_1'\alpha_2\dots\alpha_n'\rangle \nonumber \\
	&+& \dots \nonumber \\
	&+& \sum_{\alpha_2', \alpha_n'} \langle \alpha_2'\alpha_n'|\hat{v}|\alpha_2\alpha_n\rangle
		|\alpha_1\alpha_2'\dots\alpha_n'\rangle \nonumber \\
	 &+& \dots \label{eq:2-34}
\end{align}

We introduce second quantization via the relation
\begin{align}
	&& a_{\alpha_k'}^{\dagger} a_{\alpha_l'}^{\dagger} a_{\alpha_l} a_{\alpha_k} 
		|\alpha_1\alpha_2\dots\alpha_k\dots\alpha_l\dots\alpha_n\rangle \nonumber \\
	&=& (-1)^{k-1} (-1)^{l-2} a_{\alpha_k'}^{\dagger} a_{\alpha_l'}^{\dagger} a_{\alpha_l} a_{\alpha_k}
		|\alpha_k\alpha_l \underbrace{\alpha_1\alpha_2\dots\alpha_n}_{\neq \alpha_k,\alpha_l}\rangle \nonumber \\
	&=& (-1)^{k-1} (-1)^{l-2} 
	|\alpha_k'\alpha_l' \underbrace{\alpha_1\alpha_2\dots\alpha_n}_{\neq \alpha_k',\alpha_l'}\rangle \nonumber \\
	&=& |\alpha_1\alpha_2\dots\alpha_k'\dots\alpha_l'\dots\alpha_n\rangle \label{eq:2-35}
\end{align}

Inserting this in (\ref{eq:2-34}) gives
\begin{align}
	H_I |\alpha_1\alpha_2\dots\alpha_n\rangle
	&=& \sum_{\alpha_1', \alpha_2'} \langle \alpha_1'\alpha_2'|\hat{v}|\alpha_1\alpha_2\rangle
		a_{\alpha_1'}^{\dagger} a_{\alpha_2'}^{\dagger} a_{\alpha_2} a_{\alpha_1}
		|\alpha_1\alpha_2\dots\alpha_n\rangle \nonumber \\
	&+& \dots \nonumber \\
	&=& \sum_{\alpha_1', \alpha_n'} \langle \alpha_1'\alpha_n'|\hat{v}|\alpha_1\alpha_n\rangle
		a_{\alpha_1'}^{\dagger} a_{\alpha_n'}^{\dagger} a_{\alpha_n} a_{\alpha_1}
		|\alpha_1\alpha_2\dots\alpha_n\rangle \nonumber \\
	&+& \dots \nonumber \\
	&=& \sum_{\alpha_2', \alpha_n'} \langle \alpha_2'\alpha_n'|\hat{v}|\alpha_2\alpha_n\rangle
		a_{\alpha_2'}^{\dagger} a_{\alpha_n'}^{\dagger} a_{\alpha_n} a_{\alpha_2}
		|\alpha_1\alpha_2\dots\alpha_n\rangle \nonumber \\
	&+& \dots \nonumber \\
	&=& \sum_{\alpha, \beta, \gamma, \delta} ' \langle \alpha\beta|\hat{v}|\gamma\delta\rangle
		a^{\dagger}_\alpha a^{\dagger}_\beta a_\delta a_\gamma
		|\alpha_1\alpha_2\dots\alpha_n\rangle \label{eq:2-36}
\end{align}

Here we let $\sum'$ indicate that the sums running over $\alpha$ and $\beta$ run over all
single-particle states, while the summations  $\gamma$ and $\delta$ 
run over all pairs of single-particle states. We wish to remove this restriction and since
\begin{equation}
	\langle \alpha\beta|\hat{v}|\gamma\delta\rangle = \langle \beta\alpha|\hat{v}|\delta\gamma\rangle \label{eq:2-37}
\end{equation}
we get
\begin{align}
	\sum_{\alpha\beta} \langle \alpha\beta|\hat{v}|\gamma\delta\rangle a^{\dagger}_\alpha a^{\dagger}_\beta a_\delta a_\gamma &=& 
		\sum_{\alpha\beta} \langle \beta\alpha|\hat{v}|\delta\gamma\rangle 
		a^{\dagger}_\alpha a^{\dagger}_\beta a_\delta a_\gamma \label{eq:2-38a} \\
	&=& \sum_{\alpha\beta}\langle \beta\alpha|\hat{v}|\delta\gamma\rangle
		a^{\dagger}_\beta a^{\dagger}_\alpha a_\gamma a_\delta \label{eq:2-38b}
\end{align}
where we  have used the anti-commutation rules.

Changing the summation indices 
$\alpha$ and $\beta$ in (\ref{eq:2-38b}) we obtain
\begin{equation}
	\sum_{\alpha\beta} \langle \alpha\beta|\hat{v}|\gamma\delta\rangle a^{\dagger}_\alpha a^{\dagger}_\beta a_\delta a_\gamma =
		 \sum_{\alpha\beta} \langle \alpha\beta|\hat{v}|\delta\gamma\rangle 
		  a^{\dagger}_\alpha a^{\dagger}_\beta  a_\gamma a_\delta \label{eq:2-38c}
\end{equation}
From this it follows that the restriction on the summation over $\gamma$ and $\delta$ can be removed if we multiply with a factor $\frac{1}{2}$, resulting in 
\begin{equation}
	\hat{H}_I = \frac{1}{2} \sum_{\alpha\beta\gamma\delta} \langle \alpha\beta|\hat{v}|\gamma\delta\rangle
		a^{\dagger}_\alpha a^{\dagger}_\beta a_\delta a_\gamma \label{eq:2-39}
\end{equation}
where we sum freely over all single-particle states $\alpha$, 
$\beta$, $\gamma$ og $\delta$.

With this expression we can now verify that the second quantization form of $\hat{H}_I$ in Eq.~(\ref{eq:2-39}) 
results in the same matrix between two anti-symmetrized two-particle states as its corresponding coordinate
space representation. We have  
\begin{equation}
	\langle \alpha_1 \alpha_2|\hat{H}_I|\beta_1 \beta_2\rangle =
		\frac{1}{2} \sum_{\alpha\beta\gamma\delta}
			\langle \alpha\beta|\hat{v}|\gamma\delta\rangle \langle 0|a_{\alpha_2} a_{\alpha_1} 
			 a^{\dagger}_\alpha a^{\dagger}_\beta a_\delta a_\gamma 
			 a_{\beta_1}^{\dagger} a_{\beta_2}^{\dagger}|0\rangle. \label{eq:2-40}
\end{equation}

Using the commutation relations we get 
\begin{align}
	&& a_{\alpha_2} a_{\alpha_1}a^{\dagger}_\alpha a^{\dagger}_\beta 
		a_\delta a_\gamma a_{\beta_1}^{\dagger} a_{\beta_2}^{\dagger} \nonumber \\
	&=& a_{\alpha_2} a_{\alpha_1}a^{\dagger}_\alpha a^{\dagger}_\beta 
		( a_\delta \delta_{\gamma \beta_1} a_{\beta_2}^{\dagger} - 
		a_\delta  a_{\beta_1}^{\dagger} a_\gamma a_{\beta_2}^{\dagger} ) \nonumber \\
	&=& a_{\alpha_2} a_{\alpha_1}a^{\dagger}_\alpha a^{\dagger}_\beta 
		(\delta_{\gamma \beta_1} \delta_{\delta \beta_2} - \delta_{\gamma \beta_1} a_{\beta_2}^{\dagger} a_\delta -
		a_\delta a_{\beta_1}^{\dagger}\delta_{\gamma \beta_2} +
		a_\delta a_{\beta_1}^{\dagger} a_{\beta_2}^{\dagger} a_\gamma ) \nonumber \\
	&=& a_{\alpha_2} a_{\alpha_1}a^{\dagger}_\alpha a^{\dagger}_\beta 
		(\delta_{\gamma \beta_1} \delta_{\delta \beta_2} - \delta_{\gamma \beta_1} a_{\beta_2}^{\dagger} a_\delta \nonumber \\
		&& \qquad - \delta_{\delta \beta_1} \delta_{\gamma \beta_2} + \delta_{\gamma \beta_2} a_{\beta_1}^{\dagger} a_\delta
		+ a_\delta a_{\beta_1}^{\dagger} a_{\beta_2}^{\dagger} a_\gamma ) \label{eq:2-41}
\end{align}

The vacuum expectation value of this product of operators becomes
\begin{align}
	&& \langle 0|a_{\alpha_2} a_{\alpha_1} a^{\dagger}_\alpha a^{\dagger}_\beta a_\delta a_\gamma 
		a_{\beta_1}^{\dagger} a_{\beta_2}^{\dagger}|0\rangle \nonumber \\
	&=& (\delta_{\gamma \beta_1} \delta_{\delta \beta_2} -
		\delta_{\delta \beta_1} \delta_{\gamma \beta_2} ) 
		\langle 0|a_{\alpha_2} a_{\alpha_1}a^{\dagger}_\alpha a^{\dagger}_\beta|0\rangle \nonumber \\
	&=& (\delta_{\gamma \beta_1} \delta_{\delta \beta_2} -\delta_{\delta \beta_1} \delta_{\gamma \beta_2} )
	(\delta_{\alpha \alpha_1} \delta_{\beta \alpha_2} -\delta_{\beta \alpha_1} \delta_{\alpha \alpha_2} ) \label{eq:2-42b}
\end{align}

Insertion of 
Eq.~(\ref{eq:2-42b}) in Eq.~(\ref{eq:2-40}) results in
\begin{align}
	\langle \alpha_1\alpha_2|\hat{H}_I|\beta_1\beta_2\rangle &=& \frac{1}{2} \big[ 
		\langle \alpha_1\alpha_2|\hat{v}|\beta_1\beta_2\rangle- \langle \alpha_1\alpha_2|\hat{v}|\beta_2\beta_1\rangle \nonumber \\
		&& - \langle \alpha_2\alpha_1|\hat{v}|\beta_1\beta_2\rangle + \langle \alpha_2\alpha_1|\hat{v}|\beta_2\beta_1\rangle \big] \nonumber \\
	&=& \langle \alpha_1\alpha_2|\hat{v}|\beta_1\beta_2\rangle - \langle \alpha_1\alpha_2|\hat{v}|\beta_2\beta_1\rangle \nonumber \\
	&=& \langle \alpha_1\alpha_2|\hat{v}|\beta_1\beta_2\rangle_{\mathrm{AS}}. \label{eq:2-43b}
\end{align}

The two-body operator can also be expressed in terms of the anti-symmetrized matrix elements we discussed previously as
\begin{align}
	\hat{H}_I &=& \frac{1}{2} \sum_{\alpha\beta\gamma\delta}  \langle \alpha \beta|\hat{v}|\gamma \delta\rangle
		a_\alpha^{\dagger} a_\beta^{\dagger} a_\delta a_\gamma \nonumber \\
	&=& \frac{1}{4} \sum_{\alpha\beta\gamma\delta} \left[ \langle \alpha \beta|\hat{v}|\gamma \delta\rangle -
		\langle \alpha \beta|\hat{v}|\delta\gamma \rangle \right] 
		a_\alpha^{\dagger} a_\beta^{\dagger} a_\delta a_\gamma \nonumber \\
	&=& \frac{1}{4} \sum_{\alpha\beta\gamma\delta} \langle \alpha \beta|\hat{v}|\gamma \delta\rangle_{\mathrm{AS}}
		a_\alpha^{\dagger} a_\beta^{\dagger} a_\delta a_\gamma \label{eq:2-45}
\end{align}

The factors in front of the operator, either  $\frac{1}{4}$ or 
$\frac{1}{2}$ tells whether we use antisymmetrized matrix elements or not. 

We can now express the Hamiltonian operator for a many-fermion system  in the occupation basis representation
as  
\begin{equation}
	H = \sum_{\alpha, \beta} \langle \alpha|\hat{t}+\hat{u}_{\mathrm{ext}}|\beta\rangle a_\alpha^{\dagger} a_\beta + \frac{1}{4} \sum_{\alpha\beta\gamma\delta}
		\langle \alpha \beta|\hat{v}|\gamma \delta\rangle a_\alpha^{\dagger} a_\beta^{\dagger} a_\delta a_\gamma. \label{eq:2-46b}
\end{equation}
This is the form we will use in the rest of these lectures, assuming that we work with anti-symmetrized two-body matrix elements.

\subsection*{Particle-hole formalism}

Second quantization is a useful and elegant formalism  for constructing many-body  states and 
quantum mechanical operators. One can express and translate many physical processes
into simple pictures such as Feynman diagrams. Expecation values of many-body states are also easily calculated.
However, although the equations are seemingly easy to set up, from  a practical point of view, that is
the solution of Schroedinger's equation, there is no particular gain.
The many-body equation is equally hard to solve, irrespective of representation. 
The cliche that 
there is no free lunch brings us down to earth again.  
Note however that a transformation to a particular
basis, for cases where the interaction obeys specific symmetries, can ease the solution of Schroedinger's equation. 

But there is at least one important case where second quantization comes to our rescue.
It is namely easy to introduce another reference state than the pure vacuum $|0\rangle $, where all single-particle states are active.
With many particles present it is often useful to introduce another reference state  than the vacuum state$|0\rangle $. We will label this state $|c\rangle$ ($c$ for core) and as we will see it can reduce 
considerably the complexity and thereby the dimensionality of the many-body problem. It allows us to sum up to infinite order specific many-body correlations.  The particle-hole representation is one of these handy representations. 

In the original particle representation these states are products of the creation operators  $a_{\alpha_i}^\dagger$ acting on the true vacuum $|0\rangle $.
Following Eq.~(\ref{eq:2-2}) we have 
\begin{align}
 |\alpha_1\alpha_2\dots\alpha_{n-1}\alpha_n\rangle &=& a_{\alpha_1}^\dagger a_{\alpha_2}^\dagger \dots
					a_{\alpha_{n-1}}^\dagger a_{\alpha_n}^\dagger |0\rangle  \label{eq:2-47a} \\
	|\alpha_1\alpha_2\dots\alpha_{n-1}\alpha_n\alpha_{n+1}\rangle &=&
		a_{\alpha_1}^\dagger a_{\alpha_2}^\dagger \dots a_{\alpha_{n-1}}^\dagger a_{\alpha_n}^\dagger
		a_{\alpha_{n+1}}^\dagger |0\rangle  \label{eq:2-47b} \\
	|\alpha_1\alpha_2\dots\alpha_{n-1}\rangle &=& a_{\alpha_1}^\dagger a_{\alpha_2}^\dagger \dots
		a_{\alpha_{n-1}}^\dagger |0\rangle  \label{eq:2-47c}
\end{align}

If we use Eq.~(\ref{eq:2-47a}) as our new reference state, we can simplify considerably the representation of 
this state
\begin{equation}
	|c\rangle  \equiv |\alpha_1\alpha_2\dots\alpha_{n-1}\alpha_n\rangle =
		a_{\alpha_1}^\dagger a_{\alpha_2}^\dagger \dots a_{\alpha_{n-1}}^\dagger a_{\alpha_n}^\dagger |0\rangle  \label{eq:2-48a}
\end{equation}
The new reference states for the $n+1$ and $n-1$ states can then be written as
\begin{align}
	|\alpha_1\alpha_2\dots\alpha_{n-1}\alpha_n\alpha_{n+1}\rangle &=& (-1)^n a_{\alpha_{n+1}}^\dagger |c\rangle 
		\equiv (-1)^n |\alpha_{n+1}\rangle_c \label{eq:2-48b} \\
	|\alpha_1\alpha_2\dots\alpha_{n-1}\rangle &=& (-1)^{n-1} a_{\alpha_n} |c\rangle  
		\equiv (-1)^{n-1} |\alpha_{n-1}\rangle_c \label{eq:2-48c} 
\end{align}

The first state has one additional particle with respect to the new vacuum state
$|c\rangle $  and is normally referred to as a one-particle state or one particle added to the 
many-body reference state. 
The second state has one particle less than the reference vacuum state  $|c\rangle $ and is referred to as
a one-hole state. 
When dealing with a new reference state it is often convenient to introduce 
new creation and annihilation operators since we have 
from Eq.~(\ref{eq:2-48c})
\begin{equation}
	a_\alpha |c\rangle  \neq 0 \label{eq:2-49}
\end{equation}
since  $\alpha$ is contained  in $|c\rangle $, while for the true vacuum we have 
$a_\alpha |0\rangle  = 0$ for all $\alpha$.

The new reference state leads to the definition of new creation and annihilation operators
which satisfy the following relations
\begin{align}
	b_\alpha |c\rangle  &=& 0 \label{eq:2-50a} \\
	\{b_\alpha^\dagger , b_\beta^\dagger \} = \{b_\alpha , b_\beta \} &=& 0 \nonumber  \\
	\{b_\alpha^\dagger , b_\beta \} &=& \delta_{\alpha \beta} \label{eq:2-50c}
\end{align}
We assume also that the new reference state is properly normalized
\begin{equation}
	\langle c | c \rangle = 1 \label{eq:2-51}
\end{equation}

The physical interpretation of these new operators is that of so-called quasiparticle states.
This means that a state defined by the addition of one extra particle to a reference state $|c\rangle $ may not necesseraly be interpreted as one particle coupled to a core.
We define now new creation operators that act on a state $\alpha$ creating a new quasiparticle state
\begin{equation}
	b_\alpha^\dagger|c\rangle  = \Bigg\{ \begin{array}{ll}
		a_\alpha^\dagger |c\rangle  = |\alpha\rangle, & \alpha > F \\
		\\
		a_\alpha |c\rangle  = |\alpha^{-1}\rangle, & \alpha \leq F
	\end{array} \label{eq:2-52}
\end{equation}
where $F$ is the Fermi level representing the last  occupied single-particle orbit 
of the new reference state $|c\rangle $. 

The annihilation is the hermitian conjugate of the creation operator
\[
	b_\alpha = (b_\alpha^\dagger)^\dagger,
\]
resulting in
\begin{equation}
	b_\alpha^\dagger = \Bigg\{ \begin{array}{ll}
		a_\alpha^\dagger & \alpha > F \\
		\\
		a_\alpha & \alpha \leq F
	\end{array} \qquad 
	b_\alpha = \Bigg\{ \begin{array}{ll}
		a_\alpha & \alpha > F \\
		\\
		 a_\alpha^\dagger & \alpha \leq F
	\end{array} \label{eq:2-54}
\end{equation}

With the new creation and annihilation operator  we can now construct 
many-body quasiparticle states, with one-particle-one-hole states, two-particle-two-hole
states etc in the same fashion as we previously constructed many-particle states. 
We can write a general particle-hole state as
\begin{equation}
	|\beta_1\beta_2\dots \beta_{n_p} \gamma_1^{-1} \gamma_2^{-1} \dots \gamma_{n_h}^{-1}\rangle \equiv
		\underbrace{b_{\beta_1}^\dagger b_{\beta_2}^\dagger \dots b_{\beta_{n_p}}^\dagger}_{>F}
		\underbrace{b_{\gamma_1}^\dagger b_{\gamma_2}^\dagger \dots b_{\gamma_{n_h}}^\dagger}_{\leq F} |c\rangle \label{eq:2-56}
\end{equation}
We can now rewrite our one-body and two-body operators in terms of the new creation and annihilation operators.
The number operator becomes
\begin{equation}
	\hat{N} = \sum_\alpha a_\alpha^\dagger a_\alpha= 
\sum_{\alpha > F} b_\alpha^\dagger b_\alpha + n_c - \sum_{\alpha \leq F} b_\alpha^\dagger b_\alpha \label{eq:2-57b}
\end{equation}
where $n_c$ is the number of particle in the new vacuum state $|c\rangle $.  
The action of $\hat{N}$ on a many-body state results in 
\begin{equation}
	N |\beta_1\beta_2\dots \beta_{n_p} \gamma_1^{-1} \gamma_2^{-1} \dots \gamma_{n_h}^{-1}\rangle = (n_p + n_c - n_h) |\beta_1\beta_2\dots \beta_{n_p} \gamma_1^{-1} \gamma_2^{-1} \dots \gamma_{n_h}^{-1}\rangle \label{2-59}
\end{equation}
Here  $n=n_p +n_c - n_h$ is the total number of particles in the quasi-particle state of 
Eq.~(\ref{eq:2-56}). Note that  $\hat{N}$ counts the total number of particles  present 
\begin{equation}
	N_{qp} = \sum_\alpha b_\alpha^\dagger b_\alpha, \label{eq:2-60}
\end{equation}
gives us the number of quasi-particles as can be seen by computing
\begin{equation}
	N_{qp}= |\beta_1\beta_2\dots \beta_{n_p} \gamma_1^{-1} \gamma_2^{-1} \dots \gamma_{n_h}^{-1}\rangle
		= (n_p + n_h)|\beta_1\beta_2\dots \beta_{n_p} \gamma_1^{-1} \gamma_2^{-1} \dots \gamma_{n_h}^{-1}\rangle \label{eq:2-61}
\end{equation}
where $n_{qp} = n_p + n_h$ is the total number of quasi-particles.

We express the one-body operator $\hat{H}_0$ in terms of the quasi-particle creation and annihilation operators, resulting in
\begin{align}
	\hat{H}_0 &=& \sum_{\alpha\beta > F} \langle \alpha|\hat{h}_0|\beta\rangle  b_\alpha^\dagger b_\beta +
		\sum_{\alpha > F, \beta \leq F } \left[\langle \alpha|\hat{h}_0|\beta\rangle b_\alpha^\dagger b_\beta^\dagger + \langle \beta|\hat{h}_0|\alpha\rangle b_\beta  b_\alpha \right] \nonumber \\
	&+& \sum_{\alpha \leq F} \langle \alpha|\hat{h}_0|\alpha\rangle - \sum_{\alpha\beta \leq F} \langle \beta|\hat{h}_0|\alpha\rangle b_\alpha^\dagger b_\beta \label{eq:2-63b}
\end{align}
The first term  gives contribution only for particle states, while the last one
contributes only for holestates. The second term can create or destroy a set of
quasi-particles and 
the third term is the contribution  from the vacuum state $|c\rangle$.

Before we continue with the expressions for the two-body operator, we introduce a nomenclature we will use for the rest of this
text. It is inspired by the notation used in quantum chemistry.
We reserve the labels $i,j,k,\dots$ for hole states and $a,b,c,\dots$ for states above $F$, viz.~particle states.
This means also that we will skip the constraint $\leq F$ or $> F$ in the summation symbols. 
Our operator $\hat{H}_0$  reads now 
\begin{align}
	\hat{H}_0 &=& \sum_{ab} \langle a|\hat{h}|b\rangle b_a^\dagger b_b +
		\sum_{ai} \left[
		\langle a|\hat{h}|i\rangle b_a^\dagger b_i^\dagger + 
		\langle i|\hat{h}|a\rangle b_i  b_a \right] \nonumber \\
	&+& \sum_{i} \langle i|\hat{h}|i\rangle - 
		\sum_{ij} \langle j|\hat{h}|i\rangle
		b_i^\dagger b_j \label{eq:2-63c}
\end{align} 

The two-particle operator in the particle-hole formalism  is more complicated since we have
to translate four indices $\alpha\beta\gamma\delta$ to the possible combinations of particle and hole
states.  When performing the commutator algebra we can regroup the operator in five different terms
\begin{equation}
	\hat{H}_I = \hat{H}_I^{(a)} + \hat{H}_I^{(b)} + \hat{H}_I^{(c)} + \hat{H}_I^{(d)} + \hat{H}_I^{(e)} \label{eq:2-65}
\end{equation}
Using anti-symmetrized  matrix elements, 
bthe term  $\hat{H}_I^{(a)}$ is  
\begin{equation}
	\hat{H}_I^{(a)} = \frac{1}{4}
	\sum_{abcd} \langle ab|\hat{V}|cd\rangle 
		b_a^\dagger b_b^\dagger b_d b_c \label{eq:2-66}
\end{equation}

The next term $\hat{H}_I^{(b)}$  reads
\begin{equation}
	 \hat{H}_I^{(b)} = \frac{1}{4} \sum_{abci}\left(\langle ab|\hat{V}|ci\rangle b_a^\dagger b_b^\dagger b_i^\dagger b_c +\langle ai|\hat{V}|cb\rangle b_a^\dagger b_i b_b b_c\right) \label{eq:2-67b}
\end{equation}
This term conserves the number of quasiparticles but creates or removes a 
three-particle-one-hole  state. 
For $\hat{H}_I^{(c)}$  we have
\begin{align}
	\hat{H}_I^{(c)}& =& \frac{1}{4}
		\sum_{abij}\left(\langle ab|\hat{V}|ij\rangle b_a^\dagger b_b^\dagger b_j^\dagger b_i^\dagger +
		\langle ij|\hat{V}|ab\rangle b_a  b_b b_j b_i \right)+  \nonumber \\
	&&	\frac{1}{2}\sum_{abij}\langle ai|\hat{V}|bj\rangle b_a^\dagger b_j^\dagger b_b b_i + 
		\frac{1}{2}\sum_{abi}\langle ai|\hat{V}|bi\rangle b_a^\dagger b_b. \label{eq:2-68c}
\end{align}

The first line stands for the creation of a two-particle-two-hole state, while the second line represents
the creation to two one-particle-one-hole pairs
while the last term represents a contribution to the particle single-particle energy
from the hole states, that is an interaction between the particle states and the hole states
within the new vacuum  state.
The fourth term reads
\begin{align}
	 \hat{H}_I^{(d)}& = &\frac{1}{4} 
	 	\sum_{aijk}\left(\langle ai|\hat{V}|jk\rangle b_a^\dagger b_k^\dagger b_j^\dagger b_i+
\langle ji|\hat{V}|ak\rangle b_k^\dagger b_j b_i b_a\right)+\nonumber \\
&&\frac{1}{4}\sum_{aij}\left(\langle ai|\hat{V}|ji\rangle b_a^\dagger b_j^\dagger+
\langle ji|\hat{V}|ai\rangle - \langle ji|\hat{V}|ia\rangle b_j b_a \right). \label{eq:2-69d} 
\end{align}
The terms in the first line  stand for the creation of a particle-hole state 
interacting with hole states, we will label this as a two-hole-one-particle contribution. 
The remaining terms are a particle-hole state interacting with the holes in the vacuum state. 
Finally we have 
\begin{equation}
	\hat{H}_I^{(e)} = \frac{1}{4}
		 \sum_{ijkl}
		 \langle kl|\hat{V}|ij\rangle b_i^\dagger b_j^\dagger b_l b_k+
	        \frac{1}{2}\sum_{ijk}\langle ij|\hat{V}|kj\rangle b_k^\dagger b_i
	        +\frac{1}{2}\sum_{ij}\langle ij|\hat{V}|ij\rangle \label{eq:2-70d}
\end{equation}
The first terms represents the 
interaction between two holes while the second stands for the interaction between a hole and the remaining holes in the vacuum state.
It represents a contribution to single-hole energy  to first order.
The last term collects all contributions to the energy of the ground state of a closed-shell system arising
from hole-hole correlations.

\subsection*{Summarizing and defining a normal-ordered Hamiltonian}

\[
  \Phi_{AS}(\alpha_1, \dots, \alpha_A; x_1, \dots x_A)=
            \frac{1}{\sqrt{A}} \sum_{\hat{P}} (-1)^P \hat{P} \prod_{i=1}^A \psi_{\alpha_i}(x_i),
\]
which is equivalent with $|\alpha_1 \dots \alpha_A\rangle= a_{\alpha_1}^{\dagger} \dots a_{\alpha_A}^{\dagger} |0\rangle$. We have also
    \[
        a_p^\dagger|0\rangle = |p\rangle, \quad a_p |q\rangle = \delta_{pq}|0\rangle
    \]
\[
  \delta_{pq} = \left\{a_p, a_q^\dagger \right\},
\]
and 
\[
0 = \left\{a_p^\dagger, a_q \right\} = \left\{a_p, a_q \right\} = \left\{a_p^\dagger, a_q^\dagger \right\}
\]
\[
|\Phi_0\rangle = |\alpha_1 \dots \alpha_A\rangle, \quad \alpha_1, \dots, \alpha_A \leq \alpha_F
\]

\[
\left\{a_p^\dagger, a_q \right\}= \delta_{pq}, p, q \leq \alpha_F 
\]
\[
\left\{a_p, a_q^\dagger \right\} = \delta_{pq}, p, q > \alpha_F
\]
with         $i,j,\ldots \leq \alpha_F, \quad a,b,\ldots > \alpha_F, \quad p,q, \ldots - \textrm{any}$
\[
        a_i|\Phi_0\rangle = |\Phi_i\rangle, \hspace{0.5cm} a_a^\dagger|\Phi_0\rangle = |\Phi^a\rangle
\]
and         
\[
a_i^\dagger|\Phi_0\rangle = 0 \hspace{0.5cm}  a_a|\Phi_0\rangle = 0
\]

The one-body operator is defined as
\[
 \hat{F} = \sum_{pq} \langle p|\hat{f}|q\rangle a_p^\dagger a_q
\]
while the two-body opreator is defined as
\[
\hat{V} = \frac{1}{4} \sum_{pqrs} \langle pq|\hat{v}|rs\rangle_{AS} a_p^\dagger a_q^\dagger a_s a_r
\]
where we have defined the antisymmetric matrix elements
\[
\langle pq|\hat{v}|rs\rangle_{AS} = \langle pq|\hat{v}|rs\rangle - \langle pq|\hat{v}|sr\rangle.
\]

We can also define a three-body operator
\[
\hat{V}_3 = \frac{1}{36} \sum_{pqrstu} \langle pqr|\hat{v}_3|stu\rangle_{AS} 
                a_p^\dagger a_q^\dagger a_r^\dagger a_u a_t a_s
\]
with the antisymmetrized matrix element
\begin{align}
            \langle pqr|\hat{v}_3|stu\rangle_{AS} = \langle pqr|\hat{v}_3|stu\rangle + \langle pqr|\hat{v}_3|tus\rangle + \langle pqr|\hat{v}_3|ust\rangle- \langle pqr|\hat{v}_3|sut\rangle - \langle pqr|\hat{v}_3|tsu\rangle - \langle pqr|\hat{v}_3|uts\rangle.
\end{align}

\subsection*{Operators in second quantization}

In the build-up of a shell-model or FCI code that is meant to tackle large dimensionalities
is the action of the Hamiltonian $\hat{H}$ on a
Slater determinant represented in second quantization as
\[
 |\alpha_1\dots \alpha_n\rangle = a_{\alpha_1}^{\dagger} a_{\alpha_2}^{\dagger} \dots a_{\alpha_n}^{\dagger} |0\rangle.
\]
The time consuming part stems from the action of the Hamiltonian
on the above determinant,
\[
\left(\sum_{\alpha\beta} \langle \alpha|t+u|\beta\rangle a_\alpha^{\dagger} a_\beta + \frac{1}{4} \sum_{\alpha\beta\gamma\delta}
                \langle \alpha \beta|\hat{v}|\gamma \delta\rangle a_\alpha^{\dagger} a_\beta^{\dagger} a_\delta a_\gamma\right)a_{\alpha_1}^{\dagger} a_{\alpha_2}^{\dagger} \dots a_{\alpha_n}^{\dagger} |0\rangle.
\]
A practically useful way to implement this action is to encode a Slater determinant as a bit pattern.

Assume that we have at our disposal $n$ different single-particle orbits
$\alpha_0,\alpha_2,\dots,\alpha_{n-1}$ and that we can distribute  among these orbits $N\le n$ particles.

A Slater  determinant can then be coded as an integer of $n$ bits. As an example, if we have $n=16$ single-particle states
$\alpha_0,\alpha_1,\dots,\alpha_{15}$ and $N=4$ fermions occupying the states $\alpha_3$, $\alpha_6$, $\alpha_{10}$ and $\alpha_{13}$
we could write this Slater determinant as  
\[
\Phi_{\Lambda} = a_{\alpha_3}^{\dagger} a_{\alpha_6}^{\dagger} a_{\alpha_{10}}^{\dagger} a_{\alpha_{13}}^{\dagger} |0\rangle.
\]
The unoccupied single-particle states have bit value $0$ while the occupied ones are represented by bit state $1$. 
In the binary notation we would write this   16 bits long integer as
\[
\begin{array}{cccccccccccccccc}
{\alpha_0}&{\alpha_1}&{\alpha_2}&{\alpha_3}&{\alpha_4}&{\alpha_5}&{\alpha_6}&{\alpha_7} & {\alpha_8} &{\alpha_9} & {\alpha_{10}} &{\alpha_{11}} &{\alpha_{12}} &{\alpha_{13}} &{\alpha_{14}} & {\alpha_{15}} \\
{0} & {0} &{0} &{1} &{0} &{0} &{1} &{0} &{0} &{0} &{1} &{0} &{0} &{1} &{0} & {0} \\
\end{array}
\]
which translates into the decimal number
\[
2^3+2^6+2^{10}+2^{13}=9288.
\]
We can thus encode a Slater determinant as a bit pattern.

With $N$ particles that can be distributed over $n$ single-particle states, the total number of Slater determinats (and defining thereby the dimensionality of the system) is
\[
\mathrm{dim}(\mathcal{H}) = \left(\begin{array}{c} n \\N\end{array}\right).
\]
The total number of bit patterns is $2^n$. 

We assume again that we have at our disposal $n$ different single-particle orbits
$\alpha_0,\alpha_2,\dots,\alpha_{n-1}$ and that we can distribute  among these orbits $N\le n$ particles.
The ordering among these states is important as it defines the order of the creation operators.
We will write the determinant 
\[
\Phi_{\Lambda} = a_{\alpha_3}^{\dagger} a_{\alpha_6}^{\dagger} a_{\alpha_{10}}^{\dagger} a_{\alpha_{13}}^{\dagger} |0\rangle,
\]
in a more compact way as 
\[
\Phi_{3,6,10,13} = |0001001000100100\rangle.
\]
The action of a creation operator is thus
\[
a^{\dagger}_{\alpha_4}\Phi_{3,6,10,13} = a^{\dagger}_{\alpha_4}|0001001000100100\rangle=a^{\dagger}_{\alpha_4}a_{\alpha_3}^{\dagger} a_{\alpha_6}^{\dagger} a_{\alpha_{10}}^{\dagger} a_{\alpha_{13}}^{\dagger} |0\rangle,
\]
which becomes
\[
-a_{\alpha_3}^{\dagger} a^{\dagger}_{\alpha_4} a_{\alpha_6}^{\dagger} a_{\alpha_{10}}^{\dagger} a_{\alpha_{13}}^{\dagger} |0\rangle=-|0001101000100100\rangle.
\]

Similarly
\[
a^{\dagger}_{\alpha_6}\Phi_{3,6,10,13} = a^{\dagger}_{\alpha_6}|0001001000100100\rangle=a^{\dagger}_{\alpha_6}a_{\alpha_3}^{\dagger} a_{\alpha_6}^{\dagger} a_{\alpha_{10}}^{\dagger} a_{\alpha_{13}}^{\dagger} |0\rangle,
\]
which becomes
\[
-a^{\dagger}_{\alpha_4} (a_{\alpha_6}^{\dagger})^ 2 a_{\alpha_{10}}^{\dagger} a_{\alpha_{13}}^{\dagger} |0\rangle=0!
\]
This gives a simple recipe:  
\begin{itemize}
\item If one of the bits $b_j$ is $1$ and we act with a creation operator on this bit, we return a null vector

\item If $b_j=0$, we set it to $1$ and return a sign factor $(-1)^l$, where $l$ is the number of bits set before bit $j$.
\end{itemize}

\noindent
Consider the action of $a^{\dagger}_{\alpha_2}$ on various slater determinants:
\[
\begin{array}{ccc}
a^{\dagger}_{\alpha_2}\Phi_{00111}& = a^{\dagger}_{\alpha_2}|00111\rangle&=0\times |00111\rangle\\
a^{\dagger}_{\alpha_2}\Phi_{01011}& = a^{\dagger}_{\alpha_2}|01011\rangle&=(-1)\times |01111\rangle\\
a^{\dagger}_{\alpha_2}\Phi_{01101}& = a^{\dagger}_{\alpha_2}|01101\rangle&=0\times |01101\rangle\\
a^{\dagger}_{\alpha_2}\Phi_{01110}& = a^{\dagger}_{\alpha_2}|01110\rangle&=0\times |01110\rangle\\
a^{\dagger}_{\alpha_2}\Phi_{10011}& = a^{\dagger}_{\alpha_2}|10011\rangle&=(-1)\times |10111\rangle\\
a^{\dagger}_{\alpha_2}\Phi_{10101}& = a^{\dagger}_{\alpha_2}|10101\rangle&=0\times |10101\rangle\\
a^{\dagger}_{\alpha_2}\Phi_{10110}& = a^{\dagger}_{\alpha_2}|10110\rangle&=0\times |10110\rangle\\
a^{\dagger}_{\alpha_2}\Phi_{11001}& = a^{\dagger}_{\alpha_2}|11001\rangle&=(+1)\times |11101\rangle\\
a^{\dagger}_{\alpha_2}\Phi_{11010}& = a^{\dagger}_{\alpha_2}|11010\rangle&=(+1)\times |11110\rangle\\
\end{array}
\]
What is the simplest way to obtain the phase when we act with one annihilation(creation) operator
on the given Slater determinant representation?

We have an SD representation
\[
\Phi_{\Lambda} = a_{\alpha_0}^{\dagger} a_{\alpha_3}^{\dagger} a_{\alpha_6}^{\dagger} a_{\alpha_{10}}^{\dagger} a_{\alpha_{13}}^{\dagger} |0\rangle,
\]
in a more compact way as
\[
\Phi_{0,3,6,10,13} = |1001001000100100\rangle.
\]
The action of
\[
a^{\dagger}_{\alpha_4}a_{\alpha_0}\Phi_{0,3,6,10,13} = a^{\dagger}_{\alpha_4}|0001001000100100\rangle=a^{\dagger}_{\alpha_4}a_{\alpha_3}^{\dagger} a_{\alpha_6}^{\dagger} a_{\alpha_{10}}^{\dagger} a_{\alpha_{13}}^{\dagger} |0\rangle,
\]
which becomes
\[
-a_{\alpha_3}^{\dagger} a^{\dagger}_{\alpha_4} a_{\alpha_6}^{\dagger} a_{\alpha_{10}}^{\dagger} a_{\alpha_{13}}^{\dagger} |0\rangle=-|0001101000100100\rangle.
\]

The action
\[
a_{\alpha_0}\Phi_{0,3,6,10,13} = |0001001000100100\rangle,
\]
can be obtained by subtracting the logical sum (AND operation) of $\Phi_{0,3,6,10,13}$ and 
a word which represents only $\alpha_0$, that is
\[
|1000000000000000\rangle,
\] 
from $\Phi_{0,3,6,10,13}= |1001001000100100\rangle$.

This operation gives $|0001001000100100\rangle$. 

Similarly, we can form $a^{\dagger}_{\alpha_4}a_{\alpha_0}\Phi_{0,3,6,10,13}$, say, by adding 
$|0000100000000000\rangle$ to $a_{\alpha_0}\Phi_{0,3,6,10,13}$, first checking that their logical sum
is zero in order to make sure that orbital $\alpha_4$ is not already occupied. 

It is trickier however to get the phase $(-1)^l$. 
One possibility is as follows
\begin{itemize}
\item Let $S_1$ be a word that represents the $1-$bit to be removed and all others set to zero.
\end{itemize}

\noindent
In the previous example $S_1=|1000000000000000\rangle$
\begin{itemize}
\item Define $S_2$ as the similar word that represents the bit to be added, that is in our case
\end{itemize}

\noindent
$S_2=|0000100000000000\rangle$.
\begin{itemize}
\item Compute then $S=S_1-S_2$, which here becomes
\end{itemize}

\noindent
\[
S=|0111000000000000\rangle
\]
\begin{itemize}
\item Perform then the logical AND operation of $S$ with the word containing 
\end{itemize}

\noindent
\[
\Phi_{0,3,6,10,13} = |1001001000100100\rangle,
\]
which results in $|0001000000000000\rangle$. Counting the number of $1-$bits gives the phase.  Here you need however an algorithm for bitcounting. Several efficient ones available. 


 \clearemptydoublepage
         
% ------------------- main content ----------------------

\chapter{Many-body Hamiltonians, basic linear algebra and Second Quantization}

\subsection*{Definitions and notations}

Before we proceed we need some definitions.
We will assume that the interacting part of the Hamiltonian
can be approximated by a two-body interaction.
This means that our Hamiltonian is written as the sum of some onebody part and a twobody part
\begin{equation}
    \hat{H} = \hat{H}_0 + \hat{H}_I 
    = \sum_{i=1}^A \hat{h}_0(x_i) + \sum_{i < j}^A \hat{v}(r_{ij}),
\label{Hnuclei}
\end{equation}
with 
\begin{equation}
  H_0=\sum_{i=1}^A \hat{h}_0(x_i).
\label{hinuclei}
\end{equation}
The onebody part $u_{\mathrm{ext}}(x_i)$ is normally approximated by a harmonic oscillator potential or the Coulomb interaction an electron feels from the nucleus. However, other potentials are fully possible, such as 
one derived from the self-consistent solution of the Hartree-Fock equations to be discussed here.

Our Hamiltonian is invariant under the permutation (interchange) of two particles.
Since we deal with fermions however, the total wave function is antisymmetric.
Let $\hat{P}$ be an operator which interchanges two particles.
Due to the symmetries we have ascribed to our Hamiltonian, this operator commutes with the total Hamiltonian,
\[
[\hat{H},\hat{P}] = 0,
 \]
meaning that $\Psi_{\lambda}(x_1, x_2, \dots , x_A)$ is an eigenfunction of 
$\hat{P}$ as well, that is
\[
\hat{P}_{ij}\Psi_{\lambda}(x_1, x_2, \dots,x_i,\dots,x_j,\dots,x_A)=
\beta\Psi_{\lambda}(x_1, x_2, \dots,x_i,\dots,x_j,\dots,x_A),
\]
where $\beta$ is the eigenvalue of $\hat{P}$. We have introduced the suffix $ij$ in order to indicate that we permute particles $i$ and $j$.
The Pauli principle tells us that the total wave function for a system of fermions
has to be antisymmetric, resulting in the eigenvalue $\beta = -1$.   

In our case we assume that  we can approximate the exact eigenfunction with a Slater determinant
\begin{equation}
   \Phi(x_1, x_2,\dots ,x_A,\alpha,\beta,\dots, \sigma)=\frac{1}{\sqrt{A!}}
\left| \begin{array}{ccccc} \psi_{\alpha}(x_1)& \psi_{\alpha}(x_2)& \dots & \dots & \psi_{\alpha}(x_A)\\
                            \psi_{\beta}(x_1)&\psi_{\beta}(x_2)& \dots & \dots & \psi_{\beta}(x_A)\\  
                            \dots & \dots & \dots & \dots & \dots \\
                            \dots & \dots & \dots & \dots & \dots \\
                     \psi_{\sigma}(x_1)&\psi_{\sigma}(x_2)& \dots & \dots & \psi_{\sigma}(x_A)\end{array} \right|, \label{eq:HartreeFockDet}
\end{equation}
where  $x_i$  stand for the coordinates and spin values of a particle $i$ and $\alpha,\beta,\dots, \gamma$ 
are quantum numbers needed to describe remaining quantum numbers.  

\paragraph{Brief reminder on some linear algebra properties.}
Before we proceed with a more compact representation of a Slater determinant, we would like to repeat some linear algebra properties which will be useful for our derivations of the energy as function of a Slater determinant, Hartree-Fock theory and later the nuclear shell model.

The inverse of a matrix is defined by

\[
\mathbf{A}^{-1} \cdot \mathbf{A} = I
\]
A unitary matrix $\mathbf{A}$ is one whose inverse is its adjoint
\[
\mathbf{A}^{-1}=\mathbf{A}^{\dagger}
\]
A real unitary matrix is called orthogonal and its inverse is equal to its transpose.
A hermitian matrix is its own self-adjoint, that  is
\[
\mathbf{A}=\mathbf{A}^{\dagger}. 
\]


\begin{quote}
\begin{tabular}{ccc}
\hline
\multicolumn{1}{c}{ Relations } & \multicolumn{1}{c}{ Name } & \multicolumn{1}{c}{ matrix elements } \\
\hline
$A = A^{T}$                            & symmetric       & $a_{ij} = a_{ji}$                                                       \\
$A = \left (A^{T} \right )^{-1}$       & real orthogonal & $\sum_k a_{ik} a_{jk} = \sum_k a_{ki} a_{kj} = \delta_{ij}$             \\
$A = A^{ * }$                          & real matrix     & $a_{ij} = a_{ij}^{ * }$                                                 \\
$A = A^{\dagger}$                      & hermitian       & $a_{ij} = a_{ji}^{ * }$                                                 \\
$A = \left (A^{\dagger} \right )^{-1}$ & unitary         & $\sum_k a_{ik} a_{jk}^{ * } = \sum_k a_{ki}^{ * } a_{kj} = \delta_{ij}$ \\
\hline
\end{tabular}
\end{quote}

\noindent
Since we will deal with Fermions (identical and indistinguishable particles) we will 
form an ansatz for a given state in terms of so-called Slater determinants determined
by a chosen basis of single-particle functions. 

For a given $n\times n$ matrix $\mathbf{A}$ we can write its determinant
\[
   det(\mathbf{A})=|\mathbf{A}|=
\left| \begin{array}{ccccc} a_{11}& a_{12}& \dots & \dots & a_{1n}\\
                            a_{21}&a_{22}& \dots & \dots & a_{2n}\\  
                            \dots & \dots & \dots & \dots & \dots \\
                            \dots & \dots & \dots & \dots & \dots \\
                            a_{n1}& a_{n2}& \dots & \dots & a_{nn}\end{array} \right|,
\]
in a more compact form as 
\[
|\mathbf{A}|= \sum_{i=1}^{n!}(-1)^{p_i}\hat{P}_i a_{11}a_{22}\dots a_{nn},
\]
where $\hat{P}_i$ is a permutation operator which permutes the column indices $1,2,3,\dots,n$
and the sum runs over all $n!$ permutations.  The quantity $p_i$ represents the number of transpositions of column indices that are needed in order to bring a given permutation back to its initial ordering, in our case given by $a_{11}a_{22}\dots a_{nn}$ here.

A simple $2\times 2$ determinant illustrates this. We have
\[
   det(\mathbf{A})=
\left| \begin{array}{cc} a_{11}& a_{12}\\
                            a_{21}&a_{22}\end{array} \right|= (-1)^0a_{11}a_{22}+(-1)^1a_{12}a_{21},
\]
where in the last term we have interchanged the column indices $1$ and $2$. The natural ordering we have chosen is $a_{11}a_{22}$. 

\paragraph{Back to the derivation of the energy.}
The single-particle function $\psi_{\alpha}(x_i)$  are eigenfunctions of the onebody
Hamiltonian $h_i$, that is
\[
\hat{h}_0(x_i)=\hat{t}(x_i) + \hat{u}_{\mathrm{ext}}(x_i),
\]
with eigenvalues 
\[
\hat{h}_0(x_i) \psi_{\alpha}(x_i)=\left(\hat{t}(x_i) + \hat{u}_{\mathrm{ext}}(x_i)\right)\psi_{\alpha}(x_i)=\varepsilon_{\alpha}\psi_{\alpha}(x_i).
\]
The energies $\varepsilon_{\alpha}$ are the so-called non-interacting single-particle energies, or unperturbed energies. 
The total energy is in this case the sum over all  single-particle energies, if no two-body or more complicated
many-body interactions are present.

Let us denote the ground state energy by $E_0$. According to the
variational principle we have
\[
  E_0 \le E[\Phi] = \int \Phi^*\hat{H}\Phi d\mathbf{\tau}
\]
where $\Phi$ is a trial function which we assume to be normalized
\[
  \int \Phi^*\Phi d\mathbf{\tau} = 1,
\]
where we have used the shorthand $d\mathbf{\tau}=dx_1dr_2\dots dr_A$.

In the Hartree-Fock method the trial function is the Slater
determinant of Eq.~(\ref{eq:HartreeFockDet}) which can be rewritten as 
\[
  \Phi(x_1,x_2,\dots,x_A,\alpha,\beta,\dots,\nu) = \frac{1}{\sqrt{A!}}\sum_{P} (-)^P\hat{P}\psi_{\alpha}(x_1)
    \psi_{\beta}(x_2)\dots\psi_{\nu}(x_A)=\sqrt{A!}\hat{A}\Phi_H,
\]
where we have introduced the antisymmetrization operator $\hat{A}$ defined by the 
summation over all possible permutations of two particles.

It is defined as
\begin{equation}
  \hat{A} = \frac{1}{A!}\sum_{p} (-)^p\hat{P},
\label{antiSymmetryOperator}
\end{equation}
with $p$ standing for the number of permutations. We have introduced for later use the so-called
Hartree-function, defined by the simple product of all possible single-particle functions
\[
  \Phi_H(x_1,x_2,\dots,x_A,\alpha,\beta,\dots,\nu) =
  \psi_{\alpha}(x_1)
    \psi_{\beta}(x_2)\dots\psi_{\nu}(x_A).
\]

Both $\hat{H}_0$ and $\hat{H}_I$ are invariant under all possible permutations of any two particles
and hence commute with $\hat{A}$
\begin{equation}
  [H_0,\hat{A}] = [H_I,\hat{A}] = 0. \label{commutionAntiSym}
\end{equation}
Furthermore, $\hat{A}$ satisfies
\begin{equation}
  \hat{A}^2 = \hat{A},  \label{AntiSymSquared}
\end{equation}
since every permutation of the Slater
determinant reproduces it. 

The expectation value of $\hat{H}_0$ 
\[
  \int \Phi^*\hat{H}_0\Phi d\mathbf{\tau} 
  = A! \int \Phi_H^*\hat{A}\hat{H}_0\hat{A}\Phi_H d\mathbf{\tau}
\]
is readily reduced to
\[
  \int \Phi^*\hat{H}_0\Phi d\mathbf{\tau} 
  = A! \int \Phi_H^*\hat{H}_0\hat{A}\Phi_H d\mathbf{\tau},
\]
where we have used Eqs.~(\ref{commutionAntiSym}) and
(\ref{AntiSymSquared}). The next step is to replace the antisymmetrization
operator by its definition and to
replace $\hat{H}_0$ with the sum of one-body operators
\[
  \int \Phi^*\hat{H}_0\Phi  d\mathbf{\tau}
  = \sum_{i=1}^A \sum_{p} (-)^p\int 
  \Phi_H^*\hat{h}_0\hat{P}\Phi_H d\mathbf{\tau}.
\]

The integral vanishes if two or more particles are permuted in only one
of the Hartree-functions $\Phi_H$ because the individual single-particle wave functions are
orthogonal. We obtain then
\[
  \int \Phi^*\hat{H}_0\Phi  d\mathbf{\tau}= \sum_{i=1}^A \int \Phi_H^*\hat{h}_0\Phi_H  d\mathbf{\tau}.
\]
Orthogonality of the single-particle functions allows us to further simplify the integral, and we
arrive at the following expression for the expectation values of the
sum of one-body Hamiltonians 
\begin{equation}
  \int \Phi^*\hat{H}_0\Phi  d\mathbf{\tau}
  = \sum_{\mu=1}^A \int \psi_{\mu}^*(x)\hat{h}_0\psi_{\mu}(x)dx
  d\mathbf{r}.
  \label{H1Expectation}
\end{equation}

We introduce the following shorthand for the above integral
\[
\langle \mu | \hat{h}_0 | \mu \rangle = \int \psi_{\mu}^*(x)\hat{h}_0\psi_{\mu}(x)dx,
\]
and rewrite Eq.~(\ref{H1Expectation}) as
\begin{equation}
  \int \Phi^*\hat{H}_0\Phi  d\tau
  = \sum_{\mu=1}^A \langle \mu | \hat{h}_0 | \mu \rangle.
  \label{H1Expectation1}
\end{equation}

The expectation value of the two-body part of the Hamiltonian is obtained in a
similar manner. We have
\[
  \int \Phi^*\hat{H}_I\Phi d\mathbf{\tau} 
  = A! \int \Phi_H^*\hat{A}\hat{H}_I\hat{A}\Phi_H d\mathbf{\tau},
\]
which reduces to
\[
 \int \Phi^*\hat{H}_I\Phi d\mathbf{\tau} 
  = \sum_{i\le j=1}^A \sum_{p} (-)^p\int 
  \Phi_H^*\hat{v}(r_{ij})\hat{P}\Phi_H d\mathbf{\tau},
\]
by following the same arguments as for the one-body
Hamiltonian. 

Because of the dependence on the inter-particle distance $r_{ij}$,  permutations of
any two particles no longer vanish, and we get
\[
  \int \Phi^*\hat{H}_I\Phi d\mathbf{\tau} 
  = \sum_{i < j=1}^A \int  
  \Phi_H^*\hat{v}(r_{ij})(1-P_{ij})\Phi_H d\mathbf{\tau}.
\]
where $P_{ij}$ is the permutation operator that interchanges
particle $i$ and particle $j$. Again we use the assumption that the single-particle wave functions
are orthogonal. 

We obtain
\begin{align}
  \int \Phi^*\hat{H}_I\Phi d\mathbf{\tau} 
  = \frac{1}{2}\sum_{\mu=1}^A\sum_{\nu=1}^A
    &\left[ \int \psi_{\mu}^*(x_i)\psi_{\nu}^*(x_j)\hat{v}(r_{ij})\psi_{\mu}(x_i)\psi_{\nu}(x_j)
    dx_idx_j \right.\\
  &\left.
  - \int \psi_{\mu}^*(x_i)\psi_{\nu}^*(x_j)
  \hat{v}(r_{ij})\psi_{\nu}(x_i)\psi_{\mu}(x_j)
  dx_idx_j
  \right]. \label{H2Expectation}
\end{align}
The first term is the so-called direct term. It is frequently also called the  Hartree term, 
while the second is due to the Pauli principle and is called
the exchange term or just the Fock term.
The factor  $1/2$ is introduced because we now run over
all pairs twice. 

The last equation allows us to  introduce some further definitions.  
The single-particle wave functions $\psi_{\mu}(x)$, defined by the quantum numbers $\mu$ and $x$
are defined as the overlap 
\[
   \psi_{\alpha}(x)  = \langle x | \alpha \rangle .
\]

We introduce the following shorthands for the above two integrals
\[
\langle \mu\nu|\hat{v}|\mu\nu\rangle =  \int \psi_{\mu}^*(x_i)\psi_{\nu}^*(x_j)\hat{v}(r_{ij})\psi_{\mu}(x_i)\psi_{\nu}(x_j)
    dx_idx_j,
\]
and
\[
\langle \mu\nu|\hat{v}|\nu\mu\rangle = \int \psi_{\mu}^*(x_i)\psi_{\nu}^*(x_j)
  \hat{v}(r_{ij})\psi_{\nu}(x_i)\psi_{\mu}(x_j)
  dx_idx_j.  
\]

\subsection*{Preparing for later studies: varying the coefficients of a wave function expansion and orthogonal transformations}

It is common to  expand the single-particle functions in a known basis  and vary the coefficients, 
that is, the new single-particle wave function is written as a linear expansion
in terms of a fixed chosen orthogonal basis (for example the well-known harmonic oscillator functions or the hydrogen-like functions etc).
We define our new single-particle basis (this is a normal approach for Hartree-Fock theory) by performing a unitary transformation 
on our previous basis (labelled with greek indices) as
\begin{equation}
\psi_p^{new}  = \sum_{\lambda} C_{p\lambda}\phi_{\lambda}. \label{eq:newbasis}
\end{equation}
In this case we vary the coefficients $C_{p\lambda}$. If the basis has infinitely many solutions, we need
to truncate the above sum.  We assume that the basis $\phi_{\lambda}$ is orthogonal.

It is normal to choose a single-particle basis defined as the eigenfunctions
of parts of the full Hamiltonian. The typical situation consists of the solutions of the one-body part of the Hamiltonian, that is we have
\[
\hat{h}_0\phi_{\lambda}=\epsilon_{\lambda}\phi_{\lambda}.
\]
The single-particle wave functions $\phi_{\lambda}(\mathbf{r})$, defined by the quantum numbers $\lambda$ and $\mathbf{r}$
are defined as the overlap 
\[
   \phi_{\lambda}(\mathbf{r})  = \langle \mathbf{r} | \lambda \rangle .
\]

In deriving the Hartree-Fock equations, we  will expand the single-particle functions in a known basis  and vary the coefficients, 
that is, the new single-particle wave function is written as a linear expansion
in terms of a fixed chosen orthogonal basis (for example the well-known harmonic oscillator functions or the hydrogen-like functions etc).

We stated that a unitary transformation keeps the orthogonality. To see this consider first a basis of vectors $\mathbf{v}_i$,
\[
\mathbf{v}_i = \begin{bmatrix} v_{i1} \\ \dots \\ \dots \\v_{in} \end{bmatrix}
\]
We assume that the basis is orthogonal, that is 
\[
\mathbf{v}_j^T\mathbf{v}_i = \delta_{ij}.
\]
An orthogonal or unitary transformation
\[
\mathbf{w}_i=\mathbf{U}\mathbf{v}_i,
\]
preserves the dot product and orthogonality since
\[
\mathbf{w}_j^T\mathbf{w}_i=(\mathbf{U}\mathbf{v}_j)^T\mathbf{U}\mathbf{v}_i=\mathbf{v}_j^T\mathbf{U}^T\mathbf{U}\mathbf{v}_i= \mathbf{v}_j^T\mathbf{v}_i = \delta_{ij}.
\]

This means that if the coefficients $C_{p\lambda}$ belong to a unitary or orthogonal trasformation (using the Dirac bra-ket notation)
\[
\vert p\rangle  = \sum_{\lambda} C_{p\lambda}\vert\lambda\rangle,
\]
orthogonality is preserved, that is $\langle \alpha \vert \beta\rangle = \delta_{\alpha\beta}$
and $\langle p \vert q\rangle = \delta_{pq}$. 

This propertry is extremely useful when we build up a basis of many-body Stater determinant based states. 

\textbf{Note also that although a basis $\vert \alpha\rangle$ contains an infinity of states, for practical calculations we have always to make some truncations.} 

Before we develop for example the Hartree-Fock equations, there is another very useful property of determinants that we will use both in connection with Hartree-Fock calculations and later shell-model calculations.  

Consider the following determinant
\[
\left| \begin{array}{cc} \alpha_1b_{11}+\alpha_2sb_{12}& a_{12}\\
                         \alpha_1b_{21}+\alpha_2b_{22}&a_{22}\end{array} \right|=\alpha_1\left|\begin{array}{cc} b_{11}& a_{12}\\
                         b_{21}&a_{22}\end{array} \right|+\alpha_2\left| \begin{array}{cc} b_{12}& a_{12}\\b_{22}&a_{22}\end{array} \right|
\]

We can generalize this to  an $n\times n$ matrix and have 
\[
\left| \begin{array}{cccccc} a_{11}& a_{12} & \dots & \sum_{k=1}^n c_k b_{1k} &\dots & a_{1n}\\
a_{21}& a_{22} & \dots & \sum_{k=1}^n c_k b_{2k} &\dots & a_{2n}\\
\dots & \dots & \dots & \dots & \dots & \dots \\
\dots & \dots & \dots & \dots & \dots & \dots \\
a_{n1}& a_{n2} & \dots & \sum_{k=1}^n c_k b_{nk} &\dots & a_{nn}\end{array} \right|=
\sum_{k=1}^n c_k\left| \begin{array}{cccccc} a_{11}& a_{12} & \dots &  b_{1k} &\dots & a_{1n}\\
a_{21}& a_{22} & \dots &  b_{2k} &\dots & a_{2n}\\
\dots & \dots & \dots & \dots & \dots & \dots\\
\dots & \dots & \dots & \dots & \dots & \dots\\
a_{n1}& a_{n2} & \dots &  b_{nk} &\dots & a_{nn}\end{array} \right| .
\]
This is a property we will use in our Hartree-Fock discussions. 

We can generalize the previous results, now 
with all elements $a_{ij}$  being given as functions of 
linear combinations  of various coefficients $c$ and elements $b_{ij}$,
\[
\left| \begin{array}{cccccc} \sum_{k=1}^n b_{1k}c_{k1}& \sum_{k=1}^n b_{1k}c_{k2} & \dots & \sum_{k=1}^n b_{1k}c_{kj}  &\dots & \sum_{k=1}^n b_{1k}c_{kn}\\
\sum_{k=1}^n b_{2k}c_{k1}& \sum_{k=1}^n b_{2k}c_{k2} & \dots & \sum_{k=1}^n b_{2k}c_{kj} &\dots & \sum_{k=1}^n b_{2k}c_{kn}\\
\dots & \dots & \dots & \dots & \dots & \dots \\
\dots & \dots & \dots & \dots & \dots &\dots \\
\sum_{k=1}^n b_{nk}c_{k1}& \sum_{k=1}^n b_{nk}c_{k2} & \dots & \sum_{k=1}^n b_{nk}c_{kj} &\dots & \sum_{k=1}^n b_{nk}c_{kn}\end{array} \right|=det(\mathbf{C})det(\mathbf{B}),
\]
where $det(\mathbf{C})$ and $det(\mathbf{B})$ are the determinants of $n\times n$ matrices
with elements $c_{ij}$ and $b_{ij}$ respectively.  
This is a property we will use in our Hartree-Fock discussions. Convince yourself about the correctness of the above expression by setting $n=2$. 

With our definition of the new basis in terms of an orthogonal basis we have
\[
\psi_p(x)  = \sum_{\lambda} C_{p\lambda}\phi_{\lambda}(x).
\]
If the coefficients $C_{p\lambda}$ belong to an orthogonal or unitary matrix, the new basis
is also orthogonal. 
Our Slater determinant in the new basis $\psi_p(x)$ is written as
\[
\frac{1}{\sqrt{A!}}
\left| \begin{array}{ccccc} \psi_{p}(x_1)& \psi_{p}(x_2)& \dots & \dots & \psi_{p}(x_A)\\
                            \psi_{q}(x_1)&\psi_{q}(x_2)& \dots & \dots & \psi_{q}(x_A)\\  
                            \dots & \dots & \dots & \dots & \dots \\
                            \dots & \dots & \dots & \dots & \dots \\
                     \psi_{t}(x_1)&\psi_{t}(x_2)& \dots & \dots & \psi_{t}(x_A)\end{array} \right|=\frac{1}{\sqrt{A!}}
\left| \begin{array}{ccccc} \sum_{\lambda} C_{p\lambda}\phi_{\lambda}(x_1)& \sum_{\lambda} C_{p\lambda}\phi_{\lambda}(x_2)& \dots & \dots & \sum_{\lambda} C_{p\lambda}\phi_{\lambda}(x_A)\\
                            \sum_{\lambda} C_{q\lambda}\phi_{\lambda}(x_1)&\sum_{\lambda} C_{q\lambda}\phi_{\lambda}(x_2)& \dots & \dots & \sum_{\lambda} C_{q\lambda}\phi_{\lambda}(x_A)\\  
                            \dots & \dots & \dots & \dots & \dots \\
                            \dots & \dots & \dots & \dots & \dots \\
                     \sum_{\lambda} C_{t\lambda}\phi_{\lambda}(x_1)&\sum_{\lambda} C_{t\lambda}\phi_{\lambda}(x_2)& \dots & \dots & \sum_{\lambda} C_{t\lambda}\phi_{\lambda}(x_A)\end{array} \right|,
\]
which is nothing but $det(\mathbf{C})det(\Phi)$, with $det(\Phi)$ being the determinant given by the basis functions $\phi_{\lambda}(x)$. 

In our discussions hereafter we will use our definitions of single-particle states above and below the Fermi ($F$) level given by the labels
$ijkl\dots \le F$ for so-called single-hole states and $abcd\dots > F$ for so-called particle states.
For general single-particle states we employ the labels $pqrs\dots$. 

The energy functional is
\[
  E[\Phi] 
  = \sum_{\mu=1}^A \langle \mu | h | \mu \rangle +
  \frac{1}{2}\sum_{{\mu}=1}^A\sum_{{\nu}=1}^A \langle \mu\nu|\hat{v}|\mu\nu\rangle_{AS},
\]
we found the expression for the energy functional in terms of the basis function $\phi_{\lambda}(\mathbf{r})$. We then  varied the above energy functional with respect to the basis functions $|\mu \rangle$. 
Now we are interested in defining a new basis defined in terms of
a chosen basis as defined in Eq.~(\ref{eq:newbasis}). We can then rewrite the energy functional as
\begin{equation}
  E[\Phi^{New}] 
  = \sum_{i=1}^A \langle i | h | i \rangle +
  \frac{1}{2}\sum_{ij=1}^A\langle ij|\hat{v}|ij\rangle_{AS}, \label{FunctionalEPhi2}
\end{equation}
where $\Phi^{New}$ is the new Slater determinant defined by the new basis of Eq.~(\ref{eq:newbasis}). 

Using Eq.~(\ref{eq:newbasis}) we can rewrite Eq.~(\ref{FunctionalEPhi2}) as 
\begin{equation}
  E[\Psi] 
  = \sum_{i=1}^A \sum_{\alpha\beta} C^*_{i\alpha}C_{i\beta}\langle \alpha | h | \beta \rangle +
  \frac{1}{2}\sum_{ij=1}^A\sum_{{\alpha\beta\gamma\delta}} C^*_{i\alpha}C^*_{j\beta}C_{i\gamma}C_{j\delta}\langle \alpha\beta|\hat{v}|\gamma\delta\rangle_{AS}. \label{FunctionalEPhi3}
\end{equation}

\subsection*{Second quantization}

We introduce the time-independent  operators
$a_\alpha^{\dagger}$ and $a_\alpha$   which create and annihilate, respectively, a particle 
in the single-particle state 
$\varphi_\alpha$. 
We define the fermion creation operator
$a_\alpha^{\dagger}$ 
\begin{equation}
	a_\alpha^{\dagger}|0\rangle \equiv  |\alpha\rangle  \label{eq:2-1a},
\end{equation}
and
\begin{equation}
	a_\alpha^{\dagger}|\alpha_1\dots \alpha_n\rangle_{\mathrm{AS}} \equiv  |\alpha\alpha_1\dots \alpha_n\rangle_{\mathrm{AS}} \label{eq:2-1b}
\end{equation}

In Eq.~(\ref{eq:2-1a}) 
the operator  $a_\alpha^{\dagger}$  acts on the vacuum state 
$|0\rangle$, which does not contain any particles. Alternatively, we could define  a closed-shell nucleus or atom as our new vacuum, but then
we need to introduce the particle-hole  formalism, see the discussion to come. 

In Eq.~(\ref{eq:2-1b}) $a_\alpha^{\dagger}$ acts on an antisymmetric $n$-particle state and 
creates an antisymmetric $(n+1)$-particle state, where the one-body state 
$\varphi_\alpha$ is occupied, under the condition that
$\alpha \ne \alpha_1, \alpha_2, \dots, \alpha_n$. 
It follows that we can express an antisymmetric state as the product of the creation
operators acting on the vacuum state.  
\begin{equation}
	|\alpha_1\dots \alpha_n\rangle_{\mathrm{AS}} = a_{\alpha_1}^{\dagger} a_{\alpha_2}^{\dagger} \dots a_{\alpha_n}^{\dagger} |0\rangle \label{eq:2-2}
\end{equation}

It is easy to derive the commutation and anticommutation rules  for the fermionic creation operators 
$a_\alpha^{\dagger}$. Using the antisymmetry of the states 
(\ref{eq:2-2})
\begin{equation}
	|\alpha_1\dots \alpha_i\dots \alpha_k\dots \alpha_n\rangle_{\mathrm{AS}} = 
		- |\alpha_1\dots \alpha_k\dots \alpha_i\dots \alpha_n\rangle_{\mathrm{AS}} \label{eq:2-3a}
\end{equation}
we obtain
\begin{equation}
	 a_{\alpha_i}^{\dagger}  a_{\alpha_k}^{\dagger} = - a_{\alpha_k}^{\dagger} a_{\alpha_i}^{\dagger} \label{eq:2-3b}
\end{equation}

Using the Pauli principle
\begin{equation}
	|\alpha_1\dots \alpha_i\dots \alpha_i\dots \alpha_n\rangle_{\mathrm{AS}} = 0 \label{eq:2-4a}
\end{equation}
it follows that
\begin{equation}
	a_{\alpha_i}^{\dagger}  a_{\alpha_i}^{\dagger} = 0. \label{eq:2-4b}
\end{equation}
If we combine Eqs.~(\ref{eq:2-3b}) and (\ref{eq:2-4b}), we obtain the well-known anti-commutation rule
\begin{equation}
	a_{\alpha}^{\dagger}  a_{\beta}^{\dagger} + a_{\beta}^{\dagger}  a_{\alpha}^{\dagger} \equiv 
		\{a_{\alpha}^{\dagger},a_{\beta}^{\dagger}\} = 0 \label{eq:2-5}
\end{equation}

The hermitian conjugate  of $a_\alpha^{\dagger}$ is
\begin{equation}
	a_{\alpha} = ( a_{\alpha}^{\dagger} )^{\dagger} \label{eq:2-6}
\end{equation}
If we take the hermitian conjugate of Eq.~(\ref{eq:2-5}), we arrive at 
\begin{equation}
	\{a_{\alpha},a_{\beta}\} = 0 \label{eq:2-7}
\end{equation}

What is the physical interpretation of the operator $a_\alpha$ and what is the effect of 
$a_\alpha$ on a given state $|\alpha_1\alpha_2\dots\alpha_n\rangle_{\mathrm{AS}}$? 
Consider the following matrix element
\begin{equation}
	\langle\alpha_1\alpha_2 \dots \alpha_n|a_\alpha|\alpha_1'\alpha_2' \dots \alpha_m'\rangle \label{eq:2-8}
\end{equation}
where both sides are antisymmetric. We  distinguish between two cases. The first (1) is when
$\alpha \in \{\alpha_i\}$. Using the Pauli principle of Eq.~(\ref{eq:2-4a}) it follows
\begin{equation}
		\langle\alpha_1\alpha_2 \dots \alpha_n|a_\alpha = 0 \label{eq:2-9a}
\end{equation}
The second (2) case is when $\alpha \notin \{\alpha_i\}$. It follows that an hermitian conjugation
\begin{equation}
		\langle \alpha_1\alpha_2 \dots \alpha_n|a_\alpha = \langle\alpha\alpha_1\alpha_2 \dots \alpha_n|  \label{eq:2-9b}
\end{equation}

Eq.~(\ref{eq:2-9b}) holds for case (1) since the lefthand side is zero due to the Pauli principle. We write
Eq.~(\ref{eq:2-8}) as
\begin{equation}
	\langle\alpha_1\alpha_2 \dots \alpha_n|a_\alpha|\alpha_1'\alpha_2' \dots \alpha_m'\rangle = 
	\langle \alpha_1\alpha_2 \dots \alpha_n|\alpha\alpha_1'\alpha_2' \dots \alpha_m'\rangle \label{eq:2-10}
\end{equation}
Here we must have $m = n+1$ if Eq.~(\ref{eq:2-10}) has to be trivially different from zero.

For the last case, the minus and plus signs apply when the sequence 
$\alpha ,\alpha_1, \alpha_2, \dots, \alpha_n$ and 
$\alpha_1', \alpha_2', \dots, \alpha_{n+1}'$ are related to each other via even and odd permutations.
If we assume that  $\alpha \notin \{\alpha_i\}$ we obtain 
\begin{equation}
	\langle\alpha_1\alpha_2 \dots \alpha_n|a_\alpha|\alpha_1'\alpha_2' \dots \alpha_{n+1}'\rangle = 0 \label{eq:2-12}
\end{equation}
when $\alpha \in \{\alpha_i'\}$. If $\alpha \notin \{\alpha_i'\}$, we obtain
\begin{equation}
	a_\alpha\underbrace{|\alpha_1'\alpha_2' \dots \alpha_{n+1}'}\rangle_{\neq \alpha} = 0 \label{eq:2-13a}
\end{equation}
and in particular
\begin{equation}
	a_\alpha |0\rangle = 0 \label{eq:2-13b}
\end{equation}

If $\{\alpha\alpha_i\} = \{\alpha_i'\}$, performing the right permutations, the sequence
$\alpha ,\alpha_1, \alpha_2, \dots, \alpha_n$ is identical with the sequence
$\alpha_1', \alpha_2', \dots, \alpha_{n+1}'$. This results in
\begin{equation}
	\langle\alpha_1\alpha_2 \dots \alpha_n|a_\alpha|\alpha\alpha_1\alpha_2 \dots \alpha_{n}\rangle = 1 \label{eq:2-14}
\end{equation}
and thus
\begin{equation}
	a_\alpha |\alpha\alpha_1\alpha_2 \dots \alpha_{n}\rangle = |\alpha_1\alpha_2 \dots \alpha_{n}\rangle \label{eq:2-15}
\end{equation}

The action of the operator 
$a_\alpha$ from the left on a state vector  is to to remove  one particle in the state
$\alpha$. 
If the state vector does not contain the single-particle state $\alpha$, the outcome of the operation is zero.
The operator  $a_\alpha$ is normally called for a destruction or annihilation operator.

The next step is to establish the  commutator algebra of $a_\alpha^{\dagger}$ and
$a_\beta$. 

The action of the anti-commutator 
$\{a_\alpha^{\dagger}$,$a_\alpha\}$ on a given $n$-particle state is
\begin{align}
	a_\alpha^{\dagger} a_\alpha \underbrace{|\alpha_1\alpha_2 \dots \alpha_{n}\rangle}_{\neq \alpha} &= 0 \nonumber \\
	a_\alpha a_\alpha^{\dagger} \underbrace{|\alpha_1\alpha_2 \dots \alpha_{n}\rangle}_{\neq \alpha} &=
	a_\alpha \underbrace{|\alpha \alpha_1\alpha_2 \dots \alpha_{n}\rangle}_{\neq \alpha} = 
	\underbrace{|\alpha_1\alpha_2 \dots \alpha_{n}\rangle}_{\neq \alpha} \label{eq:2-16a}
\end{align}
if the single-particle state $\alpha$ is not contained in the state.

 If it is present
we arrive at
\begin{align}
	a_\alpha^{\dagger} a_\alpha |\alpha_1\alpha_2 \dots \alpha_{k}\alpha \alpha_{k+1} \dots \alpha_{n-1}\rangle &=
	a_\alpha^{\dagger} a_\alpha (-1)^k |\alpha \alpha_1\alpha_2 \dots \alpha_{n-1}\rangle \nonumber \\
	= (-1)^k |\alpha \alpha_1\alpha_2 \dots \alpha_{n-1}\rangle &=
	|\alpha_1\alpha_2 \dots \alpha_{k}\alpha \alpha_{k+1} \dots \alpha_{n-1}\rangle \nonumber \\
	a_\alpha a_\alpha^{\dagger}|\alpha_1\alpha_2 \dots \alpha_{k}\alpha \alpha_{k+1} \dots \alpha_{n-1}\rangle &= 0 \label{eq:2-16b}
\end{align}
From Eqs.~(\ref{eq:2-16a}) and  (\ref{eq:2-16b}) we arrive at 
\begin{equation}
	\{a_\alpha^{\dagger} , a_\alpha \} = a_\alpha^{\dagger} a_\alpha + a_\alpha a_\alpha^{\dagger} = 1 \label{eq:2-17}
\end{equation}

The action of $\left\{a_\alpha^{\dagger}, a_\beta\right\}$, with 
$\alpha \ne \beta$ on a given state yields three possibilities. 
The first case is a state vector which contains both $\alpha$ and $\beta$, then either 
$\alpha$ or $\beta$ and finally none of them.

The first case results in
\begin{align}
	a_\alpha^{\dagger} a_\beta |\alpha\beta\alpha_1\alpha_2 \dots \alpha_{n-2}\rangle = 0 \nonumber \\
	a_\beta a_\alpha^{\dagger} |\alpha\beta\alpha_1\alpha_2 \dots \alpha_{n-2}\rangle = 0 \label{eq:2-18a}
\end{align}
while the second case gives
\begin{align}
	 a_\alpha^{\dagger} a_\beta |\beta \underbrace{\alpha_1\alpha_2 \dots \alpha_{n-1}}_{\neq \alpha}\rangle =& 
	 	|\alpha \underbrace{\alpha_1\alpha_2 \dots \alpha_{n-1}}_{\neq  \alpha}\rangle \nonumber \\
	a_\beta a_\alpha^{\dagger} |\beta \underbrace{\alpha_1\alpha_2 \dots \alpha_{n-1}}_{\neq \alpha}\rangle =&
		a_\beta |\alpha\beta\underbrace{\beta \alpha_1\alpha_2 \dots \alpha_{n-1}}_{\neq \alpha}\rangle \nonumber \\
	=& - |\alpha\underbrace{\alpha_1\alpha_2 \dots \alpha_{n-1}}_{\neq \alpha}\rangle \label{eq:2-18b}
\end{align}

Finally if the state vector does not contain $\alpha$ and $\beta$
\begin{align}
	a_\alpha^{\dagger} a_\beta |\underbrace{\alpha_1\alpha_2 \dots \alpha_{n}}_{\neq \alpha,\beta}\rangle &=& 0 \nonumber \\
	a_\beta a_\alpha^{\dagger} |\underbrace{\alpha_1\alpha_2 \dots \alpha_{n}}_{\neq \alpha,\beta}\rangle &=& 
		a_\beta |\alpha \underbrace{\alpha_1\alpha_2 \dots \alpha_{n}}_{\neq \alpha,\beta}\rangle = 0 \label{eq:2-18c}
\end{align}
For all three cases we have
\begin{equation}
	\{a_\alpha^{\dagger},a_\beta \} = a_\alpha^{\dagger} a_\beta + a_\beta a_\alpha^{\dagger} = 0, \quad \alpha \neq \beta \label{eq:2-19}
\end{equation}

We can summarize  our findings in Eqs.~(\ref{eq:2-17}) and (\ref{eq:2-19}) as 
\begin{equation}
	\{a_\alpha^{\dagger},a_\beta \} = \delta_{\alpha\beta} \label{eq:2-20}
\end{equation}
with  $\delta_{\alpha\beta}$ is the Kroenecker $\delta$-symbol.

The properties of the creation and annihilation operators can be summarized as (for fermions)
\[
	a_\alpha^{\dagger}|0\rangle \equiv  |\alpha\rangle,
\]
and
\[
	a_\alpha^{\dagger}|\alpha_1\dots \alpha_n\rangle_{\mathrm{AS}} \equiv  |\alpha\alpha_1\dots \alpha_n\rangle_{\mathrm{AS}}. 
\]
from which follows
\[
        |\alpha_1\dots \alpha_n\rangle_{\mathrm{AS}} = a_{\alpha_1}^{\dagger} a_{\alpha_2}^{\dagger} \dots a_{\alpha_n}^{\dagger} |0\rangle.
\]

The hermitian conjugate has the folowing properties
\[
        a_{\alpha} = ( a_{\alpha}^{\dagger} )^{\dagger}.
\]
Finally we found 
\[
	a_\alpha\underbrace{|\alpha_1'\alpha_2' \dots \alpha_{n+1}'}\rangle_{\neq \alpha} = 0, \quad
		\textrm{in particular } a_\alpha |0\rangle = 0,
\]
and
\[
 a_\alpha |\alpha\alpha_1\alpha_2 \dots \alpha_{n}\rangle = |\alpha_1\alpha_2 \dots \alpha_{n}\rangle,
\]
and the corresponding commutator algebra
\[
	\{a_{\alpha}^{\dagger},a_{\beta}^{\dagger}\} = \{a_{\alpha},a_{\beta}\} = 0 \hspace{0.5cm} 
\{a_\alpha^{\dagger},a_\beta \} = \delta_{\alpha\beta}.
\]

\subsection*{One-body operators in second quantization}

A very useful operator is the so-called number-operator.  Most physics cases  we will
study in this text conserve the total number of particles.  The number operator is therefore
a useful quantity which allows us to test that our many-body formalism  conserves the number of particles.
In for example $(d,p)$ or $(p,d)$ reactions it is important to be able to describe quantum mechanical states
where particles get added or removed.
A creation operator $a_\alpha^{\dagger}$ adds one particle to the single-particle state
$\alpha$ of a give many-body state vector, while an annihilation operator $a_\alpha$ 
removes a particle from a single-particle
state $\alpha$. 

Let us consider an operator proportional with $a_\alpha^{\dagger} a_\beta$ and 
$\alpha=\beta$. It acts on an $n$-particle state 
resulting in
\begin{equation}
	a_\alpha^{\dagger} a_\alpha |\alpha_1\alpha_2 \dots \alpha_{n}\rangle = 
	\begin{cases}
		0  &\alpha \notin \{\alpha_i\} \\
		\\
		|\alpha_1\alpha_2 \dots \alpha_{n}\rangle & \alpha \in \{\alpha_i\}
	\end{cases}
\end{equation}
Summing over all possible one-particle states we arrive at
\begin{equation}
	\left( \sum_\alpha a_\alpha^{\dagger} a_\alpha \right) |\alpha_1\alpha_2 \dots \alpha_{n}\rangle = 
	n |\alpha_1\alpha_2 \dots \alpha_{n}\rangle \label{eq:2-21}
\end{equation}

The operator 
\begin{equation}
	\hat{N} = \sum_\alpha a_\alpha^{\dagger} a_\alpha \label{eq:2-22}
\end{equation}
is called the number operator since it counts the number of particles in a give state vector when it acts 
on the different single-particle states.  It acts on one single-particle state at the time and falls 
therefore under category one-body operators.
Next we look at another important one-body operator, namely $\hat{H}_0$ and study its operator form in the 
occupation number representation.

We want to obtain an expression for a one-body operator which conserves the number of particles.
Here we study the one-body operator for the kinetic energy plus an eventual external one-body potential.
The action of this operator on a particular $n$-body state with its pertinent expectation value has already
been studied in coordinate  space.
In coordinate space the operator reads
\begin{equation}
	\hat{H}_0 = \sum_i \hat{h}_0(x_i) \label{eq:2-23}
\end{equation}
and the anti-symmetric $n$-particle Slater determinant is defined as 
\[
\Phi(x_1, x_2,\dots ,x_n,\alpha_1,\alpha_2,\dots, \alpha_n)= \frac{1}{\sqrt{n!}} \sum_p (-1)^p\hat{P}\psi_{\alpha_1}(x_1)\psi_{\alpha_2}(x_2) \dots \psi_{\alpha_n}(x_n).
\]

Defining
\begin{equation}
	\hat{h}_0(x_i) \psi_{\alpha_i}(x_i) = \sum_{\alpha_k'} \psi_{\alpha_k'}(x_i) \langle\alpha_k'|\hat{h}_0|\alpha_k\rangle \label{eq:2-25}
\end{equation}
we can easily  evaluate the action of $\hat{H}_0$ on each product of one-particle functions in Slater determinant.
From Eq.~(\ref{eq:2-25})  we obtain the following result without  permuting any particle pair 
\begin{align}
	&& \left( \sum_i \hat{h}_0(x_i) \right) \psi_{\alpha_1}(x_1)\psi_{\alpha_2}(x_2) \dots \psi_{\alpha_n}(x_n) \nonumber \\
	& =&\sum_{\alpha_1'} \langle \alpha_1'|\hat{h}_0|\alpha_1\rangle 
		\psi_{\alpha_1'}(x_1)\psi_{\alpha_2}(x_2) \dots \psi_{\alpha_n}(x_n) \nonumber \\
	&+&\sum_{\alpha_2'} \langle \alpha_2'|\hat{h}_0|\alpha_2\rangle
		\psi_{\alpha_1}(x_1)\psi_{\alpha_2'}(x_2) \dots \psi_{\alpha_n}(x_n) \nonumber \\
	&+& \dots \nonumber \\
	&+&\sum_{\alpha_n'} \langle \alpha_n'|\hat{h}_0|\alpha_n\rangle
		\psi_{\alpha_1}(x_1)\psi_{\alpha_2}(x_2) \dots \psi_{\alpha_n'}(x_n) \label{eq:2-26}
\end{align}

If we interchange particles $1$ and $2$  we obtain
\begin{align}
	&& \left( \sum_i \hat{h}_0(x_i) \right) \psi_{\alpha_1}(x_2)\psi_{\alpha_1}(x_2) \dots \psi_{\alpha_n}(x_n) \nonumber \\
	& =&\sum_{\alpha_2'} \langle \alpha_2'|\hat{h}_0|\alpha_2\rangle 
		\psi_{\alpha_1}(x_2)\psi_{\alpha_2'}(x_1) \dots \psi_{\alpha_n}(x_n) \nonumber \\
	&+&\sum_{\alpha_1'} \langle \alpha_1'|\hat{h}_0|\alpha_1\rangle
		\psi_{\alpha_1'}(x_2)\psi_{\alpha_2}(x_1) \dots \psi_{\alpha_n}(x_n) \nonumber \\
	&+& \dots \nonumber \\
	&+&\sum_{\alpha_n'} \langle \alpha_n'|\hat{h}_0|\alpha_n\rangle
		\psi_{\alpha_1}(x_2)\psi_{\alpha_1}(x_2) \dots \psi_{\alpha_n'}(x_n) \label{eq:2-27}
\end{align}

We can continue by computing all possible permutations. 
We rewrite also our Slater determinant in its second quantized form and skip the dependence on the quantum numbers $x_i.$
Summing up all contributions and taking care of all phases
$(-1)^p$ we arrive at 
\begin{align}
	\hat{H}_0|\alpha_1,\alpha_2,\dots, \alpha_n\rangle &=& \sum_{\alpha_1'}\langle \alpha_1'|\hat{h}_0|\alpha_1\rangle
		|\alpha_1'\alpha_2 \dots \alpha_{n}\rangle \nonumber \\
	&+& \sum_{\alpha_2'} \langle \alpha_2'|\hat{h}_0|\alpha_2\rangle
		|\alpha_1\alpha_2' \dots \alpha_{n}\rangle \nonumber \\
	&+& \dots \nonumber \\
	&+& \sum_{\alpha_n'} \langle \alpha_n'|\hat{h}_0|\alpha_n\rangle
		|\alpha_1\alpha_2 \dots \alpha_{n}'\rangle \label{eq:2-28}
\end{align}

In Eq.~(\ref{eq:2-28}) 
we have expressed the action of the one-body operator
of Eq.~(\ref{eq:2-23}) on the  $n$-body state in its second quantized form.
This equation can be further manipulated if we use the properties of the creation and annihilation operator
on each primed quantum number, that is
\begin{equation}
	|\alpha_1\alpha_2 \dots \alpha_k' \dots \alpha_{n}\rangle = 
		a_{\alpha_k'}^{\dagger}  a_{\alpha_k} |\alpha_1\alpha_2 \dots \alpha_k \dots \alpha_{n}\rangle \label{eq:2-29}
\end{equation}
Inserting this in the right-hand side of Eq.~(\ref{eq:2-28}) results in
\begin{align}
	\hat{H}_0|\alpha_1\alpha_2 \dots \alpha_{n}\rangle &=& \sum_{\alpha_1'}\langle \alpha_1'|\hat{h}_0|\alpha_1\rangle
		a_{\alpha_1'}^{\dagger}  a_{\alpha_1} |\alpha_1\alpha_2 \dots \alpha_{n}\rangle \nonumber \\
	&+& \sum_{\alpha_2'} \langle \alpha_2'|\hat{h}_0|\alpha_2\rangle
		a_{\alpha_2'}^{\dagger}  a_{\alpha_2} |\alpha_1\alpha_2 \dots \alpha_{n}\rangle \nonumber \\
	&+& \dots \nonumber \\
	&+& \sum_{\alpha_n'} \langle \alpha_n'|\hat{h}_0|\alpha_n\rangle
		a_{\alpha_n'}^{\dagger}  a_{\alpha_n} |\alpha_1\alpha_2 \dots \alpha_{n}\rangle \nonumber \\
	&=& \sum_{\alpha, \beta} \langle \alpha|\hat{h}_0|\beta\rangle a_\alpha^{\dagger} a_\beta 
		|\alpha_1\alpha_2 \dots \alpha_{n}\rangle \label{eq:2-30a}
\end{align}

In the number occupation representation or second quantization we get the following expression for a one-body 
operator which conserves the number of particles
\begin{equation}
	\hat{H}_0 = \sum_{\alpha\beta} \langle \alpha|\hat{h}_0|\beta\rangle a_\alpha^{\dagger} a_\beta \label{eq:2-30b}
\end{equation}
Obviously, $\hat{H}_0$ can be replaced by any other one-body  operator which preserved the number
of particles. The stucture of the operator is therefore not limited to say the kinetic or single-particle energy only.

The opearator $\hat{H}_0$ takes a particle from the single-particle state $\beta$  to the single-particle state $\alpha$ 
with a probability for the transition given by the expectation value $\langle \alpha|\hat{h}_0|\beta\rangle$.

It is instructive to verify Eq.~(\ref{eq:2-30b}) by computing the expectation value of $\hat{H}_0$ 
between two single-particle states
\begin{equation}
	\langle \alpha_1|\hat{h}_0|\alpha_2\rangle = \sum_{\alpha\beta} \langle \alpha|\hat{h}_0|\beta\rangle
		\langle 0|a_{\alpha_1}a_\alpha^{\dagger} a_\beta a_{\alpha_2}^{\dagger}|0\rangle \label{eq:2-30c}
\end{equation}

Using the commutation relations for the creation and annihilation operators we have 
\begin{equation}
a_{\alpha_1}a_\alpha^{\dagger} a_\beta a_{\alpha_2}^{\dagger} = (\delta_{\alpha \alpha_1} - a_\alpha^{\dagger} a_{\alpha_1} )(\delta_{\beta \alpha_2} - a_{\alpha_2}^{\dagger} a_{\beta} ), \label{eq:2-30d}
\end{equation}
which results in
\begin{equation}
\langle 0|a_{\alpha_1}a_\alpha^{\dagger} a_\beta a_{\alpha_2}^{\dagger}|0\rangle = \delta_{\alpha \alpha_1} \delta_{\beta \alpha_2} \label{eq:2-30e}
\end{equation}
and
\begin{equation}
\langle \alpha_1|\hat{h}_0|\alpha_2\rangle = \sum_{\alpha\beta} \langle \alpha|\hat{h}_0|\beta\rangle\delta_{\alpha \alpha_1} \delta_{\beta \alpha_2} = \langle \alpha_1|\hat{h}_0|\alpha_2\rangle \label{eq:2-30f}
\end{equation}

\subsection*{Two-body operators in second quantization}

Let us now derive the expression for our two-body interaction part, which also conserves the number of particles.
We can proceed in exactly the same way as for the one-body operator. In the coordinate representation our
two-body interaction part takes the following expression
\begin{equation}
	\hat{H}_I = \sum_{i < j} V(x_i,x_j) \label{eq:2-31}
\end{equation}
where the summation runs over distinct pairs. The term $V$ can be an interaction model for the nucleon-nucleon interaction
or the interaction between two electrons. It can also include additional two-body interaction terms. 

The action of this operator on a product of 
two single-particle functions is defined as 
\begin{equation}
	V(x_i,x_j) \psi_{\alpha_k}(x_i) \psi_{\alpha_l}(x_j) = \sum_{\alpha_k'\alpha_l'} 
		\psi_{\alpha_k}'(x_i)\psi_{\alpha_l}'(x_j) 
		\langle \alpha_k'\alpha_l'|\hat{v}|\alpha_k\alpha_l\rangle \label{eq:2-32}
\end{equation}

We can now let $\hat{H}_I$ act on all terms in the linear combination for $|\alpha_1\alpha_2\dots\alpha_n\rangle$. Without any permutations we have
\begin{align}
	&& \left( \sum_{i < j} V(x_i,x_j) \right) \psi_{\alpha_1}(x_1)\psi_{\alpha_2}(x_2)\dots \psi_{\alpha_n}(x_n) \nonumber \\
	&=& \sum_{\alpha_1'\alpha_2'} \langle \alpha_1'\alpha_2'|\hat{v}|\alpha_1\alpha_2\rangle
		\psi_{\alpha_1}'(x_1)\psi_{\alpha_2}'(x_2)\dots \psi_{\alpha_n}(x_n) \nonumber \\
	& +& \dots \nonumber \\
	&+& \sum_{\alpha_1'\alpha_n'} \langle \alpha_1'\alpha_n'|\hat{v}|\alpha_1\alpha_n\rangle
		\psi_{\alpha_1}'(x_1)\psi_{\alpha_2}(x_2)\dots \psi_{\alpha_n}'(x_n) \nonumber \\
	& +& \dots \nonumber \\
	&+& \sum_{\alpha_2'\alpha_n'} \langle \alpha_2'\alpha_n'|\hat{v}|\alpha_2\alpha_n\rangle
		\psi_{\alpha_1}(x_1)\psi_{\alpha_2}'(x_2)\dots \psi_{\alpha_n}'(x_n) \nonumber \\
	 & +& \dots \label{eq:2-33}
\end{align}
where on the rhs we have a term for each distinct pairs. 

For the other terms on the rhs we obtain similar expressions  and summing over all terms we obtain
\begin{align}
	H_I |\alpha_1\alpha_2\dots\alpha_n\rangle &=& \sum_{\alpha_1', \alpha_2'} \langle \alpha_1'\alpha_2'|\hat{v}|\alpha_1\alpha_2\rangle
		|\alpha_1'\alpha_2'\dots\alpha_n\rangle \nonumber \\
	&+& \dots \nonumber \\
	&+& \sum_{\alpha_1', \alpha_n'} \langle \alpha_1'\alpha_n'|\hat{v}|\alpha_1\alpha_n\rangle
		|\alpha_1'\alpha_2\dots\alpha_n'\rangle \nonumber \\
	&+& \dots \nonumber \\
	&+& \sum_{\alpha_2', \alpha_n'} \langle \alpha_2'\alpha_n'|\hat{v}|\alpha_2\alpha_n\rangle
		|\alpha_1\alpha_2'\dots\alpha_n'\rangle \nonumber \\
	 &+& \dots \label{eq:2-34}
\end{align}

We introduce second quantization via the relation
\begin{align}
	&& a_{\alpha_k'}^{\dagger} a_{\alpha_l'}^{\dagger} a_{\alpha_l} a_{\alpha_k} 
		|\alpha_1\alpha_2\dots\alpha_k\dots\alpha_l\dots\alpha_n\rangle \nonumber \\
	&=& (-1)^{k-1} (-1)^{l-2} a_{\alpha_k'}^{\dagger} a_{\alpha_l'}^{\dagger} a_{\alpha_l} a_{\alpha_k}
		|\alpha_k\alpha_l \underbrace{\alpha_1\alpha_2\dots\alpha_n}_{\neq \alpha_k,\alpha_l}\rangle \nonumber \\
	&=& (-1)^{k-1} (-1)^{l-2} 
	|\alpha_k'\alpha_l' \underbrace{\alpha_1\alpha_2\dots\alpha_n}_{\neq \alpha_k',\alpha_l'}\rangle \nonumber \\
	&=& |\alpha_1\alpha_2\dots\alpha_k'\dots\alpha_l'\dots\alpha_n\rangle \label{eq:2-35}
\end{align}

Inserting this in (\ref{eq:2-34}) gives
\begin{align}
	H_I |\alpha_1\alpha_2\dots\alpha_n\rangle
	&=& \sum_{\alpha_1', \alpha_2'} \langle \alpha_1'\alpha_2'|\hat{v}|\alpha_1\alpha_2\rangle
		a_{\alpha_1'}^{\dagger} a_{\alpha_2'}^{\dagger} a_{\alpha_2} a_{\alpha_1}
		|\alpha_1\alpha_2\dots\alpha_n\rangle \nonumber \\
	&+& \dots \nonumber \\
	&=& \sum_{\alpha_1', \alpha_n'} \langle \alpha_1'\alpha_n'|\hat{v}|\alpha_1\alpha_n\rangle
		a_{\alpha_1'}^{\dagger} a_{\alpha_n'}^{\dagger} a_{\alpha_n} a_{\alpha_1}
		|\alpha_1\alpha_2\dots\alpha_n\rangle \nonumber \\
	&+& \dots \nonumber \\
	&=& \sum_{\alpha_2', \alpha_n'} \langle \alpha_2'\alpha_n'|\hat{v}|\alpha_2\alpha_n\rangle
		a_{\alpha_2'}^{\dagger} a_{\alpha_n'}^{\dagger} a_{\alpha_n} a_{\alpha_2}
		|\alpha_1\alpha_2\dots\alpha_n\rangle \nonumber \\
	&+& \dots \nonumber \\
	&=& \sum_{\alpha, \beta, \gamma, \delta} ' \langle \alpha\beta|\hat{v}|\gamma\delta\rangle
		a^{\dagger}_\alpha a^{\dagger}_\beta a_\delta a_\gamma
		|\alpha_1\alpha_2\dots\alpha_n\rangle \label{eq:2-36}
\end{align}

Here we let $\sum'$ indicate that the sums running over $\alpha$ and $\beta$ run over all
single-particle states, while the summations  $\gamma$ and $\delta$ 
run over all pairs of single-particle states. We wish to remove this restriction and since
\begin{equation}
	\langle \alpha\beta|\hat{v}|\gamma\delta\rangle = \langle \beta\alpha|\hat{v}|\delta\gamma\rangle \label{eq:2-37}
\end{equation}
we get
\begin{align}
	\sum_{\alpha\beta} \langle \alpha\beta|\hat{v}|\gamma\delta\rangle a^{\dagger}_\alpha a^{\dagger}_\beta a_\delta a_\gamma &=& 
		\sum_{\alpha\beta} \langle \beta\alpha|\hat{v}|\delta\gamma\rangle 
		a^{\dagger}_\alpha a^{\dagger}_\beta a_\delta a_\gamma \label{eq:2-38a} \\
	&=& \sum_{\alpha\beta}\langle \beta\alpha|\hat{v}|\delta\gamma\rangle
		a^{\dagger}_\beta a^{\dagger}_\alpha a_\gamma a_\delta \label{eq:2-38b}
\end{align}
where we  have used the anti-commutation rules.

Changing the summation indices 
$\alpha$ and $\beta$ in (\ref{eq:2-38b}) we obtain
\begin{equation}
	\sum_{\alpha\beta} \langle \alpha\beta|\hat{v}|\gamma\delta\rangle a^{\dagger}_\alpha a^{\dagger}_\beta a_\delta a_\gamma =
		 \sum_{\alpha\beta} \langle \alpha\beta|\hat{v}|\delta\gamma\rangle 
		  a^{\dagger}_\alpha a^{\dagger}_\beta  a_\gamma a_\delta \label{eq:2-38c}
\end{equation}
From this it follows that the restriction on the summation over $\gamma$ and $\delta$ can be removed if we multiply with a factor $\frac{1}{2}$, resulting in 
\begin{equation}
	\hat{H}_I = \frac{1}{2} \sum_{\alpha\beta\gamma\delta} \langle \alpha\beta|\hat{v}|\gamma\delta\rangle
		a^{\dagger}_\alpha a^{\dagger}_\beta a_\delta a_\gamma \label{eq:2-39}
\end{equation}
where we sum freely over all single-particle states $\alpha$, 
$\beta$, $\gamma$ og $\delta$.

With this expression we can now verify that the second quantization form of $\hat{H}_I$ in Eq.~(\ref{eq:2-39}) 
results in the same matrix between two anti-symmetrized two-particle states as its corresponding coordinate
space representation. We have  
\begin{equation}
	\langle \alpha_1 \alpha_2|\hat{H}_I|\beta_1 \beta_2\rangle =
		\frac{1}{2} \sum_{\alpha\beta\gamma\delta}
			\langle \alpha\beta|\hat{v}|\gamma\delta\rangle \langle 0|a_{\alpha_2} a_{\alpha_1} 
			 a^{\dagger}_\alpha a^{\dagger}_\beta a_\delta a_\gamma 
			 a_{\beta_1}^{\dagger} a_{\beta_2}^{\dagger}|0\rangle. \label{eq:2-40}
\end{equation}

Using the commutation relations we get 
\begin{align}
	&& a_{\alpha_2} a_{\alpha_1}a^{\dagger}_\alpha a^{\dagger}_\beta 
		a_\delta a_\gamma a_{\beta_1}^{\dagger} a_{\beta_2}^{\dagger} \nonumber \\
	&=& a_{\alpha_2} a_{\alpha_1}a^{\dagger}_\alpha a^{\dagger}_\beta 
		( a_\delta \delta_{\gamma \beta_1} a_{\beta_2}^{\dagger} - 
		a_\delta  a_{\beta_1}^{\dagger} a_\gamma a_{\beta_2}^{\dagger} ) \nonumber \\
	&=& a_{\alpha_2} a_{\alpha_1}a^{\dagger}_\alpha a^{\dagger}_\beta 
		(\delta_{\gamma \beta_1} \delta_{\delta \beta_2} - \delta_{\gamma \beta_1} a_{\beta_2}^{\dagger} a_\delta -
		a_\delta a_{\beta_1}^{\dagger}\delta_{\gamma \beta_2} +
		a_\delta a_{\beta_1}^{\dagger} a_{\beta_2}^{\dagger} a_\gamma ) \nonumber \\
	&=& a_{\alpha_2} a_{\alpha_1}a^{\dagger}_\alpha a^{\dagger}_\beta 
		(\delta_{\gamma \beta_1} \delta_{\delta \beta_2} - \delta_{\gamma \beta_1} a_{\beta_2}^{\dagger} a_\delta \nonumber \\
		&& \qquad - \delta_{\delta \beta_1} \delta_{\gamma \beta_2} + \delta_{\gamma \beta_2} a_{\beta_1}^{\dagger} a_\delta
		+ a_\delta a_{\beta_1}^{\dagger} a_{\beta_2}^{\dagger} a_\gamma ) \label{eq:2-41}
\end{align}

The vacuum expectation value of this product of operators becomes
\begin{align}
	&& \langle 0|a_{\alpha_2} a_{\alpha_1} a^{\dagger}_\alpha a^{\dagger}_\beta a_\delta a_\gamma 
		a_{\beta_1}^{\dagger} a_{\beta_2}^{\dagger}|0\rangle \nonumber \\
	&=& (\delta_{\gamma \beta_1} \delta_{\delta \beta_2} -
		\delta_{\delta \beta_1} \delta_{\gamma \beta_2} ) 
		\langle 0|a_{\alpha_2} a_{\alpha_1}a^{\dagger}_\alpha a^{\dagger}_\beta|0\rangle \nonumber \\
	&=& (\delta_{\gamma \beta_1} \delta_{\delta \beta_2} -\delta_{\delta \beta_1} \delta_{\gamma \beta_2} )
	(\delta_{\alpha \alpha_1} \delta_{\beta \alpha_2} -\delta_{\beta \alpha_1} \delta_{\alpha \alpha_2} ) \label{eq:2-42b}
\end{align}

Insertion of 
Eq.~(\ref{eq:2-42b}) in Eq.~(\ref{eq:2-40}) results in
\begin{align}
	\langle \alpha_1\alpha_2|\hat{H}_I|\beta_1\beta_2\rangle &=& \frac{1}{2} \big[ 
		\langle \alpha_1\alpha_2|\hat{v}|\beta_1\beta_2\rangle- \langle \alpha_1\alpha_2|\hat{v}|\beta_2\beta_1\rangle \nonumber \\
		&& - \langle \alpha_2\alpha_1|\hat{v}|\beta_1\beta_2\rangle + \langle \alpha_2\alpha_1|\hat{v}|\beta_2\beta_1\rangle \big] \nonumber \\
	&=& \langle \alpha_1\alpha_2|\hat{v}|\beta_1\beta_2\rangle - \langle \alpha_1\alpha_2|\hat{v}|\beta_2\beta_1\rangle \nonumber \\
	&=& \langle \alpha_1\alpha_2|\hat{v}|\beta_1\beta_2\rangle_{\mathrm{AS}}. \label{eq:2-43b}
\end{align}

The two-body operator can also be expressed in terms of the anti-symmetrized matrix elements we discussed previously as
\begin{align}
	\hat{H}_I &=& \frac{1}{2} \sum_{\alpha\beta\gamma\delta}  \langle \alpha \beta|\hat{v}|\gamma \delta\rangle
		a_\alpha^{\dagger} a_\beta^{\dagger} a_\delta a_\gamma \nonumber \\
	&=& \frac{1}{4} \sum_{\alpha\beta\gamma\delta} \left[ \langle \alpha \beta|\hat{v}|\gamma \delta\rangle -
		\langle \alpha \beta|\hat{v}|\delta\gamma \rangle \right] 
		a_\alpha^{\dagger} a_\beta^{\dagger} a_\delta a_\gamma \nonumber \\
	&=& \frac{1}{4} \sum_{\alpha\beta\gamma\delta} \langle \alpha \beta|\hat{v}|\gamma \delta\rangle_{\mathrm{AS}}
		a_\alpha^{\dagger} a_\beta^{\dagger} a_\delta a_\gamma \label{eq:2-45}
\end{align}

The factors in front of the operator, either  $\frac{1}{4}$ or 
$\frac{1}{2}$ tells whether we use antisymmetrized matrix elements or not. 

We can now express the Hamiltonian operator for a many-fermion system  in the occupation basis representation
as  
\begin{equation}
	H = \sum_{\alpha, \beta} \langle \alpha|\hat{t}+\hat{u}_{\mathrm{ext}}|\beta\rangle a_\alpha^{\dagger} a_\beta + \frac{1}{4} \sum_{\alpha\beta\gamma\delta}
		\langle \alpha \beta|\hat{v}|\gamma \delta\rangle a_\alpha^{\dagger} a_\beta^{\dagger} a_\delta a_\gamma. \label{eq:2-46b}
\end{equation}
This is the form we will use in the rest of these lectures, assuming that we work with anti-symmetrized two-body matrix elements.

\subsection*{Particle-hole formalism}

Second quantization is a useful and elegant formalism  for constructing many-body  states and 
quantum mechanical operators. One can express and translate many physical processes
into simple pictures such as Feynman diagrams. Expecation values of many-body states are also easily calculated.
However, although the equations are seemingly easy to set up, from  a practical point of view, that is
the solution of Schroedinger's equation, there is no particular gain.
The many-body equation is equally hard to solve, irrespective of representation. 
The cliche that 
there is no free lunch brings us down to earth again.  
Note however that a transformation to a particular
basis, for cases where the interaction obeys specific symmetries, can ease the solution of Schroedinger's equation. 

But there is at least one important case where second quantization comes to our rescue.
It is namely easy to introduce another reference state than the pure vacuum $|0\rangle $, where all single-particle states are active.
With many particles present it is often useful to introduce another reference state  than the vacuum state$|0\rangle $. We will label this state $|c\rangle$ ($c$ for core) and as we will see it can reduce 
considerably the complexity and thereby the dimensionality of the many-body problem. It allows us to sum up to infinite order specific many-body correlations.  The particle-hole representation is one of these handy representations. 

In the original particle representation these states are products of the creation operators  $a_{\alpha_i}^\dagger$ acting on the true vacuum $|0\rangle $.
Following Eq.~(\ref{eq:2-2}) we have 
\begin{align}
 |\alpha_1\alpha_2\dots\alpha_{n-1}\alpha_n\rangle &=& a_{\alpha_1}^\dagger a_{\alpha_2}^\dagger \dots
					a_{\alpha_{n-1}}^\dagger a_{\alpha_n}^\dagger |0\rangle  \label{eq:2-47a} \\
	|\alpha_1\alpha_2\dots\alpha_{n-1}\alpha_n\alpha_{n+1}\rangle &=&
		a_{\alpha_1}^\dagger a_{\alpha_2}^\dagger \dots a_{\alpha_{n-1}}^\dagger a_{\alpha_n}^\dagger
		a_{\alpha_{n+1}}^\dagger |0\rangle  \label{eq:2-47b} \\
	|\alpha_1\alpha_2\dots\alpha_{n-1}\rangle &=& a_{\alpha_1}^\dagger a_{\alpha_2}^\dagger \dots
		a_{\alpha_{n-1}}^\dagger |0\rangle  \label{eq:2-47c}
\end{align}

If we use Eq.~(\ref{eq:2-47a}) as our new reference state, we can simplify considerably the representation of 
this state
\begin{equation}
	|c\rangle  \equiv |\alpha_1\alpha_2\dots\alpha_{n-1}\alpha_n\rangle =
		a_{\alpha_1}^\dagger a_{\alpha_2}^\dagger \dots a_{\alpha_{n-1}}^\dagger a_{\alpha_n}^\dagger |0\rangle  \label{eq:2-48a}
\end{equation}
The new reference states for the $n+1$ and $n-1$ states can then be written as
\begin{align}
	|\alpha_1\alpha_2\dots\alpha_{n-1}\alpha_n\alpha_{n+1}\rangle &=& (-1)^n a_{\alpha_{n+1}}^\dagger |c\rangle 
		\equiv (-1)^n |\alpha_{n+1}\rangle_c \label{eq:2-48b} \\
	|\alpha_1\alpha_2\dots\alpha_{n-1}\rangle &=& (-1)^{n-1} a_{\alpha_n} |c\rangle  
		\equiv (-1)^{n-1} |\alpha_{n-1}\rangle_c \label{eq:2-48c} 
\end{align}

The first state has one additional particle with respect to the new vacuum state
$|c\rangle $  and is normally referred to as a one-particle state or one particle added to the 
many-body reference state. 
The second state has one particle less than the reference vacuum state  $|c\rangle $ and is referred to as
a one-hole state. 
When dealing with a new reference state it is often convenient to introduce 
new creation and annihilation operators since we have 
from Eq.~(\ref{eq:2-48c})
\begin{equation}
	a_\alpha |c\rangle  \neq 0 \label{eq:2-49}
\end{equation}
since  $\alpha$ is contained  in $|c\rangle $, while for the true vacuum we have 
$a_\alpha |0\rangle  = 0$ for all $\alpha$.

The new reference state leads to the definition of new creation and annihilation operators
which satisfy the following relations
\begin{align}
	b_\alpha |c\rangle  &=& 0 \label{eq:2-50a} \\
	\{b_\alpha^\dagger , b_\beta^\dagger \} = \{b_\alpha , b_\beta \} &=& 0 \nonumber  \\
	\{b_\alpha^\dagger , b_\beta \} &=& \delta_{\alpha \beta} \label{eq:2-50c}
\end{align}
We assume also that the new reference state is properly normalized
\begin{equation}
	\langle c | c \rangle = 1 \label{eq:2-51}
\end{equation}

The physical interpretation of these new operators is that of so-called quasiparticle states.
This means that a state defined by the addition of one extra particle to a reference state $|c\rangle $ may not necesseraly be interpreted as one particle coupled to a core.
We define now new creation operators that act on a state $\alpha$ creating a new quasiparticle state
\begin{equation}
	b_\alpha^\dagger|c\rangle  = \Bigg\{ \begin{array}{ll}
		a_\alpha^\dagger |c\rangle  = |\alpha\rangle, & \alpha > F \\
		\\
		a_\alpha |c\rangle  = |\alpha^{-1}\rangle, & \alpha \leq F
	\end{array} \label{eq:2-52}
\end{equation}
where $F$ is the Fermi level representing the last  occupied single-particle orbit 
of the new reference state $|c\rangle $. 

The annihilation is the hermitian conjugate of the creation operator
\[
	b_\alpha = (b_\alpha^\dagger)^\dagger,
\]
resulting in
\begin{equation}
	b_\alpha^\dagger = \Bigg\{ \begin{array}{ll}
		a_\alpha^\dagger & \alpha > F \\
		\\
		a_\alpha & \alpha \leq F
	\end{array} \qquad 
	b_\alpha = \Bigg\{ \begin{array}{ll}
		a_\alpha & \alpha > F \\
		\\
		 a_\alpha^\dagger & \alpha \leq F
	\end{array} \label{eq:2-54}
\end{equation}

With the new creation and annihilation operator  we can now construct 
many-body quasiparticle states, with one-particle-one-hole states, two-particle-two-hole
states etc in the same fashion as we previously constructed many-particle states. 
We can write a general particle-hole state as
\begin{equation}
	|\beta_1\beta_2\dots \beta_{n_p} \gamma_1^{-1} \gamma_2^{-1} \dots \gamma_{n_h}^{-1}\rangle \equiv
		\underbrace{b_{\beta_1}^\dagger b_{\beta_2}^\dagger \dots b_{\beta_{n_p}}^\dagger}_{>F}
		\underbrace{b_{\gamma_1}^\dagger b_{\gamma_2}^\dagger \dots b_{\gamma_{n_h}}^\dagger}_{\leq F} |c\rangle \label{eq:2-56}
\end{equation}
We can now rewrite our one-body and two-body operators in terms of the new creation and annihilation operators.
The number operator becomes
\begin{equation}
	\hat{N} = \sum_\alpha a_\alpha^\dagger a_\alpha= 
\sum_{\alpha > F} b_\alpha^\dagger b_\alpha + n_c - \sum_{\alpha \leq F} b_\alpha^\dagger b_\alpha \label{eq:2-57b}
\end{equation}
where $n_c$ is the number of particle in the new vacuum state $|c\rangle $.  
The action of $\hat{N}$ on a many-body state results in 
\begin{equation}
	N |\beta_1\beta_2\dots \beta_{n_p} \gamma_1^{-1} \gamma_2^{-1} \dots \gamma_{n_h}^{-1}\rangle = (n_p + n_c - n_h) |\beta_1\beta_2\dots \beta_{n_p} \gamma_1^{-1} \gamma_2^{-1} \dots \gamma_{n_h}^{-1}\rangle \label{2-59}
\end{equation}
Here  $n=n_p +n_c - n_h$ is the total number of particles in the quasi-particle state of 
Eq.~(\ref{eq:2-56}). Note that  $\hat{N}$ counts the total number of particles  present 
\begin{equation}
	N_{qp} = \sum_\alpha b_\alpha^\dagger b_\alpha, \label{eq:2-60}
\end{equation}
gives us the number of quasi-particles as can be seen by computing
\begin{equation}
	N_{qp}= |\beta_1\beta_2\dots \beta_{n_p} \gamma_1^{-1} \gamma_2^{-1} \dots \gamma_{n_h}^{-1}\rangle
		= (n_p + n_h)|\beta_1\beta_2\dots \beta_{n_p} \gamma_1^{-1} \gamma_2^{-1} \dots \gamma_{n_h}^{-1}\rangle \label{eq:2-61}
\end{equation}
where $n_{qp} = n_p + n_h$ is the total number of quasi-particles.

We express the one-body operator $\hat{H}_0$ in terms of the quasi-particle creation and annihilation operators, resulting in
\begin{align}
	\hat{H}_0 &=& \sum_{\alpha\beta > F} \langle \alpha|\hat{h}_0|\beta\rangle  b_\alpha^\dagger b_\beta +
		\sum_{\alpha > F, \beta \leq F } \left[\langle \alpha|\hat{h}_0|\beta\rangle b_\alpha^\dagger b_\beta^\dagger + \langle \beta|\hat{h}_0|\alpha\rangle b_\beta  b_\alpha \right] \nonumber \\
	&+& \sum_{\alpha \leq F} \langle \alpha|\hat{h}_0|\alpha\rangle - \sum_{\alpha\beta \leq F} \langle \beta|\hat{h}_0|\alpha\rangle b_\alpha^\dagger b_\beta \label{eq:2-63b}
\end{align}
The first term  gives contribution only for particle states, while the last one
contributes only for holestates. The second term can create or destroy a set of
quasi-particles and 
the third term is the contribution  from the vacuum state $|c\rangle$.

Before we continue with the expressions for the two-body operator, we introduce a nomenclature we will use for the rest of this
text. It is inspired by the notation used in quantum chemistry.
We reserve the labels $i,j,k,\dots$ for hole states and $a,b,c,\dots$ for states above $F$, viz.~particle states.
This means also that we will skip the constraint $\leq F$ or $> F$ in the summation symbols. 
Our operator $\hat{H}_0$  reads now 
\begin{align}
	\hat{H}_0 &=& \sum_{ab} \langle a|\hat{h}|b\rangle b_a^\dagger b_b +
		\sum_{ai} \left[
		\langle a|\hat{h}|i\rangle b_a^\dagger b_i^\dagger + 
		\langle i|\hat{h}|a\rangle b_i  b_a \right] \nonumber \\
	&+& \sum_{i} \langle i|\hat{h}|i\rangle - 
		\sum_{ij} \langle j|\hat{h}|i\rangle
		b_i^\dagger b_j \label{eq:2-63c}
\end{align} 

The two-particle operator in the particle-hole formalism  is more complicated since we have
to translate four indices $\alpha\beta\gamma\delta$ to the possible combinations of particle and hole
states.  When performing the commutator algebra we can regroup the operator in five different terms
\begin{equation}
	\hat{H}_I = \hat{H}_I^{(a)} + \hat{H}_I^{(b)} + \hat{H}_I^{(c)} + \hat{H}_I^{(d)} + \hat{H}_I^{(e)} \label{eq:2-65}
\end{equation}
Using anti-symmetrized  matrix elements, 
bthe term  $\hat{H}_I^{(a)}$ is  
\begin{equation}
	\hat{H}_I^{(a)} = \frac{1}{4}
	\sum_{abcd} \langle ab|\hat{V}|cd\rangle 
		b_a^\dagger b_b^\dagger b_d b_c \label{eq:2-66}
\end{equation}

The next term $\hat{H}_I^{(b)}$  reads
\begin{equation}
	 \hat{H}_I^{(b)} = \frac{1}{4} \sum_{abci}\left(\langle ab|\hat{V}|ci\rangle b_a^\dagger b_b^\dagger b_i^\dagger b_c +\langle ai|\hat{V}|cb\rangle b_a^\dagger b_i b_b b_c\right) \label{eq:2-67b}
\end{equation}
This term conserves the number of quasiparticles but creates or removes a 
three-particle-one-hole  state. 
For $\hat{H}_I^{(c)}$  we have
\begin{align}
	\hat{H}_I^{(c)}& =& \frac{1}{4}
		\sum_{abij}\left(\langle ab|\hat{V}|ij\rangle b_a^\dagger b_b^\dagger b_j^\dagger b_i^\dagger +
		\langle ij|\hat{V}|ab\rangle b_a  b_b b_j b_i \right)+  \nonumber \\
	&&	\frac{1}{2}\sum_{abij}\langle ai|\hat{V}|bj\rangle b_a^\dagger b_j^\dagger b_b b_i + 
		\frac{1}{2}\sum_{abi}\langle ai|\hat{V}|bi\rangle b_a^\dagger b_b. \label{eq:2-68c}
\end{align}

The first line stands for the creation of a two-particle-two-hole state, while the second line represents
the creation to two one-particle-one-hole pairs
while the last term represents a contribution to the particle single-particle energy
from the hole states, that is an interaction between the particle states and the hole states
within the new vacuum  state.
The fourth term reads
\begin{align}
	 \hat{H}_I^{(d)}& = &\frac{1}{4} 
	 	\sum_{aijk}\left(\langle ai|\hat{V}|jk\rangle b_a^\dagger b_k^\dagger b_j^\dagger b_i+
\langle ji|\hat{V}|ak\rangle b_k^\dagger b_j b_i b_a\right)+\nonumber \\
&&\frac{1}{4}\sum_{aij}\left(\langle ai|\hat{V}|ji\rangle b_a^\dagger b_j^\dagger+
\langle ji|\hat{V}|ai\rangle - \langle ji|\hat{V}|ia\rangle b_j b_a \right). \label{eq:2-69d} 
\end{align}
The terms in the first line  stand for the creation of a particle-hole state 
interacting with hole states, we will label this as a two-hole-one-particle contribution. 
The remaining terms are a particle-hole state interacting with the holes in the vacuum state. 
Finally we have 
\begin{equation}
	\hat{H}_I^{(e)} = \frac{1}{4}
		 \sum_{ijkl}
		 \langle kl|\hat{V}|ij\rangle b_i^\dagger b_j^\dagger b_l b_k+
	        \frac{1}{2}\sum_{ijk}\langle ij|\hat{V}|kj\rangle b_k^\dagger b_i
	        +\frac{1}{2}\sum_{ij}\langle ij|\hat{V}|ij\rangle \label{eq:2-70d}
\end{equation}
The first terms represents the 
interaction between two holes while the second stands for the interaction between a hole and the remaining holes in the vacuum state.
It represents a contribution to single-hole energy  to first order.
The last term collects all contributions to the energy of the ground state of a closed-shell system arising
from hole-hole correlations.

\subsection*{Summarizing and defining a normal-ordered Hamiltonian}

\[
  \Phi_{AS}(\alpha_1, \dots, \alpha_A; x_1, \dots x_A)=
            \frac{1}{\sqrt{A}} \sum_{\hat{P}} (-1)^P \hat{P} \prod_{i=1}^A \psi_{\alpha_i}(x_i),
\]
which is equivalent with $|\alpha_1 \dots \alpha_A\rangle= a_{\alpha_1}^{\dagger} \dots a_{\alpha_A}^{\dagger} |0\rangle$. We have also
    \[
        a_p^\dagger|0\rangle = |p\rangle, \quad a_p |q\rangle = \delta_{pq}|0\rangle
    \]
\[
  \delta_{pq} = \left\{a_p, a_q^\dagger \right\},
\]
and 
\[
0 = \left\{a_p^\dagger, a_q \right\} = \left\{a_p, a_q \right\} = \left\{a_p^\dagger, a_q^\dagger \right\}
\]
\[
|\Phi_0\rangle = |\alpha_1 \dots \alpha_A\rangle, \quad \alpha_1, \dots, \alpha_A \leq \alpha_F
\]

\[
\left\{a_p^\dagger, a_q \right\}= \delta_{pq}, p, q \leq \alpha_F 
\]
\[
\left\{a_p, a_q^\dagger \right\} = \delta_{pq}, p, q > \alpha_F
\]
with         $i,j,\ldots \leq \alpha_F, \quad a,b,\ldots > \alpha_F, \quad p,q, \ldots - \textrm{any}$
\[
        a_i|\Phi_0\rangle = |\Phi_i\rangle, \hspace{0.5cm} a_a^\dagger|\Phi_0\rangle = |\Phi^a\rangle
\]
and         
\[
a_i^\dagger|\Phi_0\rangle = 0 \hspace{0.5cm}  a_a|\Phi_0\rangle = 0
\]

The one-body operator is defined as
\[
 \hat{F} = \sum_{pq} \langle p|\hat{f}|q\rangle a_p^\dagger a_q
\]
while the two-body opreator is defined as
\[
\hat{V} = \frac{1}{4} \sum_{pqrs} \langle pq|\hat{v}|rs\rangle_{AS} a_p^\dagger a_q^\dagger a_s a_r
\]
where we have defined the antisymmetric matrix elements
\[
\langle pq|\hat{v}|rs\rangle_{AS} = \langle pq|\hat{v}|rs\rangle - \langle pq|\hat{v}|sr\rangle.
\]

We can also define a three-body operator
\[
\hat{V}_3 = \frac{1}{36} \sum_{pqrstu} \langle pqr|\hat{v}_3|stu\rangle_{AS} 
                a_p^\dagger a_q^\dagger a_r^\dagger a_u a_t a_s
\]
with the antisymmetrized matrix element
\begin{align}
            \langle pqr|\hat{v}_3|stu\rangle_{AS} = \langle pqr|\hat{v}_3|stu\rangle + \langle pqr|\hat{v}_3|tus\rangle + \langle pqr|\hat{v}_3|ust\rangle- \langle pqr|\hat{v}_3|sut\rangle - \langle pqr|\hat{v}_3|tsu\rangle - \langle pqr|\hat{v}_3|uts\rangle.
\end{align}

\subsection*{Operators in second quantization}

In the build-up of a shell-model or FCI code that is meant to tackle large dimensionalities
is the action of the Hamiltonian $\hat{H}$ on a
Slater determinant represented in second quantization as
\[
 |\alpha_1\dots \alpha_n\rangle = a_{\alpha_1}^{\dagger} a_{\alpha_2}^{\dagger} \dots a_{\alpha_n}^{\dagger} |0\rangle.
\]
The time consuming part stems from the action of the Hamiltonian
on the above determinant,
\[
\left(\sum_{\alpha\beta} \langle \alpha|t+u|\beta\rangle a_\alpha^{\dagger} a_\beta + \frac{1}{4} \sum_{\alpha\beta\gamma\delta}
                \langle \alpha \beta|\hat{v}|\gamma \delta\rangle a_\alpha^{\dagger} a_\beta^{\dagger} a_\delta a_\gamma\right)a_{\alpha_1}^{\dagger} a_{\alpha_2}^{\dagger} \dots a_{\alpha_n}^{\dagger} |0\rangle.
\]
A practically useful way to implement this action is to encode a Slater determinant as a bit pattern.

Assume that we have at our disposal $n$ different single-particle orbits
$\alpha_0,\alpha_2,\dots,\alpha_{n-1}$ and that we can distribute  among these orbits $N\le n$ particles.

A Slater  determinant can then be coded as an integer of $n$ bits. As an example, if we have $n=16$ single-particle states
$\alpha_0,\alpha_1,\dots,\alpha_{15}$ and $N=4$ fermions occupying the states $\alpha_3$, $\alpha_6$, $\alpha_{10}$ and $\alpha_{13}$
we could write this Slater determinant as  
\[
\Phi_{\Lambda} = a_{\alpha_3}^{\dagger} a_{\alpha_6}^{\dagger} a_{\alpha_{10}}^{\dagger} a_{\alpha_{13}}^{\dagger} |0\rangle.
\]
The unoccupied single-particle states have bit value $0$ while the occupied ones are represented by bit state $1$. 
In the binary notation we would write this   16 bits long integer as
\[
\begin{array}{cccccccccccccccc}
{\alpha_0}&{\alpha_1}&{\alpha_2}&{\alpha_3}&{\alpha_4}&{\alpha_5}&{\alpha_6}&{\alpha_7} & {\alpha_8} &{\alpha_9} & {\alpha_{10}} &{\alpha_{11}} &{\alpha_{12}} &{\alpha_{13}} &{\alpha_{14}} & {\alpha_{15}} \\
{0} & {0} &{0} &{1} &{0} &{0} &{1} &{0} &{0} &{0} &{1} &{0} &{0} &{1} &{0} & {0} \\
\end{array}
\]
which translates into the decimal number
\[
2^3+2^6+2^{10}+2^{13}=9288.
\]
We can thus encode a Slater determinant as a bit pattern.

With $N$ particles that can be distributed over $n$ single-particle states, the total number of Slater determinats (and defining thereby the dimensionality of the system) is
\[
\mathrm{dim}(\mathcal{H}) = \left(\begin{array}{c} n \\N\end{array}\right).
\]
The total number of bit patterns is $2^n$. 

We assume again that we have at our disposal $n$ different single-particle orbits
$\alpha_0,\alpha_2,\dots,\alpha_{n-1}$ and that we can distribute  among these orbits $N\le n$ particles.
The ordering among these states is important as it defines the order of the creation operators.
We will write the determinant 
\[
\Phi_{\Lambda} = a_{\alpha_3}^{\dagger} a_{\alpha_6}^{\dagger} a_{\alpha_{10}}^{\dagger} a_{\alpha_{13}}^{\dagger} |0\rangle,
\]
in a more compact way as 
\[
\Phi_{3,6,10,13} = |0001001000100100\rangle.
\]
The action of a creation operator is thus
\[
a^{\dagger}_{\alpha_4}\Phi_{3,6,10,13} = a^{\dagger}_{\alpha_4}|0001001000100100\rangle=a^{\dagger}_{\alpha_4}a_{\alpha_3}^{\dagger} a_{\alpha_6}^{\dagger} a_{\alpha_{10}}^{\dagger} a_{\alpha_{13}}^{\dagger} |0\rangle,
\]
which becomes
\[
-a_{\alpha_3}^{\dagger} a^{\dagger}_{\alpha_4} a_{\alpha_6}^{\dagger} a_{\alpha_{10}}^{\dagger} a_{\alpha_{13}}^{\dagger} |0\rangle=-|0001101000100100\rangle.
\]

Similarly
\[
a^{\dagger}_{\alpha_6}\Phi_{3,6,10,13} = a^{\dagger}_{\alpha_6}|0001001000100100\rangle=a^{\dagger}_{\alpha_6}a_{\alpha_3}^{\dagger} a_{\alpha_6}^{\dagger} a_{\alpha_{10}}^{\dagger} a_{\alpha_{13}}^{\dagger} |0\rangle,
\]
which becomes
\[
-a^{\dagger}_{\alpha_4} (a_{\alpha_6}^{\dagger})^ 2 a_{\alpha_{10}}^{\dagger} a_{\alpha_{13}}^{\dagger} |0\rangle=0!
\]
This gives a simple recipe:  
\begin{itemize}
\item If one of the bits $b_j$ is $1$ and we act with a creation operator on this bit, we return a null vector

\item If $b_j=0$, we set it to $1$ and return a sign factor $(-1)^l$, where $l$ is the number of bits set before bit $j$.
\end{itemize}

\noindent
Consider the action of $a^{\dagger}_{\alpha_2}$ on various slater determinants:
\[
\begin{array}{ccc}
a^{\dagger}_{\alpha_2}\Phi_{00111}& = a^{\dagger}_{\alpha_2}|00111\rangle&=0\times |00111\rangle\\
a^{\dagger}_{\alpha_2}\Phi_{01011}& = a^{\dagger}_{\alpha_2}|01011\rangle&=(-1)\times |01111\rangle\\
a^{\dagger}_{\alpha_2}\Phi_{01101}& = a^{\dagger}_{\alpha_2}|01101\rangle&=0\times |01101\rangle\\
a^{\dagger}_{\alpha_2}\Phi_{01110}& = a^{\dagger}_{\alpha_2}|01110\rangle&=0\times |01110\rangle\\
a^{\dagger}_{\alpha_2}\Phi_{10011}& = a^{\dagger}_{\alpha_2}|10011\rangle&=(-1)\times |10111\rangle\\
a^{\dagger}_{\alpha_2}\Phi_{10101}& = a^{\dagger}_{\alpha_2}|10101\rangle&=0\times |10101\rangle\\
a^{\dagger}_{\alpha_2}\Phi_{10110}& = a^{\dagger}_{\alpha_2}|10110\rangle&=0\times |10110\rangle\\
a^{\dagger}_{\alpha_2}\Phi_{11001}& = a^{\dagger}_{\alpha_2}|11001\rangle&=(+1)\times |11101\rangle\\
a^{\dagger}_{\alpha_2}\Phi_{11010}& = a^{\dagger}_{\alpha_2}|11010\rangle&=(+1)\times |11110\rangle\\
\end{array}
\]
What is the simplest way to obtain the phase when we act with one annihilation(creation) operator
on the given Slater determinant representation?

We have an SD representation
\[
\Phi_{\Lambda} = a_{\alpha_0}^{\dagger} a_{\alpha_3}^{\dagger} a_{\alpha_6}^{\dagger} a_{\alpha_{10}}^{\dagger} a_{\alpha_{13}}^{\dagger} |0\rangle,
\]
in a more compact way as
\[
\Phi_{0,3,6,10,13} = |1001001000100100\rangle.
\]
The action of
\[
a^{\dagger}_{\alpha_4}a_{\alpha_0}\Phi_{0,3,6,10,13} = a^{\dagger}_{\alpha_4}|0001001000100100\rangle=a^{\dagger}_{\alpha_4}a_{\alpha_3}^{\dagger} a_{\alpha_6}^{\dagger} a_{\alpha_{10}}^{\dagger} a_{\alpha_{13}}^{\dagger} |0\rangle,
\]
which becomes
\[
-a_{\alpha_3}^{\dagger} a^{\dagger}_{\alpha_4} a_{\alpha_6}^{\dagger} a_{\alpha_{10}}^{\dagger} a_{\alpha_{13}}^{\dagger} |0\rangle=-|0001101000100100\rangle.
\]

The action
\[
a_{\alpha_0}\Phi_{0,3,6,10,13} = |0001001000100100\rangle,
\]
can be obtained by subtracting the logical sum (AND operation) of $\Phi_{0,3,6,10,13}$ and 
a word which represents only $\alpha_0$, that is
\[
|1000000000000000\rangle,
\] 
from $\Phi_{0,3,6,10,13}= |1001001000100100\rangle$.

This operation gives $|0001001000100100\rangle$. 

Similarly, we can form $a^{\dagger}_{\alpha_4}a_{\alpha_0}\Phi_{0,3,6,10,13}$, say, by adding 
$|0000100000000000\rangle$ to $a_{\alpha_0}\Phi_{0,3,6,10,13}$, first checking that their logical sum
is zero in order to make sure that orbital $\alpha_4$ is not already occupied. 

It is trickier however to get the phase $(-1)^l$. 
One possibility is as follows
\begin{itemize}
\item Let $S_1$ be a word that represents the $1-$bit to be removed and all others set to zero.
\end{itemize}

\noindent
In the previous example $S_1=|1000000000000000\rangle$
\begin{itemize}
\item Define $S_2$ as the similar word that represents the bit to be added, that is in our case
\end{itemize}

\noindent
$S_2=|0000100000000000\rangle$.
\begin{itemize}
\item Compute then $S=S_1-S_2$, which here becomes
\end{itemize}

\noindent
\[
S=|0111000000000000\rangle
\]
\begin{itemize}
\item Perform then the logical AND operation of $S$ with the word containing 
\end{itemize}

\noindent
\[
\Phi_{0,3,6,10,13} = |1001001000100100\rangle,
\]
which results in $|0001000000000000\rangle$. Counting the number of $1-$bits gives the phase.  Here you need however an algorithm for bitcounting. Several efficient ones available. 


 \clearemptydoublepage
      
\chapter{Full configuration interaction theory}

\subsection*{Slater determinants as basis states, Repetition}

% --- begin paragraph admon ---
\paragraph{}
The simplest possible choice for many-body wavefunctions are \textbf{product} wavefunctions.
That is
\[ 
\Psi(x_1, x_2, x_3, \ldots, x_A) \approx \phi_1(x_1) \phi_2(x_2) \phi_3(x_3) \ldots
\]
because we are really only good  at thinking about one particle at a time. Such 
product wavefunctions, without correlations, are easy to 
work with; for example, if the single-particle states $\phi_i(x)$ are orthonormal, then 
the product wavefunctions are easy to orthonormalize.   

Similarly, computing matrix elements of operators are relatively easy, because the 
integrals factorize.

The price we pay is the lack of correlations, which we must build up by using many, many product 
wavefunctions. (Thus we have a trade-off: compact representation of correlations but 
difficult integrals versus easy integrals but many states required.)
% --- end paragraph admon ---



\subsection*{Slater determinants as basis states, repetition}

% --- begin paragraph admon ---
\paragraph{}
Because we have fermions, we are required to have antisymmetric wavefunctions, e.g.
\[
\Psi(x_1, x_2, x_3, \ldots, x_A) = - \Psi(x_2, x_1, x_3, \ldots, x_A)
\]
etc. This is accomplished formally by using the determinantal formalism
\[
\Psi(x_1, x_2, \ldots, x_A) 
= \frac{1}{\sqrt{A!}} 
\det \left | 
\begin{array}{cccc}
\phi_1(x_1) & \phi_1(x_2) & \ldots & \phi_1(x_A) \\
\phi_2(x_1) & \phi_2(x_2) & \ldots & \phi_2(x_A) \\
 \vdots & & &  \\
\phi_A(x_1) & \phi_A(x_2) & \ldots & \phi_A(x_A) 
\end{array}
\right |
\]
Product wavefunction + antisymmetry = Slater determinant.
% --- end paragraph admon ---



\subsection*{Slater determinants as basis states}

% --- begin paragraph admon ---
\paragraph{}
\[
\Psi(x_1, x_2, \ldots, x_A) 
= \frac{1}{\sqrt{A!}} 
\det \left | 
\begin{array}{cccc}
\phi_1(x_1) & \phi_1(x_2) & \ldots & \phi_1(x_A) \\
\phi_2(x_1) & \phi_2(x_2) & \ldots & \phi_2(x_A) \\
 \vdots & & &  \\
\phi_A(x_1) & \phi_A(x_2) & \ldots & \phi_A(x_A) 
\end{array}
\right |
\]
Properties of the determinant (interchange of any two rows or 
any two columns yields a change in sign; thus no two rows and no 
two columns can be the same) lead to the Pauli principle:

\begin{itemize}
\item No two particles can be at the same place (two columns the same); and

\item No two particles can be in the same state (two rows the same).
\end{itemize}

\noindent
% --- end paragraph admon ---



\subsection*{Slater determinants as basis states}

% --- begin paragraph admon ---
\paragraph{}
As a practical matter, however, Slater determinants beyond $N=4$ quickly become 
unwieldy. Thus we turn to the \textbf{occupation representation} or \textbf{second quantization} to simplify calculations. 

The occupation representation or number representation, using fermion \textbf{creation} and \textbf{annihilation} 
operators, is compact and efficient. It is also abstract and, at first encounter, not easy to 
internalize. It is inspired by other operator formalism, such as the ladder operators for 
the harmonic oscillator or for angular momentum, but unlike those cases, the operators \textbf{do not have coordinate space representations}.

Instead, one can think of fermion creation/annihilation operators as a game of symbols that 
compactly reproduces what one would do, albeit clumsily, with full coordinate-space Slater 
determinants.
% --- end paragraph admon ---



\subsection*{Quick repetition of the occupation representation}

% --- begin paragraph admon ---
\paragraph{}
We start with a set of orthonormal single-particle states $\{ \phi_i(x) \}$. 
(Note: this requirement, and others, can be relaxed, but leads to a 
more involved formalism.) \textbf{Any} orthonormal set will do. 

To each single-particle state $\phi_i(x)$ we associate a creation operator 
$\hat{a}^\dagger_i$ and an annihilation operator $\hat{a}_i$. 

When acting on the vacuum state $| 0 \rangle$, the creation operator $\hat{a}^\dagger_i$ causes 
a particle to occupy the single-particle state $\phi_i(x)$:
\[
\phi_i(x) \rightarrow \hat{a}^\dagger_i |0 \rangle
\]
% --- end paragraph admon ---



\subsection*{Quick repetition  of the occupation representation}

% --- begin paragraph admon ---
\paragraph{}
But with multiple creation operators we can occupy multiple states:
\[
\phi_i(x) \phi_j(x^\prime) \phi_k(x^{\prime \prime}) 
\rightarrow \hat{a}^\dagger_i \hat{a}^\dagger_j \hat{a}^\dagger_k |0 \rangle.
\]

Now we impose antisymmetry, by having the fermion operators satisfy  \textbf{anticommutation relations}:
\[
\hat{a}^\dagger_i \hat{a}^\dagger_j + \hat{a}^\dagger_j \hat{a}^\dagger_i
= [ \hat{a}^\dagger_i ,\hat{a}^\dagger_j ]_+ 
= \{ \hat{a}^\dagger_i ,\hat{a}^\dagger_j \} = 0
\]
so that 
\[
\hat{a}^\dagger_i \hat{a}^\dagger_j = - \hat{a}^\dagger_j \hat{a}^\dagger_i
\]
% --- end paragraph admon ---



\subsection*{Quick repetition  of the occupation representation}

% --- begin paragraph admon ---
\paragraph{}
Because of this property, automatically $\hat{a}^\dagger_i \hat{a}^\dagger_i = 0$, 
enforcing the Pauli exclusion principle.  Thus when writing a Slater determinant 
using creation operators, 
\[
\hat{a}^\dagger_i \hat{a}^\dagger_j \hat{a}^\dagger_k \ldots |0 \rangle
\]
each index $i,j,k, \ldots$ must be unique.

For some relevant exercises with solutions see chapter 8 of \href{{http://www.springer.com/us/book/9783319533353}}{Lecture Notes in Physics, volume 936}.
% --- end paragraph admon ---



\subsection*{Full Configuration Interaction Theory}

% --- begin paragraph admon ---
\paragraph{}
We have defined the ansatz for the ground state as 
\[
|\Phi_0\rangle = \left(\prod_{i\le F}\hat{a}_{i}^{\dagger}\right)|0\rangle,
\]
where the index $i$ defines different single-particle states up to the Fermi level. We have assumed that we have $N$ fermions. 
A given one-particle-one-hole ($1p1h$) state can be written as
\[
|\Phi_i^a\rangle = \hat{a}_{a}^{\dagger}\hat{a}_i|\Phi_0\rangle,
\]
while a $2p2h$ state can be written as
\[
|\Phi_{ij}^{ab}\rangle = \hat{a}_{a}^{\dagger}\hat{a}_{b}^{\dagger}\hat{a}_j\hat{a}_i|\Phi_0\rangle,
\]
and a general $NpNh$ state as 
\[
|\Phi_{ijk\dots}^{abc\dots}\rangle = \hat{a}_{a}^{\dagger}\hat{a}_{b}^{\dagger}\hat{a}_{c}^{\dagger}\dots\hat{a}_k\hat{a}_j\hat{a}_i|\Phi_0\rangle.
\]
% --- end paragraph admon ---



\subsection*{Full Configuration Interaction Theory}

% --- begin paragraph admon ---
\paragraph{}
We can then expand our exact state function for the ground state 
as
\[
|\Psi_0\rangle=C_0|\Phi_0\rangle+\sum_{ai}C_i^a|\Phi_i^a\rangle+\sum_{abij}C_{ij}^{ab}|\Phi_{ij}^{ab}\rangle+\dots
=(C_0+\hat{C})|\Phi_0\rangle,
\]
where we have introduced the so-called correlation operator 
\[
\hat{C}=\sum_{ai}C_i^a\hat{a}_{a}^{\dagger}\hat{a}_i  +\sum_{abij}C_{ij}^{ab}\hat{a}_{a}^{\dagger}\hat{a}_{b}^{\dagger}\hat{a}_j\hat{a}_i+\dots
\]
Since the normalization of $\Psi_0$ is at our disposal and since $C_0$ is by hypothesis non-zero, we may arbitrarily set $C_0=1$ with 
corresponding proportional changes in all other coefficients. Using this so-called intermediate normalization we have
\[
\langle \Psi_0 | \Phi_0 \rangle = \langle \Phi_0 | \Phi_0 \rangle = 1, 
\]
resulting in 
\[
|\Psi_0\rangle=(1+\hat{C})|\Phi_0\rangle.
\]
% --- end paragraph admon ---



\subsection*{Full Configuration Interaction Theory}

% --- begin paragraph admon ---
\paragraph{}
We rewrite 
\[
|\Psi_0\rangle=C_0|\Phi_0\rangle+\sum_{ai}C_i^a|\Phi_i^a\rangle+\sum_{abij}C_{ij}^{ab}|\Phi_{ij}^{ab}\rangle+\dots,
\]
in a more compact form as 
\[
|\Psi_0\rangle=\sum_{PH}C_H^P\Phi_H^P=\left(\sum_{PH}C_H^P\hat{A}_H^P\right)|\Phi_0\rangle,
\]
where $H$ stands for $0,1,\dots,n$ hole states and $P$ for $0,1,\dots,n$ particle states. 
Our requirement of unit normalization gives
\[
\langle \Psi_0 | \Phi_0 \rangle = \sum_{PH}|C_H^P|^2= 1,
\]
and the energy can be written as 
\[
E= \langle \Psi_0 | \hat{H} |\Phi_0 \rangle= \sum_{PP'HH'}C_H^{*P}\langle \Phi_H^P | \hat{H} |\Phi_{H'}^{P'} \rangle C_{H'}^{P'}.
\]
% --- end paragraph admon ---



\subsection*{Full Configuration Interaction Theory}

% --- begin paragraph admon ---
\paragraph{}
Normally 
\[
E= \langle \Psi_0 | \hat{H} |\Phi_0 \rangle= \sum_{PP'HH'}C_H^{*P}\langle \Phi_H^P | \hat{H} |\Phi_{H'}^{P'} \rangle C_{H'}^{P'},
\]
is solved by diagonalization setting up the Hamiltonian matrix defined by the basis of all possible Slater determinants. A diagonalization
% to do: add text about Rayleigh-Ritz
is equivalent to finding the variational minimum   of 
\[
 \langle \Psi_0 | \hat{H} |\Phi_0 \rangle-\lambda \langle \Psi_0 |\Phi_0 \rangle,
\]
where $\lambda$ is a variational multiplier to be identified with the energy of the system.
The minimization process results in 
\[
\delta\left[ \langle \Psi_0 | \hat{H} |\Phi_0 \rangle-\lambda \langle \Psi_0 |\Phi_0 \rangle\right]=
\]
\[
\sum_{P'H'}\left\{\delta[C_H^{*P}]\langle \Phi_H^P | \hat{H} |\Phi_{H'}^{P'} \rangle C_{H'}^{P'}+
C_H^{*P}\langle \Phi_H^P | \hat{H} |\Phi_{H'}^{P'} \rangle \delta[C_{H'}^{P'}]-
\lambda( \delta[C_H^{*P}]C_{H'}^{P'}+C_H^{*P}\delta[C_{H'}^{P'}]\right\} = 0.
\]
Since the coefficients $\delta[C_H^{*P}]$ and $\delta[C_{H'}^{P'}]$ are complex conjugates it is necessary and sufficient to require the quantities that multiply with $\delta[C_H^{*P}]$ to vanish.
% --- end paragraph admon ---



\subsection*{Full Configuration Interaction Theory}

% --- begin paragraph admon ---
\paragraph{}

This leads to 
\[
\sum_{P'H'}\langle \Phi_H^P | \hat{H} |\Phi_{H'}^{P'} \rangle C_{H'}^{P'}-\lambda C_H^{P}=0,
\]
for all sets of $P$ and $H$.

If we then multiply by the corresponding $C_H^{*P}$ and sum over $PH$ we obtain
\[ 
\sum_{PP'HH'}C_H^{*P}\langle \Phi_H^P | \hat{H} |\Phi_{H'}^{P'} \rangle C_{H'}^{P'}-\lambda\sum_{PH}|C_H^P|^2=0,
\]
leading to the identification $\lambda = E$. This means that we have for all $PH$ sets
\begin{equation}
\sum_{P'H'}\langle \Phi_H^P | \hat{H} -E|\Phi_{H'}^{P'} \rangle = 0. \label{eq:fullci}
\end{equation}
% --- end paragraph admon ---



\subsection*{Full Configuration Interaction Theory}

% --- begin paragraph admon ---
\paragraph{}
An alternative way to derive the last equation is to start from 
\[
(\hat{H} -E)|\Psi_0\rangle = (\hat{H} -E)\sum_{P'H'}C_{H'}^{P'}|\Phi_{H'}^{P'} \rangle=0, 
\]
and if this equation is successively projected against all $\Phi_H^P$ in the expansion of $\Psi$, then the last equation on the previous slide
results.   As stated previously, one solves this equation normally by diagonalization. If we are able to solve this equation exactly (that is
numerically exactly) in a large Hilbert space (it will be truncated in terms of the number of single-particle states included in the definition
of Slater determinants), it can then serve as a benchmark for other many-body methods which approximate the correlation operator
$\hat{C}$.
% --- end paragraph admon ---



\subsection*{Example of a Hamiltonian matrix}

% --- begin paragraph admon ---
\paragraph{}
Suppose, as an example, that we have six fermions below the Fermi level.
This means that we can make at most $6p-6h$ excitations. If we have an infinity of single particle states above the Fermi level, we will obviously have an infinity of say $2p-2h$ excitations. Each such way to configure the particles is called a \textbf{configuration}. We will always have to truncate in the basis of single-particle states.
This gives us a finite number of possible Slater determinants. Our Hamiltonian matrix would then look like (where each block can have a large dimensionalities):


\begin{quote}
\begin{tabular}{cccccccc}
\hline
\multicolumn{1}{c}{  } & \multicolumn{1}{c}{ $0p-0h$ } & \multicolumn{1}{c}{ $1p-1h$ } & \multicolumn{1}{c}{ $2p-2h$ } & \multicolumn{1}{c}{ $3p-3h$ } & \multicolumn{1}{c}{ $4p-4h$ } & \multicolumn{1}{c}{ $5p-5h$ } & \multicolumn{1}{c}{ $6p-6h$ } \\
\hline
$0p-0h$ & x       & x       & x       & 0       & 0       & 0       & 0       \\
$1p-1h$ & x       & x       & x       & x       & 0       & 0       & 0       \\
$2p-2h$ & x       & x       & x       & x       & x       & 0       & 0       \\
$3p-3h$ & 0       & x       & x       & x       & x       & x       & 0       \\
$4p-4h$ & 0       & 0       & x       & x       & x       & x       & x       \\
$5p-5h$ & 0       & 0       & 0       & x       & x       & x       & x       \\
$6p-6h$ & 0       & 0       & 0       & 0       & x       & x       & x       \\
\hline
\end{tabular}
\end{quote}

\noindent
with a two-body force. Why are there non-zero blocks of elements?
% --- end paragraph admon ---



\subsection*{Example of a Hamiltonian matrix with a Hartree-Fock basis}

% --- begin paragraph admon ---
\paragraph{}
If we use a Hartree-Fock basis, this corresponds to a particular unitary transformation where matrix elements of the type $\langle 0p-0h \vert \hat{H} \vert 1p-1h\rangle =\langle \Phi_0 | \hat{H}|\Phi_{i}^{a}\rangle=0$ and our Hamiltonian matrix becomes 


\begin{quote}
\begin{tabular}{cccccccc}
\hline
\multicolumn{1}{c}{  } & \multicolumn{1}{c}{ $0p-0h$ } & \multicolumn{1}{c}{ $1p-1h$ } & \multicolumn{1}{c}{ $2p-2h$ } & \multicolumn{1}{c}{ $3p-3h$ } & \multicolumn{1}{c}{ $4p-4h$ } & \multicolumn{1}{c}{ $5p-5h$ } & \multicolumn{1}{c}{ $6p-6h$ } \\
\hline
$0p-0h$ & $\tilde{x}$ & 0           & $\tilde{x}$ & 0           & 0           & 0           & 0           \\
$1p-1h$ & 0           & $\tilde{x}$ & $\tilde{x}$ & $\tilde{x}$ & 0           & 0           & 0           \\
$2p-2h$ & $\tilde{x}$ & $\tilde{x}$ & $\tilde{x}$ & $\tilde{x}$ & $\tilde{x}$ & 0           & 0           \\
$3p-3h$ & 0           & $\tilde{x}$ & $\tilde{x}$ & $\tilde{x}$ & $\tilde{x}$ & $\tilde{x}$ & 0           \\
$4p-4h$ & 0           & 0           & $\tilde{x}$ & $\tilde{x}$ & $\tilde{x}$ & $\tilde{x}$ & $\tilde{x}$ \\
$5p-5h$ & 0           & 0           & 0           & $\tilde{x}$ & $\tilde{x}$ & $\tilde{x}$ & $\tilde{x}$ \\
$6p-6h$ & 0           & 0           & 0           & 0           & $\tilde{x}$ & $\tilde{x}$ & $\tilde{x}$ \\
\hline
\end{tabular}
\end{quote}

\noindent
% --- end paragraph admon ---



\subsection*{Shell-model jargon}

% --- begin paragraph admon ---
\paragraph{}
If we do not make any truncations in the possible sets of Slater determinants (many-body states) we can make by distributing $A$ nucleons among $n$ single-particle states, we call such a calculation for \textbf{Full configuration interaction theory}

If we make truncations, we have different possibilities

\begin{itemize}
\item The standard nuclear shell-model. Here we define an effective Hilbert space with respect to a given core. The calculations are normally then performed for all many-body states that can be constructed from the effective Hilbert spaces. This approach requires a properly defined effective Hamiltonian

\item We can truncate in the number of excitations. For example, we can limit the possible Slater determinants to only $1p-1h$ and $2p-2h$ excitations. This is called a configuration interaction calculation at the level of singles and doubles excitations, or just CISD. 

\item We can limit the number of excitations in terms of the excitation energies. If we do not define a core, this defines normally what is called the no-core shell-model approach. 
\end{itemize}

\noindent
What happens if we have a three-body interaction and a Hartree-Fock basis?
% --- end paragraph admon ---



\subsection*{FCI and the exponential growth}

% --- begin paragraph admon ---
\paragraph{}
Full configuration interaction theory calculations provide in principle, if we can diagonalize numerically, all states of interest. The dimensionality of the problem explodes however quickly.

The total number of Slater determinants which can be built with say $N$ neutrons distributed among $n$ single particle states is
\[
\left (\begin{array}{c} n \\ N\end{array} \right) =\frac{n!}{(n-N)!N!}. 
\]

For a model space which comprises the first for major shells only $0s$, $0p$, $1s0d$ and $1p0f$ we have $40$ single particle states for neutrons and protons.  For the eight neutrons of oxygen-16 we would then have
\[
\left (\begin{array}{c} 40 \\ 8\end{array} \right) =\frac{40!}{(32)!8!}\sim 10^{9}, 
\]
and multiplying this with the number of proton Slater determinants we end up with approximately with a dimensionality $d$ of $d\sim 10^{18}$.
% --- end paragraph admon ---



\subsection*{Exponential wall}

% --- begin paragraph admon ---
\paragraph{}
This number can be reduced if we look at specific symmetries only. However, the dimensionality explodes quickly!

\begin{itemize}
\item For Hamiltonian matrices of dimensionalities  which are smaller than $d\sim 10^5$, we would use so-called direct methods for diagonalizing the Hamiltonian matrix

\item For larger dimensionalities iterative eigenvalue solvers like Lanczos' method are used. The most efficient codes at present can handle matrices of $d\sim 10^{10}$. 
\end{itemize}

\noindent
% --- end paragraph admon ---



\subsection*{A non-practical way of solving the eigenvalue problem}

% --- begin paragraph admon ---
\paragraph{}
To see this, we look at the contributions arising from 
\[
\langle \Phi_H^P | = \langle \Phi_0|
\]
in  Eq.~(\ref{eq:fullci}), that is we multiply with $\langle \Phi_0 |$
from the left in 
\[
(\hat{H} -E)\sum_{P'H'}C_{H'}^{P'}|\Phi_{H'}^{P'} \rangle=0. 
\]
If we assume that we have a two-body operator at most, Slater's rule gives then an equation for the 
correlation energy in terms of $C_i^a$ and $C_{ij}^{ab}$ only.  We get then
\[
\langle \Phi_0 | \hat{H} -E| \Phi_0\rangle + \sum_{ai}\langle \Phi_0 | \hat{H} -E|\Phi_{i}^{a} \rangle C_{i}^{a}+
\sum_{abij}\langle \Phi_0 | \hat{H} -E|\Phi_{ij}^{ab} \rangle C_{ij}^{ab}=0,
\]
or 
\[
E-E_0 =\Delta E=\sum_{ai}\langle \Phi_0 | \hat{H}|\Phi_{i}^{a} \rangle C_{i}^{a}+
\sum_{abij}\langle \Phi_0 | \hat{H}|\Phi_{ij}^{ab} \rangle C_{ij}^{ab},
\]
where the energy $E_0$ is the reference energy and $\Delta E$ defines the so-called correlation energy.
The single-particle basis functions  could be the results of a Hartree-Fock calculation or just the eigenstates of the non-interacting part of the Hamiltonian.
% --- end paragraph admon ---



\subsection*{A non-practical way of solving the eigenvalue problem}

% --- begin paragraph admon ---
\paragraph{}
To see this, we look at the contributions arising from 
\[
\langle \Phi_H^P | = \langle \Phi_0|
\]
in  Eq.~(\ref{eq:fullci}), that is we multiply with $\langle \Phi_0 |$
from the left in 
\[
(\hat{H} -E)\sum_{P'H'}C_{H'}^{P'}|\Phi_{H'}^{P'} \rangle=0. 
\]
% --- end paragraph admon ---



\subsection*{A non-practical way of solving the eigenvalue problem}

% --- begin paragraph admon ---
\paragraph{}
If we assume that we have a two-body operator at most, Slater's rule gives then an equation for the 
correlation energy in terms of $C_i^a$ and $C_{ij}^{ab}$ only.  We get then
\[
\langle \Phi_0 | \hat{H} -E| \Phi_0\rangle + \sum_{ai}\langle \Phi_0 | \hat{H} -E|\Phi_{i}^{a} \rangle C_{i}^{a}+
\sum_{abij}\langle \Phi_0 | \hat{H} -E|\Phi_{ij}^{ab} \rangle C_{ij}^{ab}=0,
\]
or 
\[
E-E_0 =\Delta E=\sum_{ai}\langle \Phi_0 | \hat{H}|\Phi_{i}^{a} \rangle C_{i}^{a}+
\sum_{abij}\langle \Phi_0 | \hat{H}|\Phi_{ij}^{ab} \rangle C_{ij}^{ab},
\]
where the energy $E_0$ is the reference energy and $\Delta E$ defines the so-called correlation energy.
The single-particle basis functions  could be the results of a Hartree-Fock calculation or just the eigenstates of the non-interacting part of the Hamiltonian.
% --- end paragraph admon ---



\subsection*{Rewriting the FCI equation}

% --- begin paragraph admon ---
\paragraph{}
In our notes on Hartree-Fock calculations, 
we have already computed the matrix $\langle \Phi_0 | \hat{H}|\Phi_{i}^{a}\rangle $ and $\langle \Phi_0 | \hat{H}|\Phi_{ij}^{ab}\rangle$.  If we are using a Hartree-Fock basis, then the matrix elements
$\langle \Phi_0 | \hat{H}|\Phi_{i}^{a}\rangle=0$ and we are left with a \emph{correlation energy} given by
\[
E-E_0 =\Delta E^{HF}=\sum_{abij}\langle \Phi_0 | \hat{H}|\Phi_{ij}^{ab} \rangle C_{ij}^{ab}. 
\]
% --- end paragraph admon ---




\subsection*{Rewriting the FCI equation}

% --- begin paragraph admon ---
\paragraph{}
Inserting the various matrix elements we can rewrite the previous equation as
\[
\Delta E=\sum_{ai}\langle i| \hat{f}|a \rangle C_{i}^{a}+
\sum_{abij}\langle ij | \hat{v}| ab \rangle C_{ij}^{ab}.
\]
This equation determines the correlation energy but not the coefficients $C$.
% --- end paragraph admon ---



\subsection*{Rewriting the FCI equation, does not stop here}

% --- begin paragraph admon ---
\paragraph{}
We need more equations. Our next step is to set up
\[
\langle \Phi_i^a | \hat{H} -E| \Phi_0\rangle + \sum_{bj}\langle \Phi_i^a | \hat{H} -E|\Phi_{j}^{b} \rangle C_{j}^{b}+
\sum_{bcjk}\langle \Phi_i^a | \hat{H} -E|\Phi_{jk}^{bc} \rangle C_{jk}^{bc}+
\sum_{bcdjkl}\langle \Phi_i^a | \hat{H} -E|\Phi_{jkl}^{bcd} \rangle C_{jkl}^{bcd}=0,
\]
as this equation will allow us to find an expression for the coefficents $C_i^a$ since we can rewrite this equation as 
\[
\langle i | \hat{f}| a\rangle +\langle \Phi_i^a | \hat{H}|\Phi_{i}^{a} \rangle C_{i}^{a}+ \sum_{bj\ne ai}\langle \Phi_i^a | \hat{H}|\Phi_{j}^{b} \rangle C_{j}^{b}+
\sum_{bcjk}\langle \Phi_i^a | \hat{H}|\Phi_{jk}^{bc} \rangle C_{jk}^{bc}+
\sum_{bcdjkl}\langle \Phi_i^a | \hat{H}|\Phi_{jkl}^{bcd} \rangle C_{jkl}^{bcd}=EC_i^a.
\]
% --- end paragraph admon ---



\subsection*{Rewriting the FCI equation, please stop here}

% --- begin paragraph admon ---
\paragraph{}
We see that on the right-hand side we have the energy $E$. This leads to a non-linear equation in the unknown coefficients. 
These equations are normally solved iteratively ( that is we can start with a guess for the coefficients $C_i^a$). A common choice is to use perturbation theory for the first guess, setting thereby
\[
 C_{i}^{a}=\frac{\langle i | \hat{f}| a\rangle}{\epsilon_i-\epsilon_a}.
\]
% --- end paragraph admon ---



\subsection*{Rewriting the FCI equation, more to add}

% --- begin paragraph admon ---
\paragraph{}
The observant reader will however see that we need an equation for $C_{jk}^{bc}$ and $C_{jkl}^{bcd}$ as well.
To find equations for these coefficients we need then to continue our multiplications from the left with the various
$\Phi_{H}^P$ terms. 

For $C_{jk}^{bc}$ we need then
\[
\langle \Phi_{ij}^{ab} | \hat{H} -E| \Phi_0\rangle + \sum_{kc}\langle \Phi_{ij}^{ab} | \hat{H} -E|\Phi_{k}^{c} \rangle C_{k}^{c}+
\]
\[
\sum_{cdkl}\langle \Phi_{ij}^{ab} | \hat{H} -E|\Phi_{kl}^{cd} \rangle C_{kl}^{cd}+\sum_{cdeklm}\langle \Phi_{ij}^{ab} | \hat{H} -E|\Phi_{klm}^{cde} \rangle C_{klm}^{cde}+\sum_{cdefklmn}\langle \Phi_{ij}^{ab} | \hat{H} -E|\Phi_{klmn}^{cdef} \rangle C_{klmn}^{cdef}=0,
\]
and we can isolate the coefficients $C_{kl}^{cd}$ in a similar way as we did for the coefficients $C_{i}^{a}$.
% --- end paragraph admon ---



\subsection*{Rewriting the FCI equation, more to add}

% --- begin paragraph admon ---
\paragraph{}
A standard choice for the first iteration is to set 
\[
C_{ij}^{ab} =\frac{\langle ij \vert \hat{v} \vert ab \rangle}{\epsilon_i+\epsilon_j-\epsilon_a-\epsilon_b}.
\]
At the end we can rewrite our solution of the Schroedinger equation in terms of $n$ coupled equations for the coefficients $C_H^P$.
This is a very cumbersome way of solving the equation. However, by using this iterative scheme we can illustrate how we can compute the
various terms in the wave operator or correlation operator $\hat{C}$. We will later identify the calculation of the various terms $C_H^P$
as parts of different many-body approximations to full CI. In particular, we can  relate this non-linear scheme with Coupled Cluster theory and
many-body perturbation theory.
% --- end paragraph admon ---



\subsection*{Summarizing FCI and bringing in approximative methods}

% --- begin paragraph admon ---
\paragraph{}

If we can diagonalize large matrices, FCI is the method of choice since:
\begin{itemize}
\item It gives all eigenvalues, ground state and excited states

\item The eigenvectors are obtained directly from the coefficients $C_H^P$ which result from the diagonalization

\item We can compute easily expectation values of other operators, as well as transition probabilities

\item Correlations are easy to understand in terms of contributions to a given operator beyond the Hartree-Fock contribution. This is the standard approach in  many-body theory. 
\end{itemize}

\noindent
% --- end paragraph admon ---



\subsection*{Definition of the correlation energy}

% --- begin paragraph admon ---
\paragraph{}
The correlation energy is defined as, with a two-body Hamiltonian,  
\[
\Delta E=\sum_{ai}\langle i| \hat{f}|a \rangle C_{i}^{a}+
\sum_{abij}\langle ij | \hat{v}| ab \rangle C_{ij}^{ab}.
\]
The coefficients $C$ result from the solution of the eigenvalue problem. 
The energy of say the ground state is then
\[
E=E_{ref}+\Delta E,
\]
where the so-called reference energy is the energy we obtain from a Hartree-Fock calculation, that is
\[
E_{ref}=\langle \Phi_0 \vert \hat{H} \vert \Phi_0 \rangle.
\]
% --- end paragraph admon ---



\subsection*{FCI equation and the coefficients}

% --- begin paragraph admon ---
\paragraph{}

However, as we have seen, even for a small case like the four first major shells and a nucleus like oxygen-16, the dimensionality becomes quickly intractable. If we wish to include single-particle states that reflect weakly bound systems, we need a much larger single-particle basis. We need thus approximative methods that sum specific correlations to infinite order. 

Popular methods are
\begin{itemize}
\item \href{{http://www.sciencedirect.com/science/article/pii/0370157395000126}}{Many-body perturbation theory (in essence a Taylor expansion)}

\item \href{{http://iopscience.iop.org/article/10.1088/0034-4885/77/9/096302/meta}}{Coupled cluster theory (coupled non-linear equations)}

\item Green's function approaches (matrix inversion)

\item \href{{http://journals.aps.org/prl/abstract/10.1103/PhysRevLett.106.222502}}{Similarity group transformation methods (coupled ordinary differential equations)}
\end{itemize}

\noindent
All these methods start normally with a Hartree-Fock basis as the calculational basis.
% --- end paragraph admon ---



\subsection*{Important ingredients to have in codes}

% --- begin paragraph admon ---
\paragraph{}

\begin{itemize}
\item Be able to validate and verify  the  algorithms. 

\item Include concepts like unit testing. Gives the possibility to test and validate several or all parts of the code.

\item Validation and verification are then included \emph{naturally} and one can develop a better attitude to what is meant with an ethically sound scientific approach.
\end{itemize}

\noindent
% --- end paragraph admon ---



\subsection*{A structured approach to solving problems}

% --- begin paragraph admon ---
\paragraph{}
In the steps that lead to the development of clean code you should  think of 
\begin{enumerate}
  \item How to structure a code in terms of functions  (use IDEs or advanced text editors like sublime or atom)

  \item How to make a module

  \item How to read input data flexibly from the command line or files

  \item How to create graphical/web user interfaces

  \item How to write unit tests  

  \item How to refactor code in terms of classes (instead of functions only)

  \item How to conduct and automate large-scale numerical experiments

  \item How to write scientific reports in various formats ({\LaTeX}, HTML, doconce)
\end{enumerate}

\noindent
% --- end paragraph admon ---



\subsection*{Additional benefits}

% --- begin paragraph admon ---
\paragraph{}
Many of the above aspetcs  will save you a lot of time when you incrementally extend software over time from simpler to more complicated problems. In particular, you will benefit from many good habits:
\begin{enumerate}
\item New code is added in a modular fashion to a library (modules)

\item Programs are run through convenient user interfaces

\item It takes one quick command to let all your code undergo heavy testing 

\item Tedious manual work with running programs is automated,

\item Your scientific investigations are reproducible, scientific reports with top quality typesetting are produced both for paper and electronic devices. Use version control software like \href{{https://git-scm.com/}}{git} and repositories like \href{{https://github.com/}}{github}
\end{enumerate}

\noindent
% --- end paragraph admon ---



\subsection*{Unit Testing}

% --- begin paragraph admon ---
\paragraph{}
Unit Testing is the practice of testing the smallest testable parts,
called units, of an application individually and independently to
determine if they behave exactly as expected. 

Unit tests (short code
fragments) are usually written such that they can be preformed at any
time during the development to continually verify the behavior of the
code. 

In this way, possible bugs will be identified early in the
development cycle, making the debugging at later stages much
easier.
% --- end paragraph admon ---



\subsection*{Unit Testing, benefits}

% --- begin paragraph admon ---
\paragraph{}
There are many benefits associated with Unit Testing, such as
\begin{itemize}
  \item It increases confidence in changing and maintaining code. Big changes can be made to the code quickly, since the tests will ensure that everything still is working properly.

  \item Since the code needs to be modular to make Unit Testing possible, the code will be easier to reuse. This improves the code design.

  \item Debugging is easier, since when a test fails, only the latest changes need to be debugged.
\begin{itemize}

   \item Different parts of a project can be tested without the need to wait for the other parts to be available.

\end{itemize}

\noindent
  \item A unit test can serve as a documentation on the functionality of a unit of the code.
\end{itemize}

\noindent
% --- end paragraph admon ---



\subsection*{Simple example of unit test}

% --- begin paragraph admon ---
\paragraph{}
Look up the guide on how to install unit tests for c++ at course webpage. This is the version with classes.



















\begin{minted}[fontsize=\fontsize{9pt}{9pt},linenos=false,mathescape,baselinestretch=1.0,fontfamily=tt,xleftmargin=7mm]{c++}
#include <unittest++/UnitTest++.h>

class MyMultiplyClass{
public:
    double multiply(double x, double y) {
        return x * y;
    }
};

TEST(MyMath) {
    MyMultiplyClass my;
    CHECK_EQUAL(56, my.multiply(7,8));
}

int main()
{
    return UnitTest::RunAllTests();
}

\end{minted}
% --- end paragraph admon ---



\subsection*{Simple example of unit test}

% --- begin paragraph admon ---
\paragraph{}
And without classes
















\begin{minted}[fontsize=\fontsize{9pt}{9pt},linenos=false,mathescape,baselinestretch=1.0,fontfamily=tt,xleftmargin=7mm]{c++}
#include <unittest++/UnitTest++.h>


double multiply(double x, double y) {
    return x * y;
}

TEST(MyMath) {
    CHECK_EQUAL(56, multiply(7,8));
}

int main()
{
    return UnitTest::RunAllTests();
} 

\end{minted}

For Fortran users, the link at \href{{http://sourceforge.net/projects/fortranxunit/}}{\nolinkurl{http://sourceforge.net/projects/fortranxunit/}} contains a similar
software for unit testing. For Python go to \href{{https://docs.python.org/2/library/unittest.html}}{\nolinkurl{https://docs.python.org/2/library/unittest.html}}.
% --- end paragraph admon ---



\subsection*{\href{{https://github.com/philsquared/Catch/blob/master/docs/tutorial.md}}{Unit tests}}

% --- begin paragraph admon ---
\paragraph{}
There are many types of \textbf{unit test} libraries. One which is very popular with C++ programmers is \href{{https://github.com/philsquared/Catch/blob/master/docs/tutorial.md}}{Catch}

Catch is header only. All you need to do is drop the file(s) somewhere reachable from your project - either in some central location you can set your header search path to find, or directly into your project tree itself! 

This is a particularly good option for other Open-Source projects that want to use Catch for their test suite.
% --- end paragraph admon ---



\subsection*{Examples}

Computing factorials




\begin{minted}[fontsize=\fontsize{9pt}{9pt},linenos=false,mathescape,baselinestretch=1.0,fontfamily=tt,xleftmargin=7mm]{c++}
inline unsigned int Factorial( unsigned int number ) {
  return number > 1 ? Factorial(number-1)*number : 1;
}

\end{minted}


\subsection*{Factorial Example}

Simple test where we put everything in a single file 
















\begin{minted}[fontsize=\fontsize{9pt}{9pt},linenos=false,mathescape,baselinestretch=1.0,fontfamily=tt,xleftmargin=7mm]{c++}
#define CATCH_CONFIG_MAIN  // This tells Catch to provide a main()
#include "catch.hpp"
inline unsigned int Factorial( unsigned int number ) {
  return number > 1 ? Factorial(number-1)*number : 1;
}

TEST_CASE( "Factorials are computed", "[factorial]" ) {
    REQUIRE( Factorial(0) == 1 );
    REQUIRE( Factorial(1) == 1 );
    REQUIRE( Factorial(2) == 2 );
    REQUIRE( Factorial(3) == 6 );
    REQUIRE( Factorial(10) == 3628800 );
}


\end{minted}

This will compile to a complete executable which responds to command line arguments. If you just run it with no arguments it will execute all test cases (in this case there is just one), report any failures, report a summary of how many tests passed and failed and return the number of failed tests.

\subsection*{What did we do (1)?}
All we did was 


\begin{minted}[fontsize=\fontsize{9pt}{9pt},linenos=false,mathescape,baselinestretch=1.0,fontfamily=tt,xleftmargin=7mm]{c++}
#define 

\end{minted}

one identifier and 


\begin{minted}[fontsize=\fontsize{9pt}{9pt},linenos=false,mathescape,baselinestretch=1.0,fontfamily=tt,xleftmargin=7mm]{c++}
#include 

\end{minted}

one header and we got everything - even an implementation of main() that will respond to command line arguments. 
Once you have more than one file with unit tests in you'll just need to 


\begin{minted}[fontsize=\fontsize{9pt}{9pt},linenos=false,mathescape,baselinestretch=1.0,fontfamily=tt,xleftmargin=7mm]{c++}
#include "catch.hpp" 

\end{minted}

and go. Usually it's a good idea to have a dedicated implementation file that just has 



\begin{minted}[fontsize=\fontsize{9pt}{9pt},linenos=false,mathescape,baselinestretch=1.0,fontfamily=tt,xleftmargin=7mm]{c++}
#define CATCH_CONFIG_MAIN 
#include "catch.hpp". 

\end{minted}

You can also provide your own implementation of main and drive Catch yourself.

\subsection*{What did we do (2)?}
We introduce test cases with the 


\begin{minted}[fontsize=\fontsize{9pt}{9pt},linenos=false,mathescape,baselinestretch=1.0,fontfamily=tt,xleftmargin=7mm]{c++}
TEST_CASE 

\end{minted}

macro.

The test name must be unique. You can run sets of tests by specifying a wildcarded test name or a tag expression. 
All we did was \textbf{define} one identifier and \textbf{include} one header and we got everything.

We write our individual test assertions using the 


\begin{minted}[fontsize=\fontsize{9pt}{9pt},linenos=false,mathescape,baselinestretch=1.0,fontfamily=tt,xleftmargin=7mm]{c++}
REQUIRE 

\end{minted}

macro.

\subsection*{Unit test summary and testing approach}

% --- begin paragraph admon ---
\paragraph{}
Three levels of tests
\begin{enumerate}
\item Microscopic level: testing small parts of code, use often unit test libraries

\item Mesoscopic level: testing the integration of various parts  of your code

\item Macroscopic level: testing that the final result is ok
\end{enumerate}

\noindent
% --- end paragraph admon ---

 

\subsection*{Coding Recommendations}
Writing clean and clear code is an art and reflects 
your understanding of 

\begin{enumerate}
\item derivation, verification, and implementation of algorithms

\item what can go wrong with algorithms

\item overview of important, known algorithms

\item how algorithms are used to solve mathematical problems

\item reproducible science and ethics

\item algorithmic thinking for gaining deeper insights about scientific problems
\end{enumerate}

\noindent
Computing is understanding and your understanding is reflected in your abilities to
write clear and clean code.

\subsection*{Summary and recommendations}
Some simple hints and tips in order to write clean and clear code
\begin{enumerate}
\item Spell out the algorithm and have a top-down approach to the flow of data

\item Start with coding as close as possible to eventual mathematical expressions

\item Use meaningful names for variables

\item Split tasks in simple functions and modules/classes

\item Functions should return as few as possible variables

\item Use unit tests and make sure your codes are producing the correct results

\item Where possible use symbolic coding to autogenerate code and check results

\item Make a proper timing of your algorithms

\item Use version control and make your science reproducible

\item Use IDEs or smart editors with debugging and analysis tools.

\item Automatize your computations interfacing high-level and compiled languages like C++ and Fortran.

\item .....
\end{enumerate}

\noindent
\subsection*{Building a many-body basis}

% --- begin paragraph admon ---
\paragraph{}
Here we will discuss how we can set up a single-particle basis which we can use in the various parts of our projects, from the simple pairing model to infinite nuclear matter. We will use here the simple pairing model to illustrate in particular how to set up a single-particle basis. We will also use this do discuss standard FCI approaches like:
\begin{enumerate}
 \item Standard shell-model basis in one or two major shells

 \item Full CI in a given basis and no truncations

 \item CISD and CISDT approximations

 \item No-core shell model and truncation in excitation energy
\end{enumerate}

\noindent
% --- end paragraph admon ---



\subsection*{Building a many-body basis}

% --- begin paragraph admon ---
\paragraph{}
An important step in an FCI code  is to construct the many-body basis.  

While the formalism is independent of the choice of basis, the \textbf{effectiveness} of a calculation 
will certainly be basis dependent. 

Furthermore there are common conventions useful to know.

First, the single-particle basis has angular momentum as a good quantum number.  You can 
imagine the single-particle wavefunctions being generated by a one-body Hamiltonian, 
for example a harmonic oscillator.  Modifications include harmonic oscillator plus 
spin-orbit splitting, or self-consistent mean-field potentials, or the Woods-Saxon potential which mocks 
up the self-consistent mean-field. 
For nuclei, the harmonic oscillator, modified by spin-orbit splitting, provides a useful language 
for describing single-particle states.
% --- end paragraph admon ---



\subsection*{Building a many-body basis}

% --- begin paragraph admon ---
\paragraph{}
Each single-particle state is labeled by the following quantum numbers: 

\begin{itemize}
\item Orbital angular momentum $l$

\item Intrinsic spin $s$ = 1/2 for protons and neutrons

\item Angular momentum $j = l \pm 1/2$

\item $z$-component $j_z$ (or $m$)

\item Some labeling of the radial wavefunction, typically $n$ the number of nodes in  the radial wavefunction, but in the case of harmonic oscillator one can also use the principal quantum number $N$, where the harmonic oscillator energy is $(N+3/2)\hbar \omega$.  
\end{itemize}

\noindent
In this format one labels states by $n(l)_j$, with $(l)$ replaced by a letter:
$s$ for $l=0$, $p$ for $l=1$, $d$ for $l=2$, $f$ for $l=3$, and thenceforth alphabetical.
% --- end paragraph admon ---



\subsection*{Building a many-body basis}

% --- begin paragraph admon ---
\paragraph{}
 In practice the single-particle space has to be severely truncated.  This truncation is 
typically based upon the single-particle energies, which is the effective energy 
from a mean-field potential. 

Sometimes we freeze the core and only consider a valence space. For example, one 
may assume a frozen $^{4}\mbox{He}$ core, with two protons and two neutrons in the $0s_{1/2}$ 
shell, and then only allow active particles in the $0p_{1/2}$ and $0p_{3/2}$ orbits. 

Another example is a frozen $^{16}\mbox{O}$ core, with eight protons and eight neutrons filling the 
$0s_{1/2}$,  $0p_{1/2}$ and $0p_{3/2}$ orbits, with valence particles in the 
$0d_{5/2}, 1s_{1/2}$ and $0d_{3/2}$ orbits.

Sometimes we refer to nuclei by the valence space where their last nucleons go.  
So, for example, we call $^{12}\mbox{C}$ a $p$-shell nucleus, while $^{26}\mbox{Al}$ is an 
$sd$-shell nucleus and $^{56}\mbox{Fe}$ is a $pf$-shell nucleus.
% --- end paragraph admon ---



\subsection*{Building a many-body basis}

% --- begin paragraph admon ---
\paragraph{}
There are different kinds of truncations.

\begin{itemize}
\item For example, one can start with `filled' orbits (almost always the lowest), and then  allow one, two, three... particles excited out of those filled orbits. These are called  1p-1h, 2p-2h, 3p-3h excitations. 

\item Alternately, one can state a maximal orbit and allow all possible configurations with  particles occupying states up to that maximum. This is called \emph{full configuration}.

\item Finally, for particular use in nuclear physics, there is the \emph{energy} truncation, also  called the $N\hbar\Omega$ or $N_{max}$ truncation. 
\end{itemize}

\noindent
% --- end paragraph admon ---



\subsection*{Building a many-body basis}

% --- begin paragraph admon ---
\paragraph{}
Here one works in a harmonic oscillator basis, with each major oscillator shell assigned  a principal quantum number $N=0,1,2,3,...$. 
The $N\hbar\Omega$ or $N_{max}$ truncation: Any configuration is given an noninteracting energy, which is the sum 
of the single-particle harmonic oscillator energies. (Thus this ignores 
spin-orbit splitting.)

Excited state are labeled relative to the lowest configuration by the 
number of harmonic oscillator quanta.

This truncation is useful because if one includes \emph{all} configuration up to 
some $N_{max}$, and has a translationally invariant interaction, then the intrinsic 
motion and the center-of-mass motion factor. In other words, we can know exactly 
the center-of-mass wavefunction. 

In almost all cases, the many-body Hamiltonian is rotationally invariant. This means 
it commutes with the operators $\hat{J}^2, \hat{J}_z$ and so eigenstates will have 
good $J,M$. Furthermore, the eigenenergies do not depend upon the orientation $M$. 

Therefore we can choose to construct a many-body basis which has fixed $M$; this is 
called an $M$-scheme basis. 

Alternately, one can construct a many-body basis which has fixed $J$, or a $J$-scheme 
basis.
% --- end paragraph admon ---



\subsection*{Building a many-body basis}

% --- begin paragraph admon ---
\paragraph{}
The Hamiltonian matrix will have smaller dimensions (a factor of 10 or more) in the $J$-scheme than in the $M$-scheme. 
On the other hand, as we'll show in the next slide, the $M$-scheme is very easy to 
construct with Slater determinants, while the $J$-scheme basis states, and thus the 
matrix elements, are more complicated, almost always being linear combinations of 
$M$-scheme states. $J$-scheme bases are important and useful, but we'll focus on the 
simpler $M$-scheme.

The quantum number $m$ is additive (because the underlying group is Abelian): 
if a Slater determinant $\hat{a}_i^\dagger \hat{a}^\dagger_j \hat{a}^\dagger_k \ldots | 0 \rangle$ 
is built from single-particle states all with good $m$, then the total 
\[
M = m_i + m_j + m_k + \ldots
\]
This is \emph{not} true of $J$, because the angular momentum group SU(2) is not Abelian.
% --- end paragraph admon ---



\subsection*{Building a many-body basis}

% --- begin paragraph admon ---
\paragraph{}

The upshot is that 
\begin{itemize}
\item It is easy to construct a Slater determinant with good total $M$;

\item It is trivial to calculate $M$ for each Slater determinant;

\item So it is easy to construct an $M$-scheme basis with fixed total $M$.
\end{itemize}

\noindent
Note that the individual $M$-scheme basis states will \emph{not}, in general, 
have good total $J$. 
Because the Hamiltonian is rotationally invariant, however, the eigenstates will 
have good $J$. (The situation is muddied when one has states of different $J$ that are 
nonetheless degenerate.)
% --- end paragraph admon ---



\subsection*{Building a many-body basis}

% --- begin paragraph admon ---
\paragraph{}
Example: two $j=1/2$ orbits


\begin{quote}
\begin{tabular}{ccccc}
\hline
\multicolumn{1}{c}{ Index } & \multicolumn{1}{c}{ $n$ } & \multicolumn{1}{c}{ $l$ } & \multicolumn{1}{c}{ $j$ } & \multicolumn{1}{c}{ $m_j$ } \\
\hline
1     & 0   & 0   & 1/2 & -1/2  \\
2     & 0   & 0   & 1/2 & 1/2   \\
3     & 1   & 0   & 1/2 & -1/2  \\
4     & 1   & 0   & 1/2 & 1/2   \\
\hline
\end{tabular}
\end{quote}

\noindent
Note that the order is arbitrary.
% --- end paragraph admon ---



\subsection*{Building a many-body basis}

% --- begin paragraph admon ---
\paragraph{}
There are $\left ( \begin{array}{c} 4 \\ 2 \end{array} \right) = 6$ two-particle states, 
which we list with the total $M$:


\begin{quote}
\begin{tabular}{cc}
\hline
\multicolumn{1}{c}{ Occupied } & \multicolumn{1}{c}{ $M$ } \\
\hline
1,2      & 0   \\
1,3      & -1  \\
1,4      & 0   \\
2,3      & 0   \\
2,4      & 1   \\
3,4      & 0   \\
\hline
\end{tabular}
\end{quote}

\noindent
There are 4 states with $M= 0$, 
and 1 each with $M = \pm 1$.
% --- end paragraph admon ---



\subsection*{Building a many-body basis}

% --- begin paragraph admon ---
\paragraph{}
As another example, consider using only single particle states from the $0d_{5/2}$ space. 
They have the following quantum numbers


\begin{quote}
\begin{tabular}{ccccc}
\hline
\multicolumn{1}{c}{ Index } & \multicolumn{1}{c}{ $n$ } & \multicolumn{1}{c}{ $l$ } & \multicolumn{1}{c}{ $j$ } & \multicolumn{1}{c}{ $m_j$ } \\
\hline
1     & 0   & 2   & 5/2 & -5/2  \\
2     & 0   & 2   & 5/2 & -3/2  \\
3     & 0   & 2   & 5/2 & -1/2  \\
4     & 0   & 2   & 5/2 & 1/2   \\
5     & 0   & 2   & 5/2 & 3/2   \\
6     & 0   & 2   & 5/2 & 5/2   \\
\hline
\end{tabular}
\end{quote}

\noindent
% --- end paragraph admon ---



\subsection*{Building a many-body basis}

% --- begin paragraph admon ---
\paragraph{}
There are $\left ( \begin{array}{c} 6 \\ 2 \end{array} \right) = 15$ two-particle states, 
which we list with the total $M$:


\begin{quote}
\begin{tabular}{cccccc}
\hline
\multicolumn{1}{c}{ Occupied } & \multicolumn{1}{c}{ $M$ } & \multicolumn{1}{c}{ Occupied } & \multicolumn{1}{c}{ $M$ } & \multicolumn{1}{c}{ Occupied } & \multicolumn{1}{c}{ $M$ } \\
\hline
1,2      & -4  & 2,3      & -2  & 3,5      & 1   \\
1,3      & -3  & 2,4      & -1  & 3,6      & 2   \\
1,4      & -2  & 2,5      & 0   & 4,5      & 2   \\
1,5      & -1  & 2,6      & 1   & 4,6      & 3   \\
1,6      & 0   & 3,4      & 0   & 5,6      & 4   \\
\hline
\end{tabular}
\end{quote}

\noindent
There are 3 states with $M= 0$, 2 with $M = 1$, and so on.
% --- end paragraph admon ---



\subsection*{Shell-model project}

% --- begin paragraph admon ---
\paragraph{}

The first step  is to construct the $M$-scheme basis of Slater determinants.
Here $M$-scheme means the total $J_z$ of the many-body states is fixed.

The steps could be:

\begin{itemize}
\item Read in a user-supplied file of single-particle states (examples can be given) or just code these internally;

\item Ask for the total $M$ of the system and the number of particles $N$;

\item Construct all the $N$-particle states with given $M$.  You will validate the code by  comparing both the number of states and specific states.
\end{itemize}

\noindent
% --- end paragraph admon ---



\subsection*{Shell-model project}

% --- begin paragraph admon ---
\paragraph{}
The format of a possible input  file could be

\begin{quote}
\begin{tabular}{ccccc}
\hline
\multicolumn{1}{c}{ Index } & \multicolumn{1}{c}{ $n$ } & \multicolumn{1}{c}{ $l$ } & \multicolumn{1}{c}{ $2j$ } & \multicolumn{1}{c}{ $2m_j$ } \\
\hline
1     & 1   & 0   & 1    & -1     \\
2     & 1   & 0   & 1    & 1      \\
3     & 0   & 2   & 3    & -3     \\
4     & 0   & 2   & 3    & -1     \\
5     & 0   & 2   & 3    & 1      \\
6     & 0   & 2   & 3    & 3      \\
7     & 0   & 2   & 5    & -5     \\
8     & 0   & 2   & 5    & -3     \\
9     & 0   & 2   & 5    & -1     \\
10    & 0   & 2   & 5    & 1      \\
11    & 0   & 2   & 5    & 3      \\
12    & 0   & 2   & 5    & 5      \\
\hline
\end{tabular}
\end{quote}

\noindent
This represents the $1s_{1/2}0d_{3/2}0d_{5/2}$ valence space, or just the $sd$-space.  There are 
twelve single-particle states, labeled by an overall index, and which have associated quantum 
numbers the number of radial nodes, the orbital angular momentum $l$, and the 
angular momentum $j$ and third component $j_z$.  To keep everything as integers, we could store $2 \times j$ and 
$2 \times j_z$.
% --- end paragraph admon ---



\subsection*{Shell-model project}

% --- begin paragraph admon ---
\paragraph{}
To read in the single-particle states you need to:
\begin{itemize}
\item Open the file 
\begin{itemize}

 \item Read the number of single-particle states (in the above example, 12);  allocate memory; all you need is a single array storing $2\times j_z$ for each state, labeled by the index.

\end{itemize}

\noindent
\item Read in the quantum numbers and store $2 \times j_z$ (and anything else you happen to want).
\end{itemize}

\noindent
% --- end paragraph admon ---



\subsection*{Shell-model project}

% --- begin paragraph admon ---
\paragraph{}

The next step is to read in the number of particles $N$ and the fixed total $M$ (or, actually, $2 \times M$). 
For this project we assume only a single species of particles, say neutrons, although this can be 
relaxed. \textbf{Note}: Although it is often a good idea to try to write a more general code, given the 
short time alloted we would suggest you keep your ambition in check, at least in the initial phases of the 
project.  

You should probably write an error trap to make sure $N$ and $M$ are congruent; if $N$ is even, then 
$2 \times M$ should be even, and if $N$ is odd then $2\times M$ should be odd.
% --- end paragraph admon ---



\subsection*{Shell-model project}

% --- begin paragraph admon ---
\paragraph{}
The final step is to generate the set of $N$-particle Slater determinants with fixed $M$. 
The Slater determinants will be stored in occupation representation.  Although in many codes
this representation is done compactly in bit notation with ones and zeros, but for 
greater transparency and simplicity we will list the occupied single particle states.

 Hence we can 
store the Slater determinant basis states as $sd(i,j)$, that is an 
array of dimension $N_{SD}$, the number of Slater determinants, by $N$, the number of occupied 
state. So if for the 7th Slater determinant the 2nd, 3rd, and 9th single-particle states are occupied, 
then $sd(7,1) = 2$, $sd(7,2) = 3$, and $sd(7,3) = 9$.
% --- end paragraph admon ---



\subsection*{Shell-model project}

% --- begin paragraph admon ---
\paragraph{}

We can construct an occupation representation of Slater determinants by the \emph{odometer}
method.  Consider $N_{sp} = 12$ and $N=4$. 
Start with the first 4 states occupied, that is:

\begin{itemize}
\item $sd(1,:)= 1,2,3,4$ (also written as $|1,2,3,4 \rangle$)
\end{itemize}

\noindent
Now increase the last occupancy recursively:
\begin{itemize}
\item $sd(2,:)= 1,2,3,5$

\item $sd(3,:)= 1,2,3,6$

\item $sd(4,:)= 1,2,3,7$

\item $\ldots$

\item $sd(9,:)= 1,2,3,12$
\end{itemize}

\noindent
Then start over with 
\begin{itemize}
\item $sd(10,:)= 1,2,4,5$
\end{itemize}

\noindent
and again increase the rightmost digit

\begin{itemize}
\item $sd(11,:)= 1,2,4,6$

\item $sd(12,:)= 1,2,4,7$

\item $\ldots$

\item $sd(17,:)= 1,2,4,12$
\end{itemize}

\noindent
% --- end paragraph admon ---



\subsection*{Shell-model project}

% --- begin paragraph admon ---
\paragraph{}
When we restrict ourselves to an $M$-scheme basis, we could choose two paths. 
The first is simplest (and simplest is often best, at 
least in the first draft of a code): generate all possible Slater determinants, 
and then extract from this initial list a list of those Slater determinants with a given 
$M$. (You will need to write a short function or routine that computes $M$ for any 
given occupation.)  

Alternately, and not too difficult, is to run the odometer routine twice: each time, as 
as a Slater determinant is calculated, compute $M$, but do not store the Slater determinants 
except the current one. You can then count up the number of Slater determinants with a 
chosen $M$.  Then allocated storage for the Slater determinants, and run the odometer 
algorithm again, this time storing Slater determinants with the desired $M$ (this can be 
done with a simple logical flag).
% --- end paragraph admon ---



\subsection*{Shell-model project}

% --- begin paragraph admon ---
\paragraph{}

\emph{Some example solutions}:  Let's begin with a simple case, the $0d_{5/2}$ space containing six single-particle states


\begin{quote}
\begin{tabular}{ccccc}
\hline
\multicolumn{1}{c}{ Index } & \multicolumn{1}{c}{ $n$ } & \multicolumn{1}{c}{ $l$ } & \multicolumn{1}{c}{ $j$ } & \multicolumn{1}{c}{ $m_j$ } \\
\hline
1     & 0   & 2   & 5/2 & -5/2  \\
2     & 0   & 2   & 5/2 & -3/2  \\
3     & 0   & 2   & 5/2 & -1/2  \\
4     & 0   & 2   & 5/2 & 1/2   \\
5     & 0   & 2   & 5/2 & 3/2   \\
6     & 0   & 2   & 5/2 & 5/2   \\
\hline
\end{tabular}
\end{quote}

\noindent
For two particles, there are a total of 15 states, which we list here with the total $M$:
\begin{itemize}
\item $\vert 1,2 \rangle$, $M= -4$,  $\vert 1,3 \rangle$, $M= -3$

\item $\vert  1,4 \rangle$, $M= -2$, $\vert 1,5 \rangle$, $M= -1$

\item $\vert 1,5 \rangle$, $M= 0$, $vert 2,3 \rangle$, $M= -2$

\item $\vert 2,4 \rangle$, $M= -1$, $\vert 2,5 \rangle$, $M= 0$

\item $\vert 2,6 \rangle$, $M= 1$, $\vert 3,4 \rangle$, $M= 0$

\item $\vert 3,5 \rangle$, $M= 1$, $\vert 3,6 \rangle$, $M= 2$

\item $\vert 4,5 \rangle$, $M= 2$, $\vert 4,6 \rangle$, $M= 3$

\item $\vert 5,6 \rangle$, $M= 4$
\end{itemize}

\noindent
Of these, there are only 3 states with $M=0$.
% --- end paragraph admon ---



\subsection*{Shell-model project}

% --- begin paragraph admon ---
\paragraph{}
\emph{You should try} by hand to show that in this same single-particle space, that for 
$N=3$ there are 3 states with $M=1/2$ and for $N= 4$ there are also only 3 states with $M=0$. 

\emph{To test your code}, confirm the above. 

Also, 
for the $sd$-space given above, for $N=2$ there are 14 states with $M=0$, for $N=3$ there are 37 
states with $M=1/2$, for $N=4$ there are 81 states with $M=0$.
% --- end paragraph admon ---



\subsection*{Shell-model project}

% --- begin paragraph admon ---
\paragraph{}
For our project, we will only consider the pairing model.
A simple space is the $(1/2)^2$ space with four single-particle states


\begin{quote}
\begin{tabular}{ccccc}
\hline
\multicolumn{1}{c}{ Index } & \multicolumn{1}{c}{ $n$ } & \multicolumn{1}{c}{ $l$ } & \multicolumn{1}{c}{ $s$ } & \multicolumn{1}{c}{ $m_s$ } \\
\hline
1     & 0   & 0   & 1/2 & -1/2  \\
2     & 0   & 0   & 1/2 & 1/2   \\
3     & 1   & 0   & 1/2 & -1/2  \\
4     & 1   & 0   & 1/2 & 1/2   \\
\hline
\end{tabular}
\end{quote}

\noindent
For $N=2$ there are 4 states with $M=0$; show this by hand and confirm your code reproduces it.
% --- end paragraph admon ---



\subsection*{Shell-model project}

% --- begin paragraph admon ---
\paragraph{}
Another, slightly more challenging space is the $(1/2)^4$ space, that is, 
with eight  single-particle states we have


\begin{quote}
\begin{tabular}{ccccc}
\hline
\multicolumn{1}{c}{ Index } & \multicolumn{1}{c}{ $n$ } & \multicolumn{1}{c}{ $l$ } & \multicolumn{1}{c}{ $s$ } & \multicolumn{1}{c}{ $m_s$ } \\
\hline
1     & 0   & 0   & 1/2 & -1/2  \\
2     & 0   & 0   & 1/2 & 1/2   \\
3     & 1   & 0   & 1/2 & -1/2  \\
4     & 1   & 0   & 1/2 & 1/2   \\
5     & 2   & 0   & 1/2 & -1/2  \\
6     & 2   & 0   & 1/2 & 1/2   \\
7     & 3   & 0   & 1/2 & -1/2  \\
8     & 3   & 0   & 1/2 & 1/2   \\
\hline
\end{tabular}
\end{quote}

\noindent
For $N=2$ there are 16 states with $M=0$; for $N=3$ there are 24 states with $M=1/2$, and for 
$N=4$ there are 36 states with $M=0$.
% --- end paragraph admon ---



\subsection*{Shell-model project}

% --- begin paragraph admon ---
\paragraph{}
In the shell-model context we can interpret this as 4 $s_{1/2}$ levels, with $m = \pm 1/2$, we can also think of these are simple four pairs,  $\pm k, k = 1,2,3,4$. Later on we will 
assign single-particle energies,  depending on the radial quantum number $n$, that is, 
$\epsilon_k = |k| \delta$ so that they are equally spaced.
% --- end paragraph admon ---



\subsection*{Shell-model project}

% --- begin paragraph admon ---
\paragraph{}

For application in the pairing model we can go further and consider only states with 
no ``broken pairs,'' that is, if $+k$ is filled (or $m = +1/2$, so is $-k$ ($m=-1/2$). 
If you want, you can write your code to accept only these, and obtain the following 
six states:

\begin{itemize}
\item $|           1,           2 ,          3         ,       4  \rangle , $

\item $|            1      ,     2        ,        5         ,       6 \rangle , $

\item $|            1         ,       2     ,           7         ,       8  \rangle , $

\item $|            3        ,        4      ,          5          ,      6  \rangle , $

\item $|            3        ,        4      ,          7         ,       8  \rangle , $

\item $|            5        ,        6     ,           7     ,           8  \rangle $
\end{itemize}

\noindent
% --- end paragraph admon ---



\subsection*{Shell-model project}

% --- begin paragraph admon ---
\paragraph{Hints for coding.}

\begin{itemize}
\item Write small modules (routines/functions) ; avoid big functions  that do everything. (But not too small.)

\item Use Unit tests! Write lots of error traps, even for things that are `obvious.'

\item Document as you go along. The Unit tests serve as documentation. For each function write a header that includes: 
\begin{enumerate}

\item Main purpose of function and/or unit test

\item names and  brief explanation of input variables, if any 

\item names and brief explanation of output variables, if any

\item functions called by this function

\item called by which functions
\end{enumerate}

\noindent
\end{itemize}

\noindent
% --- end paragraph admon ---



\subsection*{Shell-model project}

% --- begin paragraph admon ---
\paragraph{}

Hints for coding

\begin{itemize}
\item Unit tests will save time. Use also IDEs for debugging. If you insist on brute force debugging, print out intermediate values. It's almost impossible to debug a  code by looking at it--the code will almost always win a `staring contest.'

\item Validate code with SIMPLE CASES. Validate early and often.  Unit tests!! 
\end{itemize}

\noindent
The number one mistake is using a too complex a system to test. For example ,
if you are computing particles in a potential in a box, try removing the potential--you should get 
particles in a box. And start with one particle, then two, then three... Don't start with 
eight particles.
% --- end paragraph admon ---



\subsection*{Shell-model project}

% --- begin paragraph admon ---
\paragraph{}

Our recommended occupation representation, e.g.~$| 1,2,4,8 \rangle$, is 
easy to code, but numerically inefficient when one has hundreds of 
millions of Slater determinants.

In state-of-the-art shell-model codes, one generally uses bit 
representation, i.e.~$|1101000100... \rangle$ where one stores 
the Slater determinant as a single (or a small number of) integer.

This is much more compact, but more intricate to code with considerable 
more overhead. There exist 
bit-manipulation functions. We will discuss these in more detail at the beginning of the third week.
% --- end paragraph admon ---



\subsection*{Example case: pairing Hamiltonian}

% --- begin paragraph admon ---
\paragraph{}

We consider a space with $2\Omega$ single-particle states, with each 
state labeled by 
$k = 1, 2, 3, \Omega$ and $m = \pm 1/2$. The convention is that 
the state with $k>0$ has $m = + 1/2$ while $-k$ has $m = -1/2$.

The Hamiltonian we consider is 
\[
\hat{H} = -G \hat{P}_+ \hat{P}_-,
\]
where
\[
\hat{P}_+ = \sum_{k > 0} \hat{a}^\dagger_k \hat{a}^\dagger_{-{k}}.
\]
and $\hat{P}_- = ( \hat{P}_+)^\dagger$.

This problem can be solved using what is called the quasi-spin formalism to obtain the 
exact results. Thereafter we will try again using the explicit Slater determinant formalism.
% --- end paragraph admon ---



\subsection*{Example case: pairing Hamiltonian}

% --- begin paragraph admon ---
\paragraph{}

One can show (and this is part of the project) that
\[
\left [ \hat{P}_+, \hat{P}_- \right ] = \sum_{k> 0} \left( \hat{a}^\dagger_k \hat{a}_k 
+ \hat{a}^\dagger_{-{k}} \hat{a}_{-{k}} - 1 \right) = \hat{N} - \Omega.
\]
Now define 
\[
\hat{P}_z = \frac{1}{2} ( \hat{N} -\Omega).
\]
Finally you can show
\[
\left [ \hat{P}_z , \hat{P}_\pm \right ] = \pm \hat{P}_\pm.
\]
This means the operators $\hat{P}_\pm, \hat{P}_z$ form a so-called  $SU(2)$ algebra, and we can 
use all our insights about angular momentum, even though there is no actual 
angular momentum involved.

So we rewrite the Hamiltonian to make this explicit:
\[
\hat{H} = -G \hat{P}_+ \hat{P}_- 
= -G \left( \hat{P}^2 - \hat{P}_z^2 + \hat{P}_z\right)
\]
% --- end paragraph admon ---



\subsection*{Example case: pairing Hamiltonian}

% --- begin paragraph admon ---
\paragraph{}

Because of the SU(2) algebra, we know that the eigenvalues of 
$\hat{P}^2$ must be of the form $p(p+1)$, with $p$ either integer or half-integer, and the eigenvalues of $\hat{P}_z$ 
are $m_p$ with $p \geq | m_p|$, with $m_p$ also integer or half-integer. 

But because $\hat{P}_z = (1/2)(\hat{N}-\Omega)$, we know that for $N$ particles 
the value $m_p = (N-\Omega)/2$. Furthermore, the values of $m_p$ range from 
$-\Omega/2$ (for $N=0$) to $+\Omega/2$ (for $N=2\Omega$, with all states filled). 

We deduce the maximal $p = \Omega/2$ and for a given $n$ the 
values range of $p$ range from $|N-\Omega|/2$ to $\Omega/2$ in steps of 1 
(for an even number of particles) 

Following Racah we introduce the notation
$p = (\Omega - v)/2$
where $v = 0, 2, 4,..., \Omega - |N-\Omega|$ 
With this it is easy to deduce that the eigenvalues of the pairing Hamiltonian are
\[
-G(N-v)(2\Omega +2-N-v)/4
\]
This also works for $N$ odd, with $v= 1,3,5, \dots$.
% --- end paragraph admon ---



\subsection*{Example case: pairing Hamiltonian}

% --- begin paragraph admon ---
\paragraph{}

Let's take a specific example: $\Omega = 3$ so there are 6 single-particle states, 
and $N = 3$, with $v= 1,3$. Therefore there are two distinct eigenvalues, 
\[
E = -2G, 0
\]
Now let's work this out explicitly. The single particle degrees of freedom are defined as


\begin{quote}
\begin{tabular}{ccc}
\hline
\multicolumn{1}{c}{ Index } & \multicolumn{1}{c}{ $k$ } & \multicolumn{1}{c}{ $m$ } \\
\hline
1     & 1   & -1/2 \\
2     & -1  & 1/2  \\
3     & 2   & -1/2 \\
4     & -2  & 1/2  \\
5     & 3   & -1/2 \\
6     & -3  & 1/2  \\
\hline
\end{tabular}
\end{quote}

\noindent
 There are  $\left( \begin{array}{c}6 \\ 3 \end{array} \right) = 20$ three-particle states, but there 
are 9 states with $M = +1/2$, namely
$| 1,2,3 \rangle, |1,2,5\rangle, | 1,4,6 \rangle, | 2,3,4 \rangle, |2,3,6 \rangle, | 2,4,5 \rangle, | 2, 5, 6 \rangle, |3,4,6 \rangle, | 4,5,6 \rangle$.
% --- end paragraph admon ---



\subsection*{Example case: pairing Hamiltonian}

% --- begin paragraph admon ---
\paragraph{}

In this basis, the operator 
\[
\hat{P}_+
= \hat{a}^\dagger_1 \hat{a}^\dagger_2 + \hat{a}^\dagger_3 \hat{a}^\dagger_4 +
\hat{a}^\dagger_5 \hat{a}^\dagger_6 
\]
From this we can determine that 
\[
\hat{P}_- | 1, 4, 6 \rangle = \hat{P}_- | 2, 3, 6 \rangle
= \hat{P}_- | 2, 4, 5 \rangle = 0
\]
so those states all have eigenvalue 0.
% --- end paragraph admon ---



\subsection*{Example case: pairing Hamiltonian}

% --- begin paragraph admon ---
\paragraph{}
Now for further example, 
\[
\hat{P}_- | 1,2,3 \rangle = | 3 \rangle
\]
so
\[
\hat{P}_+ \hat{P}_- | 1,2,3\rangle = | 1,2,3\rangle+ | 3,4,3\rangle + | 5,6,3\rangle
\]
The second term vanishes because state 3 is occupied twice, and reordering the last 
term we
get
\[
\hat{P}_+ \hat{P}_- | 1,2,3\rangle = | 1,2,3\rangle+ |3, 5,6\rangle
\]
without picking up a phase.
% --- end paragraph admon ---



\subsection*{Example case: pairing Hamiltonian}

% --- begin paragraph admon ---
\paragraph{}

Continuing in this fashion, with the previous ordering of the many-body states
(  $| 1,2,3 \rangle, |1,2,5\rangle, | 1,4,6 \rangle, | 2,3,4 \rangle, |2,3,6 \rangle, | 2,4,5 \rangle, | 2, 5, 6 \rangle, |3,4,6 \rangle, | 4,5,6 \rangle$) the 
Hamiltonian matrix of this system is 
\[
H = -G\left( 
\begin{array}{ccccccccc}
1 & 0 & 0 & 0 & 0 & 0 & 0 & 0 & 1  \\
0 & 1 & 0 & 0 & 0 & 0 & 0 & 1 & 0  \\
0 & 0 & 0 & 0 & 0 & 0 & 0 & 0 & 0  \\
0 & 0 & 0 & 1 & 0 & 0 & 1 & 0 & 0  \\
0 & 0 & 0 & 0 & 0 & 0 & 0 & 0 & 0  \\
0 & 0 & 0 & 0 & 0 & 0 & 0 & 0 & 0  \\
0 & 0 & 0 & 1 & 0 & 0 & 1 & 0 & 0  \\
0 & 0 & 0 & 0 & 0 & 0 & 0 & 0 & 0  \\
0 & 1 & 0 & 0 & 0 & 0 & 0 & 1 & 0  \\
1 & 0 & 0 & 0 & 0 & 0 & 0 & 0 & 1  
\end{array} \right )
\] 
This is useful for our project.  One can by hand confirm 
that there are 3 eigenvalues $-2G$ and 6 with value zero.
% --- end paragraph admon ---



\subsection*{Example case: pairing Hamiltonian}

% --- begin paragraph admon ---
\paragraph{}

Another example
Using the $(1/2)^4$ single-particle space, resulting in eight single-particle states


\begin{quote}
\begin{tabular}{ccccc}
\hline
\multicolumn{1}{c}{ Index } & \multicolumn{1}{c}{ $n$ } & \multicolumn{1}{c}{ $l$ } & \multicolumn{1}{c}{ $s$ } & \multicolumn{1}{c}{ $m_s$ } \\
\hline
1     & 0   & 0   & 1/2 & -1/2  \\
2     & 0   & 0   & 1/2 & 1/2   \\
3     & 1   & 0   & 1/2 & -1/2  \\
4     & 1   & 0   & 1/2 & 1/2   \\
5     & 2   & 0   & 1/2 & -1/2  \\
6     & 2   & 0   & 1/2 & 1/2   \\
7     & 3   & 0   & 1/2 & -1/2  \\
8     & 3   & 0   & 1/2 & 1/2   \\
\hline
\end{tabular}
\end{quote}

\noindent
and then taking only 4-particle, $M=0$ states that have no `broken pairs', there are six basis Slater 
determinants:

\begin{itemize}
\item $|           1,           2 ,          3         ,       4  \rangle , $

\item $|            1      ,     2        ,        5         ,       6 \rangle , $

\item $|            1         ,       2     ,           7         ,       8  \rangle , $

\item $|            3        ,        4      ,          5          ,      6  \rangle , $

\item $|            3        ,        4      ,          7         ,       8  \rangle , $

\item $|            5        ,        6     ,           7     ,           8  \rangle $
\end{itemize}

\noindent
% --- end paragraph admon ---



\subsection*{Example case: pairing Hamiltonian}

% --- begin paragraph admon ---
\paragraph{}

Now we take the following Hamiltonian
\[
\hat{H} = \sum_n n \delta \hat{N}_n  - G \hat{P}^\dagger \hat{P}
\]
where 
\[
\hat{N}_n = \hat{a}^\dagger_{n, m=+1/2} \hat{a}_{n, m=+1/2} +
\hat{a}^\dagger_{n, m=-1/2} \hat{a}_{n, m=-1/2}
\]
and
\[
\hat{P}^\dagger = \sum_{n} \hat{a}^\dagger_{n, m=+1/2} \hat{a}^\dagger_{n, m=-1/2} 
\]
We can write down the $ 6 \times 6$  Hamiltonian in the basis from the prior slide:
\[
H = \left ( 
\begin{array}{cccccc}
2\delta -2G & -G & -G & -G & -G & 0 \\
 -G & 4\delta -2G & -G & -G & -0 & -G \\
-G & -G & 6\delta -2G & 0 & -G & -G \\
 -G & -G & 0 & 6\delta-2G & -G & -G \\
 -G & 0 & -G & -G & 8\delta-2G & -G \\
0 & -G & -G & -G & -G & 10\delta -2G 
\end{array} \right )
\]
(You should check by hand that this is correct.) 

For $\delta = 0$ we have the closed form solution of  the g.s. energy given by $-6G$.
% --- end paragraph admon ---



\subsection*{Building a Hamiltonian matrix}

% --- begin paragraph admon ---
\paragraph{}
The goal is to compute the matrix elements of the Hamiltonian, specifically
matrix elements between many-body states (Slater determinants) of two-body
operators
\[
\sum_{p < q, r < s}V_{pqr} \hat{a}^\dagger_p \hat{a}^\dagger_q\hat{a}_s \hat{a}_r
\]
In particular we will need to compute
\[
\langle \beta | \hat{a}^\dagger_p \hat{a}^\dagger_q\hat{a}_s \hat{a}_r |\alpha \rangle
\]
where $\alpha, \beta$ are indices labeling Slater determinants and $p,q,r,s$ label
single-particle states.
% --- end paragraph admon ---



\subsection*{Building a Hamiltonian matrix}

% --- begin paragraph admon ---
\paragraph{}
Note: there are other, more efficient ways to do this than the method we describe, 
but you will
be able to produce a working code quickly.

As we coded in the first step,
a Slater determinant $| \alpha \rangle$ with index $\alpha$ is a
list of $N$ occupied single-particle states $i_1 < i_2 < i_3 \ldots i_N$.

Furthermore, for the two-body matrix elements $V_{pqrs}$ we normally assume
$p < q$ and $r < s$. For our specific project, the interaction is much simpler and you can use this to simplify considerably the setup of a shell-model code for project 2.

What follows here is a more general, but still brute force, approach.
% --- end paragraph admon ---



\subsection*{Building a Hamiltonian matrix}

% --- begin paragraph admon ---
\paragraph{}
Write a function that:
\begin{enumerate}
\item Has as input the single-particle indices $p,q,r,s$ for the two-body operator and the index $\alpha$ for the ket Slater determinant;

\item Returns the index $\beta$ of the unique (if any) Slater determinant such that
\end{enumerate}

\noindent
\[
| \beta \rangle = \pm \hat{a}^\dagger_p \hat{a}^\dagger_q\hat{a}_s \hat{a}_r |\alpha \rangle
\]
as well as the phase

This is equivalent to computing
\[
\langle \beta | \hat{a}^\dagger_p \hat{a}^\dagger_q\hat{a}_s \hat{a}_r |\alpha \rangle
\]
% --- end paragraph admon ---



\subsection*{Building a Hamiltonian matrix, first step}

% --- begin paragraph admon ---
\paragraph{}
The first step can take as input an initial Slater determinant
(whose position in the list of basis Slater determinants is $\alpha$) written as an
ordered listed of occupied single-particle states, e.g.~$1,2,5,8$, and the
indices $p,q,r,s$ from the two-body operator. 

It will return another final Slater determinant if the single-particle states $r$ and $s$ are occupied, else it will return an 
empty Slater determinant
(all zeroes). 

If $r$ and $s$ are in the list of occupied single particle states, then
replace the initial single-particle states $ij$ as $i \rightarrow r$ and $j \rightarrow r$.
% --- end paragraph admon ---



\subsection*{Building a Hamiltonian matrix, second step}

% --- begin paragraph admon ---
\paragraph{}
The second step will take the final Slater determinant 
from the first step (if not empty),
and then order by pairwise permutations (i.e., if the Slater determinant is
$i_1, i_2, i_3, \ldots$, then if $i_n > i_{n+1}$, interchange 
$i_n \leftrightarrow i_{n+1}$.
% --- end paragraph admon ---



\subsection*{Building a Hamiltonian matrix}

% --- begin paragraph admon ---
\paragraph{}

It will also output a phase.  If any two single-particle occupancies are repeated, 
the phase is
0.  Otherwise it is +1 for an even permutation and -1 for an odd permutation to 
bring the final
Slater determinant into ascending order, $j_1 < j_2 < j_3 \ldots$.
% --- end paragraph admon ---



\subsection*{Building a Hamiltonian matrix}

% --- begin paragraph admon ---
\paragraph{}
\textbf{Example}: Suppose in the $sd$ single-particle space that the initial 
Slater determinant
is $1,3,9,12$. If $p,q,r,s = 2,8,1,12$, then after the first step the final Slater determinant
is $2,3,9,8$.  The second step will return $2,3,8,9$ and a phase of -1, 
because an odd number  of interchanges is required.
% --- end paragraph admon ---



\subsection*{Building a Hamiltonian matrix}

% --- begin paragraph admon ---
\paragraph{}

\textbf{Example}: Suppose in the $sd$ single-particle space that the initial 
Slater determinant
is $1,3,9,12$. If $p,q,r,s = 3,8,1,12$, then after the first step the 
final  Slater determinant
is $3,3,9,8$, but after the second step the phase is 0 
because the single-particle state 3 is
occupied twice.

Lastly, the final step  takes the ordered final Slater determinant and 
we search through the basis list to
determine its index in the many-body basis, that is, $\beta$.
% --- end paragraph admon ---



\subsection*{Building a Hamiltonian matrix}

% --- begin paragraph admon ---
\paragraph{}

The Hamiltonian is then stored as an $N_{SD} \times N_{SD}$ array of real numbers, which
can be allocated once you have created the many-body basis and know $N_{SD}$.
% --- end paragraph admon ---



\subsection*{Building a Hamiltonian matrix}

% --- begin paragraph admon ---
\paragraph{}

\begin{enumerate}
\item Initialize $H(\alpha,\beta)=0.0$

\item Set up an outer loop over $\beta$

\item Loop over $\alpha = 1, NSD$

\item For each $\alpha$, loop over $a=1,ntbme$  and fetch $V(a)$ and the single-particle indices $p,q,r,s$ 

\item If $V(a) = 0$ skip.  Otherwise, apply $\hat{a}^\dagger_p\hat{a}^\dagger_q \hat{a}_s \hat{a}_r$ to the Slater determinant labeled by $\alpha$.

\item Find, if any, the label $\beta$ of the resulting Slater determinant and the phase (which is 0, +1, -1).

\item If phase $\neq 0$, then update $H(\alpha,\beta)$  as $H(\alpha,\beta) + phase*V(a)$. The sum is important because multiple operators might contribute to the same matrix element.

\item Continue loop over $a$

\item Continue loop over $\alpha$.

\item End the outer loop over $\beta$.
\end{enumerate}

\noindent
You should force the resulting matrix $H$ to be symmetric. To do this, when
updating $H(\alpha,\beta)$, if $\alpha \neq \beta$, also update $H(\beta,\alpha)$.
% --- end paragraph admon ---



\subsection*{Building a Hamiltonian matrix}

% --- begin paragraph admon ---
\paragraph{}

You will also need to include the single-particle energies. This is easy: they only
contribute to diagonal matrix elements, that is, $H(\alpha,\alpha)$.  
Simply find the occupied single-particle states $i$ and add the corresponding $\epsilon(i)$.
% --- end paragraph admon ---



\subsection*{Hamiltonian matrix without the bit representation}

% --- begin paragraph admon ---
\paragraph{}

Consider the many-body state $\Psi_{\lambda}$ expressed as linear combinations of
Slater determinants ($SD$) of orthonormal single-particle states $\phi({\bf r})$:
\begin{equation}
\Psi_{\lambda} = \sum_i C_{\lambda i} SD_i
\end{equation}
Using the Slater-Condon rules the matrix elements of any one-body
($\cal{O}_1$) or two-body ($\cal{O}_2$) operator expressed in the
determinant space have simple expressions involving one- and two-fermion
integrals in our given single-particle basis.
The diagonal elements are given by:
\begin{eqnarray}
  \langle SD | \cal{O}_1 | SD \rangle & = & \sum_{i \in SD} \langle \phi_i | \cal{O}_1 | \phi_i \rangle \\
  \langle SD | \cal{O}_2 | SD \rangle & = & \frac{1}{2} \sum_{(i,j) \in SD}  
      \langle \phi_i \phi_j | \cal{O}_2 | \phi_i \phi_j \rangle - \nonumber \\
 & & 
      \langle \phi_i \phi_j | \cal{O}_2 | \phi_j \phi_i \rangle \nonumber 
\end{eqnarray}
% --- end paragraph admon ---



\subsection*{Hamiltonian matrix without the bit representation, one and two-body operators}

% --- begin paragraph admon ---
\paragraph{}

For two determinants which differ only by the substitution of single-particle states $i$ with
a single-particle state $j$:
\begin{eqnarray}
  \langle SD | \cal{O}_1 | SD_i^j \rangle & = & \langle \phi_i | \cal{O}_1 | \phi_j \rangle \\
  \langle SD | \cal{O}_2 | SD_i^j \rangle & = & \sum_{k \in SD} 
      \langle \phi_i \phi_k | \cal{O}_2 | \phi_j \phi_k \rangle - 
      \langle \phi_i \phi_k | \cal{O}_2 | \phi_k \phi_j \rangle \nonumber
\end{eqnarray}
For two determinants which differ by two single-particle states
\begin{eqnarray}
  \langle SD | \cal{O}_1 | SD_{ik}^{jl} \rangle & = & 0 \\
  \langle SD | \cal{O}_2 | SD_{ik}^{jl} \rangle & = & 
      \langle \phi_i \phi_k | \cal{O}_2 | \phi_j \phi_l \rangle -
      \langle \phi_i \phi_k | \cal{O}_2 | \phi_l \phi_j \rangle \nonumber 
\end{eqnarray}
All other matrix elements involving determinants with more than two
substitutions are zero.
% --- end paragraph admon ---



\subsection*{Strategies for setting up an algorithm}

% --- begin paragraph admon ---
\paragraph{}

An efficient implementation of these rules requires

\begin{itemize}
\item to find the number of single-particle state substitutions between two determinants

\item to find which single-particle states are involved in the substitution

\item to compute the phase factor if a reordering of the single-particle states has occured
\end{itemize}

\noindent
We can solve this problem using our odometric approach or alternatively using a bit representation as discussed below and in more detail in 

\begin{itemize}
\item \href{{https://github.com/scemama/slater_condon}}{Scemama and Gimer's article (Fortran codes)}

\item \href{{https://arxiv.org/abs/0810.2644}}{Simen Kvaal's article on how to build an FCI code (C++ code)}
\end{itemize}

\noindent
We recommend in particular the article by Simen Kvaal. It contains nice general classes for creation and annihilation operators as well as the calculation of the phase (see below).
% --- end paragraph admon ---



\subsection*{Computing expectation values and transitions in the shell-model}

% --- begin paragraph admon ---
\paragraph{}
When we diagonalize the Hamiltonian matrix, the eigenvectors are the coefficients $C_{\lambda i}$ used 
to express the many-body state $\Psi_{\lambda}$ in terms of  a linear combinations of
Slater determinants ($SD$) of orthonormal single-particle states $\phi({\bf r})$.

With these eigenvectors we can compute say the transition likelyhood of a one-body operator as
\[
\langle \Psi_{\lambda} \vert \cal{O}_1 \vert \Psi_{\sigma} \rangle  = 
\sum_{ij}C_{\lambda i}^*C_{\sigma j}  \langle SD_i | \cal{O}_1 | SD_j \rangle .
\]
Writing the one-body operator in second quantization as 
\[
\cal{O}_1  = \sum_{pq} \langle p \vert \cal{o}_1 \vert q\rangle a_p^{\dagger} a_q, 
\]
we have
\[
\langle \Psi_{\lambda} \vert \cal{O}_1 \vert \Psi_{\sigma} \rangle  = 
\sum_{pq}\langle p \vert \cal{o}_1 \vert q\rangle \sum_{ij}C_{\lambda i}^*C_{\sigma j}  \langle SD_i |a_p^{\dagger} a_q | SD_j \rangle .
\]
% --- end paragraph admon ---



\subsection*{Computing expectation values and transitions in the shell-model and spectroscopic factors}

% --- begin paragraph admon ---
\paragraph{}
The terms we need to evalute then are just the elements 
\[
\langle SD_i |a_p^{\dagger} a_q | SD_j \rangle, 
\]
which can be rewritten in terms of spectroscopic factors by inserting a complete set of Slater determinats as
\[
\langle SD_i |a_p^{\dagger} a_q | SD_j \rangle = \sum_{l}\langle SD_i \vert a_p^{\dagger}\vert SD_l\rangle \langle SD_l \vert  a_q \vert SD_j \rangle,
\]
where $\langle SD_l\vert a_q(a_p^{\dagger})\vert SD_j\rangle$ are the spectroscopic factors. These can be easily evaluated in $m$-scheme. Using the Wigner-Eckart theorem we can transform these to a $J$-coupled scheme through so-called reduced matrix elements.
% --- end paragraph admon ---



\subsection*{Operators in second quantization}

% --- begin paragraph admon ---
\paragraph{}
In the build-up of a shell-model or FCI code that is meant to tackle large dimensionalities
we need to deal with the action of the Hamiltonian $\hat{H}$ on a
Slater determinant represented in second quantization as
\[
 |\alpha_1\dots \alpha_n\rangle = a_{\alpha_1}^{\dagger} a_{\alpha_2}^{\dagger} \dots a_{\alpha_n}^{\dagger} |0\rangle.
\]
The time consuming part stems from the action of the Hamiltonian
on the above determinant,
\[
\left(\sum_{\alpha\beta} \langle \alpha|t+u|\beta\rangle a_\alpha^{\dagger} a_\beta + \frac{1}{4} \sum_{\alpha\beta\gamma\delta}
                \langle \alpha \beta|\hat{v}|\gamma \delta\rangle a_\alpha^{\dagger} a_\beta^{\dagger} a_\delta a_\gamma\right)a_{\alpha_1}^{\dagger} a_{\alpha_2}^{\dagger} \dots a_{\alpha_n}^{\dagger} |0\rangle.
\]
A practically useful way to implement this action is to encode a Slater determinant as a bit pattern.
% --- end paragraph admon ---



\subsection*{Operators in second quantization}

% --- begin paragraph admon ---
\paragraph{}
Assume that we have at our disposal $n$ different single-particle states
$\alpha_0,\alpha_2,\dots,\alpha_{n-1}$ and that we can distribute  among these states $N\le n$ particles.

A Slater  determinant can then be coded as an integer of $n$ bits. As an example, if we have $n=16$ single-particle states
$\alpha_0,\alpha_1,\dots,\alpha_{15}$ and $N=4$ fermions occupying the states $\alpha_3$, $\alpha_6$, $\alpha_{10}$ and $\alpha_{13}$
we could write this Slater determinant as  
\[
\Phi_{\Lambda} = a_{\alpha_3}^{\dagger} a_{\alpha_6}^{\dagger} a_{\alpha_{10}}^{\dagger} a_{\alpha_{13}}^{\dagger} |0\rangle.
\]
The unoccupied single-particle states have bit value $0$ while the occupied ones are represented by bit state $1$. 
In the binary notation we would write this   16 bits long integer as
\[
\begin{array}{cccccccccccccccc}
{\alpha_0}&{\alpha_1}&{\alpha_2}&{\alpha_3}&{\alpha_4}&{\alpha_5}&{\alpha_6}&{\alpha_7} & {\alpha_8} &{\alpha_9} & {\alpha_{10}} &{\alpha_{11}} &{\alpha_{12}} &{\alpha_{13}} &{\alpha_{14}} & {\alpha_{15}} \\
{0} & {0} &{0} &{1} &{0} &{0} &{1} &{0} &{0} &{0} &{1} &{0} &{0} &{1} &{0} & {0} \\
\end{array}
\]
which translates into the decimal number
\[
2^3+2^6+2^{10}+2^{13}=9288.
\]
We can thus encode a Slater determinant as a bit pattern.
% --- end paragraph admon ---



\subsection*{Operators in second quantization}

% --- begin paragraph admon ---
\paragraph{}
With $N$ particles that can be distributed over $n$ single-particle states, the total number of Slater determinats (and defining thereby the dimensionality of the system) is
\[
\mathrm{dim}(\mathcal{H}) = \left(\begin{array}{c} n \\N\end{array}\right).
\]
The total number of bit patterns is $2^n$.
% --- end paragraph admon ---



\subsection*{Operators in second quantization}

% --- begin paragraph admon ---
\paragraph{}
We assume again that we have at our disposal $n$ different single-particle orbits
$\alpha_0,\alpha_2,\dots,\alpha_{n-1}$ and that we can distribute  among these orbits $N\le n$ particles.
The ordering among these states is important as it defines the order of the creation operators.
We will write the determinant 
\[
\Phi_{\Lambda} = a_{\alpha_3}^{\dagger} a_{\alpha_6}^{\dagger} a_{\alpha_{10}}^{\dagger} a_{\alpha_{13}}^{\dagger} |0\rangle,
\]
in a more compact way as 
\[
\Phi_{3,6,10,13} = |0001001000100100\rangle.
\]
The action of a creation operator is thus
\[
a^{\dagger}_{\alpha_4}\Phi_{3,6,10,13} = a^{\dagger}_{\alpha_4}|0001001000100100\rangle=a^{\dagger}_{\alpha_4}a_{\alpha_3}^{\dagger} a_{\alpha_6}^{\dagger} a_{\alpha_{10}}^{\dagger} a_{\alpha_{13}}^{\dagger} |0\rangle,
\]
which becomes
\[
-a_{\alpha_3}^{\dagger} a^{\dagger}_{\alpha_4} a_{\alpha_6}^{\dagger} a_{\alpha_{10}}^{\dagger} a_{\alpha_{13}}^{\dagger} |0\rangle=-|0001101000100100\rangle.
\]
% --- end paragraph admon ---



\subsection*{Operators in second quantization}

% --- begin paragraph admon ---
\paragraph{}
Similarly
\[
a^{\dagger}_{\alpha_6}\Phi_{3,6,10,13} = a^{\dagger}_{\alpha_6}|0001001000100100\rangle=a^{\dagger}_{\alpha_6}a_{\alpha_3}^{\dagger} a_{\alpha_6}^{\dagger} a_{\alpha_{10}}^{\dagger} a_{\alpha_{13}}^{\dagger} |0\rangle,
\]
which becomes
\[
-a^{\dagger}_{\alpha_4} (a_{\alpha_6}^{\dagger})^ 2 a_{\alpha_{10}}^{\dagger} a_{\alpha_{13}}^{\dagger} |0\rangle=0!
\]
This gives a simple recipe:  
\begin{itemize}
\item If one of the bits $b_j$ is $1$ and we act with a creation operator on this bit, we return a null vector

\item If $b_j=0$, we set it to $1$ and return a sign factor $(-1)^l$, where $l$ is the number of bits set before bit $j$.
\end{itemize}

\noindent
% --- end paragraph admon ---



\subsection*{Operators in second quantization}

% --- begin paragraph admon ---
\paragraph{}
Consider the action of $a^{\dagger}_{\alpha_2}$ on various slater determinants:
\[
\begin{array}{ccc}
a^{\dagger}_{\alpha_2}\Phi_{00111}& = a^{\dagger}_{\alpha_2}|00111\rangle&=0\times |00111\rangle\\
a^{\dagger}_{\alpha_2}\Phi_{01011}& = a^{\dagger}_{\alpha_2}|01011\rangle&=(-1)\times |01111\rangle\\
a^{\dagger}_{\alpha_2}\Phi_{01101}& = a^{\dagger}_{\alpha_2}|01101\rangle&=0\times |01101\rangle\\
a^{\dagger}_{\alpha_2}\Phi_{01110}& = a^{\dagger}_{\alpha_2}|01110\rangle&=0\times |01110\rangle\\
a^{\dagger}_{\alpha_2}\Phi_{10011}& = a^{\dagger}_{\alpha_2}|10011\rangle&=(-1)\times |10111\rangle\\
a^{\dagger}_{\alpha_2}\Phi_{10101}& = a^{\dagger}_{\alpha_2}|10101\rangle&=0\times |10101\rangle\\
a^{\dagger}_{\alpha_2}\Phi_{10110}& = a^{\dagger}_{\alpha_2}|10110\rangle&=0\times |10110\rangle\\
a^{\dagger}_{\alpha_2}\Phi_{11001}& = a^{\dagger}_{\alpha_2}|11001\rangle&=(+1)\times |11101\rangle\\
a^{\dagger}_{\alpha_2}\Phi_{11010}& = a^{\dagger}_{\alpha_2}|11010\rangle&=(+1)\times |11110\rangle\\
\end{array}
\]
What is the simplest way to obtain the phase when we act with one annihilation(creation) operator
on the given Slater determinant representation?
% --- end paragraph admon ---



\subsection*{Operators in second quantization}

% --- begin paragraph admon ---
\paragraph{}
We have an SD representation
\[
\Phi_{\Lambda} = a_{\alpha_0}^{\dagger} a_{\alpha_3}^{\dagger} a_{\alpha_6}^{\dagger} a_{\alpha_{10}}^{\dagger} a_{\alpha_{13}}^{\dagger} |0\rangle,
\]
in a more compact way as
\[
\Phi_{0,3,6,10,13} = |1001001000100100\rangle.
\]
The action of
\[
a^{\dagger}_{\alpha_4}a_{\alpha_0}\Phi_{0,3,6,10,13} = a^{\dagger}_{\alpha_4}|0001001000100100\rangle=a^{\dagger}_{\alpha_4}a_{\alpha_3}^{\dagger} a_{\alpha_6}^{\dagger} a_{\alpha_{10}}^{\dagger} a_{\alpha_{13}}^{\dagger} |0\rangle,
\]
which becomes
\[
-a_{\alpha_3}^{\dagger} a^{\dagger}_{\alpha_4} a_{\alpha_6}^{\dagger} a_{\alpha_{10}}^{\dagger} a_{\alpha_{13}}^{\dagger} |0\rangle=-|0001101000100100\rangle.
\]
% --- end paragraph admon ---



\subsection*{Operators in second quantization}

% --- begin paragraph admon ---
\paragraph{}
The action
\[
a_{\alpha_0}\Phi_{0,3,6,10,13} = |0001001000100100\rangle,
\]
can be obtained by subtracting the logical sum (AND operation) of $\Phi_{0,3,6,10,13}$ and 
a word which represents only $\alpha_0$, that is
\[
|1000000000000000\rangle,
\] 
from $\Phi_{0,3,6,10,13}= |1001001000100100\rangle$.

This operation gives $|0001001000100100\rangle$. 

Similarly, we can form $a^{\dagger}_{\alpha_4}a_{\alpha_0}\Phi_{0,3,6,10,13}$, say, by adding 
$|0000100000000000\rangle$ to $a_{\alpha_0}\Phi_{0,3,6,10,13}$, first checking that their logical sum
is zero in order to make sure that the state $\alpha_4$ is not already occupied.
% --- end paragraph admon ---



\subsection*{Operators in second quantization}

% --- begin paragraph admon ---
\paragraph{}
It is trickier however to get the phase $(-1)^l$. 
One possibility is as follows
\begin{itemize}
\item Let $S_1$ be a word that represents the 1-bit to be removed and all others set to zero.
\end{itemize}

\noindent
In the previous example $S_1=|1000000000000000\rangle$
\begin{itemize}
\item Define $S_2$ as the similar word that represents the bit to be added, that is in our case
\end{itemize}

\noindent
$S_2=|0000100000000000\rangle$.
\begin{itemize}
\item Compute then $S=S_1-S_2$, which here becomes
\end{itemize}

\noindent
\[
S=|0111000000000000\rangle
\]
\begin{itemize}
\item Perform then the logical AND operation of $S$ with the word containing 
\end{itemize}

\noindent
\[
\Phi_{0,3,6,10,13} = |1001001000100100\rangle,
\]
which results in $|0001000000000000\rangle$. Counting the number of 1-bits gives the phase.  Here you need however an algorithm for bitcounting.
% --- end paragraph admon ---



\subsection*{Bit counting}

% --- begin paragraph admon ---
\paragraph{}

We include here a python program which may aid in this direction. It uses bit manipulation functions from \href{{http://wiki.python.org/moin/BitManipulation}}{\nolinkurl{http://wiki.python.org/moin/BitManipulation}}.














































































































\begin{minted}[fontsize=\fontsize{9pt}{9pt},linenos=false,mathescape,baselinestretch=1.0,fontfamily=tt,xleftmargin=7mm]{python}
import math

"""
A simple Python class for Slater determinant manipulation
Bit-manipulation stolen from:

http://wiki.python.org/moin/BitManipulation
"""

# bitCount() counts the number of bits set (not an optimal function)

def bitCount(int_type):
    """ Count bits set in integer """
    count = 0
    while(int_type):
        int_type &= int_type - 1
        count += 1
    return(count)


# testBit() returns a nonzero result, 2**offset, if the bit at 'offset' is one.

def testBit(int_type, offset):
    mask = 1 << offset
    return(int_type & mask) >> offset

# setBit() returns an integer with the bit at 'offset' set to 1.

def setBit(int_type, offset):
    mask = 1 << offset
    return(int_type | mask)

# clearBit() returns an integer with the bit at 'offset' cleared.

def clearBit(int_type, offset):
    mask = ~(1 << offset)
    return(int_type & mask)

# toggleBit() returns an integer with the bit at 'offset' inverted, 0 -> 1 and 1 -> 0.

def toggleBit(int_type, offset):
    mask = 1 << offset
    return(int_type ^ mask)

# binary string made from number

def bin0(s):
    return str(s) if s<=1 else bin0(s>>1) + str(s&1)

def bin(s, L = 0):
    ss = bin0(s)
    if L > 0:
        return '0'*(L-len(ss)) + ss
    else:
        return ss
    
    

class Slater:
    """ Class for Slater determinants """
    def __init__(self):
        self.word = int(0)

    def create(self, j):
        print "c^+_" + str(j) + " |" + bin(self.word) + ">  = ",
        # Assume bit j is set, then we return zero.
        s = 0
        # Check if bit j is set.
        isset = testBit(self.word, j)
        if isset == 0:
            bits = bitCount(self.word & ((1<<j)-1))
            s = pow(-1, bits)
            self.word = setBit(self.word, j)

        print str(s) + " x |" + bin(self.word) + ">"
        return s
        
    def annihilate(self, j):
        print "c_" + str(j) + " |" + bin(self.word) + ">  = ",
        # Assume bit j is not set, then we return zero.
        s = 0
        # Check if bit j is set.
        isset = testBit(self.word, j)
        if isset == 1:
            bits = bitCount(self.word & ((1<<j)-1))
            s = pow(-1, bits)
            self.word = clearBit(self.word, j)

        print str(s) + " x |" + bin(self.word) + ">"
        return s



# Do some testing:

phi = Slater()
phi.create(0)
phi.create(1)
phi.create(2)
phi.create(3)

print


s = phi.annihilate(2)
s = phi.create(7)
s = phi.annihilate(0)
s = phi.create(200)


\end{minted}
% --- end paragraph admon ---

    

\subsection*{Eigenvalue problems, basic definitions}

% --- begin paragraph admon ---
\paragraph{}
Let us consider the matrix $\mathbf{A}$ of dimension $n$. The eigenvalues of
$\mathbf{A}$ are defined through the matrix equation 
\[
   \mathbf{A}\mathbf{x}^{(\nu)} = \lambda^{(\nu)}\mathbf{x}^{(\nu)},
\]
where $\lambda^{(\nu)}$ are the eigenvalues and $\mathbf{x}^{(\nu)}$ the
corresponding eigenvectors.
Unless otherwise stated, when we use the wording eigenvector we mean the
right eigenvector. The left eigenvalue problem is defined as 
\[
\mathbf{x}^{(\nu)}_L\mathbf{A} = \lambda^{(\nu)}\mathbf{x}^{(\nu)}_L
\]
The above right eigenvector problem is equivalent to a set of $n$ equations with $n$ unknowns
$x_i$.
% --- end paragraph admon ---



\subsection*{Eigenvalue problems, basic definitions}

% --- begin paragraph admon ---
\paragraph{}
The eigenvalue problem can be rewritten as 
\[
   \left( \mathbf{A}-\lambda^{(\nu)} \mathbf{I} \right) \mathbf{x}^{(\nu)} = 0,
\]
with $\mathbf{I}$ being the unity matrix. This equation provides
a solution to the problem if and only if the determinant
is zero, namely
\[
   \left| \mathbf{A}-\lambda^{(\nu)}\mathbf{I}\right| = 0,
\]
which in turn means that the determinant is a polynomial
of degree $n$ in $\lambda$ and in general we will have 
$n$ distinct zeros.
% --- end paragraph admon ---



\subsection*{Eigenvalue problems, basic definitions}

% --- begin paragraph admon ---
\paragraph{}
The eigenvalues of a matrix 
$\mathbf{A}\in {\mathbb{C}}^{n\times n}$
are thus the $n$ roots of its characteristic polynomial 
\[
P(\lambda) = det(\lambda\mathbf{I}-\mathbf{A}),
\]
or 
\[
  P(\lambda)= \prod_{i=1}^{n}\left(\lambda_i-\lambda\right).
\]
The set of these roots is called the spectrum and is denoted as
$\lambda(\mathbf{A})$.
If $\lambda(\mathbf{A})=\left\{\lambda_1,\lambda_2,\dots ,\lambda_n\right\}$ then we have
\[
   det(\mathbf{A})= \lambda_1\lambda_2\dots\lambda_n, 
\]
and if we define the trace of $\mathbf{A}$ as
\[
Tr(\mathbf{A})=\sum_{i=1}^n a_{ii}\]
then
\[
Tr(\mathbf{A})=\lambda_1+\lambda_2+\dots+\lambda_n.
\]
% --- end paragraph admon ---



\subsection*{Abel-Ruffini Impossibility Theorem}

% --- begin paragraph admon ---
\paragraph{}
The \emph{Abel-Ruffini} theorem (also known as Abel's impossibility theorem) 
states that there is no general solution in radicals to polynomial equations of degree five or higher.

The content of this theorem is frequently misunderstood. It does not assert that higher-degree polynomial equations are unsolvable. 
In fact, if the polynomial has real or complex coefficients, and we allow complex solutions, then every polynomial equation has solutions; this is the fundamental theorem of algebra. Although these solutions cannot always be computed exactly with radicals, they can be computed to any desired degree of accuracy using numerical methods such as the Newton-Raphson method or Laguerre method, and in this way they are no different from solutions to polynomial equations of the second, third, or fourth degrees.

The theorem only concerns the form that such a solution must take. The content of the theorem is 
that the solution of a higher-degree equation cannot in all cases be expressed in terms of the polynomial coefficients with a finite number of operations of addition, subtraction, multiplication, division and root extraction. Some polynomials of arbitrary degree, of which the simplest nontrivial example is the monomial equation $ax^n = b$, are always solvable with a radical.
% --- end paragraph admon ---



\subsection*{Abel-Ruffini Impossibility Theorem}

% --- begin paragraph admon ---
\paragraph{}

The \emph{Abel-Ruffini} theorem says that there are some fifth-degree equations whose solution cannot be so expressed. 
The equation $x^5 - x + 1 = 0$ is an example. Some other fifth degree equations can be solved by radicals, 
for example $x^5 - x^4 - x + 1 = 0$. The precise criterion that distinguishes between those equations that can be solved 
by radicals and those that cannot was given by Galois and is now part of Galois theory: 
a polynomial equation can be solved by radicals if and only if its Galois group is a solvable group.

Today, in the modern algebraic context, we say that second, third and fourth degree polynomial 
equations can always be solved by radicals because the symmetric groups $S_2, S_3$ and $S_4$ are solvable groups, 
whereas $S_n$ is not solvable for $n \ge 5$.
% --- end paragraph admon ---



\subsection*{Eigenvalue problems, basic definitions}

% --- begin paragraph admon ---
\paragraph{}
In the present discussion we assume that our matrix is real and symmetric, that is 
$\mathbf{A}\in {\mathbb{R}}^{n\times n}$.
The matrix $\mathbf{A}$ has $n$ eigenvalues
$\lambda_1\dots \lambda_n$ (distinct or not). Let $\mathbf{D}$ be the
diagonal matrix with the eigenvalues on the diagonal
\[
\mathbf{D}=    \left( \begin{array}{ccccccc} \lambda_1 & 0 & 0   & 0    & \dots  &0     & 0 \\
                                0 & \lambda_2 & 0 & 0    & \dots  &0     &0 \\
                                0   & 0 & \lambda_3 & 0  &0       &\dots & 0\\
                                \dots  & \dots & \dots & \dots  &\dots      &\dots & \dots\\
                                0   & \dots & \dots & \dots  &\dots       &\lambda_{n-1} & \\
                                0   & \dots & \dots & \dots  &\dots       &0 & \lambda_n
             \end{array} \right).
\]
If $\mathbf{A}$ is real and symmetric then there exists a real orthogonal matrix $\mathbf{S}$ such that
\[
     \mathbf{S}^T \mathbf{A}\mathbf{S}= \mathrm{diag}(\lambda_1,\lambda_2,\dots ,\lambda_n),
\]
and for $j=1:n$ we have $\mathbf{A}\mathbf{S}(:,j) = \lambda_j \mathbf{S}(:,j)$.
% --- end paragraph admon ---



\subsection*{Eigenvalue problems, basic definitions}

% --- begin paragraph admon ---
\paragraph{}
To obtain the eigenvalues of $\mathbf{A}\in {\mathbb{R}}^{n\times n}$,
the strategy is to
perform a series of similarity transformations on the original
matrix $\mathbf{A}$, in order to reduce it either into a  diagonal form as above
or into a  tridiagonal form. 

We say that a matrix $\mathbf{B}$ is a similarity
transform  of  $\mathbf{A}$ if 
\[
     \mathbf{B}= \mathbf{S}^T \mathbf{A}\mathbf{S}, \hspace{1cm} \mathrm{where} \hspace{1cm}  \mathbf{S}^T\mathbf{S}=\mathbf{S}^{-1}\mathbf{S} =\mathbf{I}.
\]
The importance of a similarity transformation lies in the fact that
the resulting matrix has the same
eigenvalues, but the eigenvectors are in general different.
% --- end paragraph admon ---



\subsection*{Eigenvalue problems, basic definitions}

% --- begin paragraph admon ---
\paragraph{}
To prove this we
start with  the eigenvalue problem and a similarity transformed matrix $\mathbf{B}$.
\[
   \mathbf{A}\mathbf{x}=\lambda\mathbf{x} \hspace{1cm} \mathrm{and}\hspace{1cm} 
    \mathbf{B}= \mathbf{S}^T \mathbf{A}\mathbf{S}.
\]
We multiply the first equation on the left by $\mathbf{S}^T$ and insert
$\mathbf{S}^{T}\mathbf{S} = \mathbf{I}$ between $\mathbf{A}$ and $\mathbf{x}$. Then we get
\begin{equation}
   (\mathbf{S}^T\mathbf{A}\mathbf{S})(\mathbf{S}^T\mathbf{x})=\lambda\mathbf{S}^T\mathbf{x} ,
\end{equation}  
which is the same as 
\[
   \mathbf{B} \left ( \mathbf{S}^T\mathbf{x} \right ) = \lambda \left (\mathbf{S}^T\mathbf{x}\right ).
\]
The variable  $\lambda$ is an eigenvalue of $\mathbf{B}$ as well, but with
eigenvector $\mathbf{S}^T\mathbf{x}$.
% --- end paragraph admon ---



\subsection*{Eigenvalue problems, basic definitions}

% --- begin paragraph admon ---
\paragraph{}
The basic philosophy is to
\begin{itemize}
 \item Either apply subsequent similarity transformations (direct method) so that 
\end{itemize}

\noindent
\begin{equation}
   \mathbf{S}_N^T\dots \mathbf{S}_1^T\mathbf{A}\mathbf{S}_1\dots \mathbf{S}_N=\mathbf{D} ,
\end{equation}
\begin{itemize}
 \item Or apply subsequent similarity transformations so that $\mathbf{A}$ becomes tridiagonal (Householder) or upper/lower triangular (the \emph{QR} method to be discussed later). 

 \item Thereafter, techniques for obtaining eigenvalues from tridiagonal matrices can be used.

 \item Or use so-called power methods

 \item Or use iterative methods (Krylov, Lanczos, Arnoldi). These methods are popular for huge matrix problems.
\end{itemize}

\noindent
% --- end paragraph admon ---



\subsection*{Discussion of  methods for eigenvalues}

% --- begin paragraph admon ---
\paragraph{The general overview.}

One speaks normally of two main approaches to solving the eigenvalue problem.
\begin{itemize}
 \item The first is the formal method, involving determinants and the  characteristic polynomial. This proves how many eigenvalues  there are, and is the way most of you learned about how to solve the eigenvalue problem, but for matrices of dimensions greater than 2 or 3, it is rather impractical.

 \item The other general approach is to use similarity or unitary tranformations  to reduce a matrix to diagonal form. This is normally done in two steps: first reduce to for example a \emph{tridiagonal} form, and then to diagonal form. The main algorithms we will discuss in detail, Jacobi's and  Householder's  (so-called direct method) and Lanczos algorithms (an iterative method), follow this methodology. 
\end{itemize}

\noindent
% --- end paragraph admon ---



\subsection*{Eigenvalues methods}

% --- begin paragraph admon ---
\paragraph{}
Direct or non-iterative methods  require for matrices of dimensionality $n\times n$ typically $O(n^3)$ operations. These methods are normally called standard methods and are used for dimensionalities
$n \sim 10^5$ or smaller. A brief historical overview  


\begin{quote}
\begin{tabular}{ccc}
\hline
\multicolumn{1}{c}{ Year } & \multicolumn{1}{c}{ $n$ } & \multicolumn{1}{c}{  } \\
\hline
1950        & $n=20$       & (Wilkinson)       \\
1965        & $n=200$      & (Forsythe et al.) \\
1980        & $n=2000$     & Linpack           \\
1995        & $n=20000$    & Lapack            \\
This decade & $n\sim 10^5$ & Lapack            \\
\hline
\end{tabular}
\end{quote}

\noindent
shows that in the course of 60 years the dimension that  direct diagonalization methods can handle  has increased by almost a factor of
$10^4$ (note this is for serial versions). However, it pales beside the progress achieved by computer hardware, from flops to petaflops, a factor of almost $10^{15}$. We see clearly played out in history the $O(n^3)$ bottleneck  of direct matrix algorithms.

Sloppily speaking, when  $n\sim 10^4$ is cubed we have $O(10^{12})$ operations, which is smaller than the $10^{15}$ increase in flops.
% --- end paragraph admon ---



\subsection*{Discussion of methods for eigenvalues}

% --- begin paragraph admon ---
\paragraph{}
If the matrix to diagonalize is large and sparse, direct methods simply become impractical, 
also because
many of the direct methods tend to destroy sparsity. As a result large dense matrices may arise during the diagonalization procedure.  The idea behind iterative methods is to project the 
$n-$dimensional problem in smaller spaces, so-called Krylov subspaces. 
Given a matrix $\mathbf{A}$ and a vector $\mathbf{v}$, the associated Krylov sequences of vectors
(and thereby subspaces) 
$\mathbf{v}$, $\mathbf{A}\mathbf{v}$, $\mathbf{A}^2\mathbf{v}$, $\mathbf{A}^3\mathbf{v},\dots$, represent
successively larger Krylov subspaces. 


\begin{quote}
\begin{tabular}{lll}
\hline
\multicolumn{1}{c}{ Matrix } & \multicolumn{1}{c}{ $\mathbf{A}\mathbf{x}=\mathbf{b}$ } & \multicolumn{1}{c}{ $\mathbf{A}\mathbf{x}=\lambda\mathbf{x}$ } \\
\hline
$\mathbf{A}=\mathbf{A}^*$    & Conjugate gradient                & Lanczos                                  \\
$\mathbf{A}\ne \mathbf{A}^*$ & GMRES etc                         & Arnoldi                                  \\
\hline
\end{tabular}
\end{quote}

\noindent
% --- end paragraph admon ---



\subsection*{Eigenvalues and Lanczos' method}

% --- begin paragraph admon ---
\paragraph{}
Basic features with a real symmetric matrix (and normally huge $n> 10^6$ and sparse) 
$\hat{A}$ of dimension $n\times n$:

\begin{itemize}
\item Lanczos' algorithm generates a sequence of real tridiagonal matrices $T_k$ of dimension $k\times k$ with $k\le n$, with the property that the extremal eigenvalues of $T_k$ are progressively better estimates of $\hat{A}$' extremal eigenvalues.* The method converges to the extremal eigenvalues.

\item The similarity transformation is 
\end{itemize}

\noindent
\[
\hat{T}= \hat{Q}^{T}\hat{A}\hat{Q},
\]
with the first vector $\hat{Q}\hat{e}_1=\hat{q}_1$.

We are going to solve iteratively
\[
\hat{T}= \hat{Q}^{T}\hat{A}\hat{Q},
\]
with the first vector $\hat{Q}\hat{e}_1=\hat{q}_1$.
We can write out the matrix $\hat{Q}$ in terms of its column vectors 
\[
\hat{Q}=\left[\hat{q}_1\hat{q}_2\dots\hat{q}_n\right].
\]
% --- end paragraph admon ---



\subsection*{Eigenvalues and Lanczos' method, tridiagonal matrix}

% --- begin paragraph admon ---
\paragraph{}
The matrix
\[
\hat{T}= \hat{Q}^{T}\hat{A}\hat{Q},
\]
can be written as 
\[
    \hat{T} = \left(\begin{array}{cccccc}
                           \alpha_1& \beta_1 & 0 &\dots   & \dots &0 \\
                           \beta_1 & \alpha_2 & \beta_2 &0 &\dots &0 \\
                           0& \beta_2 & \alpha_3 & \beta_3 & \dots &0 \\
                           \dots& \dots   & \dots &\dots   &\dots & 0 \\
                           \dots&   &  &\beta_{n-2}  &\alpha_{n-1}& \beta_{n-1} \\
                           0&  \dots  &\dots  &0   &\beta_{n-1} & \alpha_{n} \\
                      \end{array} \right)
\]
% --- end paragraph admon ---



\subsection*{Eigenvalues and Lanczos' method, tridiagonal and orthogonal matrices}

% --- begin paragraph admon ---
\paragraph{}
Using the fact that 
\[
\hat{Q}\hat{Q}^T=\hat{I}, 
\]
we can rewrite 
\[
\hat{T}= \hat{Q}^{T}\hat{A}\hat{Q},
\]
as 
\[
\hat{Q}\hat{T}= \hat{A}\hat{Q}.
\]
% --- end paragraph admon ---



\subsection*{Eigenvalues and Lanczos' method}

% --- begin paragraph admon ---
\paragraph{}
If we equate columns 
\[
\hat{T} = \left(\begin{array}{cccccc}
        \alpha_1& \beta_1 & 0 &\dots   & \dots &0 \\
        \beta_1 & \alpha_2 & \beta_2 &0 &\dots &0 \\
        0& \beta_2 & \alpha_3 & \beta_3 & \dots &0 \\
        \dots& \dots   & \dots &\dots   &\dots & 0 \\
        \dots&   &  &\beta_{n-2}  &\alpha_{n-1}& \beta_{n-1} \\
        0&  \dots  &\dots  &0   &\beta_{n-1} & \alpha_{n} \\
        \end{array} \right)
\]
we obtain
\[
\hat{A}\hat{q}_k=\beta_{k-1}\hat{q}_{k-1}+\alpha_k\hat{q}_k+\beta_k\hat{q}_{k+1}.
\]
% --- end paragraph admon ---



\subsection*{Eigenvalues and Lanczos' method, defining the Lanczos' vectors}

% --- begin paragraph admon ---
\paragraph{}
We have thus
\[
\hat{A}\hat{q}_k=\beta_{k-1}\hat{q}_{k-1}+\alpha_k\hat{q}_k+\beta_k\hat{q}_{k+1},
\]
with $\beta_0\hat{q}_0=0$ for $k=1:n-1$. Remember that the vectors $\hat{q}_k$  are orthornormal and this implies
\[
\alpha_k=\hat{q}_k^T\hat{A}\hat{q}_k,
\]
and these vectors are called Lanczos vectors.
% --- end paragraph admon ---



\subsection*{Eigenvalues and Lanczos' method, basic steps}

% --- begin paragraph admon ---
\paragraph{}
We have thus
\[
\hat{A}\hat{q}_k=\beta_{k-1}\hat{q}_{k-1}+\alpha_k\hat{q}_k+\beta_k\hat{q}_{k+1},
\]
with $\beta_0\hat{q}_0=0$ for $k=1:n-1$ and 
\[
\alpha_k=\hat{q}_k^T\hat{A}\hat{q}_k.
\]
If 
\[
\hat{r}_k=(\hat{A}-\alpha_k\hat{I})\hat{q}_k-\beta_{k-1}\hat{q}_{k-1},
\]
is non-zero, then 
\[
\hat{q}_{k+1}=\hat{r}_{k}/\beta_k,
\]
with $\beta_k=\pm ||\hat{r}_{k}||_2$.
% --- end paragraph admon ---




% ------------------- end of main content ---------------

\end{document}


 \part{Mean-field theories and post Hartree-Fock methods}
 \clearemptydoublepage
      
\chapter{Hartree-Fock methods}

\subsection*{Why Hartree-Fock? Derivation of Hartree-Fock equations in coordinate space}

Hartree-Fock (HF) theory is an algorithm for finding an approximative expression for the ground state of a given Hamiltonian. The basic ingredients are
\begin{itemize}
  \item Define a single-particle basis $\{\psi_{\alpha}\}$ so that
\end{itemize}

\noindent
\[ 
\hat{h}^{\mathrm{HF}}\psi_{\alpha} = \varepsilon_{\alpha}\psi_{\alpha}
\]
with the Hartree-Fock Hamiltonian defined as
\[
\hat{h}^{\mathrm{HF}}=\hat{t}+\hat{u}_{\mathrm{ext}}+\hat{u}^{\mathrm{HF}}
\]
\begin{itemize}
  \item The term  $\hat{u}^{\mathrm{HF}}$ is a single-particle potential to be determined by the HF algorithm.

  \item The HF algorithm means to choose $\hat{u}^{\mathrm{HF}}$ in order to have 
\end{itemize}

\noindent
\[ \langle \hat{H} \rangle = E^{\mathrm{HF}}= \langle \Phi_0 | \hat{H}|\Phi_0 \rangle
\]
that is to find a local minimum with a Slater determinant $\Phi_0$ being the ansatz for the ground state. 
\begin{itemize}
  \item The variational principle ensures that $E^{\mathrm{HF}} \ge E_0$, with $E_0$ the exact ground state energy.
\end{itemize}

\noindent
We will show that the Hartree-Fock Hamiltonian $\hat{h}^{\mathrm{HF}}$ equals our definition of the operator $\hat{f}$ discussed in connection with the new definition of the normal-ordered Hamiltonian (see later lectures), that is we have, for a specific matrix element
\[
\langle p |\hat{h}^{\mathrm{HF}}| q \rangle =\langle p |\hat{f}| q \rangle=\langle p|\hat{t}+\hat{u}_{\mathrm{ext}}|q \rangle +\sum_{i\le F} \langle pi | \hat{V} | qi\rangle_{AS},
\]
meaning that
\[
\langle p|\hat{u}^{\mathrm{HF}}|q\rangle = \sum_{i\le F} \langle pi | \hat{V} | qi\rangle_{AS}.
\]
The so-called Hartree-Fock potential $\hat{u}^{\mathrm{HF}}$ brings an explicit medium dependence due to the summation over all single-particle states below the Fermi level $F$. It brings also in an explicit dependence on the two-body interaction (in nuclear physics we can also have complicated three- or higher-body forces). The two-body interaction, with its contribution from the other bystanding fermions, creates an effective mean field in which a given fermion moves, in addition to the external potential $\hat{u}_{\mathrm{ext}}$ which confines the motion of the fermion. For systems like nuclei, there is no external confining potential. Nuclei are examples of self-bound systems, where the binding arises due to the intrinsic nature of the strong force. For nuclear systems thus, there would be no external one-body potential in the Hartree-Fock Hamiltonian. 

\subsection*{Variational Calculus and Lagrangian Multipliers}

The calculus of variations involves 
problems where the quantity to be minimized or maximized is an integral. 

In the general case we have an integral of the type
\[ 
E[\Phi]= \int_a^b f(\Phi(x),\frac{\partial \Phi}{\partial x},x)dx,
\]
where $E$ is the quantity which is sought minimized or maximized.
The problem is that although $f$ is a function of the variables $\Phi$, $\partial \Phi/\partial x$ and $x$, the exact dependence of
$\Phi$ on $x$ is not known.  This means again that even though the integral has fixed limits $a$ and $b$, the path of integration is
not known. In our case the unknown quantities are the single-particle wave functions and we wish to choose an integration path which makes
the functional $E[\Phi]$ stationary. This means that we want to find minima, or maxima or saddle points. In physics we search normally for minima.
Our task is therefore to find the minimum of $E[\Phi]$ so that its variation $\delta E$ is zero  subject to specific
constraints. In our case the constraints appear as the integral which expresses the orthogonality of the  single-particle wave functions.
The constraints can be treated via the technique of Lagrangian multipliers

Let us specialize to the expectation value of the energy for one particle in three-dimensions.
This expectation value reads
\[
  E=\int dxdydz \psi^*(x,y,z) \hat{H} \psi(x,y,z),
\]
with the constraint
\[
 \int dxdydz \psi^*(x,y,z) \psi(x,y,z)=1,
\]
and a Hamiltonian
\[
\hat{H}=-\frac{1}{2}\nabla^2+V(x,y,z).
\]
We will, for the sake of notational convenience,  skip the variables $x,y,z$ below, and write for example $V(x,y,z)=V$.

The integral involving the kinetic energy can be written as, with the function $\psi$ vanishing
strongly for large values of $x,y,z$ (given here by the limits $a$ and $b$), 
 \[
  \int_a^b dxdydz \psi^* \left(-\frac{1}{2}\nabla^2\right) \psi dxdydz = \psi^*\nabla\psi|_a^b+\int_a^b dxdydz\frac{1}{2}\nabla\psi^*\nabla\psi.
\]
We will drop the limits $a$ and $b$ in the remaining discussion. 
Inserting this expression into the expectation value for the energy and taking the variational minimum  we obtain
\[
\delta E = \delta \left\{\int dxdydz\left( \frac{1}{2}\nabla\psi^*\nabla\psi+V\psi^*\psi\right)\right\} = 0.
\]

The constraint appears in integral form as 
\[
 \int dxdydz \psi^* \psi=\mathrm{constant},
\]
and multiplying with a Lagrangian multiplier $\lambda$ and taking the variational minimum we obtain the final variational equation
\[
\delta \left\{\int dxdydz\left( \frac{1}{2}\nabla\psi^*\nabla\psi+V\psi^*\psi-\lambda\psi^*\psi\right)\right\} = 0.
\]
We introduce the function  $f$
\[
  f =  \frac{1}{2}\nabla\psi^*\nabla\psi+V\psi^*\psi-\lambda\psi^*\psi=
\frac{1}{2}(\psi^*_x\psi_x+\psi^*_y\psi_y+\psi^*_z\psi_z)+V\psi^*\psi-\lambda\psi^*\psi,
\]
where we have skipped the dependence on $x,y,z$ and introduced the shorthand $\psi_x$, $\psi_y$ and $\psi_z$  for the various derivatives.

For $\psi^*$ the Euler-Lagrange  equations yield
\[
\frac{\partial f}{\partial \psi^*}- \frac{\partial }{\partial x}\frac{\partial f}{\partial \psi^*_x}-\frac{\partial }{\partial y}\frac{\partial f}{\partial \psi^*_y}-\frac{\partial }{\partial z}\frac{\partial f}{\partial \psi^*_z}=0,
\] 
which results in 
\[
    -\frac{1}{2}(\psi_{xx}+\psi_{yy}+\psi_{zz})+V\psi=\lambda \psi.
\]
We can then identify the  Lagrangian multiplier as the energy of the system. The last equation is 
nothing but the standard 
Schroedinger equation and the variational  approach discussed here provides 
a powerful method for obtaining approximate solutions of the wave function.

\subsection*{Derivation of Hartree-Fock equations in coordinate space}

Let us denote the ground state energy by $E_0$. According to the
variational principle we have
\[
  E_0 \le E[\Phi] = \int \Phi^*\hat{H}\Phi d\mathbf{\tau}
\]
where $\Phi$ is a trial function which we assume to be normalized
\[
  \int \Phi^*\Phi d\mathbf{\tau} = 1,
\]
where we have used the shorthand $d\mathbf{\tau}=dx_1dx_2\dots dx_A$.

In the Hartree-Fock method the trial function is a Slater
determinant which can be rewritten as 
\[
  \Psi(x_1,x_2,\dots,x_A,\alpha,\beta,\dots,\nu) = \frac{1}{\sqrt{A!}}\sum_{P} (-)^PP\psi_{\alpha}(x_1)
    \psi_{\beta}(x_2)\dots\psi_{\nu}(x_A)=\sqrt{A!}\hat{A}\Phi_H,
\]
where we have introduced the anti-symmetrization operator $\hat{A}$ defined by the 
summation over all possible permutations \emph{p} of two fermions.
It is defined as
\[
  \hat{A} = \frac{1}{A!}\sum_{p} (-)^p\hat{P},
\]
with the the Hartree-function given by the simple product of all possible single-particle function
\[
  \Phi_H(x_1,x_2,\dots,x_A,\alpha,\beta,\dots,\nu) =
  \psi_{\alpha}(x_1)
    \psi_{\beta}(x_2)\dots\psi_{\nu}(x_A).
\]

Our functional is written as
\[
  E[\Phi] = \sum_{\mu=1}^A \int \psi_{\mu}^*(x_i)\hat{h}_0(x_i)\psi_{\mu}(x_i) dx_i 
  + \frac{1}{2}\sum_{\mu=1}^A\sum_{\nu=1}^A
   \left[ \int \psi_{\mu}^*(x_i)\psi_{\nu}^*(x_j)\hat{v}(r_{ij})\psi_{\mu}(x_i)\psi_{\nu}(x_j)dx_idx_j- \int \psi_{\mu}^*(x_i)\psi_{\nu}^*(x_j)
 \hat{v}(r_{ij})\psi_{\nu}(x_i)\psi_{\mu}(x_j)dx_idx_j\right]
\]
The more compact version reads
\[
  E[\Phi] 
  = \sum_{\mu}^A \langle \mu | \hat{h}_0 | \mu\rangle+ \frac{1}{2}\sum_{\mu\nu}^A\left[\langle \mu\nu |\hat{v}|\mu\nu\rangle-\langle \nu\mu |\hat{v}|\mu\nu\rangle\right].
\]

Since the interaction is invariant under the interchange of two particles it means for example that we have
\[
\langle \mu\nu|\hat{v}|\mu\nu\rangle =  \langle \nu\mu|\hat{v}|\nu\mu\rangle,  
\]
or in the more general case
\[
\langle \mu\nu|\hat{v}|\sigma\tau\rangle =  \langle \nu\mu|\hat{v}|\tau\sigma\rangle.  
\]

The direct and exchange matrix elements can be  brought together if we define the antisymmetrized matrix element
\[
\langle \mu\nu|\hat{v}|\mu\nu\rangle_{AS}= \langle \mu\nu|\hat{v}|\mu\nu\rangle-\langle \mu\nu|\hat{v}|\nu\mu\rangle,
\]
or for a general matrix element  
\[
\langle \mu\nu|\hat{v}|\sigma\tau\rangle_{AS}= \langle \mu\nu|\hat{v}|\sigma\tau\rangle-\langle \mu\nu|\hat{v}|\tau\sigma\rangle.
\]
It has the symmetry property
\[
\langle \mu\nu|\hat{v}|\sigma\tau\rangle_{AS}= -\langle \mu\nu|\hat{v}|\tau\sigma\rangle_{AS}=-\langle \nu\mu|\hat{v}|\sigma\tau\rangle_{AS}.
\]
The antisymmetric matrix element is also hermitian, implying 
\[
\langle \mu\nu|\hat{v}|\sigma\tau\rangle_{AS}= \langle \sigma\tau|\hat{v}|\mu\nu\rangle_{AS}.
\]

With these notations we rewrite the Hartree-Fock functional as
\begin{equation}
  \int \Phi^*\hat{H_I}\Phi d\mathbf{\tau} 
  = \frac{1}{2}\sum_{\mu=1}^A\sum_{\nu=1}^A \langle \mu\nu|\hat{v}|\mu\nu\rangle_{AS}. \label{H2Expectation2}
\end{equation}

Adding the contribution from the one-body operator $\hat{H}_0$ to
(\ref{H2Expectation2}) we obtain the energy functional 
\begin{equation}
  E[\Phi] 
  = \sum_{\mu=1}^A \langle \mu | h | \mu \rangle +
  \frac{1}{2}\sum_{{\mu}=1}^A\sum_{{\nu}=1}^A \langle \mu\nu|\hat{v}|\mu\nu\rangle_{AS}. \label{FunctionalEPhi}
\end{equation}
In our coordinate space derivations below we will spell out the Hartree-Fock equations in terms of their integrals.

If we generalize the Euler-Lagrange equations to more variables 
and introduce $N^2$ Lagrange multipliers which we denote by 
$\epsilon_{\mu\nu}$, we can write the variational equation for the functional of $E$
\[
  \delta E - \sum_{\mu\nu}^A \epsilon_{\mu\nu} \delta
  \int \psi_{\mu}^* \psi_{\nu} = 0.
\]
For the orthogonal wave functions $\psi_{i}$ this reduces to
\[
  \delta E - \sum_{\mu=1}^A \epsilon_{\mu} \delta
  \int \psi_{\mu}^* \psi_{\mu} = 0.
\]

Variation with respect to the single-particle wave functions $\psi_{\mu}$ yields then
\[
  \sum_{\mu=1}^A \int \delta\psi_{\mu}^*\hat{h_0}(x_i)\psi_{\mu}
  dx_i  
  + \frac{1}{2}\sum_{{\mu}=1}^A\sum_{{\nu}=1}^A \left[ \int
  \delta\psi_{\mu}^*\psi_{\nu}^*\hat{v}(r_{ij})\psi_{\mu}\psi_{\nu} dx_idx_j- \int
  \delta\psi_{\mu}^*\psi_{\nu}^*\hat{v}(r_{ij})\psi_{\nu}\psi_{\mu}
  dx_idx_j \right]+ 
\]
\[
\sum_{\mu=1}^A \int \psi_{\mu}^*\hat{h_0}(x_i)\delta\psi_{\mu}
  dx_i 
  + \frac{1}{2}\sum_{{\mu}=1}^A\sum_{{\nu}=1}^A \left[ \int
  \psi_{\mu}^*\psi_{\nu}^*\hat{v}(r_{ij})\delta\psi_{\mu}\psi_{\nu} dx_idx_j- \int
  \psi_{\mu}^*\psi_{\nu}^*\hat{v}(r_{ij})\psi_{\nu}\delta\psi_{\mu}
  dx_idx_j \right]-  \sum_{{\mu}=1}^A E_{\mu} \int \delta\psi_{\mu}^*
  \psi_{\mu}dx_i
  -  \sum_{{\mu}=1}^A E_{\mu} \int \psi_{\mu}^*
  \delta\psi_{\mu}dx_i = 0.
\]

Although the variations $\delta\psi$ and $\delta\psi^*$ are not
independent, they may in fact be treated as such, so that the 
terms dependent on either $\delta\psi$ and $\delta\psi^*$ individually 
may be set equal to zero. To see this, simply 
replace the arbitrary variation $\delta\psi$ by $i\delta\psi$, so that
$\delta\psi^*$ is replaced by $-i\delta\psi^*$, and combine the two
equations. We thus arrive at the Hartree-Fock equations
\begin{equation}
\left[ -\frac{1}{2}\nabla_i^2+ \sum_{\nu=1}^A\int \psi_{\nu}^*(x_j)\hat{v}(r_{ij})\psi_{\nu}(x_j)dx_j \right]\psi_{\mu}(x_i) - \left[ \sum_{{\nu}=1}^A \int\psi_{\nu}^*(x_j)\hat{v}(r_{ij})\psi_{\mu}(x_j) dx_j\right] \psi_{\nu}(x_i) = \epsilon_{\mu} \psi_{\mu}(x_i).  \label{eq:hartreefockcoordinatespace}
\end{equation}
Notice that the integration $\int dx_j$ implies an
integration over the spatial coordinates $\mathbf{r_j}$ and a summation
over the spin-coordinate of fermion $j$. We note that the factor of $1/2$ in front of the sum involving the two-body interaction, has been removed. This is due to the fact that we need to vary both $\delta\psi_{\mu}^*$ and
$\delta\psi_{\nu}^*$. Using the symmetry properties of the two-body interaction and interchanging $\mu$ and $\nu$
as summation indices, we obtain two identical terms. 

The two first terms in the last equation are the one-body kinetic energy and the
electron-nucleus potential. The third or \emph{direct} term is the averaged electronic repulsion of the other
electrons. As written, the
term includes the \emph{self-interaction} of 
electrons when $\mu=\nu$. The self-interaction is cancelled in the fourth
term, or the \emph{exchange} term. The exchange term results from our
inclusion of the Pauli principle and the assumed determinantal form of
the wave-function. Equation (\ref{eq:hartreefockcoordinatespace}), in addition to the kinetic energy and the attraction from the atomic nucleus that confines the motion of a single electron,   represents now the motion of a single-particle modified by the two-body interaction. The additional contribution to the Schroedinger equation due to the two-body interaction, represents a mean field set up by all the other bystanding electrons, the latter given by the sum over all single-particle states occupied by $N$ electrons. 

The Hartree-Fock equation is an example of an integro-differential equation. These equations involve repeated calculations of integrals, in addition to the solution of a set of coupled differential equations. 
The Hartree-Fock equations can also be rewritten in terms of an eigenvalue problem. The solution of an eigenvalue problem represents often a more practical algorithm and the  solution of  coupled  integro-differential equations.
This alternative derivation of the Hartree-Fock equations is given below.

\subsection*{Analysis of Hartree-Fock equations in coordinate space}

  A theoretically convenient form of the
Hartree-Fock equation is to regard the direct and exchange operator
defined through 
\begin{equation*}
  V_{\mu}^{d}(x_i) = \int \psi_{\mu}^*(x_j) 
 \hat{v}(r_{ij})\psi_{\mu}(x_j) dx_j
\end{equation*}
and
\begin{equation*}
  V_{\mu}^{ex}(x_i) g(x_i) 
  = \left(\int \psi_{\mu}^*(x_j) 
 \hat{v}(r_{ij})g(x_j) dx_j
  \right)\psi_{\mu}(x_i),
\end{equation*}
respectively. 

The function $g(x_i)$ is an arbitrary function,
and by the substitution $g(x_i) = \psi_{\nu}(x_i)$
we get
\begin{equation*}
  V_{\mu}^{ex}(x_i) \psi_{\nu}(x_i) 
  = \left(\int \psi_{\mu}^*(x_j) 
 \hat{v}(r_{ij})\psi_{\nu}(x_j)
  dx_j\right)\psi_{\mu}(x_i).
\end{equation*}
We may then rewrite the Hartree-Fock equations as
\[
  \hat{h}^{HF}(x_i) \psi_{\nu}(x_i) = \epsilon_{\nu}\psi_{\nu}(x_i),
\]
with
\[
  \hat{h}^{HF}(x_i)= \hat{h}_0(x_i) + \sum_{\mu=1}^AV_{\mu}^{d}(x_i) -
  \sum_{\mu=1}^AV_{\mu}^{ex}(x_i),
\]
and where $\hat{h}_0(i)$ is the one-body part. The latter is normally chosen as a part which yields solutions in closed form. The harmonic oscilltor is a classical problem thereof.
We normally rewrite the last equation as
\[
  \hat{h}^{HF}(x_i)= \hat{h}_0(x_i) + \hat{u}^{HF}(x_i). 
\]

\subsection*{Hartree-Fock by varying the coefficients of a wave function expansion}

Another possibility is to expand the single-particle functions in a known basis  and vary the coefficients, 
that is, the new single-particle wave function is written as a linear expansion
in terms of a fixed chosen orthogonal basis (for example the well-known harmonic oscillator functions or the hydrogen-like functions etc).
We define our new Hartree-Fock single-particle basis by performing a unitary transformation 
on our previous basis (labelled with greek indices) as
\begin{equation}
\psi_p^{HF}  = \sum_{\lambda} C_{p\lambda}\phi_{\lambda}. \label{eq:newbasis}
\end{equation}
In this case we vary the coefficients $C_{p\lambda}$. If the basis has infinitely many solutions, we need
to truncate the above sum.  We assume that the basis $\phi_{\lambda}$ is orthogonal.

It is normal to choose a single-particle basis defined as the eigenfunctions
of parts of the full Hamiltonian. The typical situation consists of the solutions of the one-body part of the Hamiltonian, that is we have
\[
\hat{h}_0\phi_{\lambda}=\epsilon_{\lambda}\phi_{\lambda}.
\]
The single-particle wave functions $\phi_{\lambda}(\mathbf{r})$, defined by the quantum numbers $\lambda$ and $\mathbf{r}$
are defined as the overlap 
\[
   \phi_{\lambda}(\mathbf{r})  = \langle \mathbf{r} | \lambda \rangle .
\]

In deriving the Hartree-Fock equations, we  will expand the single-particle functions in a known basis  and vary the coefficients, 
that is, the new single-particle wave function is written as a linear expansion
in terms of a fixed chosen orthogonal basis (for example the well-known harmonic oscillator functions or the hydrogen-like functions etc).

We stated that a unitary transformation keeps the orthogonality. To see this consider first a basis of vectors $\mathbf{v}_i$,
\[
\mathbf{v}_i = \begin{bmatrix} v_{i1} \\ \dots \\ \dots \\v_{in} \end{bmatrix}
\]
We assume that the basis is orthogonal, that is 
\[
\mathbf{v}_j^T\mathbf{v}_i = \delta_{ij}.
\]
An orthogonal or unitary transformation
\[
\mathbf{w}_i=\mathbf{U}\mathbf{v}_i,
\]
preserves the dot product and orthogonality since
\[
\mathbf{w}_j^T\mathbf{w}_i=(\mathbf{U}\mathbf{v}_j)^T\mathbf{U}\mathbf{v}_i=\mathbf{v}_j^T\mathbf{U}^T\mathbf{U}\mathbf{v}_i= \mathbf{v}_j^T\mathbf{v}_i = \delta_{ij}.
\]

This means that if the coefficients $C_{p\lambda}$ belong to a unitary or orthogonal trasformation (using the Dirac bra-ket notation)
\[
\vert p\rangle  = \sum_{\lambda} C_{p\lambda}\vert\lambda\rangle,
\]
orthogonality is preserved, that is $\langle \alpha \vert \beta\rangle = \delta_{\alpha\beta}$
and $\langle p \vert q\rangle = \delta_{pq}$. 

This propertry is extremely useful when we build up a basis of many-body Stater determinant based states. 

\textbf{Note also that although a basis $\vert \alpha\rangle$ contains an infinity of states, for practical calculations we have always to make some truncations.} 

Before we develop the Hartree-Fock equations, there is another very useful property of determinants that we will use both in connection with Hartree-Fock calculations and later shell-model calculations.  

Consider the following determinant
\[
\left| \begin{array}{cc} \alpha_1b_{11}+\alpha_2sb_{12}& a_{12}\\
                         \alpha_1b_{21}+\alpha_2b_{22}&a_{22}\end{array} \right|=\alpha_1\left|\begin{array}{cc} b_{11}& a_{12}\\
                         b_{21}&a_{22}\end{array} \right|+\alpha_2\left| \begin{array}{cc} b_{12}& a_{12}\\b_{22}&a_{22}\end{array} \right|
\]

We can generalize this to  an $n\times n$ matrix and have 
\[
\left| \begin{array}{cccccc} a_{11}& a_{12} & \dots & \sum_{k=1}^n c_k b_{1k} &\dots & a_{1n}\\
a_{21}& a_{22} & \dots & \sum_{k=1}^n c_k b_{2k} &\dots & a_{2n}\\
\dots & \dots & \dots & \dots & \dots & \dots \\
\dots & \dots & \dots & \dots & \dots & \dots \\
a_{n1}& a_{n2} & \dots & \sum_{k=1}^n c_k b_{nk} &\dots & a_{nn}\end{array} \right|=
\sum_{k=1}^n c_k\left| \begin{array}{cccccc} a_{11}& a_{12} & \dots &  b_{1k} &\dots & a_{1n}\\
a_{21}& a_{22} & \dots &  b_{2k} &\dots & a_{2n}\\
\dots & \dots & \dots & \dots & \dots & \dots\\
\dots & \dots & \dots & \dots & \dots & \dots\\
a_{n1}& a_{n2} & \dots &  b_{nk} &\dots & a_{nn}\end{array} \right| .
\]
This is a property we will use in our Hartree-Fock discussions. 

We can generalize the previous results, now 
with all elements $a_{ij}$  being given as functions of 
linear combinations  of various coefficients $c$ and elements $b_{ij}$,
\[
\left| \begin{array}{cccccc} \sum_{k=1}^n b_{1k}c_{k1}& \sum_{k=1}^n b_{1k}c_{k2} & \dots & \sum_{k=1}^n b_{1k}c_{kj}  &\dots & \sum_{k=1}^n b_{1k}c_{kn}\\
\sum_{k=1}^n b_{2k}c_{k1}& \sum_{k=1}^n b_{2k}c_{k2} & \dots & \sum_{k=1}^n b_{2k}c_{kj} &\dots & \sum_{k=1}^n b_{2k}c_{kn}\\
\dots & \dots & \dots & \dots & \dots & \dots \\
\dots & \dots & \dots & \dots & \dots &\dots \\
\sum_{k=1}^n b_{nk}c_{k1}& \sum_{k=1}^n b_{nk}c_{k2} & \dots & \sum_{k=1}^n b_{nk}c_{kj} &\dots & \sum_{k=1}^n b_{nk}c_{kn}\end{array} \right|=det(\mathbf{C})det(\mathbf{B}),
\]
where $det(\mathbf{C})$ and $det(\mathbf{B})$ are the determinants of $n\times n$ matrices
with elements $c_{ij}$ and $b_{ij}$ respectively.  
This is a property we will use in our Hartree-Fock discussions. Convince yourself about the correctness of the above expression by setting $n=2$. 

With our definition of the new basis in terms of an orthogonal basis we have
\[
\psi_p(x)  = \sum_{\lambda} C_{p\lambda}\phi_{\lambda}(x).
\]
If the coefficients $C_{p\lambda}$ belong to an orthogonal or unitary matrix, the new basis
is also orthogonal. 
Our Slater determinant in the new basis $\psi_p(x)$ is written as
\[
\frac{1}{\sqrt{A!}}
\left| \begin{array}{ccccc} \psi_{p}(x_1)& \psi_{p}(x_2)& \dots & \dots & \psi_{p}(x_A)\\
                            \psi_{q}(x_1)&\psi_{q}(x_2)& \dots & \dots & \psi_{q}(x_A)\\  
                            \dots & \dots & \dots & \dots & \dots \\
                            \dots & \dots & \dots & \dots & \dots \\
                     \psi_{t}(x_1)&\psi_{t}(x_2)& \dots & \dots & \psi_{t}(x_A)\end{array} \right|=\frac{1}{\sqrt{A!}}
\left| \begin{array}{ccccc} \sum_{\lambda} C_{p\lambda}\phi_{\lambda}(x_1)& \sum_{\lambda} C_{p\lambda}\phi_{\lambda}(x_2)& \dots & \dots & \sum_{\lambda} C_{p\lambda}\phi_{\lambda}(x_A)\\
                            \sum_{\lambda} C_{q\lambda}\phi_{\lambda}(x_1)&\sum_{\lambda} C_{q\lambda}\phi_{\lambda}(x_2)& \dots & \dots & \sum_{\lambda} C_{q\lambda}\phi_{\lambda}(x_A)\\  
                            \dots & \dots & \dots & \dots & \dots \\
                            \dots & \dots & \dots & \dots & \dots \\
                     \sum_{\lambda} C_{t\lambda}\phi_{\lambda}(x_1)&\sum_{\lambda} C_{t\lambda}\phi_{\lambda}(x_2)& \dots & \dots & \sum_{\lambda} C_{t\lambda}\phi_{\lambda}(x_A)\end{array} \right|,
\]
which is nothing but $det(\mathbf{C})det(\Phi)$, with $det(\Phi)$ being the determinant given by the basis functions $\phi_{\lambda}(x)$. 

In our discussions hereafter we will use our definitions of single-particle states above and below the Fermi ($F$) level given by the labels
$ijkl\dots \le F$ for so-called single-hole states and $abcd\dots > F$ for so-called particle states.
For general single-particle states we employ the labels $pqrs\dots$. 

In Eq.~(\ref{FunctionalEPhi}), restated here
\[
  E[\Phi] 
  = \sum_{\mu=1}^A \langle \mu | h | \mu \rangle +
  \frac{1}{2}\sum_{{\mu}=1}^A\sum_{{\nu}=1}^A \langle \mu\nu|\hat{v}|\mu\nu\rangle_{AS},
\]
we found the expression for the energy functional in terms of the basis function $\phi_{\lambda}(\mathbf{r})$. We then  varied the above energy functional with respect to the basis functions $|\mu \rangle$. 
Now we are interested in defining a new basis defined in terms of
a chosen basis as defined in Eq.~(\ref{eq:newbasis}). We can then rewrite the energy functional as
\begin{equation}
  E[\Phi^{HF}] 
  = \sum_{i=1}^A \langle i | h | i \rangle +
  \frac{1}{2}\sum_{ij=1}^A\langle ij|\hat{v}|ij\rangle_{AS}, \label{FunctionalEPhi2}
\end{equation}
where $\Phi^{HF}$ is the new Slater determinant defined by the new basis of Eq.~(\ref{eq:newbasis}). 

Using Eq.~(\ref{eq:newbasis}) we can rewrite Eq.~(\ref{FunctionalEPhi2}) as 
\begin{equation}
  E[\Psi] 
  = \sum_{i=1}^A \sum_{\alpha\beta} C^*_{i\alpha}C_{i\beta}\langle \alpha | h | \beta \rangle +
  \frac{1}{2}\sum_{ij=1}^A\sum_{{\alpha\beta\gamma\delta}} C^*_{i\alpha}C^*_{j\beta}C_{i\gamma}C_{j\delta}\langle \alpha\beta|\hat{v}|\gamma\delta\rangle_{AS}. \label{FunctionalEPhi3}
\end{equation}

We wish now to minimize the above functional. We introduce again a set of Lagrange multipliers, noting that
since $\langle i | j \rangle = \delta_{i,j}$ and $\langle \alpha | \beta \rangle = \delta_{\alpha,\beta}$, 
the coefficients $C_{i\gamma}$ obey the relation
\[
 \langle i | j \rangle=\delta_{i,j}=\sum_{\alpha\beta} C^*_{i\alpha}C_{i\beta}\langle \alpha | \beta \rangle=
\sum_{\alpha} C^*_{i\alpha}C_{i\alpha},
\]
which allows us to define a functional to be minimized that reads
\begin{equation}
  F[\Phi^{HF}]=E[\Phi^{HF}] - \sum_{i=1}^A\epsilon_i\sum_{\alpha} C^*_{i\alpha}C_{i\alpha}.
\end{equation}

Minimizing with respect to $C^*_{i\alpha}$, remembering that the equations for $C^*_{i\alpha}$ and $C_{i\alpha}$
can be written as two  independent equations, we obtain
\[
\frac{d}{dC^*_{i\alpha}}\left[  E[\Phi^{HF}] - \sum_{j}\epsilon_j\sum_{\alpha} C^*_{j\alpha}C_{j\alpha}\right]=0,
\]
which yields for every single-particle state $i$ and index $\alpha$ (recalling that the coefficients $C_{i\alpha}$ are matrix elements of a unitary (or orthogonal for a real symmetric matrix) matrix)
the following Hartree-Fock equations
\[
\sum_{\beta} C_{i\beta}\langle \alpha | h | \beta \rangle+
\sum_{j=1}^A\sum_{\beta\gamma\delta} C^*_{j\beta}C_{j\delta}C_{i\gamma}\langle \alpha\beta|\hat{v}|\gamma\delta\rangle_{AS}=\epsilon_i^{HF}C_{i\alpha}.
\]

We can rewrite this equation as (changing dummy variables)
\[
\sum_{\beta} \left\{\langle \alpha | h | \beta \rangle+
\sum_{j}^A\sum_{\gamma\delta} C^*_{j\gamma}C_{j\delta}\langle \alpha\gamma|\hat{v}|\beta\delta\rangle_{AS}\right\}C_{i\beta}=\epsilon_i^{HF}C_{i\alpha}.
\]
Note that the sums over greek indices run over the number of basis set functions (in principle an infinite number).

Defining 
\[
h_{\alpha\beta}^{HF}=\langle \alpha | h | \beta \rangle+
\sum_{j=1}^A\sum_{\gamma\delta} C^*_{j\gamma}C_{j\delta}\langle \alpha\gamma|\hat{v}|\beta\delta\rangle_{AS},
\]
we can rewrite the new equations as 
\begin{equation}
\sum_{\beta}h_{\alpha\beta}^{HF}C_{i\beta}=\epsilon_i^{HF}C_{i\alpha}. \label{eq:newhf}
\end{equation}
The latter is nothing but a standard eigenvalue problem. Compared with Eq.~(\ref{eq:hartreefockcoordinatespace}),
we see that we do not need to compute any integrals in an iterative procedure for solving the equations.
It suffices to tabulate the matrix elements $\langle \alpha | h | \beta \rangle$ and $\langle \alpha\gamma|\hat{v}|\beta\delta\rangle_{AS}$ once and for all. Successive iterations require thus only a look-up in tables over one-body and two-body matrix elements. These details will be discussed below when we solve the Hartree-Fock equations numerical. 

\subsection*{Hartree-Fock algorithm}

Our Hartree-Fock matrix  is thus
\[
\hat{h}_{\alpha\beta}^{HF}=\langle \alpha | \hat{h}_0 | \beta \rangle+
\sum_{j=1}^A\sum_{\gamma\delta} C^*_{j\gamma}C_{j\delta}\langle \alpha\gamma|\hat{v}|\beta\delta\rangle_{AS}.
\]
The Hartree-Fock equations are solved in an iterative waym starting with a guess for the coefficients $C_{j\gamma}=\delta_{j,\gamma}$ and solving the equations by diagonalization till the new single-particle energies
$\epsilon_i^{\mathrm{HF}}$ do not change anymore by a prefixed quantity. 

Normally we assume that the single-particle basis $|\beta\rangle$ forms an eigenbasis for the operator
$\hat{h}_0$, meaning that the Hartree-Fock matrix becomes  
\[
\hat{h}_{\alpha\beta}^{HF}=\epsilon_{\alpha}\delta_{\alpha,\beta}+
\sum_{j=1}^A\sum_{\gamma\delta} C^*_{j\gamma}C_{j\delta}\langle \alpha\gamma|\hat{v}|\beta\delta\rangle_{AS}.
\]
The Hartree-Fock eigenvalue problem
\[
\sum_{\beta}\hat{h}_{\alpha\beta}^{HF}C_{i\beta}=\epsilon_i^{\mathrm{HF}}C_{i\alpha},
\]
can be written out in a more compact form as
\[
\hat{h}^{HF}\hat{C}=\epsilon^{\mathrm{HF}}\hat{C}. 
\]

The Hartree-Fock equations are, in their simplest form, solved in an iterative way, starting with a guess for the
coefficients $C_{i\alpha}$. We label the coefficients as $C_{i\alpha}^{(n)}$, where the subscript $n$ stands for iteration $n$.
To set up the algorithm we can proceed as follows:

\begin{itemize}
 \item We start with a guess $C_{i\alpha}^{(0)}=\delta_{i,\alpha}$. Alternatively, we could have used random starting values as long as the vectors are normalized. Another possibility is to give states below the Fermi level a larger weight.

 \item The Hartree-Fock matrix simplifies then to (assuming that the coefficients $C_{i\alpha} $  are real)
\end{itemize}

\noindent
\[
\hat{h}_{\alpha\beta}^{HF}=\epsilon_{\alpha}\delta_{\alpha,\beta}+
\sum_{j = 1}^A\sum_{\gamma\delta} C_{j\gamma}^{(0)}C_{j\delta}^{(0)}\langle \alpha\gamma|\hat{v}|\beta\delta\rangle_{AS}.
\]

Solving the Hartree-Fock eigenvalue problem yields then new eigenvectors $C_{i\alpha}^{(1)}$ and eigenvalues
$\epsilon_i^{HF(1)}$. 
\begin{itemize}
 \item With the new eigenvalues we can set up a new Hartree-Fock potential 
\end{itemize}

\noindent
\[
\sum_{j = 1}^A\sum_{\gamma\delta} C_{j\gamma}^{(1)}C_{j\delta}^{(1)}\langle \alpha\gamma|\hat{v}|\beta\delta\rangle_{AS}.
\]
The diagonalization with the new Hartree-Fock potential yields new eigenvectors and eigenvalues.
This process is continued till for example
\[
\frac{\sum_{p} |\epsilon_i^{(n)}-\epsilon_i^{(n-1)}|}{m} \le \lambda,  
\]
where $\lambda$ is a user prefixed quantity ($\lambda \sim 10^{-8}$ or smaller) and $p$ runs over all calculated single-particle
energies and $m$ is the number of single-particle states.

\subsection*{Analysis of Hartree-Fock equations and Koopman's theorem}

We can rewrite the ground state energy by adding and subtracting $\hat{u}^{HF}(x_i)$ 
\[
  E_0^{HF} =\langle \Phi_0 | \hat{H} | \Phi_0\rangle = 
\sum_{i\le F}^A \langle i | \hat{h}_0 +\hat{u}^{HF}| j\rangle+ \frac{1}{2}\sum_{i\le F}^A\sum_{j \le F}^A\left[\langle ij |\hat{v}|ij \rangle-\langle ij|\hat{v}|ji\rangle\right]-\sum_{i\le F}^A \langle i |\hat{u}^{HF}| i\rangle,
\]
which results in
\[
  E_0^{HF}
  = \sum_{i\le F}^A \varepsilon_i^{HF} + \frac{1}{2}\sum_{i\le F}^A\sum_{j \le F}^A\left[\langle ij |\hat{v}|ij \rangle-\langle ij|\hat{v}|ji\rangle\right]-\sum_{i\le F}^A \langle i |\hat{u}^{HF}| i\rangle.
\]
Our single-particle states $ijk\dots$ are now single-particle states obtained from the solution of the Hartree-Fock equations.

Using our definition of the Hartree-Fock single-particle energies we obtain then the following expression for the total ground-state energy
\[
  E_0^{HF}
  = \sum_{i\le F}^A \varepsilon_i - \frac{1}{2}\sum_{i\le F}^A\sum_{j \le F}^A\left[\langle ij |\hat{v}|ij \rangle-\langle ij|\hat{v}|ji\rangle\right].
\]
This form will be used in our discussion of Koopman's theorem.

In the   atomic physics case we have 
\[
  E[\Phi^{\mathrm{HF}}(N)] 
  = \sum_{i=1}^H \langle i | \hat{h}_0 | i \rangle +
  \frac{1}{2}\sum_{ij=1}^N\langle ij|\hat{v}|ij\rangle_{AS},
\]
where $\Phi^{\mathrm{HF}}(N)$ is the new Slater determinant defined by the new basis of Eq.~(\ref{eq:newbasis})
for $N$ electrons (same $Z$).  If we assume that the single-particle wave functions in the new basis do not change 
when we remove one electron or add one electron, we can then define the corresponding energy for the $N-1$ systems as 
\[
  E[\Phi^{\mathrm{HF}}(N-1)] 
  = \sum_{i=1; i\ne k}^N \langle i | \hat{h}_0 | i \rangle +
  \frac{1}{2}\sum_{ij=1;i,j\ne k}^N\langle ij|\hat{v}|ij\rangle_{AS},
\]
where we have removed a single-particle state $k\le F$, that is a state below the Fermi level.  

Calculating the difference 
\[
  E[\Phi^{\mathrm{HF}}(N)]-   E[\Phi^{\mathrm{HF}}(N-1)] = \langle k | \hat{h}_0 | k \rangle +
  \frac{1}{2}\sum_{i=1;i\ne k}^N\langle ik|\hat{v}|ik\rangle_{AS} + \frac{1}{2}\sum_{j=1;j\ne k}^N\langle kj|\hat{v}|kj\rangle_{AS},
\]
we obtain
\[
  E[\Phi^{\mathrm{HF}}(N)]-   E[\Phi^{\mathrm{HF}}(N-1)] = \langle k | \hat{h}_0 | k \rangle +\sum_{j=1}^N\langle kj|\hat{v}|kj\rangle_{AS}
\]
which is just our definition of the Hartree-Fock single-particle energy
\[
  E[\Phi^{\mathrm{HF}}(N)]-   E[\Phi^{\mathrm{HF}}(N-1)] = \epsilon_k^{\mathrm{HF}} 
\]

Similarly, we can now compute the difference (we label the single-particle states above the Fermi level as $abcd > F$)
\[
  E[\Phi^{\mathrm{HF}}(N+1)]-   E[\Phi^{\mathrm{HF}}(N)]= \epsilon_a^{\mathrm{HF}}. 
\]
These two equations can thus be used to the electron affinity or ionization energies, respectively. 
Koopman's theorem states that for example the ionization energy of a closed-shell system is given by the energy of the highest occupied single-particle state.  If we assume that changing the number of electrons from $N$ to $N+1$ does not change the Hartree-Fock single-particle energies and eigenfunctions, then Koopman's theorem simply states that the ionization energy of an atom is given by the single-particle energy of the last bound state. In a similar way, we can also define the electron affinities. 

As an example, consider a simple model for atomic sodium, Na. Neutral sodium has eleven electrons, 
with the weakest bound one being confined the $3s$ single-particle quantum numbers. The energy needed to remove an electron from neutral sodium is rather small, 5.1391 eV, a feature which pertains to all alkali metals.
Having performed a  Hartree-Fock calculation for neutral sodium would then allows us to compute the
ionization energy by using the single-particle energy for the $3s$ states, namely $\epsilon_{3s}^{\mathrm{HF}}$. 

From these considerations, we see that Hartree-Fock theory allows us to make a connection between experimental 
observables (here ionization and affinity energies) and the underlying interactions between particles.  
In this sense, we are now linking the dynamics and structure of a many-body system with the laws of motion which govern the system. Our approach is a reductionistic one, meaning that we want to understand the laws of motion 
in terms of the particles or degrees of freedom which we believe are the fundamental ones. Our Slater determinant, being constructed as the product of various single-particle functions, follows this philosophy.

With similar arguments as in atomic physics, we can now use Hartree-Fock theory to make a link
between nuclear forces and separation energies. Changing to nuclear system, we define
\[
  E[\Phi^{\mathrm{HF}}(A)] 
  = \sum_{i=1}^A \langle i | \hat{h}_0 | i \rangle +
  \frac{1}{2}\sum_{ij=1}^A\langle ij|\hat{v}|ij\rangle_{AS},
\]
where $\Phi^{\mathrm{HF}}(A)$ is the new Slater determinant defined by the new basis of Eq.~(\ref{eq:newbasis})
for $A$ nucleons, where $A=N+Z$, with $N$ now being the number of neutrons and $Z$ th enumber of protons.  If we assume again that the single-particle wave functions in the new basis do not change from a nucleus with $A$ nucleons to a nucleus with $A-1$  nucleons, we can then define the corresponding energy for the $A-1$ systems as 
\[
  E[\Phi^{\mathrm{HF}}(A-1)] 
  = \sum_{i=1; i\ne k}^A \langle i | \hat{h}_0 | i \rangle +
  \frac{1}{2}\sum_{ij=1;i,j\ne k}^A\langle ij|\hat{v}|ij\rangle_{AS},
\]
where we have removed a single-particle state $k\le F$, that is a state below the Fermi level.  

Calculating the difference 
\[
  E[\Phi^{\mathrm{HF}}(A)]-   E[\Phi^{\mathrm{HF}}(A-1)] 
  = \langle k | \hat{h}_0 | k \rangle +
  \frac{1}{2}\sum_{i=1;i\ne k}^A\langle ik|\hat{v}|ik\rangle_{AS} + \frac{1}{2}\sum_{j=1;j\ne k}^A\langle kj|\hat{v}|kj\rangle_{AS},
\]
which becomes 
\[
  E[\Phi^{\mathrm{HF}}(A)]-   E[\Phi^{\mathrm{HF}}(A-1)] 
  = \langle k | \hat{h}_0 | k \rangle +\sum_{j=1}^A\langle kj|\hat{v}|kj\rangle_{AS}
\]
which is just our definition of the Hartree-Fock single-particle energy
\[
  E[\Phi^{\mathrm{HF}}(A)]-   E[\Phi^{\mathrm{HF}}(A-1)] 
  = \epsilon_k^{\mathrm{HF}} 
\]

Similarly, we can now compute the difference (recall that the single-particle states $abcd > F$)
\[
  E[\Phi^{\mathrm{HF}}(A+1)]-   E[\Phi^{\mathrm{HF}}(A)]= \epsilon_a^{\mathrm{HF}}. 
\]
If we then recall that the binding energy differences 
\[
BE(A)-BE(A-1) \hspace{0.5cm} \mathrm{and} \hspace{0.5cm} BE(A+1)-BE(A), 
\]
define the separation energies, we see that the Hartree-Fock single-particle energies can be used to
define separation energies. We have thus our first link between nuclear forces (included in the potential energy term) and an observable quantity defined by differences in binding energies. 

We have thus the following interpretations (if the single-particle fields do not change)
\[
BE(A)-BE(A-1)\approx  E[\Phi^{\mathrm{HF}}(A)]-   E[\Phi^{\mathrm{HF}}(A-1)] 
  = \epsilon_k^{\mathrm{HF}}, 
\]
and
\[
BE(A+1)-BE(A)\approx  E[\Phi^{\mathrm{HF}}(A+1)]-   E[\Phi^{\mathrm{HF}}(A)] =  \epsilon_a^{\mathrm{HF}}. 
\]
If  we use $^{16}\mbox{O}$ as our closed-shell nucleus, we could then interpret the separation energy
\[
BE(^{16}\mathrm{O})-BE(^{15}\mathrm{O})\approx \epsilon_{0p^{\nu}_{1/2}}^{\mathrm{HF}}, 
\]
and
\[
BE(^{16}\mathrm{O})-BE(^{15}\mathrm{N})\approx \epsilon_{0p^{\pi}_{1/2}}^{\mathrm{HF}}.
\]

Similalry, we could interpret
\[
BE(^{17}\mathrm{O})-BE(^{16}\mathrm{O})\approx \epsilon_{0d^{\nu}_{5/2}}^{\mathrm{HF}}, 
\]
and 
\[
BE(^{17}\mathrm{F})-BE(^{16}\mathrm{O})\approx\epsilon_{0d^{\pi}_{5/2}}^{\mathrm{HF}}.
\]
We can continue like this for all $A\pm 1$ nuclei where $A$ is a good closed-shell (or subshell closure)
nucleus. Examples are $^{22}\mbox{O}$, $^{24}\mbox{O}$, $^{40}\mbox{Ca}$, $^{48}\mbox{Ca}$, $^{52}\mbox{Ca}$, $^{54}\mbox{Ca}$, $^{56}\mbox{Ni}$, 
$^{68}\mbox{Ni}$, $^{78}\mbox{Ni}$, $^{90}\mbox{Zr}$, $^{88}\mbox{Sr}$, $^{100}\mbox{Sn}$, $^{132}\mbox{Sn}$ and $^{208}\mbox{Pb}$, to mention some possile cases.

We can thus make our first interpretation of the separation energies in terms of the simplest
possible many-body theory. 
If we also recall that the so-called energy gap for neutrons (or protons) is defined as
\[
\Delta S_n= 2BE(N,Z)-BE(N-1,Z)-BE(N+1,Z),
\]
for neutrons and the corresponding gap for protons
\[
\Delta S_p= 2BE(N,Z)-BE(N,Z-1)-BE(N,Z+1),
\]
we can define the neutron and proton energy gaps for $^{16}\mbox{O}$ as
\[
\Delta S_{\nu}=\epsilon_{0d^{\nu}_{5/2}}^{\mathrm{HF}}-\epsilon_{0p^{\nu}_{1/2}}^{\mathrm{HF}}, 
\]
and 
\[
\Delta S_{\pi}=\epsilon_{0d^{\pi}_{5/2}}^{\mathrm{HF}}-\epsilon_{0p^{\pi}_{1/2}}^{\mathrm{HF}}. 
\]

% --- begin exercise ---
\begin{doconceexercise}
\refstepcounter{doconceexercisecounter}

\exercisesection*{Exercise \thedoconceexercisecounter: Derivation of Hartree-Fock equations}
                             

Consider a Slater determinant built up of single-particle orbitals $\psi_{\lambda}$, 
with $\lambda = 1,2,\dots,N$.

The unitary transformation
\[
\psi_a  = \sum_{\lambda} C_{a\lambda}\phi_{\lambda},
\]
brings us into the new basis.  
The new basis has quantum numbers $a=1,2,\dots,N$.

% --- begin subexercise ---
\subex{a)}
Show that the new basis is orthonormal.

% --- end subexercise ---

% --- begin subexercise ---
\subex{b)}
Show that the new Slater determinant constructed from the new single-particle wave functions can be
written as the determinant based on the previous basis and the determinant of the matrix $C$.

% --- end subexercise ---

% --- begin subexercise ---
\subex{c)}
Show that the old and the new Slater determinants are equal up to a complex constant with absolute value unity.

% --- begin hint in exercise ---

\paragraph{Hint.}
Use the fact that $C$ is a unitary matrix.

% --- end hint in exercise ---

% --- end subexercise ---

\end{doconceexercise}
% --- end exercise ---

% --- begin exercise ---
\begin{doconceexercise}
\refstepcounter{doconceexercisecounter}

\exercisesection*{Exercise \thedoconceexercisecounter: Derivation of Hartree-Fock equations}
                             

Consider the  Slater  determinant
\[
\Phi_{0}=\frac{1}{\sqrt{n!}}\sum_{p}(-)^{p}P
\prod_{i=1}^{n}\psi_{\alpha_{i}}(x_{i}).
\]
A small variation in this function is given by
\[
\delta\Phi_{0}=\frac{1}{\sqrt{n!}}\sum_{p}(-)^{p}P
\psi_{\alpha_{1}}(x_{1})\psi_{\alpha_{2}}(x_{2})\dots
\psi_{\alpha_{i-1}}(x_{i-1})(\delta\psi_{\alpha_{i}}(x_{i}))
\psi_{\alpha_{i+1}}(x_{i+1})\dots\psi_{\alpha_{n}}(x_{n}).
\]

% --- begin subexercise ---
\subex{a)}
Show that
\[
\langle \delta\Phi_{0}|\sum_{i=1}^{n}\left\{t(x_{i})+u(x_{i})
\right\}+\frac{1}{2}
\sum_{i\neq j=1}^{n}v(x_{i},x_{j})|\Phi_{0}\rangle=\sum_{i=1}^{n}\langle \delta\psi_{\alpha_{i}}|\hat{t}+\hat{u}
|\phi_{\alpha_{i}}\rangle
+\sum_{i\neq j=1}^{n}\left\{\langle\delta\psi_{\alpha_{i}}
\psi_{\alpha_{j}}|\hat{v}|\psi_{\alpha_{i}}\psi_{\alpha_{j}}\rangle-
\langle\delta\psi_{\alpha_{i}}\psi_{\alpha_{j}}|\hat{v}
|\psi_{\alpha_{j}}\psi_{\alpha_{i}}\rangle\right\}
\]

% --- end subexercise ---

\end{doconceexercise}
% --- end exercise ---

% --- begin exercise ---
\begin{doconceexercise}
\refstepcounter{doconceexercisecounter}

\exercisesection*{Exercise \thedoconceexercisecounter: Developing a  Hartree-Fock program}
                             

Neutron drops are a powerful theoretical laboratory for testing,
validating and improving nuclear structure models. Indeed, all
approaches to nuclear structure, from ab initio theory to shell model
to density functional theory are applicable in such systems. We will,
therefore, use neutron drops as a test system for setting up a
Hartree-Fock code.  This program can later be extended to studies of
the binding energy of nuclei like $^{16}$O or $^{40}$Ca. The
single-particle energies obtained by solving the Hartree-Fock
equations can then be directly related to experimental separation
energies. 
Since Hartree-Fock theory is the starting point for
several many-body techniques (density functional theory, random-phase
approximation, shell-model etc), the aim here is to develop a computer
program to solve the Hartree-Fock equations in a given single-particle basis,
here the harmonic oscillator.

The Hamiltonian for a system of $N$ neutron drops confined in a
harmonic potential reads
\[
\hat{H} = \sum_{i=1}^{N} \frac{\hat{p}_{i}^{2}}{2m}+\sum_{i=1}^{N} \frac{1}{2} m\omega {r}_{i}^{2}+\sum_{i<j} \hat{V}_{ij},
\]
with $\hbar^{2}/2m = 20.73$ fm$^{2}$, $mc^{2} = 938.90590$ MeV, and 
$\hat{V}_{ij}$ is the two-body interaction potential whose 
matrix elements are precalculated
and to be read in by you.

The Hartree-Fock algorithm can be broken down as follows. We recall that  our Hartree-Fock matrix  is 
\[
\hat{h}_{\alpha\beta}^{HF}=\langle \alpha \vert\hat{h}_0 \vert \beta \rangle+
\sum_{j=1}^N\sum_{\gamma\delta} C^*_{j\gamma}C_{j\delta}\langle \alpha\gamma|V|\beta\delta\rangle_{AS}.
\]
Normally we assume that the single-particle basis $\vert\beta\rangle$
forms an eigenbasis for the operator $\hat{h}_0$ (this is our case), meaning that the
Hartree-Fock matrix becomes
\[
\hat{h}_{\alpha\beta}^{HF}=\epsilon_{\alpha}\delta_{\alpha,\beta}+
\sum_{j=1}^N\sum_{\gamma\delta} C^*_{j\gamma}C_{j\delta}\langle \alpha\gamma|V|\beta\delta\rangle_{AS}.
\]
The Hartree-Fock eigenvalue problem
\[
\sum_{\beta}\hat{h}_{\alpha\beta}^{HF}C_{i\beta}=\epsilon_i^{\mathrm{HF}}C_{i\alpha},
\]
can be written out in a more compact form as
\[
\hat{h}^{HF}\hat{C}=\epsilon^{\mathrm{HF}}\hat{C}. 
\]

The equations are often rewritten in terms of a so-called density matrix,
which is defined as 
\begin{equation}
\rho_{\gamma\delta}=\sum_{i=1}^{N}\langle\gamma|i\rangle\langle i|\delta\rangle = \sum_{i=1}^{N}C_{i\gamma}C^*_{i\delta}.
\end{equation}
It means that we can rewrite the Hartree-Fock Hamiltonian as
\[
\hat{h}_{\alpha\beta}^{HF}=\epsilon_{\alpha}\delta_{\alpha,\beta}+
\sum_{\gamma\delta} \rho_{\gamma\delta}\langle \alpha\gamma|V|\beta\delta\rangle_{AS}.
\]
It is convenient to use the density matrix since we can precalculate in every iteration the product of two eigenvector components $C$. 

Note that $\langle \alpha\vert\hat{h}_0\vert\beta \rangle$ denotes the
matrix elements of the one-body part of the starting hamiltonian. For
self-bound nuclei $\langle \alpha\vert\hat{h}_0\vert\beta \rangle$ is the
kinetic energy, whereas for neutron drops, $\langle \alpha \vert \hat{h}_0 \vert \beta \rangle$ represents the harmonic oscillator hamiltonian since
the system is confined in a harmonic trap. If we are working in a
harmonic oscillator basis with the same $\omega$ as the trapping
potential, then $\langle \alpha\vert\hat{h}_0 \vert \beta \rangle$ is
diagonal.

The python
\href{{https://github.com/CompPhysics/ManyBodyMethods/tree/master/doc/src/hfock/Code}}{program}
shows how one can, in a brute force way read in matrix elements in
$m$-scheme and compute the Hartree-Fock single-particle energies for
four major shells. The interaction which has been used is the
so-called N3LO interaction of \href{{http://journals.aps.org/prc/abstract/10.1103/PhysRevC.68.041001}}{Machleidt and
Entem}
using the \href{{http://journals.aps.org/prc/abstract/10.1103/PhysRevC.75.061001}}{Similarity Renormalization
Group}
approach method to renormalize the interaction, using an oscillator
energy $\hbar\omega=10$ MeV.

The nucleon-nucleon two-body matrix elements are in $m$-scheme and are fully anti-symmetrized. The Hartree-Fock programs uses the density matrix discussed above in order to compute the Hartree-Fock matrix.
Here we display the Hartree-Fock part only, assuming that single-particle data and two-body matrix elements have already been read in. 





































































































\begin{minted}[fontsize=\fontsize{9pt}{9pt},linenos=false,mathescape,baselinestretch=1.0,fontfamily=tt,xleftmargin=7mm]{python}
import numpy as np 
from decimal import Decimal
# expectation value for the one body part, Harmonic oscillator in three dimensions
def onebody(i, n, l):
        homega = 10.0
        return homega*(2*n[i] + l[i] + 1.5)

if __name__ == '__main__':
        
    Nparticles = 16
    """ Read quantum numbers from file """
    index = []
    n = []
    l = []
    j = []
    mj = []
    tz = []
    spOrbitals = 0
    with open("nucleispnumbers.dat", "r") as qnumfile:
                for line in qnumfile:
                        nums = line.split()
                        if len(nums) != 0:
                                index.append(int(nums[0]))
                                n.append(int(nums[1]))
                                l.append(int(nums[2]))
                                j.append(int(nums[3]))
                                mj.append(int(nums[4]))
                                tz.append(int(nums[5]))
                                spOrbitals += 1


    """ Read two-nucleon interaction elements (integrals) from file, brute force 4-dim array """
    nninteraction = np.zeros([spOrbitals, spOrbitals, spOrbitals, spOrbitals])
    with open("nucleitwobody.dat", "r") as infile:
        for line in infile:
                number = line.split()
                a = int(number[0]) - 1
                b = int(number[1]) - 1
                c = int(number[2]) - 1
                d = int(number[3]) - 1
                nninteraction[a][b][c][d] = Decimal(number[4])
        """ Set up single-particle integral """
        singleparticleH = np.zeros(spOrbitals)
        for i in range(spOrbitals):
                singleparticleH[i] = Decimal(onebody(i, n, l))
        
        """ Star HF-iterations, preparing variables and density matrix """

        """ Coefficients for setting up density matrix, assuming only one along the diagonals """
        C = np.eye(spOrbitals) # HF coefficients
        DensityMatrix = np.zeros([spOrbitals,spOrbitals])
        for gamma in range(spOrbitals):
            for delta in range(spOrbitals):
                sum = 0.0
                for i in range(Nparticles):
                    sum += C[gamma][i]*C[delta][i]
                DensityMatrix[gamma][delta] = Decimal(sum)
        maxHFiter = 100
        epsilon =  1.0e-5 
        difference = 1.0
        hf_count = 0
        oldenergies = np.zeros(spOrbitals)
        newenergies = np.zeros(spOrbitals)
        while hf_count < maxHFiter and difference > epsilon:
                print("############### Iteration %i ###############" % hf_count)
                HFmatrix = np.zeros([spOrbitals,spOrbitals])            
                for alpha in range(spOrbitals):
                        for beta in range(spOrbitals):
                            """  If tests for three-dimensional systems, including isospin conservation """
                            if l[alpha] != l[beta] and j[alpha] != j[beta] and mj[alpha] != mj[beta] and tz[alpha] != tz[beta]: continue
                            """  Setting up the Fock matrix using the density matrix and antisymmetrized NN interaction in m-scheme """
                            sumFockTerm = 0.0
                            for gamma in range(spOrbitals):
                                for delta in range(spOrbitals):
                                    if (mj[alpha]+mj[gamma]) != (mj[beta]+mj[delta]) and (tz[alpha]+tz[gamma]) != (tz[beta]+tz[delta]): continue
                                    sumFockTerm += DensityMatrix[gamma][delta]*nninteraction[alpha][gamma][beta][delta]
                            HFmatrix[alpha][beta] = Decimal(sumFockTerm)
                            """  Adding the one-body term, here plain harmonic oscillator """
                            if beta == alpha:   HFmatrix[alpha][alpha] += singleparticleH[alpha]
                spenergies, C = np.linalg.eigh(HFmatrix)
                """ Setting up new density matrix in m-scheme """
                DensityMatrix = np.zeros([spOrbitals,spOrbitals])
                for gamma in range(spOrbitals):
                    for delta in range(spOrbitals):
                        sum = 0.0
                        for i in range(Nparticles):
                            sum += C[gamma][i]*C[delta][i]
                        DensityMatrix[gamma][delta] = Decimal(sum)
                newenergies = spenergies
                """ Brute force computation of difference between previous and new sp HF energies """
                sum =0.0
                for i in range(spOrbitals):
                    sum += (abs(newenergies[i]-oldenergies[i]))/spOrbitals
                difference = sum
                oldenergies = newenergies
                print ("Single-particle energies, ordering may have changed ")
                for i in range(spOrbitals):
                    print('{0:4d}  {1:.4f}'.format(i, Decimal(oldenergies[i])))
                hf_count += 1


\end{minted}

Running the program, one finds that the lowest-lying states for a nucleus like $^{16}\mbox{O}$, we see that the nucleon-nucleon force brings a natural spin-orbit splitting for the $0p$ states (or other states except the $s$-states).
Since we are using the $m$-scheme for our calculations, we observe that there are several states with the same
eigenvalues. The number of eigenvalues corresponds to the degeneracy $2j+1$ and is well respected in our calculations, as see from the table here.

The values of the lowest-lying states are ($\pi$ for protons and $\nu$ for neutrons)

\begin{quote}
\begin{tabular}{cc}
\hline
\multicolumn{1}{c}{ Quantum numbers } & \multicolumn{1}{c}{ Energy [MeV] } \\
\hline
$0s_{1/2}^{\pi}$ & -40.4602     \\
$0s_{1/2}^{\pi}$ & -40.4602     \\
$0s_{1/2}^{\nu}$ & -40.6426     \\
$0s_{1/2}^{\nu}$ & -40.6426     \\
$0p_{1/2}^{\pi}$ & -6.7133      \\
$0p_{1/2}^{\pi}$ & -6.7133      \\
$0p_{1/2}^{\nu}$ & -6.8403      \\
$0p_{1/2}^{\nu}$ & -6.8403      \\
$0p_{3/2}^{\pi}$ & -11.5886     \\
$0p_{3/2}^{\pi}$ & -11.5886     \\
$0p_{3/2}^{\pi}$ & -11.5886     \\
$0p_{3/2}^{\pi}$ & -11.5886     \\
$0p_{3/2}^{\nu}$ & -11.7201     \\
$0p_{3/2}^{\nu}$ & -11.7201     \\
$0p_{3/2}^{\nu}$ & -11.7201     \\
$0p_{3/2}^{\nu}$ & -11.7201     \\
$0d_{5/2}^{\pi}$ & 18.7589      \\
$0d_{5/2}^{\nu}$ & 18.8082      \\
\hline
\end{tabular}
\end{quote}

\noindent
We can use these results to attempt our first link with experimental data, namely to compute the shell gap or the separation energies. The shell gap for neutrons is given by
\[
\Delta S_n= 2BE(N,Z)-BE(N-1,Z)-BE(N+1,Z).
\]
For $^{16}\mbox{O}$  we have an experimental value for the  shell gap of $11.51$ MeV for neutrons, while our Hartree-Fock calculations result in $25.65$ MeV. This means that correlations beyond a simple Hartree-Fock calculation with a two-body force play an important role in nuclear physics.
The splitting between the $0p_{3/2}^{\nu}$ and the $0p_{1/2}^{\nu}$ state is 4.88 MeV, while the experimental value for the gap between the ground state $1/2^{-}$ and the first excited $3/2^{-}$ states is 6.08 MeV. The two-nucleon spin-orbit force plays a central role here. In our discussion of nuclear forces we will see how the spin-orbit force comes into play here.

\end{doconceexercise}
% --- end exercise ---

\subsection*{Hartree-Fock in second quantization and stability of HF solution}

We wish now to derive the Hartree-Fock equations using our second-quantized formalism and study the stability of the equations. 
Our ansatz for the ground state of the system is approximated as (this is our representation of a Slater determinant in second quantization)
\[   
|\Phi_0\rangle = |c\rangle = a^{\dagger}_i a^{\dagger}_j \dots a^{\dagger}_l|0\rangle.
\]
We wish to determine $\hat{u}^{HF}$ so that 
$E_0^{HF}= \langle c|\hat{H}| c\rangle$ becomes a local minimum. 

In our analysis here we will need Thouless' theorem, which states that
an arbitrary Slater determinant $|c'\rangle$ which is not orthogonal to a determinant
$| c\rangle ={\displaystyle\prod_{i=1}^{n}}
a_{\alpha_{i}}^{\dagger}|0\rangle$, can be written as
\[
|c'\rangle=exp\left\{\sum_{a>F}\sum_{i\le F}C_{ai}a_{a}^{\dagger}a_{i}\right\}| c\rangle 
\]

Let us give a simple proof of Thouless' theorem. The theorem states that we can make a linear combination av particle-hole excitations  with respect to a given reference state $\vert c\rangle$. With this linear combination, we can make a new Slater determinant $\vert c'\rangle $ which is not orthogonal to 
$\vert c\rangle$, that is
\[
\langle c|c'\rangle \ne 0.
\] 
To show this we need some intermediate steps. The exponential product of two operators  $\exp{\hat{A}}\times\exp{\hat{B}}$ is equal to $\exp{(\hat{A}+\hat{B})}$ only if the two operators commute, that is
\[
[\hat{A},\hat{B}] = 0.
\]

\subsection*{Thouless' theorem}

If the operators do not commute, we need to resort to the \href{{http://www.encyclopediaofmath.org/index.php/Campbell%E2%80%93Hausdorff_formula}}{Baker-Campbell-Hauersdorf}. This relation states that
\[
\exp{\hat{C}}=\exp{\hat{A}}\exp{\hat{B}},
\]
with 
\[
\hat{C}=\hat{A}+\hat{B}+\frac{1}{2}[\hat{A},\hat{B}]+\frac{1}{12}[[\hat{A},\hat{B}],\hat{B}]-\frac{1}{12}[[\hat{A},\hat{B}],\hat{A}]+\dots
\]
From these relations, we note that 
in our expression  for $|c'\rangle$ we have commutators of the type
\[
[a_{a}^{\dagger}a_{i},a_{b}^{\dagger}a_{j}],
\]
and it is easy to convince oneself that these commutators, or higher powers thereof, are all zero. This means that we can write out our new representation of a Slater determinant as
\[
|c'\rangle=exp\left\{\sum_{a>F}\sum_{i\le F}C_{ai}a_{a}^{\dagger}a_{i}\right\}| c\rangle=\prod_{i}\left\{1+\sum_{a>F}C_{ai}a_{a}^{\dagger}a_{i}+\left(\sum_{a>F}C_{ai}a_{a}^{\dagger}a_{i}\right)^2+\dots\right\}| c\rangle
\]

We note that
\[
\prod_{i}\sum_{a>F}C_{ai}a_{a}^{\dagger}a_{i}\sum_{b>F}C_{bi}a_{b}^{\dagger}a_{i}| c\rangle =0,
\]
and all higher-order powers of these combinations of creation and annihilation operators disappear 
due to the fact that $(a_i)^n| c\rangle =0$ when $n > 1$. This allows us to rewrite the expression for $|c'\rangle $ as
\[
|c'\rangle=\prod_{i}\left\{1+\sum_{a>F}C_{ai}a_{a}^{\dagger}a_{i}\right\}| c\rangle,
\]
which we can rewrite as 
\[
|c'\rangle=\prod_{i}\left\{1+\sum_{a>F}C_{ai}a_{a}^{\dagger}a_{i}\right\}| a^{\dagger}_{i_1} a^{\dagger}_{i_2} \dots a^{\dagger}_{i_n}|0\rangle.
\]
The last equation can be written as
\begin{align}
|c'\rangle&=\prod_{i}\left\{1+\sum_{a>F}C_{ai}a_{a}^{\dagger}a_{i}\right\}| a^{\dagger}_{i_1} a^{\dagger}_{i_2} \dots a^{\dagger}_{i_n}|0\rangle=\left(1+\sum_{a>F}C_{ai_1}a_{a}^{\dagger}a_{i_1}\right)a^{\dagger}_{i_1} \\
& \times\left(1+\sum_{a>F}C_{ai_2}a_{a}^{\dagger}a_{i_2}\right)a^{\dagger}_{i_2} \dots |0\rangle=\prod_{i}\left(a^{\dagger}_{i}+\sum_{a>F}C_{ai}a_{a}^{\dagger}\right)|0\rangle.
\end{align}

\subsection*{New operators}

If we define a new creation operator 
\begin{equation}
b^{\dagger}_{i}=a^{\dagger}_{i}+\sum_{a>F}C_{ai}a_{a}^{\dagger}, \label{eq:newb}
\end{equation}
we have 
\[
|c'\rangle=\prod_{i}b^{\dagger}_{i}|0\rangle=\prod_{i}\left(a^{\dagger}_{i}+\sum_{a>F}C_{ai}a_{a}^{\dagger}\right)|0\rangle,
\]
meaning that the new representation of the Slater determinant in second quantization, $|c'\rangle$, looks like our previous ones. However, this representation is not general enough since we have a restriction on the sum over single-particle states in Eq.~(\ref{eq:newb}). The single-particle states have all to be above the Fermi level.
The question then is whether we can construct a general representation of a Slater determinant with a creation operator 
\[
\tilde{b}^{\dagger}_{i}=\sum_{p}f_{ip}a_{p}^{\dagger},
\]
where $f_{ip}$ is a matrix element of a unitary matrix which transforms our creation and annihilation operators
$a^{\dagger}$ and $a$ to $\tilde{b}^{\dagger}$ and $\tilde{b}$. These new operators define a new representation of a Slater determinant as
\[
|\tilde{c}\rangle=\prod_{i}\tilde{b}^{\dagger}_{i}|0\rangle.
\]

\subsection*{Showing that $|\tilde{c}\rangle= |c'\rangle$}

We need to show that $|\tilde{c}\rangle= |c'\rangle$. We need also to assume that the new state
is not orthogonal to $|c\rangle$, that is $\langle c| \tilde{c}\rangle \ne 0$. From this it follows that 
\[
\langle c| \tilde{c}\rangle=\langle 0| a_{i_n}\dots a_{i_1}\left(\sum_{p=i_1}^{i_n}f_{i_1p}a_{p}^{\dagger} \right)\left(\sum_{q=i_1}^{i_n}f_{i_2q}a_{q}^{\dagger} \right)\dots \left(\sum_{t=i_1}^{i_n}f_{i_nt}a_{t}^{\dagger} \right)|0\rangle,
\]
which is nothing but the determinant $det(f_{ip})$ which we can, using the intermediate normalization condition, 
normalize to one, that is
\[
det(f_{ip})=1,
\]
meaning that $f$ has an inverse defined as (since we are dealing with orthogonal, and in our case unitary as well, transformations)
\[
\sum_{k} f_{ik}f^{-1}_{kj} = \delta_{ij},
\]
and 
\[
\sum_{j} f^{-1}_{ij}f_{jk} = \delta_{ik}.
\]

Using these relations we can then define the linear combination of creation (and annihilation as well) 
operators as
\[
\sum_{i}f^{-1}_{ki}\tilde{b}^{\dagger}_{i}=\sum_{i}f^{-1}_{ki}\sum_{p=i_1}^{\infty}f_{ip}a_{p}^{\dagger}=a_{k}^{\dagger}+\sum_{i}\sum_{p=i_{n+1}}^{\infty}f^{-1}_{ki}f_{ip}a_{p}^{\dagger}.
\]
Defining 
\[
c_{kp}=\sum_{i \le F}f^{-1}_{ki}f_{ip},
\]
we can redefine 
\[
a_{k}^{\dagger}+\sum_{i}\sum_{p=i_{n+1}}^{\infty}f^{-1}_{ki}f_{ip}a_{p}^{\dagger}=a_{k}^{\dagger}+\sum_{p=i_{n+1}}^{\infty}c_{kp}a_{p}^{\dagger}=b_k^{\dagger},
\]
our starting point. We have shown that our general representation of a Slater determinant 
\[
|\tilde{c}\rangle=\prod_{i}\tilde{b}^{\dagger}_{i}|0\rangle=|c'\rangle=\prod_{i}b^{\dagger}_{i}|0\rangle,
\]
with 
\[
b_k^{\dagger}=a_{k}^{\dagger}+\sum_{p=i_{n+1}}^{\infty}c_{kp}a_{p}^{\dagger}.
\]

This means that we can actually write an ansatz for the ground state of the system as a linear combination of
terms which contain the ansatz itself $|c\rangle$ with  an admixture from an infinity of one-particle-one-hole states. The latter has important consequences when we wish to interpret the Hartree-Fock equations and their stability. We can rewrite the new representation as 
\[
|c'\rangle = |c\rangle+|\delta c\rangle,
\]
where $|\delta c\rangle$ can now be interpreted as a small variation. If we approximate this term with 
contributions from one-particle-one-hole (\emph{1p-1h}) states only, we arrive at 
\[
|c'\rangle = \left(1+\sum_{ai}\delta C_{ai}a_{a}^{\dagger}a_i\right)|c\rangle.
\]
In our derivation of the Hartree-Fock equations we have shown that 
\[
\langle \delta c| \hat{H} | c\rangle =0,
\]
which means that we have to satisfy
\[
\langle c|\sum_{ai}\delta C_{ai}\left\{a_{a}^{\dagger}a_i\right\} \hat{H} | c\rangle =0.
\]
With this as a background, we are now ready to study the stability of the Hartree-Fock equations.

\subsection*{Hartree-Fock in second quantization and stability of HF solution}

The variational condition for deriving the Hartree-Fock equations guarantees only that the expectation value $\langle c | \hat{H} | c \rangle$ has an extreme value, not necessarily a minimum. To figure out whether the extreme value we have found  is a minimum, we can use second quantization to analyze our results and find a criterion 
for the above expectation value to a local minimum. We will use Thouless' theorem and show that
\[
\frac{\langle c' |\hat{H} | c'\rangle}{\langle c' |c'\rangle} \ge \langle c |\hat{H} | c\rangle= E_0,
\]
with
\[
 {|c'\rangle} = {|c\rangle + |\delta c\rangle}.
\]
Using Thouless' theorem we can write out $|c'\rangle$ as
\begin{align}
 {|c'\rangle}&=\exp\left\{\sum_{a > F}\sum_{i \le F}\delta C_{ai}a_{a}^{\dagger}a_{i}\right\}| c\rangle\\ 
&=\left\{1+\sum_{a > F}\sum_{i \le F}\delta C_{ai}a_{a}^{\dagger}
a_{i}+\frac{1}{2!}\sum_{ab > F}\sum_{ij \le F}\delta C_{ai}\delta C_{bj}a_{a}^{\dagger}a_{i}a_{b}^{\dagger}a_{j}+\dots\right\}
\end{align}
where the amplitudes $\delta C$ are small.

The norm of $|c'\rangle$ is given by (using the intermediate normalization condition $\langle c' |c\rangle=1$) 
\[
\langle c' | c'\rangle = 1+\sum_{a>F}
\sum_{i\le F}|\delta C_{ai}|^2+O(\delta C_{ai}^3).
\]
The expectation value for the energy is now given by (using the Hartree-Fock condition)
\[
\langle c' |\hat{H} | c'\rangle=\langle c |\hat{H} | c\rangle +
\sum_{ab>F}
\sum_{ij\le F}\delta C_{ai}^*\delta C_{bj}\langle c |a_{i}^{\dagger}a_{a}\hat{H}a_{b}^{\dagger}a_{j}|c\rangle+
\]
\[
\frac{1}{2!}\sum_{ab>F}
\sum_{ij\le F}\delta C_{ai}\delta C_{bj}\langle c |\hat{H}a_{a}^{\dagger}a_{i}a_{b}^{\dagger}a_{j}|c\rangle+\frac{1}{2!}\sum_{ab>F}
\sum_{ij\le F}\delta C_{ai}^*\delta C_{bj}^*\langle c|a_{j}^{\dagger}a_{b}a_{i}^{\dagger}a_{a}\hat{H}|c\rangle
+\dots
\] 

We have already calculated the second term on the right-hand side of the previous equation
\begin{align}
\langle c | \left(\{a^\dagger_i a_a\} \hat{H} \{a^\dagger_b a_j\} \right) | c\rangle&=\sum_{pq} \sum_{ijab}\delta C_{ai}^*\delta C_{bj} \langle p|\hat{h}_0 |q\rangle 
            \langle c | \left(\{a^{\dagger}_i a_a\}\{a^{\dagger}_pa_q\} 
             \{a^{\dagger}_b a_j\} \right)| c\rangle\\
& +\frac{1}{4} \sum_{pqrs} \sum_{ijab}\delta C_{ai}^*\delta C_{bj} \langle pq| \hat{v}|rs\rangle 
            \langle c | \left(\{a^\dagger_i a_a\}\{a^{\dagger}_p a^{\dagger}_q a_s  a_r\} \{a^{\dagger}_b a_j\} \right)| c\rangle ,
\end{align}
resulting in
\[
E_0\sum_{ai}|\delta C_{ai}|^2+\sum_{ai}|\delta C_{ai}|^2(\varepsilon_a-\varepsilon_i)-\sum_{ijab} \langle aj|\hat{v}| bi\rangle \delta C_{ai}^*\delta C_{bj}.
\]

\[
\frac{1}{2!}\langle c |\left(\{a^\dagger_j a_b\} \{a^\dagger_i a_a\} \hat{V}_N  \right) | c\rangle  = 
\frac{1}{2!}\langle c |\left( \hat{V}_N \{a^\dagger_a a_i\} \{a^\dagger_b a_j\} \right)^{\dagger} | c\rangle 
\]
which is nothing but
\[
\frac{1}{2!}\langle c |  \left( \hat{V}_N \{a^\dagger_a a_i\} \{a^\dagger_b a_j\} \right) | c\rangle^*
=\frac{1}{2} \sum_{ijab} (\langle ij|\hat{v}|ab\rangle)^*\delta C_{ai}^*\delta C_{bj}^*
\]
or 
\[
\frac{1}{2} \sum_{ijab} (\langle ab|\hat{v}|ij\rangle)\delta C_{ai}^*\delta C_{bj}^*
\]
where we have used the relation
\[ 
\langle a |\hat{A} | b\rangle =  (\langle b |\hat{A}^{\dagger} | a\rangle)^*
\]
due to the hermiticity of $\hat{H}$ and $\hat{V}$.

We define two matrix elements
\[
A_{ai,bj}=-\langle aj|\hat{v} bi\rangle
\]
and
\[
B_{ai,bj}=\langle ab|\hat{v}|ij\rangle
\]
both being anti-symmetrized.

With these definitions we write out the energy as
\begin{align}
\langle c'|H|c'\rangle& = \left(1+\sum_{ai}|\delta C_{ai}|^2\right)\langle c |H|c\rangle+\sum_{ai}|\delta C_{ai}|^2(\varepsilon_a^{HF}-\varepsilon_i^{HF})+\sum_{ijab}A_{ai,bj}\delta C_{ai}^*\delta C_{bj}+\\
&\frac{1}{2} \sum_{ijab} B_{ai,bj}^*\delta C_{ai}\delta C_{bj}+\frac{1}{2} \sum_{ijab} B_{ai,bj}\delta C_{ai}^*\delta C_{bj}^*
+O(\delta C_{ai}^3),
\end{align}
which can be rewritten as
\[
\langle c'|H|c'\rangle = \left(1+\sum_{ai}|\delta C_{ai}|^2\right)\langle c |H|c\rangle+\Delta E+O(\delta C_{ai}^3),
\]
and skipping higher-order terms we arrived
\[
\frac{\langle c' |\hat{H} | c'\rangle}{\langle c' |c'\rangle} =E_0+\frac{\Delta E}{\left(1+\sum_{ai}|\delta C_{ai}|^2\right)}.
\]

We have defined 
\[
\Delta E = \frac{1}{2} \langle \chi | \hat{M}| \chi \rangle
\]
with the vectors 
\[ 
\chi = \left[ \delta C\hspace{0.2cm} \delta C^*\right]^T
\]
and the matrix 
\[
\hat{M}=\left(\begin{array}{cc} \Delta + A & B \\ B^* & \Delta + A^*\end{array}\right),
\]
with $\Delta_{ai,bj} = (\varepsilon_a-\varepsilon_i)\delta_{ab}\delta_{ij}$.

The condition
\[
\Delta E = \frac{1}{2} \langle \chi | \hat{M}| \chi \rangle \ge 0
\]
for an arbitrary  vector 
\[ 
\chi = \left[ \delta C\hspace{0.2cm} \delta C^*\right]^T
\]
means that all eigenvalues of the matrix have to be larger than or equal zero. 
A necessary (but no sufficient) condition is that the matrix elements (for all $ai$ )
\[
(\varepsilon_a-\varepsilon_i)\delta_{ab}\delta_{ij}+A_{ai,bj} \ge 0.
\]
This equation can be used as a first test of the stability of the Hartree-Fock equation.


% ------------------- end of main content ---------------

\end{document}


 \clearemptydoublepage

\chapter{Many-body perturbation theory}

We assume here that we are only interested in the ground state of the system and 
expand the exact wave function in term of a series of Slater determinants
\[
\vert \Psi_0\rangle = \vert \Phi_0\rangle + \sum_{m=1}^{\infty}C_m\vert \Phi_m\rangle,
\]
where we have assumed that the true ground state is dominated by the 
solution of the unperturbed problem, that is
\[
\hat{H}_0\vert \Phi_0\rangle= W_0\vert \Phi_0\rangle.
\]
The state $\vert \Psi_0\rangle$ is not normalized, rather we have used an intermediate 
normalization $\langle \Phi_0 \vert \Psi_0\rangle=1$ since we have $\langle \Phi_0\vert \Phi_0\rangle=1$. 

The Schroedinger equation is
\[
\hat{H}\vert \Psi_0\rangle = E\vert \Psi_0\rangle,
\]
and multiplying the latter from the left with $\langle \Phi_0\vert $ gives
\[
\langle \Phi_0\vert \hat{H}\vert \Psi_0\rangle = E\langle \Phi_0\vert \Psi_0\rangle=E,
\]
and subtracting from this equation
\[
\langle \Psi_0\vert \hat{H}_0\vert \Phi_0\rangle= W_0\langle \Psi_0\vert \Phi_0\rangle=W_0,
\]
and using the fact that the both operators $\hat{H}$ and $\hat{H}_0$ are hermitian 
results in
\[
\Delta E=E-W_0=\langle \Phi_0\vert \hat{H}_I\vert \Psi_0\rangle,
\]
which is an exact result. We call this quantity the correlation energy.

This equation forms the starting point for all perturbative derivations. However,
as it stands it represents nothing but a mere formal rewriting of Schroedinger's equation and is not of much practical use. The exact wave function $\vert \Psi_0\rangle$ is unknown. In order to obtain a perturbative expansion, we need to expand the exact wave function in terms of the interaction $\hat{H}_I$. 

Here we have assumed that our model space defined by the operator $\hat{P}$ is one-dimensional, meaning that
\[
\hat{P}= \vert \Phi_0\rangle \langle \Phi_0\vert ,
\]
and
\[
\hat{Q}=\sum_{m=1}^{\infty}\vert \Phi_m\rangle \langle \Phi_m\vert .
\]

We can thus rewrite the exact wave function as
\[
\vert \Psi_0\rangle= (\hat{P}+\hat{Q})\vert \Psi_0\rangle=\vert \Phi_0\rangle+\hat{Q}\vert \Psi_0\rangle.
\]
Going back to the Schr\"odinger equation, we can rewrite it as, adding and a subtracting a term $\omega \vert \Psi_0\rangle$ as
\[
\left(\omega-\hat{H}_0\right)\vert \Psi_0\rangle=\left(\omega-E+\hat{H}_I\right)\vert \Psi_0\rangle,
\]
where $\omega$ is an energy variable to be specified later. 

We assume also that the resolvent of $\left(\omega-\hat{H}_0\right)$ exits, that is
it has an inverse which defined the unperturbed Green's function as
\[
\left(\omega-\hat{H}_0\right)^{-1}=\frac{1}{\left(\omega-\hat{H}_0\right)}.
\]

We can rewrite Schroedinger's equation as
\[
\vert \Psi_0\rangle=\frac{1}{\omega-\hat{H}_0}\left(\omega-E+\hat{H}_I\right)\vert \Psi_0\rangle,
\]
and multiplying from the left with $\hat{Q}$ results in
\[
\hat{Q}\vert \Psi_0\rangle=\frac{\hat{Q}}{\omega-\hat{H}_0}\left(\omega-E+\hat{H}_I\right)\vert \Psi_0\rangle,
\]
which is possible since we have defined the operator $\hat{Q}$ in terms of the eigenfunctions of $\hat{H}$.

These operators commute meaning that
\[
\hat{Q}\frac{1}{\left(\omega-\hat{H}_0\right)}\hat{Q}=\hat{Q}\frac{1}{\left(\omega-\hat{H}_0\right)}=\frac{\hat{Q}}{\left(\omega-\hat{H}_0\right)}.
\]
With these definitions we can in turn define the wave function as 
\[
\vert \Psi_0\rangle=\vert \Phi_0\rangle+\frac{\hat{Q}}{\omega-\hat{H}_0}\left(\omega-E+\hat{H}_I\right)\vert \Psi_0\rangle.
\]
This equation is again nothing but a formal rewrite of Schr\"odinger's equation
and does not represent a practical calculational scheme.  
It is a non-linear equation in two unknown quantities, the energy $E$ and the exact
wave function $\vert \Psi_0\rangle$. We can however start with a guess for $\vert \Psi_0\rangle$ on the right hand side of the last equation.

 The most common choice is to start with the function which is expected to exhibit the largest overlap with the wave function we are searching after, namely $\vert \Phi_0\rangle$. This can again be inserted in the solution for $\vert \Psi_0\rangle$ in an iterative fashion and if we continue along these lines we end up with
\[
\vert \Psi_0\rangle=\sum_{i=0}^{\infty}\left\{\frac{\hat{Q}}{\omega-\hat{H}_0}\left(\omega-E+\hat{H}_I\right)\right\}^i\vert \Phi_0\rangle, 
\]
for the wave function and
\[
\Delta E=\sum_{i=0}^{\infty}\langle \Phi_0\vert \hat{H}_I\left\{\frac{\hat{Q}}{\omega-\hat{H}_0}\left(\omega-E+\hat{H}_I\right)\right\}^i\vert \Phi_0\rangle, 
\]
which is now  a perturbative expansion of the exact energy in terms of the interaction
$\hat{H}_I$ and the unperturbed wave function $\vert \Psi_0\rangle$.

In our equations for $\vert \Psi_0\rangle$ and $\Delta E$ in terms of the unperturbed
solutions $\vert \Phi_i\rangle$  we have still an undetermined parameter $\omega$
and a dependecy on the exact energy $E$. Not much has been gained thus from a practical computational point of view. 

In Brilluoin-Wigner perturbation theory it is customary to set $\omega=E$. This results in the following perturbative expansion for the energy $\Delta E$
\[
\Delta E=\sum_{i=0}^{\infty}\langle \Phi_0\vert \hat{H}_I\left\{\frac{\hat{Q}}{\omega-\hat{H}_0}\left(\omega-E+\hat{H}_I\right)\right\}^i\vert \Phi_0\rangle=
\]
\[
\langle \Phi_0\vert \left(\hat{H}_I+\hat{H}_I\frac{\hat{Q}}{E-\hat{H}_0}\hat{H}_I+
\hat{H}_I\frac{\hat{Q}}{E-\hat{H}_0}\hat{H}_I\frac{\hat{Q}}{E-\hat{H}_0}\hat{H}_I+\dots\right)\vert \Phi_0\rangle. 
\]

\[
\Delta E=\sum_{i=0}^{\infty}\langle \Phi_0\vert \hat{H}_I\left\{\frac{\hat{Q}}{\omega-\hat{H}_0}\left(\omega-E+\hat{H}_I\right)\right\}^i\vert \Phi_0\rangle=\]
\[
\langle \Phi_0\vert \left(\hat{H}_I+\hat{H}_I\frac{\hat{Q}}{E-\hat{H}_0}\hat{H}_I+
\hat{H}_I\frac{\hat{Q}}{E-\hat{H}_0}\hat{H}_I\frac{\hat{Q}}{E-\hat{H}_0}\hat{H}_I+\dots\right)\vert \Phi_0\rangle. 
\]
This expression depends however on the exact energy $E$ and is again not very convenient from a practical point of view. It can obviously be solved iteratively, by starting with a guess for  $E$ and then solve till some kind of self-consistency criterion has been reached. 

Actually, the above expression is nothing but a rewrite again of the full Schr\"odinger equation. 

Defining $e=E-\hat{H}_0$ and recalling that $\hat{H}_0$ commutes with 
$\hat{Q}$ by construction and that $\hat{Q}$ is an idempotent operator
$\hat{Q}^2=\hat{Q}$. 
Using this equation in the above expansion for $\Delta E$ we can write the denominator 
\[
\hat{Q}\frac{1}{\hat{e}-\hat{Q}\hat{H}_I\hat{Q}}=
\]
\[
\hat{Q}\left[\frac{1}{\hat{e}}+\frac{1}{\hat{e}}\hat{Q}\hat{H}_I\hat{Q}
\frac{1}{\hat{e}}+\frac{1}{\hat{e}}\hat{Q}\hat{H}_I\hat{Q}
\frac{1}{\hat{e}}\hat{Q}\hat{H}_I\hat{Q}\frac{1}{\hat{e}}+\dots\right]\hat{Q}.
\]

Inserted in the expression for $\Delta E$ leads to 
\[
\Delta E=
\langle \Phi_0\vert \hat{H}_I+\hat{H}_I\hat{Q}\frac{1}{E-\hat{H}_0-\hat{Q}\hat{H}_I\hat{Q}}\hat{Q}\hat{H}_I\vert \Phi_0\rangle. 
\]
In RS perturbation theory we set $\omega = W_0$ and obtain the following expression for the energy difference
\[
\Delta E=\sum_{i=0}^{\infty}\langle \Phi_0\vert \hat{H}_I\left\{\frac{\hat{Q}}{W_0-\hat{H}_0}\left(\hat{H}_I-\Delta E\right)\right\}^i\vert \Phi_0\rangle=
\]
\[
\langle \Phi_0\vert \left(\hat{H}_I+\hat{H}_I\frac{\hat{Q}}{W_0-\hat{H}_0}(\hat{H}_I-\Delta E)+
\hat{H}_I\frac{\hat{Q}}{W_0-\hat{H}_0}(\hat{H}_I-\Delta E)\frac{\hat{Q}}{W_0-\hat{H}_0}(\hat{H}_I-\Delta E)+\dots\right)\vert \Phi_0\rangle.
\]

Recalling that $\hat{Q}$ commutes with $\hat{H_0}$ and since $\Delta E$ is a constant we obtain that
\[
\hat{Q}\Delta E\vert \Phi_0\rangle = \hat{Q}\Delta E\vert \hat{Q}\Phi_0\rangle = 0.
\]
Inserting this results in the expression for the energy results in
\[
\Delta E=\langle \Phi_0\vert \left(\hat{H}_I+\hat{H}_I\frac{\hat{Q}}{W_0-\hat{H}_0}\hat{H}_I+
\hat{H}_I\frac{\hat{Q}}{W_0-\hat{H}_0}(\hat{H}_I-\Delta E)\frac{\hat{Q}}{W_0-\hat{H}_0}\hat{H}_I+\dots\right)\vert \Phi_0\rangle.
\]

We can now this expression in terms of a perturbative expression in terms
of $\hat{H}_I$ where we iterate the last expression in terms of $\Delta E$
\[
\Delta E=\sum_{i=1}^{\infty}\Delta E^{(i)}.
\]
We get the following expression for $\Delta E^{(i)}$
\[
\Delta E^{(1)}=\langle \Phi_0\vert \hat{H}_I\vert \Phi_0\rangle,
\] 
which is just the contribution to first order in perturbation theory,
\[
\Delta E^{(2)}=\langle\Phi_0\vert \hat{H}_I\frac{\hat{Q}}{W_0-\hat{H}_0}\hat{H}_I\vert \Phi_0\rangle, 
\]
which is the contribution to second order.

\[
\Delta E^{(3)}=\langle \Phi_0\vert \hat{H}_I\frac{\hat{Q}}{W_0-\hat{H}_0}\hat{H}_I\frac{\hat{Q}}{W_0-\hat{H}_0}\hat{H}_I\Phi_0\rangle-
\langle\Phi_0\vert \hat{H}_I\frac{\hat{Q}}{W_0-\hat{H}_0}\langle \Phi_0\vert \hat{H}_I\vert \Phi_0\rangle\frac{\hat{Q}}{W_0-\hat{H}_0}\hat{H}_I\vert \Phi_0\rangle,
\]
being the third-order contribution. 

\subsection*{Interpreting the correlation energy and the wave operator}

In the shell-model lectures we showed that we could rewrite the exact state function for say the ground state, as a linear expansion in terms of all possible Slater determinants. That is, we 
define the ansatz for the ground state as 
\[
|\Phi_0\rangle = \left(\prod_{i\le F}\hat{a}_{i}^{\dagger}\right)|0\rangle,
\]
where the index $i$ defines different single-particle states up to the Fermi level. We have assumed that we have $N$ fermions. 
A given one-particle-one-hole ($1p1h$) state can be written as
\[
|\Phi_i^a\rangle = \hat{a}_{a}^{\dagger}\hat{a}_i|\Phi_0\rangle,
\]
while a $2p2h$ state can be written as
\[
|\Phi_{ij}^{ab}\rangle = \hat{a}_{a}^{\dagger}\hat{a}_{b}^{\dagger}\hat{a}_j\hat{a}_i|\Phi_0\rangle,
\]
and a general $ApAh$ state as 
\[
|\Phi_{ijk\dots}^{abc\dots}\rangle = \hat{a}_{a}^{\dagger}\hat{a}_{b}^{\dagger}\hat{a}_{c}^{\dagger}\dots\hat{a}_k\hat{a}_j\hat{a}_i|\Phi_0\rangle.
\]

We use letters $ijkl\dots$ for states below the Fermi level and $abcd\dots$ for states above the Fermi level. A general single-particle state is given by letters $pqrs\dots$.

We can then expand our exact state function for the ground state 
as
\[
|\Psi_0\rangle=C_0|\Phi_0\rangle+\sum_{ai}C_i^a|\Phi_i^a\rangle+\sum_{abij}C_{ij}^{ab}|\Phi_{ij}^{ab}\rangle+\dots
=(C_0+\hat{C})|\Phi_0\rangle,
\]
where we have introduced the so-called correlation operator 
\[
\hat{C}=\sum_{ai}C_i^a\hat{a}_{a}^{\dagger}\hat{a}_i  +\sum_{abij}C_{ij}^{ab}\hat{a}_{a}^{\dagger}\hat{a}_{b}^{\dagger}\hat{a}_j\hat{a}_i+\dots
\]
Since the normalization of $\Psi_0$ is at our disposal and since $C_0$ is by hypothesis non-zero, we may arbitrarily set $C_0=1$ with 
corresponding proportional changes in all other coefficients. Using this so-called intermediate normalization we have
\[
\langle \Psi_0 | \Phi_0 \rangle = \langle \Phi_0 | \Phi_0 \rangle = 1, 
\]
resulting in 
\[
|\Psi_0\rangle=(1+\hat{C})|\Phi_0\rangle.
\]

In a shell-model calculation, the unknown coefficients in $\hat{C}$ are the 
eigenvectors which result from the diagonalization of the Hamiltonian matrix.

How can we use perturbation theory to determine the same coefficients? Let us study the contributions to second order in the interaction, namely
\[
\Delta E^{(2)}=\langle\Phi_0\vert \hat{H}_I\frac{\hat{Q}}{W_0-\hat{H}_0}\hat{H}_I\vert \Phi_0\rangle.
\]

The intermediate states given by $\hat{Q}$ can at most be of a $2p-2h$ nature if we have a two-body Hamiltonian. This means that second order in the perturbation theory can have $1p-1h$ and $2p-2h$ at most as intermediate states. When we diagonalize, these contributions are included to infinite order. This means that higher-orders in perturbation theory bring in more complicated correlations. 

If we limit the attention to a Hartree-Fock basis, then we have that
$\langle\Phi_0\vert \hat{H}_I \vert 2p-2h\rangle$ is the only contribution and the contribution to the energy reduces to
\[
\Delta E^{(2)}=\frac{1}{4}\sum_{abij}\langle ij\vert \hat{v}\vert ab\rangle \frac{\langle ab\vert \hat{v}\vert ij\rangle}{\epsilon_i+\epsilon_j-\epsilon_a-\epsilon_b}.
\]

If we compare this to the correlation energy obtained from full configuration interaction theory with a Hartree-Fock basis, we found that
\[
E-E_0 =\Delta E=
\sum_{abij}\langle ij | \hat{v}| ab \rangle C_{ij}^{ab},
\]
where the energy $E_0$ is the reference energy and $\Delta E$ defines the so-called correlation energy.

We see that if we set
\[
C_{ij}^{ab} =\frac{1}{4}\frac{\langle ab \vert \hat{v} \vert ij \rangle}{\epsilon_i+\epsilon_j-\epsilon_a-\epsilon_b},
\]
we have a perfect agreement between FCI and MBPT. However, FCI includes such $2p-2h$ correlations to infinite order. In order to make a meaningful comparison we would at least need to sum such correlations to infinite order in perturbation theory. 

Summing up, we can see that
\begin{itemize}
\item MBPT introduces order-by-order specific correlations and we make comparisons with exact calculations like FCI

\item At every order, we can calculate all contributions since they are well-known and either tabulated or calculated on the fly.

\item MBPT is a non-variational theory and there is no guarantee that higher orders will improve the convergence. 

\item However, since FCI calculations are limited by the size of the Hamiltonian matrices to diagonalize (today's most efficient codes can attach dimensionalities of ten billion basis states, MBPT can function as an approximative method which gives a straightforward (but tedious) calculation recipe. 

\item MBPT has been widely used to compute effective interactions for the nuclear shell-model.

\item But there are better methods which sum to infinite order important correlations. Coupled cluster theory is one of these methods. 
\end{itemize}


 \part{Monte Carlo Methods}
 
\chapter{Variational Monte Carlo methods}

\subsection*{Quantum Monte Carlo Motivation}

We start with the variational principle.
Given a hamiltonian $H$ and a trial wave function $\Psi_T$, the variational principle states that the expectation value of $\langle H \rangle$, defined through 
\[
   E[H]= \langle H \rangle =
   \frac{\int d\bm{R}\Psi^{\ast}_T(\bm{R})H(\bm{R})\Psi_T(\bm{R})}
        {\int d\bm{R}\Psi^{\ast}_T(\bm{R})\Psi_T(\bm{R})},
\]
is an upper bound to the ground state energy $E_0$ of the hamiltonian $H$, that is 
\[
    E_0 \le \langle H \rangle .
\]
In general, the integrals involved in the calculation of various  expectation values  are multi-dimensional ones. Traditional integration methods such as the Gauss-Legendre will not be adequate for say the  computation of the energy of a many-body system.

The trial wave function can be expanded in the eigenstates of the hamiltonian since they form a complete set, viz.,
\[
   \Psi_T(\bm{R})=\sum_i a_i\Psi_i(\bm{R}),
\]
and assuming the set of eigenfunctions to be normalized one obtains 
\[
     \frac{\sum_{nm}a^*_ma_n \int d\bm{R}\Psi^{\ast}_m(\bm{R})H(\bm{R})\Psi_n(\bm{R})}
        {\sum_{nm}a^*_ma_n \int d\bm{R}\Psi^{\ast}_m(\bm{R})\Psi_n(\bm{R})} =\frac{\sum_{n}a^2_n E_n}
        {\sum_{n}a^2_n} \ge E_0,
\]
where we used that $H(\bm{R})\Psi_n(\bm{R})=E_n\Psi_n(\bm{R})$.
In general, the integrals involved in the calculation of various  expectation
values  are multi-dimensional ones. 
The variational principle yields the lowest state of a given symmetry.

In most cases, a wave function has only small values in large parts of 
configuration space, and a straightforward procedure which uses
homogenously distributed random points in configuration space 
will most likely lead to poor results. This may suggest that some kind
of importance sampling combined with e.g., the Metropolis algorithm 
may be  a more efficient way of obtaining the ground state energy.
The hope is then that those regions of configurations space where
the wave function assumes appreciable values are sampled more 
efficiently. 

The tedious part in a VMC calculation is the search for the variational
minimum. A good knowledge of the system is required in order to carry out
reasonable VMC calculations. This is not always the case, 
and often VMC calculations 
serve rather as the starting
point for so-called diffusion Monte Carlo calculations (DMC). DMC is a way of
solving exactly the many-body Schroedinger equation by means of 
a stochastic procedure. A good guess on the binding energy
and its wave function is however necessary. 
A carefully performed VMC calculation can aid in this context. 

The basic recipe in a VMC calculation consists of the following elements:

\begin{itemize}
\item Construct first a trial wave function $\psi_T(\bm{R},\bm{\alpha})$,  for a many-body system consisting of $N$ particles located at positions  $\bm{R}=(\bm{R}_1,\dots ,\bm{R}_N)$. The trial wave function depends on $\alpha$ variational parameters $\bm{\alpha}=(\alpha_1,\dots ,\alpha_M)$.

\item Then we evaluate the expectation value of the hamiltonian $H$ 
\end{itemize}

\noindent
\[
   E[H]=\langle H \rangle =
   \frac{\int d\bm{R}\Psi^{\ast}_{T}(\bm{R},\bm{\alpha})H(\bm{R})\Psi_{T}(\bm{R},\bm{\alpha})}
        {\int d\bm{R}\Psi^{\ast}_{T}(\bm{R},\bm{\alpha})\Psi_{T}(\bm{R},\bm{\alpha})}.
\]
\begin{itemize}
\item Thereafter we vary $\alpha$ according to some minimization algorithm and return to the first step.
\end{itemize}

\noindent
With a trial wave function $\psi_T(\bm{R})$ we can in turn construct the quantum mechanical probability distribution
\[
   P(\bm{R})= \frac{\left|\psi_T(\bm{R})\right|^2}{\int \left|\psi_T(\bm{R})\right|^2d\bm{R}}.
\]
This is our new probability distribution function  (PDF).
The approximation to the expectation value of the Hamiltonian is now 
\[
   E[H(\bm{\alpha})] = 
   \frac{\int d\bm{R}\Psi^{\ast}_T(\bm{R},\bm{\alpha})H(\bm{R})\Psi_T(\bm{R},\bm{\alpha})}
        {\int d\bm{R}\Psi^{\ast}_T(\bm{R},\bm{\alpha})\Psi_T(\bm{R},\bm{\alpha})}.
\]

Define a new quantity
\[
   E_L(\bm{R},\bm{\alpha})=\frac{1}{\psi_T(\bm{R},\bm{\alpha})}H\psi_T(\bm{R},\bm{\alpha}),
   \label{eq:locale1}
\]
called the local energy, which, together with our trial PDF yields
\[
  E[H(\bm{\alpha})]=\int P(\bm{R})E_L(\bm{R}) d\bm{R}\approx \frac{1}{N}\sum_{i=1}^NP(\bm{R_i},\bm{\alpha})E_L(\bm{R_i},\bm{\alpha})
  \label{eq:vmc1}
\]
with $N$ being the number of Monte Carlo samples.

The Algorithm for performing a variational Monte Carlo calculations runs thus as this

\begin{itemize}
   \item Initialisation: Fix the number of Monte Carlo steps. Choose an initial $\bm{R}$ and variational parameters $\alpha$ and calculate $\left|\psi_T^{\alpha}(\bm{R})\right|^2$. 

   \item Initialise the energy and the variance and start the Monte Carlo calculation.
\begin{itemize}

      \item Calculate  a trial position  $\bm{R}_p=\bm{R}+r*step$ where $r$ is a random variable $r \in [0,1]$.

      \item Metropolis algorithm to accept or reject this move  $w = P(\bm{R}_p)/P(\bm{R})$.

      \item If the step is accepted, then we set $\bm{R}=\bm{R}_p$. 

      \item Update averages

\end{itemize}

\noindent
   \item Finish and compute final averages.
\end{itemize}

\noindent
Observe that the jumping in space is governed by the variable \emph{step}. This is Called brute-force sampling.
Need importance sampling to get more relevant sampling, see lectures below.

\paragraph{Quantum Monte Carlo: hydrogen atom.}
The radial Schroedinger equation for the hydrogen atom can be
written as
\[
-\frac{\hbar^2}{2m}\frac{\partial^2 u(r)}{\partial r^2}-
\left(\frac{ke^2}{r}-\frac{\hbar^2l(l+1)}{2mr^2}\right)u(r)=Eu(r),
\]
or with dimensionless variables
\[
-\frac{1}{2}\frac{\partial^2 u(\rho)}{\partial \rho^2}-
\frac{u(\rho)}{\rho}+\frac{l(l+1)}{2\rho^2}u(\rho)-\lambda u(\rho)=0,
\label{eq:hydrodimless1}
\]
with the hamiltonian
\[
H=-\frac{1}{2}\frac{\partial^2 }{\partial \rho^2}-
\frac{1}{\rho}+\frac{l(l+1)}{2\rho^2}.
\]
Use variational parameter $\alpha$ in the trial
wave function 
\[
   u_T^{\alpha}(\rho)=\alpha\rho e^{-\alpha\rho}. 
   \label{eq:trialhydrogen}
\]

Inserting this wave function into the expression for the
local energy $E_L$ gives
\[
   E_L(\rho)=-\frac{1}{\rho}-
              \frac{\alpha}{2}\left(\alpha-\frac{2}{\rho}\right).
\]
A simple variational Monte Carlo calculation results in

\begin{quote}
\begin{tabular}{cccc}
\hline
\multicolumn{1}{c}{ $\alpha$ } & \multicolumn{1}{c}{ $\langle H \rangle $ } & \multicolumn{1}{c}{ $\sigma^2$ } & \multicolumn{1}{c}{ $\sigma/\sqrt{N}$ } \\
\hline
7.00000E-01 & -4.57759E-01         & 4.51201E-02 & 6.71715E-04       \\
8.00000E-01 & -4.81461E-01         & 3.05736E-02 & 5.52934E-04       \\
9.00000E-01 & -4.95899E-01         & 8.20497E-03 & 2.86443E-04       \\
1.00000E-00 & -5.00000E-01         & 0.00000E+00 & 0.00000E+00       \\
1.10000E+00 & -4.93738E-01         & 1.16989E-02 & 3.42036E-04       \\
1.20000E+00 & -4.75563E-01         & 8.85899E-02 & 9.41222E-04       \\
1.30000E+00 & -4.54341E-01         & 1.45171E-01 & 1.20487E-03       \\
\hline
\end{tabular}
\end{quote}

\noindent
We note that at $\alpha=1$ we obtain the exact
result, and the variance is zero, as it should. The reason is that 
we then have the exact wave function, and the action of the hamiltionan
on the wave function
\[
   H\psi = \mathrm{constant}\times \psi,
\]
yields just a constant. The integral which defines various 
expectation values involving moments of the hamiltonian becomes then
\[
   \langle H^n \rangle =
   \frac{\int d\bm{R}\Psi^{\ast}_T(\bm{R})H^n(\bm{R})\Psi_T(\bm{R})}
        {\int d\bm{R}\Psi^{\ast}_T(\bm{R})\Psi_T(\bm{R})}=
\mathrm{constant}\times\frac{\int d\bm{R}\Psi^{\ast}_T(\bm{R})\Psi_T(\bm{R})}
        {\int d\bm{R}\Psi^{\ast}_T(\bm{R})\Psi_T(\bm{R})}=\mathrm{constant}.
\]
\textbf{This gives an important information: the exact wave function leads to zero variance!}
Variation is then performed by minimizing both the energy and the variance.

For bosons in a harmonic oscillator-like  trap we will use is a spherical (S)
 or an elliptical (E) harmonic trap in one, two and finally three
 dimensions, with the latter given by
 \begin{equation}
 V_{ext}(\mathbf{r}) = \Bigg\{
 \begin{array}{ll}
	 \frac{1}{2}m\omega_{ho}^2r^2 & (S)\\
 \strut
	 \frac{1}{2}m[\omega_{ho}^2(x^2+y^2) + \omega_z^2z^2] & (E)
 \label{trap_eqn}
 \end{array}
 \end{equation}
where (S) stands for symmetric and 
\begin{equation}
     \hat{H} = \sum_i^N \left(
	 \frac{-\hbar^2}{2m}
	 { \bigtriangledown }_{i}^2 +
	 V_{ext}({\bf{r}}_i)\right)  +
	 \sum_{i<j}^{N} V_{int}({\bf{r}}_i,{\bf{r}}_j),
\end{equation}
as the two-body Hamiltonian of the system.  

 We will represent the inter-boson interaction by a pairwise, repulsive potential
\begin{equation}
 V_{int}(|\mathbf{r}_i-\mathbf{r}_j|) =  \Bigg\{
 \begin{array}{ll}
	 \infty & {|\mathbf{r}_i-\mathbf{r}_j|} \leq {a}\\
	 0 & {|\mathbf{r}_i-\mathbf{r}_j|} > {a}
 \end{array}
 \end{equation}
 where $a$ is the so-called hard-core diameter of the bosons.
 Clearly, $V_{int}(|\mathbf{r}_i-\mathbf{r}_j|)$ is zero if the bosons are
 separated by a distance $|\mathbf{r}_i-\mathbf{r}_j|$ greater than $a$ but
 infinite if they attempt to come within a distance $|\mathbf{r}_i-\mathbf{r}_j| \leq a$.

 Our trial wave function for the ground state with $N$ atoms is given by
 \begin{equation}
 \Psi_T(\mathbf{R})=\Psi_T(\mathbf{r}_1, \mathbf{r}_2, \dots \mathbf{r}_N,\alpha,\beta)=\prod_i g(\alpha,\beta,\mathbf{r}_i)\prod_{i<j}f(a,|\mathbf{r}_i-\mathbf{r}_j|),
 \label{eq:trialwf}
 \end{equation}
 where $\alpha$ and $\beta$ are variational parameters. The
 single-particle wave function is proportional to the harmonic
 oscillator function for the ground state
\begin{equation}
    g(\alpha,\beta,\mathbf{r}_i)= \exp{[-\alpha(x_i^2+y_i^2+\beta z_i^2)]}.
 \end{equation}

For spherical traps we have $\beta = 1$ and for non-interacting
bosons ($a=0$) we have $\alpha = 1/2a_{ho}^2$.  The correlation wave
 function is
 \begin{equation}
    f(a,|\mathbf{r}_i-\mathbf{r}_j|)=\Bigg\{
 \begin{array}{ll}
	 0 & {|\mathbf{r}_i-\mathbf{r}_j|} \leq {a}\\
	 (1-\frac{a}{|\mathbf{r}_i-\mathbf{r}_j|}) & {|\mathbf{r}_i-\mathbf{r}_j|} > {a}.
 \end{array}
 \end{equation}  

\paragraph{A simple Python code that solves the two-boson or two-fermion case in two-dimensions.}













































































































\begin{minted}[fontsize=\fontsize{9pt}{9pt},linenos=false,mathescape,baselinestretch=1.0,fontfamily=tt,xleftmargin=7mm]{python}
# Importing various packages
from math import exp, sqrt
from random import random, seed
import numpy as np
import matplotlib.pyplot as plt
from mpl_toolkits.mplot3d import Axes3D
from matplotlib import cm
from matplotlib.ticker import LinearLocator, FormatStrFormatter
import sys

#Trial wave function for quantum dots in two dims
def WaveFunction(r,alpha,beta):
    r1 = r[0,0]**2 + r[0,1]**2
    r2 = r[1,0]**2 + r[1,1]**2
    r12 = sqrt((r[0,0]-r[1,0])**2 + (r[0,1]-r[1,1])**2)
    deno = r12/(1+beta*r12)
    return exp(-0.5*alpha*(r1+r2)+deno)

#Local energy  for quantum dots in two dims, using analytical local energy
def LocalEnergy(r,alpha,beta):
    
    r1 = (r[0,0]**2 + r[0,1]**2)
    r2 = (r[1,0]**2 + r[1,1]**2)
    r12 = sqrt((r[0,0]-r[1,0])**2 + (r[0,1]-r[1,1])**2)
    deno = 1.0/(1+beta*r12)
    deno2 = deno*deno
    return 0.5*(1-alpha*alpha)*(r1 + r2) +2.0*alpha + 1.0/r12+deno2*(alpha*r12-deno2+2*beta*deno-1.0/r12)

# The Monte Carlo sampling with the Metropolis algo
def MonteCarloSampling():

    NumberMCcycles= 100000
    StepSize = 1.0
    # positions
    PositionOld = np.zeros((NumberParticles,Dimension), np.double)
    PositionNew = np.zeros((NumberParticles,Dimension), np.double)
    # seed for rng generator
    seed()
    # start variational parameter
    alpha = 0.9
    for ia in range(MaxVariations):
        alpha += .025
        AlphaValues[ia] = alpha
        beta = 0.2 
        for jb in range(MaxVariations):
            beta += .01
            BetaValues[jb] = beta
            energy = energy2 = 0.0
            DeltaE = 0.0
            #Initial position
            for i in range(NumberParticles):
                for j in range(Dimension):
                    PositionOld[i,j] = StepSize * (random() - .5)
            wfold = WaveFunction(PositionOld,alpha,beta)

            #Loop over MC MCcycles
            for MCcycle in range(NumberMCcycles):
                #Trial position
                for i in range(NumberParticles):
                    for j in range(Dimension):
                        PositionNew[i,j] = PositionOld[i,j] + StepSize * (random() - .5)
                wfnew = WaveFunction(PositionNew,alpha,beta)

                #Metropolis test to see whether we accept the move
                if random() < wfnew**2 / wfold**2:
                   PositionOld = PositionNew.copy()
                   wfold = wfnew
                   DeltaE = LocalEnergy(PositionOld,alpha,beta)
                energy += DeltaE
                energy2 += DeltaE**2

            #We calculate mean, variance and error ...
            energy /= NumberMCcycles
            energy2 /= NumberMCcycles
            variance = energy2 - energy**2
            error = sqrt(variance/NumberMCcycles)
            Energies[ia,jb] = energy    
    return Energies, AlphaValues, BetaValues


#Here starts the main program with variable declarations
NumberParticles = 2
Dimension = 2
MaxVariations = 10
Energies = np.zeros((MaxVariations,MaxVariations))
AlphaValues = np.zeros(MaxVariations)
BetaValues = np.zeros(MaxVariations)
(Energies, AlphaValues, BetaValues) = MonteCarloSampling()

# Prepare for plots
fig = plt.figure()
ax = fig.gca(projection='3d')
# Plot the surface.
X, Y = np.meshgrid(AlphaValues, BetaValues)
surf = ax.plot_surface(X, Y, Energies,cmap=cm.coolwarm,linewidth=0, antialiased=False)
# Customize the z axis.
zmin = np.matrix(Energies).min()
zmax = np.matrix(Energies).max()
ax.set_zlim(zmin, zmax)
ax.set_xlabel(r'$\alpha$')
ax.set_ylabel(r'$\beta$')
ax.set_zlabel(r'$\langle E \rangle$')
ax.zaxis.set_major_locator(LinearLocator(10))
ax.zaxis.set_major_formatter(FormatStrFormatter('%.02f'))
# Add a color bar which maps values to colors.
fig.colorbar(surf, shrink=0.5, aspect=5)
plt.show()


\end{minted}


\subsection*{Quantum Monte Carlo: the helium atom}

The helium atom consists of two electrons and a nucleus with
charge $Z=2$. 
The contribution  
to the potential energy due to the attraction from the nucleus is
\[
   -\frac{2ke^2}{r_1}-\frac{2ke^2}{r_2},
\] 
and if we add the repulsion arising from the two 
interacting electrons, we obtain the potential energy
\[
 V(r_1, r_2)=-\frac{2ke^2}{r_1}-\frac{2ke^2}{r_2}+
               \frac{ke^2}{r_{12}},
\]
with the electrons separated at a distance 
$r_{12}=|\bm{r}_1-\bm{r}_2|$.

The hamiltonian becomes then
\[
   \hat{H}=-\frac{\hbar^2\nabla_1^2}{2m}-\frac{\hbar^2\nabla_2^2}{2m}
          -\frac{2ke^2}{r_1}-\frac{2ke^2}{r_2}+
               \frac{ke^2}{r_{12}},
\]
and  Schroedingers equation reads
\[
   \hat{H}\psi=E\psi.
\]
All observables are evaluated with respect to the probability distribution
\[
   P(\bm{R})= \frac{\left|\psi_T(\bm{R})\right|^2}{\int \left|\psi_T(\bm{R})\right|^2d\bm{R}}.
\]
generated by the trial wave function.   
The trial wave function must approximate an exact 
eigenstate in order that accurate results are to be obtained. 

Choice of trial wave function for Helium:
Assume $r_1 \rightarrow 0$.
\[
   E_L(\bm{R})=\frac{1}{\psi_T(\bm{R})}H\psi_T(\bm{R})=
     \frac{1}{\psi_T(\bm{R})}\left(-\frac{1}{2}\nabla^2_1
     -\frac{Z}{r_1}\right)\psi_T(\bm{R}) + \mathrm{finite \hspace{0.1cm}terms}.
\]
\[ 
    E_L(R)=
    \frac{1}{\mathbf{R}_T(r_1)}\left(-\frac{1}{2}\frac{d^2}{dr_1^2}-
     \frac{1}{r_1}\frac{d}{dr_1}
     -\frac{Z}{r_1}\right)\mathbf{R}_T(r_1) + \mathrm{finite\hspace{0.1cm} terms}
\]
For small values of $r_1$, the terms which dominate are
\[ 
    \lim_{r_1 \rightarrow 0}E_L(R)=
    \frac{1}{\mathbf{R}_T(r_1)}\left(-
     \frac{1}{r_1}\frac{d}{dr_1}
     -\frac{Z}{r_1}\right)\mathbf{R}_T(r_1),
\]
since the second derivative does not diverge due to the finiteness of  $\Psi$ at the origin.

This results in
\[
     \frac{1}{\mathbf{R}_T(r_1)}\frac{d \mathbf{R}_T(r_1)}{dr_1}=-Z,
\]
and
\[
   \mathbf{R}_T(r_1)\propto e^{-Zr_1}.
\]
A similar condition applies to electron 2 as well. 
For orbital momenta $l > 0$ we have 
\[
     \frac{1}{\mathbf{R}_T(r)}\frac{d \mathbf{R}_T(r)}{dr}=-\frac{Z}{l+1}.
\]
Similarly, studying the case $r_{12}\rightarrow 0$ we can write 
a possible trial wave function as
\[
   \psi_T(\bm{R})=e^{-\alpha(r_1+r_2)}e^{\beta r_{12}}.
    \label{eq:wavehelium2}
\]
The last equation can be generalized to
\[
   \psi_T(\bm{R})=\phi(\bm{r}_1)\phi(\bm{r}_2)\dots\phi(\bm{r}_N)
                   \prod_{i < j}f(r_{ij}),
\]
for a system with $N$ electrons or particles. 

During the development of our code we need to make several checks. It is also very instructive to compute a closed form expression for the local energy. Since our wave function is rather simple  it is straightforward
to find an analytic expressions.  Consider first the case of the simple helium function 
\[
   \Psi_T(\bm{r}_1,\bm{r}_2) = e^{-\alpha(r_1+r_2)}
\]
The local energy is for this case 
\[ 
E_{L1} = \left(\alpha-Z\right)\left(\frac{1}{r_1}+\frac{1}{r_2}\right)+\frac{1}{r_{12}}-\alpha^2
\]
which gives an expectation value for the local energy given by
\[
\langle E_{L1} \rangle = \alpha^2-2\alpha\left(Z-\frac{5}{16}\right)
\]

With closed form formulae we  can speed up the computation of the correlation. In our case
we write it as 
\[
\Psi_C= \exp{\left\{\sum_{i < j}\frac{ar_{ij}}{1+\beta r_{ij}}\right\}},
\]
which means that the gradient needed for the so-called quantum force and local energy 
can be calculated analytically.
This will speed up your code since the computation of the correlation part and the Slater determinant are the most 
time consuming parts in your code.  

We will refer to this correlation function as $\Psi_C$ or the \emph{linear Pade-Jastrow}.

We can test this by computing the local energy for our helium wave function
\[
   \psi_{T}(\bm{r}_1,\bm{r}_2) = 
   \exp{\left(-\alpha(r_1+r_2)\right)}
   \exp{\left(\frac{r_{12}}{2(1+\beta r_{12})}\right)}, 
\]
with $\alpha$ and $\beta$ as variational parameters.

The local energy is for this case 
\[ 
E_{L2} = E_{L1}+\frac{1}{2(1+\beta r_{12})^2}\left\{\frac{\alpha(r_1+r_2)}{r_{12}}(1-\frac{\bm{r}_1\bm{r}_2}{r_1r_2})-\frac{1}{2(1+\beta r_{12})^2}-\frac{2}{r_{12}}+\frac{2\beta}{1+\beta r_{12}}\right\}
\]
It is very useful to test your code against these expressions. It means also that you don't need to
compute a derivative numerically as discussed in the code example below. 

For the computation of various derivatives with different types of wave functions, you will find it useful to use python with symbolic python, that is sympy, see \href{{http://docs.sympy.org/latest/index.html}}{online manual}.  Using sympy allows you autogenerate both Latex code as well c++, python or Fortran codes. Here you will find some simple examples. We choose 
the $2s$ hydrogen-orbital  (not normalized) as an example
\[
 \phi_{2s}(\bm{r}) = (Zr - 2)\exp{-(\frac{1}{2}Zr)},
\]
with $ r^2 = x^2 + y^2 + z^2$.









\begin{minted}[fontsize=\fontsize{9pt}{9pt},linenos=false,mathescape,baselinestretch=1.0,fontfamily=tt,xleftmargin=7mm]{python}
from sympy import symbols, diff, exp, sqrt
x, y, z, Z = symbols('x y z Z')
r = sqrt(x*x + y*y + z*z)
r
phi = (Z*r - 2)*exp(-Z*r/2)
phi
diff(phi, x)

\end{minted}

This doesn't look very nice, but sympy provides several functions that allow for improving and simplifying the output.

We can improve our output by factorizing and substituting expressions









\begin{minted}[fontsize=\fontsize{9pt}{9pt},linenos=false,mathescape,baselinestretch=1.0,fontfamily=tt,xleftmargin=7mm]{python}
from sympy import symbols, diff, exp, sqrt, factor, Symbol, printing
x, y, z, Z = symbols('x y z Z')
r = sqrt(x*x + y*y + z*z)
phi = (Z*r - 2)*exp(-Z*r/2)
R = Symbol('r') #Creates a symbolic equivalent of r
#print latex and c++ code
print printing.latex(diff(phi, x).factor().subs(r, R))
print printing.ccode(diff(phi, x).factor().subs(r, R))

\end{minted}


We can in turn look at second derivatives












\begin{minted}[fontsize=\fontsize{9pt}{9pt},linenos=false,mathescape,baselinestretch=1.0,fontfamily=tt,xleftmargin=7mm]{python}
from sympy import symbols, diff, exp, sqrt, factor, Symbol, printing
x, y, z, Z = symbols('x y z Z')
r = sqrt(x*x + y*y + z*z)
phi = (Z*r - 2)*exp(-Z*r/2)
R = Symbol('r') #Creates a symbolic equivalent of r
(diff(diff(phi, x), x) + diff(diff(phi, y), y) + diff(diff(phi, z), z)).factor().subs(r, R)
# Collect the Z values
(diff(diff(phi, x), x) + diff(diff(phi, y), y) +diff(diff(phi, z), z)).factor().collect(Z).subs(r, R)
# Factorize also the r**2 terms
(diff(diff(phi, x), x) + diff(diff(phi, y), y) + diff(diff(phi, z), z)).factor().collect(Z).subs(r, R).subs(r**2, R**2).factor()
print printing.ccode((diff(diff(phi, x), x) + diff(diff(phi, y), y) + diff(diff(phi, z), z)).factor().collect(Z).subs(r, R).subs(r**2, R**2).factor())

\end{minted}

With some practice this allows one to be able to check one's own calculation and translate automatically into code lines.

\subsection*{The Metropolis algorithm}

The Metropolis algorithm , see \href{{http://scitation.aip.org/content/aip/journal/jcp/21/6/10.1063/1.1699114}}{the original article} was invented by Metropolis et. al
and is often simply called the Metropolis algorithm.
It is a method to sample a normalized probability
distribution by a stochastic process. We define $\mathbf{P}_i^{(n)}$ to
be the probability for finding the system in the state $i$ at step $n$.
The algorithm is then

\begin{itemize}
\item Sample a possible new state $j$ with some probability $T_{i\rightarrow j}$.

\item Accept the new state $j$ with probability $A_{i \rightarrow j}$ and use it as the next sample. With probability $1-A_{i\rightarrow j}$ the move is rejected and the original state $i$ is used again as a sample.
\end{itemize}

\noindent
We wish to derive the required properties of $T$ and $A$ such that
$\mathbf{P}_i^{(n\rightarrow \infty)} \rightarrow p_i$ so that starting
from any distribution, the method converges to the correct distribution.
Note that the description here is for a discrete probability distribution.
Replacing probabilities $p_i$ with expressions like $p(x_i)dx_i$ will
take all of these over to the corresponding continuum expressions.

The dynamical equation for $\mathbf{P}_i^{(n)}$ can be written directly from
the description above. The probability of being in the state $i$ at step $n$
is given by the probability of being in any state $j$ at the previous step,
and making an accepted transition to $i$ added to the probability of
being in the state $i$, making a transition to any state $j$ and
rejecting the move:
\[
\mathbf{P}^{(n)}_i = \sum_j \left [
\mathbf{P}^{(n-1)}_jT_{j\rightarrow i} A_{j\rightarrow i} 
+\mathbf{P}^{(n-1)}_iT_{i\rightarrow j}\left ( 1- A_{i\rightarrow j} \right)
\right ] \,.
\]
Since the probability of making some transition must be 1,
$\sum_j T_{i\rightarrow j} = 1$, and the above equation becomes
\[
\mathbf{P}^{(n)}_i = \mathbf{P}^{(n-1)}_i +
 \sum_j \left [
\mathbf{P}^{(n-1)}_jT_{j\rightarrow i} A_{j\rightarrow i} 
-\mathbf{P}^{(n-1)}_iT_{i\rightarrow j}A_{i\rightarrow j}
\right ] \,.
\]

For large $n$ we require that $\mathbf{P}^{(n\rightarrow \infty)}_i = p_i$,
the desired probability distribution. Taking this limit, gives the
balance requirement
\[
 \sum_j \left [
p_jT_{j\rightarrow i} A_{j\rightarrow i}
-p_iT_{i\rightarrow j}A_{i\rightarrow j}
\right ] = 0 \,.
\]
The balance requirement is very weak. Typically the much stronger detailed
balance requirement is enforced, that is rather than the sum being
set to zero, we set each term separately to zero and use this
to determine the acceptance probabilities. Rearranging, the result is
\[
\frac{ A_{j\rightarrow i}}{A_{i\rightarrow j}}
= \frac{p_iT_{i\rightarrow j}}{ p_jT_{j\rightarrow i}} \,.
\]

The Metropolis choice is to maximize the $A$ values, that is
\[
A_{j \rightarrow i} = \min \left ( 1,
\frac{p_iT_{i\rightarrow j}}{ p_jT_{j\rightarrow i}}\right ).
\]
Other choices are possible, but they all correspond to multilplying
$A_{i\rightarrow j}$ and $A_{j\rightarrow i}$ by the same constant
smaller than unity.\footnote{The penalty function method uses just such
a factor to compensate for $p_i$ that are evaluated stochastically
and are therefore noisy.}

Having chosen the acceptance probabilities, we have guaranteed that
if the  $\mathbf{P}_i^{(n)}$ has equilibrated, that is if it is equal to $p_i$,
it will remain equilibrated. Next we need to find the circumstances for
convergence to equilibrium.

The dynamical equation can be written as
\[
\mathbf{P}^{(n)}_i = \sum_j M_{ij}\mathbf{P}^{(n-1)}_j
\]
with the matrix $M$ given by
\[
M_{ij} = \delta_{ij}\left [ 1 -\sum_k T_{i\rightarrow k} A_{i \rightarrow k}
\right ] + T_{j\rightarrow i} A_{j\rightarrow i} \,.
\]
Summing over $i$ shows that $\sum_i M_{ij} = 1$, and since
$\sum_k T_{i\rightarrow k} = 1$, and $A_{i \rightarrow k} \leq 1$, the
elements of the matrix satisfy $M_{ij} \geq 0$. The matrix $M$ is therefore
a stochastic matrix.

The Metropolis method is simply the power method for computing the
right eigenvector of $M$ with the largest magnitude eigenvalue.
By construction, the correct probability distribution is a right eigenvector
with eigenvalue 1. Therefore, for the Metropolis method to converge
to this result, we must show that $M$ has only one eigenvalue with this
magnitude, and all other eigenvalues are smaller.

\subsection*{Importance sampling}

We need to replace the brute force
Metropolis algorithm with a walk in coordinate space biased by the trial wave function.
This approach is based on the Fokker-Planck equation and the Langevin equation for generating a trajectory in coordinate space.  The link between the Fokker-Planck equation and the Langevin equations are explained, only partly, in the slides below.
An excellent reference on topics like Brownian motion, Markov chains, the Fokker-Planck equation and the Langevin equation is the text by  \href{{http://www.elsevier.com/books/stochastic-processes-in-physics-and-chemistry/van-kampen/978-0-444-52965-7}}{Van Kampen}
Here we will focus first on the implementation part first.

For a diffusion process characterized by a time-dependent probability density $P(x,t)$ in one dimension the Fokker-Planck
equation reads (for one particle /walker) 
\[
   \frac{\partial P}{\partial t} = D\frac{\partial }{\partial x}\left(\frac{\partial }{\partial x} -F\right)P(x,t),
\]
where $F$ is a drift term and $D$ is the diffusion coefficient. 

The new positions in coordinate space are given as the solutions of the Langevin equation using Euler's method, namely,
we go from the Langevin equation
\[ 
   \frac{\partial x(t)}{\partial t} = DF(x(t)) +\eta,
\]
with $\eta$ a random variable,
yielding a new position 
\[
   y = x+DF(x)\Delta t +\xi\sqrt{\Delta t},
\]
where $\xi$ is gaussian random variable and $\Delta t$ is a chosen time step. 
The quantity $D$ is, in atomic units, equal to $1/2$ and comes from the factor $1/2$ in the kinetic energy operator. Note that $\Delta t$ is to be viewed as a parameter. Values of $\Delta t \in [0.001,0.01]$ yield in general rather stable values of the ground state energy.  

The process of isotropic diffusion characterized by a time-dependent probability density $P(\mathbf{x},t)$ obeys (as an approximation) the so-called Fokker-Planck equation 
\[
   \frac{\partial P}{\partial t} = \sum_i D\frac{\partial }{\partial \mathbf{x_i}}\left(\frac{\partial }{\partial \mathbf{x_i}} -\mathbf{F_i}\right)P(\mathbf{x},t),
\]
where $\mathbf{F_i}$ is the $i^{th}$ component of the drift term (drift velocity) caused by an external potential, and $D$ is the diffusion coefficient. The convergence to a stationary probability density can be obtained by setting the left hand side to zero. The resulting equation will be satisfied if and only if all the terms of the sum are equal zero,
\[
\frac{\partial^2 P}{\partial {\mathbf{x_i}^2}} = P\frac{\partial}{\partial {\mathbf{x_i}}}\mathbf{F_i} + \mathbf{F_i}\frac{\partial}{\partial {\mathbf{x_i}}}P.
\]

The drift vector should be of the form $\mathbf{F} = g(\mathbf{x}) \frac{\partial P}{\partial \mathbf{x}}$. Then,
\[
\frac{\partial^2 P}{\partial {\mathbf{x_i}^2}} = P\frac{\partial g}{\partial P}\left( \frac{\partial P}{\partial {\mathbf{x}_i}}  \right)^2 + P g \frac{\partial ^2 P}{\partial {\mathbf{x}_i^2}}  + g \left( \frac{\partial P}{\partial {\mathbf{x}_i}}  \right)^2.
\]
The condition of stationary density means that the left hand side equals zero. In other words, the terms containing first and second derivatives have to cancel each other. It is possible only if $g = \frac{1}{P}$, which yields
\[
\mathbf{F} = 2\frac{1}{\Psi_T}\nabla\Psi_T,
\]
which is known as the so-called \emph{quantum force}. This term is responsible for pushing the walker towards regions of configuration space where the trial wave function is large, increasing the efficiency of the simulation in contrast to the Metropolis algorithm where the walker has the same probability of moving in every direction.

The Fokker-Planck equation yields a (the solution to the equation) transition probability given by the Green's function
\[
  G(y,x,\Delta t) = \frac{1}{(4\pi D\Delta t)^{3N/2}} \exp{\left(-(y-x-D\Delta t F(x))^2/4D\Delta t\right)}
\]
which in turn means that our brute force Metropolis algorithm
\[ 
    A(y,x) = \mathrm{min}(1,q(y,x))),
\]
with $q(y,x) = |\Psi_T(y)|^2/|\Psi_T(x)|^2$ is now replaced by the \href{{http://scitation.aip.org/content/aip/journal/jcp/21/6/10.1063/1.1699114}}{Metropolis-Hastings algorithm} as well as \href{{http://biomet.oxfordjournals.org/content/57/1/97.abstract}}{Hasting's article}, 
\[
q(y,x) = \frac{G(x,y,\Delta t)|\Psi_T(y)|^2}{G(y,x,\Delta t)|\Psi_T(x)|^2}
\]

\subsection*{Importance sampling, program elements}

The general derivative formula of the Jastrow factor is (the subscript $C$ stands for Correlation)
\[
\frac{1}{\Psi_C}\frac{\partial \Psi_C}{\partial x_k} =
\sum_{i=1}^{k-1}\frac{\partial g_{ik}}{\partial x_k}
+
\sum_{i=k+1}^{N}\frac{\partial g_{ki}}{\partial x_k}
\]
However, 
with our written in way which can be reused later as
\[
\Psi_C=\prod_{i< j}g(r_{ij})= \exp{\left\{\sum_{i<j}f(r_{ij})\right\}},
\]
the gradient needed for the quantum force and local energy is easy to compute.  
The function $f(r_{ij})$ will depends on the system under study. In the equations below we will keep this general form.

In the Metropolis/Hasting algorithm, the \emph{acceptance ratio} determines the probability for a particle  to be accepted at a new position. The ratio of the trial wave functions evaluated at the new and current positions is given by ($OB$ for the onebody  part)
\[
R \equiv \frac{\Psi_{T}^{new}}{\Psi_{T}^{old}} = 
\frac{\Psi_{OB}^{new}}{\Psi_{OB}^{old}}\frac{\Psi_{C}^{new}}{\Psi_{C}^{old}}
\]
Here $\Psi_{OB}$ is our onebody part (Slater determinant or product of boson single-particle states)  while $\Psi_{C}$ is our correlation function, or Jastrow factor. 
We need to optimize the $\nabla \Psi_T / \Psi_T$ ratio and the second derivative as well, that is
the $\mathbf{\nabla}^2 \Psi_T/\Psi_T$ ratio. The first is needed when we compute the so-called quantum force in importance sampling.
The second is needed when we compute the kinetic energy term of the local energy.
\[
\frac{\mathbf{\mathbf{\nabla}}  \Psi}{\Psi}  = \frac{\mathbf{\nabla}  (\Psi_{OB} \, \Psi_{C})}{\Psi_{OB} \, \Psi_{C}}  =  \frac{ \Psi_C \mathbf{\nabla}  \Psi_{OB} + \Psi_{OB} \mathbf{\nabla}  \Psi_{C}}{\Psi_{OB} \Psi_{C}} = \frac{\mathbf{\nabla}  \Psi_{OB}}{\Psi_{OB}} + \frac{\mathbf{\nabla}   \Psi_C}{ \Psi_C}
\]

The expectation value of the kinetic energy expressed in atomic units for electron $i$ is 
\[
 \langle \hat{K}_i \rangle = -\frac{1}{2}\frac{\langle\Psi|\mathbf{\nabla}_{i}^2|\Psi \rangle}{\langle\Psi|\Psi \rangle},
\]
\[
\hat{K}_i = -\frac{1}{2}\frac{\mathbf{\nabla}_{i}^{2} \Psi}{\Psi}.
\]

The second derivative which enters the definition of the local energy is 
\[
\frac{\mathbf{\nabla}^2 \Psi}{\Psi}=\frac{\mathbf{\nabla}^2 \Psi_{OB}}{\Psi_{OB}} + \frac{\mathbf{\nabla}^2  \Psi_C}{ \Psi_C} + 2 \frac{\mathbf{\nabla}  \Psi_{OB}}{\Psi_{OB}}\cdot\frac{\mathbf{\nabla}   \Psi_C}{ \Psi_C}
\]
We discuss here how to calculate these quantities in an optimal way,

We have defined the correlated function as
\[
\Psi_C=\prod_{i< j}g(r_{ij})=\prod_{i< j}^Ng(r_{ij})= \prod_{i=1}^N\prod_{j=i+1}^Ng(r_{ij}),
\]
with 
$r_{ij}=|\mathbf{r}_i-\mathbf{r}_j|=\sqrt{(x_i-x_j)^2+(y_i-y_j)^2+(z_i-z_j)^2}$ in three dimensions or
$r_{ij}=|\mathbf{r}_i-\mathbf{r}_j|=\sqrt{(x_i-x_j)^2+(y_i-y_j)^2}$ if we work with two-dimensional systems.

In our particular case we have
\[
\Psi_C=\prod_{i< j}g(r_{ij})=\exp{\left\{\sum_{i<j}f(r_{ij})\right\}}.
\]

The total number of different relative distances $r_{ij}$ is $N(N-1)/2$. In a matrix storage format, the relative distances  form a strictly upper triangular matrix
\[
 \mathbf{r} \equiv \begin{pmatrix}
  0 & r_{1,2} & r_{1,3} & \cdots & r_{1,N} \\
  \vdots & 0       & r_{2,3} & \cdots & r_{2,N} \\
  \vdots & \vdots  & 0  & \ddots & \vdots  \\
  \vdots & \vdots  & \vdots  & \ddots  & r_{N-1,N} \\
  0 & 0  & 0  & \cdots  & 0
 \end{pmatrix}.
\]
This applies to  $\mathbf{g} = \mathbf{g}(r_{ij})$ as well. 

In our algorithm we will move one particle  at the time, say the $kth$-particle.  This sampling will be seen to be particularly efficient when we are going to compute a Slater determinant. 

We have that the ratio between Jastrow factors $R_C$ is given by
\[
R_{C} = \frac{\Psi_{C}^\mathrm{new}}{\Psi_{C}^\mathrm{cur}} =
\prod_{i=1}^{k-1}\frac{g_{ik}^\mathrm{new}}{g_{ik}^\mathrm{cur}}
\prod_{i=k+1}^{N}\frac{ g_{ki}^\mathrm{new}} {g_{ki}^\mathrm{cur}}.
\]
For the Pade-Jastrow form
\[
 R_{C} = \frac{\Psi_{C}^\mathrm{new}}{\Psi_{C}^\mathrm{cur}} = 
\frac{\exp{U_{new}}}{\exp{U_{cur}}} = \exp{\Delta U},
\]
where
\[
\Delta U =
\sum_{i=1}^{k-1}\big(f_{ik}^\mathrm{new}-f_{ik}^\mathrm{cur}\big)
+
\sum_{i=k+1}^{N}\big(f_{ki}^\mathrm{new}-f_{ki}^\mathrm{cur}\big)
\]

One needs to develop a special algorithm 
that runs only through the elements of the upper triangular
matrix $\mathbf{g}$ and have $k$ as an index. 

The expression to be derived in the following is of interest when computing the quantum force and the kinetic energy. It has the form
\[
\frac{\mathbf{\nabla}_i\Psi_C}{\Psi_C} = \frac{1}{\Psi_C}\frac{\partial \Psi_C}{\partial x_i},
\]
for all dimensions and with $i$ running over all particles.

For the first derivative only $N-1$ terms survive the ratio because the $g$-terms that are not differentiated cancel with their corresponding ones in the denominator. Then,
\[
\frac{1}{\Psi_C}\frac{\partial \Psi_C}{\partial x_k} =
\sum_{i=1}^{k-1}\frac{1}{g_{ik}}\frac{\partial g_{ik}}{\partial x_k}
+
\sum_{i=k+1}^{N}\frac{1}{g_{ki}}\frac{\partial g_{ki}}{\partial x_k}.
\]
An equivalent equation is obtained for the exponential form after replacing $g_{ij}$ by $\exp(f_{ij})$, yielding:
\[
\frac{1}{\Psi_C}\frac{\partial \Psi_C}{\partial x_k} =
\sum_{i=1}^{k-1}\frac{\partial g_{ik}}{\partial x_k}
+
\sum_{i=k+1}^{N}\frac{\partial g_{ki}}{\partial x_k},
\]
with both expressions scaling as $\mathcal{O}(N)$.

Using the identity 
\[
\frac{\partial}{\partial x_i}g_{ij} = -\frac{\partial}{\partial x_j}g_{ij},
\]
we get expressions where all the derivatives acting on the particle  are represented by the \emph{second} index of $g$:
\[
\frac{1}{\Psi_C}\frac{\partial \Psi_C}{\partial x_k} =
\sum_{i=1}^{k-1}\frac{1}{g_{ik}}\frac{\partial g_{ik}}{\partial x_k}
-\sum_{i=k+1}^{N}\frac{1}{g_{ki}}\frac{\partial g_{ki}}{\partial x_i},
\]
and for the exponential case:
\[
\frac{1}{\Psi_C}\frac{\partial \Psi_C}{\partial x_k} =
\sum_{i=1}^{k-1}\frac{\partial g_{ik}}{\partial x_k}
-\sum_{i=k+1}^{N}\frac{\partial g_{ki}}{\partial x_i}.
\]

For correlation forms depending only on the scalar distances $r_{ij}$ we can use the chain rule. Noting that 
\[
\frac{\partial g_{ij}}{\partial x_j} = \frac{\partial g_{ij}}{\partial r_{ij}} \frac{\partial r_{ij}}{\partial x_j} = \frac{x_j - x_i}{r_{ij}} \frac{\partial g_{ij}}{\partial r_{ij}},
\]
we arrive at
\[
\frac{1}{\Psi_C}\frac{\partial \Psi_C}{\partial x_k} = 
\sum_{i=1}^{k-1}\frac{1}{g_{ik}} \frac{\mathbf{r_{ik}}}{r_{ik}} \frac{\partial g_{ik}}{\partial r_{ik}}
-\sum_{i=k+1}^{N}\frac{1}{g_{ki}}\frac{\mathbf{r_{ki}}}{r_{ki}}\frac{\partial g_{ki}}{\partial r_{ki}}.
\]

Note that for the Pade-Jastrow form we can set $g_{ij} \equiv g(r_{ij}) = e^{f(r_{ij})} = e^{f_{ij}}$ and 
\[
\frac{\partial g_{ij}}{\partial r_{ij}} = g_{ij} \frac{\partial f_{ij}}{\partial r_{ij}}.
\]
Therefore, 
\[
\frac{1}{\Psi_{C}}\frac{\partial \Psi_{C}}{\partial x_k} =
\sum_{i=1}^{k-1}\frac{\mathbf{r_{ik}}}{r_{ik}}\frac{\partial f_{ik}}{\partial r_{ik}}
-\sum_{i=k+1}^{N}\frac{\mathbf{r_{ki}}}{r_{ki}}\frac{\partial f_{ki}}{\partial r_{ki}},
\]
where 
\[
 \mathbf{r}_{ij} = |\mathbf{r}_j - \mathbf{r}_i| = (x_j - x_i)\mathbf{e}_1 + (y_j - y_i)\mathbf{e}_2 + (z_j - z_i)\mathbf{e}_3
\]
is the relative distance. 

The second derivative of the Jastrow factor divided by the Jastrow factor (the way it enters the kinetic energy) is
\[
\left[\frac{\mathbf{\nabla}^2 \Psi_C}{\Psi_C}\right]_x =\  
2\sum_{k=1}^{N}
\sum_{i=1}^{k-1}\frac{\partial^2 g_{ik}}{\partial x_k^2}\ +\ 
\sum_{k=1}^N
\left(
\sum_{i=1}^{k-1}\frac{\partial g_{ik}}{\partial x_k} -
\sum_{i=k+1}^{N}\frac{\partial g_{ki}}{\partial x_i}
\right)^2
\]

But we have a simple form for the function, namely
\[
\Psi_{C}=\prod_{i< j}\exp{f(r_{ij})},
\]
and it is easy to see that for particle  $k$
we have
\[
  \frac{\mathbf{\nabla}^2_k \Psi_C}{\Psi_C }=
\sum_{ij\ne k}\frac{(\mathbf{r}_k-\mathbf{r}_i)(\mathbf{r}_k-\mathbf{r}_j)}{r_{ki}r_{kj}}f'(r_{ki})f'(r_{kj})+
\sum_{j\ne k}\left( f''(r_{kj})+\frac{2}{r_{kj}}f'(r_{kj})\right)
\]

\subsection*{Importance sampling, Fokker-Planck and Langevin equations}

A stochastic process is simply a function of two variables, one is the time,
the other is a stochastic variable $X$, defined by specifying
\begin{itemize}
\item the set $\left\{x\right\}$ of possible values for $X$;

\item the probability distribution, $w_X(x)$,  over this set, or briefly $w(x)$
\end{itemize}

\noindent
The set of values $\left\{x\right\}$ for $X$ 
may be discrete, or continuous. If the set of
values is continuous, then $w_X (x)$ is a probability density so that 
$w_X (x)dx$
is the probability that one finds the stochastic variable $X$ to have values
in the range $[x, x + dx]$ .

     An arbitrary number of other stochastic variables may be derived from
$X$. For example, any $Y$ given by a mapping of $X$, is also a stochastic
variable. The mapping may also be time-dependent, that is, the mapping
depends on an additional variable $t$
\[
                              Y_X (t) = f (X, t) .
\]
The quantity $Y_X (t)$ is called a random function, or, since $t$ often is time,
a stochastic process. A stochastic process is a function of two variables,
one is the time, the other is a stochastic variable $X$. Let $x$ be one of the
possible values of $X$ then
\[
                               y(t) = f (x, t),
\]
is a function of $t$, called a sample function or realization of the process.
In physics one considers the stochastic process to be an ensemble of such
sample functions.

     For many physical systems initial distributions of a stochastic 
variable $y$ tend to equilibrium distributions: $w(y, t)\rightarrow w_0(y)$ 
as $t\rightarrow\infty$. In
equilibrium detailed balance constrains the transition rates
\[
     W(y\rightarrow y')w(y ) = W(y'\rightarrow y)w_0 (y),
\]
where $W(y'\rightarrow y)$ 
is the probability, per unit time, that the system changes
from a state $|y\rangle$ , characterized by the value $y$ 
for the stochastic variable $Y$ , to a state $|y'\rangle$.

Note that for a system in equilibrium the transition rate 
$W(y'\rightarrow y)$ and
the reverse $W(y\rightarrow y')$ may be very different. 

Consider, for instance, a simple
system that has only two energy levels $\epsilon_0 = 0$ and 
$\epsilon_1 = \Delta E$. 

For a system governed by the Boltzmann distribution we find (the partition function has been taken out)
\[
     W(0\rightarrow 1)\exp{-(\epsilon_0/kT)} = W(1\rightarrow 0)\exp{-(\epsilon_1/kT)}
\]
We get then
\[
     \frac{W(1\rightarrow 0)}{W(0 \rightarrow 1)}=\exp{-(\Delta E/kT)},
\]
which goes to zero when $T$ tends to zero.

If we assume a discrete set of events,
our initial probability
distribution function can be  given by 
\[
   w_i(0) = \delta_{i,0},
\]
and its time-development after a given time step $\Delta t=\epsilon$ is
\[ 
   w_i(t) = \sum_{j}W(j\rightarrow i)w_j(t=0).
\] 
The continuous analog to $w_i(0)$ is
\[
   w(\mathbf{x})\rightarrow \delta(\mathbf{x}),
\]
where we now have generalized the one-dimensional position $x$ to a generic-dimensional  
vector $\mathbf{x}$. The Kroenecker $\delta$ function is replaced by the $\delta$ distribution
function $\delta(\mathbf{x})$ at  $t=0$.  

The transition from a state $j$ to a state $i$ is now replaced by a transition
to a state with position $\mathbf{y}$ from a state with position $\mathbf{x}$. 
The discrete sum of transition probabilities can then be replaced by an integral
and we obtain the new distribution at a time $t+\Delta t$ as 
\[
   w(\mathbf{y},t+\Delta t)= \int W(\mathbf{y},t+\Delta t| \mathbf{x},t)w(\mathbf{x},t)d\mathbf{x},
\]
and after $m$ time steps we have
\[
   w(\mathbf{y},t+m\Delta t)= \int W(\mathbf{y},t+m\Delta t| \mathbf{x},t)w(\mathbf{x},t)d\mathbf{x}.
\]
When equilibrium is reached we have
\[
   w(\mathbf{y})= \int W(\mathbf{y}|\mathbf{x}, t)w(\mathbf{x})d\mathbf{x},
\]
that is no time-dependence. Note our change of notation for $W$

We can solve the equation for $w(\mathbf{y},t)$ by making a Fourier transform to
momentum space. 
The PDF $w(\mathbf{x},t)$ is related to its Fourier transform
$\tilde{w}(\mathbf{k},t)$ through
\[
   w(\mathbf{x},t) = \int_{-\infty}^{\infty}d\mathbf{k} \exp{(i\mathbf{kx})}\tilde{w}(\mathbf{k},t),
\]
and using the definition of the 
$\delta$-function 
\[
   \delta(\mathbf{x}) = \frac{1}{2\pi} \int_{-\infty}^{\infty}d\mathbf{k} \exp{(i\mathbf{kx})},
\]
 we see that
\[
   \tilde{w}(\mathbf{k},0)=1/2\pi.
\]

We can then use the Fourier-transformed diffusion equation 
\[
    \frac{\partial \tilde{w}(\mathbf{k},t)}{\partial t} = -D\mathbf{k}^2\tilde{w}(\mathbf{k},t),
\]
with the obvious solution
\[
   \tilde{w}(\mathbf{k},t)=\tilde{w}(\mathbf{k},0)\exp{\left[-(D\mathbf{k}^2t)\right)}=
    \frac{1}{2\pi}\exp{\left[-(D\mathbf{k}^2t)\right]}. 
\]

With the Fourier transform we obtain 
\[
   w(\mathbf{x},t)=\int_{-\infty}^{\infty}d\mathbf{k} \exp{\left[i\mathbf{kx}\right]}\frac{1}{2\pi}\exp{\left[-(D\mathbf{k}^2t)\right]}=
    \frac{1}{\sqrt{4\pi Dt}}\exp{\left[-(\mathbf{x}^2/4Dt)\right]}, 
\]
with the normalization condition
\[
   \int_{-\infty}^{\infty}w(\mathbf{x},t)d\mathbf{x}=1.
\]

The solution represents the probability of finding
our random walker at position $\mathbf{x}$ at time $t$ if the initial distribution 
was placed at $\mathbf{x}=0$ at $t=0$. 

There is another interesting feature worth observing. The discrete transition probability $W$
itself is given by a binomial distribution.
The results from the central limit theorem state that 
transition probability in the limit $n\rightarrow \infty$ converges to the normal 
distribution. It is then possible to show that
\[
    W(il-jl,n\epsilon)\rightarrow W(\mathbf{y},t+\Delta t|\mathbf{x},t)=
    \frac{1}{\sqrt{4\pi D\Delta t}}\exp{\left[-((\mathbf{y}-\mathbf{x})^2/4D\Delta t)\right]},
\]
and that it satisfies the normalization condition and is itself a solution
to the diffusion equation.

Let us now assume that we have three PDFs for times $t_0 < t' < t$, that is
$w(\mathbf{x}_0,t_0)$, $w(\mathbf{x}',t')$ and $w(\mathbf{x},t)$.
We have then  
\[
   w(\mathbf{x},t)= \int_{-\infty}^{\infty} W(\mathbf{x}.t|\mathbf{x}'.t')w(\mathbf{x}',t')d\mathbf{x}',
\]
and
\[
   w(\mathbf{x},t)= \int_{-\infty}^{\infty} W(\mathbf{x}.t|\mathbf{x}_0.t_0)w(\mathbf{x}_0,t_0)d\mathbf{x}_0,
\]
and
\[
   w(\mathbf{x}',t')= \int_{-\infty}^{\infty} W(\mathbf{x}'.t'|\mathbf{x}_0,t_0)w(\mathbf{x}_0,t_0)d\mathbf{x}_0.
\]

We can combine these equations and arrive at the famous Einstein-Smoluchenski-Kolmogorov-Chapman (ESKC) relation
\[
 W(\mathbf{x}t|\mathbf{x}_0t_0)  = \int_{-\infty}^{\infty} W(\mathbf{x},t|\mathbf{x}',t')W(\mathbf{x}',t'|\mathbf{x}_0,t_0)d\mathbf{x}'.
\]
We can replace the spatial dependence with a dependence upon say the velocity
(or momentum), that is we have
\[
 W(\mathbf{v},t|\mathbf{v}_0,t_0)  = \int_{-\infty}^{\infty} W(\mathbf{v},t|\mathbf{v}',t')W(\mathbf{v}',t'|\mathbf{v}_0,t_0)d\mathbf{x}'.
\]

We will now derive the Fokker-Planck equation. 
We start from the ESKC equation
\[
 W(\mathbf{x},t|\mathbf{x}_0,t_0)  = \int_{-\infty}^{\infty} W(\mathbf{x},t|\mathbf{x}',t')W(\mathbf{x}',t'|\mathbf{x}_0,t_0)d\mathbf{x}'.
\]
Define $s=t'-t_0$, $\tau=t-t'$ and $t-t_0=s+\tau$. We have then
\[
 W(\mathbf{x},s+\tau|\mathbf{x}_0)  = \int_{-\infty}^{\infty} W(\mathbf{x},\tau|\mathbf{x}')W(\mathbf{x}',s|\mathbf{x}_0)d\mathbf{x}'.
\]

Assume now that $\tau$ is very small so that we can make an expansion in terms of a small step $xi$, with $\mathbf{x}'=\mathbf{x}-\xi$, that is
\[
 W(\mathbf{x},s|\mathbf{x}_0)+\frac{\partial W}{\partial s}\tau +O(\tau^2) = \int_{-\infty}^{\infty} W(\mathbf{x},\tau|\mathbf{x}-\xi)W(\mathbf{x}-\xi,s|\mathbf{x}_0)d\mathbf{x}'.
\]
We assume that $W(\mathbf{x},\tau|\mathbf{x}-\xi)$ takes non-negligible values only when $\xi$ is small. This is just another way of stating the Master equation!!

We say thus that $\mathbf{x}$ changes only by a small amount in the time interval $\tau$. 
This means that we can make a Taylor expansion in terms of $\xi$, that is we
expand
\[
W(\mathbf{x},\tau|\mathbf{x}-\xi)W(\mathbf{x}-\xi,s|\mathbf{x}_0) =
\sum_{n=0}^{\infty}\frac{(-\xi)^n}{n!}\frac{\partial^n}{\partial x^n}\left[W(\mathbf{x}+\xi,\tau|\mathbf{x})W(\mathbf{x},s|\mathbf{x}_0)
\right].
\]

We can then rewrite the ESKC equation as 
\[
\frac{\partial W}{\partial s}\tau=-W(\mathbf{x},s|\mathbf{x}_0)+
\sum_{n=0}^{\infty}\frac{(-\xi)^n}{n!}\frac{\partial^n}{\partial x^n}
\left[W(\mathbf{x},s|\mathbf{x}_0)\int_{-\infty}^{\infty} \xi^nW(\mathbf{x}+\xi,\tau|\mathbf{x})d\xi\right].
\]
We have neglected higher powers of $\tau$ and have used that for $n=0$ 
we get simply $W(\mathbf{x},s|\mathbf{x}_0)$ due to normalization.

We say thus that $\mathbf{x}$ changes only by a small amount in the time interval $\tau$. 
This means that we can make a Taylor expansion in terms of $\xi$, that is we
expand
\[
W(\mathbf{x},\tau|\mathbf{x}-\xi)W(\mathbf{x}-\xi,s|\mathbf{x}_0) =
\sum_{n=0}^{\infty}\frac{(-\xi)^n}{n!}\frac{\partial^n}{\partial x^n}\left[W(\mathbf{x}+\xi,\tau|\mathbf{x})W(\mathbf{x},s|\mathbf{x}_0)
\right].
\]

We can then rewrite the ESKC equation as 
\[
\frac{\partial W(\mathbf{x},s|\mathbf{x}_0)}{\partial s}\tau=-W(\mathbf{x},s|\mathbf{x}_0)+
\sum_{n=0}^{\infty}\frac{(-\xi)^n}{n!}\frac{\partial^n}{\partial x^n}
\left[W(\mathbf{x},s|\mathbf{x}_0)\int_{-\infty}^{\infty} \xi^nW(\mathbf{x}+\xi,\tau|\mathbf{x})d\xi\right].
\]
We have neglected higher powers of $\tau$ and have used that for $n=0$ 
we get simply $W(\mathbf{x},s|\mathbf{x}_0)$ due to normalization.

We simplify the above by introducing the moments 
\[
M_n=\frac{1}{\tau}\int_{-\infty}^{\infty} \xi^nW(\mathbf{x}+\xi,\tau|\mathbf{x})d\xi=
\frac{\langle [\Delta x(\tau)]^n\rangle}{\tau},
\]
resulting in
\[
\frac{\partial W(\mathbf{x},s|\mathbf{x}_0)}{\partial s}=
\sum_{n=1}^{\infty}\frac{(-\xi)^n}{n!}\frac{\partial^n}{\partial x^n}
\left[W(\mathbf{x},s|\mathbf{x}_0)M_n\right].
\]

When $\tau \rightarrow 0$ we assume that $\langle [\Delta x(\tau)]^n\rangle \rightarrow 0$ more rapidly than $\tau$ itself if $n > 2$. 
When $\tau$ is much larger than the standard correlation time of 
system then $M_n$ for $n > 2$ can normally be neglected.
This means that fluctuations become negligible at large time scales.

If we neglect such terms we can rewrite the ESKC equation as 
\[
\frac{\partial W(\mathbf{x},s|\mathbf{x}_0)}{\partial s}=
-\frac{\partial M_1W(\mathbf{x},s|\mathbf{x}_0)}{\partial x}+
\frac{1}{2}\frac{\partial^2 M_2W(\mathbf{x},s|\mathbf{x}_0)}{\partial x^2}.
\]

In a more compact form we have
\[
\frac{\partial W}{\partial s}=
-\frac{\partial M_1W}{\partial x}+
\frac{1}{2}\frac{\partial^2 M_2W}{\partial x^2},
\]
which is the Fokker-Planck equation!  It is trivial to replace 
position with velocity (momentum).

Consider a particle  suspended in a liquid. On its path through the liquid it will continuously collide with the liquid molecules. Because on average the particle  will collide more often on the front side than on the back side, it will experience a systematic force proportional with its velocity, and directed opposite to its velocity. Besides this systematic force the particle  will experience a stochastic force  $\mathbf{F}(t)$. 
The equations of motion are 
\begin{itemize}
\item $\frac{d\mathbf{r}}{dt}=\mathbf{v}$ and 

\item $\frac{d\mathbf{v}}{dt}=-\xi \mathbf{v}+\mathbf{F}$.
\end{itemize}

\noindent
From hydrodynamics  we know that the friction constant  $\xi$ is given by
\[
\xi =6\pi \eta a/m 
\]
where $\eta$ is the viscosity  of the solvent and a is the radius of the particle .

Solving the second equation in the previous slide we get 
\[
\mathbf{v}(t)=\mathbf{v}_{0}e^{-\xi t}+\int_{0}^{t}d\tau e^{-\xi (t-\tau )}\mathbf{F }(\tau ). 
\]

If we want to get some useful information out of this, we have to average over all possible realizations of 
$\mathbf{F}(t)$, with the initial velocity as a condition. A useful quantity for example is
\[ 
\langle \mathbf{v}(t)\cdot \mathbf{v}(t)\rangle_{\mathbf{v}_{0}}=v_{0}^{-\xi 2t}
+2\int_{0}^{t}d\tau e^{-\xi (2t-\tau)}\mathbf{v}_{0}\cdot \langle \mathbf{F}(\tau )\rangle_{\mathbf{v}_{0}}
\]
\[  	  	
 +\int_{0}^{t}d\tau ^{\prime }\int_{0}^{t}d\tau e^{-\xi (2t-\tau -\tau ^{\prime })}
\langle \mathbf{F}(\tau )\cdot \mathbf{F}(\tau ^{\prime })\rangle_{ \mathbf{v}_{0}}.
\]

In order to continue we have to make some assumptions about the conditional averages of the stochastic forces. 
In view of the chaotic character of the stochastic forces the following 
assumptions seem to be appropriate
\[ 
\langle \mathbf{F}(t)\rangle=0, 
\]
and
\[
\langle \mathbf{F}(t)\cdot \mathbf{F}(t^{\prime })\rangle_{\mathbf{v}_{0}}=  C_{\mathbf{v}_{0}}\delta (t-t^{\prime }).
\] 	

We omit the subscript $\mathbf{v}_{0}$, when the quantity of interest turns out to be independent of $\mathbf{v}_{0}$. Using the last three equations we get
 \[
\langle \mathbf{v}(t)\cdot \mathbf{v}(t)\rangle_{\mathbf{v}_{0}}=v_{0}^{2}e^{-2\xi t}+\frac{C_{\mathbf{v}_{0}}}{2\xi }(1-e^{-2\xi t}).
\]
For large t this should be equal to 3kT/m, from which it follows that
\[
\langle \mathbf{F}(t)\cdot \mathbf{F}(t^{\prime })\rangle =6\frac{kT}{m}\xi \delta (t-t^{\prime }). 
\]
This result is called the fluctuation-dissipation theorem .

Integrating 
 \[ 
\mathbf{v}(t)=\mathbf{v}_{0}e^{-\xi t}+\int_{0}^{t}d\tau e^{-\xi (t-\tau )}\mathbf{F }(\tau ), 
\] 
we get
\[
\mathbf{r}(t)=\mathbf{r}_{0}+\mathbf{v}_{0}\frac{1}{\xi }(1-e^{-\xi t})+
\int_0^td\tau \int_0^{\tau}\tau ^{\prime } e^{-\xi (\tau -\tau ^{\prime })}\mathbf{F}(\tau ^{\prime }), 
\]
from which we calculate the mean square displacement 
\[
\langle ( \mathbf{r}(t)-\mathbf{r}_{0})^{2}\rangle _{\mathbf{v}_{0}}=\frac{v_0^2}{\xi}(1-e^{-\xi t})^{2}+\frac{3kT}{m\xi ^{2}}(2\xi t-3+4e^{-\xi t}-e^{-2\xi t}). 
\]

For very large $t$ this becomes
\[
\langle (\mathbf{r}(t)-\mathbf{r}_{0})^{2}\rangle =\frac{6kT}{m\xi }t 
\] 
from which we get the Einstein relation  
 \[ 
D= \frac{kT}{m\xi } 
\] 	
where we have used $\langle (\mathbf{r}(t)-\mathbf{r}_{0})^{2}\rangle =6Dt$.

\subsection*{Code example for two electrons in a quantum dots}
























































































































































\begin{minted}[fontsize=\fontsize{9pt}{9pt},linenos=false,mathescape,baselinestretch=1.0,fontfamily=tt,xleftmargin=7mm]{python}
# 2-electron VMC code for 2dim quantum dot with importance sampling
# Using gaussian rng for new positions and Metropolis- Hastings 
# No energy minimization
from math import exp, sqrt
from random import random, seed, normalvariate
import numpy as np
import matplotlib.pyplot as plt
from mpl_toolkits.mplot3d import Axes3D
from matplotlib import cm
from matplotlib.ticker import LinearLocator, FormatStrFormatter
import sys
from numba import jit,njit


#Read name of output file from command line
if len(sys.argv) == 2:
    outfilename = sys.argv[1]
else:
    print('\nError: Name of output file must be given as command line argument.\n')
outfile = open(outfilename,'w')

# Trial wave function for the 2-electron quantum dot in two dims
def WaveFunction(r,alpha,beta):
    r1 = r[0,0]**2 + r[0,1]**2
    r2 = r[1,0]**2 + r[1,1]**2
    r12 = sqrt((r[0,0]-r[1,0])**2 + (r[0,1]-r[1,1])**2)
    deno = r12/(1+beta*r12)
    return exp(-0.5*alpha*(r1+r2)+deno)

# Local energy  for the 2-electron quantum dot in two dims, using analytical local energy
def LocalEnergy(r,alpha,beta):
    
    r1 = (r[0,0]**2 + r[0,1]**2)
    r2 = (r[1,0]**2 + r[1,1]**2)
    r12 = sqrt((r[0,0]-r[1,0])**2 + (r[0,1]-r[1,1])**2)
    deno = 1.0/(1+beta*r12)
    deno2 = deno*deno
    return 0.5*(1-alpha*alpha)*(r1 + r2) +2.0*alpha + 1.0/r12+deno2*(alpha*r12-deno2+2*beta*deno-1.0/r12)

# Setting up the quantum force for the two-electron quantum dot, recall that it is a vector
def QuantumForce(r,alpha,beta):

    qforce = np.zeros((NumberParticles,Dimension), np.double)
    r12 = sqrt((r[0,0]-r[1,0])**2 + (r[0,1]-r[1,1])**2)
    deno = 1.0/(1+beta*r12)
    qforce[0,:] = -2*r[0,:]*alpha*(r[0,:]-r[1,:])*deno*deno/r12
    qforce[1,:] = -2*r[1,:]*alpha*(r[1,:]-r[0,:])*deno*deno/r12
    return qforce
    
# The Monte Carlo sampling with the Metropolis algo
# jit decorator tells Numba to compile this function.
# The argument types will be inferred by Numba when function is called.
@jit()
def MonteCarloSampling():

    NumberMCcycles= 100000
    # Parameters in the Fokker-Planck simulation of the quantum force
    D = 0.5
    TimeStep = 0.05
    # positions
    PositionOld = np.zeros((NumberParticles,Dimension), np.double)
    PositionNew = np.zeros((NumberParticles,Dimension), np.double)
    # Quantum force
    QuantumForceOld = np.zeros((NumberParticles,Dimension), np.double)
    QuantumForceNew = np.zeros((NumberParticles,Dimension), np.double)

    # seed for rng generator 
    seed()
    # start variational parameter  loops, two parameters here
    alpha = 0.9
    for ia in range(MaxVariations):
        alpha += .025
        AlphaValues[ia] = alpha
        beta = 0.2 
        for jb in range(MaxVariations):
            beta += .01
            BetaValues[jb] = beta
            energy = energy2 = 0.0
            DeltaE = 0.0
            #Initial position
            for i in range(NumberParticles):
                for j in range(Dimension):
                    PositionOld[i,j] = normalvariate(0.0,1.0)*sqrt(TimeStep)
            wfold = WaveFunction(PositionOld,alpha,beta)
            QuantumForceOld = QuantumForce(PositionOld,alpha, beta)

            #Loop over MC MCcycles
            for MCcycle in range(NumberMCcycles):
                #Trial position moving one particle at the time
                for i in range(NumberParticles):
                    for j in range(Dimension):
                        PositionNew[i,j] = PositionOld[i,j]+normalvariate(0.0,1.0)*sqrt(TimeStep)+\
                                           QuantumForceOld[i,j]*TimeStep*D
                    wfnew = WaveFunction(PositionNew,alpha,beta)
                    QuantumForceNew = QuantumForce(PositionNew,alpha, beta)
                    GreensFunction = 0.0
                    for j in range(Dimension):
                        GreensFunction += 0.5*(QuantumForceOld[i,j]+QuantumForceNew[i,j])*\
	                              (D*TimeStep*0.5*(QuantumForceOld[i,j]-QuantumForceNew[i,j])-\
                                      PositionNew[i,j]+PositionOld[i,j])
      
                    GreensFunction = exp(GreensFunction)
                    ProbabilityRatio = GreensFunction*wfnew**2/wfold**2
                    #Metropolis-Hastings test to see whether we accept the move
                    if random() <= ProbabilityRatio:
                       for j in range(Dimension):
                           PositionOld[i,j] = PositionNew[i,j]
                           QuantumForceOld[i,j] = QuantumForceNew[i,j]
                       wfold = wfnew
                DeltaE = LocalEnergy(PositionOld,alpha,beta)
                energy += DeltaE
                energy2 += DeltaE**2
            # We calculate mean, variance and error (no blocking applied)
            energy /= NumberMCcycles
            energy2 /= NumberMCcycles
            variance = energy2 - energy**2
            error = sqrt(variance/NumberMCcycles)
            Energies[ia,jb] = energy    
            outfile.write('%f %f %f %f %f\n' %(alpha,beta,energy,variance,error))
    return Energies, AlphaValues, BetaValues


#Here starts the main program with variable declarations
NumberParticles = 2
Dimension = 2
MaxVariations = 10
Energies = np.zeros((MaxVariations,MaxVariations))
AlphaValues = np.zeros(MaxVariations)
BetaValues = np.zeros(MaxVariations)
(Energies, AlphaValues, BetaValues) = MonteCarloSampling()
outfile.close()
# Prepare for plots
fig = plt.figure()
ax = fig.gca(projection='3d')
# Plot the surface.
X, Y = np.meshgrid(AlphaValues, BetaValues)
surf = ax.plot_surface(X, Y, Energies,cmap=cm.coolwarm,linewidth=0, antialiased=False)
# Customize the z axis.
zmin = np.matrix(Energies).min()
zmax = np.matrix(Energies).max()
ax.set_zlim(zmin, zmax)
ax.set_xlabel(r'$\alpha$')
ax.set_ylabel(r'$\beta$')
ax.set_zlabel(r'$\langle E \rangle$')
ax.zaxis.set_major_locator(LinearLocator(10))
ax.zaxis.set_major_formatter(FormatStrFormatter('%.02f'))
# Add a color bar which maps values to colors.
fig.colorbar(surf, shrink=0.5, aspect=5)
plt.show()



\end{minted}


\paragraph{Bringing the gradient optmization.}
The simple one-particle case in a harmonic oscillator trap


















\begin{minted}[fontsize=\fontsize{9pt}{9pt},linenos=false,mathescape,baselinestretch=1.0,fontfamily=tt,xleftmargin=7mm]{python}
# Gradient descent stepping with analytical derivative
import numpy as np
from scipy.optimize import minimize
def DerivativeE(x):
    return x-1.0/(4*x*x*x);

def Energy(x):
   return x*x*0.5+1.0/(8*x*x);
x0 = 1.0
eta = 0.1
Niterations = 100

for iter in range(Niterations):
    gradients = DerivativeE(x0)
    x0 -= eta*gradients

print(x0)

\end{minted}




































































































































\begin{minted}[fontsize=\fontsize{9pt}{9pt},linenos=false,mathescape,baselinestretch=1.0,fontfamily=tt,xleftmargin=7mm]{python}
# 2-electron VMC code for 2dim quantum dot with importance sampling
# Using gaussian rng for new positions and Metropolis- Hastings 
from math import exp, sqrt
from random import random, seed, normalvariate
import numpy as np
import matplotlib.pyplot as plt
from mpl_toolkits.mplot3d import Axes3D
from matplotlib import cm
from matplotlib.ticker import LinearLocator, FormatStrFormatter
import sys
from numba import jit


# Trial wave function for the 2-electron quantum dot in two dims
def WaveFunction(r,alpha):
    r1 = r[0,0]**2 + r[0,1]**2
    r2 = r[1,0]**2 + r[1,1]**2
    return exp(-0.5*alpha*(r1+r2))

# Local energy  for the 2-electron quantum dot in two dims, using analytical local energy
def LocalEnergy(r,alpha):
    
    r1 = (r[0,0]**2 + r[0,1]**2)
    r2 = (r[1,0]**2 + r[1,1]**2)
    return 0.5*(1-alpha*alpha)*(r1 + r2) +2.0*alpha

# Derivate of wave function ansatz as function of variational parameters
def DerivativeWFansatz(r,alpha):
    
    r1 = (r[0,0]**2 + r[0,1]**2)
    r2 = (r[1,0]**2 + r[1,1]**2)
    WfDer = -(r1+r2)
    return  WfDer

# Setting up the quantum force for the two-electron quantum dot, recall that it is a vector
def QuantumForce(r,alpha):

    qforce = np.zeros((NumberParticles,Dimension), np.double)
    qforce[0,:] = -2*r[0,:]*alpha
    qforce[1,:] = -2*r[1,:]*alpha
    return qforce
    
# Computing the derivative of the energy and the energy 
# jit decorator tells Numba to compile this function.
# The argument types will be inferred by Numba when function is called.
@jit
def EnergyMinimization(alpha):

    NumberMCcycles= 1000
    # Parameters in the Fokker-Planck simulation of the quantum force
    D = 0.5
    TimeStep = 0.05
    # positions
    PositionOld = np.zeros((NumberParticles,Dimension), np.double)
    PositionNew = np.zeros((NumberParticles,Dimension), np.double)
    # Quantum force
    QuantumForceOld = np.zeros((NumberParticles,Dimension), np.double)
    QuantumForceNew = np.zeros((NumberParticles,Dimension), np.double)

    # seed for rng generator 
    seed()
    energy = 0.0
    DeltaE = 0.0
    EnergyDer = 0.0
    DeltaPsi = 0.0
    DerivativePsiE = 0.0
    #Initial position
    for i in range(NumberParticles):
        for j in range(Dimension):
            PositionOld[i,j] = normalvariate(0.0,1.0)*sqrt(TimeStep)
    wfold = WaveFunction(PositionOld,alpha)
    QuantumForceOld = QuantumForce(PositionOld,alpha)

    #Loop over MC MCcycles
    for MCcycle in range(NumberMCcycles):
        #Trial position moving one particle at the time
        for i in range(NumberParticles):
            for j in range(Dimension):
                PositionNew[i,j] = PositionOld[i,j]+normalvariate(0.0,1.0)*sqrt(TimeStep)+\
                                       QuantumForceOld[i,j]*TimeStep*D
            wfnew = WaveFunction(PositionNew,alpha)
            QuantumForceNew = QuantumForce(PositionNew,alpha)
            GreensFunction = 0.0
            for j in range(Dimension):
                GreensFunction += 0.5*(QuantumForceOld[i,j]+QuantumForceNew[i,j])*\
	                              (D*TimeStep*0.5*(QuantumForceOld[i,j]-QuantumForceNew[i,j])-\
                                      PositionNew[i,j]+PositionOld[i,j])
      
            GreensFunction = exp(GreensFunction)
            ProbabilityRatio = GreensFunction*wfnew**2/wfold**2
            #Metropolis-Hastings test to see whether we accept the move
            if random() <= ProbabilityRatio:
                for j in range(Dimension):
                    PositionOld[i,j] = PositionNew[i,j]
                    QuantumForceOld[i,j] = QuantumForceNew[i,j]
                wfold = wfnew
        DeltaE = LocalEnergy(PositionOld,alpha)
        DeltaPsi = DerivativeWFansatz(PositionOld,alpha)
        energy += DeltaE
        DerivativePsiE += DeltaPsi*DeltaE
            
    # We calculate mean, variance and error (no blocking applied)
    energy /= NumberMCcycles
    DerivativePsiE /= NumberMCcycles
    DeltaPsi /= NumberMCcycles
    EnergyDer  = 2*(DerivativePsiE-DeltaPsi*energy)
    return energy, EnergyDer


#Here starts the main program with variable declarations
NumberParticles = 2
Dimension = 2
# guess for variational parameters
x0 = 1.5
# Set up iteration using stochastic gradient method
Energy =0 ; EnergyDer = 0
Energy, EnergyDer = EnergyMinimization(x0)
print(Energy, EnergyDer)

eta = 0.01
Niterations = 100

for iter in range(Niterations):
    gradients = EnergyDer
    x0 -= eta*gradients
    Energy, EnergyDer = EnergyMinimization(x0)

print(x0)


\end{minted}


\subsection*{VMC for fermions: Efficient calculation of Slater determinants}
The potentially most time-consuming part is the
evaluation of the gradient and the Laplacian of an $N$-particle  Slater
determinant. 

We have to differentiate the determinant with respect to
all spatial coordinates of all particles. A brute force
differentiation would involve $N\cdot d$ evaluations of the entire
determinant which would even worsen the already undesirable time
scaling, making it $Nd\cdot O(N^3)\sim O(d\cdot N^4)$.

This poses serious hindrances to the overall efficiency of our code.

The efficiency can be improved however if we move only one electron at the time.
The Slater determinant matrix $\hat{D}$ is defined by the matrix elements
\[
d_{ij}=\phi_j(x_i)
\]
where $\phi_j(\mathbf{r}_i)$ is a single particle  wave function.
The columns correspond to the position of a given particle 
while the rows stand for the various quantum numbers.

What we need to realize is that when differentiating a Slater
determinant with respect to some given coordinate, only one row of the
corresponding Slater matrix is changed. 

Therefore, by recalculating
the whole determinant we risk producing redundant information. The
solution turns out to be an algorithm that requires to keep track of
the \emph{inverse} of the Slater matrix.

Let the current position in phase space be represented by the $(N\cdot d)$-element 
vector $\mathbf{r}^{\mathrm{old}}$ and the new suggested
position by the vector $\mathbf{r}^{\mathrm{new}}$.

The inverse of $\hat{D}$ can be expressed in terms of its
cofactors $C_{ij}$ and its determinant (this our notation for a determinant) $\vert\hat{D}\vert$:
\begin{equation}
d_{ij}^{-1} = \frac{C_{ji}}{\vert\hat{D}\vert}
\label{eq:inverse_cofactor}
\end{equation}
Notice that the interchanged indices indicate that the matrix of cofactors is to be transposed.

If $\hat{D}$ is invertible, then we must obviously have $\hat{D}^{-1}\hat{D}= \mathbf{1}$, or explicitly in terms of the individual
elements of $\hat{D}$ and $\hat{D}^{-1}$:
\begin{equation}
\sum_{k=1}^N d_{ik}^{\phantom X}d_{kj}^{-1} = \delta_{ij}^{\phantom X}
\label{eq:unity_explicitely}
\end{equation}

Consider the ratio, which we shall call $R$, between $\vert\hat{D}(\mathbf{r}^{\mathrm{new}})\vert$ and $\vert\hat{D}(\mathbf{r}^{\mathrm{old}})\vert$. 
By definition, each of these determinants can
individually be expressed in terms of the \emph{i}-th row of its cofactor
matrix
\begin{equation}
R\equiv\frac{\vert\hat{D}(\mathbf{r}^{\mathrm{new}})\vert}
{\vert\hat{D}(\mathbf{r}^{\mathrm{old}})\vert} =
\frac{\sum_{j=1}^N d_{ij}(\mathbf{r}^{\mathrm{new}})\,
C_{ij}(\mathbf{r}^{\mathrm{new}})}
{\sum_{j=1}^N d_{ij}(\mathbf{r}^{\mathrm{old}})\,
C_{ij}(\mathbf{r}^{\mathrm{old}})}
\label{eq:detratio_cofactors}
\end{equation}

Suppose now that we move only one particle  at a time, meaning that
$\mathbf{r}^{\mathrm{new}}$ differs from $\mathbf{r}^{\mathrm{old}}$ by the
position of only one, say the \emph{i}-th, particle . This means that $\hat{D}(\mathbf{r}^{\mathrm{new}})$ and $\hat{D}(\mathbf{r}^{\mathrm{old}})$ differ
only by the entries of the \emph{i}-th row.  Recall also that the \emph{i}-th row
of a cofactor matrix $\hat{C}$ is independent of the entries of the
\emph{i}-th row of its corresponding matrix $\hat{D}$. In this particular
case we therefore get that the \emph{i}-th row of $\hat{C}(\mathbf{r}^{\mathrm{new}})$ 
and $\hat{C}(\mathbf{r}^{\mathrm{old}})$ must be
equal. Explicitly, we have:
\begin{equation}
C_{ij}(\mathbf{r}^{\mathrm{new}}) = C_{ij}(\mathbf{r}^{\mathrm{old}})\quad
\forall\ j\in\{1,\dots,N\}
\end{equation}

Inserting this into the numerator of eq.~(\ref{eq:detratio_cofactors})
and using eq.~(\ref{eq:inverse_cofactor}) to substitute the cofactors
with the elements of the inverse matrix, we get:
\begin{equation}
R =\frac{\sum_{j=1}^N d_{ij}(\mathbf{r}^{\mathrm{new}})\,
C_{ij}(\mathbf{r}^{\mathrm{old}})}
{\sum_{j=1}^N d_{ij}(\mathbf{r}^{\mathrm{old}})\,
C_{ij}(\mathbf{r}^{\mathrm{old}})} =
\frac{\sum_{j=1}^N d_{ij}(\mathbf{r}^{\mathrm{new}})\,
d_{ji}^{-1}(\mathbf{r}^{\mathrm{old}})}
{\sum_{j=1}^N d_{ij}(\mathbf{r}^{\mathrm{old}})\,
d_{ji}^{-1}(\mathbf{r}^{\mathrm{old}})}
\end{equation}

Now by eq.~(\ref{eq:unity_explicitely}) the denominator of the rightmost
expression must be unity, so that we finally arrive at:
\begin{equation}
R =
\sum_{j=1}^N d_{ij}(\mathbf{r}^{\mathrm{new}})\,
d_{ji}^{-1}(\mathbf{r}^{\mathrm{old}}) = 
\sum_{j=1}^N \phi_j(\mathbf{r}_i^{\mathrm{new}})\,
d_{ji}^{-1}(\mathbf{r}^{\mathrm{old}})
\label{eq:detratio_inverse}
\end{equation}
What this means is that in order to get the ratio when only the \emph{i}-th
particle  has been moved, we only need to calculate the dot
product of the vector $\left(\phi_1(\mathbf{r}_i^\mathrm{new}),\,\dots,\,\phi_N(\mathbf{r}_i^\mathrm{new})\right)$ of single particle  wave functions
evaluated at this new position with the \emph{i}-th column of the inverse
matrix $\hat{D}^{-1}$ evaluated at the original position. Such
an operation has a time scaling of $O(N)$. The only extra thing we
need to do is to maintain the inverse matrix $\hat{D}^{-1}(\mathbf{x}^{\mathrm{old}})$.

If the new position $\mathbf{r}^{\mathrm{new}}$ is accepted, then the
inverse matrix can by suitably updated by an algorithm having a time
scaling of $O(N^2)$.  This algorithm goes as
follows. First we update all but the \emph{i}-th column of $\hat{D}^{-1}$. For each column $j\neq i$, we first calculate the quantity:
\begin{equation}
S_j =
(\hat{D}(\mathbf{r}^{\mathrm{new}})\times
\hat{D}^{-1}(\mathbf{r}^{\mathrm{old}}))_{ij} =
\sum_{l=1}^N d_{il}(\mathbf{r}^{\mathrm{new}})\,
d^{-1}_{lj}(\mathbf{r}^{\mathrm{old}})
\label{eq:inverse_update_1}
\end{equation}

The new elements of the \emph{j}-th column of $\hat{D}^{-1}$ are then given
by:
\begin{equation}
d_{kj}^{-1}(\mathbf{r}^{\mathrm{new}}) =
d_{kj}^{-1}(\mathbf{r}^{\mathrm{old}}) -
\frac{S_j}{R}\,d_{ki}^{-1}(\mathbf{r}^{\mathrm{old}})\quad
\begin{array}{ll}
\forall\ \ k\in\{1,\dots,N\}\\j\neq i
\end{array}
\label{eq:inverse_update_2}
\end{equation}

Finally the elements of the \emph{i}-th column of $\hat{D}^{-1}$ are updated
simply as follows:
\begin{equation}
d_{ki}^{-1}(\mathbf{r}^{\mathrm{new}}) =
\frac{1}{R}\,d_{ki}^{-1}(\mathbf{r}^{\mathrm{old}})\quad
\forall\ \ k\in\{1,\dots,N\}
\label{eq:inverse_update_3}
\end{equation}
We see from these formulas that the time scaling of an update of
$\hat{D}^{-1}$ after changing one row of $\hat{D}$ is $O(N^2)$.

The scheme is also applicable for the calculation of the ratios
involving derivatives. It turns
out that differentiating the Slater determinant with respect
to the coordinates of a single particle  $\mathbf{r}_i$ changes only the
\emph{i}-th row of the corresponding Slater matrix. 

\paragraph{The gradient and the Laplacian.}
The gradient and the Laplacian can therefore be calculated as follows:
\[
\frac{\vec\nabla_i\vert\hat{D}(\mathbf{r})\vert}{\vert\hat{D}(\mathbf{r})\vert} =
\sum_{j=1}^N \vec\nabla_i d_{ij}(\mathbf{r})d_{ji}^{-1}(\mathbf{r}) =
\sum_{j=1}^N \vec\nabla_i \phi_j(\mathbf{r}_i)d_{ji}^{-1}(\mathbf{r})
\]
and
\[
\frac{\nabla^2_i\vert\hat{D}(\mathbf{r})\vert}{\vert\hat{D}(\mathbf{r})\vert} =
\sum_{j=1}^N \nabla^2_i d_{ij}(\mathbf{r})d_{ji}^{-1}(\mathbf{r}) =
\sum_{j=1}^N \nabla^2_i \phi_j(\mathbf{r}_i)\,d_{ji}^{-1}(\mathbf{r})
\]

Thus, to calculate all the derivatives of the Slater determinant, we
only need the derivatives of the single particle  wave functions
($\vec\nabla_i \phi_j(\mathbf{r}_i)$ and $\nabla^2_i \phi_j(\mathbf{r}_i)$)
and the elements of the corresponding inverse Slater matrix ($\hat{D}^{-1}(\mathbf{r}_i)$). A calculation of a single derivative is by the
above result an $O(N)$ operation. Since there are $d\cdot N$
derivatives, the time scaling of the total evaluation becomes
$O(d\cdot N^2)$. With an $O(N^2)$ updating algorithm for the
inverse matrix, the total scaling is no worse, which is far better
than the brute force approach yielding $O(d\cdot N^4)$.

\textbf{Important note}: In most cases you end with closed form expressions for the single-particle  wave functions. It is then useful to calculate the various derivatives and make separate functions
for them.

The Slater determinant takes the form  
\[
   \Phi(\mathbf{r}_1,\mathbf{r}_2,,\mathbf{r}_3,\mathbf{r}_4, \alpha,\beta,\gamma,\delta)=\frac{1}{\sqrt{4!}}
\left| \begin{array}{cccc} \psi_{100\uparrow}(\mathbf{r}_1)& \psi_{100\uparrow}(\mathbf{r}_2)& \psi_{100\uparrow}(\mathbf{r}_3)&\psi_{100\uparrow}(\mathbf{r}_4) \\
\psi_{100\downarrow}(\mathbf{r}_1)& \psi_{100\downarrow}(\mathbf{r}_2)& \psi_{100\downarrow}(\mathbf{r}_3)&\psi_{100\downarrow}(\mathbf{r}_4) \\
\psi_{200\uparrow}(\mathbf{r}_1)& \psi_{200\uparrow}(\mathbf{r}_2)& \psi_{200\uparrow}(\mathbf{r}_3)&\psi_{200\uparrow}(\mathbf{r}_4) \\
\psi_{200\downarrow}(\mathbf{r}_1)& \psi_{200\downarrow}(\mathbf{r}_2)& \psi_{200\downarrow}(\mathbf{r}_3)&\psi_{200\downarrow}(\mathbf{r}_4) \end{array} \right|.
\]
The Slater determinant as written is zero since the spatial wave functions for the spin up and spin down 
states are equal.  
But we can rewrite it as the product of two Slater determinants, one for spin up and one for spin down.

We can rewrite it as 
\[
   \Phi(\mathbf{r}_1,\mathbf{r}_2,,\mathbf{r}_3,\mathbf{r}_4, \alpha,\beta,\gamma,\delta)=\det\uparrow(1,2)\det\downarrow(3,4)-\det\uparrow(1,3)\det\downarrow(2,4)
\]
\[
-\det\uparrow(1,4)\det\downarrow(3,2)+\det\uparrow(2,3)\det\downarrow(1,4)-\det\uparrow(2,4)\det\downarrow(1,3)
\]
\[
+\det\uparrow(3,4)\det\downarrow(1,2),
\]
where we have defined
\[
\det\uparrow(1,2)=\frac{1}{\sqrt{2}}\left| \begin{array}{cc} \psi_{100\uparrow}(\mathbf{r}_1)& \psi_{100\uparrow}(\mathbf{r}_2)\\
\psi_{200\uparrow}(\mathbf{r}_1)& \psi_{200\uparrow}(\mathbf{r}_2) \end{array} \right|,
\]
and 
\[
\det\downarrow(3,4)=\frac{1}{\sqrt{2}}\left| \begin{array}{cc} \psi_{100\downarrow}(\mathbf{r}_3)& \psi_{100\downarrow}(\mathbf{r}_4)\\
\psi_{200\downarrow}(\mathbf{r}_3)& \psi_{200\downarrow}(\mathbf{r}_4) \end{array} \right|.
\]

We want to avoid to sum over spin variables, in particular when the interaction does not depend on spin.

It can be shown, see for example Moskowitz and Kalos, \href{{http://onlinelibrary.wiley.com/doi/10.1002/qua.560200508/abstract}}{Int.~J.~Quantum Chem. \textbf{20} 1107 (1981)}, that for the variational energy
we can approximate the Slater determinant as  
\[
   \Phi(\mathbf{r}_1,\mathbf{r}_2,,\mathbf{r}_3,\mathbf{r}_4, \alpha,\beta,\gamma,\delta) \propto \det\uparrow(1,2)\det\downarrow(3,4),
\]
or more generally as 
\[
   \Phi(\mathbf{r}_1,\mathbf{r}_2,\dots \mathbf{r}_N) \propto \det\uparrow \det\downarrow,
\]
where we have the Slater determinant as the product of a spin up part involving the number of electrons with spin up only (2 for beryllium and 5 for neon) and a spin down part involving the electrons with spin down.

This ansatz is not antisymmetric under the exchange of electrons with  opposite spins but it can be shown (show this) that it gives the same
expectation value for the energy as the full Slater determinant.

As long as the Hamiltonian is spin independent, the above is correct. It is rather straightforward to see this if you go back to the equations for the energy discussed earlier  this semester.

We will thus
factorize the full determinant $\vert\hat{D}\vert$ into two smaller ones, where 
each can be identified with $\uparrow$ and $\downarrow$
respectively:
\[
\vert\hat{D}\vert = \vert\hat{D}\vert_\uparrow\cdot \vert\hat{D}\vert_\downarrow
\]

The combined dimensionality of the two smaller determinants equals the
dimensionality of the full determinant. Such a factorization is
advantageous in that it makes it possible to perform the calculation
of the ratio $R$ and the updating of the inverse matrix separately for
$\vert\hat{D}\vert_\uparrow$ and $\vert\hat{D}\vert_\downarrow$:
\[
\frac{\vert\hat{D}\vert^\mathrm{new}}{\vert\hat{D}\vert^\mathrm{old}} =
\frac{\vert\hat{D}\vert^\mathrm{new}_\uparrow}
{\vert\hat{D}\vert^\mathrm{old}_\uparrow}\cdot
\frac{\vert\hat{D}\vert^\mathrm{new}_\downarrow
}{\vert\hat{D}\vert^\mathrm{old}_\downarrow}
\]

This reduces the calculation time by a constant factor. The maximal
time reduction happens in a system of equal numbers of $\uparrow$ and
$\downarrow$ particles, so that the two factorized determinants are
half the size of the original one.

Consider the case of moving only one particle  at a time which
originally had the following time scaling for one transition:
\[
O_R(N)+O_\mathrm{inverse}(N^2)
\]
For the factorized determinants one of the two determinants is
obviously unaffected by the change so that it cancels from the ratio
$R$. 

Therefore, only one determinant of size $N/2$ is involved in each
calculation of $R$ and update of the inverse matrix. The scaling of
each transition then becomes:
\[
O_R(N/2)+O_\mathrm{inverse}(N^2/4)
\]
and the time scaling when the transitions for all $N$ particles are
put together:
\[
O_R(N^2/2)+O_\mathrm{inverse}(N^3/4)
\]
which gives the same reduction as in the case of moving all particles
at once.

Computing the ratios discussed above requires that we maintain 
the inverse of the Slater matrix evaluated at the current position. 
Each time a trial position is accepted, the row number $i$ of the Slater 
matrix changes and updating its inverse has to be carried out. 
Getting the inverse of an $N \times N$ matrix by Gaussian elimination has a 
complexity of order of $\mathcal{O}(N^3)$ operations, a luxury that we 
cannot afford for each time a particle  move is accepted.
We will use the expression
\begin{equation}
\label{updatingInverse}
d^{-1}_{kj}(\mathbf{x^{new}}) = \left\{\begin{array}{l l}
  d^{-1}_{kj}(\mathbf{x^{old}}) - \frac{d^{-1}_{ki}(\mathbf{x^{old}})}{R} \sum_{l=1}^{N} d_{il}(\mathbf{x^{new}})  d^{-1}_{lj}(\mathbf{x^{old}}) & \mbox{if $j \neq i$}\nonumber \\ \\
 \frac{d^{-1}_{ki}(\mathbf{x^{old}})}{R} \sum_{l=1}^{N} d_{il}(\mathbf{x^{old}}) d^{-1}_{lj}(\mathbf{x^{old}}) & \mbox{if $j=i$}
\end{array} \right.
\end{equation}

This equation scales as $O(N^2)$.
The evaluation of the determinant of an $N \times N$ matrix by standard Gaussian elimination 
requires $\mathbf{O}(N^3)$
calculations. 
As there are $Nd$ independent coordinates we need to evaluate $Nd$ Slater determinants 
for the gradient (quantum force) and $Nd$ for the Laplacian (kinetic energy). 
With the updating algorithm we need only to invert the Slater 
determinant matrix once. This can be done by standard LU decomposition methods.

\paragraph{Expectation value of the kinetic energy.}
The expectation value of the kinetic energy expressed in atomic units for electron $i$ is 
\[
 \langle \hat{K}_i \rangle = -\frac{1}{2}\frac{\langle\Psi|\nabla_{i}^2|\Psi \rangle}{\langle\Psi|\Psi \rangle},
\]
\begin{equation}
\label{kineticE}
K_i = -\frac{1}{2}\frac{\nabla_{i}^{2} \Psi}{\Psi}.
\end{equation}
\begin{align}
\frac{\nabla^2 \Psi}{\Psi} & =  \frac{\nabla^2 ({\Psi_{D} \,  \Psi_C})}{\Psi_{D} \,  \Psi_C} = \frac{\nabla  \cdot [\nabla  {(\Psi_{D} \,  \Psi_C)}]}{\Psi_{D} \,  \Psi_C} = \frac{\nabla  \cdot [ \Psi_C \nabla  \Psi_{D} + \Psi_{D} \nabla   \Psi_C]}{\Psi_{D} \,  \Psi_C}\nonumber\\
&  =  \frac{\nabla   \Psi_C \cdot \nabla  \Psi_{D} +  \Psi_C \nabla^2 \Psi_{D} + \nabla  \Psi_{D} \cdot \nabla   \Psi_C + \Psi_{D} \nabla^2  \Psi_C}{\Psi_{D} \,  \Psi_C}\nonumber\\
\end{align}
\begin{align}
\frac{\nabla^2 \Psi}{\Psi}
& =  \frac{\nabla^2 \Psi_{D}}{\Psi_{D}} + \frac{\nabla^2  \Psi_C}{ \Psi_C} + 2 \frac{\nabla  \Psi_{D}}{\Psi_{D}}\cdot\frac{\nabla   \Psi_C}{ \Psi_C}
\end{align}

The second derivative of the Jastrow factor divided by the Jastrow factor (the way it enters the kinetic energy) is
\[
\left[\frac{\nabla^2 \Psi_C}{\Psi_C}\right]_x =\  
2\sum_{k=1}^{N}
\sum_{i=1}^{k-1}\frac{\partial^2 g_{ik}}{\partial x_k^2}\ +\ 
\sum_{k=1}^N
\left(
\sum_{i=1}^{k-1}\frac{\partial g_{ik}}{\partial x_k} -
\sum_{i=k+1}^{N}\frac{\partial g_{ki}}{\partial x_i}
\right)^2
\]

But we have a simple form for the function, namely
\[
\Psi_{C}=\prod_{i< j}\exp{f(r_{ij})}= \exp{\left\{\sum_{i<j}\frac{ar_{ij}}{1+\beta r_{ij}}\right\}},
\]
and it is easy to see that for particle  $k$
we have
\[
  \frac{\nabla^2_k \Psi_C}{\Psi_C }=
\sum_{ij\ne k}\frac{(\mathbf{r}_k-\mathbf{r}_i)(\mathbf{r}_k-\mathbf{r}_j)}{r_{ki}r_{kj}}f'(r_{ki})f'(r_{kj})+
\sum_{j\ne k}\left( f''(r_{kj})+\frac{2}{r_{kj}}f'(r_{kj})\right)
\]

Using 
\[
f(r_{ij})= \frac{ar_{ij}}{1+\beta r_{ij}},
\]
and $g'(r_{kj})=dg(r_{kj})/dr_{kj}$ and 
$g''(r_{kj})=d^2g(r_{kj})/dr_{kj}^2$  we find that for particle  $k$
we have
\[
  \frac{\nabla^2_k \Psi_C}{\Psi_C }=
\sum_{ij\ne k}\frac{(\mathbf{r}_k-\mathbf{r}_i)(\mathbf{r}_k-\mathbf{r}_j)}{r_{ki}r_{kj}}\frac{a}{(1+\beta r_{ki})^2}
\frac{a}{(1+\beta r_{kj})^2}+
\sum_{j\ne k}\left(\frac{2a}{r_{kj}(1+\beta r_{kj})^2}-\frac{2a\beta}{(1+\beta r_{kj})^3}\right)
\]

The gradient and
Laplacian can be calculated as follows:
\[
\frac{\mathbf{\nabla}_i\vert\hat{D}(\mathbf{r})\vert}
{\vert\hat{D}(\mathbf{r})\vert} =
\sum_{j=1}^N \vec\nabla_i d_{ij}(\mathbf{r})\,
d_{ji}^{-1}(\mathbf{r}) =
\sum_{j=1}^N \vec\nabla_i \phi_j(\mathbf{r}_i)\,
d_{ji}^{-1}(\mathbf{r})
\]
and
\[
\frac{\nabla^2_i\vert\hat{D}(\mathbf{r})\vert}
{\vert\hat{D}(\mathbf{r})\vert} =
\sum_{j=1}^N \nabla^2_i d_{ij}(\mathbf{r})\,
d_{ji}^{-1}(\mathbf{r}) =
\sum_{j=1}^N \nabla^2_i \phi_j(\mathbf{r}_i)\,
d_{ji}^{-1}(\mathbf{r})
\]

The gradient for the determinant is 
\[
\frac{\mathbf{\nabla}_i\vert\hat{D}(\mathbf{r})\vert}
{\vert\hat{D}(\mathbf{r})\vert} =
\sum_{j=1}^N \mathbf{\nabla}_i d_{ij}(\mathbf{r})\,
d_{ji}^{-1}(\mathbf{r}) =
\sum_{j=1}^N \mathbf{\nabla}_i \phi_j(\mathbf{r}_i)\,
d_{ji}^{-1}(\mathbf{r}).
\]

We have
\[
\Psi_C=\prod_{i< j}g(r_{ij})= \exp{\left\{\sum_{i<j}\frac{ar_{ij}}{1+\beta r_{ij}}\right\}},
\]
the gradient needed for the quantum force and local energy is easy to compute.  
We get for particle  $k$
\[
\frac{ \nabla_k \Psi_C}{ \Psi_C }= \sum_{j\ne k}\frac{\mathbf{r}_{kj}}{r_{kj}}\frac{a}{(1+\beta r_{kj})^2},
\]
which is rather easy to code.  Remember to sum over all particles  when you compute the local energy.

We need to compute the ratio between wave functions, in particular  for the Slater determinants.
\[
R =\sum_{j=1}^N d_{ij}(\mathbf{r}^{\mathrm{new}})\,
d_{ji}^{-1}(\mathbf{r}^{\mathrm{old}}) = 
\sum_{j=1}^N \phi_j(\mathbf{r}_i^{\mathrm{new}})\,
d_{ji}^{-1}(\mathbf{r}^{\mathrm{old}})
\]
What this means is that in order to get the ratio when only the \emph{i}-th
particle  has been moved, we only need to calculate the dot
product of the vector $\left(\phi_1(\mathbf{r}_i^\mathrm{new}),\,\dots,\,
\phi_N(\mathbf{r}_i^\mathrm{new})\right)$ of single particle  wave functions
evaluated at this new position with the \emph{i}-th column of the inverse
matrix $\hat{D}^{-1}$ evaluated at the original position. Such
an operation has a time scaling of $O(N)$. The only extra thing we
need to do is to maintain the inverse matrix 
$\hat{D}^{-1}(\mathbf{x}^{\mathrm{old}})$.


        \part{Machine Learning}
         \part{Quantum Computing}   

 \bibliographystyle{unsrt}

\backmatter%%%%%%%%%%%%%%%%%%%%%%%%%%%%%%%%%%%%%%%%%%%%%%%%%%%%%%%
%\include{glossary}
%\include{solutions}
\printindex



\end{document}





