%%%%%%%%%%%%%%%%%%%% book.tex %%%%%%%%%%%%%%%%%%%%%%%%%%%%%
%
% sample root file for the chapters of your "monograph"
%
% Use this file as a template for your own input.
%
%%%%%%%%%%%%%%%% Springer-Verlag %%%%%%%%%%%%%%%%%%%%%%%%%%


% RECOMMENDED %%%%%%%%%%%%%%%%%%%%%%%%%%%%%%%%%%%%%%%%%%%%%%%%%%%
\documentclass[graybox,envcountchap,sectrefs]{svmono}

% choose options for [] as required from the list
% in the Reference Guide

\usepackage{mathptmx}
\usepackage{helvet}
\usepackage{courier}
%
\usepackage{type1cm}         

\usepackage{makeidx}         % allows index generation
\usepackage{graphicx}        % standard LaTeX graphics tool
                             % when including figure files
\usepackage{multicol}        % used for the two-column index
\usepackage[bottom]{footmisc}% places footnotes at page bottom

% see the list of further useful packages
% in the Reference Guide

\makeindex             % used for the subject index
                       % please use the style svind.ist with
                       % your makeindex program
\usepackage[usenames,dvipsnames,x11names]{xcolor}
\usepackage{tikz}
\usetikzlibrary{arrows,snakes,shapes}

 \usepackage{listings}
 \usepackage{graphicx}
 \usepackage{epic}
 \usepackage{eepic}
 \usepackage{a4wide}
 \usepackage{color}
 \usepackage{amsmath}
 \usepackage{amssymb}
 \usepackage[dvips]{epsfig}
 \usepackage{psfig}
 \usepackage[T1]{fontenc}
 \usepackage{cite} % [2,3,4] --> [2--4]
 \usepackage{shadow}
 \usepackage{hyperref}
 \usepackage{bezier}
 \usepackage{pstricks}
 %\usepackage{refcheck}
 \setcounter{tocdepth}{2}
%\usepackage{gnuplot-lua-tikz}


\usepackage{textcomp,type1ec,pdfpages}
\usepackage{bera}

\definecolor{dkgreen}{rgb}{0,0.6,0}
\definecolor{gray}{rgb}{0.5,0.5,0.5}
\definecolor{mauve}{rgb}{0.58,0,0.82}

 \lstset{language=c++}
 \lstset{alsolanguage=[90]Fortran}
 \lstset{alsolanguage=python}
% \lstset{basicstyle=\small}
 \lstset{backgroundcolor=\color{white}}
 \lstset{frame=single}
 \lstset{stringstyle=\ttfamily}
 \lstset{keywordstyle=\color{red}\bfseries}
 \lstset{commentstyle=\itshape\color{blue}}
 \lstset{showspaces=false}
 \lstset{showstringspaces=false}
 \lstset{showtabs=false}
 \lstset{breaklines}
 

% Default settings for code listings
% \lstnewenvironment{Python}[1]{
\lstset{%frame=tb,
  language=c++,
  alsolanguage=python,
  %aboveskip=3mm,
 % belowskip=3mm,
  showstringspaces=false,
  columns=flexible,
  basicstyle={\footnotesize\ttfamily},
  numbers=none,
  numberstyle=\tiny\color{gray},
  commentstyle=\color{dkgreen},
  stringstyle=\color{mauve},
  frame=single,  
  breaklines=true,
  %%%% FOR PYTHON 
  otherkeywords={\ , \}, \{},
  keywordstyle=\color{blue},
  emph={void, ||, &&, break, class,continue, delete, else,
  for, if, include, return,try,while},
  emphstyle=\color{black}\bfseries,
  emph={[2]True, False, None, self},
  emphstyle=[2]\color{dkgreen},
  emphstyle=[2]\color{red},
  emph={[3]from, import, as},
  emphstyle=[3]\color{blue},
  upquote=true,
  morecomment=[s]{"""}{"""},
  commentstyle=\color{green}\slshape, %%% cambie gray por green
  emph={[4]1, 2, 3, 4, 5, 6, 7, 8, 9, 0},
  emphstyle=[4]\color{blue},
  breakatwhitespace=true,
  tabsize=2
}

\renewcommand{\lstlistlistingname}{Code Listings}
\renewcommand{\lstlistingname}{Code Listing}
\definecolor{gray}{gray}{0.5}
\definecolor{green}{rgb}{0,0.5,0}

\lstnewenvironment{Python}[1]{
\lstset{
language=python,
basicstyle=\footnotesize\setstretch{1},
stringstyle=\color{red},
showstringspaces=false,
alsoletter={1234567890},
otherkeywords={\ , \}, \{},
keywordstyle=\color{blue},
emph={access,and,break,class,continue,def,del,elif ,else,%
except,exec,finally,for,from,global,if,import,in,is,%
lambda,not,or,pass,print,raise,return,try,while},
emphstyle=\color{black}\bfseries,
emph={[2]True, False, None, self},
emphstyle=[2]\color{red},
emph={[3]from, import, as},
emphstyle=[3]\color{blue},
upquote=true,
morecomment=[s]{"""}{"""},
commentstyle=\color{dkgreen}\slshape, % el color era gray pero lo cambie a verde
emph={[4]1, 2, 3, 4, 5, 6, 7, 8, 9, 0},
emphstyle=[4]\color{blue},
framexleftmargin=1mm, framextopmargin=1mm, rulesepcolor=\color{blue},
breakatwhitespace=true,
tabsize=2
}}{}


\lstnewenvironment{C++}[1]{
\lstset{
language=c++,
% basicstyle=\ttfamily\small\setstretch{1},
basicstyle=\footnotesize\setstretch{1},
stringstyle=\color{red},
showstringspaces=false,
alsoletter={1234567890},
otherkeywords={\ , \}, \{},
keywordstyle=\color{blue},
emph={access,and,break,class,continue,def,del,elif ,else,%
except,exec,finally,for,from,global,if,import,in,is,%
lambda,not,or,pass,print,raise,return,try,while},
emphstyle=\color{black}\bfseries,
emph={[2]True, False, None, self},
emphstyle=[2]\color{red},
emph={[3]from, import, as},
emphstyle=[3]\color{blue},
upquote=true,
morecomment=[s]{"""}{"""},
commentstyle=\color{dkgreen}\slshape, % el color era gray pero lo cambie a verde
emph={[4]1, 2, 3, 4, 5, 6, 7, 8, 9, 0},
emphstyle=[4]\color{blue},
% literate=*{:}{{\textcolor{blue}:}}{1}%
% {=}{{\textcolor{blue}=}}{1}%
% {-}{{\textcolor{blue}-}}{1}%
% {+}{{\textcolor{blue}+}}{1}%
% {*}{{\textcolor{blue}*}}{1}%
% {!}{{\textcolor{blue}!}}{1}%
% {(}{{\textcolor{blue}(}}{1}%
% {)}{{\textcolor{blue})}}{1}%
% {[}{{\textcolor{blue}[}}{1}%
% {]}{{\textcolor{blue}]}}{1}%
% {<}{{\textcolor{blue}<}}{1}%
% {>}{{\textcolor{blue}>}}{1},%
framexleftmargin=1mm, framextopmargin=1mm, rulesepcolor=\color{blue},
breakatwhitespace=true,
tabsize=2
}}{}




\usepackage{tikz}
\usetikzlibrary{shapes,arrows}

% Define block styles
\tikzstyle{decision} = [diamond, draw, fill=blue!20,
    text width=3.5em, text badly centered, node distance=2.5cm, inner sep=0pt]
\tikzstyle{block} = [rectangle, draw, fill=blue!20,
    text width=8em, text centered, rounded corners, minimum height=4em]
\tikzstyle{line} = [draw, very thick, color=black!50, -latex']
\tikzstyle{cloud} = [draw, ellipse,fill=red!20, node distance=2.5cm,
    minimum height=2em]

\def\radius{.7mm} 
\tikzstyle{branch}=[fill,shape=circle,minimum size=3pt,inner sep=0pt]


\newcommand{\bfv}[1]{\boldsymbol{#1}} 
\newcommand{\Div}[1]{\nabla \bullet \vbf{#1}}           % define divergence
\newcommand{\Grad}[1]{\boldsymbol{\nabla}{#1}}
 \newcommand{\OP}[1]{{\bf\widehat{#1}}}
 \newcommand{\be}{\begin{equation}}
 \newcommand{\ee}{\end{equation}}
\newcommand{\beN}{\begin{equation*}}
\newcommand{\bea}{\begin{eqnarray}}
\newcommand{\beaN}{\begin{eqnarray*}}
\newcommand{\eeN}{\end{equation*}}
\newcommand{\eea}{\end{eqnarray}}
\newcommand{\eeaN}{\end{eqnarray*}}
\newcommand{\bdm}{\begin{displaymath}}
\newcommand{\edm}{\end{displaymath}}
\newcommand{\bsubeqs}{\begin{subequations}}
\newcommand{\esubeqs}{\end{subequations}}
\newcommand{\Obs}[1]{\langle{\Op{#1}\rangle}}             % define observable
\newcommand{\PsiT}{\bfv{\Psi_T}(\bfv{R})}                       % symbol for trial wave function
%\newcommand{\braket}[2]{\langle{#1}|\Op{#2}|{#1}\rangle}
\newcommand{\Det}[1]{{|\bfv{#1}|}}
\newcommand{\uvec}[1]{\mbox{\boldmath$\hat{#1}$\unboldmath}}
\newcommand{\Op}[1]{{\bf\widehat{#1}}}    
\newcommand{\eqbrace}[4]{\left\{
\begin{array}{ll}
#1 & #2 \\[0.5cm]
#3 & #4
\end{array}\right.}
\newcommand{\eqbraced}[4]{\left\{
\begin{array}{ll}
#1 & #2 \\[0.5cm]
#3 & #4
\end{array}\right\}}
\newcommand{\eqbracetriple}[6]{\left\{
\begin{array}{ll}
#1 & #2 \\
#3 & #4 \\
#5 & #6
\end{array}\right.}
\newcommand{\eqbracedtriple}[6]{\left\{
\begin{array}{ll}
#1 & #2 \\
#3 & #4 \\
#5 & #6
\end{array}\right\}}

\newcommand{\mybox}[3]{\mbox{\makebox[#1][#2]{$#3$}}}
\newcommand{\myframedbox}[3]{\mbox{\framebox[#1][#2]{$#3$}}}

%% Infinitesimal (and double infinitesimal), useful at end of integrals
%\newcommand{\ud}[1]{\mathrm d#1}
\newcommand{\ud}[1]{d#1}
\newcommand{\udd}[1]{d^2\!#1}

%% Operators, algebraic matrices, algebraic vectors

%% Operator (hat, bold or bold symbol, whichever you like best):
\newcommand{\op}[1]{\widehat{#1}}
%\newcommand{\op}[1]{\mathbf{#1}}
%\newcommand{\op}[1]{\boldsymbol{#1}}

%% Vector:
\renewcommand{\vec}[1]{\boldsymbol{#1}}

%% Matrix symbol:
\newcommand{\matr}[1]{\boldsymbol{#1}}
%\newcommand{\bb}[1]{\mathbb{#1}}

%% Determinant symbol:
\renewcommand{\det}[1]{|#1|}

%% Means (expectation values) of varius sizes
\newcommand{\mean}[1]{\langle #1 \rangle}
\newcommand{\meanb}[1]{\big\langle #1 \big\rangle}
\newcommand{\meanbb}[1]{\Big\langle #1 \Big\rangle}
\newcommand{\meanbbb}[1]{\bigg\langle #1 \bigg\rangle}
\newcommand{\meanbbbb}[1]{\Bigg\langle #1 \Bigg\rangle}

%% Shorthands for text set in roman font
\newcommand{\prob}[0]{\mathrm{Prob}} %probability
\newcommand{\cov}[0]{\mathrm{Cov}}   %covariance
\newcommand{\var}[0]{\mathrm{Var}}   %variancd

%% Big-O (typically for specifying the speed scaling of an algorithm)
\newcommand{\bigO}{\mathcal{O}}

%% Real value of a complex number
\newcommand{\real}[1]{\mathrm{Re}\!\left\{#1\right\}}

%% Quantum mechanical state vectors and matrix elements (of different sizes)
\newcommand{\brab}[1]{\big\langle #1 \big|}
\newcommand{\brabb}[1]{\Big\langle #1 \Big|}
\newcommand{\brabbb}[1]{\bigg\langle #1 \bigg|}
\newcommand{\brabbbb}[1]{\Bigg\langle #1 \Bigg|}
\newcommand{\ketb}[1]{\big| #1 \big\rangle}
\newcommand{\ketbb}[1]{\Big| #1 \Big\rangle}
\newcommand{\ketbbb}[1]{\bigg| #1 \bigg\rangle}
\newcommand{\ketbbbb}[1]{\Bigg| #1 \Bigg\rangle}
\newcommand{\overlap}[2]{\langle #1 | #2 \rangle}
\newcommand{\overlapb}[2]{\big\langle #1 \big| #2 \big\rangle}
\newcommand{\overlapbb}[2]{\Big\langle #1 \Big| #2 \Big\rangle}
\newcommand{\overlapbbb}[2]{\bigg\langle #1 \bigg| #2 \bigg\rangle}
\newcommand{\overlapbbbb}[2]{\Bigg\langle #1 \Bigg| #2 \Bigg\rangle}
\newcommand{\bracket}[3]{\langle #1 | #2 | #3 \rangle}
\newcommand{\bracketb}[3]{\big\langle #1 \big| #2 \big| #3 \big\rangle}
\newcommand{\bracketbb}[3]{\Big\langle #1 \Big| #2 \Big| #3 \Big\rangle}
\newcommand{\bracketbbb}[3]{\bigg\langle #1 \bigg| #2 \bigg| #3 \bigg\rangle}
\newcommand{\bracketbbbb}[3]{\Bigg\langle #1 \Bigg| #2 \Bigg| #3 \Bigg\rangle}
\newcommand{\projection}[2]
{| #1 \rangle \langle  #2 |}
\newcommand{\projectionb}[2]
{\big| #1 \big\rangle \big\langle #2 \big|}
\newcommand{\projectionbb}[2]
{ \Big| #1 \Big\rangle \Big\langle #2 \Big|}
\newcommand{\projectionbbb}[2]
{ \bigg| #1 \bigg\rangle \bigg\langle #2 \bigg|}
\newcommand{\projectionbbbb}[2]
{ \Bigg| #1 \Bigg\rangle \Bigg\langle #2 \Bigg|}


%\proton{xposition,yposition}
\newcommand{\proton}[1]{%
    \shade[ball color=red] (#1) circle (.25);\draw (#1) node{$+$};
}

%\neutron{xposition,yposition}
\newcommand{\neutron}[1]{%
    \shade[ball color=green] (#1) circle (.25);
}

%\electron{xwidth,ywidth,rotation angle}
\newcommand{\electron}[3]{%
    \draw[rotate = #3](0,0) ellipse (#1 and #2)[color=blue];
    \shade[ball color=Gold2] (0,#2)[rotate=#3] circle (.1);
}

\newcommand{\nucleus}{%
    \neutron{0.1,0.3}
    \proton{0,0}
    \neutron{0.3,0.2}
    \proton{-0.2,0.1}
    \neutron{-0.1,0.3}
    \proton{0.2,-0.15}
    \neutron{-0.05,-0.12}
    \proton{0.17,0.21}
}

%\photoelectron{xwidth,ywidth,rotation angle}
\newcommand{\photoelectron}[3]{%
    \draw[rotate = #3](0,0) ellipse (#1 and #2)[color=blue];%
    \draw[snake=coil,%
        line after snake=0pt, segment aspect=0,%
        segment length=20pt,color=red!50!blue](#3:#1)-- +(-6,0)%
        node[fill=white!70!Gold2,draw=red!80!white, above=0.2cm,pos=0.5]%
            {Incoming $\gamma$-photon};%
    \draw[-stealth,Gold2](#3:#1) -- ++ (5,0.625);%
    \shade[ball color=Gold2](#3:#1)  --  ++(4,0.5)%
        node[fill=white!70!Gold2,draw=red!80!white,%
        text width=3cm, below right=0.2cm]%
            {Photoelectron from an inner shell} circle(0.1);%
    \fill  (#1,0)[rotate=#3,color=white,opaque] circle (.1);%
    \draw  (#1,0)[rotate=#3,color=Gold2] circle (.1) ;%
}

%\comptonelectron{xwidth,ywidth,rotation angle}
\newcommand{\comptonelectron}[3]{%
    \draw[rotate = #3](0,0) ellipse (#1 and #2)[color=blue];%
    \draw[snake=coil, line after snake=0pt,%
        segment aspect=0, segment length=10pt,color=red!50!blue]%
        (#3:#1)-- +(-6,0)%
        node[fill=white!70!Gold2,draw=red!80!white, above=0.2cm,pos=0.5]%
            {Incoming $\gamma$-photon};%
    \draw[-stealth,Gold2](#3:#1) -- ++ (5,2.5);%
    \shade[ball color=Gold2](#3:#1)  --  ++(4,2.0)%
        node[fill=white!70!Gold2,draw=red!80!white, text width=3cm,%
        below right=0.2cm]{Scattered electron from an outer shell} circle(0.1);%
    \fill  (#1,0)[rotate=#3,color=white,opaque] circle (.1);%
    \draw  (#1,0)[rotate=#3,color=Gold2] circle (.1) ;%
    \draw[snake=coil, line after snake=1mm, segment aspect=0,%
        segment length=15pt,color=red!50!blue,-stealth] (#3:#1)-- ++(6,-3)%
        node[fill=white!70!Gold2,draw=red!80!white, right=1cm,pos=0.5]%
            {Scattered $\gamma$-photon};%
}

%\paircreation{impact parameter}
\newcommand{\paircreation}[1]{%
    \draw[snake=coil, line after snake=0pt, segment aspect=0,%
        segment length=5pt,color=red!50!blue] (0,#1)-- +(-6,0)%
        node[fill=white!70!Gold2,draw=red!80!white, above=0.2cm,pos=0.5]%
            {Incoming $\gamma$-photon};%
    \draw[-stealth,Gold2](0,#1) -- ++ (5,2.5);%
    \shade[ball color=Gold2](0,#1)  --  ++(4,2.0)%
        node[fill=white!70!Gold2,draw=red!80!white, below right=0.2cm]%
            {Positron} circle(0.1);%
    \draw[-stealth,Gold2](0,#1) -- ++ (4,-2.0);%
    \shade[ball color=Gold2](0,#1)  --  ++(3.2,-1.6)%
        node[fill=white!70!Gold2!,draw=red!80!white, above right=0.2cm]%
            {Electron} circle(0.1);%
}





%%%%%%%%%%%%%%%%%%%%%%%%%%%%%%%%%%%%%%%%%%%%%%%%%%%%%%%%%%%%%%%%%%%%%

\begin{document}

\author{Morten Hjorth-Jensen and Anders Malthe-S\o renssen}
\title{Computational Physics}
%\subtitle{-- Monograph --}
\maketitle

\frontmatter%%%%%%%%%%%%%%%%%%%%%%%%%%%%%%%%%%%%%%%%%%%%%%%%%%%%%%


%%%%%%%%%%%%%%%%%%%%%%% dedic.tex %%%%%%%%%%%%%%%%%%%%%%%%%%%%%%%%%
%
% sample dedication
%
% Use this file as a template for your own input.
%
%%%%%%%%%%%%%%%%%%%%%%%% Springer %%%%%%%%%%%%%%%%%%%%%%%%%%

\begin{dedication}
Use the template \emph{dedic.tex} together with the Springer document class SVMono for monograph-type books or SVMult for contributed volumes to style a quotation or a dedication\index{dedication} at the very beginning of your book in the Springer layout
\end{dedication}





\include{foreword}
\preface
%  last update : 24/8/2013  mhj

\begin{quotation}
So, ultimately, in order to understand nature it may be necessary to
have a deeper understanding of mathematical relationships. But the
real reason is that the subject is enjoyable, and although we humans
cut nature up in different ways, and we have different courses in
different departments, such compartmentalization is really artificial,
and we should take our intellectual pleasures where we find them. 
{\em Richard Feynman, The Laws of Thermodynamics.}
\end{quotation}

Why a preface you may ask? Isn't that just a mere exposition of a
raison d'$\mathrm{\hat{e}}$tre of an author's choice of material,
preferences, biases, teaching philosophy etc.?  To a large extent I
can answer in the affirmative to that. A preface ought to be personal.
Indeed, what you will see in the various chapters of these notes
represents how I perceive computational physics should be taught.

 This set of lecture notes serves the scope of presenting to you and
train you in an algorithmic approach to problems in the sciences,
represented here by the unity of three disciplines, physics,
mathematics and informatics. This trinity outlines the emerging field
of computational physics.

Our insight in a physical system, combined with numerical mathematics
gives us the rules for setting up an algorithm, viz.~a set of rules
for solving a particular problem.  Our understanding of the physical
system under study is obviously gauged by the natural laws at play,
the initial conditions, boundary conditions and other external
constraints which influence the given system. Having spelled out the
physics, for example in the form of a set of coupled partial
differential equations, we need efficient numerical methods in order
to set up the final algorithm.  This algorithm is in turn coded into a
computer program and executed on available computing facilities.  To
develop such an algorithmic approach, you will be exposed to several
physics cases, spanning from the classical pendulum to quantum
mechanical systems. We will also present some of the most popular
algorithms from numerical mathematics used to solve a plethora of
problems in the sciences.  Finally we will codify these algorithms
using some of the most widely used programming languages, presently C,
C++ and Fortran and its most recent standard Fortran
2008\footnote{Throughout this text we refer to Fortran 2008 as
Fortran, implying the latest standard.}. However, a high-level and fully
object-oriented language like Python is now emerging as a good
alternative although C++ and Fortran still outperform Python when it
comes to computational speed.  In this text we offer an approach where
one can write all programs in C/C++ or Fortran.  We will also show you
how to develop large programs in Python interfacing C++ and/or Fortran
functions for those parts of the program which are CPU intensive.
Such an approach allows you to structure the flow of data in a
high-level language like Python while tasks of a mere repetitive and
CPU intensive nature are left to low-level languages like C++ or
Fortran. Python allows you also to smoothly interface your program
with other software, such as plotting programs or operating system
instructions. A typical Python program you may end up writing contains
everything from compiling and running your codes to preparing the body
of a file for writing up your report.



Computer simulations are nowadays an integral part of contemporary
basic and applied research in the sciences.  Computation is becoming
as important as theory and experiment. In physics, computational
physics, theoretical physics and experimental physics are all equally
important in our daily research and studies of physical
systems. Physics is the unity of theory, experiment and
computation\footnote{We mentioned previously the trinity of physics,
mathematics and informatics. Viewing physics as the trinity of theory,
experiment and simulations is yet another example. It is obviously
tempting to go beyond the sciences. History shows that triunes,
trinities and for example triple deities permeate the Indo-European
cultures (and probably all human cultures), from the ancient Celts and
Hindus to modern days.  The ancient Celts revered many such trinues,
their world was divided into earth, sea and air, nature was divided in
animal, vegetable and mineral and the cardinal colours were red,
yellow and blue, just to mention a few.  As a curious digression, it
was a Gaulish Celt, Hilary, philosopher and bishop of Poitiers (AD
315-367) in his work {\em De Trinitate} who formulated the Holy
Trinity concept of Christianity, perhaps in order to accomodate
millenia of human divination practice.}.  Moreover, the ability "to
compute" forms part of the essential repertoire of research
scientists. Several new fields within computational science have
emerged and strengthened their positions in the last years, such as
computational materials science, bioinformatics, computational
mathematics and mechanics, computational chemistry and physics and so
forth, just to mention a few.  These fields underscore the importance
of simulations as a means to gain novel insights into physical
systems, especially for those cases where no analytical solutions can
be found or an experiment is too complicated or expensive to carry
out.  To be able to simulate large quantal systems with many degrees
of freedom such as strongly interacting electrons in a quantum dot
will be of great importance for future directions in novel fields like
nano-techonology.  This ability often combines knowledge from many
different subjects, in our case essentially from the physical
sciences, numerical mathematics, computing languages, topics from
high-performace computing and some knowledge of computers.


In 1999, when I started this course at the department of physics in
Oslo, computational physics and computational science in general were
still perceived by the majority of physicists and scientists as topics
dealing with just mere tools and number crunching, and not as subjects
of their own.  The computational background of most students enlisting
for the course on computational physics could span from dedicated
hackers and computer freaks to people who basically had never used a
PC. The majority of undergraduate and graduate students had a very
rudimentary knowledge of computational techniques and methods.
Questions like 'do you know of better methods for numerical
integration than the trapezoidal rule' were not uncommon. I do happen
to know of colleagues who applied for time at a supercomputing centre
because they needed to invert matrices of the size of $10^4\times
10^4$ since they were using the trapezoidal rule to compute
integrals. With Gaussian quadrature this dimensionality was easily
reduced to matrix problems of the size of $10^2\times 10^2$, with much
better precision.

More than a decade later most students have now been exposed to a
fairly uniform introduction to computers, basic programming skills and
use of numerical exercises.  Practically every undergraduate student
in physics has now made a Matlab or Maple simulation of for example
the pendulum, with or without chaotic motion.  Nowadays most of you
are familiar, through various undergraduate courses in physics and
mathematics, with interpreted languages such as Maple, Matlab and/or
Mathematica. In addition, the interest in scripting languages such as
Python or Perl has increased considerably in recent years.  The modern
programmer would typically combine several tools, computing
environments and programming languages. A typical example is the
following. Suppose you are working on a project which demands
extensive visualizations of the results. To obtain these results, that
is to solve a physics problems like obtaining the density profile of a
Bose-Einstein condensate, you need however a program which is fairly
fast when computational speed matters.  In this case you would most
likely write a high-performance computing program using Monte Carlo
methods in languages which are tailored for that. These are
represented by programming languages like Fortran and C++.  However,
to visualize the results you would find interpreted languages like
Matlab or scripting languages like Python extremely suitable for your
tasks.  You will therefore end up writing for example a script in
Matlab which calls a Fortran or C++ program where the number crunching
is done and then visualize the results of say a wave equation solver
via Matlab's large library of visualization tools. Alternatively, you
could organize everything into a Python or Perl script which does
everything for you, calls the Fortran and/or C++ programs and performs
the visualization in Matlab or Python. Used correctly, these tools,
spanning from scripting languages to high-performance computing
languages and vizualization programs, speed up your capability to
solve complicated problems.  Being multilingual is thus an advantage
which not only applies to our globalized modern society but to
computing environments as well.  This text shows you how to use C++
and Fortran as programming languages.

There is however more to the picture than meets the eye.  Although
interpreted languages like Matlab, Mathematica and Maple allow you
nowadays to solve very complicated problems, and high-level languages
like Python can be used to solve computational problems, computational
speed and the capability to write an efficient code are topics which
still do matter. To this end, the majority of scientists still use
languages like C++ and Fortran to solve scientific problems.  When you
embark on a master or PhD thesis, you will most likely meet these
high-performance computing languages.  This course emphasizes thus the
use of programming languages like Fortran, Python and C++ instead of
interpreted ones like Matlab or Maple. You should however note that
there are still large differences in computer time between for example
numerical Python and a corresponding C++ program for many numerical
applications in the physical sciences, with a code in C++ or Fortran
being the fastest.

Computational speed is not the only reason for this choice of
programming languages. Another important reason is that we feel that
at a certain stage one needs to have some insights into the algorithm
used, its stability conditions, possible pitfalls like loss of
precision, ranges of applicability, the possibility to improve the
algorithm and taylor it to special purposes etc etc.  One of our major
aims here is to present to you what we would dub 'the algorithmic
approach', a set of rules for doing mathematics or a precise
description of how to solve a problem. To device an algorithm and
thereafter write a code for solving physics problems is a marvelous
way of gaining insight into complicated physical systems. The
algorithm you end up writing reflects in essentially all cases your
own understanding of the physics and the mathematics (the way you
express yourself) of the problem.  We do therefore devote quite some
space to the algorithms behind various functions presented in the
text. Especially, insight into how errors propagate and how to avoid
them is a topic we would like you to pay special attention to. Only
then can you avoid problems like underflow, overflow and loss of
precision. Such a control is not always achievable with interpreted
languages and canned functions where the underlying algorithm and/or
code is not easily accesible.  Although we will at various stages
recommend the use of library routines for say linear
algebra\footnote{Such library functions are often taylored to a given
machine's architecture and should accordingly run faster than user
provided ones.}, our belief is that one should understand what the
given function does, at least to have a mere idea.  With such a
starting point, we strongly believe that it can be easier to develope
more complicated programs on your own using Fortran, C++ or Python.

We have several other aims as well, namely:
\begin{itemize}
\item We would like to give you  an opportunity to gain a 
      deeper understanding of the physics you have learned in other
      courses. In most courses one is normally confronted with simple
      systems which provide exact solutions and mimic to a certain
      extent the realistic cases. Many are however the comments like
      'why can't we do something else than the particle in a box
      potential?'.  In several of the projects we hope to present some
      more 'realistic' cases to solve by various numerical
      methods. This also means that we wish to give examples of how
      physics can be applied in a much broader context than it is
      discussed in the traditional physics undergraduate curriculum.
\item To encourage you to "discover" physics in a way similar to how 
researchers learn in the context of research.
\item Hopefully also to introduce numerical methods and new areas of physics that 
      can be studied with the methods discussed.
\item To teach   structured programming in the context of doing science. 
\item The projects we propose are meant to mimic to a certain extent 
      the situation encountered during a thesis or project work. You
      will tipically have at your disposal 2-3 weeks to solve
      numerically a given project. In so doing you may need to do a
      literature study as well. Finally, we would like you to write a
      report for every project.
\end{itemize}
Our overall goal is to encourage you to learn about science through
experience and by asking questions. Our objective is always
understanding and the purpose of computing is further insight, not
mere numbers!  Simulations can often be considered as
experiments. Rerunning a simulation need not be as costly as rerunning
an experiment.


 
Needless to say, these lecture notes are upgraded continuously, from
typos to new input.  And we do always benefit from your comments,
suggestions and ideas for making these notes better.  It's through the
scientific discourse and critics we advance.  Moreover, I have
benefitted immensely from many discussions with fellow colleagues and
students. In particular I must mention Hans Petter Langtangen, Anders
Malthe-S\o renssen, Knut M\o rken and \O yvind Ryan, whose input
during the last fifteen years has considerably improved these lecture
notes.  Furthermore, the time we have spent and keep spending together
on the Computing in Science Education project at the University, is
just marvelous. Thanks so much. Concerning the Computing in Science
Education initiative, you can read more
at \url{http://www.mn.uio.no/english/about/collaboration/cse/}.


Finally, I would like to add a petit note on referencing. These notes
have evolved over many years and the idea is that they should end up
in the format of a web-based learning environment for doing
computational science. It will be fully free and hopefully represent a
much more efficient way of conveying teaching material than
traditional textbooks.  I have not yet settled on a specific format,
so any input is welcome. At present however, it is very easy for me to
upgrade and improve the material on say a yearly basis, from simple
typos to adding new material.  When accessing the web page of the
course, you will have noticed that you can obtain all source files for
the programs discussed in the text.  Many people have thus written to
me about how they should properly reference this material and whether
they can freely use it. My answer is rather simple.  You are
encouraged to use these codes, modify them, include them in
publications, thesis work, your lectures etc.  As long as your use is
part of the dialectics of science you can use this material freely.
However, since many weekends have elapsed in writing several of these
programs, testing them, sweating over bugs, swearing in front of a
f*@?\%g code which didn't compile properly ten minutes before monday
morning's eight o'clock lecture etc etc, I would dearly appreciate in
case you find these codes of any use, to reference them properly. That
can be done in a simple way, refer to M.~Hjorth-Jensen, {\em
Computational Physics}, University of Oslo (2013). The weblink to the
course should also be included. Hope it is not too much to ask
for. Enjoy!

%%%%%%%%%%%%%%%%%%%%%%acknow.tex%%%%%%%%%%%%%%%%%%%%%%%%%%%%%%%%%%%%%%%%%
% sample acknowledgement chapter
%
% Use this file as a template for your own input.
%
%%%%%%%%%%%%%%%%%%%%%%%% Springer %%%%%%%%%%%%%%%%%%%%%%%%%%

\extrachap{Acknowledgements}

Use the template \emph{acknow.tex} together with the Springer document class SVMono (monograph-type books) or SVMult (edited books) if you prefer to set your acknowledgement section as a separate chapter instead of including it as last part of your preface.



\tableofcontents

%%%%%%%%%%%%%%%%%%%%%%acronym.tex%%%%%%%%%%%%%%%%%%%%%%%%%%%%%%%%%%%%%%%%%
% sample list of acronyms
%
% Use this file as a template for your own input.
%
%%%%%%%%%%%%%%%%%%%%%%%% Springer %%%%%%%%%%%%%%%%%%%%%%%%%%

\extrachap{Acronyms}

Use the template \emph{acronym.tex} together with the Springer document class SVMono (monograph-type books) or SVMult (edited books) to style your list(s) of abbreviations or symbols in the Springer layout.

Lists of abbreviations\index{acronyms, list of}, symbols\index{symbols, list of} and the like are easily formatted with the help of the Springer-enhanced \verb|description| environment.

\begin{description}[CABR]
\item[ABC]{Spelled-out abbreviation and definition}
\item[BABI]{Spelled-out abbreviation and definition}
\item[CABR]{Spelled-out abbreviation and definition}
\end{description}


\mainmatter%%%%%%%%%%%%%%%%%%%%%%%%%%%%%%%%%%%%%%%%%%%%%%%%%%%%%%%
 %  Introductory chapters
         \part{Introduction to Programming and Numerical Methods}
 %%  Introduction
         
% ------------------- main content ----------------------

\chapter{Many-body Hamiltonians, basic linear algebra and Second Quantization}

\subsection*{Definitions and notations}

Before we proceed we need some definitions.
We will assume that the interacting part of the Hamiltonian
can be approximated by a two-body interaction.
This means that our Hamiltonian is written as the sum of some onebody part and a twobody part
\begin{equation}
    \hat{H} = \hat{H}_0 + \hat{H}_I 
    = \sum_{i=1}^A \hat{h}_0(x_i) + \sum_{i < j}^A \hat{v}(r_{ij}),
\label{Hnuclei}
\end{equation}
with 
\begin{equation}
  H_0=\sum_{i=1}^A \hat{h}_0(x_i).
\label{hinuclei}
\end{equation}
The onebody part $u_{\mathrm{ext}}(x_i)$ is normally approximated by a harmonic oscillator potential or the Coulomb interaction an electron feels from the nucleus. However, other potentials are fully possible, such as 
one derived from the self-consistent solution of the Hartree-Fock equations to be discussed here.

Our Hamiltonian is invariant under the permutation (interchange) of two particles.
Since we deal with fermions however, the total wave function is antisymmetric.
Let $\hat{P}$ be an operator which interchanges two particles.
Due to the symmetries we have ascribed to our Hamiltonian, this operator commutes with the total Hamiltonian,
\[
[\hat{H},\hat{P}] = 0,
 \]
meaning that $\Psi_{\lambda}(x_1, x_2, \dots , x_A)$ is an eigenfunction of 
$\hat{P}$ as well, that is
\[
\hat{P}_{ij}\Psi_{\lambda}(x_1, x_2, \dots,x_i,\dots,x_j,\dots,x_A)=
\beta\Psi_{\lambda}(x_1, x_2, \dots,x_i,\dots,x_j,\dots,x_A),
\]
where $\beta$ is the eigenvalue of $\hat{P}$. We have introduced the suffix $ij$ in order to indicate that we permute particles $i$ and $j$.
The Pauli principle tells us that the total wave function for a system of fermions
has to be antisymmetric, resulting in the eigenvalue $\beta = -1$.   

In our case we assume that  we can approximate the exact eigenfunction with a Slater determinant
\begin{equation}
   \Phi(x_1, x_2,\dots ,x_A,\alpha,\beta,\dots, \sigma)=\frac{1}{\sqrt{A!}}
\left| \begin{array}{ccccc} \psi_{\alpha}(x_1)& \psi_{\alpha}(x_2)& \dots & \dots & \psi_{\alpha}(x_A)\\
                            \psi_{\beta}(x_1)&\psi_{\beta}(x_2)& \dots & \dots & \psi_{\beta}(x_A)\\  
                            \dots & \dots & \dots & \dots & \dots \\
                            \dots & \dots & \dots & \dots & \dots \\
                     \psi_{\sigma}(x_1)&\psi_{\sigma}(x_2)& \dots & \dots & \psi_{\sigma}(x_A)\end{array} \right|, \label{eq:HartreeFockDet}
\end{equation}
where  $x_i$  stand for the coordinates and spin values of a particle $i$ and $\alpha,\beta,\dots, \gamma$ 
are quantum numbers needed to describe remaining quantum numbers.  

\paragraph{Brief reminder on some linear algebra properties.}
Before we proceed with a more compact representation of a Slater determinant, we would like to repeat some linear algebra properties which will be useful for our derivations of the energy as function of a Slater determinant, Hartree-Fock theory and later the nuclear shell model.

The inverse of a matrix is defined by

\[
\mathbf{A}^{-1} \cdot \mathbf{A} = I
\]
A unitary matrix $\mathbf{A}$ is one whose inverse is its adjoint
\[
\mathbf{A}^{-1}=\mathbf{A}^{\dagger}
\]
A real unitary matrix is called orthogonal and its inverse is equal to its transpose.
A hermitian matrix is its own self-adjoint, that  is
\[
\mathbf{A}=\mathbf{A}^{\dagger}. 
\]


\begin{quote}
\begin{tabular}{ccc}
\hline
\multicolumn{1}{c}{ Relations } & \multicolumn{1}{c}{ Name } & \multicolumn{1}{c}{ matrix elements } \\
\hline
$A = A^{T}$                            & symmetric       & $a_{ij} = a_{ji}$                                                       \\
$A = \left (A^{T} \right )^{-1}$       & real orthogonal & $\sum_k a_{ik} a_{jk} = \sum_k a_{ki} a_{kj} = \delta_{ij}$             \\
$A = A^{ * }$                          & real matrix     & $a_{ij} = a_{ij}^{ * }$                                                 \\
$A = A^{\dagger}$                      & hermitian       & $a_{ij} = a_{ji}^{ * }$                                                 \\
$A = \left (A^{\dagger} \right )^{-1}$ & unitary         & $\sum_k a_{ik} a_{jk}^{ * } = \sum_k a_{ki}^{ * } a_{kj} = \delta_{ij}$ \\
\hline
\end{tabular}
\end{quote}

\noindent
Since we will deal with Fermions (identical and indistinguishable particles) we will 
form an ansatz for a given state in terms of so-called Slater determinants determined
by a chosen basis of single-particle functions. 

For a given $n\times n$ matrix $\mathbf{A}$ we can write its determinant
\[
   det(\mathbf{A})=|\mathbf{A}|=
\left| \begin{array}{ccccc} a_{11}& a_{12}& \dots & \dots & a_{1n}\\
                            a_{21}&a_{22}& \dots & \dots & a_{2n}\\  
                            \dots & \dots & \dots & \dots & \dots \\
                            \dots & \dots & \dots & \dots & \dots \\
                            a_{n1}& a_{n2}& \dots & \dots & a_{nn}\end{array} \right|,
\]
in a more compact form as 
\[
|\mathbf{A}|= \sum_{i=1}^{n!}(-1)^{p_i}\hat{P}_i a_{11}a_{22}\dots a_{nn},
\]
where $\hat{P}_i$ is a permutation operator which permutes the column indices $1,2,3,\dots,n$
and the sum runs over all $n!$ permutations.  The quantity $p_i$ represents the number of transpositions of column indices that are needed in order to bring a given permutation back to its initial ordering, in our case given by $a_{11}a_{22}\dots a_{nn}$ here.

A simple $2\times 2$ determinant illustrates this. We have
\[
   det(\mathbf{A})=
\left| \begin{array}{cc} a_{11}& a_{12}\\
                            a_{21}&a_{22}\end{array} \right|= (-1)^0a_{11}a_{22}+(-1)^1a_{12}a_{21},
\]
where in the last term we have interchanged the column indices $1$ and $2$. The natural ordering we have chosen is $a_{11}a_{22}$. 

\paragraph{Back to the derivation of the energy.}
The single-particle function $\psi_{\alpha}(x_i)$  are eigenfunctions of the onebody
Hamiltonian $h_i$, that is
\[
\hat{h}_0(x_i)=\hat{t}(x_i) + \hat{u}_{\mathrm{ext}}(x_i),
\]
with eigenvalues 
\[
\hat{h}_0(x_i) \psi_{\alpha}(x_i)=\left(\hat{t}(x_i) + \hat{u}_{\mathrm{ext}}(x_i)\right)\psi_{\alpha}(x_i)=\varepsilon_{\alpha}\psi_{\alpha}(x_i).
\]
The energies $\varepsilon_{\alpha}$ are the so-called non-interacting single-particle energies, or unperturbed energies. 
The total energy is in this case the sum over all  single-particle energies, if no two-body or more complicated
many-body interactions are present.

Let us denote the ground state energy by $E_0$. According to the
variational principle we have
\[
  E_0 \le E[\Phi] = \int \Phi^*\hat{H}\Phi d\mathbf{\tau}
\]
where $\Phi$ is a trial function which we assume to be normalized
\[
  \int \Phi^*\Phi d\mathbf{\tau} = 1,
\]
where we have used the shorthand $d\mathbf{\tau}=dx_1dr_2\dots dr_A$.

In the Hartree-Fock method the trial function is the Slater
determinant of Eq.~(\ref{eq:HartreeFockDet}) which can be rewritten as 
\[
  \Phi(x_1,x_2,\dots,x_A,\alpha,\beta,\dots,\nu) = \frac{1}{\sqrt{A!}}\sum_{P} (-)^P\hat{P}\psi_{\alpha}(x_1)
    \psi_{\beta}(x_2)\dots\psi_{\nu}(x_A)=\sqrt{A!}\hat{A}\Phi_H,
\]
where we have introduced the antisymmetrization operator $\hat{A}$ defined by the 
summation over all possible permutations of two particles.

It is defined as
\begin{equation}
  \hat{A} = \frac{1}{A!}\sum_{p} (-)^p\hat{P},
\label{antiSymmetryOperator}
\end{equation}
with $p$ standing for the number of permutations. We have introduced for later use the so-called
Hartree-function, defined by the simple product of all possible single-particle functions
\[
  \Phi_H(x_1,x_2,\dots,x_A,\alpha,\beta,\dots,\nu) =
  \psi_{\alpha}(x_1)
    \psi_{\beta}(x_2)\dots\psi_{\nu}(x_A).
\]

Both $\hat{H}_0$ and $\hat{H}_I$ are invariant under all possible permutations of any two particles
and hence commute with $\hat{A}$
\begin{equation}
  [H_0,\hat{A}] = [H_I,\hat{A}] = 0. \label{commutionAntiSym}
\end{equation}
Furthermore, $\hat{A}$ satisfies
\begin{equation}
  \hat{A}^2 = \hat{A},  \label{AntiSymSquared}
\end{equation}
since every permutation of the Slater
determinant reproduces it. 

The expectation value of $\hat{H}_0$ 
\[
  \int \Phi^*\hat{H}_0\Phi d\mathbf{\tau} 
  = A! \int \Phi_H^*\hat{A}\hat{H}_0\hat{A}\Phi_H d\mathbf{\tau}
\]
is readily reduced to
\[
  \int \Phi^*\hat{H}_0\Phi d\mathbf{\tau} 
  = A! \int \Phi_H^*\hat{H}_0\hat{A}\Phi_H d\mathbf{\tau},
\]
where we have used Eqs.~(\ref{commutionAntiSym}) and
(\ref{AntiSymSquared}). The next step is to replace the antisymmetrization
operator by its definition and to
replace $\hat{H}_0$ with the sum of one-body operators
\[
  \int \Phi^*\hat{H}_0\Phi  d\mathbf{\tau}
  = \sum_{i=1}^A \sum_{p} (-)^p\int 
  \Phi_H^*\hat{h}_0\hat{P}\Phi_H d\mathbf{\tau}.
\]

The integral vanishes if two or more particles are permuted in only one
of the Hartree-functions $\Phi_H$ because the individual single-particle wave functions are
orthogonal. We obtain then
\[
  \int \Phi^*\hat{H}_0\Phi  d\mathbf{\tau}= \sum_{i=1}^A \int \Phi_H^*\hat{h}_0\Phi_H  d\mathbf{\tau}.
\]
Orthogonality of the single-particle functions allows us to further simplify the integral, and we
arrive at the following expression for the expectation values of the
sum of one-body Hamiltonians 
\begin{equation}
  \int \Phi^*\hat{H}_0\Phi  d\mathbf{\tau}
  = \sum_{\mu=1}^A \int \psi_{\mu}^*(x)\hat{h}_0\psi_{\mu}(x)dx
  d\mathbf{r}.
  \label{H1Expectation}
\end{equation}

We introduce the following shorthand for the above integral
\[
\langle \mu | \hat{h}_0 | \mu \rangle = \int \psi_{\mu}^*(x)\hat{h}_0\psi_{\mu}(x)dx,
\]
and rewrite Eq.~(\ref{H1Expectation}) as
\begin{equation}
  \int \Phi^*\hat{H}_0\Phi  d\tau
  = \sum_{\mu=1}^A \langle \mu | \hat{h}_0 | \mu \rangle.
  \label{H1Expectation1}
\end{equation}

The expectation value of the two-body part of the Hamiltonian is obtained in a
similar manner. We have
\[
  \int \Phi^*\hat{H}_I\Phi d\mathbf{\tau} 
  = A! \int \Phi_H^*\hat{A}\hat{H}_I\hat{A}\Phi_H d\mathbf{\tau},
\]
which reduces to
\[
 \int \Phi^*\hat{H}_I\Phi d\mathbf{\tau} 
  = \sum_{i\le j=1}^A \sum_{p} (-)^p\int 
  \Phi_H^*\hat{v}(r_{ij})\hat{P}\Phi_H d\mathbf{\tau},
\]
by following the same arguments as for the one-body
Hamiltonian. 

Because of the dependence on the inter-particle distance $r_{ij}$,  permutations of
any two particles no longer vanish, and we get
\[
  \int \Phi^*\hat{H}_I\Phi d\mathbf{\tau} 
  = \sum_{i < j=1}^A \int  
  \Phi_H^*\hat{v}(r_{ij})(1-P_{ij})\Phi_H d\mathbf{\tau}.
\]
where $P_{ij}$ is the permutation operator that interchanges
particle $i$ and particle $j$. Again we use the assumption that the single-particle wave functions
are orthogonal. 

We obtain
\begin{align}
  \int \Phi^*\hat{H}_I\Phi d\mathbf{\tau} 
  = \frac{1}{2}\sum_{\mu=1}^A\sum_{\nu=1}^A
    &\left[ \int \psi_{\mu}^*(x_i)\psi_{\nu}^*(x_j)\hat{v}(r_{ij})\psi_{\mu}(x_i)\psi_{\nu}(x_j)
    dx_idx_j \right.\\
  &\left.
  - \int \psi_{\mu}^*(x_i)\psi_{\nu}^*(x_j)
  \hat{v}(r_{ij})\psi_{\nu}(x_i)\psi_{\mu}(x_j)
  dx_idx_j
  \right]. \label{H2Expectation}
\end{align}
The first term is the so-called direct term. It is frequently also called the  Hartree term, 
while the second is due to the Pauli principle and is called
the exchange term or just the Fock term.
The factor  $1/2$ is introduced because we now run over
all pairs twice. 

The last equation allows us to  introduce some further definitions.  
The single-particle wave functions $\psi_{\mu}(x)$, defined by the quantum numbers $\mu$ and $x$
are defined as the overlap 
\[
   \psi_{\alpha}(x)  = \langle x | \alpha \rangle .
\]

We introduce the following shorthands for the above two integrals
\[
\langle \mu\nu|\hat{v}|\mu\nu\rangle =  \int \psi_{\mu}^*(x_i)\psi_{\nu}^*(x_j)\hat{v}(r_{ij})\psi_{\mu}(x_i)\psi_{\nu}(x_j)
    dx_idx_j,
\]
and
\[
\langle \mu\nu|\hat{v}|\nu\mu\rangle = \int \psi_{\mu}^*(x_i)\psi_{\nu}^*(x_j)
  \hat{v}(r_{ij})\psi_{\nu}(x_i)\psi_{\mu}(x_j)
  dx_idx_j.  
\]

\subsection*{Preparing for later studies: varying the coefficients of a wave function expansion and orthogonal transformations}

It is common to  expand the single-particle functions in a known basis  and vary the coefficients, 
that is, the new single-particle wave function is written as a linear expansion
in terms of a fixed chosen orthogonal basis (for example the well-known harmonic oscillator functions or the hydrogen-like functions etc).
We define our new single-particle basis (this is a normal approach for Hartree-Fock theory) by performing a unitary transformation 
on our previous basis (labelled with greek indices) as
\begin{equation}
\psi_p^{new}  = \sum_{\lambda} C_{p\lambda}\phi_{\lambda}. \label{eq:newbasis}
\end{equation}
In this case we vary the coefficients $C_{p\lambda}$. If the basis has infinitely many solutions, we need
to truncate the above sum.  We assume that the basis $\phi_{\lambda}$ is orthogonal.

It is normal to choose a single-particle basis defined as the eigenfunctions
of parts of the full Hamiltonian. The typical situation consists of the solutions of the one-body part of the Hamiltonian, that is we have
\[
\hat{h}_0\phi_{\lambda}=\epsilon_{\lambda}\phi_{\lambda}.
\]
The single-particle wave functions $\phi_{\lambda}(\mathbf{r})$, defined by the quantum numbers $\lambda$ and $\mathbf{r}$
are defined as the overlap 
\[
   \phi_{\lambda}(\mathbf{r})  = \langle \mathbf{r} | \lambda \rangle .
\]

In deriving the Hartree-Fock equations, we  will expand the single-particle functions in a known basis  and vary the coefficients, 
that is, the new single-particle wave function is written as a linear expansion
in terms of a fixed chosen orthogonal basis (for example the well-known harmonic oscillator functions or the hydrogen-like functions etc).

We stated that a unitary transformation keeps the orthogonality. To see this consider first a basis of vectors $\mathbf{v}_i$,
\[
\mathbf{v}_i = \begin{bmatrix} v_{i1} \\ \dots \\ \dots \\v_{in} \end{bmatrix}
\]
We assume that the basis is orthogonal, that is 
\[
\mathbf{v}_j^T\mathbf{v}_i = \delta_{ij}.
\]
An orthogonal or unitary transformation
\[
\mathbf{w}_i=\mathbf{U}\mathbf{v}_i,
\]
preserves the dot product and orthogonality since
\[
\mathbf{w}_j^T\mathbf{w}_i=(\mathbf{U}\mathbf{v}_j)^T\mathbf{U}\mathbf{v}_i=\mathbf{v}_j^T\mathbf{U}^T\mathbf{U}\mathbf{v}_i= \mathbf{v}_j^T\mathbf{v}_i = \delta_{ij}.
\]

This means that if the coefficients $C_{p\lambda}$ belong to a unitary or orthogonal trasformation (using the Dirac bra-ket notation)
\[
\vert p\rangle  = \sum_{\lambda} C_{p\lambda}\vert\lambda\rangle,
\]
orthogonality is preserved, that is $\langle \alpha \vert \beta\rangle = \delta_{\alpha\beta}$
and $\langle p \vert q\rangle = \delta_{pq}$. 

This propertry is extremely useful when we build up a basis of many-body Stater determinant based states. 

\textbf{Note also that although a basis $\vert \alpha\rangle$ contains an infinity of states, for practical calculations we have always to make some truncations.} 

Before we develop for example the Hartree-Fock equations, there is another very useful property of determinants that we will use both in connection with Hartree-Fock calculations and later shell-model calculations.  

Consider the following determinant
\[
\left| \begin{array}{cc} \alpha_1b_{11}+\alpha_2sb_{12}& a_{12}\\
                         \alpha_1b_{21}+\alpha_2b_{22}&a_{22}\end{array} \right|=\alpha_1\left|\begin{array}{cc} b_{11}& a_{12}\\
                         b_{21}&a_{22}\end{array} \right|+\alpha_2\left| \begin{array}{cc} b_{12}& a_{12}\\b_{22}&a_{22}\end{array} \right|
\]

We can generalize this to  an $n\times n$ matrix and have 
\[
\left| \begin{array}{cccccc} a_{11}& a_{12} & \dots & \sum_{k=1}^n c_k b_{1k} &\dots & a_{1n}\\
a_{21}& a_{22} & \dots & \sum_{k=1}^n c_k b_{2k} &\dots & a_{2n}\\
\dots & \dots & \dots & \dots & \dots & \dots \\
\dots & \dots & \dots & \dots & \dots & \dots \\
a_{n1}& a_{n2} & \dots & \sum_{k=1}^n c_k b_{nk} &\dots & a_{nn}\end{array} \right|=
\sum_{k=1}^n c_k\left| \begin{array}{cccccc} a_{11}& a_{12} & \dots &  b_{1k} &\dots & a_{1n}\\
a_{21}& a_{22} & \dots &  b_{2k} &\dots & a_{2n}\\
\dots & \dots & \dots & \dots & \dots & \dots\\
\dots & \dots & \dots & \dots & \dots & \dots\\
a_{n1}& a_{n2} & \dots &  b_{nk} &\dots & a_{nn}\end{array} \right| .
\]
This is a property we will use in our Hartree-Fock discussions. 

We can generalize the previous results, now 
with all elements $a_{ij}$  being given as functions of 
linear combinations  of various coefficients $c$ and elements $b_{ij}$,
\[
\left| \begin{array}{cccccc} \sum_{k=1}^n b_{1k}c_{k1}& \sum_{k=1}^n b_{1k}c_{k2} & \dots & \sum_{k=1}^n b_{1k}c_{kj}  &\dots & \sum_{k=1}^n b_{1k}c_{kn}\\
\sum_{k=1}^n b_{2k}c_{k1}& \sum_{k=1}^n b_{2k}c_{k2} & \dots & \sum_{k=1}^n b_{2k}c_{kj} &\dots & \sum_{k=1}^n b_{2k}c_{kn}\\
\dots & \dots & \dots & \dots & \dots & \dots \\
\dots & \dots & \dots & \dots & \dots &\dots \\
\sum_{k=1}^n b_{nk}c_{k1}& \sum_{k=1}^n b_{nk}c_{k2} & \dots & \sum_{k=1}^n b_{nk}c_{kj} &\dots & \sum_{k=1}^n b_{nk}c_{kn}\end{array} \right|=det(\mathbf{C})det(\mathbf{B}),
\]
where $det(\mathbf{C})$ and $det(\mathbf{B})$ are the determinants of $n\times n$ matrices
with elements $c_{ij}$ and $b_{ij}$ respectively.  
This is a property we will use in our Hartree-Fock discussions. Convince yourself about the correctness of the above expression by setting $n=2$. 

With our definition of the new basis in terms of an orthogonal basis we have
\[
\psi_p(x)  = \sum_{\lambda} C_{p\lambda}\phi_{\lambda}(x).
\]
If the coefficients $C_{p\lambda}$ belong to an orthogonal or unitary matrix, the new basis
is also orthogonal. 
Our Slater determinant in the new basis $\psi_p(x)$ is written as
\[
\frac{1}{\sqrt{A!}}
\left| \begin{array}{ccccc} \psi_{p}(x_1)& \psi_{p}(x_2)& \dots & \dots & \psi_{p}(x_A)\\
                            \psi_{q}(x_1)&\psi_{q}(x_2)& \dots & \dots & \psi_{q}(x_A)\\  
                            \dots & \dots & \dots & \dots & \dots \\
                            \dots & \dots & \dots & \dots & \dots \\
                     \psi_{t}(x_1)&\psi_{t}(x_2)& \dots & \dots & \psi_{t}(x_A)\end{array} \right|=\frac{1}{\sqrt{A!}}
\left| \begin{array}{ccccc} \sum_{\lambda} C_{p\lambda}\phi_{\lambda}(x_1)& \sum_{\lambda} C_{p\lambda}\phi_{\lambda}(x_2)& \dots & \dots & \sum_{\lambda} C_{p\lambda}\phi_{\lambda}(x_A)\\
                            \sum_{\lambda} C_{q\lambda}\phi_{\lambda}(x_1)&\sum_{\lambda} C_{q\lambda}\phi_{\lambda}(x_2)& \dots & \dots & \sum_{\lambda} C_{q\lambda}\phi_{\lambda}(x_A)\\  
                            \dots & \dots & \dots & \dots & \dots \\
                            \dots & \dots & \dots & \dots & \dots \\
                     \sum_{\lambda} C_{t\lambda}\phi_{\lambda}(x_1)&\sum_{\lambda} C_{t\lambda}\phi_{\lambda}(x_2)& \dots & \dots & \sum_{\lambda} C_{t\lambda}\phi_{\lambda}(x_A)\end{array} \right|,
\]
which is nothing but $det(\mathbf{C})det(\Phi)$, with $det(\Phi)$ being the determinant given by the basis functions $\phi_{\lambda}(x)$. 

In our discussions hereafter we will use our definitions of single-particle states above and below the Fermi ($F$) level given by the labels
$ijkl\dots \le F$ for so-called single-hole states and $abcd\dots > F$ for so-called particle states.
For general single-particle states we employ the labels $pqrs\dots$. 

The energy functional is
\[
  E[\Phi] 
  = \sum_{\mu=1}^A \langle \mu | h | \mu \rangle +
  \frac{1}{2}\sum_{{\mu}=1}^A\sum_{{\nu}=1}^A \langle \mu\nu|\hat{v}|\mu\nu\rangle_{AS},
\]
we found the expression for the energy functional in terms of the basis function $\phi_{\lambda}(\mathbf{r})$. We then  varied the above energy functional with respect to the basis functions $|\mu \rangle$. 
Now we are interested in defining a new basis defined in terms of
a chosen basis as defined in Eq.~(\ref{eq:newbasis}). We can then rewrite the energy functional as
\begin{equation}
  E[\Phi^{New}] 
  = \sum_{i=1}^A \langle i | h | i \rangle +
  \frac{1}{2}\sum_{ij=1}^A\langle ij|\hat{v}|ij\rangle_{AS}, \label{FunctionalEPhi2}
\end{equation}
where $\Phi^{New}$ is the new Slater determinant defined by the new basis of Eq.~(\ref{eq:newbasis}). 

Using Eq.~(\ref{eq:newbasis}) we can rewrite Eq.~(\ref{FunctionalEPhi2}) as 
\begin{equation}
  E[\Psi] 
  = \sum_{i=1}^A \sum_{\alpha\beta} C^*_{i\alpha}C_{i\beta}\langle \alpha | h | \beta \rangle +
  \frac{1}{2}\sum_{ij=1}^A\sum_{{\alpha\beta\gamma\delta}} C^*_{i\alpha}C^*_{j\beta}C_{i\gamma}C_{j\delta}\langle \alpha\beta|\hat{v}|\gamma\delta\rangle_{AS}. \label{FunctionalEPhi3}
\end{equation}


 \clearemptydoublepage
 %%  introduction to C++ and F90 programming
         \input{basicCF90.tex}
 \clearemptydoublepage
 %% numerical derivation
      \input{differentiate.tex}
 \clearemptydoublepage
 %% Classes.
%\input{classes}
% \clearemptydoublepage
 %% numerical integration
      \input{integrate.tex}
 \clearemptydoublepage
 % this section should be included at an earlier stage
      \input{nonlinear.tex}
 \clearemptydoublepage
% \clearemptydoublepage
         \part{Linear Algebra and Eigenvalue problems}
 %% Linear algebra.
 \input{linalgebra}
% Put diagonalization chapter here
 \clearemptydoublepage
 %% interpolation  this chapter should come earlier
%      \input{interpolate.tex}
% \clearemptydoublepage
 %% eigenvalue systems
     \input{eigenvalue.tex}
 \clearemptydoublepage
        \part{Ordinary and Partial Differential Equations}
 %%  Differential equations 
         \input{diffeq.tex}
 \clearemptydoublepage
 %% Two point boundary value problems. 
         \input{twopboundary.tex}
 \clearemptydoublepage
 %% Partial differential equations, finite difference
 \input{partdiff.tex}
 \clearemptydoublepage
        \part{Monte Carlo Methods}
 %% Monte Carlo methods
      \input{montecarlo_intro.tex}
 \clearemptydoublepage
 %% random walks and the diffusion equation
      \input{randowalks.tex}
 \clearemptydoublepage
 %% Monte carlo applications, stat phys
      \input{stat_phys.tex}
 \clearemptydoublepage
% \chapter{Modelling Phase Transitions in Statistical Physics}\label{chap:advancedstatphys}
% \input{advancedsm.tex}
% \clearemptydoublepage
 %% Monte carlo applications, quantum mechanics
      \input{vmc.tex}
 \clearemptydoublepage




 %  Advanced topics
 \part{Advanced topics}   

 \chapter{Many-body approaches to studies of electronic systems: Hartree-Fock theory}\label{chap:advancedatoms}

 \input{advancedatoms}

 \clearemptydoublepage
 \chapter{Bose-Einstein condensation and Diffusion Monte Carlo}\label{chap:advancedqmc}

 \input{advancedqm.tex}



 \bibliographystyle{unsrt}

 \bibliography{../../../../Library_of_Topics/mylib}


%\include{part}
%\include{chapter}
%\include{appendix}

\backmatter%%%%%%%%%%%%%%%%%%%%%%%%%%%%%%%%%%%%%%%%%%%%%%%%%%%%%%%
%\include{glossary}
%\include{solutions}
\printindex

%%%%%%%%%%%%%%%%%%%%%%%%%%%%%%%%%%%%%%%%%%%%%%%%%%%%%%%%%%%%%%%%%%%%%%

\end{document}






 \end{document}









