%%%%%%%%%%%%%%%%%%%% book.tex %%%%%%%%%%%%%%%%%%%%%%%%%%%%%
%
% sample root file for the chapters of your "monograph"
%
% Use this file as a template for your own input.
%
%%%%%%%%%%%%%%%% Springer-Verlag %%%%%%%%%%%%%%%%%%%%%%%%%%


% RECOMMENDED %%%%%%%%%%%%%%%%%%%%%%%%%%%%%%%%%%%%%%%%%%%%%%%%%%%
\documentclass[graybox,envcountchap,sectrefs]{svmono}

% choose options for [] as required from the list
% in the Reference Guide

\usepackage{mathptmx}
\usepackage{helvet}
\usepackage{courier}
%
\usepackage{type1cm}         

\usepackage{makeidx}         % allows index generation
\usepackage{graphicx}        % standard LaTeX graphics tool
                             % when including figure files
\usepackage{multicol}        % used for the two-column index
\usepackage[bottom]{footmisc}% places footnotes at page bottom

% see the list of further useful packages
% in the Reference Guide

\makeindex             % used for the subject index
                       % please use the style svind.ist with
                       % your makeindex program
\usepackage[usenames,dvipsnames,x11names]{xcolor}
\usepackage{tikz}
\usetikzlibrary{arrows,snakes,shapes}

 \usepackage{listings}
 \usepackage{graphicx}
 \usepackage{epic}
 \usepackage{eepic}
 \usepackage{a4wide}
 \usepackage{color}
 \usepackage{amsmath}
 \usepackage{amssymb}
 \usepackage[dvips]{epsfig}
 \usepackage{psfig}
 \usepackage[T1]{fontenc}
 \usepackage{cite} % [2,3,4] --> [2--4]
 \usepackage{shadow}
 \usepackage{hyperref}
 \usepackage{bezier}
 \usepackage{pstricks}
 %\usepackage{refcheck}
 \setcounter{tocdepth}{2}
%\usepackage{gnuplot-lua-tikz}


\usepackage{textcomp,type1ec,pdfpages}
\usepackage{bera}

\definecolor{dkgreen}{rgb}{0,0.6,0}
\definecolor{gray}{rgb}{0.5,0.5,0.5}
\definecolor{mauve}{rgb}{0.58,0,0.82}

 \lstset{language=c++}
 \lstset{alsolanguage=[90]Fortran}
 \lstset{alsolanguage=python}
% \lstset{basicstyle=\small}
 \lstset{backgroundcolor=\color{white}}
 \lstset{frame=single}
 \lstset{stringstyle=\ttfamily}
 \lstset{keywordstyle=\color{red}\bfseries}
 \lstset{commentstyle=\itshape\color{blue}}
 \lstset{showspaces=false}
 \lstset{showstringspaces=false}
 \lstset{showtabs=false}
 \lstset{breaklines}
 

% Default settings for code listings
% \lstnewenvironment{Python}[1]{
\lstset{%frame=tb,
  language=c++,
  alsolanguage=python,
  %aboveskip=3mm,
 % belowskip=3mm,
  showstringspaces=false,
  columns=flexible,
  basicstyle={\footnotesize\ttfamily},
  numbers=none,
  numberstyle=\tiny\color{gray},
  commentstyle=\color{dkgreen},
  stringstyle=\color{mauve},
  frame=single,  
  breaklines=true,
  %%%% FOR PYTHON 
  otherkeywords={\ , \}, \{},
  keywordstyle=\color{blue},
  emph={void, ||, &&, break, class,continue, delete, else,
  for, if, include, return,try,while},
  emphstyle=\color{black}\bfseries,
  emph={[2]True, False, None, self},
  emphstyle=[2]\color{dkgreen},
  emphstyle=[2]\color{red},
  emph={[3]from, import, as},
  emphstyle=[3]\color{blue},
  upquote=true,
  morecomment=[s]{"""}{"""},
  commentstyle=\color{green}\slshape, %%% cambie gray por green
  emph={[4]1, 2, 3, 4, 5, 6, 7, 8, 9, 0},
  emphstyle=[4]\color{blue},
  breakatwhitespace=true,
  tabsize=2
}

\renewcommand{\lstlistlistingname}{Code Listings}
\renewcommand{\lstlistingname}{Code Listing}
\definecolor{gray}{gray}{0.5}
\definecolor{green}{rgb}{0,0.5,0}

\lstnewenvironment{Python}[1]{
\lstset{
language=python,
basicstyle=\footnotesize\setstretch{1},
stringstyle=\color{red},
showstringspaces=false,
alsoletter={1234567890},
otherkeywords={\ , \}, \{},
keywordstyle=\color{blue},
emph={access,and,break,class,continue,def,del,elif ,else,%
except,exec,finally,for,from,global,if,import,in,is,%
lambda,not,or,pass,print,raise,return,try,while},
emphstyle=\color{black}\bfseries,
emph={[2]True, False, None, self},
emphstyle=[2]\color{red},
emph={[3]from, import, as},
emphstyle=[3]\color{blue},
upquote=true,
morecomment=[s]{"""}{"""},
commentstyle=\color{dkgreen}\slshape, % el color era gray pero lo cambie a verde
emph={[4]1, 2, 3, 4, 5, 6, 7, 8, 9, 0},
emphstyle=[4]\color{blue},
framexleftmargin=1mm, framextopmargin=1mm, rulesepcolor=\color{blue},
breakatwhitespace=true,
tabsize=2
}}{}


\lstnewenvironment{C++}[1]{
\lstset{
language=c++,
% basicstyle=\ttfamily\small\setstretch{1},
basicstyle=\footnotesize\setstretch{1},
stringstyle=\color{red},
showstringspaces=false,
alsoletter={1234567890},
otherkeywords={\ , \}, \{},
keywordstyle=\color{blue},
emph={access,and,break,class,continue,def,del,elif ,else,%
except,exec,finally,for,from,global,if,import,in,is,%
lambda,not,or,pass,print,raise,return,try,while},
emphstyle=\color{black}\bfseries,
emph={[2]True, False, None, self},
emphstyle=[2]\color{red},
emph={[3]from, import, as},
emphstyle=[3]\color{blue},
upquote=true,
morecomment=[s]{"""}{"""},
commentstyle=\color{dkgreen}\slshape, % el color era gray pero lo cambie a verde
emph={[4]1, 2, 3, 4, 5, 6, 7, 8, 9, 0},
emphstyle=[4]\color{blue},
% literate=*{:}{{\textcolor{blue}:}}{1}%
% {=}{{\textcolor{blue}=}}{1}%
% {-}{{\textcolor{blue}-}}{1}%
% {+}{{\textcolor{blue}+}}{1}%
% {*}{{\textcolor{blue}*}}{1}%
% {!}{{\textcolor{blue}!}}{1}%
% {(}{{\textcolor{blue}(}}{1}%
% {)}{{\textcolor{blue})}}{1}%
% {[}{{\textcolor{blue}[}}{1}%
% {]}{{\textcolor{blue}]}}{1}%
% {<}{{\textcolor{blue}<}}{1}%
% {>}{{\textcolor{blue}>}}{1},%
framexleftmargin=1mm, framextopmargin=1mm, rulesepcolor=\color{blue},
breakatwhitespace=true,
tabsize=2
}}{}




\usepackage{tikz}
\usetikzlibrary{shapes,arrows}

% Define block styles
\tikzstyle{decision} = [diamond, draw, fill=blue!20,
    text width=3.5em, text badly centered, node distance=2.5cm, inner sep=0pt]
\tikzstyle{block} = [rectangle, draw, fill=blue!20,
    text width=8em, text centered, rounded corners, minimum height=4em]
\tikzstyle{line} = [draw, very thick, color=black!50, -latex']
\tikzstyle{cloud} = [draw, ellipse,fill=red!20, node distance=2.5cm,
    minimum height=2em]

\def\radius{.7mm} 
\tikzstyle{branch}=[fill,shape=circle,minimum size=3pt,inner sep=0pt]


\newcommand{\bfv}[1]{\boldsymbol{#1}} 
\newcommand{\Div}[1]{\nabla \bullet \vbf{#1}}           % define divergence
\newcommand{\Grad}[1]{\boldsymbol{\nabla}{#1}}
 \newcommand{\OP}[1]{{\bf\widehat{#1}}}
 \newcommand{\be}{\begin{equation}}
 \newcommand{\ee}{\end{equation}}
\newcommand{\beN}{\begin{equation*}}
\newcommand{\bea}{\begin{eqnarray}}
\newcommand{\beaN}{\begin{eqnarray*}}
\newcommand{\eeN}{\end{equation*}}
\newcommand{\eea}{\end{eqnarray}}
\newcommand{\eeaN}{\end{eqnarray*}}
\newcommand{\bdm}{\begin{displaymath}}
\newcommand{\edm}{\end{displaymath}}
\newcommand{\bsubeqs}{\begin{subequations}}
\newcommand{\esubeqs}{\end{subequations}}
\newcommand{\Obs}[1]{\langle{\Op{#1}\rangle}}             % define observable
\newcommand{\PsiT}{\bfv{\Psi_T}(\bfv{R})}                       % symbol for trial wave function
%\newcommand{\braket}[2]{\langle{#1}|\Op{#2}|{#1}\rangle}
\newcommand{\Det}[1]{{|\bfv{#1}|}}
\newcommand{\uvec}[1]{\mbox{\boldmath$\hat{#1}$\unboldmath}}
\newcommand{\Op}[1]{{\bf\widehat{#1}}}    
\newcommand{\eqbrace}[4]{\left\{
\begin{array}{ll}
#1 & #2 \\[0.5cm]
#3 & #4
\end{array}\right.}
\newcommand{\eqbraced}[4]{\left\{
\begin{array}{ll}
#1 & #2 \\[0.5cm]
#3 & #4
\end{array}\right\}}
\newcommand{\eqbracetriple}[6]{\left\{
\begin{array}{ll}
#1 & #2 \\
#3 & #4 \\
#5 & #6
\end{array}\right.}
\newcommand{\eqbracedtriple}[6]{\left\{
\begin{array}{ll}
#1 & #2 \\
#3 & #4 \\
#5 & #6
\end{array}\right\}}

\newcommand{\mybox}[3]{\mbox{\makebox[#1][#2]{$#3$}}}
\newcommand{\myframedbox}[3]{\mbox{\framebox[#1][#2]{$#3$}}}

%% Infinitesimal (and double infinitesimal), useful at end of integrals
%\newcommand{\ud}[1]{\mathrm d#1}
\newcommand{\ud}[1]{d#1}
\newcommand{\udd}[1]{d^2\!#1}

%% Operators, algebraic matrices, algebraic vectors

%% Operator (hat, bold or bold symbol, whichever you like best):
\newcommand{\op}[1]{\widehat{#1}}
%\newcommand{\op}[1]{\mathbf{#1}}
%\newcommand{\op}[1]{\boldsymbol{#1}}

%% Vector:
\renewcommand{\vec}[1]{\boldsymbol{#1}}

%% Matrix symbol:
\newcommand{\matr}[1]{\boldsymbol{#1}}
%\newcommand{\bb}[1]{\mathbb{#1}}

%% Determinant symbol:
\renewcommand{\det}[1]{|#1|}

%% Means (expectation values) of varius sizes
\newcommand{\mean}[1]{\langle #1 \rangle}
\newcommand{\meanb}[1]{\big\langle #1 \big\rangle}
\newcommand{\meanbb}[1]{\Big\langle #1 \Big\rangle}
\newcommand{\meanbbb}[1]{\bigg\langle #1 \bigg\rangle}
\newcommand{\meanbbbb}[1]{\Bigg\langle #1 \Bigg\rangle}

%% Shorthands for text set in roman font
\newcommand{\prob}[0]{\mathrm{Prob}} %probability
\newcommand{\cov}[0]{\mathrm{Cov}}   %covariance
\newcommand{\var}[0]{\mathrm{Var}}   %variancd

%% Big-O (typically for specifying the speed scaling of an algorithm)
\newcommand{\bigO}{\mathcal{O}}

%% Real value of a complex number
\newcommand{\real}[1]{\mathrm{Re}\!\left\{#1\right\}}

%% Quantum mechanical state vectors and matrix elements (of different sizes)
\newcommand{\brab}[1]{\big\langle #1 \big|}
\newcommand{\brabb}[1]{\Big\langle #1 \Big|}
\newcommand{\brabbb}[1]{\bigg\langle #1 \bigg|}
\newcommand{\brabbbb}[1]{\Bigg\langle #1 \Bigg|}
\newcommand{\ketb}[1]{\big| #1 \big\rangle}
\newcommand{\ketbb}[1]{\Big| #1 \Big\rangle}
\newcommand{\ketbbb}[1]{\bigg| #1 \bigg\rangle}
\newcommand{\ketbbbb}[1]{\Bigg| #1 \Bigg\rangle}
\newcommand{\overlap}[2]{\langle #1 | #2 \rangle}
\newcommand{\overlapb}[2]{\big\langle #1 \big| #2 \big\rangle}
\newcommand{\overlapbb}[2]{\Big\langle #1 \Big| #2 \Big\rangle}
\newcommand{\overlapbbb}[2]{\bigg\langle #1 \bigg| #2 \bigg\rangle}
\newcommand{\overlapbbbb}[2]{\Bigg\langle #1 \Bigg| #2 \Bigg\rangle}
\newcommand{\bracket}[3]{\langle #1 | #2 | #3 \rangle}
\newcommand{\bracketb}[3]{\big\langle #1 \big| #2 \big| #3 \big\rangle}
\newcommand{\bracketbb}[3]{\Big\langle #1 \Big| #2 \Big| #3 \Big\rangle}
\newcommand{\bracketbbb}[3]{\bigg\langle #1 \bigg| #2 \bigg| #3 \bigg\rangle}
\newcommand{\bracketbbbb}[3]{\Bigg\langle #1 \Bigg| #2 \Bigg| #3 \Bigg\rangle}
\newcommand{\projection}[2]
{| #1 \rangle \langle  #2 |}
\newcommand{\projectionb}[2]
{\big| #1 \big\rangle \big\langle #2 \big|}
\newcommand{\projectionbb}[2]
{ \Big| #1 \Big\rangle \Big\langle #2 \Big|}
\newcommand{\projectionbbb}[2]
{ \bigg| #1 \bigg\rangle \bigg\langle #2 \bigg|}
\newcommand{\projectionbbbb}[2]
{ \Bigg| #1 \Bigg\rangle \Bigg\langle #2 \Bigg|}







%%%%%%%%%%%%%%%%%%%%%%%%%%%%%%%%%%%%%%%%%%%%%%%%%%%%%%%%%%%%%%%%%%%%%

\begin{document}

\author{Morten Hjorth-Jensen}
\title{Quantum mechanics for many-particle systems}
\subtitle{From standard methods to quantum computing and machine learning}
\maketitle

\frontmatter%%%%%%%%%%%%%%%%%%%%%%%%%%%%%%%%%%%%%%%%%%%%%%%%%%%%%%


%%%%%%%%%%%%%%%%%%%%%%% dedic.tex %%%%%%%%%%%%%%%%%%%%%%%%%%%%%%%%%
%
% sample dedication
%
% Use this file as a template for your own input.
%
%%%%%%%%%%%%%%%%%%%%%%%% Springer %%%%%%%%%%%%%%%%%%%%%%%%%%

\begin{dedication}
Use the template \emph{dedic.tex} together with the Springer document class SVMono for monograph-type books or SVMult for contributed volumes to style a quotation or a dedication\index{dedication} at the very beginning of your book in the Springer layout
\end{dedication}





\include{foreword}
\preface
%  last update : 24/8/2013  mhj

\begin{quotation}
So, ultimately, in order to understand nature it may be necessary to
have a deeper understanding of mathematical relationships. But the
real reason is that the subject is enjoyable, and although we humans
cut nature up in different ways, and we have different courses in
different departments, such compartmentalization is really artificial,
and we should take our intellectual pleasures where we find them. 
{\em Richard Feynman, The Laws of Thermodynamics.}
\end{quotation}

Why a preface you may ask? Isn't that just a mere exposition of a
raison d'$\mathrm{\hat{e}}$tre of an author's choice of material,
preferences, biases, teaching philosophy etc.?  To a large extent I
can answer in the affirmative to that. A preface ought to be personal.
Indeed, what you will see in the various chapters of these notes
represents how I perceive computational physics should be taught.

 This set of lecture notes serves the scope of presenting to you and
train you in an algorithmic approach to problems in the sciences,
represented here by the unity of three disciplines, physics,
mathematics and informatics. This trinity outlines the emerging field
of computational physics.

Our insight in a physical system, combined with numerical mathematics
gives us the rules for setting up an algorithm, viz.~a set of rules
for solving a particular problem.  Our understanding of the physical
system under study is obviously gauged by the natural laws at play,
the initial conditions, boundary conditions and other external
constraints which influence the given system. Having spelled out the
physics, for example in the form of a set of coupled partial
differential equations, we need efficient numerical methods in order
to set up the final algorithm.  This algorithm is in turn coded into a
computer program and executed on available computing facilities.  To
develop such an algorithmic approach, you will be exposed to several
physics cases, spanning from the classical pendulum to quantum
mechanical systems. We will also present some of the most popular
algorithms from numerical mathematics used to solve a plethora of
problems in the sciences.  Finally we will codify these algorithms
using some of the most widely used programming languages, presently C,
C++ and Fortran and its most recent standard Fortran
2008\footnote{Throughout this text we refer to Fortran 2008 as
Fortran, implying the latest standard.}. However, a high-level and fully
object-oriented language like Python is now emerging as a good
alternative although C++ and Fortran still outperform Python when it
comes to computational speed.  In this text we offer an approach where
one can write all programs in C/C++ or Fortran.  We will also show you
how to develop large programs in Python interfacing C++ and/or Fortran
functions for those parts of the program which are CPU intensive.
Such an approach allows you to structure the flow of data in a
high-level language like Python while tasks of a mere repetitive and
CPU intensive nature are left to low-level languages like C++ or
Fortran. Python allows you also to smoothly interface your program
with other software, such as plotting programs or operating system
instructions. A typical Python program you may end up writing contains
everything from compiling and running your codes to preparing the body
of a file for writing up your report.



Computer simulations are nowadays an integral part of contemporary
basic and applied research in the sciences.  Computation is becoming
as important as theory and experiment. In physics, computational
physics, theoretical physics and experimental physics are all equally
important in our daily research and studies of physical
systems. Physics is the unity of theory, experiment and
computation\footnote{We mentioned previously the trinity of physics,
mathematics and informatics. Viewing physics as the trinity of theory,
experiment and simulations is yet another example. It is obviously
tempting to go beyond the sciences. History shows that triunes,
trinities and for example triple deities permeate the Indo-European
cultures (and probably all human cultures), from the ancient Celts and
Hindus to modern days.  The ancient Celts revered many such trinues,
their world was divided into earth, sea and air, nature was divided in
animal, vegetable and mineral and the cardinal colours were red,
yellow and blue, just to mention a few.  As a curious digression, it
was a Gaulish Celt, Hilary, philosopher and bishop of Poitiers (AD
315-367) in his work {\em De Trinitate} who formulated the Holy
Trinity concept of Christianity, perhaps in order to accomodate
millenia of human divination practice.}.  Moreover, the ability "to
compute" forms part of the essential repertoire of research
scientists. Several new fields within computational science have
emerged and strengthened their positions in the last years, such as
computational materials science, bioinformatics, computational
mathematics and mechanics, computational chemistry and physics and so
forth, just to mention a few.  These fields underscore the importance
of simulations as a means to gain novel insights into physical
systems, especially for those cases where no analytical solutions can
be found or an experiment is too complicated or expensive to carry
out.  To be able to simulate large quantal systems with many degrees
of freedom such as strongly interacting electrons in a quantum dot
will be of great importance for future directions in novel fields like
nano-techonology.  This ability often combines knowledge from many
different subjects, in our case essentially from the physical
sciences, numerical mathematics, computing languages, topics from
high-performace computing and some knowledge of computers.


In 1999, when I started this course at the department of physics in
Oslo, computational physics and computational science in general were
still perceived by the majority of physicists and scientists as topics
dealing with just mere tools and number crunching, and not as subjects
of their own.  The computational background of most students enlisting
for the course on computational physics could span from dedicated
hackers and computer freaks to people who basically had never used a
PC. The majority of undergraduate and graduate students had a very
rudimentary knowledge of computational techniques and methods.
Questions like 'do you know of better methods for numerical
integration than the trapezoidal rule' were not uncommon. I do happen
to know of colleagues who applied for time at a supercomputing centre
because they needed to invert matrices of the size of $10^4\times
10^4$ since they were using the trapezoidal rule to compute
integrals. With Gaussian quadrature this dimensionality was easily
reduced to matrix problems of the size of $10^2\times 10^2$, with much
better precision.

More than a decade later most students have now been exposed to a
fairly uniform introduction to computers, basic programming skills and
use of numerical exercises.  Practically every undergraduate student
in physics has now made a Matlab or Maple simulation of for example
the pendulum, with or without chaotic motion.  Nowadays most of you
are familiar, through various undergraduate courses in physics and
mathematics, with interpreted languages such as Maple, Matlab and/or
Mathematica. In addition, the interest in scripting languages such as
Python or Perl has increased considerably in recent years.  The modern
programmer would typically combine several tools, computing
environments and programming languages. A typical example is the
following. Suppose you are working on a project which demands
extensive visualizations of the results. To obtain these results, that
is to solve a physics problems like obtaining the density profile of a
Bose-Einstein condensate, you need however a program which is fairly
fast when computational speed matters.  In this case you would most
likely write a high-performance computing program using Monte Carlo
methods in languages which are tailored for that. These are
represented by programming languages like Fortran and C++.  However,
to visualize the results you would find interpreted languages like
Matlab or scripting languages like Python extremely suitable for your
tasks.  You will therefore end up writing for example a script in
Matlab which calls a Fortran or C++ program where the number crunching
is done and then visualize the results of say a wave equation solver
via Matlab's large library of visualization tools. Alternatively, you
could organize everything into a Python or Perl script which does
everything for you, calls the Fortran and/or C++ programs and performs
the visualization in Matlab or Python. Used correctly, these tools,
spanning from scripting languages to high-performance computing
languages and vizualization programs, speed up your capability to
solve complicated problems.  Being multilingual is thus an advantage
which not only applies to our globalized modern society but to
computing environments as well.  This text shows you how to use C++
and Fortran as programming languages.

There is however more to the picture than meets the eye.  Although
interpreted languages like Matlab, Mathematica and Maple allow you
nowadays to solve very complicated problems, and high-level languages
like Python can be used to solve computational problems, computational
speed and the capability to write an efficient code are topics which
still do matter. To this end, the majority of scientists still use
languages like C++ and Fortran to solve scientific problems.  When you
embark on a master or PhD thesis, you will most likely meet these
high-performance computing languages.  This course emphasizes thus the
use of programming languages like Fortran, Python and C++ instead of
interpreted ones like Matlab or Maple. You should however note that
there are still large differences in computer time between for example
numerical Python and a corresponding C++ program for many numerical
applications in the physical sciences, with a code in C++ or Fortran
being the fastest.

Computational speed is not the only reason for this choice of
programming languages. Another important reason is that we feel that
at a certain stage one needs to have some insights into the algorithm
used, its stability conditions, possible pitfalls like loss of
precision, ranges of applicability, the possibility to improve the
algorithm and taylor it to special purposes etc etc.  One of our major
aims here is to present to you what we would dub 'the algorithmic
approach', a set of rules for doing mathematics or a precise
description of how to solve a problem. To device an algorithm and
thereafter write a code for solving physics problems is a marvelous
way of gaining insight into complicated physical systems. The
algorithm you end up writing reflects in essentially all cases your
own understanding of the physics and the mathematics (the way you
express yourself) of the problem.  We do therefore devote quite some
space to the algorithms behind various functions presented in the
text. Especially, insight into how errors propagate and how to avoid
them is a topic we would like you to pay special attention to. Only
then can you avoid problems like underflow, overflow and loss of
precision. Such a control is not always achievable with interpreted
languages and canned functions where the underlying algorithm and/or
code is not easily accesible.  Although we will at various stages
recommend the use of library routines for say linear
algebra\footnote{Such library functions are often taylored to a given
machine's architecture and should accordingly run faster than user
provided ones.}, our belief is that one should understand what the
given function does, at least to have a mere idea.  With such a
starting point, we strongly believe that it can be easier to develope
more complicated programs on your own using Fortran, C++ or Python.

We have several other aims as well, namely:
\begin{itemize}
\item We would like to give you  an opportunity to gain a 
      deeper understanding of the physics you have learned in other
      courses. In most courses one is normally confronted with simple
      systems which provide exact solutions and mimic to a certain
      extent the realistic cases. Many are however the comments like
      'why can't we do something else than the particle in a box
      potential?'.  In several of the projects we hope to present some
      more 'realistic' cases to solve by various numerical
      methods. This also means that we wish to give examples of how
      physics can be applied in a much broader context than it is
      discussed in the traditional physics undergraduate curriculum.
\item To encourage you to "discover" physics in a way similar to how 
researchers learn in the context of research.
\item Hopefully also to introduce numerical methods and new areas of physics that 
      can be studied with the methods discussed.
\item To teach   structured programming in the context of doing science. 
\item The projects we propose are meant to mimic to a certain extent 
      the situation encountered during a thesis or project work. You
      will tipically have at your disposal 2-3 weeks to solve
      numerically a given project. In so doing you may need to do a
      literature study as well. Finally, we would like you to write a
      report for every project.
\end{itemize}
Our overall goal is to encourage you to learn about science through
experience and by asking questions. Our objective is always
understanding and the purpose of computing is further insight, not
mere numbers!  Simulations can often be considered as
experiments. Rerunning a simulation need not be as costly as rerunning
an experiment.


 
Needless to say, these lecture notes are upgraded continuously, from
typos to new input.  And we do always benefit from your comments,
suggestions and ideas for making these notes better.  It's through the
scientific discourse and critics we advance.  Moreover, I have
benefitted immensely from many discussions with fellow colleagues and
students. In particular I must mention Hans Petter Langtangen, Anders
Malthe-S\o renssen, Knut M\o rken and \O yvind Ryan, whose input
during the last fifteen years has considerably improved these lecture
notes.  Furthermore, the time we have spent and keep spending together
on the Computing in Science Education project at the University, is
just marvelous. Thanks so much. Concerning the Computing in Science
Education initiative, you can read more
at \url{http://www.mn.uio.no/english/about/collaboration/cse/}.


Finally, I would like to add a petit note on referencing. These notes
have evolved over many years and the idea is that they should end up
in the format of a web-based learning environment for doing
computational science. It will be fully free and hopefully represent a
much more efficient way of conveying teaching material than
traditional textbooks.  I have not yet settled on a specific format,
so any input is welcome. At present however, it is very easy for me to
upgrade and improve the material on say a yearly basis, from simple
typos to adding new material.  When accessing the web page of the
course, you will have noticed that you can obtain all source files for
the programs discussed in the text.  Many people have thus written to
me about how they should properly reference this material and whether
they can freely use it. My answer is rather simple.  You are
encouraged to use these codes, modify them, include them in
publications, thesis work, your lectures etc.  As long as your use is
part of the dialectics of science you can use this material freely.
However, since many weekends have elapsed in writing several of these
programs, testing them, sweating over bugs, swearing in front of a
f*@?\%g code which didn't compile properly ten minutes before monday
morning's eight o'clock lecture etc etc, I would dearly appreciate in
case you find these codes of any use, to reference them properly. That
can be done in a simple way, refer to M.~Hjorth-Jensen, {\em
Computational Physics}, University of Oslo (2013). The weblink to the
course should also be included. Hope it is not too much to ask
for. Enjoy!

%%%%%%%%%%%%%%%%%%%%%%acknow.tex%%%%%%%%%%%%%%%%%%%%%%%%%%%%%%%%%%%%%%%%%
% sample acknowledgement chapter
%
% Use this file as a template for your own input.
%
%%%%%%%%%%%%%%%%%%%%%%%% Springer %%%%%%%%%%%%%%%%%%%%%%%%%%

\extrachap{Acknowledgements}

Use the template \emph{acknow.tex} together with the Springer document class SVMono (monograph-type books) or SVMult (edited books) if you prefer to set your acknowledgement section as a separate chapter instead of including it as last part of your preface.



\tableofcontents

%%%%%%%%%%%%%%%%%%%%%%acronym.tex%%%%%%%%%%%%%%%%%%%%%%%%%%%%%%%%%%%%%%%%%
% sample list of acronyms
%
% Use this file as a template for your own input.
%
%%%%%%%%%%%%%%%%%%%%%%%% Springer %%%%%%%%%%%%%%%%%%%%%%%%%%

\extrachap{Acronyms}

Use the template \emph{acronym.tex} together with the Springer document class SVMono (monograph-type books) or SVMult (edited books) to style your list(s) of abbreviations or symbols in the Springer layout.

Lists of abbreviations\index{acronyms, list of}, symbols\index{symbols, list of} and the like are easily formatted with the help of the Springer-enhanced \verb|description| environment.

\begin{description}[CABR]
\item[ABC]{Spelled-out abbreviation and definition}
\item[BABI]{Spelled-out abbreviation and definition}
\item[CABR]{Spelled-out abbreviation and definition}
\end{description}


\mainmatter%%%%%%%%%%%%%%%%%%%%%%%%%%%%%%%%%%%%%%%%%%%%%%%%%%%%%%%
 %  Introductory chapters
 \part{Linear algebra and second quantization}
         
% ------------------- main content ----------------------

\chapter{Many-body Hamiltonians, basic linear algebra and Second Quantization}

\subsection*{Definitions and notations}

Before we proceed we need some definitions.
We will assume that the interacting part of the Hamiltonian
can be approximated by a two-body interaction.
This means that our Hamiltonian is written as the sum of some onebody part and a twobody part
\begin{equation}
    \hat{H} = \hat{H}_0 + \hat{H}_I 
    = \sum_{i=1}^A \hat{h}_0(x_i) + \sum_{i < j}^A \hat{v}(r_{ij}),
\label{Hnuclei}
\end{equation}
with 
\begin{equation}
  H_0=\sum_{i=1}^A \hat{h}_0(x_i).
\label{hinuclei}
\end{equation}
The onebody part $u_{\mathrm{ext}}(x_i)$ is normally approximated by a harmonic oscillator potential or the Coulomb interaction an electron feels from the nucleus. However, other potentials are fully possible, such as 
one derived from the self-consistent solution of the Hartree-Fock equations to be discussed here.

Our Hamiltonian is invariant under the permutation (interchange) of two particles.
Since we deal with fermions however, the total wave function is antisymmetric.
Let $\hat{P}$ be an operator which interchanges two particles.
Due to the symmetries we have ascribed to our Hamiltonian, this operator commutes with the total Hamiltonian,
\[
[\hat{H},\hat{P}] = 0,
 \]
meaning that $\Psi_{\lambda}(x_1, x_2, \dots , x_A)$ is an eigenfunction of 
$\hat{P}$ as well, that is
\[
\hat{P}_{ij}\Psi_{\lambda}(x_1, x_2, \dots,x_i,\dots,x_j,\dots,x_A)=
\beta\Psi_{\lambda}(x_1, x_2, \dots,x_i,\dots,x_j,\dots,x_A),
\]
where $\beta$ is the eigenvalue of $\hat{P}$. We have introduced the suffix $ij$ in order to indicate that we permute particles $i$ and $j$.
The Pauli principle tells us that the total wave function for a system of fermions
has to be antisymmetric, resulting in the eigenvalue $\beta = -1$.   

In our case we assume that  we can approximate the exact eigenfunction with a Slater determinant
\begin{equation}
   \Phi(x_1, x_2,\dots ,x_A,\alpha,\beta,\dots, \sigma)=\frac{1}{\sqrt{A!}}
\left| \begin{array}{ccccc} \psi_{\alpha}(x_1)& \psi_{\alpha}(x_2)& \dots & \dots & \psi_{\alpha}(x_A)\\
                            \psi_{\beta}(x_1)&\psi_{\beta}(x_2)& \dots & \dots & \psi_{\beta}(x_A)\\  
                            \dots & \dots & \dots & \dots & \dots \\
                            \dots & \dots & \dots & \dots & \dots \\
                     \psi_{\sigma}(x_1)&\psi_{\sigma}(x_2)& \dots & \dots & \psi_{\sigma}(x_A)\end{array} \right|, \label{eq:HartreeFockDet}
\end{equation}
where  $x_i$  stand for the coordinates and spin values of a particle $i$ and $\alpha,\beta,\dots, \gamma$ 
are quantum numbers needed to describe remaining quantum numbers.  

\paragraph{Brief reminder on some linear algebra properties.}
Before we proceed with a more compact representation of a Slater determinant, we would like to repeat some linear algebra properties which will be useful for our derivations of the energy as function of a Slater determinant, Hartree-Fock theory and later the nuclear shell model.

The inverse of a matrix is defined by

\[
\mathbf{A}^{-1} \cdot \mathbf{A} = I
\]
A unitary matrix $\mathbf{A}$ is one whose inverse is its adjoint
\[
\mathbf{A}^{-1}=\mathbf{A}^{\dagger}
\]
A real unitary matrix is called orthogonal and its inverse is equal to its transpose.
A hermitian matrix is its own self-adjoint, that  is
\[
\mathbf{A}=\mathbf{A}^{\dagger}. 
\]


\begin{quote}
\begin{tabular}{ccc}
\hline
\multicolumn{1}{c}{ Relations } & \multicolumn{1}{c}{ Name } & \multicolumn{1}{c}{ matrix elements } \\
\hline
$A = A^{T}$                            & symmetric       & $a_{ij} = a_{ji}$                                                       \\
$A = \left (A^{T} \right )^{-1}$       & real orthogonal & $\sum_k a_{ik} a_{jk} = \sum_k a_{ki} a_{kj} = \delta_{ij}$             \\
$A = A^{ * }$                          & real matrix     & $a_{ij} = a_{ij}^{ * }$                                                 \\
$A = A^{\dagger}$                      & hermitian       & $a_{ij} = a_{ji}^{ * }$                                                 \\
$A = \left (A^{\dagger} \right )^{-1}$ & unitary         & $\sum_k a_{ik} a_{jk}^{ * } = \sum_k a_{ki}^{ * } a_{kj} = \delta_{ij}$ \\
\hline
\end{tabular}
\end{quote}

\noindent
Since we will deal with Fermions (identical and indistinguishable particles) we will 
form an ansatz for a given state in terms of so-called Slater determinants determined
by a chosen basis of single-particle functions. 

For a given $n\times n$ matrix $\mathbf{A}$ we can write its determinant
\[
   det(\mathbf{A})=|\mathbf{A}|=
\left| \begin{array}{ccccc} a_{11}& a_{12}& \dots & \dots & a_{1n}\\
                            a_{21}&a_{22}& \dots & \dots & a_{2n}\\  
                            \dots & \dots & \dots & \dots & \dots \\
                            \dots & \dots & \dots & \dots & \dots \\
                            a_{n1}& a_{n2}& \dots & \dots & a_{nn}\end{array} \right|,
\]
in a more compact form as 
\[
|\mathbf{A}|= \sum_{i=1}^{n!}(-1)^{p_i}\hat{P}_i a_{11}a_{22}\dots a_{nn},
\]
where $\hat{P}_i$ is a permutation operator which permutes the column indices $1,2,3,\dots,n$
and the sum runs over all $n!$ permutations.  The quantity $p_i$ represents the number of transpositions of column indices that are needed in order to bring a given permutation back to its initial ordering, in our case given by $a_{11}a_{22}\dots a_{nn}$ here.

A simple $2\times 2$ determinant illustrates this. We have
\[
   det(\mathbf{A})=
\left| \begin{array}{cc} a_{11}& a_{12}\\
                            a_{21}&a_{22}\end{array} \right|= (-1)^0a_{11}a_{22}+(-1)^1a_{12}a_{21},
\]
where in the last term we have interchanged the column indices $1$ and $2$. The natural ordering we have chosen is $a_{11}a_{22}$. 

\paragraph{Back to the derivation of the energy.}
The single-particle function $\psi_{\alpha}(x_i)$  are eigenfunctions of the onebody
Hamiltonian $h_i$, that is
\[
\hat{h}_0(x_i)=\hat{t}(x_i) + \hat{u}_{\mathrm{ext}}(x_i),
\]
with eigenvalues 
\[
\hat{h}_0(x_i) \psi_{\alpha}(x_i)=\left(\hat{t}(x_i) + \hat{u}_{\mathrm{ext}}(x_i)\right)\psi_{\alpha}(x_i)=\varepsilon_{\alpha}\psi_{\alpha}(x_i).
\]
The energies $\varepsilon_{\alpha}$ are the so-called non-interacting single-particle energies, or unperturbed energies. 
The total energy is in this case the sum over all  single-particle energies, if no two-body or more complicated
many-body interactions are present.

Let us denote the ground state energy by $E_0$. According to the
variational principle we have
\[
  E_0 \le E[\Phi] = \int \Phi^*\hat{H}\Phi d\mathbf{\tau}
\]
where $\Phi$ is a trial function which we assume to be normalized
\[
  \int \Phi^*\Phi d\mathbf{\tau} = 1,
\]
where we have used the shorthand $d\mathbf{\tau}=dx_1dr_2\dots dr_A$.

In the Hartree-Fock method the trial function is the Slater
determinant of Eq.~(\ref{eq:HartreeFockDet}) which can be rewritten as 
\[
  \Phi(x_1,x_2,\dots,x_A,\alpha,\beta,\dots,\nu) = \frac{1}{\sqrt{A!}}\sum_{P} (-)^P\hat{P}\psi_{\alpha}(x_1)
    \psi_{\beta}(x_2)\dots\psi_{\nu}(x_A)=\sqrt{A!}\hat{A}\Phi_H,
\]
where we have introduced the antisymmetrization operator $\hat{A}$ defined by the 
summation over all possible permutations of two particles.

It is defined as
\begin{equation}
  \hat{A} = \frac{1}{A!}\sum_{p} (-)^p\hat{P},
\label{antiSymmetryOperator}
\end{equation}
with $p$ standing for the number of permutations. We have introduced for later use the so-called
Hartree-function, defined by the simple product of all possible single-particle functions
\[
  \Phi_H(x_1,x_2,\dots,x_A,\alpha,\beta,\dots,\nu) =
  \psi_{\alpha}(x_1)
    \psi_{\beta}(x_2)\dots\psi_{\nu}(x_A).
\]

Both $\hat{H}_0$ and $\hat{H}_I$ are invariant under all possible permutations of any two particles
and hence commute with $\hat{A}$
\begin{equation}
  [H_0,\hat{A}] = [H_I,\hat{A}] = 0. \label{commutionAntiSym}
\end{equation}
Furthermore, $\hat{A}$ satisfies
\begin{equation}
  \hat{A}^2 = \hat{A},  \label{AntiSymSquared}
\end{equation}
since every permutation of the Slater
determinant reproduces it. 

The expectation value of $\hat{H}_0$ 
\[
  \int \Phi^*\hat{H}_0\Phi d\mathbf{\tau} 
  = A! \int \Phi_H^*\hat{A}\hat{H}_0\hat{A}\Phi_H d\mathbf{\tau}
\]
is readily reduced to
\[
  \int \Phi^*\hat{H}_0\Phi d\mathbf{\tau} 
  = A! \int \Phi_H^*\hat{H}_0\hat{A}\Phi_H d\mathbf{\tau},
\]
where we have used Eqs.~(\ref{commutionAntiSym}) and
(\ref{AntiSymSquared}). The next step is to replace the antisymmetrization
operator by its definition and to
replace $\hat{H}_0$ with the sum of one-body operators
\[
  \int \Phi^*\hat{H}_0\Phi  d\mathbf{\tau}
  = \sum_{i=1}^A \sum_{p} (-)^p\int 
  \Phi_H^*\hat{h}_0\hat{P}\Phi_H d\mathbf{\tau}.
\]

The integral vanishes if two or more particles are permuted in only one
of the Hartree-functions $\Phi_H$ because the individual single-particle wave functions are
orthogonal. We obtain then
\[
  \int \Phi^*\hat{H}_0\Phi  d\mathbf{\tau}= \sum_{i=1}^A \int \Phi_H^*\hat{h}_0\Phi_H  d\mathbf{\tau}.
\]
Orthogonality of the single-particle functions allows us to further simplify the integral, and we
arrive at the following expression for the expectation values of the
sum of one-body Hamiltonians 
\begin{equation}
  \int \Phi^*\hat{H}_0\Phi  d\mathbf{\tau}
  = \sum_{\mu=1}^A \int \psi_{\mu}^*(x)\hat{h}_0\psi_{\mu}(x)dx
  d\mathbf{r}.
  \label{H1Expectation}
\end{equation}

We introduce the following shorthand for the above integral
\[
\langle \mu | \hat{h}_0 | \mu \rangle = \int \psi_{\mu}^*(x)\hat{h}_0\psi_{\mu}(x)dx,
\]
and rewrite Eq.~(\ref{H1Expectation}) as
\begin{equation}
  \int \Phi^*\hat{H}_0\Phi  d\tau
  = \sum_{\mu=1}^A \langle \mu | \hat{h}_0 | \mu \rangle.
  \label{H1Expectation1}
\end{equation}

The expectation value of the two-body part of the Hamiltonian is obtained in a
similar manner. We have
\[
  \int \Phi^*\hat{H}_I\Phi d\mathbf{\tau} 
  = A! \int \Phi_H^*\hat{A}\hat{H}_I\hat{A}\Phi_H d\mathbf{\tau},
\]
which reduces to
\[
 \int \Phi^*\hat{H}_I\Phi d\mathbf{\tau} 
  = \sum_{i\le j=1}^A \sum_{p} (-)^p\int 
  \Phi_H^*\hat{v}(r_{ij})\hat{P}\Phi_H d\mathbf{\tau},
\]
by following the same arguments as for the one-body
Hamiltonian. 

Because of the dependence on the inter-particle distance $r_{ij}$,  permutations of
any two particles no longer vanish, and we get
\[
  \int \Phi^*\hat{H}_I\Phi d\mathbf{\tau} 
  = \sum_{i < j=1}^A \int  
  \Phi_H^*\hat{v}(r_{ij})(1-P_{ij})\Phi_H d\mathbf{\tau}.
\]
where $P_{ij}$ is the permutation operator that interchanges
particle $i$ and particle $j$. Again we use the assumption that the single-particle wave functions
are orthogonal. 

We obtain
\begin{align}
  \int \Phi^*\hat{H}_I\Phi d\mathbf{\tau} 
  = \frac{1}{2}\sum_{\mu=1}^A\sum_{\nu=1}^A
    &\left[ \int \psi_{\mu}^*(x_i)\psi_{\nu}^*(x_j)\hat{v}(r_{ij})\psi_{\mu}(x_i)\psi_{\nu}(x_j)
    dx_idx_j \right.\\
  &\left.
  - \int \psi_{\mu}^*(x_i)\psi_{\nu}^*(x_j)
  \hat{v}(r_{ij})\psi_{\nu}(x_i)\psi_{\mu}(x_j)
  dx_idx_j
  \right]. \label{H2Expectation}
\end{align}
The first term is the so-called direct term. It is frequently also called the  Hartree term, 
while the second is due to the Pauli principle and is called
the exchange term or just the Fock term.
The factor  $1/2$ is introduced because we now run over
all pairs twice. 

The last equation allows us to  introduce some further definitions.  
The single-particle wave functions $\psi_{\mu}(x)$, defined by the quantum numbers $\mu$ and $x$
are defined as the overlap 
\[
   \psi_{\alpha}(x)  = \langle x | \alpha \rangle .
\]

We introduce the following shorthands for the above two integrals
\[
\langle \mu\nu|\hat{v}|\mu\nu\rangle =  \int \psi_{\mu}^*(x_i)\psi_{\nu}^*(x_j)\hat{v}(r_{ij})\psi_{\mu}(x_i)\psi_{\nu}(x_j)
    dx_idx_j,
\]
and
\[
\langle \mu\nu|\hat{v}|\nu\mu\rangle = \int \psi_{\mu}^*(x_i)\psi_{\nu}^*(x_j)
  \hat{v}(r_{ij})\psi_{\nu}(x_i)\psi_{\mu}(x_j)
  dx_idx_j.  
\]

\subsection*{Preparing for later studies: varying the coefficients of a wave function expansion and orthogonal transformations}

It is common to  expand the single-particle functions in a known basis  and vary the coefficients, 
that is, the new single-particle wave function is written as a linear expansion
in terms of a fixed chosen orthogonal basis (for example the well-known harmonic oscillator functions or the hydrogen-like functions etc).
We define our new single-particle basis (this is a normal approach for Hartree-Fock theory) by performing a unitary transformation 
on our previous basis (labelled with greek indices) as
\begin{equation}
\psi_p^{new}  = \sum_{\lambda} C_{p\lambda}\phi_{\lambda}. \label{eq:newbasis}
\end{equation}
In this case we vary the coefficients $C_{p\lambda}$. If the basis has infinitely many solutions, we need
to truncate the above sum.  We assume that the basis $\phi_{\lambda}$ is orthogonal.

It is normal to choose a single-particle basis defined as the eigenfunctions
of parts of the full Hamiltonian. The typical situation consists of the solutions of the one-body part of the Hamiltonian, that is we have
\[
\hat{h}_0\phi_{\lambda}=\epsilon_{\lambda}\phi_{\lambda}.
\]
The single-particle wave functions $\phi_{\lambda}(\mathbf{r})$, defined by the quantum numbers $\lambda$ and $\mathbf{r}$
are defined as the overlap 
\[
   \phi_{\lambda}(\mathbf{r})  = \langle \mathbf{r} | \lambda \rangle .
\]

In deriving the Hartree-Fock equations, we  will expand the single-particle functions in a known basis  and vary the coefficients, 
that is, the new single-particle wave function is written as a linear expansion
in terms of a fixed chosen orthogonal basis (for example the well-known harmonic oscillator functions or the hydrogen-like functions etc).

We stated that a unitary transformation keeps the orthogonality. To see this consider first a basis of vectors $\mathbf{v}_i$,
\[
\mathbf{v}_i = \begin{bmatrix} v_{i1} \\ \dots \\ \dots \\v_{in} \end{bmatrix}
\]
We assume that the basis is orthogonal, that is 
\[
\mathbf{v}_j^T\mathbf{v}_i = \delta_{ij}.
\]
An orthogonal or unitary transformation
\[
\mathbf{w}_i=\mathbf{U}\mathbf{v}_i,
\]
preserves the dot product and orthogonality since
\[
\mathbf{w}_j^T\mathbf{w}_i=(\mathbf{U}\mathbf{v}_j)^T\mathbf{U}\mathbf{v}_i=\mathbf{v}_j^T\mathbf{U}^T\mathbf{U}\mathbf{v}_i= \mathbf{v}_j^T\mathbf{v}_i = \delta_{ij}.
\]

This means that if the coefficients $C_{p\lambda}$ belong to a unitary or orthogonal trasformation (using the Dirac bra-ket notation)
\[
\vert p\rangle  = \sum_{\lambda} C_{p\lambda}\vert\lambda\rangle,
\]
orthogonality is preserved, that is $\langle \alpha \vert \beta\rangle = \delta_{\alpha\beta}$
and $\langle p \vert q\rangle = \delta_{pq}$. 

This propertry is extremely useful when we build up a basis of many-body Stater determinant based states. 

\textbf{Note also that although a basis $\vert \alpha\rangle$ contains an infinity of states, for practical calculations we have always to make some truncations.} 

Before we develop for example the Hartree-Fock equations, there is another very useful property of determinants that we will use both in connection with Hartree-Fock calculations and later shell-model calculations.  

Consider the following determinant
\[
\left| \begin{array}{cc} \alpha_1b_{11}+\alpha_2sb_{12}& a_{12}\\
                         \alpha_1b_{21}+\alpha_2b_{22}&a_{22}\end{array} \right|=\alpha_1\left|\begin{array}{cc} b_{11}& a_{12}\\
                         b_{21}&a_{22}\end{array} \right|+\alpha_2\left| \begin{array}{cc} b_{12}& a_{12}\\b_{22}&a_{22}\end{array} \right|
\]

We can generalize this to  an $n\times n$ matrix and have 
\[
\left| \begin{array}{cccccc} a_{11}& a_{12} & \dots & \sum_{k=1}^n c_k b_{1k} &\dots & a_{1n}\\
a_{21}& a_{22} & \dots & \sum_{k=1}^n c_k b_{2k} &\dots & a_{2n}\\
\dots & \dots & \dots & \dots & \dots & \dots \\
\dots & \dots & \dots & \dots & \dots & \dots \\
a_{n1}& a_{n2} & \dots & \sum_{k=1}^n c_k b_{nk} &\dots & a_{nn}\end{array} \right|=
\sum_{k=1}^n c_k\left| \begin{array}{cccccc} a_{11}& a_{12} & \dots &  b_{1k} &\dots & a_{1n}\\
a_{21}& a_{22} & \dots &  b_{2k} &\dots & a_{2n}\\
\dots & \dots & \dots & \dots & \dots & \dots\\
\dots & \dots & \dots & \dots & \dots & \dots\\
a_{n1}& a_{n2} & \dots &  b_{nk} &\dots & a_{nn}\end{array} \right| .
\]
This is a property we will use in our Hartree-Fock discussions. 

We can generalize the previous results, now 
with all elements $a_{ij}$  being given as functions of 
linear combinations  of various coefficients $c$ and elements $b_{ij}$,
\[
\left| \begin{array}{cccccc} \sum_{k=1}^n b_{1k}c_{k1}& \sum_{k=1}^n b_{1k}c_{k2} & \dots & \sum_{k=1}^n b_{1k}c_{kj}  &\dots & \sum_{k=1}^n b_{1k}c_{kn}\\
\sum_{k=1}^n b_{2k}c_{k1}& \sum_{k=1}^n b_{2k}c_{k2} & \dots & \sum_{k=1}^n b_{2k}c_{kj} &\dots & \sum_{k=1}^n b_{2k}c_{kn}\\
\dots & \dots & \dots & \dots & \dots & \dots \\
\dots & \dots & \dots & \dots & \dots &\dots \\
\sum_{k=1}^n b_{nk}c_{k1}& \sum_{k=1}^n b_{nk}c_{k2} & \dots & \sum_{k=1}^n b_{nk}c_{kj} &\dots & \sum_{k=1}^n b_{nk}c_{kn}\end{array} \right|=det(\mathbf{C})det(\mathbf{B}),
\]
where $det(\mathbf{C})$ and $det(\mathbf{B})$ are the determinants of $n\times n$ matrices
with elements $c_{ij}$ and $b_{ij}$ respectively.  
This is a property we will use in our Hartree-Fock discussions. Convince yourself about the correctness of the above expression by setting $n=2$. 

With our definition of the new basis in terms of an orthogonal basis we have
\[
\psi_p(x)  = \sum_{\lambda} C_{p\lambda}\phi_{\lambda}(x).
\]
If the coefficients $C_{p\lambda}$ belong to an orthogonal or unitary matrix, the new basis
is also orthogonal. 
Our Slater determinant in the new basis $\psi_p(x)$ is written as
\[
\frac{1}{\sqrt{A!}}
\left| \begin{array}{ccccc} \psi_{p}(x_1)& \psi_{p}(x_2)& \dots & \dots & \psi_{p}(x_A)\\
                            \psi_{q}(x_1)&\psi_{q}(x_2)& \dots & \dots & \psi_{q}(x_A)\\  
                            \dots & \dots & \dots & \dots & \dots \\
                            \dots & \dots & \dots & \dots & \dots \\
                     \psi_{t}(x_1)&\psi_{t}(x_2)& \dots & \dots & \psi_{t}(x_A)\end{array} \right|=\frac{1}{\sqrt{A!}}
\left| \begin{array}{ccccc} \sum_{\lambda} C_{p\lambda}\phi_{\lambda}(x_1)& \sum_{\lambda} C_{p\lambda}\phi_{\lambda}(x_2)& \dots & \dots & \sum_{\lambda} C_{p\lambda}\phi_{\lambda}(x_A)\\
                            \sum_{\lambda} C_{q\lambda}\phi_{\lambda}(x_1)&\sum_{\lambda} C_{q\lambda}\phi_{\lambda}(x_2)& \dots & \dots & \sum_{\lambda} C_{q\lambda}\phi_{\lambda}(x_A)\\  
                            \dots & \dots & \dots & \dots & \dots \\
                            \dots & \dots & \dots & \dots & \dots \\
                     \sum_{\lambda} C_{t\lambda}\phi_{\lambda}(x_1)&\sum_{\lambda} C_{t\lambda}\phi_{\lambda}(x_2)& \dots & \dots & \sum_{\lambda} C_{t\lambda}\phi_{\lambda}(x_A)\end{array} \right|,
\]
which is nothing but $det(\mathbf{C})det(\Phi)$, with $det(\Phi)$ being the determinant given by the basis functions $\phi_{\lambda}(x)$. 

In our discussions hereafter we will use our definitions of single-particle states above and below the Fermi ($F$) level given by the labels
$ijkl\dots \le F$ for so-called single-hole states and $abcd\dots > F$ for so-called particle states.
For general single-particle states we employ the labels $pqrs\dots$. 

The energy functional is
\[
  E[\Phi] 
  = \sum_{\mu=1}^A \langle \mu | h | \mu \rangle +
  \frac{1}{2}\sum_{{\mu}=1}^A\sum_{{\nu}=1}^A \langle \mu\nu|\hat{v}|\mu\nu\rangle_{AS},
\]
we found the expression for the energy functional in terms of the basis function $\phi_{\lambda}(\mathbf{r})$. We then  varied the above energy functional with respect to the basis functions $|\mu \rangle$. 
Now we are interested in defining a new basis defined in terms of
a chosen basis as defined in Eq.~(\ref{eq:newbasis}). We can then rewrite the energy functional as
\begin{equation}
  E[\Phi^{New}] 
  = \sum_{i=1}^A \langle i | h | i \rangle +
  \frac{1}{2}\sum_{ij=1}^A\langle ij|\hat{v}|ij\rangle_{AS}, \label{FunctionalEPhi2}
\end{equation}
where $\Phi^{New}$ is the new Slater determinant defined by the new basis of Eq.~(\ref{eq:newbasis}). 

Using Eq.~(\ref{eq:newbasis}) we can rewrite Eq.~(\ref{FunctionalEPhi2}) as 
\begin{equation}
  E[\Psi] 
  = \sum_{i=1}^A \sum_{\alpha\beta} C^*_{i\alpha}C_{i\beta}\langle \alpha | h | \beta \rangle +
  \frac{1}{2}\sum_{ij=1}^A\sum_{{\alpha\beta\gamma\delta}} C^*_{i\alpha}C^*_{j\beta}C_{i\gamma}C_{j\delta}\langle \alpha\beta|\hat{v}|\gamma\delta\rangle_{AS}. \label{FunctionalEPhi3}
\end{equation}

\subsection*{Second quantization}

We introduce the time-independent  operators
$a_\alpha^{\dagger}$ and $a_\alpha$   which create and annihilate, respectively, a particle 
in the single-particle state 
$\varphi_\alpha$. 
We define the fermion creation operator
$a_\alpha^{\dagger}$ 
\begin{equation}
	a_\alpha^{\dagger}|0\rangle \equiv  |\alpha\rangle  \label{eq:2-1a},
\end{equation}
and
\begin{equation}
	a_\alpha^{\dagger}|\alpha_1\dots \alpha_n\rangle_{\mathrm{AS}} \equiv  |\alpha\alpha_1\dots \alpha_n\rangle_{\mathrm{AS}} \label{eq:2-1b}
\end{equation}

In Eq.~(\ref{eq:2-1a}) 
the operator  $a_\alpha^{\dagger}$  acts on the vacuum state 
$|0\rangle$, which does not contain any particles. Alternatively, we could define  a closed-shell nucleus or atom as our new vacuum, but then
we need to introduce the particle-hole  formalism, see the discussion to come. 

In Eq.~(\ref{eq:2-1b}) $a_\alpha^{\dagger}$ acts on an antisymmetric $n$-particle state and 
creates an antisymmetric $(n+1)$-particle state, where the one-body state 
$\varphi_\alpha$ is occupied, under the condition that
$\alpha \ne \alpha_1, \alpha_2, \dots, \alpha_n$. 
It follows that we can express an antisymmetric state as the product of the creation
operators acting on the vacuum state.  
\begin{equation}
	|\alpha_1\dots \alpha_n\rangle_{\mathrm{AS}} = a_{\alpha_1}^{\dagger} a_{\alpha_2}^{\dagger} \dots a_{\alpha_n}^{\dagger} |0\rangle \label{eq:2-2}
\end{equation}

It is easy to derive the commutation and anticommutation rules  for the fermionic creation operators 
$a_\alpha^{\dagger}$. Using the antisymmetry of the states 
(\ref{eq:2-2})
\begin{equation}
	|\alpha_1\dots \alpha_i\dots \alpha_k\dots \alpha_n\rangle_{\mathrm{AS}} = 
		- |\alpha_1\dots \alpha_k\dots \alpha_i\dots \alpha_n\rangle_{\mathrm{AS}} \label{eq:2-3a}
\end{equation}
we obtain
\begin{equation}
	 a_{\alpha_i}^{\dagger}  a_{\alpha_k}^{\dagger} = - a_{\alpha_k}^{\dagger} a_{\alpha_i}^{\dagger} \label{eq:2-3b}
\end{equation}

Using the Pauli principle
\begin{equation}
	|\alpha_1\dots \alpha_i\dots \alpha_i\dots \alpha_n\rangle_{\mathrm{AS}} = 0 \label{eq:2-4a}
\end{equation}
it follows that
\begin{equation}
	a_{\alpha_i}^{\dagger}  a_{\alpha_i}^{\dagger} = 0. \label{eq:2-4b}
\end{equation}
If we combine Eqs.~(\ref{eq:2-3b}) and (\ref{eq:2-4b}), we obtain the well-known anti-commutation rule
\begin{equation}
	a_{\alpha}^{\dagger}  a_{\beta}^{\dagger} + a_{\beta}^{\dagger}  a_{\alpha}^{\dagger} \equiv 
		\{a_{\alpha}^{\dagger},a_{\beta}^{\dagger}\} = 0 \label{eq:2-5}
\end{equation}

The hermitian conjugate  of $a_\alpha^{\dagger}$ is
\begin{equation}
	a_{\alpha} = ( a_{\alpha}^{\dagger} )^{\dagger} \label{eq:2-6}
\end{equation}
If we take the hermitian conjugate of Eq.~(\ref{eq:2-5}), we arrive at 
\begin{equation}
	\{a_{\alpha},a_{\beta}\} = 0 \label{eq:2-7}
\end{equation}

What is the physical interpretation of the operator $a_\alpha$ and what is the effect of 
$a_\alpha$ on a given state $|\alpha_1\alpha_2\dots\alpha_n\rangle_{\mathrm{AS}}$? 
Consider the following matrix element
\begin{equation}
	\langle\alpha_1\alpha_2 \dots \alpha_n|a_\alpha|\alpha_1'\alpha_2' \dots \alpha_m'\rangle \label{eq:2-8}
\end{equation}
where both sides are antisymmetric. We  distinguish between two cases. The first (1) is when
$\alpha \in \{\alpha_i\}$. Using the Pauli principle of Eq.~(\ref{eq:2-4a}) it follows
\begin{equation}
		\langle\alpha_1\alpha_2 \dots \alpha_n|a_\alpha = 0 \label{eq:2-9a}
\end{equation}
The second (2) case is when $\alpha \notin \{\alpha_i\}$. It follows that an hermitian conjugation
\begin{equation}
		\langle \alpha_1\alpha_2 \dots \alpha_n|a_\alpha = \langle\alpha\alpha_1\alpha_2 \dots \alpha_n|  \label{eq:2-9b}
\end{equation}

Eq.~(\ref{eq:2-9b}) holds for case (1) since the lefthand side is zero due to the Pauli principle. We write
Eq.~(\ref{eq:2-8}) as
\begin{equation}
	\langle\alpha_1\alpha_2 \dots \alpha_n|a_\alpha|\alpha_1'\alpha_2' \dots \alpha_m'\rangle = 
	\langle \alpha_1\alpha_2 \dots \alpha_n|\alpha\alpha_1'\alpha_2' \dots \alpha_m'\rangle \label{eq:2-10}
\end{equation}
Here we must have $m = n+1$ if Eq.~(\ref{eq:2-10}) has to be trivially different from zero.

For the last case, the minus and plus signs apply when the sequence 
$\alpha ,\alpha_1, \alpha_2, \dots, \alpha_n$ and 
$\alpha_1', \alpha_2', \dots, \alpha_{n+1}'$ are related to each other via even and odd permutations.
If we assume that  $\alpha \notin \{\alpha_i\}$ we obtain 
\begin{equation}
	\langle\alpha_1\alpha_2 \dots \alpha_n|a_\alpha|\alpha_1'\alpha_2' \dots \alpha_{n+1}'\rangle = 0 \label{eq:2-12}
\end{equation}
when $\alpha \in \{\alpha_i'\}$. If $\alpha \notin \{\alpha_i'\}$, we obtain
\begin{equation}
	a_\alpha\underbrace{|\alpha_1'\alpha_2' \dots \alpha_{n+1}'}\rangle_{\neq \alpha} = 0 \label{eq:2-13a}
\end{equation}
and in particular
\begin{equation}
	a_\alpha |0\rangle = 0 \label{eq:2-13b}
\end{equation}

If $\{\alpha\alpha_i\} = \{\alpha_i'\}$, performing the right permutations, the sequence
$\alpha ,\alpha_1, \alpha_2, \dots, \alpha_n$ is identical with the sequence
$\alpha_1', \alpha_2', \dots, \alpha_{n+1}'$. This results in
\begin{equation}
	\langle\alpha_1\alpha_2 \dots \alpha_n|a_\alpha|\alpha\alpha_1\alpha_2 \dots \alpha_{n}\rangle = 1 \label{eq:2-14}
\end{equation}
and thus
\begin{equation}
	a_\alpha |\alpha\alpha_1\alpha_2 \dots \alpha_{n}\rangle = |\alpha_1\alpha_2 \dots \alpha_{n}\rangle \label{eq:2-15}
\end{equation}

The action of the operator 
$a_\alpha$ from the left on a state vector  is to to remove  one particle in the state
$\alpha$. 
If the state vector does not contain the single-particle state $\alpha$, the outcome of the operation is zero.
The operator  $a_\alpha$ is normally called for a destruction or annihilation operator.

The next step is to establish the  commutator algebra of $a_\alpha^{\dagger}$ and
$a_\beta$. 

The action of the anti-commutator 
$\{a_\alpha^{\dagger}$,$a_\alpha\}$ on a given $n$-particle state is
\begin{align}
	a_\alpha^{\dagger} a_\alpha \underbrace{|\alpha_1\alpha_2 \dots \alpha_{n}\rangle}_{\neq \alpha} &= 0 \nonumber \\
	a_\alpha a_\alpha^{\dagger} \underbrace{|\alpha_1\alpha_2 \dots \alpha_{n}\rangle}_{\neq \alpha} &=
	a_\alpha \underbrace{|\alpha \alpha_1\alpha_2 \dots \alpha_{n}\rangle}_{\neq \alpha} = 
	\underbrace{|\alpha_1\alpha_2 \dots \alpha_{n}\rangle}_{\neq \alpha} \label{eq:2-16a}
\end{align}
if the single-particle state $\alpha$ is not contained in the state.

 If it is present
we arrive at
\begin{align}
	a_\alpha^{\dagger} a_\alpha |\alpha_1\alpha_2 \dots \alpha_{k}\alpha \alpha_{k+1} \dots \alpha_{n-1}\rangle &=
	a_\alpha^{\dagger} a_\alpha (-1)^k |\alpha \alpha_1\alpha_2 \dots \alpha_{n-1}\rangle \nonumber \\
	= (-1)^k |\alpha \alpha_1\alpha_2 \dots \alpha_{n-1}\rangle &=
	|\alpha_1\alpha_2 \dots \alpha_{k}\alpha \alpha_{k+1} \dots \alpha_{n-1}\rangle \nonumber \\
	a_\alpha a_\alpha^{\dagger}|\alpha_1\alpha_2 \dots \alpha_{k}\alpha \alpha_{k+1} \dots \alpha_{n-1}\rangle &= 0 \label{eq:2-16b}
\end{align}
From Eqs.~(\ref{eq:2-16a}) and  (\ref{eq:2-16b}) we arrive at 
\begin{equation}
	\{a_\alpha^{\dagger} , a_\alpha \} = a_\alpha^{\dagger} a_\alpha + a_\alpha a_\alpha^{\dagger} = 1 \label{eq:2-17}
\end{equation}

The action of $\left\{a_\alpha^{\dagger}, a_\beta\right\}$, with 
$\alpha \ne \beta$ on a given state yields three possibilities. 
The first case is a state vector which contains both $\alpha$ and $\beta$, then either 
$\alpha$ or $\beta$ and finally none of them.

The first case results in
\begin{align}
	a_\alpha^{\dagger} a_\beta |\alpha\beta\alpha_1\alpha_2 \dots \alpha_{n-2}\rangle = 0 \nonumber \\
	a_\beta a_\alpha^{\dagger} |\alpha\beta\alpha_1\alpha_2 \dots \alpha_{n-2}\rangle = 0 \label{eq:2-18a}
\end{align}
while the second case gives
\begin{align}
	 a_\alpha^{\dagger} a_\beta |\beta \underbrace{\alpha_1\alpha_2 \dots \alpha_{n-1}}_{\neq \alpha}\rangle =& 
	 	|\alpha \underbrace{\alpha_1\alpha_2 \dots \alpha_{n-1}}_{\neq  \alpha}\rangle \nonumber \\
	a_\beta a_\alpha^{\dagger} |\beta \underbrace{\alpha_1\alpha_2 \dots \alpha_{n-1}}_{\neq \alpha}\rangle =&
		a_\beta |\alpha\beta\underbrace{\beta \alpha_1\alpha_2 \dots \alpha_{n-1}}_{\neq \alpha}\rangle \nonumber \\
	=& - |\alpha\underbrace{\alpha_1\alpha_2 \dots \alpha_{n-1}}_{\neq \alpha}\rangle \label{eq:2-18b}
\end{align}

Finally if the state vector does not contain $\alpha$ and $\beta$
\begin{align}
	a_\alpha^{\dagger} a_\beta |\underbrace{\alpha_1\alpha_2 \dots \alpha_{n}}_{\neq \alpha,\beta}\rangle &=& 0 \nonumber \\
	a_\beta a_\alpha^{\dagger} |\underbrace{\alpha_1\alpha_2 \dots \alpha_{n}}_{\neq \alpha,\beta}\rangle &=& 
		a_\beta |\alpha \underbrace{\alpha_1\alpha_2 \dots \alpha_{n}}_{\neq \alpha,\beta}\rangle = 0 \label{eq:2-18c}
\end{align}
For all three cases we have
\begin{equation}
	\{a_\alpha^{\dagger},a_\beta \} = a_\alpha^{\dagger} a_\beta + a_\beta a_\alpha^{\dagger} = 0, \quad \alpha \neq \beta \label{eq:2-19}
\end{equation}

We can summarize  our findings in Eqs.~(\ref{eq:2-17}) and (\ref{eq:2-19}) as 
\begin{equation}
	\{a_\alpha^{\dagger},a_\beta \} = \delta_{\alpha\beta} \label{eq:2-20}
\end{equation}
with  $\delta_{\alpha\beta}$ is the Kroenecker $\delta$-symbol.

The properties of the creation and annihilation operators can be summarized as (for fermions)
\[
	a_\alpha^{\dagger}|0\rangle \equiv  |\alpha\rangle,
\]
and
\[
	a_\alpha^{\dagger}|\alpha_1\dots \alpha_n\rangle_{\mathrm{AS}} \equiv  |\alpha\alpha_1\dots \alpha_n\rangle_{\mathrm{AS}}. 
\]
from which follows
\[
        |\alpha_1\dots \alpha_n\rangle_{\mathrm{AS}} = a_{\alpha_1}^{\dagger} a_{\alpha_2}^{\dagger} \dots a_{\alpha_n}^{\dagger} |0\rangle.
\]

The hermitian conjugate has the folowing properties
\[
        a_{\alpha} = ( a_{\alpha}^{\dagger} )^{\dagger}.
\]
Finally we found 
\[
	a_\alpha\underbrace{|\alpha_1'\alpha_2' \dots \alpha_{n+1}'}\rangle_{\neq \alpha} = 0, \quad
		\textrm{in particular } a_\alpha |0\rangle = 0,
\]
and
\[
 a_\alpha |\alpha\alpha_1\alpha_2 \dots \alpha_{n}\rangle = |\alpha_1\alpha_2 \dots \alpha_{n}\rangle,
\]
and the corresponding commutator algebra
\[
	\{a_{\alpha}^{\dagger},a_{\beta}^{\dagger}\} = \{a_{\alpha},a_{\beta}\} = 0 \hspace{0.5cm} 
\{a_\alpha^{\dagger},a_\beta \} = \delta_{\alpha\beta}.
\]

\subsection*{One-body operators in second quantization}

A very useful operator is the so-called number-operator.  Most physics cases  we will
study in this text conserve the total number of particles.  The number operator is therefore
a useful quantity which allows us to test that our many-body formalism  conserves the number of particles.
In for example $(d,p)$ or $(p,d)$ reactions it is important to be able to describe quantum mechanical states
where particles get added or removed.
A creation operator $a_\alpha^{\dagger}$ adds one particle to the single-particle state
$\alpha$ of a give many-body state vector, while an annihilation operator $a_\alpha$ 
removes a particle from a single-particle
state $\alpha$. 

Let us consider an operator proportional with $a_\alpha^{\dagger} a_\beta$ and 
$\alpha=\beta$. It acts on an $n$-particle state 
resulting in
\begin{equation}
	a_\alpha^{\dagger} a_\alpha |\alpha_1\alpha_2 \dots \alpha_{n}\rangle = 
	\begin{cases}
		0  &\alpha \notin \{\alpha_i\} \\
		\\
		|\alpha_1\alpha_2 \dots \alpha_{n}\rangle & \alpha \in \{\alpha_i\}
	\end{cases}
\end{equation}
Summing over all possible one-particle states we arrive at
\begin{equation}
	\left( \sum_\alpha a_\alpha^{\dagger} a_\alpha \right) |\alpha_1\alpha_2 \dots \alpha_{n}\rangle = 
	n |\alpha_1\alpha_2 \dots \alpha_{n}\rangle \label{eq:2-21}
\end{equation}

The operator 
\begin{equation}
	\hat{N} = \sum_\alpha a_\alpha^{\dagger} a_\alpha \label{eq:2-22}
\end{equation}
is called the number operator since it counts the number of particles in a give state vector when it acts 
on the different single-particle states.  It acts on one single-particle state at the time and falls 
therefore under category one-body operators.
Next we look at another important one-body operator, namely $\hat{H}_0$ and study its operator form in the 
occupation number representation.

We want to obtain an expression for a one-body operator which conserves the number of particles.
Here we study the one-body operator for the kinetic energy plus an eventual external one-body potential.
The action of this operator on a particular $n$-body state with its pertinent expectation value has already
been studied in coordinate  space.
In coordinate space the operator reads
\begin{equation}
	\hat{H}_0 = \sum_i \hat{h}_0(x_i) \label{eq:2-23}
\end{equation}
and the anti-symmetric $n$-particle Slater determinant is defined as 
\[
\Phi(x_1, x_2,\dots ,x_n,\alpha_1,\alpha_2,\dots, \alpha_n)= \frac{1}{\sqrt{n!}} \sum_p (-1)^p\hat{P}\psi_{\alpha_1}(x_1)\psi_{\alpha_2}(x_2) \dots \psi_{\alpha_n}(x_n).
\]

Defining
\begin{equation}
	\hat{h}_0(x_i) \psi_{\alpha_i}(x_i) = \sum_{\alpha_k'} \psi_{\alpha_k'}(x_i) \langle\alpha_k'|\hat{h}_0|\alpha_k\rangle \label{eq:2-25}
\end{equation}
we can easily  evaluate the action of $\hat{H}_0$ on each product of one-particle functions in Slater determinant.
From Eq.~(\ref{eq:2-25})  we obtain the following result without  permuting any particle pair 
\begin{align}
	&& \left( \sum_i \hat{h}_0(x_i) \right) \psi_{\alpha_1}(x_1)\psi_{\alpha_2}(x_2) \dots \psi_{\alpha_n}(x_n) \nonumber \\
	& =&\sum_{\alpha_1'} \langle \alpha_1'|\hat{h}_0|\alpha_1\rangle 
		\psi_{\alpha_1'}(x_1)\psi_{\alpha_2}(x_2) \dots \psi_{\alpha_n}(x_n) \nonumber \\
	&+&\sum_{\alpha_2'} \langle \alpha_2'|\hat{h}_0|\alpha_2\rangle
		\psi_{\alpha_1}(x_1)\psi_{\alpha_2'}(x_2) \dots \psi_{\alpha_n}(x_n) \nonumber \\
	&+& \dots \nonumber \\
	&+&\sum_{\alpha_n'} \langle \alpha_n'|\hat{h}_0|\alpha_n\rangle
		\psi_{\alpha_1}(x_1)\psi_{\alpha_2}(x_2) \dots \psi_{\alpha_n'}(x_n) \label{eq:2-26}
\end{align}

If we interchange particles $1$ and $2$  we obtain
\begin{align}
	&& \left( \sum_i \hat{h}_0(x_i) \right) \psi_{\alpha_1}(x_2)\psi_{\alpha_1}(x_2) \dots \psi_{\alpha_n}(x_n) \nonumber \\
	& =&\sum_{\alpha_2'} \langle \alpha_2'|\hat{h}_0|\alpha_2\rangle 
		\psi_{\alpha_1}(x_2)\psi_{\alpha_2'}(x_1) \dots \psi_{\alpha_n}(x_n) \nonumber \\
	&+&\sum_{\alpha_1'} \langle \alpha_1'|\hat{h}_0|\alpha_1\rangle
		\psi_{\alpha_1'}(x_2)\psi_{\alpha_2}(x_1) \dots \psi_{\alpha_n}(x_n) \nonumber \\
	&+& \dots \nonumber \\
	&+&\sum_{\alpha_n'} \langle \alpha_n'|\hat{h}_0|\alpha_n\rangle
		\psi_{\alpha_1}(x_2)\psi_{\alpha_1}(x_2) \dots \psi_{\alpha_n'}(x_n) \label{eq:2-27}
\end{align}

We can continue by computing all possible permutations. 
We rewrite also our Slater determinant in its second quantized form and skip the dependence on the quantum numbers $x_i.$
Summing up all contributions and taking care of all phases
$(-1)^p$ we arrive at 
\begin{align}
	\hat{H}_0|\alpha_1,\alpha_2,\dots, \alpha_n\rangle &=& \sum_{\alpha_1'}\langle \alpha_1'|\hat{h}_0|\alpha_1\rangle
		|\alpha_1'\alpha_2 \dots \alpha_{n}\rangle \nonumber \\
	&+& \sum_{\alpha_2'} \langle \alpha_2'|\hat{h}_0|\alpha_2\rangle
		|\alpha_1\alpha_2' \dots \alpha_{n}\rangle \nonumber \\
	&+& \dots \nonumber \\
	&+& \sum_{\alpha_n'} \langle \alpha_n'|\hat{h}_0|\alpha_n\rangle
		|\alpha_1\alpha_2 \dots \alpha_{n}'\rangle \label{eq:2-28}
\end{align}

In Eq.~(\ref{eq:2-28}) 
we have expressed the action of the one-body operator
of Eq.~(\ref{eq:2-23}) on the  $n$-body state in its second quantized form.
This equation can be further manipulated if we use the properties of the creation and annihilation operator
on each primed quantum number, that is
\begin{equation}
	|\alpha_1\alpha_2 \dots \alpha_k' \dots \alpha_{n}\rangle = 
		a_{\alpha_k'}^{\dagger}  a_{\alpha_k} |\alpha_1\alpha_2 \dots \alpha_k \dots \alpha_{n}\rangle \label{eq:2-29}
\end{equation}
Inserting this in the right-hand side of Eq.~(\ref{eq:2-28}) results in
\begin{align}
	\hat{H}_0|\alpha_1\alpha_2 \dots \alpha_{n}\rangle &=& \sum_{\alpha_1'}\langle \alpha_1'|\hat{h}_0|\alpha_1\rangle
		a_{\alpha_1'}^{\dagger}  a_{\alpha_1} |\alpha_1\alpha_2 \dots \alpha_{n}\rangle \nonumber \\
	&+& \sum_{\alpha_2'} \langle \alpha_2'|\hat{h}_0|\alpha_2\rangle
		a_{\alpha_2'}^{\dagger}  a_{\alpha_2} |\alpha_1\alpha_2 \dots \alpha_{n}\rangle \nonumber \\
	&+& \dots \nonumber \\
	&+& \sum_{\alpha_n'} \langle \alpha_n'|\hat{h}_0|\alpha_n\rangle
		a_{\alpha_n'}^{\dagger}  a_{\alpha_n} |\alpha_1\alpha_2 \dots \alpha_{n}\rangle \nonumber \\
	&=& \sum_{\alpha, \beta} \langle \alpha|\hat{h}_0|\beta\rangle a_\alpha^{\dagger} a_\beta 
		|\alpha_1\alpha_2 \dots \alpha_{n}\rangle \label{eq:2-30a}
\end{align}

In the number occupation representation or second quantization we get the following expression for a one-body 
operator which conserves the number of particles
\begin{equation}
	\hat{H}_0 = \sum_{\alpha\beta} \langle \alpha|\hat{h}_0|\beta\rangle a_\alpha^{\dagger} a_\beta \label{eq:2-30b}
\end{equation}
Obviously, $\hat{H}_0$ can be replaced by any other one-body  operator which preserved the number
of particles. The stucture of the operator is therefore not limited to say the kinetic or single-particle energy only.

The opearator $\hat{H}_0$ takes a particle from the single-particle state $\beta$  to the single-particle state $\alpha$ 
with a probability for the transition given by the expectation value $\langle \alpha|\hat{h}_0|\beta\rangle$.

It is instructive to verify Eq.~(\ref{eq:2-30b}) by computing the expectation value of $\hat{H}_0$ 
between two single-particle states
\begin{equation}
	\langle \alpha_1|\hat{h}_0|\alpha_2\rangle = \sum_{\alpha\beta} \langle \alpha|\hat{h}_0|\beta\rangle
		\langle 0|a_{\alpha_1}a_\alpha^{\dagger} a_\beta a_{\alpha_2}^{\dagger}|0\rangle \label{eq:2-30c}
\end{equation}

Using the commutation relations for the creation and annihilation operators we have 
\begin{equation}
a_{\alpha_1}a_\alpha^{\dagger} a_\beta a_{\alpha_2}^{\dagger} = (\delta_{\alpha \alpha_1} - a_\alpha^{\dagger} a_{\alpha_1} )(\delta_{\beta \alpha_2} - a_{\alpha_2}^{\dagger} a_{\beta} ), \label{eq:2-30d}
\end{equation}
which results in
\begin{equation}
\langle 0|a_{\alpha_1}a_\alpha^{\dagger} a_\beta a_{\alpha_2}^{\dagger}|0\rangle = \delta_{\alpha \alpha_1} \delta_{\beta \alpha_2} \label{eq:2-30e}
\end{equation}
and
\begin{equation}
\langle \alpha_1|\hat{h}_0|\alpha_2\rangle = \sum_{\alpha\beta} \langle \alpha|\hat{h}_0|\beta\rangle\delta_{\alpha \alpha_1} \delta_{\beta \alpha_2} = \langle \alpha_1|\hat{h}_0|\alpha_2\rangle \label{eq:2-30f}
\end{equation}

\subsection*{Two-body operators in second quantization}

Let us now derive the expression for our two-body interaction part, which also conserves the number of particles.
We can proceed in exactly the same way as for the one-body operator. In the coordinate representation our
two-body interaction part takes the following expression
\begin{equation}
	\hat{H}_I = \sum_{i < j} V(x_i,x_j) \label{eq:2-31}
\end{equation}
where the summation runs over distinct pairs. The term $V$ can be an interaction model for the nucleon-nucleon interaction
or the interaction between two electrons. It can also include additional two-body interaction terms. 

The action of this operator on a product of 
two single-particle functions is defined as 
\begin{equation}
	V(x_i,x_j) \psi_{\alpha_k}(x_i) \psi_{\alpha_l}(x_j) = \sum_{\alpha_k'\alpha_l'} 
		\psi_{\alpha_k}'(x_i)\psi_{\alpha_l}'(x_j) 
		\langle \alpha_k'\alpha_l'|\hat{v}|\alpha_k\alpha_l\rangle \label{eq:2-32}
\end{equation}

We can now let $\hat{H}_I$ act on all terms in the linear combination for $|\alpha_1\alpha_2\dots\alpha_n\rangle$. Without any permutations we have
\begin{align}
	&& \left( \sum_{i < j} V(x_i,x_j) \right) \psi_{\alpha_1}(x_1)\psi_{\alpha_2}(x_2)\dots \psi_{\alpha_n}(x_n) \nonumber \\
	&=& \sum_{\alpha_1'\alpha_2'} \langle \alpha_1'\alpha_2'|\hat{v}|\alpha_1\alpha_2\rangle
		\psi_{\alpha_1}'(x_1)\psi_{\alpha_2}'(x_2)\dots \psi_{\alpha_n}(x_n) \nonumber \\
	& +& \dots \nonumber \\
	&+& \sum_{\alpha_1'\alpha_n'} \langle \alpha_1'\alpha_n'|\hat{v}|\alpha_1\alpha_n\rangle
		\psi_{\alpha_1}'(x_1)\psi_{\alpha_2}(x_2)\dots \psi_{\alpha_n}'(x_n) \nonumber \\
	& +& \dots \nonumber \\
	&+& \sum_{\alpha_2'\alpha_n'} \langle \alpha_2'\alpha_n'|\hat{v}|\alpha_2\alpha_n\rangle
		\psi_{\alpha_1}(x_1)\psi_{\alpha_2}'(x_2)\dots \psi_{\alpha_n}'(x_n) \nonumber \\
	 & +& \dots \label{eq:2-33}
\end{align}
where on the rhs we have a term for each distinct pairs. 

For the other terms on the rhs we obtain similar expressions  and summing over all terms we obtain
\begin{align}
	H_I |\alpha_1\alpha_2\dots\alpha_n\rangle &=& \sum_{\alpha_1', \alpha_2'} \langle \alpha_1'\alpha_2'|\hat{v}|\alpha_1\alpha_2\rangle
		|\alpha_1'\alpha_2'\dots\alpha_n\rangle \nonumber \\
	&+& \dots \nonumber \\
	&+& \sum_{\alpha_1', \alpha_n'} \langle \alpha_1'\alpha_n'|\hat{v}|\alpha_1\alpha_n\rangle
		|\alpha_1'\alpha_2\dots\alpha_n'\rangle \nonumber \\
	&+& \dots \nonumber \\
	&+& \sum_{\alpha_2', \alpha_n'} \langle \alpha_2'\alpha_n'|\hat{v}|\alpha_2\alpha_n\rangle
		|\alpha_1\alpha_2'\dots\alpha_n'\rangle \nonumber \\
	 &+& \dots \label{eq:2-34}
\end{align}

We introduce second quantization via the relation
\begin{align}
	&& a_{\alpha_k'}^{\dagger} a_{\alpha_l'}^{\dagger} a_{\alpha_l} a_{\alpha_k} 
		|\alpha_1\alpha_2\dots\alpha_k\dots\alpha_l\dots\alpha_n\rangle \nonumber \\
	&=& (-1)^{k-1} (-1)^{l-2} a_{\alpha_k'}^{\dagger} a_{\alpha_l'}^{\dagger} a_{\alpha_l} a_{\alpha_k}
		|\alpha_k\alpha_l \underbrace{\alpha_1\alpha_2\dots\alpha_n}_{\neq \alpha_k,\alpha_l}\rangle \nonumber \\
	&=& (-1)^{k-1} (-1)^{l-2} 
	|\alpha_k'\alpha_l' \underbrace{\alpha_1\alpha_2\dots\alpha_n}_{\neq \alpha_k',\alpha_l'}\rangle \nonumber \\
	&=& |\alpha_1\alpha_2\dots\alpha_k'\dots\alpha_l'\dots\alpha_n\rangle \label{eq:2-35}
\end{align}

Inserting this in (\ref{eq:2-34}) gives
\begin{align}
	H_I |\alpha_1\alpha_2\dots\alpha_n\rangle
	&=& \sum_{\alpha_1', \alpha_2'} \langle \alpha_1'\alpha_2'|\hat{v}|\alpha_1\alpha_2\rangle
		a_{\alpha_1'}^{\dagger} a_{\alpha_2'}^{\dagger} a_{\alpha_2} a_{\alpha_1}
		|\alpha_1\alpha_2\dots\alpha_n\rangle \nonumber \\
	&+& \dots \nonumber \\
	&=& \sum_{\alpha_1', \alpha_n'} \langle \alpha_1'\alpha_n'|\hat{v}|\alpha_1\alpha_n\rangle
		a_{\alpha_1'}^{\dagger} a_{\alpha_n'}^{\dagger} a_{\alpha_n} a_{\alpha_1}
		|\alpha_1\alpha_2\dots\alpha_n\rangle \nonumber \\
	&+& \dots \nonumber \\
	&=& \sum_{\alpha_2', \alpha_n'} \langle \alpha_2'\alpha_n'|\hat{v}|\alpha_2\alpha_n\rangle
		a_{\alpha_2'}^{\dagger} a_{\alpha_n'}^{\dagger} a_{\alpha_n} a_{\alpha_2}
		|\alpha_1\alpha_2\dots\alpha_n\rangle \nonumber \\
	&+& \dots \nonumber \\
	&=& \sum_{\alpha, \beta, \gamma, \delta} ' \langle \alpha\beta|\hat{v}|\gamma\delta\rangle
		a^{\dagger}_\alpha a^{\dagger}_\beta a_\delta a_\gamma
		|\alpha_1\alpha_2\dots\alpha_n\rangle \label{eq:2-36}
\end{align}

Here we let $\sum'$ indicate that the sums running over $\alpha$ and $\beta$ run over all
single-particle states, while the summations  $\gamma$ and $\delta$ 
run over all pairs of single-particle states. We wish to remove this restriction and since
\begin{equation}
	\langle \alpha\beta|\hat{v}|\gamma\delta\rangle = \langle \beta\alpha|\hat{v}|\delta\gamma\rangle \label{eq:2-37}
\end{equation}
we get
\begin{align}
	\sum_{\alpha\beta} \langle \alpha\beta|\hat{v}|\gamma\delta\rangle a^{\dagger}_\alpha a^{\dagger}_\beta a_\delta a_\gamma &=& 
		\sum_{\alpha\beta} \langle \beta\alpha|\hat{v}|\delta\gamma\rangle 
		a^{\dagger}_\alpha a^{\dagger}_\beta a_\delta a_\gamma \label{eq:2-38a} \\
	&=& \sum_{\alpha\beta}\langle \beta\alpha|\hat{v}|\delta\gamma\rangle
		a^{\dagger}_\beta a^{\dagger}_\alpha a_\gamma a_\delta \label{eq:2-38b}
\end{align}
where we  have used the anti-commutation rules.

Changing the summation indices 
$\alpha$ and $\beta$ in (\ref{eq:2-38b}) we obtain
\begin{equation}
	\sum_{\alpha\beta} \langle \alpha\beta|\hat{v}|\gamma\delta\rangle a^{\dagger}_\alpha a^{\dagger}_\beta a_\delta a_\gamma =
		 \sum_{\alpha\beta} \langle \alpha\beta|\hat{v}|\delta\gamma\rangle 
		  a^{\dagger}_\alpha a^{\dagger}_\beta  a_\gamma a_\delta \label{eq:2-38c}
\end{equation}
From this it follows that the restriction on the summation over $\gamma$ and $\delta$ can be removed if we multiply with a factor $\frac{1}{2}$, resulting in 
\begin{equation}
	\hat{H}_I = \frac{1}{2} \sum_{\alpha\beta\gamma\delta} \langle \alpha\beta|\hat{v}|\gamma\delta\rangle
		a^{\dagger}_\alpha a^{\dagger}_\beta a_\delta a_\gamma \label{eq:2-39}
\end{equation}
where we sum freely over all single-particle states $\alpha$, 
$\beta$, $\gamma$ og $\delta$.

With this expression we can now verify that the second quantization form of $\hat{H}_I$ in Eq.~(\ref{eq:2-39}) 
results in the same matrix between two anti-symmetrized two-particle states as its corresponding coordinate
space representation. We have  
\begin{equation}
	\langle \alpha_1 \alpha_2|\hat{H}_I|\beta_1 \beta_2\rangle =
		\frac{1}{2} \sum_{\alpha\beta\gamma\delta}
			\langle \alpha\beta|\hat{v}|\gamma\delta\rangle \langle 0|a_{\alpha_2} a_{\alpha_1} 
			 a^{\dagger}_\alpha a^{\dagger}_\beta a_\delta a_\gamma 
			 a_{\beta_1}^{\dagger} a_{\beta_2}^{\dagger}|0\rangle. \label{eq:2-40}
\end{equation}

Using the commutation relations we get 
\begin{align}
	&& a_{\alpha_2} a_{\alpha_1}a^{\dagger}_\alpha a^{\dagger}_\beta 
		a_\delta a_\gamma a_{\beta_1}^{\dagger} a_{\beta_2}^{\dagger} \nonumber \\
	&=& a_{\alpha_2} a_{\alpha_1}a^{\dagger}_\alpha a^{\dagger}_\beta 
		( a_\delta \delta_{\gamma \beta_1} a_{\beta_2}^{\dagger} - 
		a_\delta  a_{\beta_1}^{\dagger} a_\gamma a_{\beta_2}^{\dagger} ) \nonumber \\
	&=& a_{\alpha_2} a_{\alpha_1}a^{\dagger}_\alpha a^{\dagger}_\beta 
		(\delta_{\gamma \beta_1} \delta_{\delta \beta_2} - \delta_{\gamma \beta_1} a_{\beta_2}^{\dagger} a_\delta -
		a_\delta a_{\beta_1}^{\dagger}\delta_{\gamma \beta_2} +
		a_\delta a_{\beta_1}^{\dagger} a_{\beta_2}^{\dagger} a_\gamma ) \nonumber \\
	&=& a_{\alpha_2} a_{\alpha_1}a^{\dagger}_\alpha a^{\dagger}_\beta 
		(\delta_{\gamma \beta_1} \delta_{\delta \beta_2} - \delta_{\gamma \beta_1} a_{\beta_2}^{\dagger} a_\delta \nonumber \\
		&& \qquad - \delta_{\delta \beta_1} \delta_{\gamma \beta_2} + \delta_{\gamma \beta_2} a_{\beta_1}^{\dagger} a_\delta
		+ a_\delta a_{\beta_1}^{\dagger} a_{\beta_2}^{\dagger} a_\gamma ) \label{eq:2-41}
\end{align}

The vacuum expectation value of this product of operators becomes
\begin{align}
	&& \langle 0|a_{\alpha_2} a_{\alpha_1} a^{\dagger}_\alpha a^{\dagger}_\beta a_\delta a_\gamma 
		a_{\beta_1}^{\dagger} a_{\beta_2}^{\dagger}|0\rangle \nonumber \\
	&=& (\delta_{\gamma \beta_1} \delta_{\delta \beta_2} -
		\delta_{\delta \beta_1} \delta_{\gamma \beta_2} ) 
		\langle 0|a_{\alpha_2} a_{\alpha_1}a^{\dagger}_\alpha a^{\dagger}_\beta|0\rangle \nonumber \\
	&=& (\delta_{\gamma \beta_1} \delta_{\delta \beta_2} -\delta_{\delta \beta_1} \delta_{\gamma \beta_2} )
	(\delta_{\alpha \alpha_1} \delta_{\beta \alpha_2} -\delta_{\beta \alpha_1} \delta_{\alpha \alpha_2} ) \label{eq:2-42b}
\end{align}

Insertion of 
Eq.~(\ref{eq:2-42b}) in Eq.~(\ref{eq:2-40}) results in
\begin{align}
	\langle \alpha_1\alpha_2|\hat{H}_I|\beta_1\beta_2\rangle &=& \frac{1}{2} \big[ 
		\langle \alpha_1\alpha_2|\hat{v}|\beta_1\beta_2\rangle- \langle \alpha_1\alpha_2|\hat{v}|\beta_2\beta_1\rangle \nonumber \\
		&& - \langle \alpha_2\alpha_1|\hat{v}|\beta_1\beta_2\rangle + \langle \alpha_2\alpha_1|\hat{v}|\beta_2\beta_1\rangle \big] \nonumber \\
	&=& \langle \alpha_1\alpha_2|\hat{v}|\beta_1\beta_2\rangle - \langle \alpha_1\alpha_2|\hat{v}|\beta_2\beta_1\rangle \nonumber \\
	&=& \langle \alpha_1\alpha_2|\hat{v}|\beta_1\beta_2\rangle_{\mathrm{AS}}. \label{eq:2-43b}
\end{align}

The two-body operator can also be expressed in terms of the anti-symmetrized matrix elements we discussed previously as
\begin{align}
	\hat{H}_I &=& \frac{1}{2} \sum_{\alpha\beta\gamma\delta}  \langle \alpha \beta|\hat{v}|\gamma \delta\rangle
		a_\alpha^{\dagger} a_\beta^{\dagger} a_\delta a_\gamma \nonumber \\
	&=& \frac{1}{4} \sum_{\alpha\beta\gamma\delta} \left[ \langle \alpha \beta|\hat{v}|\gamma \delta\rangle -
		\langle \alpha \beta|\hat{v}|\delta\gamma \rangle \right] 
		a_\alpha^{\dagger} a_\beta^{\dagger} a_\delta a_\gamma \nonumber \\
	&=& \frac{1}{4} \sum_{\alpha\beta\gamma\delta} \langle \alpha \beta|\hat{v}|\gamma \delta\rangle_{\mathrm{AS}}
		a_\alpha^{\dagger} a_\beta^{\dagger} a_\delta a_\gamma \label{eq:2-45}
\end{align}

The factors in front of the operator, either  $\frac{1}{4}$ or 
$\frac{1}{2}$ tells whether we use antisymmetrized matrix elements or not. 

We can now express the Hamiltonian operator for a many-fermion system  in the occupation basis representation
as  
\begin{equation}
	H = \sum_{\alpha, \beta} \langle \alpha|\hat{t}+\hat{u}_{\mathrm{ext}}|\beta\rangle a_\alpha^{\dagger} a_\beta + \frac{1}{4} \sum_{\alpha\beta\gamma\delta}
		\langle \alpha \beta|\hat{v}|\gamma \delta\rangle a_\alpha^{\dagger} a_\beta^{\dagger} a_\delta a_\gamma. \label{eq:2-46b}
\end{equation}
This is the form we will use in the rest of these lectures, assuming that we work with anti-symmetrized two-body matrix elements.

\subsection*{Particle-hole formalism}

Second quantization is a useful and elegant formalism  for constructing many-body  states and 
quantum mechanical operators. One can express and translate many physical processes
into simple pictures such as Feynman diagrams. Expecation values of many-body states are also easily calculated.
However, although the equations are seemingly easy to set up, from  a practical point of view, that is
the solution of Schroedinger's equation, there is no particular gain.
The many-body equation is equally hard to solve, irrespective of representation. 
The cliche that 
there is no free lunch brings us down to earth again.  
Note however that a transformation to a particular
basis, for cases where the interaction obeys specific symmetries, can ease the solution of Schroedinger's equation. 

But there is at least one important case where second quantization comes to our rescue.
It is namely easy to introduce another reference state than the pure vacuum $|0\rangle $, where all single-particle states are active.
With many particles present it is often useful to introduce another reference state  than the vacuum state$|0\rangle $. We will label this state $|c\rangle$ ($c$ for core) and as we will see it can reduce 
considerably the complexity and thereby the dimensionality of the many-body problem. It allows us to sum up to infinite order specific many-body correlations.  The particle-hole representation is one of these handy representations. 

In the original particle representation these states are products of the creation operators  $a_{\alpha_i}^\dagger$ acting on the true vacuum $|0\rangle $.
Following Eq.~(\ref{eq:2-2}) we have 
\begin{align}
 |\alpha_1\alpha_2\dots\alpha_{n-1}\alpha_n\rangle &=& a_{\alpha_1}^\dagger a_{\alpha_2}^\dagger \dots
					a_{\alpha_{n-1}}^\dagger a_{\alpha_n}^\dagger |0\rangle  \label{eq:2-47a} \\
	|\alpha_1\alpha_2\dots\alpha_{n-1}\alpha_n\alpha_{n+1}\rangle &=&
		a_{\alpha_1}^\dagger a_{\alpha_2}^\dagger \dots a_{\alpha_{n-1}}^\dagger a_{\alpha_n}^\dagger
		a_{\alpha_{n+1}}^\dagger |0\rangle  \label{eq:2-47b} \\
	|\alpha_1\alpha_2\dots\alpha_{n-1}\rangle &=& a_{\alpha_1}^\dagger a_{\alpha_2}^\dagger \dots
		a_{\alpha_{n-1}}^\dagger |0\rangle  \label{eq:2-47c}
\end{align}

If we use Eq.~(\ref{eq:2-47a}) as our new reference state, we can simplify considerably the representation of 
this state
\begin{equation}
	|c\rangle  \equiv |\alpha_1\alpha_2\dots\alpha_{n-1}\alpha_n\rangle =
		a_{\alpha_1}^\dagger a_{\alpha_2}^\dagger \dots a_{\alpha_{n-1}}^\dagger a_{\alpha_n}^\dagger |0\rangle  \label{eq:2-48a}
\end{equation}
The new reference states for the $n+1$ and $n-1$ states can then be written as
\begin{align}
	|\alpha_1\alpha_2\dots\alpha_{n-1}\alpha_n\alpha_{n+1}\rangle &=& (-1)^n a_{\alpha_{n+1}}^\dagger |c\rangle 
		\equiv (-1)^n |\alpha_{n+1}\rangle_c \label{eq:2-48b} \\
	|\alpha_1\alpha_2\dots\alpha_{n-1}\rangle &=& (-1)^{n-1} a_{\alpha_n} |c\rangle  
		\equiv (-1)^{n-1} |\alpha_{n-1}\rangle_c \label{eq:2-48c} 
\end{align}

The first state has one additional particle with respect to the new vacuum state
$|c\rangle $  and is normally referred to as a one-particle state or one particle added to the 
many-body reference state. 
The second state has one particle less than the reference vacuum state  $|c\rangle $ and is referred to as
a one-hole state. 
When dealing with a new reference state it is often convenient to introduce 
new creation and annihilation operators since we have 
from Eq.~(\ref{eq:2-48c})
\begin{equation}
	a_\alpha |c\rangle  \neq 0 \label{eq:2-49}
\end{equation}
since  $\alpha$ is contained  in $|c\rangle $, while for the true vacuum we have 
$a_\alpha |0\rangle  = 0$ for all $\alpha$.

The new reference state leads to the definition of new creation and annihilation operators
which satisfy the following relations
\begin{align}
	b_\alpha |c\rangle  &=& 0 \label{eq:2-50a} \\
	\{b_\alpha^\dagger , b_\beta^\dagger \} = \{b_\alpha , b_\beta \} &=& 0 \nonumber  \\
	\{b_\alpha^\dagger , b_\beta \} &=& \delta_{\alpha \beta} \label{eq:2-50c}
\end{align}
We assume also that the new reference state is properly normalized
\begin{equation}
	\langle c | c \rangle = 1 \label{eq:2-51}
\end{equation}

The physical interpretation of these new operators is that of so-called quasiparticle states.
This means that a state defined by the addition of one extra particle to a reference state $|c\rangle $ may not necesseraly be interpreted as one particle coupled to a core.
We define now new creation operators that act on a state $\alpha$ creating a new quasiparticle state
\begin{equation}
	b_\alpha^\dagger|c\rangle  = \Bigg\{ \begin{array}{ll}
		a_\alpha^\dagger |c\rangle  = |\alpha\rangle, & \alpha > F \\
		\\
		a_\alpha |c\rangle  = |\alpha^{-1}\rangle, & \alpha \leq F
	\end{array} \label{eq:2-52}
\end{equation}
where $F$ is the Fermi level representing the last  occupied single-particle orbit 
of the new reference state $|c\rangle $. 

The annihilation is the hermitian conjugate of the creation operator
\[
	b_\alpha = (b_\alpha^\dagger)^\dagger,
\]
resulting in
\begin{equation}
	b_\alpha^\dagger = \Bigg\{ \begin{array}{ll}
		a_\alpha^\dagger & \alpha > F \\
		\\
		a_\alpha & \alpha \leq F
	\end{array} \qquad 
	b_\alpha = \Bigg\{ \begin{array}{ll}
		a_\alpha & \alpha > F \\
		\\
		 a_\alpha^\dagger & \alpha \leq F
	\end{array} \label{eq:2-54}
\end{equation}

With the new creation and annihilation operator  we can now construct 
many-body quasiparticle states, with one-particle-one-hole states, two-particle-two-hole
states etc in the same fashion as we previously constructed many-particle states. 
We can write a general particle-hole state as
\begin{equation}
	|\beta_1\beta_2\dots \beta_{n_p} \gamma_1^{-1} \gamma_2^{-1} \dots \gamma_{n_h}^{-1}\rangle \equiv
		\underbrace{b_{\beta_1}^\dagger b_{\beta_2}^\dagger \dots b_{\beta_{n_p}}^\dagger}_{>F}
		\underbrace{b_{\gamma_1}^\dagger b_{\gamma_2}^\dagger \dots b_{\gamma_{n_h}}^\dagger}_{\leq F} |c\rangle \label{eq:2-56}
\end{equation}
We can now rewrite our one-body and two-body operators in terms of the new creation and annihilation operators.
The number operator becomes
\begin{equation}
	\hat{N} = \sum_\alpha a_\alpha^\dagger a_\alpha= 
\sum_{\alpha > F} b_\alpha^\dagger b_\alpha + n_c - \sum_{\alpha \leq F} b_\alpha^\dagger b_\alpha \label{eq:2-57b}
\end{equation}
where $n_c$ is the number of particle in the new vacuum state $|c\rangle $.  
The action of $\hat{N}$ on a many-body state results in 
\begin{equation}
	N |\beta_1\beta_2\dots \beta_{n_p} \gamma_1^{-1} \gamma_2^{-1} \dots \gamma_{n_h}^{-1}\rangle = (n_p + n_c - n_h) |\beta_1\beta_2\dots \beta_{n_p} \gamma_1^{-1} \gamma_2^{-1} \dots \gamma_{n_h}^{-1}\rangle \label{2-59}
\end{equation}
Here  $n=n_p +n_c - n_h$ is the total number of particles in the quasi-particle state of 
Eq.~(\ref{eq:2-56}). Note that  $\hat{N}$ counts the total number of particles  present 
\begin{equation}
	N_{qp} = \sum_\alpha b_\alpha^\dagger b_\alpha, \label{eq:2-60}
\end{equation}
gives us the number of quasi-particles as can be seen by computing
\begin{equation}
	N_{qp}= |\beta_1\beta_2\dots \beta_{n_p} \gamma_1^{-1} \gamma_2^{-1} \dots \gamma_{n_h}^{-1}\rangle
		= (n_p + n_h)|\beta_1\beta_2\dots \beta_{n_p} \gamma_1^{-1} \gamma_2^{-1} \dots \gamma_{n_h}^{-1}\rangle \label{eq:2-61}
\end{equation}
where $n_{qp} = n_p + n_h$ is the total number of quasi-particles.

We express the one-body operator $\hat{H}_0$ in terms of the quasi-particle creation and annihilation operators, resulting in
\begin{align}
	\hat{H}_0 &=& \sum_{\alpha\beta > F} \langle \alpha|\hat{h}_0|\beta\rangle  b_\alpha^\dagger b_\beta +
		\sum_{\alpha > F, \beta \leq F } \left[\langle \alpha|\hat{h}_0|\beta\rangle b_\alpha^\dagger b_\beta^\dagger + \langle \beta|\hat{h}_0|\alpha\rangle b_\beta  b_\alpha \right] \nonumber \\
	&+& \sum_{\alpha \leq F} \langle \alpha|\hat{h}_0|\alpha\rangle - \sum_{\alpha\beta \leq F} \langle \beta|\hat{h}_0|\alpha\rangle b_\alpha^\dagger b_\beta \label{eq:2-63b}
\end{align}
The first term  gives contribution only for particle states, while the last one
contributes only for holestates. The second term can create or destroy a set of
quasi-particles and 
the third term is the contribution  from the vacuum state $|c\rangle$.

Before we continue with the expressions for the two-body operator, we introduce a nomenclature we will use for the rest of this
text. It is inspired by the notation used in quantum chemistry.
We reserve the labels $i,j,k,\dots$ for hole states and $a,b,c,\dots$ for states above $F$, viz.~particle states.
This means also that we will skip the constraint $\leq F$ or $> F$ in the summation symbols. 
Our operator $\hat{H}_0$  reads now 
\begin{align}
	\hat{H}_0 &=& \sum_{ab} \langle a|\hat{h}|b\rangle b_a^\dagger b_b +
		\sum_{ai} \left[
		\langle a|\hat{h}|i\rangle b_a^\dagger b_i^\dagger + 
		\langle i|\hat{h}|a\rangle b_i  b_a \right] \nonumber \\
	&+& \sum_{i} \langle i|\hat{h}|i\rangle - 
		\sum_{ij} \langle j|\hat{h}|i\rangle
		b_i^\dagger b_j \label{eq:2-63c}
\end{align} 

The two-particle operator in the particle-hole formalism  is more complicated since we have
to translate four indices $\alpha\beta\gamma\delta$ to the possible combinations of particle and hole
states.  When performing the commutator algebra we can regroup the operator in five different terms
\begin{equation}
	\hat{H}_I = \hat{H}_I^{(a)} + \hat{H}_I^{(b)} + \hat{H}_I^{(c)} + \hat{H}_I^{(d)} + \hat{H}_I^{(e)} \label{eq:2-65}
\end{equation}
Using anti-symmetrized  matrix elements, 
bthe term  $\hat{H}_I^{(a)}$ is  
\begin{equation}
	\hat{H}_I^{(a)} = \frac{1}{4}
	\sum_{abcd} \langle ab|\hat{V}|cd\rangle 
		b_a^\dagger b_b^\dagger b_d b_c \label{eq:2-66}
\end{equation}

The next term $\hat{H}_I^{(b)}$  reads
\begin{equation}
	 \hat{H}_I^{(b)} = \frac{1}{4} \sum_{abci}\left(\langle ab|\hat{V}|ci\rangle b_a^\dagger b_b^\dagger b_i^\dagger b_c +\langle ai|\hat{V}|cb\rangle b_a^\dagger b_i b_b b_c\right) \label{eq:2-67b}
\end{equation}
This term conserves the number of quasiparticles but creates or removes a 
three-particle-one-hole  state. 
For $\hat{H}_I^{(c)}$  we have
\begin{align}
	\hat{H}_I^{(c)}& =& \frac{1}{4}
		\sum_{abij}\left(\langle ab|\hat{V}|ij\rangle b_a^\dagger b_b^\dagger b_j^\dagger b_i^\dagger +
		\langle ij|\hat{V}|ab\rangle b_a  b_b b_j b_i \right)+  \nonumber \\
	&&	\frac{1}{2}\sum_{abij}\langle ai|\hat{V}|bj\rangle b_a^\dagger b_j^\dagger b_b b_i + 
		\frac{1}{2}\sum_{abi}\langle ai|\hat{V}|bi\rangle b_a^\dagger b_b. \label{eq:2-68c}
\end{align}

The first line stands for the creation of a two-particle-two-hole state, while the second line represents
the creation to two one-particle-one-hole pairs
while the last term represents a contribution to the particle single-particle energy
from the hole states, that is an interaction between the particle states and the hole states
within the new vacuum  state.
The fourth term reads
\begin{align}
	 \hat{H}_I^{(d)}& = &\frac{1}{4} 
	 	\sum_{aijk}\left(\langle ai|\hat{V}|jk\rangle b_a^\dagger b_k^\dagger b_j^\dagger b_i+
\langle ji|\hat{V}|ak\rangle b_k^\dagger b_j b_i b_a\right)+\nonumber \\
&&\frac{1}{4}\sum_{aij}\left(\langle ai|\hat{V}|ji\rangle b_a^\dagger b_j^\dagger+
\langle ji|\hat{V}|ai\rangle - \langle ji|\hat{V}|ia\rangle b_j b_a \right). \label{eq:2-69d} 
\end{align}
The terms in the first line  stand for the creation of a particle-hole state 
interacting with hole states, we will label this as a two-hole-one-particle contribution. 
The remaining terms are a particle-hole state interacting with the holes in the vacuum state. 
Finally we have 
\begin{equation}
	\hat{H}_I^{(e)} = \frac{1}{4}
		 \sum_{ijkl}
		 \langle kl|\hat{V}|ij\rangle b_i^\dagger b_j^\dagger b_l b_k+
	        \frac{1}{2}\sum_{ijk}\langle ij|\hat{V}|kj\rangle b_k^\dagger b_i
	        +\frac{1}{2}\sum_{ij}\langle ij|\hat{V}|ij\rangle \label{eq:2-70d}
\end{equation}
The first terms represents the 
interaction between two holes while the second stands for the interaction between a hole and the remaining holes in the vacuum state.
It represents a contribution to single-hole energy  to first order.
The last term collects all contributions to the energy of the ground state of a closed-shell system arising
from hole-hole correlations.

\subsection*{Summarizing and defining a normal-ordered Hamiltonian}

\[
  \Phi_{AS}(\alpha_1, \dots, \alpha_A; x_1, \dots x_A)=
            \frac{1}{\sqrt{A}} \sum_{\hat{P}} (-1)^P \hat{P} \prod_{i=1}^A \psi_{\alpha_i}(x_i),
\]
which is equivalent with $|\alpha_1 \dots \alpha_A\rangle= a_{\alpha_1}^{\dagger} \dots a_{\alpha_A}^{\dagger} |0\rangle$. We have also
    \[
        a_p^\dagger|0\rangle = |p\rangle, \quad a_p |q\rangle = \delta_{pq}|0\rangle
    \]
\[
  \delta_{pq} = \left\{a_p, a_q^\dagger \right\},
\]
and 
\[
0 = \left\{a_p^\dagger, a_q \right\} = \left\{a_p, a_q \right\} = \left\{a_p^\dagger, a_q^\dagger \right\}
\]
\[
|\Phi_0\rangle = |\alpha_1 \dots \alpha_A\rangle, \quad \alpha_1, \dots, \alpha_A \leq \alpha_F
\]

\[
\left\{a_p^\dagger, a_q \right\}= \delta_{pq}, p, q \leq \alpha_F 
\]
\[
\left\{a_p, a_q^\dagger \right\} = \delta_{pq}, p, q > \alpha_F
\]
with         $i,j,\ldots \leq \alpha_F, \quad a,b,\ldots > \alpha_F, \quad p,q, \ldots - \textrm{any}$
\[
        a_i|\Phi_0\rangle = |\Phi_i\rangle, \hspace{0.5cm} a_a^\dagger|\Phi_0\rangle = |\Phi^a\rangle
\]
and         
\[
a_i^\dagger|\Phi_0\rangle = 0 \hspace{0.5cm}  a_a|\Phi_0\rangle = 0
\]

The one-body operator is defined as
\[
 \hat{F} = \sum_{pq} \langle p|\hat{f}|q\rangle a_p^\dagger a_q
\]
while the two-body opreator is defined as
\[
\hat{V} = \frac{1}{4} \sum_{pqrs} \langle pq|\hat{v}|rs\rangle_{AS} a_p^\dagger a_q^\dagger a_s a_r
\]
where we have defined the antisymmetric matrix elements
\[
\langle pq|\hat{v}|rs\rangle_{AS} = \langle pq|\hat{v}|rs\rangle - \langle pq|\hat{v}|sr\rangle.
\]

We can also define a three-body operator
\[
\hat{V}_3 = \frac{1}{36} \sum_{pqrstu} \langle pqr|\hat{v}_3|stu\rangle_{AS} 
                a_p^\dagger a_q^\dagger a_r^\dagger a_u a_t a_s
\]
with the antisymmetrized matrix element
\begin{align}
            \langle pqr|\hat{v}_3|stu\rangle_{AS} = \langle pqr|\hat{v}_3|stu\rangle + \langle pqr|\hat{v}_3|tus\rangle + \langle pqr|\hat{v}_3|ust\rangle- \langle pqr|\hat{v}_3|sut\rangle - \langle pqr|\hat{v}_3|tsu\rangle - \langle pqr|\hat{v}_3|uts\rangle.
\end{align}

\subsection*{Operators in second quantization}

In the build-up of a shell-model or FCI code that is meant to tackle large dimensionalities
is the action of the Hamiltonian $\hat{H}$ on a
Slater determinant represented in second quantization as
\[
 |\alpha_1\dots \alpha_n\rangle = a_{\alpha_1}^{\dagger} a_{\alpha_2}^{\dagger} \dots a_{\alpha_n}^{\dagger} |0\rangle.
\]
The time consuming part stems from the action of the Hamiltonian
on the above determinant,
\[
\left(\sum_{\alpha\beta} \langle \alpha|t+u|\beta\rangle a_\alpha^{\dagger} a_\beta + \frac{1}{4} \sum_{\alpha\beta\gamma\delta}
                \langle \alpha \beta|\hat{v}|\gamma \delta\rangle a_\alpha^{\dagger} a_\beta^{\dagger} a_\delta a_\gamma\right)a_{\alpha_1}^{\dagger} a_{\alpha_2}^{\dagger} \dots a_{\alpha_n}^{\dagger} |0\rangle.
\]
A practically useful way to implement this action is to encode a Slater determinant as a bit pattern.

Assume that we have at our disposal $n$ different single-particle orbits
$\alpha_0,\alpha_2,\dots,\alpha_{n-1}$ and that we can distribute  among these orbits $N\le n$ particles.

A Slater  determinant can then be coded as an integer of $n$ bits. As an example, if we have $n=16$ single-particle states
$\alpha_0,\alpha_1,\dots,\alpha_{15}$ and $N=4$ fermions occupying the states $\alpha_3$, $\alpha_6$, $\alpha_{10}$ and $\alpha_{13}$
we could write this Slater determinant as  
\[
\Phi_{\Lambda} = a_{\alpha_3}^{\dagger} a_{\alpha_6}^{\dagger} a_{\alpha_{10}}^{\dagger} a_{\alpha_{13}}^{\dagger} |0\rangle.
\]
The unoccupied single-particle states have bit value $0$ while the occupied ones are represented by bit state $1$. 
In the binary notation we would write this   16 bits long integer as
\[
\begin{array}{cccccccccccccccc}
{\alpha_0}&{\alpha_1}&{\alpha_2}&{\alpha_3}&{\alpha_4}&{\alpha_5}&{\alpha_6}&{\alpha_7} & {\alpha_8} &{\alpha_9} & {\alpha_{10}} &{\alpha_{11}} &{\alpha_{12}} &{\alpha_{13}} &{\alpha_{14}} & {\alpha_{15}} \\
{0} & {0} &{0} &{1} &{0} &{0} &{1} &{0} &{0} &{0} &{1} &{0} &{0} &{1} &{0} & {0} \\
\end{array}
\]
which translates into the decimal number
\[
2^3+2^6+2^{10}+2^{13}=9288.
\]
We can thus encode a Slater determinant as a bit pattern.

With $N$ particles that can be distributed over $n$ single-particle states, the total number of Slater determinats (and defining thereby the dimensionality of the system) is
\[
\mathrm{dim}(\mathcal{H}) = \left(\begin{array}{c} n \\N\end{array}\right).
\]
The total number of bit patterns is $2^n$. 

We assume again that we have at our disposal $n$ different single-particle orbits
$\alpha_0,\alpha_2,\dots,\alpha_{n-1}$ and that we can distribute  among these orbits $N\le n$ particles.
The ordering among these states is important as it defines the order of the creation operators.
We will write the determinant 
\[
\Phi_{\Lambda} = a_{\alpha_3}^{\dagger} a_{\alpha_6}^{\dagger} a_{\alpha_{10}}^{\dagger} a_{\alpha_{13}}^{\dagger} |0\rangle,
\]
in a more compact way as 
\[
\Phi_{3,6,10,13} = |0001001000100100\rangle.
\]
The action of a creation operator is thus
\[
a^{\dagger}_{\alpha_4}\Phi_{3,6,10,13} = a^{\dagger}_{\alpha_4}|0001001000100100\rangle=a^{\dagger}_{\alpha_4}a_{\alpha_3}^{\dagger} a_{\alpha_6}^{\dagger} a_{\alpha_{10}}^{\dagger} a_{\alpha_{13}}^{\dagger} |0\rangle,
\]
which becomes
\[
-a_{\alpha_3}^{\dagger} a^{\dagger}_{\alpha_4} a_{\alpha_6}^{\dagger} a_{\alpha_{10}}^{\dagger} a_{\alpha_{13}}^{\dagger} |0\rangle=-|0001101000100100\rangle.
\]

Similarly
\[
a^{\dagger}_{\alpha_6}\Phi_{3,6,10,13} = a^{\dagger}_{\alpha_6}|0001001000100100\rangle=a^{\dagger}_{\alpha_6}a_{\alpha_3}^{\dagger} a_{\alpha_6}^{\dagger} a_{\alpha_{10}}^{\dagger} a_{\alpha_{13}}^{\dagger} |0\rangle,
\]
which becomes
\[
-a^{\dagger}_{\alpha_4} (a_{\alpha_6}^{\dagger})^ 2 a_{\alpha_{10}}^{\dagger} a_{\alpha_{13}}^{\dagger} |0\rangle=0!
\]
This gives a simple recipe:  
\begin{itemize}
\item If one of the bits $b_j$ is $1$ and we act with a creation operator on this bit, we return a null vector

\item If $b_j=0$, we set it to $1$ and return a sign factor $(-1)^l$, where $l$ is the number of bits set before bit $j$.
\end{itemize}

\noindent
Consider the action of $a^{\dagger}_{\alpha_2}$ on various slater determinants:
\[
\begin{array}{ccc}
a^{\dagger}_{\alpha_2}\Phi_{00111}& = a^{\dagger}_{\alpha_2}|00111\rangle&=0\times |00111\rangle\\
a^{\dagger}_{\alpha_2}\Phi_{01011}& = a^{\dagger}_{\alpha_2}|01011\rangle&=(-1)\times |01111\rangle\\
a^{\dagger}_{\alpha_2}\Phi_{01101}& = a^{\dagger}_{\alpha_2}|01101\rangle&=0\times |01101\rangle\\
a^{\dagger}_{\alpha_2}\Phi_{01110}& = a^{\dagger}_{\alpha_2}|01110\rangle&=0\times |01110\rangle\\
a^{\dagger}_{\alpha_2}\Phi_{10011}& = a^{\dagger}_{\alpha_2}|10011\rangle&=(-1)\times |10111\rangle\\
a^{\dagger}_{\alpha_2}\Phi_{10101}& = a^{\dagger}_{\alpha_2}|10101\rangle&=0\times |10101\rangle\\
a^{\dagger}_{\alpha_2}\Phi_{10110}& = a^{\dagger}_{\alpha_2}|10110\rangle&=0\times |10110\rangle\\
a^{\dagger}_{\alpha_2}\Phi_{11001}& = a^{\dagger}_{\alpha_2}|11001\rangle&=(+1)\times |11101\rangle\\
a^{\dagger}_{\alpha_2}\Phi_{11010}& = a^{\dagger}_{\alpha_2}|11010\rangle&=(+1)\times |11110\rangle\\
\end{array}
\]
What is the simplest way to obtain the phase when we act with one annihilation(creation) operator
on the given Slater determinant representation?

We have an SD representation
\[
\Phi_{\Lambda} = a_{\alpha_0}^{\dagger} a_{\alpha_3}^{\dagger} a_{\alpha_6}^{\dagger} a_{\alpha_{10}}^{\dagger} a_{\alpha_{13}}^{\dagger} |0\rangle,
\]
in a more compact way as
\[
\Phi_{0,3,6,10,13} = |1001001000100100\rangle.
\]
The action of
\[
a^{\dagger}_{\alpha_4}a_{\alpha_0}\Phi_{0,3,6,10,13} = a^{\dagger}_{\alpha_4}|0001001000100100\rangle=a^{\dagger}_{\alpha_4}a_{\alpha_3}^{\dagger} a_{\alpha_6}^{\dagger} a_{\alpha_{10}}^{\dagger} a_{\alpha_{13}}^{\dagger} |0\rangle,
\]
which becomes
\[
-a_{\alpha_3}^{\dagger} a^{\dagger}_{\alpha_4} a_{\alpha_6}^{\dagger} a_{\alpha_{10}}^{\dagger} a_{\alpha_{13}}^{\dagger} |0\rangle=-|0001101000100100\rangle.
\]

The action
\[
a_{\alpha_0}\Phi_{0,3,6,10,13} = |0001001000100100\rangle,
\]
can be obtained by subtracting the logical sum (AND operation) of $\Phi_{0,3,6,10,13}$ and 
a word which represents only $\alpha_0$, that is
\[
|1000000000000000\rangle,
\] 
from $\Phi_{0,3,6,10,13}= |1001001000100100\rangle$.

This operation gives $|0001001000100100\rangle$. 

Similarly, we can form $a^{\dagger}_{\alpha_4}a_{\alpha_0}\Phi_{0,3,6,10,13}$, say, by adding 
$|0000100000000000\rangle$ to $a_{\alpha_0}\Phi_{0,3,6,10,13}$, first checking that their logical sum
is zero in order to make sure that orbital $\alpha_4$ is not already occupied. 

It is trickier however to get the phase $(-1)^l$. 
One possibility is as follows
\begin{itemize}
\item Let $S_1$ be a word that represents the $1-$bit to be removed and all others set to zero.
\end{itemize}

\noindent
In the previous example $S_1=|1000000000000000\rangle$
\begin{itemize}
\item Define $S_2$ as the similar word that represents the bit to be added, that is in our case
\end{itemize}

\noindent
$S_2=|0000100000000000\rangle$.
\begin{itemize}
\item Compute then $S=S_1-S_2$, which here becomes
\end{itemize}

\noindent
\[
S=|0111000000000000\rangle
\]
\begin{itemize}
\item Perform then the logical AND operation of $S$ with the word containing 
\end{itemize}

\noindent
\[
\Phi_{0,3,6,10,13} = |1001001000100100\rangle,
\]
which results in $|0001000000000000\rangle$. Counting the number of $1-$bits gives the phase.  Here you need however an algorithm for bitcounting. Several efficient ones available. 


 \clearemptydoublepage
         
% ------------------- main content ----------------------

\chapter{Many-body Hamiltonians, basic linear algebra and Second Quantization}

\subsection*{Definitions and notations}

Before we proceed we need some definitions.
We will assume that the interacting part of the Hamiltonian
can be approximated by a two-body interaction.
This means that our Hamiltonian is written as the sum of some onebody part and a twobody part
\begin{equation}
    \hat{H} = \hat{H}_0 + \hat{H}_I 
    = \sum_{i=1}^A \hat{h}_0(x_i) + \sum_{i < j}^A \hat{v}(r_{ij}),
\label{Hnuclei}
\end{equation}
with 
\begin{equation}
  H_0=\sum_{i=1}^A \hat{h}_0(x_i).
\label{hinuclei}
\end{equation}
The onebody part $u_{\mathrm{ext}}(x_i)$ is normally approximated by a harmonic oscillator potential or the Coulomb interaction an electron feels from the nucleus. However, other potentials are fully possible, such as 
one derived from the self-consistent solution of the Hartree-Fock equations to be discussed here.

Our Hamiltonian is invariant under the permutation (interchange) of two particles.
Since we deal with fermions however, the total wave function is antisymmetric.
Let $\hat{P}$ be an operator which interchanges two particles.
Due to the symmetries we have ascribed to our Hamiltonian, this operator commutes with the total Hamiltonian,
\[
[\hat{H},\hat{P}] = 0,
 \]
meaning that $\Psi_{\lambda}(x_1, x_2, \dots , x_A)$ is an eigenfunction of 
$\hat{P}$ as well, that is
\[
\hat{P}_{ij}\Psi_{\lambda}(x_1, x_2, \dots,x_i,\dots,x_j,\dots,x_A)=
\beta\Psi_{\lambda}(x_1, x_2, \dots,x_i,\dots,x_j,\dots,x_A),
\]
where $\beta$ is the eigenvalue of $\hat{P}$. We have introduced the suffix $ij$ in order to indicate that we permute particles $i$ and $j$.
The Pauli principle tells us that the total wave function for a system of fermions
has to be antisymmetric, resulting in the eigenvalue $\beta = -1$.   

In our case we assume that  we can approximate the exact eigenfunction with a Slater determinant
\begin{equation}
   \Phi(x_1, x_2,\dots ,x_A,\alpha,\beta,\dots, \sigma)=\frac{1}{\sqrt{A!}}
\left| \begin{array}{ccccc} \psi_{\alpha}(x_1)& \psi_{\alpha}(x_2)& \dots & \dots & \psi_{\alpha}(x_A)\\
                            \psi_{\beta}(x_1)&\psi_{\beta}(x_2)& \dots & \dots & \psi_{\beta}(x_A)\\  
                            \dots & \dots & \dots & \dots & \dots \\
                            \dots & \dots & \dots & \dots & \dots \\
                     \psi_{\sigma}(x_1)&\psi_{\sigma}(x_2)& \dots & \dots & \psi_{\sigma}(x_A)\end{array} \right|, \label{eq:HartreeFockDet}
\end{equation}
where  $x_i$  stand for the coordinates and spin values of a particle $i$ and $\alpha,\beta,\dots, \gamma$ 
are quantum numbers needed to describe remaining quantum numbers.  

\paragraph{Brief reminder on some linear algebra properties.}
Before we proceed with a more compact representation of a Slater determinant, we would like to repeat some linear algebra properties which will be useful for our derivations of the energy as function of a Slater determinant, Hartree-Fock theory and later the nuclear shell model.

The inverse of a matrix is defined by

\[
\mathbf{A}^{-1} \cdot \mathbf{A} = I
\]
A unitary matrix $\mathbf{A}$ is one whose inverse is its adjoint
\[
\mathbf{A}^{-1}=\mathbf{A}^{\dagger}
\]
A real unitary matrix is called orthogonal and its inverse is equal to its transpose.
A hermitian matrix is its own self-adjoint, that  is
\[
\mathbf{A}=\mathbf{A}^{\dagger}. 
\]


\begin{quote}
\begin{tabular}{ccc}
\hline
\multicolumn{1}{c}{ Relations } & \multicolumn{1}{c}{ Name } & \multicolumn{1}{c}{ matrix elements } \\
\hline
$A = A^{T}$                            & symmetric       & $a_{ij} = a_{ji}$                                                       \\
$A = \left (A^{T} \right )^{-1}$       & real orthogonal & $\sum_k a_{ik} a_{jk} = \sum_k a_{ki} a_{kj} = \delta_{ij}$             \\
$A = A^{ * }$                          & real matrix     & $a_{ij} = a_{ij}^{ * }$                                                 \\
$A = A^{\dagger}$                      & hermitian       & $a_{ij} = a_{ji}^{ * }$                                                 \\
$A = \left (A^{\dagger} \right )^{-1}$ & unitary         & $\sum_k a_{ik} a_{jk}^{ * } = \sum_k a_{ki}^{ * } a_{kj} = \delta_{ij}$ \\
\hline
\end{tabular}
\end{quote}

\noindent
Since we will deal with Fermions (identical and indistinguishable particles) we will 
form an ansatz for a given state in terms of so-called Slater determinants determined
by a chosen basis of single-particle functions. 

For a given $n\times n$ matrix $\mathbf{A}$ we can write its determinant
\[
   det(\mathbf{A})=|\mathbf{A}|=
\left| \begin{array}{ccccc} a_{11}& a_{12}& \dots & \dots & a_{1n}\\
                            a_{21}&a_{22}& \dots & \dots & a_{2n}\\  
                            \dots & \dots & \dots & \dots & \dots \\
                            \dots & \dots & \dots & \dots & \dots \\
                            a_{n1}& a_{n2}& \dots & \dots & a_{nn}\end{array} \right|,
\]
in a more compact form as 
\[
|\mathbf{A}|= \sum_{i=1}^{n!}(-1)^{p_i}\hat{P}_i a_{11}a_{22}\dots a_{nn},
\]
where $\hat{P}_i$ is a permutation operator which permutes the column indices $1,2,3,\dots,n$
and the sum runs over all $n!$ permutations.  The quantity $p_i$ represents the number of transpositions of column indices that are needed in order to bring a given permutation back to its initial ordering, in our case given by $a_{11}a_{22}\dots a_{nn}$ here.

A simple $2\times 2$ determinant illustrates this. We have
\[
   det(\mathbf{A})=
\left| \begin{array}{cc} a_{11}& a_{12}\\
                            a_{21}&a_{22}\end{array} \right|= (-1)^0a_{11}a_{22}+(-1)^1a_{12}a_{21},
\]
where in the last term we have interchanged the column indices $1$ and $2$. The natural ordering we have chosen is $a_{11}a_{22}$. 

\paragraph{Back to the derivation of the energy.}
The single-particle function $\psi_{\alpha}(x_i)$  are eigenfunctions of the onebody
Hamiltonian $h_i$, that is
\[
\hat{h}_0(x_i)=\hat{t}(x_i) + \hat{u}_{\mathrm{ext}}(x_i),
\]
with eigenvalues 
\[
\hat{h}_0(x_i) \psi_{\alpha}(x_i)=\left(\hat{t}(x_i) + \hat{u}_{\mathrm{ext}}(x_i)\right)\psi_{\alpha}(x_i)=\varepsilon_{\alpha}\psi_{\alpha}(x_i).
\]
The energies $\varepsilon_{\alpha}$ are the so-called non-interacting single-particle energies, or unperturbed energies. 
The total energy is in this case the sum over all  single-particle energies, if no two-body or more complicated
many-body interactions are present.

Let us denote the ground state energy by $E_0$. According to the
variational principle we have
\[
  E_0 \le E[\Phi] = \int \Phi^*\hat{H}\Phi d\mathbf{\tau}
\]
where $\Phi$ is a trial function which we assume to be normalized
\[
  \int \Phi^*\Phi d\mathbf{\tau} = 1,
\]
where we have used the shorthand $d\mathbf{\tau}=dx_1dr_2\dots dr_A$.

In the Hartree-Fock method the trial function is the Slater
determinant of Eq.~(\ref{eq:HartreeFockDet}) which can be rewritten as 
\[
  \Phi(x_1,x_2,\dots,x_A,\alpha,\beta,\dots,\nu) = \frac{1}{\sqrt{A!}}\sum_{P} (-)^P\hat{P}\psi_{\alpha}(x_1)
    \psi_{\beta}(x_2)\dots\psi_{\nu}(x_A)=\sqrt{A!}\hat{A}\Phi_H,
\]
where we have introduced the antisymmetrization operator $\hat{A}$ defined by the 
summation over all possible permutations of two particles.

It is defined as
\begin{equation}
  \hat{A} = \frac{1}{A!}\sum_{p} (-)^p\hat{P},
\label{antiSymmetryOperator}
\end{equation}
with $p$ standing for the number of permutations. We have introduced for later use the so-called
Hartree-function, defined by the simple product of all possible single-particle functions
\[
  \Phi_H(x_1,x_2,\dots,x_A,\alpha,\beta,\dots,\nu) =
  \psi_{\alpha}(x_1)
    \psi_{\beta}(x_2)\dots\psi_{\nu}(x_A).
\]

Both $\hat{H}_0$ and $\hat{H}_I$ are invariant under all possible permutations of any two particles
and hence commute with $\hat{A}$
\begin{equation}
  [H_0,\hat{A}] = [H_I,\hat{A}] = 0. \label{commutionAntiSym}
\end{equation}
Furthermore, $\hat{A}$ satisfies
\begin{equation}
  \hat{A}^2 = \hat{A},  \label{AntiSymSquared}
\end{equation}
since every permutation of the Slater
determinant reproduces it. 

The expectation value of $\hat{H}_0$ 
\[
  \int \Phi^*\hat{H}_0\Phi d\mathbf{\tau} 
  = A! \int \Phi_H^*\hat{A}\hat{H}_0\hat{A}\Phi_H d\mathbf{\tau}
\]
is readily reduced to
\[
  \int \Phi^*\hat{H}_0\Phi d\mathbf{\tau} 
  = A! \int \Phi_H^*\hat{H}_0\hat{A}\Phi_H d\mathbf{\tau},
\]
where we have used Eqs.~(\ref{commutionAntiSym}) and
(\ref{AntiSymSquared}). The next step is to replace the antisymmetrization
operator by its definition and to
replace $\hat{H}_0$ with the sum of one-body operators
\[
  \int \Phi^*\hat{H}_0\Phi  d\mathbf{\tau}
  = \sum_{i=1}^A \sum_{p} (-)^p\int 
  \Phi_H^*\hat{h}_0\hat{P}\Phi_H d\mathbf{\tau}.
\]

The integral vanishes if two or more particles are permuted in only one
of the Hartree-functions $\Phi_H$ because the individual single-particle wave functions are
orthogonal. We obtain then
\[
  \int \Phi^*\hat{H}_0\Phi  d\mathbf{\tau}= \sum_{i=1}^A \int \Phi_H^*\hat{h}_0\Phi_H  d\mathbf{\tau}.
\]
Orthogonality of the single-particle functions allows us to further simplify the integral, and we
arrive at the following expression for the expectation values of the
sum of one-body Hamiltonians 
\begin{equation}
  \int \Phi^*\hat{H}_0\Phi  d\mathbf{\tau}
  = \sum_{\mu=1}^A \int \psi_{\mu}^*(x)\hat{h}_0\psi_{\mu}(x)dx
  d\mathbf{r}.
  \label{H1Expectation}
\end{equation}

We introduce the following shorthand for the above integral
\[
\langle \mu | \hat{h}_0 | \mu \rangle = \int \psi_{\mu}^*(x)\hat{h}_0\psi_{\mu}(x)dx,
\]
and rewrite Eq.~(\ref{H1Expectation}) as
\begin{equation}
  \int \Phi^*\hat{H}_0\Phi  d\tau
  = \sum_{\mu=1}^A \langle \mu | \hat{h}_0 | \mu \rangle.
  \label{H1Expectation1}
\end{equation}

The expectation value of the two-body part of the Hamiltonian is obtained in a
similar manner. We have
\[
  \int \Phi^*\hat{H}_I\Phi d\mathbf{\tau} 
  = A! \int \Phi_H^*\hat{A}\hat{H}_I\hat{A}\Phi_H d\mathbf{\tau},
\]
which reduces to
\[
 \int \Phi^*\hat{H}_I\Phi d\mathbf{\tau} 
  = \sum_{i\le j=1}^A \sum_{p} (-)^p\int 
  \Phi_H^*\hat{v}(r_{ij})\hat{P}\Phi_H d\mathbf{\tau},
\]
by following the same arguments as for the one-body
Hamiltonian. 

Because of the dependence on the inter-particle distance $r_{ij}$,  permutations of
any two particles no longer vanish, and we get
\[
  \int \Phi^*\hat{H}_I\Phi d\mathbf{\tau} 
  = \sum_{i < j=1}^A \int  
  \Phi_H^*\hat{v}(r_{ij})(1-P_{ij})\Phi_H d\mathbf{\tau}.
\]
where $P_{ij}$ is the permutation operator that interchanges
particle $i$ and particle $j$. Again we use the assumption that the single-particle wave functions
are orthogonal. 

We obtain
\begin{align}
  \int \Phi^*\hat{H}_I\Phi d\mathbf{\tau} 
  = \frac{1}{2}\sum_{\mu=1}^A\sum_{\nu=1}^A
    &\left[ \int \psi_{\mu}^*(x_i)\psi_{\nu}^*(x_j)\hat{v}(r_{ij})\psi_{\mu}(x_i)\psi_{\nu}(x_j)
    dx_idx_j \right.\\
  &\left.
  - \int \psi_{\mu}^*(x_i)\psi_{\nu}^*(x_j)
  \hat{v}(r_{ij})\psi_{\nu}(x_i)\psi_{\mu}(x_j)
  dx_idx_j
  \right]. \label{H2Expectation}
\end{align}
The first term is the so-called direct term. It is frequently also called the  Hartree term, 
while the second is due to the Pauli principle and is called
the exchange term or just the Fock term.
The factor  $1/2$ is introduced because we now run over
all pairs twice. 

The last equation allows us to  introduce some further definitions.  
The single-particle wave functions $\psi_{\mu}(x)$, defined by the quantum numbers $\mu$ and $x$
are defined as the overlap 
\[
   \psi_{\alpha}(x)  = \langle x | \alpha \rangle .
\]

We introduce the following shorthands for the above two integrals
\[
\langle \mu\nu|\hat{v}|\mu\nu\rangle =  \int \psi_{\mu}^*(x_i)\psi_{\nu}^*(x_j)\hat{v}(r_{ij})\psi_{\mu}(x_i)\psi_{\nu}(x_j)
    dx_idx_j,
\]
and
\[
\langle \mu\nu|\hat{v}|\nu\mu\rangle = \int \psi_{\mu}^*(x_i)\psi_{\nu}^*(x_j)
  \hat{v}(r_{ij})\psi_{\nu}(x_i)\psi_{\mu}(x_j)
  dx_idx_j.  
\]

\subsection*{Preparing for later studies: varying the coefficients of a wave function expansion and orthogonal transformations}

It is common to  expand the single-particle functions in a known basis  and vary the coefficients, 
that is, the new single-particle wave function is written as a linear expansion
in terms of a fixed chosen orthogonal basis (for example the well-known harmonic oscillator functions or the hydrogen-like functions etc).
We define our new single-particle basis (this is a normal approach for Hartree-Fock theory) by performing a unitary transformation 
on our previous basis (labelled with greek indices) as
\begin{equation}
\psi_p^{new}  = \sum_{\lambda} C_{p\lambda}\phi_{\lambda}. \label{eq:newbasis}
\end{equation}
In this case we vary the coefficients $C_{p\lambda}$. If the basis has infinitely many solutions, we need
to truncate the above sum.  We assume that the basis $\phi_{\lambda}$ is orthogonal.

It is normal to choose a single-particle basis defined as the eigenfunctions
of parts of the full Hamiltonian. The typical situation consists of the solutions of the one-body part of the Hamiltonian, that is we have
\[
\hat{h}_0\phi_{\lambda}=\epsilon_{\lambda}\phi_{\lambda}.
\]
The single-particle wave functions $\phi_{\lambda}(\mathbf{r})$, defined by the quantum numbers $\lambda$ and $\mathbf{r}$
are defined as the overlap 
\[
   \phi_{\lambda}(\mathbf{r})  = \langle \mathbf{r} | \lambda \rangle .
\]

In deriving the Hartree-Fock equations, we  will expand the single-particle functions in a known basis  and vary the coefficients, 
that is, the new single-particle wave function is written as a linear expansion
in terms of a fixed chosen orthogonal basis (for example the well-known harmonic oscillator functions or the hydrogen-like functions etc).

We stated that a unitary transformation keeps the orthogonality. To see this consider first a basis of vectors $\mathbf{v}_i$,
\[
\mathbf{v}_i = \begin{bmatrix} v_{i1} \\ \dots \\ \dots \\v_{in} \end{bmatrix}
\]
We assume that the basis is orthogonal, that is 
\[
\mathbf{v}_j^T\mathbf{v}_i = \delta_{ij}.
\]
An orthogonal or unitary transformation
\[
\mathbf{w}_i=\mathbf{U}\mathbf{v}_i,
\]
preserves the dot product and orthogonality since
\[
\mathbf{w}_j^T\mathbf{w}_i=(\mathbf{U}\mathbf{v}_j)^T\mathbf{U}\mathbf{v}_i=\mathbf{v}_j^T\mathbf{U}^T\mathbf{U}\mathbf{v}_i= \mathbf{v}_j^T\mathbf{v}_i = \delta_{ij}.
\]

This means that if the coefficients $C_{p\lambda}$ belong to a unitary or orthogonal trasformation (using the Dirac bra-ket notation)
\[
\vert p\rangle  = \sum_{\lambda} C_{p\lambda}\vert\lambda\rangle,
\]
orthogonality is preserved, that is $\langle \alpha \vert \beta\rangle = \delta_{\alpha\beta}$
and $\langle p \vert q\rangle = \delta_{pq}$. 

This propertry is extremely useful when we build up a basis of many-body Stater determinant based states. 

\textbf{Note also that although a basis $\vert \alpha\rangle$ contains an infinity of states, for practical calculations we have always to make some truncations.} 

Before we develop for example the Hartree-Fock equations, there is another very useful property of determinants that we will use both in connection with Hartree-Fock calculations and later shell-model calculations.  

Consider the following determinant
\[
\left| \begin{array}{cc} \alpha_1b_{11}+\alpha_2sb_{12}& a_{12}\\
                         \alpha_1b_{21}+\alpha_2b_{22}&a_{22}\end{array} \right|=\alpha_1\left|\begin{array}{cc} b_{11}& a_{12}\\
                         b_{21}&a_{22}\end{array} \right|+\alpha_2\left| \begin{array}{cc} b_{12}& a_{12}\\b_{22}&a_{22}\end{array} \right|
\]

We can generalize this to  an $n\times n$ matrix and have 
\[
\left| \begin{array}{cccccc} a_{11}& a_{12} & \dots & \sum_{k=1}^n c_k b_{1k} &\dots & a_{1n}\\
a_{21}& a_{22} & \dots & \sum_{k=1}^n c_k b_{2k} &\dots & a_{2n}\\
\dots & \dots & \dots & \dots & \dots & \dots \\
\dots & \dots & \dots & \dots & \dots & \dots \\
a_{n1}& a_{n2} & \dots & \sum_{k=1}^n c_k b_{nk} &\dots & a_{nn}\end{array} \right|=
\sum_{k=1}^n c_k\left| \begin{array}{cccccc} a_{11}& a_{12} & \dots &  b_{1k} &\dots & a_{1n}\\
a_{21}& a_{22} & \dots &  b_{2k} &\dots & a_{2n}\\
\dots & \dots & \dots & \dots & \dots & \dots\\
\dots & \dots & \dots & \dots & \dots & \dots\\
a_{n1}& a_{n2} & \dots &  b_{nk} &\dots & a_{nn}\end{array} \right| .
\]
This is a property we will use in our Hartree-Fock discussions. 

We can generalize the previous results, now 
with all elements $a_{ij}$  being given as functions of 
linear combinations  of various coefficients $c$ and elements $b_{ij}$,
\[
\left| \begin{array}{cccccc} \sum_{k=1}^n b_{1k}c_{k1}& \sum_{k=1}^n b_{1k}c_{k2} & \dots & \sum_{k=1}^n b_{1k}c_{kj}  &\dots & \sum_{k=1}^n b_{1k}c_{kn}\\
\sum_{k=1}^n b_{2k}c_{k1}& \sum_{k=1}^n b_{2k}c_{k2} & \dots & \sum_{k=1}^n b_{2k}c_{kj} &\dots & \sum_{k=1}^n b_{2k}c_{kn}\\
\dots & \dots & \dots & \dots & \dots & \dots \\
\dots & \dots & \dots & \dots & \dots &\dots \\
\sum_{k=1}^n b_{nk}c_{k1}& \sum_{k=1}^n b_{nk}c_{k2} & \dots & \sum_{k=1}^n b_{nk}c_{kj} &\dots & \sum_{k=1}^n b_{nk}c_{kn}\end{array} \right|=det(\mathbf{C})det(\mathbf{B}),
\]
where $det(\mathbf{C})$ and $det(\mathbf{B})$ are the determinants of $n\times n$ matrices
with elements $c_{ij}$ and $b_{ij}$ respectively.  
This is a property we will use in our Hartree-Fock discussions. Convince yourself about the correctness of the above expression by setting $n=2$. 

With our definition of the new basis in terms of an orthogonal basis we have
\[
\psi_p(x)  = \sum_{\lambda} C_{p\lambda}\phi_{\lambda}(x).
\]
If the coefficients $C_{p\lambda}$ belong to an orthogonal or unitary matrix, the new basis
is also orthogonal. 
Our Slater determinant in the new basis $\psi_p(x)$ is written as
\[
\frac{1}{\sqrt{A!}}
\left| \begin{array}{ccccc} \psi_{p}(x_1)& \psi_{p}(x_2)& \dots & \dots & \psi_{p}(x_A)\\
                            \psi_{q}(x_1)&\psi_{q}(x_2)& \dots & \dots & \psi_{q}(x_A)\\  
                            \dots & \dots & \dots & \dots & \dots \\
                            \dots & \dots & \dots & \dots & \dots \\
                     \psi_{t}(x_1)&\psi_{t}(x_2)& \dots & \dots & \psi_{t}(x_A)\end{array} \right|=\frac{1}{\sqrt{A!}}
\left| \begin{array}{ccccc} \sum_{\lambda} C_{p\lambda}\phi_{\lambda}(x_1)& \sum_{\lambda} C_{p\lambda}\phi_{\lambda}(x_2)& \dots & \dots & \sum_{\lambda} C_{p\lambda}\phi_{\lambda}(x_A)\\
                            \sum_{\lambda} C_{q\lambda}\phi_{\lambda}(x_1)&\sum_{\lambda} C_{q\lambda}\phi_{\lambda}(x_2)& \dots & \dots & \sum_{\lambda} C_{q\lambda}\phi_{\lambda}(x_A)\\  
                            \dots & \dots & \dots & \dots & \dots \\
                            \dots & \dots & \dots & \dots & \dots \\
                     \sum_{\lambda} C_{t\lambda}\phi_{\lambda}(x_1)&\sum_{\lambda} C_{t\lambda}\phi_{\lambda}(x_2)& \dots & \dots & \sum_{\lambda} C_{t\lambda}\phi_{\lambda}(x_A)\end{array} \right|,
\]
which is nothing but $det(\mathbf{C})det(\Phi)$, with $det(\Phi)$ being the determinant given by the basis functions $\phi_{\lambda}(x)$. 

In our discussions hereafter we will use our definitions of single-particle states above and below the Fermi ($F$) level given by the labels
$ijkl\dots \le F$ for so-called single-hole states and $abcd\dots > F$ for so-called particle states.
For general single-particle states we employ the labels $pqrs\dots$. 

The energy functional is
\[
  E[\Phi] 
  = \sum_{\mu=1}^A \langle \mu | h | \mu \rangle +
  \frac{1}{2}\sum_{{\mu}=1}^A\sum_{{\nu}=1}^A \langle \mu\nu|\hat{v}|\mu\nu\rangle_{AS},
\]
we found the expression for the energy functional in terms of the basis function $\phi_{\lambda}(\mathbf{r})$. We then  varied the above energy functional with respect to the basis functions $|\mu \rangle$. 
Now we are interested in defining a new basis defined in terms of
a chosen basis as defined in Eq.~(\ref{eq:newbasis}). We can then rewrite the energy functional as
\begin{equation}
  E[\Phi^{New}] 
  = \sum_{i=1}^A \langle i | h | i \rangle +
  \frac{1}{2}\sum_{ij=1}^A\langle ij|\hat{v}|ij\rangle_{AS}, \label{FunctionalEPhi2}
\end{equation}
where $\Phi^{New}$ is the new Slater determinant defined by the new basis of Eq.~(\ref{eq:newbasis}). 

Using Eq.~(\ref{eq:newbasis}) we can rewrite Eq.~(\ref{FunctionalEPhi2}) as 
\begin{equation}
  E[\Psi] 
  = \sum_{i=1}^A \sum_{\alpha\beta} C^*_{i\alpha}C_{i\beta}\langle \alpha | h | \beta \rangle +
  \frac{1}{2}\sum_{ij=1}^A\sum_{{\alpha\beta\gamma\delta}} C^*_{i\alpha}C^*_{j\beta}C_{i\gamma}C_{j\delta}\langle \alpha\beta|\hat{v}|\gamma\delta\rangle_{AS}. \label{FunctionalEPhi3}
\end{equation}

\subsection*{Second quantization}

We introduce the time-independent  operators
$a_\alpha^{\dagger}$ and $a_\alpha$   which create and annihilate, respectively, a particle 
in the single-particle state 
$\varphi_\alpha$. 
We define the fermion creation operator
$a_\alpha^{\dagger}$ 
\begin{equation}
	a_\alpha^{\dagger}|0\rangle \equiv  |\alpha\rangle  \label{eq:2-1a},
\end{equation}
and
\begin{equation}
	a_\alpha^{\dagger}|\alpha_1\dots \alpha_n\rangle_{\mathrm{AS}} \equiv  |\alpha\alpha_1\dots \alpha_n\rangle_{\mathrm{AS}} \label{eq:2-1b}
\end{equation}

In Eq.~(\ref{eq:2-1a}) 
the operator  $a_\alpha^{\dagger}$  acts on the vacuum state 
$|0\rangle$, which does not contain any particles. Alternatively, we could define  a closed-shell nucleus or atom as our new vacuum, but then
we need to introduce the particle-hole  formalism, see the discussion to come. 

In Eq.~(\ref{eq:2-1b}) $a_\alpha^{\dagger}$ acts on an antisymmetric $n$-particle state and 
creates an antisymmetric $(n+1)$-particle state, where the one-body state 
$\varphi_\alpha$ is occupied, under the condition that
$\alpha \ne \alpha_1, \alpha_2, \dots, \alpha_n$. 
It follows that we can express an antisymmetric state as the product of the creation
operators acting on the vacuum state.  
\begin{equation}
	|\alpha_1\dots \alpha_n\rangle_{\mathrm{AS}} = a_{\alpha_1}^{\dagger} a_{\alpha_2}^{\dagger} \dots a_{\alpha_n}^{\dagger} |0\rangle \label{eq:2-2}
\end{equation}

It is easy to derive the commutation and anticommutation rules  for the fermionic creation operators 
$a_\alpha^{\dagger}$. Using the antisymmetry of the states 
(\ref{eq:2-2})
\begin{equation}
	|\alpha_1\dots \alpha_i\dots \alpha_k\dots \alpha_n\rangle_{\mathrm{AS}} = 
		- |\alpha_1\dots \alpha_k\dots \alpha_i\dots \alpha_n\rangle_{\mathrm{AS}} \label{eq:2-3a}
\end{equation}
we obtain
\begin{equation}
	 a_{\alpha_i}^{\dagger}  a_{\alpha_k}^{\dagger} = - a_{\alpha_k}^{\dagger} a_{\alpha_i}^{\dagger} \label{eq:2-3b}
\end{equation}

Using the Pauli principle
\begin{equation}
	|\alpha_1\dots \alpha_i\dots \alpha_i\dots \alpha_n\rangle_{\mathrm{AS}} = 0 \label{eq:2-4a}
\end{equation}
it follows that
\begin{equation}
	a_{\alpha_i}^{\dagger}  a_{\alpha_i}^{\dagger} = 0. \label{eq:2-4b}
\end{equation}
If we combine Eqs.~(\ref{eq:2-3b}) and (\ref{eq:2-4b}), we obtain the well-known anti-commutation rule
\begin{equation}
	a_{\alpha}^{\dagger}  a_{\beta}^{\dagger} + a_{\beta}^{\dagger}  a_{\alpha}^{\dagger} \equiv 
		\{a_{\alpha}^{\dagger},a_{\beta}^{\dagger}\} = 0 \label{eq:2-5}
\end{equation}

The hermitian conjugate  of $a_\alpha^{\dagger}$ is
\begin{equation}
	a_{\alpha} = ( a_{\alpha}^{\dagger} )^{\dagger} \label{eq:2-6}
\end{equation}
If we take the hermitian conjugate of Eq.~(\ref{eq:2-5}), we arrive at 
\begin{equation}
	\{a_{\alpha},a_{\beta}\} = 0 \label{eq:2-7}
\end{equation}

What is the physical interpretation of the operator $a_\alpha$ and what is the effect of 
$a_\alpha$ on a given state $|\alpha_1\alpha_2\dots\alpha_n\rangle_{\mathrm{AS}}$? 
Consider the following matrix element
\begin{equation}
	\langle\alpha_1\alpha_2 \dots \alpha_n|a_\alpha|\alpha_1'\alpha_2' \dots \alpha_m'\rangle \label{eq:2-8}
\end{equation}
where both sides are antisymmetric. We  distinguish between two cases. The first (1) is when
$\alpha \in \{\alpha_i\}$. Using the Pauli principle of Eq.~(\ref{eq:2-4a}) it follows
\begin{equation}
		\langle\alpha_1\alpha_2 \dots \alpha_n|a_\alpha = 0 \label{eq:2-9a}
\end{equation}
The second (2) case is when $\alpha \notin \{\alpha_i\}$. It follows that an hermitian conjugation
\begin{equation}
		\langle \alpha_1\alpha_2 \dots \alpha_n|a_\alpha = \langle\alpha\alpha_1\alpha_2 \dots \alpha_n|  \label{eq:2-9b}
\end{equation}

Eq.~(\ref{eq:2-9b}) holds for case (1) since the lefthand side is zero due to the Pauli principle. We write
Eq.~(\ref{eq:2-8}) as
\begin{equation}
	\langle\alpha_1\alpha_2 \dots \alpha_n|a_\alpha|\alpha_1'\alpha_2' \dots \alpha_m'\rangle = 
	\langle \alpha_1\alpha_2 \dots \alpha_n|\alpha\alpha_1'\alpha_2' \dots \alpha_m'\rangle \label{eq:2-10}
\end{equation}
Here we must have $m = n+1$ if Eq.~(\ref{eq:2-10}) has to be trivially different from zero.

For the last case, the minus and plus signs apply when the sequence 
$\alpha ,\alpha_1, \alpha_2, \dots, \alpha_n$ and 
$\alpha_1', \alpha_2', \dots, \alpha_{n+1}'$ are related to each other via even and odd permutations.
If we assume that  $\alpha \notin \{\alpha_i\}$ we obtain 
\begin{equation}
	\langle\alpha_1\alpha_2 \dots \alpha_n|a_\alpha|\alpha_1'\alpha_2' \dots \alpha_{n+1}'\rangle = 0 \label{eq:2-12}
\end{equation}
when $\alpha \in \{\alpha_i'\}$. If $\alpha \notin \{\alpha_i'\}$, we obtain
\begin{equation}
	a_\alpha\underbrace{|\alpha_1'\alpha_2' \dots \alpha_{n+1}'}\rangle_{\neq \alpha} = 0 \label{eq:2-13a}
\end{equation}
and in particular
\begin{equation}
	a_\alpha |0\rangle = 0 \label{eq:2-13b}
\end{equation}

If $\{\alpha\alpha_i\} = \{\alpha_i'\}$, performing the right permutations, the sequence
$\alpha ,\alpha_1, \alpha_2, \dots, \alpha_n$ is identical with the sequence
$\alpha_1', \alpha_2', \dots, \alpha_{n+1}'$. This results in
\begin{equation}
	\langle\alpha_1\alpha_2 \dots \alpha_n|a_\alpha|\alpha\alpha_1\alpha_2 \dots \alpha_{n}\rangle = 1 \label{eq:2-14}
\end{equation}
and thus
\begin{equation}
	a_\alpha |\alpha\alpha_1\alpha_2 \dots \alpha_{n}\rangle = |\alpha_1\alpha_2 \dots \alpha_{n}\rangle \label{eq:2-15}
\end{equation}

The action of the operator 
$a_\alpha$ from the left on a state vector  is to to remove  one particle in the state
$\alpha$. 
If the state vector does not contain the single-particle state $\alpha$, the outcome of the operation is zero.
The operator  $a_\alpha$ is normally called for a destruction or annihilation operator.

The next step is to establish the  commutator algebra of $a_\alpha^{\dagger}$ and
$a_\beta$. 

The action of the anti-commutator 
$\{a_\alpha^{\dagger}$,$a_\alpha\}$ on a given $n$-particle state is
\begin{align}
	a_\alpha^{\dagger} a_\alpha \underbrace{|\alpha_1\alpha_2 \dots \alpha_{n}\rangle}_{\neq \alpha} &= 0 \nonumber \\
	a_\alpha a_\alpha^{\dagger} \underbrace{|\alpha_1\alpha_2 \dots \alpha_{n}\rangle}_{\neq \alpha} &=
	a_\alpha \underbrace{|\alpha \alpha_1\alpha_2 \dots \alpha_{n}\rangle}_{\neq \alpha} = 
	\underbrace{|\alpha_1\alpha_2 \dots \alpha_{n}\rangle}_{\neq \alpha} \label{eq:2-16a}
\end{align}
if the single-particle state $\alpha$ is not contained in the state.

 If it is present
we arrive at
\begin{align}
	a_\alpha^{\dagger} a_\alpha |\alpha_1\alpha_2 \dots \alpha_{k}\alpha \alpha_{k+1} \dots \alpha_{n-1}\rangle &=
	a_\alpha^{\dagger} a_\alpha (-1)^k |\alpha \alpha_1\alpha_2 \dots \alpha_{n-1}\rangle \nonumber \\
	= (-1)^k |\alpha \alpha_1\alpha_2 \dots \alpha_{n-1}\rangle &=
	|\alpha_1\alpha_2 \dots \alpha_{k}\alpha \alpha_{k+1} \dots \alpha_{n-1}\rangle \nonumber \\
	a_\alpha a_\alpha^{\dagger}|\alpha_1\alpha_2 \dots \alpha_{k}\alpha \alpha_{k+1} \dots \alpha_{n-1}\rangle &= 0 \label{eq:2-16b}
\end{align}
From Eqs.~(\ref{eq:2-16a}) and  (\ref{eq:2-16b}) we arrive at 
\begin{equation}
	\{a_\alpha^{\dagger} , a_\alpha \} = a_\alpha^{\dagger} a_\alpha + a_\alpha a_\alpha^{\dagger} = 1 \label{eq:2-17}
\end{equation}

The action of $\left\{a_\alpha^{\dagger}, a_\beta\right\}$, with 
$\alpha \ne \beta$ on a given state yields three possibilities. 
The first case is a state vector which contains both $\alpha$ and $\beta$, then either 
$\alpha$ or $\beta$ and finally none of them.

The first case results in
\begin{align}
	a_\alpha^{\dagger} a_\beta |\alpha\beta\alpha_1\alpha_2 \dots \alpha_{n-2}\rangle = 0 \nonumber \\
	a_\beta a_\alpha^{\dagger} |\alpha\beta\alpha_1\alpha_2 \dots \alpha_{n-2}\rangle = 0 \label{eq:2-18a}
\end{align}
while the second case gives
\begin{align}
	 a_\alpha^{\dagger} a_\beta |\beta \underbrace{\alpha_1\alpha_2 \dots \alpha_{n-1}}_{\neq \alpha}\rangle =& 
	 	|\alpha \underbrace{\alpha_1\alpha_2 \dots \alpha_{n-1}}_{\neq  \alpha}\rangle \nonumber \\
	a_\beta a_\alpha^{\dagger} |\beta \underbrace{\alpha_1\alpha_2 \dots \alpha_{n-1}}_{\neq \alpha}\rangle =&
		a_\beta |\alpha\beta\underbrace{\beta \alpha_1\alpha_2 \dots \alpha_{n-1}}_{\neq \alpha}\rangle \nonumber \\
	=& - |\alpha\underbrace{\alpha_1\alpha_2 \dots \alpha_{n-1}}_{\neq \alpha}\rangle \label{eq:2-18b}
\end{align}

Finally if the state vector does not contain $\alpha$ and $\beta$
\begin{align}
	a_\alpha^{\dagger} a_\beta |\underbrace{\alpha_1\alpha_2 \dots \alpha_{n}}_{\neq \alpha,\beta}\rangle &=& 0 \nonumber \\
	a_\beta a_\alpha^{\dagger} |\underbrace{\alpha_1\alpha_2 \dots \alpha_{n}}_{\neq \alpha,\beta}\rangle &=& 
		a_\beta |\alpha \underbrace{\alpha_1\alpha_2 \dots \alpha_{n}}_{\neq \alpha,\beta}\rangle = 0 \label{eq:2-18c}
\end{align}
For all three cases we have
\begin{equation}
	\{a_\alpha^{\dagger},a_\beta \} = a_\alpha^{\dagger} a_\beta + a_\beta a_\alpha^{\dagger} = 0, \quad \alpha \neq \beta \label{eq:2-19}
\end{equation}

We can summarize  our findings in Eqs.~(\ref{eq:2-17}) and (\ref{eq:2-19}) as 
\begin{equation}
	\{a_\alpha^{\dagger},a_\beta \} = \delta_{\alpha\beta} \label{eq:2-20}
\end{equation}
with  $\delta_{\alpha\beta}$ is the Kroenecker $\delta$-symbol.

The properties of the creation and annihilation operators can be summarized as (for fermions)
\[
	a_\alpha^{\dagger}|0\rangle \equiv  |\alpha\rangle,
\]
and
\[
	a_\alpha^{\dagger}|\alpha_1\dots \alpha_n\rangle_{\mathrm{AS}} \equiv  |\alpha\alpha_1\dots \alpha_n\rangle_{\mathrm{AS}}. 
\]
from which follows
\[
        |\alpha_1\dots \alpha_n\rangle_{\mathrm{AS}} = a_{\alpha_1}^{\dagger} a_{\alpha_2}^{\dagger} \dots a_{\alpha_n}^{\dagger} |0\rangle.
\]

The hermitian conjugate has the folowing properties
\[
        a_{\alpha} = ( a_{\alpha}^{\dagger} )^{\dagger}.
\]
Finally we found 
\[
	a_\alpha\underbrace{|\alpha_1'\alpha_2' \dots \alpha_{n+1}'}\rangle_{\neq \alpha} = 0, \quad
		\textrm{in particular } a_\alpha |0\rangle = 0,
\]
and
\[
 a_\alpha |\alpha\alpha_1\alpha_2 \dots \alpha_{n}\rangle = |\alpha_1\alpha_2 \dots \alpha_{n}\rangle,
\]
and the corresponding commutator algebra
\[
	\{a_{\alpha}^{\dagger},a_{\beta}^{\dagger}\} = \{a_{\alpha},a_{\beta}\} = 0 \hspace{0.5cm} 
\{a_\alpha^{\dagger},a_\beta \} = \delta_{\alpha\beta}.
\]

\subsection*{One-body operators in second quantization}

A very useful operator is the so-called number-operator.  Most physics cases  we will
study in this text conserve the total number of particles.  The number operator is therefore
a useful quantity which allows us to test that our many-body formalism  conserves the number of particles.
In for example $(d,p)$ or $(p,d)$ reactions it is important to be able to describe quantum mechanical states
where particles get added or removed.
A creation operator $a_\alpha^{\dagger}$ adds one particle to the single-particle state
$\alpha$ of a give many-body state vector, while an annihilation operator $a_\alpha$ 
removes a particle from a single-particle
state $\alpha$. 

Let us consider an operator proportional with $a_\alpha^{\dagger} a_\beta$ and 
$\alpha=\beta$. It acts on an $n$-particle state 
resulting in
\begin{equation}
	a_\alpha^{\dagger} a_\alpha |\alpha_1\alpha_2 \dots \alpha_{n}\rangle = 
	\begin{cases}
		0  &\alpha \notin \{\alpha_i\} \\
		\\
		|\alpha_1\alpha_2 \dots \alpha_{n}\rangle & \alpha \in \{\alpha_i\}
	\end{cases}
\end{equation}
Summing over all possible one-particle states we arrive at
\begin{equation}
	\left( \sum_\alpha a_\alpha^{\dagger} a_\alpha \right) |\alpha_1\alpha_2 \dots \alpha_{n}\rangle = 
	n |\alpha_1\alpha_2 \dots \alpha_{n}\rangle \label{eq:2-21}
\end{equation}

The operator 
\begin{equation}
	\hat{N} = \sum_\alpha a_\alpha^{\dagger} a_\alpha \label{eq:2-22}
\end{equation}
is called the number operator since it counts the number of particles in a give state vector when it acts 
on the different single-particle states.  It acts on one single-particle state at the time and falls 
therefore under category one-body operators.
Next we look at another important one-body operator, namely $\hat{H}_0$ and study its operator form in the 
occupation number representation.

We want to obtain an expression for a one-body operator which conserves the number of particles.
Here we study the one-body operator for the kinetic energy plus an eventual external one-body potential.
The action of this operator on a particular $n$-body state with its pertinent expectation value has already
been studied in coordinate  space.
In coordinate space the operator reads
\begin{equation}
	\hat{H}_0 = \sum_i \hat{h}_0(x_i) \label{eq:2-23}
\end{equation}
and the anti-symmetric $n$-particle Slater determinant is defined as 
\[
\Phi(x_1, x_2,\dots ,x_n,\alpha_1,\alpha_2,\dots, \alpha_n)= \frac{1}{\sqrt{n!}} \sum_p (-1)^p\hat{P}\psi_{\alpha_1}(x_1)\psi_{\alpha_2}(x_2) \dots \psi_{\alpha_n}(x_n).
\]

Defining
\begin{equation}
	\hat{h}_0(x_i) \psi_{\alpha_i}(x_i) = \sum_{\alpha_k'} \psi_{\alpha_k'}(x_i) \langle\alpha_k'|\hat{h}_0|\alpha_k\rangle \label{eq:2-25}
\end{equation}
we can easily  evaluate the action of $\hat{H}_0$ on each product of one-particle functions in Slater determinant.
From Eq.~(\ref{eq:2-25})  we obtain the following result without  permuting any particle pair 
\begin{align}
	&& \left( \sum_i \hat{h}_0(x_i) \right) \psi_{\alpha_1}(x_1)\psi_{\alpha_2}(x_2) \dots \psi_{\alpha_n}(x_n) \nonumber \\
	& =&\sum_{\alpha_1'} \langle \alpha_1'|\hat{h}_0|\alpha_1\rangle 
		\psi_{\alpha_1'}(x_1)\psi_{\alpha_2}(x_2) \dots \psi_{\alpha_n}(x_n) \nonumber \\
	&+&\sum_{\alpha_2'} \langle \alpha_2'|\hat{h}_0|\alpha_2\rangle
		\psi_{\alpha_1}(x_1)\psi_{\alpha_2'}(x_2) \dots \psi_{\alpha_n}(x_n) \nonumber \\
	&+& \dots \nonumber \\
	&+&\sum_{\alpha_n'} \langle \alpha_n'|\hat{h}_0|\alpha_n\rangle
		\psi_{\alpha_1}(x_1)\psi_{\alpha_2}(x_2) \dots \psi_{\alpha_n'}(x_n) \label{eq:2-26}
\end{align}

If we interchange particles $1$ and $2$  we obtain
\begin{align}
	&& \left( \sum_i \hat{h}_0(x_i) \right) \psi_{\alpha_1}(x_2)\psi_{\alpha_1}(x_2) \dots \psi_{\alpha_n}(x_n) \nonumber \\
	& =&\sum_{\alpha_2'} \langle \alpha_2'|\hat{h}_0|\alpha_2\rangle 
		\psi_{\alpha_1}(x_2)\psi_{\alpha_2'}(x_1) \dots \psi_{\alpha_n}(x_n) \nonumber \\
	&+&\sum_{\alpha_1'} \langle \alpha_1'|\hat{h}_0|\alpha_1\rangle
		\psi_{\alpha_1'}(x_2)\psi_{\alpha_2}(x_1) \dots \psi_{\alpha_n}(x_n) \nonumber \\
	&+& \dots \nonumber \\
	&+&\sum_{\alpha_n'} \langle \alpha_n'|\hat{h}_0|\alpha_n\rangle
		\psi_{\alpha_1}(x_2)\psi_{\alpha_1}(x_2) \dots \psi_{\alpha_n'}(x_n) \label{eq:2-27}
\end{align}

We can continue by computing all possible permutations. 
We rewrite also our Slater determinant in its second quantized form and skip the dependence on the quantum numbers $x_i.$
Summing up all contributions and taking care of all phases
$(-1)^p$ we arrive at 
\begin{align}
	\hat{H}_0|\alpha_1,\alpha_2,\dots, \alpha_n\rangle &=& \sum_{\alpha_1'}\langle \alpha_1'|\hat{h}_0|\alpha_1\rangle
		|\alpha_1'\alpha_2 \dots \alpha_{n}\rangle \nonumber \\
	&+& \sum_{\alpha_2'} \langle \alpha_2'|\hat{h}_0|\alpha_2\rangle
		|\alpha_1\alpha_2' \dots \alpha_{n}\rangle \nonumber \\
	&+& \dots \nonumber \\
	&+& \sum_{\alpha_n'} \langle \alpha_n'|\hat{h}_0|\alpha_n\rangle
		|\alpha_1\alpha_2 \dots \alpha_{n}'\rangle \label{eq:2-28}
\end{align}

In Eq.~(\ref{eq:2-28}) 
we have expressed the action of the one-body operator
of Eq.~(\ref{eq:2-23}) on the  $n$-body state in its second quantized form.
This equation can be further manipulated if we use the properties of the creation and annihilation operator
on each primed quantum number, that is
\begin{equation}
	|\alpha_1\alpha_2 \dots \alpha_k' \dots \alpha_{n}\rangle = 
		a_{\alpha_k'}^{\dagger}  a_{\alpha_k} |\alpha_1\alpha_2 \dots \alpha_k \dots \alpha_{n}\rangle \label{eq:2-29}
\end{equation}
Inserting this in the right-hand side of Eq.~(\ref{eq:2-28}) results in
\begin{align}
	\hat{H}_0|\alpha_1\alpha_2 \dots \alpha_{n}\rangle &=& \sum_{\alpha_1'}\langle \alpha_1'|\hat{h}_0|\alpha_1\rangle
		a_{\alpha_1'}^{\dagger}  a_{\alpha_1} |\alpha_1\alpha_2 \dots \alpha_{n}\rangle \nonumber \\
	&+& \sum_{\alpha_2'} \langle \alpha_2'|\hat{h}_0|\alpha_2\rangle
		a_{\alpha_2'}^{\dagger}  a_{\alpha_2} |\alpha_1\alpha_2 \dots \alpha_{n}\rangle \nonumber \\
	&+& \dots \nonumber \\
	&+& \sum_{\alpha_n'} \langle \alpha_n'|\hat{h}_0|\alpha_n\rangle
		a_{\alpha_n'}^{\dagger}  a_{\alpha_n} |\alpha_1\alpha_2 \dots \alpha_{n}\rangle \nonumber \\
	&=& \sum_{\alpha, \beta} \langle \alpha|\hat{h}_0|\beta\rangle a_\alpha^{\dagger} a_\beta 
		|\alpha_1\alpha_2 \dots \alpha_{n}\rangle \label{eq:2-30a}
\end{align}

In the number occupation representation or second quantization we get the following expression for a one-body 
operator which conserves the number of particles
\begin{equation}
	\hat{H}_0 = \sum_{\alpha\beta} \langle \alpha|\hat{h}_0|\beta\rangle a_\alpha^{\dagger} a_\beta \label{eq:2-30b}
\end{equation}
Obviously, $\hat{H}_0$ can be replaced by any other one-body  operator which preserved the number
of particles. The stucture of the operator is therefore not limited to say the kinetic or single-particle energy only.

The opearator $\hat{H}_0$ takes a particle from the single-particle state $\beta$  to the single-particle state $\alpha$ 
with a probability for the transition given by the expectation value $\langle \alpha|\hat{h}_0|\beta\rangle$.

It is instructive to verify Eq.~(\ref{eq:2-30b}) by computing the expectation value of $\hat{H}_0$ 
between two single-particle states
\begin{equation}
	\langle \alpha_1|\hat{h}_0|\alpha_2\rangle = \sum_{\alpha\beta} \langle \alpha|\hat{h}_0|\beta\rangle
		\langle 0|a_{\alpha_1}a_\alpha^{\dagger} a_\beta a_{\alpha_2}^{\dagger}|0\rangle \label{eq:2-30c}
\end{equation}

Using the commutation relations for the creation and annihilation operators we have 
\begin{equation}
a_{\alpha_1}a_\alpha^{\dagger} a_\beta a_{\alpha_2}^{\dagger} = (\delta_{\alpha \alpha_1} - a_\alpha^{\dagger} a_{\alpha_1} )(\delta_{\beta \alpha_2} - a_{\alpha_2}^{\dagger} a_{\beta} ), \label{eq:2-30d}
\end{equation}
which results in
\begin{equation}
\langle 0|a_{\alpha_1}a_\alpha^{\dagger} a_\beta a_{\alpha_2}^{\dagger}|0\rangle = \delta_{\alpha \alpha_1} \delta_{\beta \alpha_2} \label{eq:2-30e}
\end{equation}
and
\begin{equation}
\langle \alpha_1|\hat{h}_0|\alpha_2\rangle = \sum_{\alpha\beta} \langle \alpha|\hat{h}_0|\beta\rangle\delta_{\alpha \alpha_1} \delta_{\beta \alpha_2} = \langle \alpha_1|\hat{h}_0|\alpha_2\rangle \label{eq:2-30f}
\end{equation}

\subsection*{Two-body operators in second quantization}

Let us now derive the expression for our two-body interaction part, which also conserves the number of particles.
We can proceed in exactly the same way as for the one-body operator. In the coordinate representation our
two-body interaction part takes the following expression
\begin{equation}
	\hat{H}_I = \sum_{i < j} V(x_i,x_j) \label{eq:2-31}
\end{equation}
where the summation runs over distinct pairs. The term $V$ can be an interaction model for the nucleon-nucleon interaction
or the interaction between two electrons. It can also include additional two-body interaction terms. 

The action of this operator on a product of 
two single-particle functions is defined as 
\begin{equation}
	V(x_i,x_j) \psi_{\alpha_k}(x_i) \psi_{\alpha_l}(x_j) = \sum_{\alpha_k'\alpha_l'} 
		\psi_{\alpha_k}'(x_i)\psi_{\alpha_l}'(x_j) 
		\langle \alpha_k'\alpha_l'|\hat{v}|\alpha_k\alpha_l\rangle \label{eq:2-32}
\end{equation}

We can now let $\hat{H}_I$ act on all terms in the linear combination for $|\alpha_1\alpha_2\dots\alpha_n\rangle$. Without any permutations we have
\begin{align}
	&& \left( \sum_{i < j} V(x_i,x_j) \right) \psi_{\alpha_1}(x_1)\psi_{\alpha_2}(x_2)\dots \psi_{\alpha_n}(x_n) \nonumber \\
	&=& \sum_{\alpha_1'\alpha_2'} \langle \alpha_1'\alpha_2'|\hat{v}|\alpha_1\alpha_2\rangle
		\psi_{\alpha_1}'(x_1)\psi_{\alpha_2}'(x_2)\dots \psi_{\alpha_n}(x_n) \nonumber \\
	& +& \dots \nonumber \\
	&+& \sum_{\alpha_1'\alpha_n'} \langle \alpha_1'\alpha_n'|\hat{v}|\alpha_1\alpha_n\rangle
		\psi_{\alpha_1}'(x_1)\psi_{\alpha_2}(x_2)\dots \psi_{\alpha_n}'(x_n) \nonumber \\
	& +& \dots \nonumber \\
	&+& \sum_{\alpha_2'\alpha_n'} \langle \alpha_2'\alpha_n'|\hat{v}|\alpha_2\alpha_n\rangle
		\psi_{\alpha_1}(x_1)\psi_{\alpha_2}'(x_2)\dots \psi_{\alpha_n}'(x_n) \nonumber \\
	 & +& \dots \label{eq:2-33}
\end{align}
where on the rhs we have a term for each distinct pairs. 

For the other terms on the rhs we obtain similar expressions  and summing over all terms we obtain
\begin{align}
	H_I |\alpha_1\alpha_2\dots\alpha_n\rangle &=& \sum_{\alpha_1', \alpha_2'} \langle \alpha_1'\alpha_2'|\hat{v}|\alpha_1\alpha_2\rangle
		|\alpha_1'\alpha_2'\dots\alpha_n\rangle \nonumber \\
	&+& \dots \nonumber \\
	&+& \sum_{\alpha_1', \alpha_n'} \langle \alpha_1'\alpha_n'|\hat{v}|\alpha_1\alpha_n\rangle
		|\alpha_1'\alpha_2\dots\alpha_n'\rangle \nonumber \\
	&+& \dots \nonumber \\
	&+& \sum_{\alpha_2', \alpha_n'} \langle \alpha_2'\alpha_n'|\hat{v}|\alpha_2\alpha_n\rangle
		|\alpha_1\alpha_2'\dots\alpha_n'\rangle \nonumber \\
	 &+& \dots \label{eq:2-34}
\end{align}

We introduce second quantization via the relation
\begin{align}
	&& a_{\alpha_k'}^{\dagger} a_{\alpha_l'}^{\dagger} a_{\alpha_l} a_{\alpha_k} 
		|\alpha_1\alpha_2\dots\alpha_k\dots\alpha_l\dots\alpha_n\rangle \nonumber \\
	&=& (-1)^{k-1} (-1)^{l-2} a_{\alpha_k'}^{\dagger} a_{\alpha_l'}^{\dagger} a_{\alpha_l} a_{\alpha_k}
		|\alpha_k\alpha_l \underbrace{\alpha_1\alpha_2\dots\alpha_n}_{\neq \alpha_k,\alpha_l}\rangle \nonumber \\
	&=& (-1)^{k-1} (-1)^{l-2} 
	|\alpha_k'\alpha_l' \underbrace{\alpha_1\alpha_2\dots\alpha_n}_{\neq \alpha_k',\alpha_l'}\rangle \nonumber \\
	&=& |\alpha_1\alpha_2\dots\alpha_k'\dots\alpha_l'\dots\alpha_n\rangle \label{eq:2-35}
\end{align}

Inserting this in (\ref{eq:2-34}) gives
\begin{align}
	H_I |\alpha_1\alpha_2\dots\alpha_n\rangle
	&=& \sum_{\alpha_1', \alpha_2'} \langle \alpha_1'\alpha_2'|\hat{v}|\alpha_1\alpha_2\rangle
		a_{\alpha_1'}^{\dagger} a_{\alpha_2'}^{\dagger} a_{\alpha_2} a_{\alpha_1}
		|\alpha_1\alpha_2\dots\alpha_n\rangle \nonumber \\
	&+& \dots \nonumber \\
	&=& \sum_{\alpha_1', \alpha_n'} \langle \alpha_1'\alpha_n'|\hat{v}|\alpha_1\alpha_n\rangle
		a_{\alpha_1'}^{\dagger} a_{\alpha_n'}^{\dagger} a_{\alpha_n} a_{\alpha_1}
		|\alpha_1\alpha_2\dots\alpha_n\rangle \nonumber \\
	&+& \dots \nonumber \\
	&=& \sum_{\alpha_2', \alpha_n'} \langle \alpha_2'\alpha_n'|\hat{v}|\alpha_2\alpha_n\rangle
		a_{\alpha_2'}^{\dagger} a_{\alpha_n'}^{\dagger} a_{\alpha_n} a_{\alpha_2}
		|\alpha_1\alpha_2\dots\alpha_n\rangle \nonumber \\
	&+& \dots \nonumber \\
	&=& \sum_{\alpha, \beta, \gamma, \delta} ' \langle \alpha\beta|\hat{v}|\gamma\delta\rangle
		a^{\dagger}_\alpha a^{\dagger}_\beta a_\delta a_\gamma
		|\alpha_1\alpha_2\dots\alpha_n\rangle \label{eq:2-36}
\end{align}

Here we let $\sum'$ indicate that the sums running over $\alpha$ and $\beta$ run over all
single-particle states, while the summations  $\gamma$ and $\delta$ 
run over all pairs of single-particle states. We wish to remove this restriction and since
\begin{equation}
	\langle \alpha\beta|\hat{v}|\gamma\delta\rangle = \langle \beta\alpha|\hat{v}|\delta\gamma\rangle \label{eq:2-37}
\end{equation}
we get
\begin{align}
	\sum_{\alpha\beta} \langle \alpha\beta|\hat{v}|\gamma\delta\rangle a^{\dagger}_\alpha a^{\dagger}_\beta a_\delta a_\gamma &=& 
		\sum_{\alpha\beta} \langle \beta\alpha|\hat{v}|\delta\gamma\rangle 
		a^{\dagger}_\alpha a^{\dagger}_\beta a_\delta a_\gamma \label{eq:2-38a} \\
	&=& \sum_{\alpha\beta}\langle \beta\alpha|\hat{v}|\delta\gamma\rangle
		a^{\dagger}_\beta a^{\dagger}_\alpha a_\gamma a_\delta \label{eq:2-38b}
\end{align}
where we  have used the anti-commutation rules.

Changing the summation indices 
$\alpha$ and $\beta$ in (\ref{eq:2-38b}) we obtain
\begin{equation}
	\sum_{\alpha\beta} \langle \alpha\beta|\hat{v}|\gamma\delta\rangle a^{\dagger}_\alpha a^{\dagger}_\beta a_\delta a_\gamma =
		 \sum_{\alpha\beta} \langle \alpha\beta|\hat{v}|\delta\gamma\rangle 
		  a^{\dagger}_\alpha a^{\dagger}_\beta  a_\gamma a_\delta \label{eq:2-38c}
\end{equation}
From this it follows that the restriction on the summation over $\gamma$ and $\delta$ can be removed if we multiply with a factor $\frac{1}{2}$, resulting in 
\begin{equation}
	\hat{H}_I = \frac{1}{2} \sum_{\alpha\beta\gamma\delta} \langle \alpha\beta|\hat{v}|\gamma\delta\rangle
		a^{\dagger}_\alpha a^{\dagger}_\beta a_\delta a_\gamma \label{eq:2-39}
\end{equation}
where we sum freely over all single-particle states $\alpha$, 
$\beta$, $\gamma$ og $\delta$.

With this expression we can now verify that the second quantization form of $\hat{H}_I$ in Eq.~(\ref{eq:2-39}) 
results in the same matrix between two anti-symmetrized two-particle states as its corresponding coordinate
space representation. We have  
\begin{equation}
	\langle \alpha_1 \alpha_2|\hat{H}_I|\beta_1 \beta_2\rangle =
		\frac{1}{2} \sum_{\alpha\beta\gamma\delta}
			\langle \alpha\beta|\hat{v}|\gamma\delta\rangle \langle 0|a_{\alpha_2} a_{\alpha_1} 
			 a^{\dagger}_\alpha a^{\dagger}_\beta a_\delta a_\gamma 
			 a_{\beta_1}^{\dagger} a_{\beta_2}^{\dagger}|0\rangle. \label{eq:2-40}
\end{equation}

Using the commutation relations we get 
\begin{align}
	&& a_{\alpha_2} a_{\alpha_1}a^{\dagger}_\alpha a^{\dagger}_\beta 
		a_\delta a_\gamma a_{\beta_1}^{\dagger} a_{\beta_2}^{\dagger} \nonumber \\
	&=& a_{\alpha_2} a_{\alpha_1}a^{\dagger}_\alpha a^{\dagger}_\beta 
		( a_\delta \delta_{\gamma \beta_1} a_{\beta_2}^{\dagger} - 
		a_\delta  a_{\beta_1}^{\dagger} a_\gamma a_{\beta_2}^{\dagger} ) \nonumber \\
	&=& a_{\alpha_2} a_{\alpha_1}a^{\dagger}_\alpha a^{\dagger}_\beta 
		(\delta_{\gamma \beta_1} \delta_{\delta \beta_2} - \delta_{\gamma \beta_1} a_{\beta_2}^{\dagger} a_\delta -
		a_\delta a_{\beta_1}^{\dagger}\delta_{\gamma \beta_2} +
		a_\delta a_{\beta_1}^{\dagger} a_{\beta_2}^{\dagger} a_\gamma ) \nonumber \\
	&=& a_{\alpha_2} a_{\alpha_1}a^{\dagger}_\alpha a^{\dagger}_\beta 
		(\delta_{\gamma \beta_1} \delta_{\delta \beta_2} - \delta_{\gamma \beta_1} a_{\beta_2}^{\dagger} a_\delta \nonumber \\
		&& \qquad - \delta_{\delta \beta_1} \delta_{\gamma \beta_2} + \delta_{\gamma \beta_2} a_{\beta_1}^{\dagger} a_\delta
		+ a_\delta a_{\beta_1}^{\dagger} a_{\beta_2}^{\dagger} a_\gamma ) \label{eq:2-41}
\end{align}

The vacuum expectation value of this product of operators becomes
\begin{align}
	&& \langle 0|a_{\alpha_2} a_{\alpha_1} a^{\dagger}_\alpha a^{\dagger}_\beta a_\delta a_\gamma 
		a_{\beta_1}^{\dagger} a_{\beta_2}^{\dagger}|0\rangle \nonumber \\
	&=& (\delta_{\gamma \beta_1} \delta_{\delta \beta_2} -
		\delta_{\delta \beta_1} \delta_{\gamma \beta_2} ) 
		\langle 0|a_{\alpha_2} a_{\alpha_1}a^{\dagger}_\alpha a^{\dagger}_\beta|0\rangle \nonumber \\
	&=& (\delta_{\gamma \beta_1} \delta_{\delta \beta_2} -\delta_{\delta \beta_1} \delta_{\gamma \beta_2} )
	(\delta_{\alpha \alpha_1} \delta_{\beta \alpha_2} -\delta_{\beta \alpha_1} \delta_{\alpha \alpha_2} ) \label{eq:2-42b}
\end{align}

Insertion of 
Eq.~(\ref{eq:2-42b}) in Eq.~(\ref{eq:2-40}) results in
\begin{align}
	\langle \alpha_1\alpha_2|\hat{H}_I|\beta_1\beta_2\rangle &=& \frac{1}{2} \big[ 
		\langle \alpha_1\alpha_2|\hat{v}|\beta_1\beta_2\rangle- \langle \alpha_1\alpha_2|\hat{v}|\beta_2\beta_1\rangle \nonumber \\
		&& - \langle \alpha_2\alpha_1|\hat{v}|\beta_1\beta_2\rangle + \langle \alpha_2\alpha_1|\hat{v}|\beta_2\beta_1\rangle \big] \nonumber \\
	&=& \langle \alpha_1\alpha_2|\hat{v}|\beta_1\beta_2\rangle - \langle \alpha_1\alpha_2|\hat{v}|\beta_2\beta_1\rangle \nonumber \\
	&=& \langle \alpha_1\alpha_2|\hat{v}|\beta_1\beta_2\rangle_{\mathrm{AS}}. \label{eq:2-43b}
\end{align}

The two-body operator can also be expressed in terms of the anti-symmetrized matrix elements we discussed previously as
\begin{align}
	\hat{H}_I &=& \frac{1}{2} \sum_{\alpha\beta\gamma\delta}  \langle \alpha \beta|\hat{v}|\gamma \delta\rangle
		a_\alpha^{\dagger} a_\beta^{\dagger} a_\delta a_\gamma \nonumber \\
	&=& \frac{1}{4} \sum_{\alpha\beta\gamma\delta} \left[ \langle \alpha \beta|\hat{v}|\gamma \delta\rangle -
		\langle \alpha \beta|\hat{v}|\delta\gamma \rangle \right] 
		a_\alpha^{\dagger} a_\beta^{\dagger} a_\delta a_\gamma \nonumber \\
	&=& \frac{1}{4} \sum_{\alpha\beta\gamma\delta} \langle \alpha \beta|\hat{v}|\gamma \delta\rangle_{\mathrm{AS}}
		a_\alpha^{\dagger} a_\beta^{\dagger} a_\delta a_\gamma \label{eq:2-45}
\end{align}

The factors in front of the operator, either  $\frac{1}{4}$ or 
$\frac{1}{2}$ tells whether we use antisymmetrized matrix elements or not. 

We can now express the Hamiltonian operator for a many-fermion system  in the occupation basis representation
as  
\begin{equation}
	H = \sum_{\alpha, \beta} \langle \alpha|\hat{t}+\hat{u}_{\mathrm{ext}}|\beta\rangle a_\alpha^{\dagger} a_\beta + \frac{1}{4} \sum_{\alpha\beta\gamma\delta}
		\langle \alpha \beta|\hat{v}|\gamma \delta\rangle a_\alpha^{\dagger} a_\beta^{\dagger} a_\delta a_\gamma. \label{eq:2-46b}
\end{equation}
This is the form we will use in the rest of these lectures, assuming that we work with anti-symmetrized two-body matrix elements.

\subsection*{Particle-hole formalism}

Second quantization is a useful and elegant formalism  for constructing many-body  states and 
quantum mechanical operators. One can express and translate many physical processes
into simple pictures such as Feynman diagrams. Expecation values of many-body states are also easily calculated.
However, although the equations are seemingly easy to set up, from  a practical point of view, that is
the solution of Schroedinger's equation, there is no particular gain.
The many-body equation is equally hard to solve, irrespective of representation. 
The cliche that 
there is no free lunch brings us down to earth again.  
Note however that a transformation to a particular
basis, for cases where the interaction obeys specific symmetries, can ease the solution of Schroedinger's equation. 

But there is at least one important case where second quantization comes to our rescue.
It is namely easy to introduce another reference state than the pure vacuum $|0\rangle $, where all single-particle states are active.
With many particles present it is often useful to introduce another reference state  than the vacuum state$|0\rangle $. We will label this state $|c\rangle$ ($c$ for core) and as we will see it can reduce 
considerably the complexity and thereby the dimensionality of the many-body problem. It allows us to sum up to infinite order specific many-body correlations.  The particle-hole representation is one of these handy representations. 

In the original particle representation these states are products of the creation operators  $a_{\alpha_i}^\dagger$ acting on the true vacuum $|0\rangle $.
Following Eq.~(\ref{eq:2-2}) we have 
\begin{align}
 |\alpha_1\alpha_2\dots\alpha_{n-1}\alpha_n\rangle &=& a_{\alpha_1}^\dagger a_{\alpha_2}^\dagger \dots
					a_{\alpha_{n-1}}^\dagger a_{\alpha_n}^\dagger |0\rangle  \label{eq:2-47a} \\
	|\alpha_1\alpha_2\dots\alpha_{n-1}\alpha_n\alpha_{n+1}\rangle &=&
		a_{\alpha_1}^\dagger a_{\alpha_2}^\dagger \dots a_{\alpha_{n-1}}^\dagger a_{\alpha_n}^\dagger
		a_{\alpha_{n+1}}^\dagger |0\rangle  \label{eq:2-47b} \\
	|\alpha_1\alpha_2\dots\alpha_{n-1}\rangle &=& a_{\alpha_1}^\dagger a_{\alpha_2}^\dagger \dots
		a_{\alpha_{n-1}}^\dagger |0\rangle  \label{eq:2-47c}
\end{align}

If we use Eq.~(\ref{eq:2-47a}) as our new reference state, we can simplify considerably the representation of 
this state
\begin{equation}
	|c\rangle  \equiv |\alpha_1\alpha_2\dots\alpha_{n-1}\alpha_n\rangle =
		a_{\alpha_1}^\dagger a_{\alpha_2}^\dagger \dots a_{\alpha_{n-1}}^\dagger a_{\alpha_n}^\dagger |0\rangle  \label{eq:2-48a}
\end{equation}
The new reference states for the $n+1$ and $n-1$ states can then be written as
\begin{align}
	|\alpha_1\alpha_2\dots\alpha_{n-1}\alpha_n\alpha_{n+1}\rangle &=& (-1)^n a_{\alpha_{n+1}}^\dagger |c\rangle 
		\equiv (-1)^n |\alpha_{n+1}\rangle_c \label{eq:2-48b} \\
	|\alpha_1\alpha_2\dots\alpha_{n-1}\rangle &=& (-1)^{n-1} a_{\alpha_n} |c\rangle  
		\equiv (-1)^{n-1} |\alpha_{n-1}\rangle_c \label{eq:2-48c} 
\end{align}

The first state has one additional particle with respect to the new vacuum state
$|c\rangle $  and is normally referred to as a one-particle state or one particle added to the 
many-body reference state. 
The second state has one particle less than the reference vacuum state  $|c\rangle $ and is referred to as
a one-hole state. 
When dealing with a new reference state it is often convenient to introduce 
new creation and annihilation operators since we have 
from Eq.~(\ref{eq:2-48c})
\begin{equation}
	a_\alpha |c\rangle  \neq 0 \label{eq:2-49}
\end{equation}
since  $\alpha$ is contained  in $|c\rangle $, while for the true vacuum we have 
$a_\alpha |0\rangle  = 0$ for all $\alpha$.

The new reference state leads to the definition of new creation and annihilation operators
which satisfy the following relations
\begin{align}
	b_\alpha |c\rangle  &=& 0 \label{eq:2-50a} \\
	\{b_\alpha^\dagger , b_\beta^\dagger \} = \{b_\alpha , b_\beta \} &=& 0 \nonumber  \\
	\{b_\alpha^\dagger , b_\beta \} &=& \delta_{\alpha \beta} \label{eq:2-50c}
\end{align}
We assume also that the new reference state is properly normalized
\begin{equation}
	\langle c | c \rangle = 1 \label{eq:2-51}
\end{equation}

The physical interpretation of these new operators is that of so-called quasiparticle states.
This means that a state defined by the addition of one extra particle to a reference state $|c\rangle $ may not necesseraly be interpreted as one particle coupled to a core.
We define now new creation operators that act on a state $\alpha$ creating a new quasiparticle state
\begin{equation}
	b_\alpha^\dagger|c\rangle  = \Bigg\{ \begin{array}{ll}
		a_\alpha^\dagger |c\rangle  = |\alpha\rangle, & \alpha > F \\
		\\
		a_\alpha |c\rangle  = |\alpha^{-1}\rangle, & \alpha \leq F
	\end{array} \label{eq:2-52}
\end{equation}
where $F$ is the Fermi level representing the last  occupied single-particle orbit 
of the new reference state $|c\rangle $. 

The annihilation is the hermitian conjugate of the creation operator
\[
	b_\alpha = (b_\alpha^\dagger)^\dagger,
\]
resulting in
\begin{equation}
	b_\alpha^\dagger = \Bigg\{ \begin{array}{ll}
		a_\alpha^\dagger & \alpha > F \\
		\\
		a_\alpha & \alpha \leq F
	\end{array} \qquad 
	b_\alpha = \Bigg\{ \begin{array}{ll}
		a_\alpha & \alpha > F \\
		\\
		 a_\alpha^\dagger & \alpha \leq F
	\end{array} \label{eq:2-54}
\end{equation}

With the new creation and annihilation operator  we can now construct 
many-body quasiparticle states, with one-particle-one-hole states, two-particle-two-hole
states etc in the same fashion as we previously constructed many-particle states. 
We can write a general particle-hole state as
\begin{equation}
	|\beta_1\beta_2\dots \beta_{n_p} \gamma_1^{-1} \gamma_2^{-1} \dots \gamma_{n_h}^{-1}\rangle \equiv
		\underbrace{b_{\beta_1}^\dagger b_{\beta_2}^\dagger \dots b_{\beta_{n_p}}^\dagger}_{>F}
		\underbrace{b_{\gamma_1}^\dagger b_{\gamma_2}^\dagger \dots b_{\gamma_{n_h}}^\dagger}_{\leq F} |c\rangle \label{eq:2-56}
\end{equation}
We can now rewrite our one-body and two-body operators in terms of the new creation and annihilation operators.
The number operator becomes
\begin{equation}
	\hat{N} = \sum_\alpha a_\alpha^\dagger a_\alpha= 
\sum_{\alpha > F} b_\alpha^\dagger b_\alpha + n_c - \sum_{\alpha \leq F} b_\alpha^\dagger b_\alpha \label{eq:2-57b}
\end{equation}
where $n_c$ is the number of particle in the new vacuum state $|c\rangle $.  
The action of $\hat{N}$ on a many-body state results in 
\begin{equation}
	N |\beta_1\beta_2\dots \beta_{n_p} \gamma_1^{-1} \gamma_2^{-1} \dots \gamma_{n_h}^{-1}\rangle = (n_p + n_c - n_h) |\beta_1\beta_2\dots \beta_{n_p} \gamma_1^{-1} \gamma_2^{-1} \dots \gamma_{n_h}^{-1}\rangle \label{2-59}
\end{equation}
Here  $n=n_p +n_c - n_h$ is the total number of particles in the quasi-particle state of 
Eq.~(\ref{eq:2-56}). Note that  $\hat{N}$ counts the total number of particles  present 
\begin{equation}
	N_{qp} = \sum_\alpha b_\alpha^\dagger b_\alpha, \label{eq:2-60}
\end{equation}
gives us the number of quasi-particles as can be seen by computing
\begin{equation}
	N_{qp}= |\beta_1\beta_2\dots \beta_{n_p} \gamma_1^{-1} \gamma_2^{-1} \dots \gamma_{n_h}^{-1}\rangle
		= (n_p + n_h)|\beta_1\beta_2\dots \beta_{n_p} \gamma_1^{-1} \gamma_2^{-1} \dots \gamma_{n_h}^{-1}\rangle \label{eq:2-61}
\end{equation}
where $n_{qp} = n_p + n_h$ is the total number of quasi-particles.

We express the one-body operator $\hat{H}_0$ in terms of the quasi-particle creation and annihilation operators, resulting in
\begin{align}
	\hat{H}_0 &=& \sum_{\alpha\beta > F} \langle \alpha|\hat{h}_0|\beta\rangle  b_\alpha^\dagger b_\beta +
		\sum_{\alpha > F, \beta \leq F } \left[\langle \alpha|\hat{h}_0|\beta\rangle b_\alpha^\dagger b_\beta^\dagger + \langle \beta|\hat{h}_0|\alpha\rangle b_\beta  b_\alpha \right] \nonumber \\
	&+& \sum_{\alpha \leq F} \langle \alpha|\hat{h}_0|\alpha\rangle - \sum_{\alpha\beta \leq F} \langle \beta|\hat{h}_0|\alpha\rangle b_\alpha^\dagger b_\beta \label{eq:2-63b}
\end{align}
The first term  gives contribution only for particle states, while the last one
contributes only for holestates. The second term can create or destroy a set of
quasi-particles and 
the third term is the contribution  from the vacuum state $|c\rangle$.

Before we continue with the expressions for the two-body operator, we introduce a nomenclature we will use for the rest of this
text. It is inspired by the notation used in quantum chemistry.
We reserve the labels $i,j,k,\dots$ for hole states and $a,b,c,\dots$ for states above $F$, viz.~particle states.
This means also that we will skip the constraint $\leq F$ or $> F$ in the summation symbols. 
Our operator $\hat{H}_0$  reads now 
\begin{align}
	\hat{H}_0 &=& \sum_{ab} \langle a|\hat{h}|b\rangle b_a^\dagger b_b +
		\sum_{ai} \left[
		\langle a|\hat{h}|i\rangle b_a^\dagger b_i^\dagger + 
		\langle i|\hat{h}|a\rangle b_i  b_a \right] \nonumber \\
	&+& \sum_{i} \langle i|\hat{h}|i\rangle - 
		\sum_{ij} \langle j|\hat{h}|i\rangle
		b_i^\dagger b_j \label{eq:2-63c}
\end{align} 

The two-particle operator in the particle-hole formalism  is more complicated since we have
to translate four indices $\alpha\beta\gamma\delta$ to the possible combinations of particle and hole
states.  When performing the commutator algebra we can regroup the operator in five different terms
\begin{equation}
	\hat{H}_I = \hat{H}_I^{(a)} + \hat{H}_I^{(b)} + \hat{H}_I^{(c)} + \hat{H}_I^{(d)} + \hat{H}_I^{(e)} \label{eq:2-65}
\end{equation}
Using anti-symmetrized  matrix elements, 
bthe term  $\hat{H}_I^{(a)}$ is  
\begin{equation}
	\hat{H}_I^{(a)} = \frac{1}{4}
	\sum_{abcd} \langle ab|\hat{V}|cd\rangle 
		b_a^\dagger b_b^\dagger b_d b_c \label{eq:2-66}
\end{equation}

The next term $\hat{H}_I^{(b)}$  reads
\begin{equation}
	 \hat{H}_I^{(b)} = \frac{1}{4} \sum_{abci}\left(\langle ab|\hat{V}|ci\rangle b_a^\dagger b_b^\dagger b_i^\dagger b_c +\langle ai|\hat{V}|cb\rangle b_a^\dagger b_i b_b b_c\right) \label{eq:2-67b}
\end{equation}
This term conserves the number of quasiparticles but creates or removes a 
three-particle-one-hole  state. 
For $\hat{H}_I^{(c)}$  we have
\begin{align}
	\hat{H}_I^{(c)}& =& \frac{1}{4}
		\sum_{abij}\left(\langle ab|\hat{V}|ij\rangle b_a^\dagger b_b^\dagger b_j^\dagger b_i^\dagger +
		\langle ij|\hat{V}|ab\rangle b_a  b_b b_j b_i \right)+  \nonumber \\
	&&	\frac{1}{2}\sum_{abij}\langle ai|\hat{V}|bj\rangle b_a^\dagger b_j^\dagger b_b b_i + 
		\frac{1}{2}\sum_{abi}\langle ai|\hat{V}|bi\rangle b_a^\dagger b_b. \label{eq:2-68c}
\end{align}

The first line stands for the creation of a two-particle-two-hole state, while the second line represents
the creation to two one-particle-one-hole pairs
while the last term represents a contribution to the particle single-particle energy
from the hole states, that is an interaction between the particle states and the hole states
within the new vacuum  state.
The fourth term reads
\begin{align}
	 \hat{H}_I^{(d)}& = &\frac{1}{4} 
	 	\sum_{aijk}\left(\langle ai|\hat{V}|jk\rangle b_a^\dagger b_k^\dagger b_j^\dagger b_i+
\langle ji|\hat{V}|ak\rangle b_k^\dagger b_j b_i b_a\right)+\nonumber \\
&&\frac{1}{4}\sum_{aij}\left(\langle ai|\hat{V}|ji\rangle b_a^\dagger b_j^\dagger+
\langle ji|\hat{V}|ai\rangle - \langle ji|\hat{V}|ia\rangle b_j b_a \right). \label{eq:2-69d} 
\end{align}
The terms in the first line  stand for the creation of a particle-hole state 
interacting with hole states, we will label this as a two-hole-one-particle contribution. 
The remaining terms are a particle-hole state interacting with the holes in the vacuum state. 
Finally we have 
\begin{equation}
	\hat{H}_I^{(e)} = \frac{1}{4}
		 \sum_{ijkl}
		 \langle kl|\hat{V}|ij\rangle b_i^\dagger b_j^\dagger b_l b_k+
	        \frac{1}{2}\sum_{ijk}\langle ij|\hat{V}|kj\rangle b_k^\dagger b_i
	        +\frac{1}{2}\sum_{ij}\langle ij|\hat{V}|ij\rangle \label{eq:2-70d}
\end{equation}
The first terms represents the 
interaction between two holes while the second stands for the interaction between a hole and the remaining holes in the vacuum state.
It represents a contribution to single-hole energy  to first order.
The last term collects all contributions to the energy of the ground state of a closed-shell system arising
from hole-hole correlations.

\subsection*{Summarizing and defining a normal-ordered Hamiltonian}

\[
  \Phi_{AS}(\alpha_1, \dots, \alpha_A; x_1, \dots x_A)=
            \frac{1}{\sqrt{A}} \sum_{\hat{P}} (-1)^P \hat{P} \prod_{i=1}^A \psi_{\alpha_i}(x_i),
\]
which is equivalent with $|\alpha_1 \dots \alpha_A\rangle= a_{\alpha_1}^{\dagger} \dots a_{\alpha_A}^{\dagger} |0\rangle$. We have also
    \[
        a_p^\dagger|0\rangle = |p\rangle, \quad a_p |q\rangle = \delta_{pq}|0\rangle
    \]
\[
  \delta_{pq} = \left\{a_p, a_q^\dagger \right\},
\]
and 
\[
0 = \left\{a_p^\dagger, a_q \right\} = \left\{a_p, a_q \right\} = \left\{a_p^\dagger, a_q^\dagger \right\}
\]
\[
|\Phi_0\rangle = |\alpha_1 \dots \alpha_A\rangle, \quad \alpha_1, \dots, \alpha_A \leq \alpha_F
\]

\[
\left\{a_p^\dagger, a_q \right\}= \delta_{pq}, p, q \leq \alpha_F 
\]
\[
\left\{a_p, a_q^\dagger \right\} = \delta_{pq}, p, q > \alpha_F
\]
with         $i,j,\ldots \leq \alpha_F, \quad a,b,\ldots > \alpha_F, \quad p,q, \ldots - \textrm{any}$
\[
        a_i|\Phi_0\rangle = |\Phi_i\rangle, \hspace{0.5cm} a_a^\dagger|\Phi_0\rangle = |\Phi^a\rangle
\]
and         
\[
a_i^\dagger|\Phi_0\rangle = 0 \hspace{0.5cm}  a_a|\Phi_0\rangle = 0
\]

The one-body operator is defined as
\[
 \hat{F} = \sum_{pq} \langle p|\hat{f}|q\rangle a_p^\dagger a_q
\]
while the two-body opreator is defined as
\[
\hat{V} = \frac{1}{4} \sum_{pqrs} \langle pq|\hat{v}|rs\rangle_{AS} a_p^\dagger a_q^\dagger a_s a_r
\]
where we have defined the antisymmetric matrix elements
\[
\langle pq|\hat{v}|rs\rangle_{AS} = \langle pq|\hat{v}|rs\rangle - \langle pq|\hat{v}|sr\rangle.
\]

We can also define a three-body operator
\[
\hat{V}_3 = \frac{1}{36} \sum_{pqrstu} \langle pqr|\hat{v}_3|stu\rangle_{AS} 
                a_p^\dagger a_q^\dagger a_r^\dagger a_u a_t a_s
\]
with the antisymmetrized matrix element
\begin{align}
            \langle pqr|\hat{v}_3|stu\rangle_{AS} = \langle pqr|\hat{v}_3|stu\rangle + \langle pqr|\hat{v}_3|tus\rangle + \langle pqr|\hat{v}_3|ust\rangle- \langle pqr|\hat{v}_3|sut\rangle - \langle pqr|\hat{v}_3|tsu\rangle - \langle pqr|\hat{v}_3|uts\rangle.
\end{align}

\subsection*{Operators in second quantization}

In the build-up of a shell-model or FCI code that is meant to tackle large dimensionalities
is the action of the Hamiltonian $\hat{H}$ on a
Slater determinant represented in second quantization as
\[
 |\alpha_1\dots \alpha_n\rangle = a_{\alpha_1}^{\dagger} a_{\alpha_2}^{\dagger} \dots a_{\alpha_n}^{\dagger} |0\rangle.
\]
The time consuming part stems from the action of the Hamiltonian
on the above determinant,
\[
\left(\sum_{\alpha\beta} \langle \alpha|t+u|\beta\rangle a_\alpha^{\dagger} a_\beta + \frac{1}{4} \sum_{\alpha\beta\gamma\delta}
                \langle \alpha \beta|\hat{v}|\gamma \delta\rangle a_\alpha^{\dagger} a_\beta^{\dagger} a_\delta a_\gamma\right)a_{\alpha_1}^{\dagger} a_{\alpha_2}^{\dagger} \dots a_{\alpha_n}^{\dagger} |0\rangle.
\]
A practically useful way to implement this action is to encode a Slater determinant as a bit pattern.

Assume that we have at our disposal $n$ different single-particle orbits
$\alpha_0,\alpha_2,\dots,\alpha_{n-1}$ and that we can distribute  among these orbits $N\le n$ particles.

A Slater  determinant can then be coded as an integer of $n$ bits. As an example, if we have $n=16$ single-particle states
$\alpha_0,\alpha_1,\dots,\alpha_{15}$ and $N=4$ fermions occupying the states $\alpha_3$, $\alpha_6$, $\alpha_{10}$ and $\alpha_{13}$
we could write this Slater determinant as  
\[
\Phi_{\Lambda} = a_{\alpha_3}^{\dagger} a_{\alpha_6}^{\dagger} a_{\alpha_{10}}^{\dagger} a_{\alpha_{13}}^{\dagger} |0\rangle.
\]
The unoccupied single-particle states have bit value $0$ while the occupied ones are represented by bit state $1$. 
In the binary notation we would write this   16 bits long integer as
\[
\begin{array}{cccccccccccccccc}
{\alpha_0}&{\alpha_1}&{\alpha_2}&{\alpha_3}&{\alpha_4}&{\alpha_5}&{\alpha_6}&{\alpha_7} & {\alpha_8} &{\alpha_9} & {\alpha_{10}} &{\alpha_{11}} &{\alpha_{12}} &{\alpha_{13}} &{\alpha_{14}} & {\alpha_{15}} \\
{0} & {0} &{0} &{1} &{0} &{0} &{1} &{0} &{0} &{0} &{1} &{0} &{0} &{1} &{0} & {0} \\
\end{array}
\]
which translates into the decimal number
\[
2^3+2^6+2^{10}+2^{13}=9288.
\]
We can thus encode a Slater determinant as a bit pattern.

With $N$ particles that can be distributed over $n$ single-particle states, the total number of Slater determinats (and defining thereby the dimensionality of the system) is
\[
\mathrm{dim}(\mathcal{H}) = \left(\begin{array}{c} n \\N\end{array}\right).
\]
The total number of bit patterns is $2^n$. 

We assume again that we have at our disposal $n$ different single-particle orbits
$\alpha_0,\alpha_2,\dots,\alpha_{n-1}$ and that we can distribute  among these orbits $N\le n$ particles.
The ordering among these states is important as it defines the order of the creation operators.
We will write the determinant 
\[
\Phi_{\Lambda} = a_{\alpha_3}^{\dagger} a_{\alpha_6}^{\dagger} a_{\alpha_{10}}^{\dagger} a_{\alpha_{13}}^{\dagger} |0\rangle,
\]
in a more compact way as 
\[
\Phi_{3,6,10,13} = |0001001000100100\rangle.
\]
The action of a creation operator is thus
\[
a^{\dagger}_{\alpha_4}\Phi_{3,6,10,13} = a^{\dagger}_{\alpha_4}|0001001000100100\rangle=a^{\dagger}_{\alpha_4}a_{\alpha_3}^{\dagger} a_{\alpha_6}^{\dagger} a_{\alpha_{10}}^{\dagger} a_{\alpha_{13}}^{\dagger} |0\rangle,
\]
which becomes
\[
-a_{\alpha_3}^{\dagger} a^{\dagger}_{\alpha_4} a_{\alpha_6}^{\dagger} a_{\alpha_{10}}^{\dagger} a_{\alpha_{13}}^{\dagger} |0\rangle=-|0001101000100100\rangle.
\]

Similarly
\[
a^{\dagger}_{\alpha_6}\Phi_{3,6,10,13} = a^{\dagger}_{\alpha_6}|0001001000100100\rangle=a^{\dagger}_{\alpha_6}a_{\alpha_3}^{\dagger} a_{\alpha_6}^{\dagger} a_{\alpha_{10}}^{\dagger} a_{\alpha_{13}}^{\dagger} |0\rangle,
\]
which becomes
\[
-a^{\dagger}_{\alpha_4} (a_{\alpha_6}^{\dagger})^ 2 a_{\alpha_{10}}^{\dagger} a_{\alpha_{13}}^{\dagger} |0\rangle=0!
\]
This gives a simple recipe:  
\begin{itemize}
\item If one of the bits $b_j$ is $1$ and we act with a creation operator on this bit, we return a null vector

\item If $b_j=0$, we set it to $1$ and return a sign factor $(-1)^l$, where $l$ is the number of bits set before bit $j$.
\end{itemize}

\noindent
Consider the action of $a^{\dagger}_{\alpha_2}$ on various slater determinants:
\[
\begin{array}{ccc}
a^{\dagger}_{\alpha_2}\Phi_{00111}& = a^{\dagger}_{\alpha_2}|00111\rangle&=0\times |00111\rangle\\
a^{\dagger}_{\alpha_2}\Phi_{01011}& = a^{\dagger}_{\alpha_2}|01011\rangle&=(-1)\times |01111\rangle\\
a^{\dagger}_{\alpha_2}\Phi_{01101}& = a^{\dagger}_{\alpha_2}|01101\rangle&=0\times |01101\rangle\\
a^{\dagger}_{\alpha_2}\Phi_{01110}& = a^{\dagger}_{\alpha_2}|01110\rangle&=0\times |01110\rangle\\
a^{\dagger}_{\alpha_2}\Phi_{10011}& = a^{\dagger}_{\alpha_2}|10011\rangle&=(-1)\times |10111\rangle\\
a^{\dagger}_{\alpha_2}\Phi_{10101}& = a^{\dagger}_{\alpha_2}|10101\rangle&=0\times |10101\rangle\\
a^{\dagger}_{\alpha_2}\Phi_{10110}& = a^{\dagger}_{\alpha_2}|10110\rangle&=0\times |10110\rangle\\
a^{\dagger}_{\alpha_2}\Phi_{11001}& = a^{\dagger}_{\alpha_2}|11001\rangle&=(+1)\times |11101\rangle\\
a^{\dagger}_{\alpha_2}\Phi_{11010}& = a^{\dagger}_{\alpha_2}|11010\rangle&=(+1)\times |11110\rangle\\
\end{array}
\]
What is the simplest way to obtain the phase when we act with one annihilation(creation) operator
on the given Slater determinant representation?

We have an SD representation
\[
\Phi_{\Lambda} = a_{\alpha_0}^{\dagger} a_{\alpha_3}^{\dagger} a_{\alpha_6}^{\dagger} a_{\alpha_{10}}^{\dagger} a_{\alpha_{13}}^{\dagger} |0\rangle,
\]
in a more compact way as
\[
\Phi_{0,3,6,10,13} = |1001001000100100\rangle.
\]
The action of
\[
a^{\dagger}_{\alpha_4}a_{\alpha_0}\Phi_{0,3,6,10,13} = a^{\dagger}_{\alpha_4}|0001001000100100\rangle=a^{\dagger}_{\alpha_4}a_{\alpha_3}^{\dagger} a_{\alpha_6}^{\dagger} a_{\alpha_{10}}^{\dagger} a_{\alpha_{13}}^{\dagger} |0\rangle,
\]
which becomes
\[
-a_{\alpha_3}^{\dagger} a^{\dagger}_{\alpha_4} a_{\alpha_6}^{\dagger} a_{\alpha_{10}}^{\dagger} a_{\alpha_{13}}^{\dagger} |0\rangle=-|0001101000100100\rangle.
\]

The action
\[
a_{\alpha_0}\Phi_{0,3,6,10,13} = |0001001000100100\rangle,
\]
can be obtained by subtracting the logical sum (AND operation) of $\Phi_{0,3,6,10,13}$ and 
a word which represents only $\alpha_0$, that is
\[
|1000000000000000\rangle,
\] 
from $\Phi_{0,3,6,10,13}= |1001001000100100\rangle$.

This operation gives $|0001001000100100\rangle$. 

Similarly, we can form $a^{\dagger}_{\alpha_4}a_{\alpha_0}\Phi_{0,3,6,10,13}$, say, by adding 
$|0000100000000000\rangle$ to $a_{\alpha_0}\Phi_{0,3,6,10,13}$, first checking that their logical sum
is zero in order to make sure that orbital $\alpha_4$ is not already occupied. 

It is trickier however to get the phase $(-1)^l$. 
One possibility is as follows
\begin{itemize}
\item Let $S_1$ be a word that represents the $1-$bit to be removed and all others set to zero.
\end{itemize}

\noindent
In the previous example $S_1=|1000000000000000\rangle$
\begin{itemize}
\item Define $S_2$ as the similar word that represents the bit to be added, that is in our case
\end{itemize}

\noindent
$S_2=|0000100000000000\rangle$.
\begin{itemize}
\item Compute then $S=S_1-S_2$, which here becomes
\end{itemize}

\noindent
\[
S=|0111000000000000\rangle
\]
\begin{itemize}
\item Perform then the logical AND operation of $S$ with the word containing 
\end{itemize}

\noindent
\[
\Phi_{0,3,6,10,13} = |1001001000100100\rangle,
\]
which results in $|0001000000000000\rangle$. Counting the number of $1-$bits gives the phase.  Here you need however an algorithm for bitcounting. Several efficient ones available. 


 \clearemptydoublepage
      
\chapter{Full configuration interaction theory}

\subsection*{Slater determinants as basis states, Repetition}

% --- begin paragraph admon ---
\paragraph{}
The simplest possible choice for many-body wavefunctions are \textbf{product} wavefunctions.
That is
\[ 
\Psi(x_1, x_2, x_3, \ldots, x_A) \approx \phi_1(x_1) \phi_2(x_2) \phi_3(x_3) \ldots
\]
because we are really only good  at thinking about one particle at a time. Such 
product wavefunctions, without correlations, are easy to 
work with; for example, if the single-particle states $\phi_i(x)$ are orthonormal, then 
the product wavefunctions are easy to orthonormalize.   

Similarly, computing matrix elements of operators are relatively easy, because the 
integrals factorize.

The price we pay is the lack of correlations, which we must build up by using many, many product 
wavefunctions. (Thus we have a trade-off: compact representation of correlations but 
difficult integrals versus easy integrals but many states required.)
% --- end paragraph admon ---



\subsection*{Slater determinants as basis states, repetition}

% --- begin paragraph admon ---
\paragraph{}
Because we have fermions, we are required to have antisymmetric wavefunctions, e.g.
\[
\Psi(x_1, x_2, x_3, \ldots, x_A) = - \Psi(x_2, x_1, x_3, \ldots, x_A)
\]
etc. This is accomplished formally by using the determinantal formalism
\[
\Psi(x_1, x_2, \ldots, x_A) 
= \frac{1}{\sqrt{A!}} 
\det \left | 
\begin{array}{cccc}
\phi_1(x_1) & \phi_1(x_2) & \ldots & \phi_1(x_A) \\
\phi_2(x_1) & \phi_2(x_2) & \ldots & \phi_2(x_A) \\
 \vdots & & &  \\
\phi_A(x_1) & \phi_A(x_2) & \ldots & \phi_A(x_A) 
\end{array}
\right |
\]
Product wavefunction + antisymmetry = Slater determinant.
% --- end paragraph admon ---



\subsection*{Slater determinants as basis states}

% --- begin paragraph admon ---
\paragraph{}
\[
\Psi(x_1, x_2, \ldots, x_A) 
= \frac{1}{\sqrt{A!}} 
\det \left | 
\begin{array}{cccc}
\phi_1(x_1) & \phi_1(x_2) & \ldots & \phi_1(x_A) \\
\phi_2(x_1) & \phi_2(x_2) & \ldots & \phi_2(x_A) \\
 \vdots & & &  \\
\phi_A(x_1) & \phi_A(x_2) & \ldots & \phi_A(x_A) 
\end{array}
\right |
\]
Properties of the determinant (interchange of any two rows or 
any two columns yields a change in sign; thus no two rows and no 
two columns can be the same) lead to the Pauli principle:

\begin{itemize}
\item No two particles can be at the same place (two columns the same); and

\item No two particles can be in the same state (two rows the same).
\end{itemize}

\noindent
% --- end paragraph admon ---



\subsection*{Slater determinants as basis states}

% --- begin paragraph admon ---
\paragraph{}
As a practical matter, however, Slater determinants beyond $N=4$ quickly become 
unwieldy. Thus we turn to the \textbf{occupation representation} or \textbf{second quantization} to simplify calculations. 

The occupation representation or number representation, using fermion \textbf{creation} and \textbf{annihilation} 
operators, is compact and efficient. It is also abstract and, at first encounter, not easy to 
internalize. It is inspired by other operator formalism, such as the ladder operators for 
the harmonic oscillator or for angular momentum, but unlike those cases, the operators \textbf{do not have coordinate space representations}.

Instead, one can think of fermion creation/annihilation operators as a game of symbols that 
compactly reproduces what one would do, albeit clumsily, with full coordinate-space Slater 
determinants.
% --- end paragraph admon ---



\subsection*{Quick repetition of the occupation representation}

% --- begin paragraph admon ---
\paragraph{}
We start with a set of orthonormal single-particle states $\{ \phi_i(x) \}$. 
(Note: this requirement, and others, can be relaxed, but leads to a 
more involved formalism.) \textbf{Any} orthonormal set will do. 

To each single-particle state $\phi_i(x)$ we associate a creation operator 
$\hat{a}^\dagger_i$ and an annihilation operator $\hat{a}_i$. 

When acting on the vacuum state $| 0 \rangle$, the creation operator $\hat{a}^\dagger_i$ causes 
a particle to occupy the single-particle state $\phi_i(x)$:
\[
\phi_i(x) \rightarrow \hat{a}^\dagger_i |0 \rangle
\]
% --- end paragraph admon ---



\subsection*{Quick repetition  of the occupation representation}

% --- begin paragraph admon ---
\paragraph{}
But with multiple creation operators we can occupy multiple states:
\[
\phi_i(x) \phi_j(x^\prime) \phi_k(x^{\prime \prime}) 
\rightarrow \hat{a}^\dagger_i \hat{a}^\dagger_j \hat{a}^\dagger_k |0 \rangle.
\]

Now we impose antisymmetry, by having the fermion operators satisfy  \textbf{anticommutation relations}:
\[
\hat{a}^\dagger_i \hat{a}^\dagger_j + \hat{a}^\dagger_j \hat{a}^\dagger_i
= [ \hat{a}^\dagger_i ,\hat{a}^\dagger_j ]_+ 
= \{ \hat{a}^\dagger_i ,\hat{a}^\dagger_j \} = 0
\]
so that 
\[
\hat{a}^\dagger_i \hat{a}^\dagger_j = - \hat{a}^\dagger_j \hat{a}^\dagger_i
\]
% --- end paragraph admon ---



\subsection*{Quick repetition  of the occupation representation}

% --- begin paragraph admon ---
\paragraph{}
Because of this property, automatically $\hat{a}^\dagger_i \hat{a}^\dagger_i = 0$, 
enforcing the Pauli exclusion principle.  Thus when writing a Slater determinant 
using creation operators, 
\[
\hat{a}^\dagger_i \hat{a}^\dagger_j \hat{a}^\dagger_k \ldots |0 \rangle
\]
each index $i,j,k, \ldots$ must be unique.

For some relevant exercises with solutions see chapter 8 of \href{{http://www.springer.com/us/book/9783319533353}}{Lecture Notes in Physics, volume 936}.
% --- end paragraph admon ---



\subsection*{Full Configuration Interaction Theory}

% --- begin paragraph admon ---
\paragraph{}
We have defined the ansatz for the ground state as 
\[
|\Phi_0\rangle = \left(\prod_{i\le F}\hat{a}_{i}^{\dagger}\right)|0\rangle,
\]
where the index $i$ defines different single-particle states up to the Fermi level. We have assumed that we have $N$ fermions. 
A given one-particle-one-hole ($1p1h$) state can be written as
\[
|\Phi_i^a\rangle = \hat{a}_{a}^{\dagger}\hat{a}_i|\Phi_0\rangle,
\]
while a $2p2h$ state can be written as
\[
|\Phi_{ij}^{ab}\rangle = \hat{a}_{a}^{\dagger}\hat{a}_{b}^{\dagger}\hat{a}_j\hat{a}_i|\Phi_0\rangle,
\]
and a general $NpNh$ state as 
\[
|\Phi_{ijk\dots}^{abc\dots}\rangle = \hat{a}_{a}^{\dagger}\hat{a}_{b}^{\dagger}\hat{a}_{c}^{\dagger}\dots\hat{a}_k\hat{a}_j\hat{a}_i|\Phi_0\rangle.
\]
% --- end paragraph admon ---



\subsection*{Full Configuration Interaction Theory}

% --- begin paragraph admon ---
\paragraph{}
We can then expand our exact state function for the ground state 
as
\[
|\Psi_0\rangle=C_0|\Phi_0\rangle+\sum_{ai}C_i^a|\Phi_i^a\rangle+\sum_{abij}C_{ij}^{ab}|\Phi_{ij}^{ab}\rangle+\dots
=(C_0+\hat{C})|\Phi_0\rangle,
\]
where we have introduced the so-called correlation operator 
\[
\hat{C}=\sum_{ai}C_i^a\hat{a}_{a}^{\dagger}\hat{a}_i  +\sum_{abij}C_{ij}^{ab}\hat{a}_{a}^{\dagger}\hat{a}_{b}^{\dagger}\hat{a}_j\hat{a}_i+\dots
\]
Since the normalization of $\Psi_0$ is at our disposal and since $C_0$ is by hypothesis non-zero, we may arbitrarily set $C_0=1$ with 
corresponding proportional changes in all other coefficients. Using this so-called intermediate normalization we have
\[
\langle \Psi_0 | \Phi_0 \rangle = \langle \Phi_0 | \Phi_0 \rangle = 1, 
\]
resulting in 
\[
|\Psi_0\rangle=(1+\hat{C})|\Phi_0\rangle.
\]
% --- end paragraph admon ---



\subsection*{Full Configuration Interaction Theory}

% --- begin paragraph admon ---
\paragraph{}
We rewrite 
\[
|\Psi_0\rangle=C_0|\Phi_0\rangle+\sum_{ai}C_i^a|\Phi_i^a\rangle+\sum_{abij}C_{ij}^{ab}|\Phi_{ij}^{ab}\rangle+\dots,
\]
in a more compact form as 
\[
|\Psi_0\rangle=\sum_{PH}C_H^P\Phi_H^P=\left(\sum_{PH}C_H^P\hat{A}_H^P\right)|\Phi_0\rangle,
\]
where $H$ stands for $0,1,\dots,n$ hole states and $P$ for $0,1,\dots,n$ particle states. 
Our requirement of unit normalization gives
\[
\langle \Psi_0 | \Phi_0 \rangle = \sum_{PH}|C_H^P|^2= 1,
\]
and the energy can be written as 
\[
E= \langle \Psi_0 | \hat{H} |\Phi_0 \rangle= \sum_{PP'HH'}C_H^{*P}\langle \Phi_H^P | \hat{H} |\Phi_{H'}^{P'} \rangle C_{H'}^{P'}.
\]
% --- end paragraph admon ---



\subsection*{Full Configuration Interaction Theory}

% --- begin paragraph admon ---
\paragraph{}
Normally 
\[
E= \langle \Psi_0 | \hat{H} |\Phi_0 \rangle= \sum_{PP'HH'}C_H^{*P}\langle \Phi_H^P | \hat{H} |\Phi_{H'}^{P'} \rangle C_{H'}^{P'},
\]
is solved by diagonalization setting up the Hamiltonian matrix defined by the basis of all possible Slater determinants. A diagonalization
% to do: add text about Rayleigh-Ritz
is equivalent to finding the variational minimum   of 
\[
 \langle \Psi_0 | \hat{H} |\Phi_0 \rangle-\lambda \langle \Psi_0 |\Phi_0 \rangle,
\]
where $\lambda$ is a variational multiplier to be identified with the energy of the system.
The minimization process results in 
\[
\delta\left[ \langle \Psi_0 | \hat{H} |\Phi_0 \rangle-\lambda \langle \Psi_0 |\Phi_0 \rangle\right]=
\]
\[
\sum_{P'H'}\left\{\delta[C_H^{*P}]\langle \Phi_H^P | \hat{H} |\Phi_{H'}^{P'} \rangle C_{H'}^{P'}+
C_H^{*P}\langle \Phi_H^P | \hat{H} |\Phi_{H'}^{P'} \rangle \delta[C_{H'}^{P'}]-
\lambda( \delta[C_H^{*P}]C_{H'}^{P'}+C_H^{*P}\delta[C_{H'}^{P'}]\right\} = 0.
\]
Since the coefficients $\delta[C_H^{*P}]$ and $\delta[C_{H'}^{P'}]$ are complex conjugates it is necessary and sufficient to require the quantities that multiply with $\delta[C_H^{*P}]$ to vanish.
% --- end paragraph admon ---



\subsection*{Full Configuration Interaction Theory}

% --- begin paragraph admon ---
\paragraph{}

This leads to 
\[
\sum_{P'H'}\langle \Phi_H^P | \hat{H} |\Phi_{H'}^{P'} \rangle C_{H'}^{P'}-\lambda C_H^{P}=0,
\]
for all sets of $P$ and $H$.

If we then multiply by the corresponding $C_H^{*P}$ and sum over $PH$ we obtain
\[ 
\sum_{PP'HH'}C_H^{*P}\langle \Phi_H^P | \hat{H} |\Phi_{H'}^{P'} \rangle C_{H'}^{P'}-\lambda\sum_{PH}|C_H^P|^2=0,
\]
leading to the identification $\lambda = E$. This means that we have for all $PH$ sets
\begin{equation}
\sum_{P'H'}\langle \Phi_H^P | \hat{H} -E|\Phi_{H'}^{P'} \rangle = 0. \label{eq:fullci}
\end{equation}
% --- end paragraph admon ---



\subsection*{Full Configuration Interaction Theory}

% --- begin paragraph admon ---
\paragraph{}
An alternative way to derive the last equation is to start from 
\[
(\hat{H} -E)|\Psi_0\rangle = (\hat{H} -E)\sum_{P'H'}C_{H'}^{P'}|\Phi_{H'}^{P'} \rangle=0, 
\]
and if this equation is successively projected against all $\Phi_H^P$ in the expansion of $\Psi$, then the last equation on the previous slide
results.   As stated previously, one solves this equation normally by diagonalization. If we are able to solve this equation exactly (that is
numerically exactly) in a large Hilbert space (it will be truncated in terms of the number of single-particle states included in the definition
of Slater determinants), it can then serve as a benchmark for other many-body methods which approximate the correlation operator
$\hat{C}$.
% --- end paragraph admon ---



\subsection*{Example of a Hamiltonian matrix}

% --- begin paragraph admon ---
\paragraph{}
Suppose, as an example, that we have six fermions below the Fermi level.
This means that we can make at most $6p-6h$ excitations. If we have an infinity of single particle states above the Fermi level, we will obviously have an infinity of say $2p-2h$ excitations. Each such way to configure the particles is called a \textbf{configuration}. We will always have to truncate in the basis of single-particle states.
This gives us a finite number of possible Slater determinants. Our Hamiltonian matrix would then look like (where each block can have a large dimensionalities):


\begin{quote}
\begin{tabular}{cccccccc}
\hline
\multicolumn{1}{c}{  } & \multicolumn{1}{c}{ $0p-0h$ } & \multicolumn{1}{c}{ $1p-1h$ } & \multicolumn{1}{c}{ $2p-2h$ } & \multicolumn{1}{c}{ $3p-3h$ } & \multicolumn{1}{c}{ $4p-4h$ } & \multicolumn{1}{c}{ $5p-5h$ } & \multicolumn{1}{c}{ $6p-6h$ } \\
\hline
$0p-0h$ & x       & x       & x       & 0       & 0       & 0       & 0       \\
$1p-1h$ & x       & x       & x       & x       & 0       & 0       & 0       \\
$2p-2h$ & x       & x       & x       & x       & x       & 0       & 0       \\
$3p-3h$ & 0       & x       & x       & x       & x       & x       & 0       \\
$4p-4h$ & 0       & 0       & x       & x       & x       & x       & x       \\
$5p-5h$ & 0       & 0       & 0       & x       & x       & x       & x       \\
$6p-6h$ & 0       & 0       & 0       & 0       & x       & x       & x       \\
\hline
\end{tabular}
\end{quote}

\noindent
with a two-body force. Why are there non-zero blocks of elements?
% --- end paragraph admon ---



\subsection*{Example of a Hamiltonian matrix with a Hartree-Fock basis}

% --- begin paragraph admon ---
\paragraph{}
If we use a Hartree-Fock basis, this corresponds to a particular unitary transformation where matrix elements of the type $\langle 0p-0h \vert \hat{H} \vert 1p-1h\rangle =\langle \Phi_0 | \hat{H}|\Phi_{i}^{a}\rangle=0$ and our Hamiltonian matrix becomes 


\begin{quote}
\begin{tabular}{cccccccc}
\hline
\multicolumn{1}{c}{  } & \multicolumn{1}{c}{ $0p-0h$ } & \multicolumn{1}{c}{ $1p-1h$ } & \multicolumn{1}{c}{ $2p-2h$ } & \multicolumn{1}{c}{ $3p-3h$ } & \multicolumn{1}{c}{ $4p-4h$ } & \multicolumn{1}{c}{ $5p-5h$ } & \multicolumn{1}{c}{ $6p-6h$ } \\
\hline
$0p-0h$ & $\tilde{x}$ & 0           & $\tilde{x}$ & 0           & 0           & 0           & 0           \\
$1p-1h$ & 0           & $\tilde{x}$ & $\tilde{x}$ & $\tilde{x}$ & 0           & 0           & 0           \\
$2p-2h$ & $\tilde{x}$ & $\tilde{x}$ & $\tilde{x}$ & $\tilde{x}$ & $\tilde{x}$ & 0           & 0           \\
$3p-3h$ & 0           & $\tilde{x}$ & $\tilde{x}$ & $\tilde{x}$ & $\tilde{x}$ & $\tilde{x}$ & 0           \\
$4p-4h$ & 0           & 0           & $\tilde{x}$ & $\tilde{x}$ & $\tilde{x}$ & $\tilde{x}$ & $\tilde{x}$ \\
$5p-5h$ & 0           & 0           & 0           & $\tilde{x}$ & $\tilde{x}$ & $\tilde{x}$ & $\tilde{x}$ \\
$6p-6h$ & 0           & 0           & 0           & 0           & $\tilde{x}$ & $\tilde{x}$ & $\tilde{x}$ \\
\hline
\end{tabular}
\end{quote}

\noindent
% --- end paragraph admon ---



\subsection*{Shell-model jargon}

% --- begin paragraph admon ---
\paragraph{}
If we do not make any truncations in the possible sets of Slater determinants (many-body states) we can make by distributing $A$ nucleons among $n$ single-particle states, we call such a calculation for \textbf{Full configuration interaction theory}

If we make truncations, we have different possibilities

\begin{itemize}
\item The standard nuclear shell-model. Here we define an effective Hilbert space with respect to a given core. The calculations are normally then performed for all many-body states that can be constructed from the effective Hilbert spaces. This approach requires a properly defined effective Hamiltonian

\item We can truncate in the number of excitations. For example, we can limit the possible Slater determinants to only $1p-1h$ and $2p-2h$ excitations. This is called a configuration interaction calculation at the level of singles and doubles excitations, or just CISD. 

\item We can limit the number of excitations in terms of the excitation energies. If we do not define a core, this defines normally what is called the no-core shell-model approach. 
\end{itemize}

\noindent
What happens if we have a three-body interaction and a Hartree-Fock basis?
% --- end paragraph admon ---



\subsection*{FCI and the exponential growth}

% --- begin paragraph admon ---
\paragraph{}
Full configuration interaction theory calculations provide in principle, if we can diagonalize numerically, all states of interest. The dimensionality of the problem explodes however quickly.

The total number of Slater determinants which can be built with say $N$ neutrons distributed among $n$ single particle states is
\[
\left (\begin{array}{c} n \\ N\end{array} \right) =\frac{n!}{(n-N)!N!}. 
\]

For a model space which comprises the first for major shells only $0s$, $0p$, $1s0d$ and $1p0f$ we have $40$ single particle states for neutrons and protons.  For the eight neutrons of oxygen-16 we would then have
\[
\left (\begin{array}{c} 40 \\ 8\end{array} \right) =\frac{40!}{(32)!8!}\sim 10^{9}, 
\]
and multiplying this with the number of proton Slater determinants we end up with approximately with a dimensionality $d$ of $d\sim 10^{18}$.
% --- end paragraph admon ---



\subsection*{Exponential wall}

% --- begin paragraph admon ---
\paragraph{}
This number can be reduced if we look at specific symmetries only. However, the dimensionality explodes quickly!

\begin{itemize}
\item For Hamiltonian matrices of dimensionalities  which are smaller than $d\sim 10^5$, we would use so-called direct methods for diagonalizing the Hamiltonian matrix

\item For larger dimensionalities iterative eigenvalue solvers like Lanczos' method are used. The most efficient codes at present can handle matrices of $d\sim 10^{10}$. 
\end{itemize}

\noindent
% --- end paragraph admon ---



\subsection*{A non-practical way of solving the eigenvalue problem}

% --- begin paragraph admon ---
\paragraph{}
To see this, we look at the contributions arising from 
\[
\langle \Phi_H^P | = \langle \Phi_0|
\]
in  Eq.~(\ref{eq:fullci}), that is we multiply with $\langle \Phi_0 |$
from the left in 
\[
(\hat{H} -E)\sum_{P'H'}C_{H'}^{P'}|\Phi_{H'}^{P'} \rangle=0. 
\]
If we assume that we have a two-body operator at most, Slater's rule gives then an equation for the 
correlation energy in terms of $C_i^a$ and $C_{ij}^{ab}$ only.  We get then
\[
\langle \Phi_0 | \hat{H} -E| \Phi_0\rangle + \sum_{ai}\langle \Phi_0 | \hat{H} -E|\Phi_{i}^{a} \rangle C_{i}^{a}+
\sum_{abij}\langle \Phi_0 | \hat{H} -E|\Phi_{ij}^{ab} \rangle C_{ij}^{ab}=0,
\]
or 
\[
E-E_0 =\Delta E=\sum_{ai}\langle \Phi_0 | \hat{H}|\Phi_{i}^{a} \rangle C_{i}^{a}+
\sum_{abij}\langle \Phi_0 | \hat{H}|\Phi_{ij}^{ab} \rangle C_{ij}^{ab},
\]
where the energy $E_0$ is the reference energy and $\Delta E$ defines the so-called correlation energy.
The single-particle basis functions  could be the results of a Hartree-Fock calculation or just the eigenstates of the non-interacting part of the Hamiltonian.
% --- end paragraph admon ---



\subsection*{A non-practical way of solving the eigenvalue problem}

% --- begin paragraph admon ---
\paragraph{}
To see this, we look at the contributions arising from 
\[
\langle \Phi_H^P | = \langle \Phi_0|
\]
in  Eq.~(\ref{eq:fullci}), that is we multiply with $\langle \Phi_0 |$
from the left in 
\[
(\hat{H} -E)\sum_{P'H'}C_{H'}^{P'}|\Phi_{H'}^{P'} \rangle=0. 
\]
% --- end paragraph admon ---



\subsection*{A non-practical way of solving the eigenvalue problem}

% --- begin paragraph admon ---
\paragraph{}
If we assume that we have a two-body operator at most, Slater's rule gives then an equation for the 
correlation energy in terms of $C_i^a$ and $C_{ij}^{ab}$ only.  We get then
\[
\langle \Phi_0 | \hat{H} -E| \Phi_0\rangle + \sum_{ai}\langle \Phi_0 | \hat{H} -E|\Phi_{i}^{a} \rangle C_{i}^{a}+
\sum_{abij}\langle \Phi_0 | \hat{H} -E|\Phi_{ij}^{ab} \rangle C_{ij}^{ab}=0,
\]
or 
\[
E-E_0 =\Delta E=\sum_{ai}\langle \Phi_0 | \hat{H}|\Phi_{i}^{a} \rangle C_{i}^{a}+
\sum_{abij}\langle \Phi_0 | \hat{H}|\Phi_{ij}^{ab} \rangle C_{ij}^{ab},
\]
where the energy $E_0$ is the reference energy and $\Delta E$ defines the so-called correlation energy.
The single-particle basis functions  could be the results of a Hartree-Fock calculation or just the eigenstates of the non-interacting part of the Hamiltonian.
% --- end paragraph admon ---



\subsection*{Rewriting the FCI equation}

% --- begin paragraph admon ---
\paragraph{}
In our notes on Hartree-Fock calculations, 
we have already computed the matrix $\langle \Phi_0 | \hat{H}|\Phi_{i}^{a}\rangle $ and $\langle \Phi_0 | \hat{H}|\Phi_{ij}^{ab}\rangle$.  If we are using a Hartree-Fock basis, then the matrix elements
$\langle \Phi_0 | \hat{H}|\Phi_{i}^{a}\rangle=0$ and we are left with a \emph{correlation energy} given by
\[
E-E_0 =\Delta E^{HF}=\sum_{abij}\langle \Phi_0 | \hat{H}|\Phi_{ij}^{ab} \rangle C_{ij}^{ab}. 
\]
% --- end paragraph admon ---




\subsection*{Rewriting the FCI equation}

% --- begin paragraph admon ---
\paragraph{}
Inserting the various matrix elements we can rewrite the previous equation as
\[
\Delta E=\sum_{ai}\langle i| \hat{f}|a \rangle C_{i}^{a}+
\sum_{abij}\langle ij | \hat{v}| ab \rangle C_{ij}^{ab}.
\]
This equation determines the correlation energy but not the coefficients $C$.
% --- end paragraph admon ---



\subsection*{Rewriting the FCI equation, does not stop here}

% --- begin paragraph admon ---
\paragraph{}
We need more equations. Our next step is to set up
\[
\langle \Phi_i^a | \hat{H} -E| \Phi_0\rangle + \sum_{bj}\langle \Phi_i^a | \hat{H} -E|\Phi_{j}^{b} \rangle C_{j}^{b}+
\sum_{bcjk}\langle \Phi_i^a | \hat{H} -E|\Phi_{jk}^{bc} \rangle C_{jk}^{bc}+
\sum_{bcdjkl}\langle \Phi_i^a | \hat{H} -E|\Phi_{jkl}^{bcd} \rangle C_{jkl}^{bcd}=0,
\]
as this equation will allow us to find an expression for the coefficents $C_i^a$ since we can rewrite this equation as 
\[
\langle i | \hat{f}| a\rangle +\langle \Phi_i^a | \hat{H}|\Phi_{i}^{a} \rangle C_{i}^{a}+ \sum_{bj\ne ai}\langle \Phi_i^a | \hat{H}|\Phi_{j}^{b} \rangle C_{j}^{b}+
\sum_{bcjk}\langle \Phi_i^a | \hat{H}|\Phi_{jk}^{bc} \rangle C_{jk}^{bc}+
\sum_{bcdjkl}\langle \Phi_i^a | \hat{H}|\Phi_{jkl}^{bcd} \rangle C_{jkl}^{bcd}=EC_i^a.
\]
% --- end paragraph admon ---



\subsection*{Rewriting the FCI equation, please stop here}

% --- begin paragraph admon ---
\paragraph{}
We see that on the right-hand side we have the energy $E$. This leads to a non-linear equation in the unknown coefficients. 
These equations are normally solved iteratively ( that is we can start with a guess for the coefficients $C_i^a$). A common choice is to use perturbation theory for the first guess, setting thereby
\[
 C_{i}^{a}=\frac{\langle i | \hat{f}| a\rangle}{\epsilon_i-\epsilon_a}.
\]
% --- end paragraph admon ---



\subsection*{Rewriting the FCI equation, more to add}

% --- begin paragraph admon ---
\paragraph{}
The observant reader will however see that we need an equation for $C_{jk}^{bc}$ and $C_{jkl}^{bcd}$ as well.
To find equations for these coefficients we need then to continue our multiplications from the left with the various
$\Phi_{H}^P$ terms. 

For $C_{jk}^{bc}$ we need then
\[
\langle \Phi_{ij}^{ab} | \hat{H} -E| \Phi_0\rangle + \sum_{kc}\langle \Phi_{ij}^{ab} | \hat{H} -E|\Phi_{k}^{c} \rangle C_{k}^{c}+
\]
\[
\sum_{cdkl}\langle \Phi_{ij}^{ab} | \hat{H} -E|\Phi_{kl}^{cd} \rangle C_{kl}^{cd}+\sum_{cdeklm}\langle \Phi_{ij}^{ab} | \hat{H} -E|\Phi_{klm}^{cde} \rangle C_{klm}^{cde}+\sum_{cdefklmn}\langle \Phi_{ij}^{ab} | \hat{H} -E|\Phi_{klmn}^{cdef} \rangle C_{klmn}^{cdef}=0,
\]
and we can isolate the coefficients $C_{kl}^{cd}$ in a similar way as we did for the coefficients $C_{i}^{a}$.
% --- end paragraph admon ---



\subsection*{Rewriting the FCI equation, more to add}

% --- begin paragraph admon ---
\paragraph{}
A standard choice for the first iteration is to set 
\[
C_{ij}^{ab} =\frac{\langle ij \vert \hat{v} \vert ab \rangle}{\epsilon_i+\epsilon_j-\epsilon_a-\epsilon_b}.
\]
At the end we can rewrite our solution of the Schroedinger equation in terms of $n$ coupled equations for the coefficients $C_H^P$.
This is a very cumbersome way of solving the equation. However, by using this iterative scheme we can illustrate how we can compute the
various terms in the wave operator or correlation operator $\hat{C}$. We will later identify the calculation of the various terms $C_H^P$
as parts of different many-body approximations to full CI. In particular, we can  relate this non-linear scheme with Coupled Cluster theory and
many-body perturbation theory.
% --- end paragraph admon ---



\subsection*{Summarizing FCI and bringing in approximative methods}

% --- begin paragraph admon ---
\paragraph{}

If we can diagonalize large matrices, FCI is the method of choice since:
\begin{itemize}
\item It gives all eigenvalues, ground state and excited states

\item The eigenvectors are obtained directly from the coefficients $C_H^P$ which result from the diagonalization

\item We can compute easily expectation values of other operators, as well as transition probabilities

\item Correlations are easy to understand in terms of contributions to a given operator beyond the Hartree-Fock contribution. This is the standard approach in  many-body theory. 
\end{itemize}

\noindent
% --- end paragraph admon ---



\subsection*{Definition of the correlation energy}

% --- begin paragraph admon ---
\paragraph{}
The correlation energy is defined as, with a two-body Hamiltonian,  
\[
\Delta E=\sum_{ai}\langle i| \hat{f}|a \rangle C_{i}^{a}+
\sum_{abij}\langle ij | \hat{v}| ab \rangle C_{ij}^{ab}.
\]
The coefficients $C$ result from the solution of the eigenvalue problem. 
The energy of say the ground state is then
\[
E=E_{ref}+\Delta E,
\]
where the so-called reference energy is the energy we obtain from a Hartree-Fock calculation, that is
\[
E_{ref}=\langle \Phi_0 \vert \hat{H} \vert \Phi_0 \rangle.
\]
% --- end paragraph admon ---



\subsection*{FCI equation and the coefficients}

% --- begin paragraph admon ---
\paragraph{}

However, as we have seen, even for a small case like the four first major shells and a nucleus like oxygen-16, the dimensionality becomes quickly intractable. If we wish to include single-particle states that reflect weakly bound systems, we need a much larger single-particle basis. We need thus approximative methods that sum specific correlations to infinite order. 

Popular methods are
\begin{itemize}
\item \href{{http://www.sciencedirect.com/science/article/pii/0370157395000126}}{Many-body perturbation theory (in essence a Taylor expansion)}

\item \href{{http://iopscience.iop.org/article/10.1088/0034-4885/77/9/096302/meta}}{Coupled cluster theory (coupled non-linear equations)}

\item Green's function approaches (matrix inversion)

\item \href{{http://journals.aps.org/prl/abstract/10.1103/PhysRevLett.106.222502}}{Similarity group transformation methods (coupled ordinary differential equations)}
\end{itemize}

\noindent
All these methods start normally with a Hartree-Fock basis as the calculational basis.
% --- end paragraph admon ---



\subsection*{Important ingredients to have in codes}

% --- begin paragraph admon ---
\paragraph{}

\begin{itemize}
\item Be able to validate and verify  the  algorithms. 

\item Include concepts like unit testing. Gives the possibility to test and validate several or all parts of the code.

\item Validation and verification are then included \emph{naturally} and one can develop a better attitude to what is meant with an ethically sound scientific approach.
\end{itemize}

\noindent
% --- end paragraph admon ---



\subsection*{A structured approach to solving problems}

% --- begin paragraph admon ---
\paragraph{}
In the steps that lead to the development of clean code you should  think of 
\begin{enumerate}
  \item How to structure a code in terms of functions  (use IDEs or advanced text editors like sublime or atom)

  \item How to make a module

  \item How to read input data flexibly from the command line or files

  \item How to create graphical/web user interfaces

  \item How to write unit tests  

  \item How to refactor code in terms of classes (instead of functions only)

  \item How to conduct and automate large-scale numerical experiments

  \item How to write scientific reports in various formats ({\LaTeX}, HTML, doconce)
\end{enumerate}

\noindent
% --- end paragraph admon ---



\subsection*{Additional benefits}

% --- begin paragraph admon ---
\paragraph{}
Many of the above aspetcs  will save you a lot of time when you incrementally extend software over time from simpler to more complicated problems. In particular, you will benefit from many good habits:
\begin{enumerate}
\item New code is added in a modular fashion to a library (modules)

\item Programs are run through convenient user interfaces

\item It takes one quick command to let all your code undergo heavy testing 

\item Tedious manual work with running programs is automated,

\item Your scientific investigations are reproducible, scientific reports with top quality typesetting are produced both for paper and electronic devices. Use version control software like \href{{https://git-scm.com/}}{git} and repositories like \href{{https://github.com/}}{github}
\end{enumerate}

\noindent
% --- end paragraph admon ---



\subsection*{Unit Testing}

% --- begin paragraph admon ---
\paragraph{}
Unit Testing is the practice of testing the smallest testable parts,
called units, of an application individually and independently to
determine if they behave exactly as expected. 

Unit tests (short code
fragments) are usually written such that they can be preformed at any
time during the development to continually verify the behavior of the
code. 

In this way, possible bugs will be identified early in the
development cycle, making the debugging at later stages much
easier.
% --- end paragraph admon ---



\subsection*{Unit Testing, benefits}

% --- begin paragraph admon ---
\paragraph{}
There are many benefits associated with Unit Testing, such as
\begin{itemize}
  \item It increases confidence in changing and maintaining code. Big changes can be made to the code quickly, since the tests will ensure that everything still is working properly.

  \item Since the code needs to be modular to make Unit Testing possible, the code will be easier to reuse. This improves the code design.

  \item Debugging is easier, since when a test fails, only the latest changes need to be debugged.
\begin{itemize}

   \item Different parts of a project can be tested without the need to wait for the other parts to be available.

\end{itemize}

\noindent
  \item A unit test can serve as a documentation on the functionality of a unit of the code.
\end{itemize}

\noindent
% --- end paragraph admon ---



\subsection*{Simple example of unit test}

% --- begin paragraph admon ---
\paragraph{}
Look up the guide on how to install unit tests for c++ at course webpage. This is the version with classes.



















\begin{minted}[fontsize=\fontsize{9pt}{9pt},linenos=false,mathescape,baselinestretch=1.0,fontfamily=tt,xleftmargin=7mm]{c++}
#include <unittest++/UnitTest++.h>

class MyMultiplyClass{
public:
    double multiply(double x, double y) {
        return x * y;
    }
};

TEST(MyMath) {
    MyMultiplyClass my;
    CHECK_EQUAL(56, my.multiply(7,8));
}

int main()
{
    return UnitTest::RunAllTests();
}

\end{minted}
% --- end paragraph admon ---



\subsection*{Simple example of unit test}

% --- begin paragraph admon ---
\paragraph{}
And without classes
















\begin{minted}[fontsize=\fontsize{9pt}{9pt},linenos=false,mathescape,baselinestretch=1.0,fontfamily=tt,xleftmargin=7mm]{c++}
#include <unittest++/UnitTest++.h>


double multiply(double x, double y) {
    return x * y;
}

TEST(MyMath) {
    CHECK_EQUAL(56, multiply(7,8));
}

int main()
{
    return UnitTest::RunAllTests();
} 

\end{minted}

For Fortran users, the link at \href{{http://sourceforge.net/projects/fortranxunit/}}{\nolinkurl{http://sourceforge.net/projects/fortranxunit/}} contains a similar
software for unit testing. For Python go to \href{{https://docs.python.org/2/library/unittest.html}}{\nolinkurl{https://docs.python.org/2/library/unittest.html}}.
% --- end paragraph admon ---



\subsection*{\href{{https://github.com/philsquared/Catch/blob/master/docs/tutorial.md}}{Unit tests}}

% --- begin paragraph admon ---
\paragraph{}
There are many types of \textbf{unit test} libraries. One which is very popular with C++ programmers is \href{{https://github.com/philsquared/Catch/blob/master/docs/tutorial.md}}{Catch}

Catch is header only. All you need to do is drop the file(s) somewhere reachable from your project - either in some central location you can set your header search path to find, or directly into your project tree itself! 

This is a particularly good option for other Open-Source projects that want to use Catch for their test suite.
% --- end paragraph admon ---



\subsection*{Examples}

Computing factorials




\begin{minted}[fontsize=\fontsize{9pt}{9pt},linenos=false,mathescape,baselinestretch=1.0,fontfamily=tt,xleftmargin=7mm]{c++}
inline unsigned int Factorial( unsigned int number ) {
  return number > 1 ? Factorial(number-1)*number : 1;
}

\end{minted}


\subsection*{Factorial Example}

Simple test where we put everything in a single file 
















\begin{minted}[fontsize=\fontsize{9pt}{9pt},linenos=false,mathescape,baselinestretch=1.0,fontfamily=tt,xleftmargin=7mm]{c++}
#define CATCH_CONFIG_MAIN  // This tells Catch to provide a main()
#include "catch.hpp"
inline unsigned int Factorial( unsigned int number ) {
  return number > 1 ? Factorial(number-1)*number : 1;
}

TEST_CASE( "Factorials are computed", "[factorial]" ) {
    REQUIRE( Factorial(0) == 1 );
    REQUIRE( Factorial(1) == 1 );
    REQUIRE( Factorial(2) == 2 );
    REQUIRE( Factorial(3) == 6 );
    REQUIRE( Factorial(10) == 3628800 );
}


\end{minted}

This will compile to a complete executable which responds to command line arguments. If you just run it with no arguments it will execute all test cases (in this case there is just one), report any failures, report a summary of how many tests passed and failed and return the number of failed tests.

\subsection*{What did we do (1)?}
All we did was 


\begin{minted}[fontsize=\fontsize{9pt}{9pt},linenos=false,mathescape,baselinestretch=1.0,fontfamily=tt,xleftmargin=7mm]{c++}
#define 

\end{minted}

one identifier and 


\begin{minted}[fontsize=\fontsize{9pt}{9pt},linenos=false,mathescape,baselinestretch=1.0,fontfamily=tt,xleftmargin=7mm]{c++}
#include 

\end{minted}

one header and we got everything - even an implementation of main() that will respond to command line arguments. 
Once you have more than one file with unit tests in you'll just need to 


\begin{minted}[fontsize=\fontsize{9pt}{9pt},linenos=false,mathescape,baselinestretch=1.0,fontfamily=tt,xleftmargin=7mm]{c++}
#include "catch.hpp" 

\end{minted}

and go. Usually it's a good idea to have a dedicated implementation file that just has 



\begin{minted}[fontsize=\fontsize{9pt}{9pt},linenos=false,mathescape,baselinestretch=1.0,fontfamily=tt,xleftmargin=7mm]{c++}
#define CATCH_CONFIG_MAIN 
#include "catch.hpp". 

\end{minted}

You can also provide your own implementation of main and drive Catch yourself.

\subsection*{What did we do (2)?}
We introduce test cases with the 


\begin{minted}[fontsize=\fontsize{9pt}{9pt},linenos=false,mathescape,baselinestretch=1.0,fontfamily=tt,xleftmargin=7mm]{c++}
TEST_CASE 

\end{minted}

macro.

The test name must be unique. You can run sets of tests by specifying a wildcarded test name or a tag expression. 
All we did was \textbf{define} one identifier and \textbf{include} one header and we got everything.

We write our individual test assertions using the 


\begin{minted}[fontsize=\fontsize{9pt}{9pt},linenos=false,mathescape,baselinestretch=1.0,fontfamily=tt,xleftmargin=7mm]{c++}
REQUIRE 

\end{minted}

macro.

\subsection*{Unit test summary and testing approach}

% --- begin paragraph admon ---
\paragraph{}
Three levels of tests
\begin{enumerate}
\item Microscopic level: testing small parts of code, use often unit test libraries

\item Mesoscopic level: testing the integration of various parts  of your code

\item Macroscopic level: testing that the final result is ok
\end{enumerate}

\noindent
% --- end paragraph admon ---

 

\subsection*{Coding Recommendations}
Writing clean and clear code is an art and reflects 
your understanding of 

\begin{enumerate}
\item derivation, verification, and implementation of algorithms

\item what can go wrong with algorithms

\item overview of important, known algorithms

\item how algorithms are used to solve mathematical problems

\item reproducible science and ethics

\item algorithmic thinking for gaining deeper insights about scientific problems
\end{enumerate}

\noindent
Computing is understanding and your understanding is reflected in your abilities to
write clear and clean code.

\subsection*{Summary and recommendations}
Some simple hints and tips in order to write clean and clear code
\begin{enumerate}
\item Spell out the algorithm and have a top-down approach to the flow of data

\item Start with coding as close as possible to eventual mathematical expressions

\item Use meaningful names for variables

\item Split tasks in simple functions and modules/classes

\item Functions should return as few as possible variables

\item Use unit tests and make sure your codes are producing the correct results

\item Where possible use symbolic coding to autogenerate code and check results

\item Make a proper timing of your algorithms

\item Use version control and make your science reproducible

\item Use IDEs or smart editors with debugging and analysis tools.

\item Automatize your computations interfacing high-level and compiled languages like C++ and Fortran.

\item .....
\end{enumerate}

\noindent
\subsection*{Building a many-body basis}

% --- begin paragraph admon ---
\paragraph{}
Here we will discuss how we can set up a single-particle basis which we can use in the various parts of our projects, from the simple pairing model to infinite nuclear matter. We will use here the simple pairing model to illustrate in particular how to set up a single-particle basis. We will also use this do discuss standard FCI approaches like:
\begin{enumerate}
 \item Standard shell-model basis in one or two major shells

 \item Full CI in a given basis and no truncations

 \item CISD and CISDT approximations

 \item No-core shell model and truncation in excitation energy
\end{enumerate}

\noindent
% --- end paragraph admon ---



\subsection*{Building a many-body basis}

% --- begin paragraph admon ---
\paragraph{}
An important step in an FCI code  is to construct the many-body basis.  

While the formalism is independent of the choice of basis, the \textbf{effectiveness} of a calculation 
will certainly be basis dependent. 

Furthermore there are common conventions useful to know.

First, the single-particle basis has angular momentum as a good quantum number.  You can 
imagine the single-particle wavefunctions being generated by a one-body Hamiltonian, 
for example a harmonic oscillator.  Modifications include harmonic oscillator plus 
spin-orbit splitting, or self-consistent mean-field potentials, or the Woods-Saxon potential which mocks 
up the self-consistent mean-field. 
For nuclei, the harmonic oscillator, modified by spin-orbit splitting, provides a useful language 
for describing single-particle states.
% --- end paragraph admon ---



\subsection*{Building a many-body basis}

% --- begin paragraph admon ---
\paragraph{}
Each single-particle state is labeled by the following quantum numbers: 

\begin{itemize}
\item Orbital angular momentum $l$

\item Intrinsic spin $s$ = 1/2 for protons and neutrons

\item Angular momentum $j = l \pm 1/2$

\item $z$-component $j_z$ (or $m$)

\item Some labeling of the radial wavefunction, typically $n$ the number of nodes in  the radial wavefunction, but in the case of harmonic oscillator one can also use the principal quantum number $N$, where the harmonic oscillator energy is $(N+3/2)\hbar \omega$.  
\end{itemize}

\noindent
In this format one labels states by $n(l)_j$, with $(l)$ replaced by a letter:
$s$ for $l=0$, $p$ for $l=1$, $d$ for $l=2$, $f$ for $l=3$, and thenceforth alphabetical.
% --- end paragraph admon ---



\subsection*{Building a many-body basis}

% --- begin paragraph admon ---
\paragraph{}
 In practice the single-particle space has to be severely truncated.  This truncation is 
typically based upon the single-particle energies, which is the effective energy 
from a mean-field potential. 

Sometimes we freeze the core and only consider a valence space. For example, one 
may assume a frozen $^{4}\mbox{He}$ core, with two protons and two neutrons in the $0s_{1/2}$ 
shell, and then only allow active particles in the $0p_{1/2}$ and $0p_{3/2}$ orbits. 

Another example is a frozen $^{16}\mbox{O}$ core, with eight protons and eight neutrons filling the 
$0s_{1/2}$,  $0p_{1/2}$ and $0p_{3/2}$ orbits, with valence particles in the 
$0d_{5/2}, 1s_{1/2}$ and $0d_{3/2}$ orbits.

Sometimes we refer to nuclei by the valence space where their last nucleons go.  
So, for example, we call $^{12}\mbox{C}$ a $p$-shell nucleus, while $^{26}\mbox{Al}$ is an 
$sd$-shell nucleus and $^{56}\mbox{Fe}$ is a $pf$-shell nucleus.
% --- end paragraph admon ---



\subsection*{Building a many-body basis}

% --- begin paragraph admon ---
\paragraph{}
There are different kinds of truncations.

\begin{itemize}
\item For example, one can start with `filled' orbits (almost always the lowest), and then  allow one, two, three... particles excited out of those filled orbits. These are called  1p-1h, 2p-2h, 3p-3h excitations. 

\item Alternately, one can state a maximal orbit and allow all possible configurations with  particles occupying states up to that maximum. This is called \emph{full configuration}.

\item Finally, for particular use in nuclear physics, there is the \emph{energy} truncation, also  called the $N\hbar\Omega$ or $N_{max}$ truncation. 
\end{itemize}

\noindent
% --- end paragraph admon ---



\subsection*{Building a many-body basis}

% --- begin paragraph admon ---
\paragraph{}
Here one works in a harmonic oscillator basis, with each major oscillator shell assigned  a principal quantum number $N=0,1,2,3,...$. 
The $N\hbar\Omega$ or $N_{max}$ truncation: Any configuration is given an noninteracting energy, which is the sum 
of the single-particle harmonic oscillator energies. (Thus this ignores 
spin-orbit splitting.)

Excited state are labeled relative to the lowest configuration by the 
number of harmonic oscillator quanta.

This truncation is useful because if one includes \emph{all} configuration up to 
some $N_{max}$, and has a translationally invariant interaction, then the intrinsic 
motion and the center-of-mass motion factor. In other words, we can know exactly 
the center-of-mass wavefunction. 

In almost all cases, the many-body Hamiltonian is rotationally invariant. This means 
it commutes with the operators $\hat{J}^2, \hat{J}_z$ and so eigenstates will have 
good $J,M$. Furthermore, the eigenenergies do not depend upon the orientation $M$. 

Therefore we can choose to construct a many-body basis which has fixed $M$; this is 
called an $M$-scheme basis. 

Alternately, one can construct a many-body basis which has fixed $J$, or a $J$-scheme 
basis.
% --- end paragraph admon ---



\subsection*{Building a many-body basis}

% --- begin paragraph admon ---
\paragraph{}
The Hamiltonian matrix will have smaller dimensions (a factor of 10 or more) in the $J$-scheme than in the $M$-scheme. 
On the other hand, as we'll show in the next slide, the $M$-scheme is very easy to 
construct with Slater determinants, while the $J$-scheme basis states, and thus the 
matrix elements, are more complicated, almost always being linear combinations of 
$M$-scheme states. $J$-scheme bases are important and useful, but we'll focus on the 
simpler $M$-scheme.

The quantum number $m$ is additive (because the underlying group is Abelian): 
if a Slater determinant $\hat{a}_i^\dagger \hat{a}^\dagger_j \hat{a}^\dagger_k \ldots | 0 \rangle$ 
is built from single-particle states all with good $m$, then the total 
\[
M = m_i + m_j + m_k + \ldots
\]
This is \emph{not} true of $J$, because the angular momentum group SU(2) is not Abelian.
% --- end paragraph admon ---



\subsection*{Building a many-body basis}

% --- begin paragraph admon ---
\paragraph{}

The upshot is that 
\begin{itemize}
\item It is easy to construct a Slater determinant with good total $M$;

\item It is trivial to calculate $M$ for each Slater determinant;

\item So it is easy to construct an $M$-scheme basis with fixed total $M$.
\end{itemize}

\noindent
Note that the individual $M$-scheme basis states will \emph{not}, in general, 
have good total $J$. 
Because the Hamiltonian is rotationally invariant, however, the eigenstates will 
have good $J$. (The situation is muddied when one has states of different $J$ that are 
nonetheless degenerate.)
% --- end paragraph admon ---



\subsection*{Building a many-body basis}

% --- begin paragraph admon ---
\paragraph{}
Example: two $j=1/2$ orbits


\begin{quote}
\begin{tabular}{ccccc}
\hline
\multicolumn{1}{c}{ Index } & \multicolumn{1}{c}{ $n$ } & \multicolumn{1}{c}{ $l$ } & \multicolumn{1}{c}{ $j$ } & \multicolumn{1}{c}{ $m_j$ } \\
\hline
1     & 0   & 0   & 1/2 & -1/2  \\
2     & 0   & 0   & 1/2 & 1/2   \\
3     & 1   & 0   & 1/2 & -1/2  \\
4     & 1   & 0   & 1/2 & 1/2   \\
\hline
\end{tabular}
\end{quote}

\noindent
Note that the order is arbitrary.
% --- end paragraph admon ---



\subsection*{Building a many-body basis}

% --- begin paragraph admon ---
\paragraph{}
There are $\left ( \begin{array}{c} 4 \\ 2 \end{array} \right) = 6$ two-particle states, 
which we list with the total $M$:


\begin{quote}
\begin{tabular}{cc}
\hline
\multicolumn{1}{c}{ Occupied } & \multicolumn{1}{c}{ $M$ } \\
\hline
1,2      & 0   \\
1,3      & -1  \\
1,4      & 0   \\
2,3      & 0   \\
2,4      & 1   \\
3,4      & 0   \\
\hline
\end{tabular}
\end{quote}

\noindent
There are 4 states with $M= 0$, 
and 1 each with $M = \pm 1$.
% --- end paragraph admon ---



\subsection*{Building a many-body basis}

% --- begin paragraph admon ---
\paragraph{}
As another example, consider using only single particle states from the $0d_{5/2}$ space. 
They have the following quantum numbers


\begin{quote}
\begin{tabular}{ccccc}
\hline
\multicolumn{1}{c}{ Index } & \multicolumn{1}{c}{ $n$ } & \multicolumn{1}{c}{ $l$ } & \multicolumn{1}{c}{ $j$ } & \multicolumn{1}{c}{ $m_j$ } \\
\hline
1     & 0   & 2   & 5/2 & -5/2  \\
2     & 0   & 2   & 5/2 & -3/2  \\
3     & 0   & 2   & 5/2 & -1/2  \\
4     & 0   & 2   & 5/2 & 1/2   \\
5     & 0   & 2   & 5/2 & 3/2   \\
6     & 0   & 2   & 5/2 & 5/2   \\
\hline
\end{tabular}
\end{quote}

\noindent
% --- end paragraph admon ---



\subsection*{Building a many-body basis}

% --- begin paragraph admon ---
\paragraph{}
There are $\left ( \begin{array}{c} 6 \\ 2 \end{array} \right) = 15$ two-particle states, 
which we list with the total $M$:


\begin{quote}
\begin{tabular}{cccccc}
\hline
\multicolumn{1}{c}{ Occupied } & \multicolumn{1}{c}{ $M$ } & \multicolumn{1}{c}{ Occupied } & \multicolumn{1}{c}{ $M$ } & \multicolumn{1}{c}{ Occupied } & \multicolumn{1}{c}{ $M$ } \\
\hline
1,2      & -4  & 2,3      & -2  & 3,5      & 1   \\
1,3      & -3  & 2,4      & -1  & 3,6      & 2   \\
1,4      & -2  & 2,5      & 0   & 4,5      & 2   \\
1,5      & -1  & 2,6      & 1   & 4,6      & 3   \\
1,6      & 0   & 3,4      & 0   & 5,6      & 4   \\
\hline
\end{tabular}
\end{quote}

\noindent
There are 3 states with $M= 0$, 2 with $M = 1$, and so on.
% --- end paragraph admon ---



\subsection*{Shell-model project}

% --- begin paragraph admon ---
\paragraph{}

The first step  is to construct the $M$-scheme basis of Slater determinants.
Here $M$-scheme means the total $J_z$ of the many-body states is fixed.

The steps could be:

\begin{itemize}
\item Read in a user-supplied file of single-particle states (examples can be given) or just code these internally;

\item Ask for the total $M$ of the system and the number of particles $N$;

\item Construct all the $N$-particle states with given $M$.  You will validate the code by  comparing both the number of states and specific states.
\end{itemize}

\noindent
% --- end paragraph admon ---



\subsection*{Shell-model project}

% --- begin paragraph admon ---
\paragraph{}
The format of a possible input  file could be

\begin{quote}
\begin{tabular}{ccccc}
\hline
\multicolumn{1}{c}{ Index } & \multicolumn{1}{c}{ $n$ } & \multicolumn{1}{c}{ $l$ } & \multicolumn{1}{c}{ $2j$ } & \multicolumn{1}{c}{ $2m_j$ } \\
\hline
1     & 1   & 0   & 1    & -1     \\
2     & 1   & 0   & 1    & 1      \\
3     & 0   & 2   & 3    & -3     \\
4     & 0   & 2   & 3    & -1     \\
5     & 0   & 2   & 3    & 1      \\
6     & 0   & 2   & 3    & 3      \\
7     & 0   & 2   & 5    & -5     \\
8     & 0   & 2   & 5    & -3     \\
9     & 0   & 2   & 5    & -1     \\
10    & 0   & 2   & 5    & 1      \\
11    & 0   & 2   & 5    & 3      \\
12    & 0   & 2   & 5    & 5      \\
\hline
\end{tabular}
\end{quote}

\noindent
This represents the $1s_{1/2}0d_{3/2}0d_{5/2}$ valence space, or just the $sd$-space.  There are 
twelve single-particle states, labeled by an overall index, and which have associated quantum 
numbers the number of radial nodes, the orbital angular momentum $l$, and the 
angular momentum $j$ and third component $j_z$.  To keep everything as integers, we could store $2 \times j$ and 
$2 \times j_z$.
% --- end paragraph admon ---



\subsection*{Shell-model project}

% --- begin paragraph admon ---
\paragraph{}
To read in the single-particle states you need to:
\begin{itemize}
\item Open the file 
\begin{itemize}

 \item Read the number of single-particle states (in the above example, 12);  allocate memory; all you need is a single array storing $2\times j_z$ for each state, labeled by the index.

\end{itemize}

\noindent
\item Read in the quantum numbers and store $2 \times j_z$ (and anything else you happen to want).
\end{itemize}

\noindent
% --- end paragraph admon ---



\subsection*{Shell-model project}

% --- begin paragraph admon ---
\paragraph{}

The next step is to read in the number of particles $N$ and the fixed total $M$ (or, actually, $2 \times M$). 
For this project we assume only a single species of particles, say neutrons, although this can be 
relaxed. \textbf{Note}: Although it is often a good idea to try to write a more general code, given the 
short time alloted we would suggest you keep your ambition in check, at least in the initial phases of the 
project.  

You should probably write an error trap to make sure $N$ and $M$ are congruent; if $N$ is even, then 
$2 \times M$ should be even, and if $N$ is odd then $2\times M$ should be odd.
% --- end paragraph admon ---



\subsection*{Shell-model project}

% --- begin paragraph admon ---
\paragraph{}
The final step is to generate the set of $N$-particle Slater determinants with fixed $M$. 
The Slater determinants will be stored in occupation representation.  Although in many codes
this representation is done compactly in bit notation with ones and zeros, but for 
greater transparency and simplicity we will list the occupied single particle states.

 Hence we can 
store the Slater determinant basis states as $sd(i,j)$, that is an 
array of dimension $N_{SD}$, the number of Slater determinants, by $N$, the number of occupied 
state. So if for the 7th Slater determinant the 2nd, 3rd, and 9th single-particle states are occupied, 
then $sd(7,1) = 2$, $sd(7,2) = 3$, and $sd(7,3) = 9$.
% --- end paragraph admon ---



\subsection*{Shell-model project}

% --- begin paragraph admon ---
\paragraph{}

We can construct an occupation representation of Slater determinants by the \emph{odometer}
method.  Consider $N_{sp} = 12$ and $N=4$. 
Start with the first 4 states occupied, that is:

\begin{itemize}
\item $sd(1,:)= 1,2,3,4$ (also written as $|1,2,3,4 \rangle$)
\end{itemize}

\noindent
Now increase the last occupancy recursively:
\begin{itemize}
\item $sd(2,:)= 1,2,3,5$

\item $sd(3,:)= 1,2,3,6$

\item $sd(4,:)= 1,2,3,7$

\item $\ldots$

\item $sd(9,:)= 1,2,3,12$
\end{itemize}

\noindent
Then start over with 
\begin{itemize}
\item $sd(10,:)= 1,2,4,5$
\end{itemize}

\noindent
and again increase the rightmost digit

\begin{itemize}
\item $sd(11,:)= 1,2,4,6$

\item $sd(12,:)= 1,2,4,7$

\item $\ldots$

\item $sd(17,:)= 1,2,4,12$
\end{itemize}

\noindent
% --- end paragraph admon ---



\subsection*{Shell-model project}

% --- begin paragraph admon ---
\paragraph{}
When we restrict ourselves to an $M$-scheme basis, we could choose two paths. 
The first is simplest (and simplest is often best, at 
least in the first draft of a code): generate all possible Slater determinants, 
and then extract from this initial list a list of those Slater determinants with a given 
$M$. (You will need to write a short function or routine that computes $M$ for any 
given occupation.)  

Alternately, and not too difficult, is to run the odometer routine twice: each time, as 
as a Slater determinant is calculated, compute $M$, but do not store the Slater determinants 
except the current one. You can then count up the number of Slater determinants with a 
chosen $M$.  Then allocated storage for the Slater determinants, and run the odometer 
algorithm again, this time storing Slater determinants with the desired $M$ (this can be 
done with a simple logical flag).
% --- end paragraph admon ---



\subsection*{Shell-model project}

% --- begin paragraph admon ---
\paragraph{}

\emph{Some example solutions}:  Let's begin with a simple case, the $0d_{5/2}$ space containing six single-particle states


\begin{quote}
\begin{tabular}{ccccc}
\hline
\multicolumn{1}{c}{ Index } & \multicolumn{1}{c}{ $n$ } & \multicolumn{1}{c}{ $l$ } & \multicolumn{1}{c}{ $j$ } & \multicolumn{1}{c}{ $m_j$ } \\
\hline
1     & 0   & 2   & 5/2 & -5/2  \\
2     & 0   & 2   & 5/2 & -3/2  \\
3     & 0   & 2   & 5/2 & -1/2  \\
4     & 0   & 2   & 5/2 & 1/2   \\
5     & 0   & 2   & 5/2 & 3/2   \\
6     & 0   & 2   & 5/2 & 5/2   \\
\hline
\end{tabular}
\end{quote}

\noindent
For two particles, there are a total of 15 states, which we list here with the total $M$:
\begin{itemize}
\item $\vert 1,2 \rangle$, $M= -4$,  $\vert 1,3 \rangle$, $M= -3$

\item $\vert  1,4 \rangle$, $M= -2$, $\vert 1,5 \rangle$, $M= -1$

\item $\vert 1,5 \rangle$, $M= 0$, $vert 2,3 \rangle$, $M= -2$

\item $\vert 2,4 \rangle$, $M= -1$, $\vert 2,5 \rangle$, $M= 0$

\item $\vert 2,6 \rangle$, $M= 1$, $\vert 3,4 \rangle$, $M= 0$

\item $\vert 3,5 \rangle$, $M= 1$, $\vert 3,6 \rangle$, $M= 2$

\item $\vert 4,5 \rangle$, $M= 2$, $\vert 4,6 \rangle$, $M= 3$

\item $\vert 5,6 \rangle$, $M= 4$
\end{itemize}

\noindent
Of these, there are only 3 states with $M=0$.
% --- end paragraph admon ---



\subsection*{Shell-model project}

% --- begin paragraph admon ---
\paragraph{}
\emph{You should try} by hand to show that in this same single-particle space, that for 
$N=3$ there are 3 states with $M=1/2$ and for $N= 4$ there are also only 3 states with $M=0$. 

\emph{To test your code}, confirm the above. 

Also, 
for the $sd$-space given above, for $N=2$ there are 14 states with $M=0$, for $N=3$ there are 37 
states with $M=1/2$, for $N=4$ there are 81 states with $M=0$.
% --- end paragraph admon ---



\subsection*{Shell-model project}

% --- begin paragraph admon ---
\paragraph{}
For our project, we will only consider the pairing model.
A simple space is the $(1/2)^2$ space with four single-particle states


\begin{quote}
\begin{tabular}{ccccc}
\hline
\multicolumn{1}{c}{ Index } & \multicolumn{1}{c}{ $n$ } & \multicolumn{1}{c}{ $l$ } & \multicolumn{1}{c}{ $s$ } & \multicolumn{1}{c}{ $m_s$ } \\
\hline
1     & 0   & 0   & 1/2 & -1/2  \\
2     & 0   & 0   & 1/2 & 1/2   \\
3     & 1   & 0   & 1/2 & -1/2  \\
4     & 1   & 0   & 1/2 & 1/2   \\
\hline
\end{tabular}
\end{quote}

\noindent
For $N=2$ there are 4 states with $M=0$; show this by hand and confirm your code reproduces it.
% --- end paragraph admon ---



\subsection*{Shell-model project}

% --- begin paragraph admon ---
\paragraph{}
Another, slightly more challenging space is the $(1/2)^4$ space, that is, 
with eight  single-particle states we have


\begin{quote}
\begin{tabular}{ccccc}
\hline
\multicolumn{1}{c}{ Index } & \multicolumn{1}{c}{ $n$ } & \multicolumn{1}{c}{ $l$ } & \multicolumn{1}{c}{ $s$ } & \multicolumn{1}{c}{ $m_s$ } \\
\hline
1     & 0   & 0   & 1/2 & -1/2  \\
2     & 0   & 0   & 1/2 & 1/2   \\
3     & 1   & 0   & 1/2 & -1/2  \\
4     & 1   & 0   & 1/2 & 1/2   \\
5     & 2   & 0   & 1/2 & -1/2  \\
6     & 2   & 0   & 1/2 & 1/2   \\
7     & 3   & 0   & 1/2 & -1/2  \\
8     & 3   & 0   & 1/2 & 1/2   \\
\hline
\end{tabular}
\end{quote}

\noindent
For $N=2$ there are 16 states with $M=0$; for $N=3$ there are 24 states with $M=1/2$, and for 
$N=4$ there are 36 states with $M=0$.
% --- end paragraph admon ---



\subsection*{Shell-model project}

% --- begin paragraph admon ---
\paragraph{}
In the shell-model context we can interpret this as 4 $s_{1/2}$ levels, with $m = \pm 1/2$, we can also think of these are simple four pairs,  $\pm k, k = 1,2,3,4$. Later on we will 
assign single-particle energies,  depending on the radial quantum number $n$, that is, 
$\epsilon_k = |k| \delta$ so that they are equally spaced.
% --- end paragraph admon ---



\subsection*{Shell-model project}

% --- begin paragraph admon ---
\paragraph{}

For application in the pairing model we can go further and consider only states with 
no ``broken pairs,'' that is, if $+k$ is filled (or $m = +1/2$, so is $-k$ ($m=-1/2$). 
If you want, you can write your code to accept only these, and obtain the following 
six states:

\begin{itemize}
\item $|           1,           2 ,          3         ,       4  \rangle , $

\item $|            1      ,     2        ,        5         ,       6 \rangle , $

\item $|            1         ,       2     ,           7         ,       8  \rangle , $

\item $|            3        ,        4      ,          5          ,      6  \rangle , $

\item $|            3        ,        4      ,          7         ,       8  \rangle , $

\item $|            5        ,        6     ,           7     ,           8  \rangle $
\end{itemize}

\noindent
% --- end paragraph admon ---



\subsection*{Shell-model project}

% --- begin paragraph admon ---
\paragraph{Hints for coding.}

\begin{itemize}
\item Write small modules (routines/functions) ; avoid big functions  that do everything. (But not too small.)

\item Use Unit tests! Write lots of error traps, even for things that are `obvious.'

\item Document as you go along. The Unit tests serve as documentation. For each function write a header that includes: 
\begin{enumerate}

\item Main purpose of function and/or unit test

\item names and  brief explanation of input variables, if any 

\item names and brief explanation of output variables, if any

\item functions called by this function

\item called by which functions
\end{enumerate}

\noindent
\end{itemize}

\noindent
% --- end paragraph admon ---



\subsection*{Shell-model project}

% --- begin paragraph admon ---
\paragraph{}

Hints for coding

\begin{itemize}
\item Unit tests will save time. Use also IDEs for debugging. If you insist on brute force debugging, print out intermediate values. It's almost impossible to debug a  code by looking at it--the code will almost always win a `staring contest.'

\item Validate code with SIMPLE CASES. Validate early and often.  Unit tests!! 
\end{itemize}

\noindent
The number one mistake is using a too complex a system to test. For example ,
if you are computing particles in a potential in a box, try removing the potential--you should get 
particles in a box. And start with one particle, then two, then three... Don't start with 
eight particles.
% --- end paragraph admon ---



\subsection*{Shell-model project}

% --- begin paragraph admon ---
\paragraph{}

Our recommended occupation representation, e.g.~$| 1,2,4,8 \rangle$, is 
easy to code, but numerically inefficient when one has hundreds of 
millions of Slater determinants.

In state-of-the-art shell-model codes, one generally uses bit 
representation, i.e.~$|1101000100... \rangle$ where one stores 
the Slater determinant as a single (or a small number of) integer.

This is much more compact, but more intricate to code with considerable 
more overhead. There exist 
bit-manipulation functions. We will discuss these in more detail at the beginning of the third week.
% --- end paragraph admon ---



\subsection*{Example case: pairing Hamiltonian}

% --- begin paragraph admon ---
\paragraph{}

We consider a space with $2\Omega$ single-particle states, with each 
state labeled by 
$k = 1, 2, 3, \Omega$ and $m = \pm 1/2$. The convention is that 
the state with $k>0$ has $m = + 1/2$ while $-k$ has $m = -1/2$.

The Hamiltonian we consider is 
\[
\hat{H} = -G \hat{P}_+ \hat{P}_-,
\]
where
\[
\hat{P}_+ = \sum_{k > 0} \hat{a}^\dagger_k \hat{a}^\dagger_{-{k}}.
\]
and $\hat{P}_- = ( \hat{P}_+)^\dagger$.

This problem can be solved using what is called the quasi-spin formalism to obtain the 
exact results. Thereafter we will try again using the explicit Slater determinant formalism.
% --- end paragraph admon ---



\subsection*{Example case: pairing Hamiltonian}

% --- begin paragraph admon ---
\paragraph{}

One can show (and this is part of the project) that
\[
\left [ \hat{P}_+, \hat{P}_- \right ] = \sum_{k> 0} \left( \hat{a}^\dagger_k \hat{a}_k 
+ \hat{a}^\dagger_{-{k}} \hat{a}_{-{k}} - 1 \right) = \hat{N} - \Omega.
\]
Now define 
\[
\hat{P}_z = \frac{1}{2} ( \hat{N} -\Omega).
\]
Finally you can show
\[
\left [ \hat{P}_z , \hat{P}_\pm \right ] = \pm \hat{P}_\pm.
\]
This means the operators $\hat{P}_\pm, \hat{P}_z$ form a so-called  $SU(2)$ algebra, and we can 
use all our insights about angular momentum, even though there is no actual 
angular momentum involved.

So we rewrite the Hamiltonian to make this explicit:
\[
\hat{H} = -G \hat{P}_+ \hat{P}_- 
= -G \left( \hat{P}^2 - \hat{P}_z^2 + \hat{P}_z\right)
\]
% --- end paragraph admon ---



\subsection*{Example case: pairing Hamiltonian}

% --- begin paragraph admon ---
\paragraph{}

Because of the SU(2) algebra, we know that the eigenvalues of 
$\hat{P}^2$ must be of the form $p(p+1)$, with $p$ either integer or half-integer, and the eigenvalues of $\hat{P}_z$ 
are $m_p$ with $p \geq | m_p|$, with $m_p$ also integer or half-integer. 

But because $\hat{P}_z = (1/2)(\hat{N}-\Omega)$, we know that for $N$ particles 
the value $m_p = (N-\Omega)/2$. Furthermore, the values of $m_p$ range from 
$-\Omega/2$ (for $N=0$) to $+\Omega/2$ (for $N=2\Omega$, with all states filled). 

We deduce the maximal $p = \Omega/2$ and for a given $n$ the 
values range of $p$ range from $|N-\Omega|/2$ to $\Omega/2$ in steps of 1 
(for an even number of particles) 

Following Racah we introduce the notation
$p = (\Omega - v)/2$
where $v = 0, 2, 4,..., \Omega - |N-\Omega|$ 
With this it is easy to deduce that the eigenvalues of the pairing Hamiltonian are
\[
-G(N-v)(2\Omega +2-N-v)/4
\]
This also works for $N$ odd, with $v= 1,3,5, \dots$.
% --- end paragraph admon ---



\subsection*{Example case: pairing Hamiltonian}

% --- begin paragraph admon ---
\paragraph{}

Let's take a specific example: $\Omega = 3$ so there are 6 single-particle states, 
and $N = 3$, with $v= 1,3$. Therefore there are two distinct eigenvalues, 
\[
E = -2G, 0
\]
Now let's work this out explicitly. The single particle degrees of freedom are defined as


\begin{quote}
\begin{tabular}{ccc}
\hline
\multicolumn{1}{c}{ Index } & \multicolumn{1}{c}{ $k$ } & \multicolumn{1}{c}{ $m$ } \\
\hline
1     & 1   & -1/2 \\
2     & -1  & 1/2  \\
3     & 2   & -1/2 \\
4     & -2  & 1/2  \\
5     & 3   & -1/2 \\
6     & -3  & 1/2  \\
\hline
\end{tabular}
\end{quote}

\noindent
 There are  $\left( \begin{array}{c}6 \\ 3 \end{array} \right) = 20$ three-particle states, but there 
are 9 states with $M = +1/2$, namely
$| 1,2,3 \rangle, |1,2,5\rangle, | 1,4,6 \rangle, | 2,3,4 \rangle, |2,3,6 \rangle, | 2,4,5 \rangle, | 2, 5, 6 \rangle, |3,4,6 \rangle, | 4,5,6 \rangle$.
% --- end paragraph admon ---



\subsection*{Example case: pairing Hamiltonian}

% --- begin paragraph admon ---
\paragraph{}

In this basis, the operator 
\[
\hat{P}_+
= \hat{a}^\dagger_1 \hat{a}^\dagger_2 + \hat{a}^\dagger_3 \hat{a}^\dagger_4 +
\hat{a}^\dagger_5 \hat{a}^\dagger_6 
\]
From this we can determine that 
\[
\hat{P}_- | 1, 4, 6 \rangle = \hat{P}_- | 2, 3, 6 \rangle
= \hat{P}_- | 2, 4, 5 \rangle = 0
\]
so those states all have eigenvalue 0.
% --- end paragraph admon ---



\subsection*{Example case: pairing Hamiltonian}

% --- begin paragraph admon ---
\paragraph{}
Now for further example, 
\[
\hat{P}_- | 1,2,3 \rangle = | 3 \rangle
\]
so
\[
\hat{P}_+ \hat{P}_- | 1,2,3\rangle = | 1,2,3\rangle+ | 3,4,3\rangle + | 5,6,3\rangle
\]
The second term vanishes because state 3 is occupied twice, and reordering the last 
term we
get
\[
\hat{P}_+ \hat{P}_- | 1,2,3\rangle = | 1,2,3\rangle+ |3, 5,6\rangle
\]
without picking up a phase.
% --- end paragraph admon ---



\subsection*{Example case: pairing Hamiltonian}

% --- begin paragraph admon ---
\paragraph{}

Continuing in this fashion, with the previous ordering of the many-body states
(  $| 1,2,3 \rangle, |1,2,5\rangle, | 1,4,6 \rangle, | 2,3,4 \rangle, |2,3,6 \rangle, | 2,4,5 \rangle, | 2, 5, 6 \rangle, |3,4,6 \rangle, | 4,5,6 \rangle$) the 
Hamiltonian matrix of this system is 
\[
H = -G\left( 
\begin{array}{ccccccccc}
1 & 0 & 0 & 0 & 0 & 0 & 0 & 0 & 1  \\
0 & 1 & 0 & 0 & 0 & 0 & 0 & 1 & 0  \\
0 & 0 & 0 & 0 & 0 & 0 & 0 & 0 & 0  \\
0 & 0 & 0 & 1 & 0 & 0 & 1 & 0 & 0  \\
0 & 0 & 0 & 0 & 0 & 0 & 0 & 0 & 0  \\
0 & 0 & 0 & 0 & 0 & 0 & 0 & 0 & 0  \\
0 & 0 & 0 & 1 & 0 & 0 & 1 & 0 & 0  \\
0 & 0 & 0 & 0 & 0 & 0 & 0 & 0 & 0  \\
0 & 1 & 0 & 0 & 0 & 0 & 0 & 1 & 0  \\
1 & 0 & 0 & 0 & 0 & 0 & 0 & 0 & 1  
\end{array} \right )
\] 
This is useful for our project.  One can by hand confirm 
that there are 3 eigenvalues $-2G$ and 6 with value zero.
% --- end paragraph admon ---



\subsection*{Example case: pairing Hamiltonian}

% --- begin paragraph admon ---
\paragraph{}

Another example
Using the $(1/2)^4$ single-particle space, resulting in eight single-particle states


\begin{quote}
\begin{tabular}{ccccc}
\hline
\multicolumn{1}{c}{ Index } & \multicolumn{1}{c}{ $n$ } & \multicolumn{1}{c}{ $l$ } & \multicolumn{1}{c}{ $s$ } & \multicolumn{1}{c}{ $m_s$ } \\
\hline
1     & 0   & 0   & 1/2 & -1/2  \\
2     & 0   & 0   & 1/2 & 1/2   \\
3     & 1   & 0   & 1/2 & -1/2  \\
4     & 1   & 0   & 1/2 & 1/2   \\
5     & 2   & 0   & 1/2 & -1/2  \\
6     & 2   & 0   & 1/2 & 1/2   \\
7     & 3   & 0   & 1/2 & -1/2  \\
8     & 3   & 0   & 1/2 & 1/2   \\
\hline
\end{tabular}
\end{quote}

\noindent
and then taking only 4-particle, $M=0$ states that have no `broken pairs', there are six basis Slater 
determinants:

\begin{itemize}
\item $|           1,           2 ,          3         ,       4  \rangle , $

\item $|            1      ,     2        ,        5         ,       6 \rangle , $

\item $|            1         ,       2     ,           7         ,       8  \rangle , $

\item $|            3        ,        4      ,          5          ,      6  \rangle , $

\item $|            3        ,        4      ,          7         ,       8  \rangle , $

\item $|            5        ,        6     ,           7     ,           8  \rangle $
\end{itemize}

\noindent
% --- end paragraph admon ---



\subsection*{Example case: pairing Hamiltonian}

% --- begin paragraph admon ---
\paragraph{}

Now we take the following Hamiltonian
\[
\hat{H} = \sum_n n \delta \hat{N}_n  - G \hat{P}^\dagger \hat{P}
\]
where 
\[
\hat{N}_n = \hat{a}^\dagger_{n, m=+1/2} \hat{a}_{n, m=+1/2} +
\hat{a}^\dagger_{n, m=-1/2} \hat{a}_{n, m=-1/2}
\]
and
\[
\hat{P}^\dagger = \sum_{n} \hat{a}^\dagger_{n, m=+1/2} \hat{a}^\dagger_{n, m=-1/2} 
\]
We can write down the $ 6 \times 6$  Hamiltonian in the basis from the prior slide:
\[
H = \left ( 
\begin{array}{cccccc}
2\delta -2G & -G & -G & -G & -G & 0 \\
 -G & 4\delta -2G & -G & -G & -0 & -G \\
-G & -G & 6\delta -2G & 0 & -G & -G \\
 -G & -G & 0 & 6\delta-2G & -G & -G \\
 -G & 0 & -G & -G & 8\delta-2G & -G \\
0 & -G & -G & -G & -G & 10\delta -2G 
\end{array} \right )
\]
(You should check by hand that this is correct.) 

For $\delta = 0$ we have the closed form solution of  the g.s. energy given by $-6G$.
% --- end paragraph admon ---



\subsection*{Building a Hamiltonian matrix}

% --- begin paragraph admon ---
\paragraph{}
The goal is to compute the matrix elements of the Hamiltonian, specifically
matrix elements between many-body states (Slater determinants) of two-body
operators
\[
\sum_{p < q, r < s}V_{pqr} \hat{a}^\dagger_p \hat{a}^\dagger_q\hat{a}_s \hat{a}_r
\]
In particular we will need to compute
\[
\langle \beta | \hat{a}^\dagger_p \hat{a}^\dagger_q\hat{a}_s \hat{a}_r |\alpha \rangle
\]
where $\alpha, \beta$ are indices labeling Slater determinants and $p,q,r,s$ label
single-particle states.
% --- end paragraph admon ---



\subsection*{Building a Hamiltonian matrix}

% --- begin paragraph admon ---
\paragraph{}
Note: there are other, more efficient ways to do this than the method we describe, 
but you will
be able to produce a working code quickly.

As we coded in the first step,
a Slater determinant $| \alpha \rangle$ with index $\alpha$ is a
list of $N$ occupied single-particle states $i_1 < i_2 < i_3 \ldots i_N$.

Furthermore, for the two-body matrix elements $V_{pqrs}$ we normally assume
$p < q$ and $r < s$. For our specific project, the interaction is much simpler and you can use this to simplify considerably the setup of a shell-model code for project 2.

What follows here is a more general, but still brute force, approach.
% --- end paragraph admon ---



\subsection*{Building a Hamiltonian matrix}

% --- begin paragraph admon ---
\paragraph{}
Write a function that:
\begin{enumerate}
\item Has as input the single-particle indices $p,q,r,s$ for the two-body operator and the index $\alpha$ for the ket Slater determinant;

\item Returns the index $\beta$ of the unique (if any) Slater determinant such that
\end{enumerate}

\noindent
\[
| \beta \rangle = \pm \hat{a}^\dagger_p \hat{a}^\dagger_q\hat{a}_s \hat{a}_r |\alpha \rangle
\]
as well as the phase

This is equivalent to computing
\[
\langle \beta | \hat{a}^\dagger_p \hat{a}^\dagger_q\hat{a}_s \hat{a}_r |\alpha \rangle
\]
% --- end paragraph admon ---



\subsection*{Building a Hamiltonian matrix, first step}

% --- begin paragraph admon ---
\paragraph{}
The first step can take as input an initial Slater determinant
(whose position in the list of basis Slater determinants is $\alpha$) written as an
ordered listed of occupied single-particle states, e.g.~$1,2,5,8$, and the
indices $p,q,r,s$ from the two-body operator. 

It will return another final Slater determinant if the single-particle states $r$ and $s$ are occupied, else it will return an 
empty Slater determinant
(all zeroes). 

If $r$ and $s$ are in the list of occupied single particle states, then
replace the initial single-particle states $ij$ as $i \rightarrow r$ and $j \rightarrow r$.
% --- end paragraph admon ---



\subsection*{Building a Hamiltonian matrix, second step}

% --- begin paragraph admon ---
\paragraph{}
The second step will take the final Slater determinant 
from the first step (if not empty),
and then order by pairwise permutations (i.e., if the Slater determinant is
$i_1, i_2, i_3, \ldots$, then if $i_n > i_{n+1}$, interchange 
$i_n \leftrightarrow i_{n+1}$.
% --- end paragraph admon ---



\subsection*{Building a Hamiltonian matrix}

% --- begin paragraph admon ---
\paragraph{}

It will also output a phase.  If any two single-particle occupancies are repeated, 
the phase is
0.  Otherwise it is +1 for an even permutation and -1 for an odd permutation to 
bring the final
Slater determinant into ascending order, $j_1 < j_2 < j_3 \ldots$.
% --- end paragraph admon ---



\subsection*{Building a Hamiltonian matrix}

% --- begin paragraph admon ---
\paragraph{}
\textbf{Example}: Suppose in the $sd$ single-particle space that the initial 
Slater determinant
is $1,3,9,12$. If $p,q,r,s = 2,8,1,12$, then after the first step the final Slater determinant
is $2,3,9,8$.  The second step will return $2,3,8,9$ and a phase of -1, 
because an odd number  of interchanges is required.
% --- end paragraph admon ---



\subsection*{Building a Hamiltonian matrix}

% --- begin paragraph admon ---
\paragraph{}

\textbf{Example}: Suppose in the $sd$ single-particle space that the initial 
Slater determinant
is $1,3,9,12$. If $p,q,r,s = 3,8,1,12$, then after the first step the 
final  Slater determinant
is $3,3,9,8$, but after the second step the phase is 0 
because the single-particle state 3 is
occupied twice.

Lastly, the final step  takes the ordered final Slater determinant and 
we search through the basis list to
determine its index in the many-body basis, that is, $\beta$.
% --- end paragraph admon ---



\subsection*{Building a Hamiltonian matrix}

% --- begin paragraph admon ---
\paragraph{}

The Hamiltonian is then stored as an $N_{SD} \times N_{SD}$ array of real numbers, which
can be allocated once you have created the many-body basis and know $N_{SD}$.
% --- end paragraph admon ---



\subsection*{Building a Hamiltonian matrix}

% --- begin paragraph admon ---
\paragraph{}

\begin{enumerate}
\item Initialize $H(\alpha,\beta)=0.0$

\item Set up an outer loop over $\beta$

\item Loop over $\alpha = 1, NSD$

\item For each $\alpha$, loop over $a=1,ntbme$  and fetch $V(a)$ and the single-particle indices $p,q,r,s$ 

\item If $V(a) = 0$ skip.  Otherwise, apply $\hat{a}^\dagger_p\hat{a}^\dagger_q \hat{a}_s \hat{a}_r$ to the Slater determinant labeled by $\alpha$.

\item Find, if any, the label $\beta$ of the resulting Slater determinant and the phase (which is 0, +1, -1).

\item If phase $\neq 0$, then update $H(\alpha,\beta)$  as $H(\alpha,\beta) + phase*V(a)$. The sum is important because multiple operators might contribute to the same matrix element.

\item Continue loop over $a$

\item Continue loop over $\alpha$.

\item End the outer loop over $\beta$.
\end{enumerate}

\noindent
You should force the resulting matrix $H$ to be symmetric. To do this, when
updating $H(\alpha,\beta)$, if $\alpha \neq \beta$, also update $H(\beta,\alpha)$.
% --- end paragraph admon ---



\subsection*{Building a Hamiltonian matrix}

% --- begin paragraph admon ---
\paragraph{}

You will also need to include the single-particle energies. This is easy: they only
contribute to diagonal matrix elements, that is, $H(\alpha,\alpha)$.  
Simply find the occupied single-particle states $i$ and add the corresponding $\epsilon(i)$.
% --- end paragraph admon ---



\subsection*{Hamiltonian matrix without the bit representation}

% --- begin paragraph admon ---
\paragraph{}

Consider the many-body state $\Psi_{\lambda}$ expressed as linear combinations of
Slater determinants ($SD$) of orthonormal single-particle states $\phi({\bf r})$:
\begin{equation}
\Psi_{\lambda} = \sum_i C_{\lambda i} SD_i
\end{equation}
Using the Slater-Condon rules the matrix elements of any one-body
($\cal{O}_1$) or two-body ($\cal{O}_2$) operator expressed in the
determinant space have simple expressions involving one- and two-fermion
integrals in our given single-particle basis.
The diagonal elements are given by:
\begin{eqnarray}
  \langle SD | \cal{O}_1 | SD \rangle & = & \sum_{i \in SD} \langle \phi_i | \cal{O}_1 | \phi_i \rangle \\
  \langle SD | \cal{O}_2 | SD \rangle & = & \frac{1}{2} \sum_{(i,j) \in SD}  
      \langle \phi_i \phi_j | \cal{O}_2 | \phi_i \phi_j \rangle - \nonumber \\
 & & 
      \langle \phi_i \phi_j | \cal{O}_2 | \phi_j \phi_i \rangle \nonumber 
\end{eqnarray}
% --- end paragraph admon ---



\subsection*{Hamiltonian matrix without the bit representation, one and two-body operators}

% --- begin paragraph admon ---
\paragraph{}

For two determinants which differ only by the substitution of single-particle states $i$ with
a single-particle state $j$:
\begin{eqnarray}
  \langle SD | \cal{O}_1 | SD_i^j \rangle & = & \langle \phi_i | \cal{O}_1 | \phi_j \rangle \\
  \langle SD | \cal{O}_2 | SD_i^j \rangle & = & \sum_{k \in SD} 
      \langle \phi_i \phi_k | \cal{O}_2 | \phi_j \phi_k \rangle - 
      \langle \phi_i \phi_k | \cal{O}_2 | \phi_k \phi_j \rangle \nonumber
\end{eqnarray}
For two determinants which differ by two single-particle states
\begin{eqnarray}
  \langle SD | \cal{O}_1 | SD_{ik}^{jl} \rangle & = & 0 \\
  \langle SD | \cal{O}_2 | SD_{ik}^{jl} \rangle & = & 
      \langle \phi_i \phi_k | \cal{O}_2 | \phi_j \phi_l \rangle -
      \langle \phi_i \phi_k | \cal{O}_2 | \phi_l \phi_j \rangle \nonumber 
\end{eqnarray}
All other matrix elements involving determinants with more than two
substitutions are zero.
% --- end paragraph admon ---



\subsection*{Strategies for setting up an algorithm}

% --- begin paragraph admon ---
\paragraph{}

An efficient implementation of these rules requires

\begin{itemize}
\item to find the number of single-particle state substitutions between two determinants

\item to find which single-particle states are involved in the substitution

\item to compute the phase factor if a reordering of the single-particle states has occured
\end{itemize}

\noindent
We can solve this problem using our odometric approach or alternatively using a bit representation as discussed below and in more detail in 

\begin{itemize}
\item \href{{https://github.com/scemama/slater_condon}}{Scemama and Gimer's article (Fortran codes)}

\item \href{{https://arxiv.org/abs/0810.2644}}{Simen Kvaal's article on how to build an FCI code (C++ code)}
\end{itemize}

\noindent
We recommend in particular the article by Simen Kvaal. It contains nice general classes for creation and annihilation operators as well as the calculation of the phase (see below).
% --- end paragraph admon ---



\subsection*{Computing expectation values and transitions in the shell-model}

% --- begin paragraph admon ---
\paragraph{}
When we diagonalize the Hamiltonian matrix, the eigenvectors are the coefficients $C_{\lambda i}$ used 
to express the many-body state $\Psi_{\lambda}$ in terms of  a linear combinations of
Slater determinants ($SD$) of orthonormal single-particle states $\phi({\bf r})$.

With these eigenvectors we can compute say the transition likelyhood of a one-body operator as
\[
\langle \Psi_{\lambda} \vert \cal{O}_1 \vert \Psi_{\sigma} \rangle  = 
\sum_{ij}C_{\lambda i}^*C_{\sigma j}  \langle SD_i | \cal{O}_1 | SD_j \rangle .
\]
Writing the one-body operator in second quantization as 
\[
\cal{O}_1  = \sum_{pq} \langle p \vert \cal{o}_1 \vert q\rangle a_p^{\dagger} a_q, 
\]
we have
\[
\langle \Psi_{\lambda} \vert \cal{O}_1 \vert \Psi_{\sigma} \rangle  = 
\sum_{pq}\langle p \vert \cal{o}_1 \vert q\rangle \sum_{ij}C_{\lambda i}^*C_{\sigma j}  \langle SD_i |a_p^{\dagger} a_q | SD_j \rangle .
\]
% --- end paragraph admon ---



\subsection*{Computing expectation values and transitions in the shell-model and spectroscopic factors}

% --- begin paragraph admon ---
\paragraph{}
The terms we need to evalute then are just the elements 
\[
\langle SD_i |a_p^{\dagger} a_q | SD_j \rangle, 
\]
which can be rewritten in terms of spectroscopic factors by inserting a complete set of Slater determinats as
\[
\langle SD_i |a_p^{\dagger} a_q | SD_j \rangle = \sum_{l}\langle SD_i \vert a_p^{\dagger}\vert SD_l\rangle \langle SD_l \vert  a_q \vert SD_j \rangle,
\]
where $\langle SD_l\vert a_q(a_p^{\dagger})\vert SD_j\rangle$ are the spectroscopic factors. These can be easily evaluated in $m$-scheme. Using the Wigner-Eckart theorem we can transform these to a $J$-coupled scheme through so-called reduced matrix elements.
% --- end paragraph admon ---



\subsection*{Operators in second quantization}

% --- begin paragraph admon ---
\paragraph{}
In the build-up of a shell-model or FCI code that is meant to tackle large dimensionalities
we need to deal with the action of the Hamiltonian $\hat{H}$ on a
Slater determinant represented in second quantization as
\[
 |\alpha_1\dots \alpha_n\rangle = a_{\alpha_1}^{\dagger} a_{\alpha_2}^{\dagger} \dots a_{\alpha_n}^{\dagger} |0\rangle.
\]
The time consuming part stems from the action of the Hamiltonian
on the above determinant,
\[
\left(\sum_{\alpha\beta} \langle \alpha|t+u|\beta\rangle a_\alpha^{\dagger} a_\beta + \frac{1}{4} \sum_{\alpha\beta\gamma\delta}
                \langle \alpha \beta|\hat{v}|\gamma \delta\rangle a_\alpha^{\dagger} a_\beta^{\dagger} a_\delta a_\gamma\right)a_{\alpha_1}^{\dagger} a_{\alpha_2}^{\dagger} \dots a_{\alpha_n}^{\dagger} |0\rangle.
\]
A practically useful way to implement this action is to encode a Slater determinant as a bit pattern.
% --- end paragraph admon ---



\subsection*{Operators in second quantization}

% --- begin paragraph admon ---
\paragraph{}
Assume that we have at our disposal $n$ different single-particle states
$\alpha_0,\alpha_2,\dots,\alpha_{n-1}$ and that we can distribute  among these states $N\le n$ particles.

A Slater  determinant can then be coded as an integer of $n$ bits. As an example, if we have $n=16$ single-particle states
$\alpha_0,\alpha_1,\dots,\alpha_{15}$ and $N=4$ fermions occupying the states $\alpha_3$, $\alpha_6$, $\alpha_{10}$ and $\alpha_{13}$
we could write this Slater determinant as  
\[
\Phi_{\Lambda} = a_{\alpha_3}^{\dagger} a_{\alpha_6}^{\dagger} a_{\alpha_{10}}^{\dagger} a_{\alpha_{13}}^{\dagger} |0\rangle.
\]
The unoccupied single-particle states have bit value $0$ while the occupied ones are represented by bit state $1$. 
In the binary notation we would write this   16 bits long integer as
\[
\begin{array}{cccccccccccccccc}
{\alpha_0}&{\alpha_1}&{\alpha_2}&{\alpha_3}&{\alpha_4}&{\alpha_5}&{\alpha_6}&{\alpha_7} & {\alpha_8} &{\alpha_9} & {\alpha_{10}} &{\alpha_{11}} &{\alpha_{12}} &{\alpha_{13}} &{\alpha_{14}} & {\alpha_{15}} \\
{0} & {0} &{0} &{1} &{0} &{0} &{1} &{0} &{0} &{0} &{1} &{0} &{0} &{1} &{0} & {0} \\
\end{array}
\]
which translates into the decimal number
\[
2^3+2^6+2^{10}+2^{13}=9288.
\]
We can thus encode a Slater determinant as a bit pattern.
% --- end paragraph admon ---



\subsection*{Operators in second quantization}

% --- begin paragraph admon ---
\paragraph{}
With $N$ particles that can be distributed over $n$ single-particle states, the total number of Slater determinats (and defining thereby the dimensionality of the system) is
\[
\mathrm{dim}(\mathcal{H}) = \left(\begin{array}{c} n \\N\end{array}\right).
\]
The total number of bit patterns is $2^n$.
% --- end paragraph admon ---



\subsection*{Operators in second quantization}

% --- begin paragraph admon ---
\paragraph{}
We assume again that we have at our disposal $n$ different single-particle orbits
$\alpha_0,\alpha_2,\dots,\alpha_{n-1}$ and that we can distribute  among these orbits $N\le n$ particles.
The ordering among these states is important as it defines the order of the creation operators.
We will write the determinant 
\[
\Phi_{\Lambda} = a_{\alpha_3}^{\dagger} a_{\alpha_6}^{\dagger} a_{\alpha_{10}}^{\dagger} a_{\alpha_{13}}^{\dagger} |0\rangle,
\]
in a more compact way as 
\[
\Phi_{3,6,10,13} = |0001001000100100\rangle.
\]
The action of a creation operator is thus
\[
a^{\dagger}_{\alpha_4}\Phi_{3,6,10,13} = a^{\dagger}_{\alpha_4}|0001001000100100\rangle=a^{\dagger}_{\alpha_4}a_{\alpha_3}^{\dagger} a_{\alpha_6}^{\dagger} a_{\alpha_{10}}^{\dagger} a_{\alpha_{13}}^{\dagger} |0\rangle,
\]
which becomes
\[
-a_{\alpha_3}^{\dagger} a^{\dagger}_{\alpha_4} a_{\alpha_6}^{\dagger} a_{\alpha_{10}}^{\dagger} a_{\alpha_{13}}^{\dagger} |0\rangle=-|0001101000100100\rangle.
\]
% --- end paragraph admon ---



\subsection*{Operators in second quantization}

% --- begin paragraph admon ---
\paragraph{}
Similarly
\[
a^{\dagger}_{\alpha_6}\Phi_{3,6,10,13} = a^{\dagger}_{\alpha_6}|0001001000100100\rangle=a^{\dagger}_{\alpha_6}a_{\alpha_3}^{\dagger} a_{\alpha_6}^{\dagger} a_{\alpha_{10}}^{\dagger} a_{\alpha_{13}}^{\dagger} |0\rangle,
\]
which becomes
\[
-a^{\dagger}_{\alpha_4} (a_{\alpha_6}^{\dagger})^ 2 a_{\alpha_{10}}^{\dagger} a_{\alpha_{13}}^{\dagger} |0\rangle=0!
\]
This gives a simple recipe:  
\begin{itemize}
\item If one of the bits $b_j$ is $1$ and we act with a creation operator on this bit, we return a null vector

\item If $b_j=0$, we set it to $1$ and return a sign factor $(-1)^l$, where $l$ is the number of bits set before bit $j$.
\end{itemize}

\noindent
% --- end paragraph admon ---



\subsection*{Operators in second quantization}

% --- begin paragraph admon ---
\paragraph{}
Consider the action of $a^{\dagger}_{\alpha_2}$ on various slater determinants:
\[
\begin{array}{ccc}
a^{\dagger}_{\alpha_2}\Phi_{00111}& = a^{\dagger}_{\alpha_2}|00111\rangle&=0\times |00111\rangle\\
a^{\dagger}_{\alpha_2}\Phi_{01011}& = a^{\dagger}_{\alpha_2}|01011\rangle&=(-1)\times |01111\rangle\\
a^{\dagger}_{\alpha_2}\Phi_{01101}& = a^{\dagger}_{\alpha_2}|01101\rangle&=0\times |01101\rangle\\
a^{\dagger}_{\alpha_2}\Phi_{01110}& = a^{\dagger}_{\alpha_2}|01110\rangle&=0\times |01110\rangle\\
a^{\dagger}_{\alpha_2}\Phi_{10011}& = a^{\dagger}_{\alpha_2}|10011\rangle&=(-1)\times |10111\rangle\\
a^{\dagger}_{\alpha_2}\Phi_{10101}& = a^{\dagger}_{\alpha_2}|10101\rangle&=0\times |10101\rangle\\
a^{\dagger}_{\alpha_2}\Phi_{10110}& = a^{\dagger}_{\alpha_2}|10110\rangle&=0\times |10110\rangle\\
a^{\dagger}_{\alpha_2}\Phi_{11001}& = a^{\dagger}_{\alpha_2}|11001\rangle&=(+1)\times |11101\rangle\\
a^{\dagger}_{\alpha_2}\Phi_{11010}& = a^{\dagger}_{\alpha_2}|11010\rangle&=(+1)\times |11110\rangle\\
\end{array}
\]
What is the simplest way to obtain the phase when we act with one annihilation(creation) operator
on the given Slater determinant representation?
% --- end paragraph admon ---



\subsection*{Operators in second quantization}

% --- begin paragraph admon ---
\paragraph{}
We have an SD representation
\[
\Phi_{\Lambda} = a_{\alpha_0}^{\dagger} a_{\alpha_3}^{\dagger} a_{\alpha_6}^{\dagger} a_{\alpha_{10}}^{\dagger} a_{\alpha_{13}}^{\dagger} |0\rangle,
\]
in a more compact way as
\[
\Phi_{0,3,6,10,13} = |1001001000100100\rangle.
\]
The action of
\[
a^{\dagger}_{\alpha_4}a_{\alpha_0}\Phi_{0,3,6,10,13} = a^{\dagger}_{\alpha_4}|0001001000100100\rangle=a^{\dagger}_{\alpha_4}a_{\alpha_3}^{\dagger} a_{\alpha_6}^{\dagger} a_{\alpha_{10}}^{\dagger} a_{\alpha_{13}}^{\dagger} |0\rangle,
\]
which becomes
\[
-a_{\alpha_3}^{\dagger} a^{\dagger}_{\alpha_4} a_{\alpha_6}^{\dagger} a_{\alpha_{10}}^{\dagger} a_{\alpha_{13}}^{\dagger} |0\rangle=-|0001101000100100\rangle.
\]
% --- end paragraph admon ---



\subsection*{Operators in second quantization}

% --- begin paragraph admon ---
\paragraph{}
The action
\[
a_{\alpha_0}\Phi_{0,3,6,10,13} = |0001001000100100\rangle,
\]
can be obtained by subtracting the logical sum (AND operation) of $\Phi_{0,3,6,10,13}$ and 
a word which represents only $\alpha_0$, that is
\[
|1000000000000000\rangle,
\] 
from $\Phi_{0,3,6,10,13}= |1001001000100100\rangle$.

This operation gives $|0001001000100100\rangle$. 

Similarly, we can form $a^{\dagger}_{\alpha_4}a_{\alpha_0}\Phi_{0,3,6,10,13}$, say, by adding 
$|0000100000000000\rangle$ to $a_{\alpha_0}\Phi_{0,3,6,10,13}$, first checking that their logical sum
is zero in order to make sure that the state $\alpha_4$ is not already occupied.
% --- end paragraph admon ---



\subsection*{Operators in second quantization}

% --- begin paragraph admon ---
\paragraph{}
It is trickier however to get the phase $(-1)^l$. 
One possibility is as follows
\begin{itemize}
\item Let $S_1$ be a word that represents the 1-bit to be removed and all others set to zero.
\end{itemize}

\noindent
In the previous example $S_1=|1000000000000000\rangle$
\begin{itemize}
\item Define $S_2$ as the similar word that represents the bit to be added, that is in our case
\end{itemize}

\noindent
$S_2=|0000100000000000\rangle$.
\begin{itemize}
\item Compute then $S=S_1-S_2$, which here becomes
\end{itemize}

\noindent
\[
S=|0111000000000000\rangle
\]
\begin{itemize}
\item Perform then the logical AND operation of $S$ with the word containing 
\end{itemize}

\noindent
\[
\Phi_{0,3,6,10,13} = |1001001000100100\rangle,
\]
which results in $|0001000000000000\rangle$. Counting the number of 1-bits gives the phase.  Here you need however an algorithm for bitcounting.
% --- end paragraph admon ---



\subsection*{Bit counting}

% --- begin paragraph admon ---
\paragraph{}

We include here a python program which may aid in this direction. It uses bit manipulation functions from \href{{http://wiki.python.org/moin/BitManipulation}}{\nolinkurl{http://wiki.python.org/moin/BitManipulation}}.














































































































\begin{minted}[fontsize=\fontsize{9pt}{9pt},linenos=false,mathescape,baselinestretch=1.0,fontfamily=tt,xleftmargin=7mm]{python}
import math

"""
A simple Python class for Slater determinant manipulation
Bit-manipulation stolen from:

http://wiki.python.org/moin/BitManipulation
"""

# bitCount() counts the number of bits set (not an optimal function)

def bitCount(int_type):
    """ Count bits set in integer """
    count = 0
    while(int_type):
        int_type &= int_type - 1
        count += 1
    return(count)


# testBit() returns a nonzero result, 2**offset, if the bit at 'offset' is one.

def testBit(int_type, offset):
    mask = 1 << offset
    return(int_type & mask) >> offset

# setBit() returns an integer with the bit at 'offset' set to 1.

def setBit(int_type, offset):
    mask = 1 << offset
    return(int_type | mask)

# clearBit() returns an integer with the bit at 'offset' cleared.

def clearBit(int_type, offset):
    mask = ~(1 << offset)
    return(int_type & mask)

# toggleBit() returns an integer with the bit at 'offset' inverted, 0 -> 1 and 1 -> 0.

def toggleBit(int_type, offset):
    mask = 1 << offset
    return(int_type ^ mask)

# binary string made from number

def bin0(s):
    return str(s) if s<=1 else bin0(s>>1) + str(s&1)

def bin(s, L = 0):
    ss = bin0(s)
    if L > 0:
        return '0'*(L-len(ss)) + ss
    else:
        return ss
    
    

class Slater:
    """ Class for Slater determinants """
    def __init__(self):
        self.word = int(0)

    def create(self, j):
        print "c^+_" + str(j) + " |" + bin(self.word) + ">  = ",
        # Assume bit j is set, then we return zero.
        s = 0
        # Check if bit j is set.
        isset = testBit(self.word, j)
        if isset == 0:
            bits = bitCount(self.word & ((1<<j)-1))
            s = pow(-1, bits)
            self.word = setBit(self.word, j)

        print str(s) + " x |" + bin(self.word) + ">"
        return s
        
    def annihilate(self, j):
        print "c_" + str(j) + " |" + bin(self.word) + ">  = ",
        # Assume bit j is not set, then we return zero.
        s = 0
        # Check if bit j is set.
        isset = testBit(self.word, j)
        if isset == 1:
            bits = bitCount(self.word & ((1<<j)-1))
            s = pow(-1, bits)
            self.word = clearBit(self.word, j)

        print str(s) + " x |" + bin(self.word) + ">"
        return s



# Do some testing:

phi = Slater()
phi.create(0)
phi.create(1)
phi.create(2)
phi.create(3)

print


s = phi.annihilate(2)
s = phi.create(7)
s = phi.annihilate(0)
s = phi.create(200)


\end{minted}
% --- end paragraph admon ---

    

\subsection*{Eigenvalue problems, basic definitions}

% --- begin paragraph admon ---
\paragraph{}
Let us consider the matrix $\mathbf{A}$ of dimension $n$. The eigenvalues of
$\mathbf{A}$ are defined through the matrix equation 
\[
   \mathbf{A}\mathbf{x}^{(\nu)} = \lambda^{(\nu)}\mathbf{x}^{(\nu)},
\]
where $\lambda^{(\nu)}$ are the eigenvalues and $\mathbf{x}^{(\nu)}$ the
corresponding eigenvectors.
Unless otherwise stated, when we use the wording eigenvector we mean the
right eigenvector. The left eigenvalue problem is defined as 
\[
\mathbf{x}^{(\nu)}_L\mathbf{A} = \lambda^{(\nu)}\mathbf{x}^{(\nu)}_L
\]
The above right eigenvector problem is equivalent to a set of $n$ equations with $n$ unknowns
$x_i$.
% --- end paragraph admon ---



\subsection*{Eigenvalue problems, basic definitions}

% --- begin paragraph admon ---
\paragraph{}
The eigenvalue problem can be rewritten as 
\[
   \left( \mathbf{A}-\lambda^{(\nu)} \mathbf{I} \right) \mathbf{x}^{(\nu)} = 0,
\]
with $\mathbf{I}$ being the unity matrix. This equation provides
a solution to the problem if and only if the determinant
is zero, namely
\[
   \left| \mathbf{A}-\lambda^{(\nu)}\mathbf{I}\right| = 0,
\]
which in turn means that the determinant is a polynomial
of degree $n$ in $\lambda$ and in general we will have 
$n$ distinct zeros.
% --- end paragraph admon ---



\subsection*{Eigenvalue problems, basic definitions}

% --- begin paragraph admon ---
\paragraph{}
The eigenvalues of a matrix 
$\mathbf{A}\in {\mathbb{C}}^{n\times n}$
are thus the $n$ roots of its characteristic polynomial 
\[
P(\lambda) = det(\lambda\mathbf{I}-\mathbf{A}),
\]
or 
\[
  P(\lambda)= \prod_{i=1}^{n}\left(\lambda_i-\lambda\right).
\]
The set of these roots is called the spectrum and is denoted as
$\lambda(\mathbf{A})$.
If $\lambda(\mathbf{A})=\left\{\lambda_1,\lambda_2,\dots ,\lambda_n\right\}$ then we have
\[
   det(\mathbf{A})= \lambda_1\lambda_2\dots\lambda_n, 
\]
and if we define the trace of $\mathbf{A}$ as
\[
Tr(\mathbf{A})=\sum_{i=1}^n a_{ii}\]
then
\[
Tr(\mathbf{A})=\lambda_1+\lambda_2+\dots+\lambda_n.
\]
% --- end paragraph admon ---



\subsection*{Abel-Ruffini Impossibility Theorem}

% --- begin paragraph admon ---
\paragraph{}
The \emph{Abel-Ruffini} theorem (also known as Abel's impossibility theorem) 
states that there is no general solution in radicals to polynomial equations of degree five or higher.

The content of this theorem is frequently misunderstood. It does not assert that higher-degree polynomial equations are unsolvable. 
In fact, if the polynomial has real or complex coefficients, and we allow complex solutions, then every polynomial equation has solutions; this is the fundamental theorem of algebra. Although these solutions cannot always be computed exactly with radicals, they can be computed to any desired degree of accuracy using numerical methods such as the Newton-Raphson method or Laguerre method, and in this way they are no different from solutions to polynomial equations of the second, third, or fourth degrees.

The theorem only concerns the form that such a solution must take. The content of the theorem is 
that the solution of a higher-degree equation cannot in all cases be expressed in terms of the polynomial coefficients with a finite number of operations of addition, subtraction, multiplication, division and root extraction. Some polynomials of arbitrary degree, of which the simplest nontrivial example is the monomial equation $ax^n = b$, are always solvable with a radical.
% --- end paragraph admon ---



\subsection*{Abel-Ruffini Impossibility Theorem}

% --- begin paragraph admon ---
\paragraph{}

The \emph{Abel-Ruffini} theorem says that there are some fifth-degree equations whose solution cannot be so expressed. 
The equation $x^5 - x + 1 = 0$ is an example. Some other fifth degree equations can be solved by radicals, 
for example $x^5 - x^4 - x + 1 = 0$. The precise criterion that distinguishes between those equations that can be solved 
by radicals and those that cannot was given by Galois and is now part of Galois theory: 
a polynomial equation can be solved by radicals if and only if its Galois group is a solvable group.

Today, in the modern algebraic context, we say that second, third and fourth degree polynomial 
equations can always be solved by radicals because the symmetric groups $S_2, S_3$ and $S_4$ are solvable groups, 
whereas $S_n$ is not solvable for $n \ge 5$.
% --- end paragraph admon ---



\subsection*{Eigenvalue problems, basic definitions}

% --- begin paragraph admon ---
\paragraph{}
In the present discussion we assume that our matrix is real and symmetric, that is 
$\mathbf{A}\in {\mathbb{R}}^{n\times n}$.
The matrix $\mathbf{A}$ has $n$ eigenvalues
$\lambda_1\dots \lambda_n$ (distinct or not). Let $\mathbf{D}$ be the
diagonal matrix with the eigenvalues on the diagonal
\[
\mathbf{D}=    \left( \begin{array}{ccccccc} \lambda_1 & 0 & 0   & 0    & \dots  &0     & 0 \\
                                0 & \lambda_2 & 0 & 0    & \dots  &0     &0 \\
                                0   & 0 & \lambda_3 & 0  &0       &\dots & 0\\
                                \dots  & \dots & \dots & \dots  &\dots      &\dots & \dots\\
                                0   & \dots & \dots & \dots  &\dots       &\lambda_{n-1} & \\
                                0   & \dots & \dots & \dots  &\dots       &0 & \lambda_n
             \end{array} \right).
\]
If $\mathbf{A}$ is real and symmetric then there exists a real orthogonal matrix $\mathbf{S}$ such that
\[
     \mathbf{S}^T \mathbf{A}\mathbf{S}= \mathrm{diag}(\lambda_1,\lambda_2,\dots ,\lambda_n),
\]
and for $j=1:n$ we have $\mathbf{A}\mathbf{S}(:,j) = \lambda_j \mathbf{S}(:,j)$.
% --- end paragraph admon ---



\subsection*{Eigenvalue problems, basic definitions}

% --- begin paragraph admon ---
\paragraph{}
To obtain the eigenvalues of $\mathbf{A}\in {\mathbb{R}}^{n\times n}$,
the strategy is to
perform a series of similarity transformations on the original
matrix $\mathbf{A}$, in order to reduce it either into a  diagonal form as above
or into a  tridiagonal form. 

We say that a matrix $\mathbf{B}$ is a similarity
transform  of  $\mathbf{A}$ if 
\[
     \mathbf{B}= \mathbf{S}^T \mathbf{A}\mathbf{S}, \hspace{1cm} \mathrm{where} \hspace{1cm}  \mathbf{S}^T\mathbf{S}=\mathbf{S}^{-1}\mathbf{S} =\mathbf{I}.
\]
The importance of a similarity transformation lies in the fact that
the resulting matrix has the same
eigenvalues, but the eigenvectors are in general different.
% --- end paragraph admon ---



\subsection*{Eigenvalue problems, basic definitions}

% --- begin paragraph admon ---
\paragraph{}
To prove this we
start with  the eigenvalue problem and a similarity transformed matrix $\mathbf{B}$.
\[
   \mathbf{A}\mathbf{x}=\lambda\mathbf{x} \hspace{1cm} \mathrm{and}\hspace{1cm} 
    \mathbf{B}= \mathbf{S}^T \mathbf{A}\mathbf{S}.
\]
We multiply the first equation on the left by $\mathbf{S}^T$ and insert
$\mathbf{S}^{T}\mathbf{S} = \mathbf{I}$ between $\mathbf{A}$ and $\mathbf{x}$. Then we get
\begin{equation}
   (\mathbf{S}^T\mathbf{A}\mathbf{S})(\mathbf{S}^T\mathbf{x})=\lambda\mathbf{S}^T\mathbf{x} ,
\end{equation}  
which is the same as 
\[
   \mathbf{B} \left ( \mathbf{S}^T\mathbf{x} \right ) = \lambda \left (\mathbf{S}^T\mathbf{x}\right ).
\]
The variable  $\lambda$ is an eigenvalue of $\mathbf{B}$ as well, but with
eigenvector $\mathbf{S}^T\mathbf{x}$.
% --- end paragraph admon ---



\subsection*{Eigenvalue problems, basic definitions}

% --- begin paragraph admon ---
\paragraph{}
The basic philosophy is to
\begin{itemize}
 \item Either apply subsequent similarity transformations (direct method) so that 
\end{itemize}

\noindent
\begin{equation}
   \mathbf{S}_N^T\dots \mathbf{S}_1^T\mathbf{A}\mathbf{S}_1\dots \mathbf{S}_N=\mathbf{D} ,
\end{equation}
\begin{itemize}
 \item Or apply subsequent similarity transformations so that $\mathbf{A}$ becomes tridiagonal (Householder) or upper/lower triangular (the \emph{QR} method to be discussed later). 

 \item Thereafter, techniques for obtaining eigenvalues from tridiagonal matrices can be used.

 \item Or use so-called power methods

 \item Or use iterative methods (Krylov, Lanczos, Arnoldi). These methods are popular for huge matrix problems.
\end{itemize}

\noindent
% --- end paragraph admon ---



\subsection*{Discussion of  methods for eigenvalues}

% --- begin paragraph admon ---
\paragraph{The general overview.}

One speaks normally of two main approaches to solving the eigenvalue problem.
\begin{itemize}
 \item The first is the formal method, involving determinants and the  characteristic polynomial. This proves how many eigenvalues  there are, and is the way most of you learned about how to solve the eigenvalue problem, but for matrices of dimensions greater than 2 or 3, it is rather impractical.

 \item The other general approach is to use similarity or unitary tranformations  to reduce a matrix to diagonal form. This is normally done in two steps: first reduce to for example a \emph{tridiagonal} form, and then to diagonal form. The main algorithms we will discuss in detail, Jacobi's and  Householder's  (so-called direct method) and Lanczos algorithms (an iterative method), follow this methodology. 
\end{itemize}

\noindent
% --- end paragraph admon ---



\subsection*{Eigenvalues methods}

% --- begin paragraph admon ---
\paragraph{}
Direct or non-iterative methods  require for matrices of dimensionality $n\times n$ typically $O(n^3)$ operations. These methods are normally called standard methods and are used for dimensionalities
$n \sim 10^5$ or smaller. A brief historical overview  


\begin{quote}
\begin{tabular}{ccc}
\hline
\multicolumn{1}{c}{ Year } & \multicolumn{1}{c}{ $n$ } & \multicolumn{1}{c}{  } \\
\hline
1950        & $n=20$       & (Wilkinson)       \\
1965        & $n=200$      & (Forsythe et al.) \\
1980        & $n=2000$     & Linpack           \\
1995        & $n=20000$    & Lapack            \\
This decade & $n\sim 10^5$ & Lapack            \\
\hline
\end{tabular}
\end{quote}

\noindent
shows that in the course of 60 years the dimension that  direct diagonalization methods can handle  has increased by almost a factor of
$10^4$ (note this is for serial versions). However, it pales beside the progress achieved by computer hardware, from flops to petaflops, a factor of almost $10^{15}$. We see clearly played out in history the $O(n^3)$ bottleneck  of direct matrix algorithms.

Sloppily speaking, when  $n\sim 10^4$ is cubed we have $O(10^{12})$ operations, which is smaller than the $10^{15}$ increase in flops.
% --- end paragraph admon ---



\subsection*{Discussion of methods for eigenvalues}

% --- begin paragraph admon ---
\paragraph{}
If the matrix to diagonalize is large and sparse, direct methods simply become impractical, 
also because
many of the direct methods tend to destroy sparsity. As a result large dense matrices may arise during the diagonalization procedure.  The idea behind iterative methods is to project the 
$n-$dimensional problem in smaller spaces, so-called Krylov subspaces. 
Given a matrix $\mathbf{A}$ and a vector $\mathbf{v}$, the associated Krylov sequences of vectors
(and thereby subspaces) 
$\mathbf{v}$, $\mathbf{A}\mathbf{v}$, $\mathbf{A}^2\mathbf{v}$, $\mathbf{A}^3\mathbf{v},\dots$, represent
successively larger Krylov subspaces. 


\begin{quote}
\begin{tabular}{lll}
\hline
\multicolumn{1}{c}{ Matrix } & \multicolumn{1}{c}{ $\mathbf{A}\mathbf{x}=\mathbf{b}$ } & \multicolumn{1}{c}{ $\mathbf{A}\mathbf{x}=\lambda\mathbf{x}$ } \\
\hline
$\mathbf{A}=\mathbf{A}^*$    & Conjugate gradient                & Lanczos                                  \\
$\mathbf{A}\ne \mathbf{A}^*$ & GMRES etc                         & Arnoldi                                  \\
\hline
\end{tabular}
\end{quote}

\noindent
% --- end paragraph admon ---



\subsection*{Eigenvalues and Lanczos' method}

% --- begin paragraph admon ---
\paragraph{}
Basic features with a real symmetric matrix (and normally huge $n> 10^6$ and sparse) 
$\hat{A}$ of dimension $n\times n$:

\begin{itemize}
\item Lanczos' algorithm generates a sequence of real tridiagonal matrices $T_k$ of dimension $k\times k$ with $k\le n$, with the property that the extremal eigenvalues of $T_k$ are progressively better estimates of $\hat{A}$' extremal eigenvalues.* The method converges to the extremal eigenvalues.

\item The similarity transformation is 
\end{itemize}

\noindent
\[
\hat{T}= \hat{Q}^{T}\hat{A}\hat{Q},
\]
with the first vector $\hat{Q}\hat{e}_1=\hat{q}_1$.

We are going to solve iteratively
\[
\hat{T}= \hat{Q}^{T}\hat{A}\hat{Q},
\]
with the first vector $\hat{Q}\hat{e}_1=\hat{q}_1$.
We can write out the matrix $\hat{Q}$ in terms of its column vectors 
\[
\hat{Q}=\left[\hat{q}_1\hat{q}_2\dots\hat{q}_n\right].
\]
% --- end paragraph admon ---



\subsection*{Eigenvalues and Lanczos' method, tridiagonal matrix}

% --- begin paragraph admon ---
\paragraph{}
The matrix
\[
\hat{T}= \hat{Q}^{T}\hat{A}\hat{Q},
\]
can be written as 
\[
    \hat{T} = \left(\begin{array}{cccccc}
                           \alpha_1& \beta_1 & 0 &\dots   & \dots &0 \\
                           \beta_1 & \alpha_2 & \beta_2 &0 &\dots &0 \\
                           0& \beta_2 & \alpha_3 & \beta_3 & \dots &0 \\
                           \dots& \dots   & \dots &\dots   &\dots & 0 \\
                           \dots&   &  &\beta_{n-2}  &\alpha_{n-1}& \beta_{n-1} \\
                           0&  \dots  &\dots  &0   &\beta_{n-1} & \alpha_{n} \\
                      \end{array} \right)
\]
% --- end paragraph admon ---



\subsection*{Eigenvalues and Lanczos' method, tridiagonal and orthogonal matrices}

% --- begin paragraph admon ---
\paragraph{}
Using the fact that 
\[
\hat{Q}\hat{Q}^T=\hat{I}, 
\]
we can rewrite 
\[
\hat{T}= \hat{Q}^{T}\hat{A}\hat{Q},
\]
as 
\[
\hat{Q}\hat{T}= \hat{A}\hat{Q}.
\]
% --- end paragraph admon ---



\subsection*{Eigenvalues and Lanczos' method}

% --- begin paragraph admon ---
\paragraph{}
If we equate columns 
\[
\hat{T} = \left(\begin{array}{cccccc}
        \alpha_1& \beta_1 & 0 &\dots   & \dots &0 \\
        \beta_1 & \alpha_2 & \beta_2 &0 &\dots &0 \\
        0& \beta_2 & \alpha_3 & \beta_3 & \dots &0 \\
        \dots& \dots   & \dots &\dots   &\dots & 0 \\
        \dots&   &  &\beta_{n-2}  &\alpha_{n-1}& \beta_{n-1} \\
        0&  \dots  &\dots  &0   &\beta_{n-1} & \alpha_{n} \\
        \end{array} \right)
\]
we obtain
\[
\hat{A}\hat{q}_k=\beta_{k-1}\hat{q}_{k-1}+\alpha_k\hat{q}_k+\beta_k\hat{q}_{k+1}.
\]
% --- end paragraph admon ---



\subsection*{Eigenvalues and Lanczos' method, defining the Lanczos' vectors}

% --- begin paragraph admon ---
\paragraph{}
We have thus
\[
\hat{A}\hat{q}_k=\beta_{k-1}\hat{q}_{k-1}+\alpha_k\hat{q}_k+\beta_k\hat{q}_{k+1},
\]
with $\beta_0\hat{q}_0=0$ for $k=1:n-1$. Remember that the vectors $\hat{q}_k$  are orthornormal and this implies
\[
\alpha_k=\hat{q}_k^T\hat{A}\hat{q}_k,
\]
and these vectors are called Lanczos vectors.
% --- end paragraph admon ---



\subsection*{Eigenvalues and Lanczos' method, basic steps}

% --- begin paragraph admon ---
\paragraph{}
We have thus
\[
\hat{A}\hat{q}_k=\beta_{k-1}\hat{q}_{k-1}+\alpha_k\hat{q}_k+\beta_k\hat{q}_{k+1},
\]
with $\beta_0\hat{q}_0=0$ for $k=1:n-1$ and 
\[
\alpha_k=\hat{q}_k^T\hat{A}\hat{q}_k.
\]
If 
\[
\hat{r}_k=(\hat{A}-\alpha_k\hat{I})\hat{q}_k-\beta_{k-1}\hat{q}_{k-1},
\]
is non-zero, then 
\[
\hat{q}_{k+1}=\hat{r}_{k}/\beta_k,
\]
with $\beta_k=\pm ||\hat{r}_{k}||_2$.

 \part{Mean-field theories and post Hartree-Fock methods}
\clearemptydoublepage
      
\chapter{Hartree-Fock methods}

\subsection*{Why Hartree-Fock? Derivation of Hartree-Fock equations in coordinate space}

Hartree-Fock (HF) theory is an algorithm for finding an approximative expression for the ground state of a given Hamiltonian. The basic ingredients are
\begin{itemize}
  \item Define a single-particle basis $\{\psi_{\alpha}\}$ so that
\end{itemize}

\noindent
\[ 
\hat{h}^{\mathrm{HF}}\psi_{\alpha} = \varepsilon_{\alpha}\psi_{\alpha}
\]
with the Hartree-Fock Hamiltonian defined as
\[
\hat{h}^{\mathrm{HF}}=\hat{t}+\hat{u}_{\mathrm{ext}}+\hat{u}^{\mathrm{HF}}
\]
\begin{itemize}
  \item The term  $\hat{u}^{\mathrm{HF}}$ is a single-particle potential to be determined by the HF algorithm.

  \item The HF algorithm means to choose $\hat{u}^{\mathrm{HF}}$ in order to have 
\end{itemize}

\noindent
\[ \langle \hat{H} \rangle = E^{\mathrm{HF}}= \langle \Phi_0 | \hat{H}|\Phi_0 \rangle
\]
that is to find a local minimum with a Slater determinant $\Phi_0$ being the ansatz for the ground state. 
\begin{itemize}
  \item The variational principle ensures that $E^{\mathrm{HF}} \ge E_0$, with $E_0$ the exact ground state energy.
\end{itemize}

\noindent
We will show that the Hartree-Fock Hamiltonian $\hat{h}^{\mathrm{HF}}$ equals our definition of the operator $\hat{f}$ discussed in connection with the new definition of the normal-ordered Hamiltonian (see later lectures), that is we have, for a specific matrix element
\[
\langle p |\hat{h}^{\mathrm{HF}}| q \rangle =\langle p |\hat{f}| q \rangle=\langle p|\hat{t}+\hat{u}_{\mathrm{ext}}|q \rangle +\sum_{i\le F} \langle pi | \hat{V} | qi\rangle_{AS},
\]
meaning that
\[
\langle p|\hat{u}^{\mathrm{HF}}|q\rangle = \sum_{i\le F} \langle pi | \hat{V} | qi\rangle_{AS}.
\]
The so-called Hartree-Fock potential $\hat{u}^{\mathrm{HF}}$ brings an explicit medium dependence due to the summation over all single-particle states below the Fermi level $F$. It brings also in an explicit dependence on the two-body interaction (in nuclear physics we can also have complicated three- or higher-body forces). The two-body interaction, with its contribution from the other bystanding fermions, creates an effective mean field in which a given fermion moves, in addition to the external potential $\hat{u}_{\mathrm{ext}}$ which confines the motion of the fermion. For systems like nuclei, there is no external confining potential. Nuclei are examples of self-bound systems, where the binding arises due to the intrinsic nature of the strong force. For nuclear systems thus, there would be no external one-body potential in the Hartree-Fock Hamiltonian. 

\subsection*{Variational Calculus and Lagrangian Multipliers}

The calculus of variations involves 
problems where the quantity to be minimized or maximized is an integral. 

In the general case we have an integral of the type
\[ 
E[\Phi]= \int_a^b f(\Phi(x),\frac{\partial \Phi}{\partial x},x)dx,
\]
where $E$ is the quantity which is sought minimized or maximized.
The problem is that although $f$ is a function of the variables $\Phi$, $\partial \Phi/\partial x$ and $x$, the exact dependence of
$\Phi$ on $x$ is not known.  This means again that even though the integral has fixed limits $a$ and $b$, the path of integration is
not known. In our case the unknown quantities are the single-particle wave functions and we wish to choose an integration path which makes
the functional $E[\Phi]$ stationary. This means that we want to find minima, or maxima or saddle points. In physics we search normally for minima.
Our task is therefore to find the minimum of $E[\Phi]$ so that its variation $\delta E$ is zero  subject to specific
constraints. In our case the constraints appear as the integral which expresses the orthogonality of the  single-particle wave functions.
The constraints can be treated via the technique of Lagrangian multipliers

Let us specialize to the expectation value of the energy for one particle in three-dimensions.
This expectation value reads
\[
  E=\int dxdydz \psi^*(x,y,z) \hat{H} \psi(x,y,z),
\]
with the constraint
\[
 \int dxdydz \psi^*(x,y,z) \psi(x,y,z)=1,
\]
and a Hamiltonian
\[
\hat{H}=-\frac{1}{2}\nabla^2+V(x,y,z).
\]
We will, for the sake of notational convenience,  skip the variables $x,y,z$ below, and write for example $V(x,y,z)=V$.

The integral involving the kinetic energy can be written as, with the function $\psi$ vanishing
strongly for large values of $x,y,z$ (given here by the limits $a$ and $b$), 
 \[
  \int_a^b dxdydz \psi^* \left(-\frac{1}{2}\nabla^2\right) \psi dxdydz = \psi^*\nabla\psi|_a^b+\int_a^b dxdydz\frac{1}{2}\nabla\psi^*\nabla\psi.
\]
We will drop the limits $a$ and $b$ in the remaining discussion. 
Inserting this expression into the expectation value for the energy and taking the variational minimum  we obtain
\[
\delta E = \delta \left\{\int dxdydz\left( \frac{1}{2}\nabla\psi^*\nabla\psi+V\psi^*\psi\right)\right\} = 0.
\]

The constraint appears in integral form as 
\[
 \int dxdydz \psi^* \psi=\mathrm{constant},
\]
and multiplying with a Lagrangian multiplier $\lambda$ and taking the variational minimum we obtain the final variational equation
\[
\delta \left\{\int dxdydz\left( \frac{1}{2}\nabla\psi^*\nabla\psi+V\psi^*\psi-\lambda\psi^*\psi\right)\right\} = 0.
\]
We introduce the function  $f$
\[
  f =  \frac{1}{2}\nabla\psi^*\nabla\psi+V\psi^*\psi-\lambda\psi^*\psi=
\frac{1}{2}(\psi^*_x\psi_x+\psi^*_y\psi_y+\psi^*_z\psi_z)+V\psi^*\psi-\lambda\psi^*\psi,
\]
where we have skipped the dependence on $x,y,z$ and introduced the shorthand $\psi_x$, $\psi_y$ and $\psi_z$  for the various derivatives.

For $\psi^*$ the Euler-Lagrange  equations yield
\[
\frac{\partial f}{\partial \psi^*}- \frac{\partial }{\partial x}\frac{\partial f}{\partial \psi^*_x}-\frac{\partial }{\partial y}\frac{\partial f}{\partial \psi^*_y}-\frac{\partial }{\partial z}\frac{\partial f}{\partial \psi^*_z}=0,
\] 
which results in 
\[
    -\frac{1}{2}(\psi_{xx}+\psi_{yy}+\psi_{zz})+V\psi=\lambda \psi.
\]
We can then identify the  Lagrangian multiplier as the energy of the system. The last equation is 
nothing but the standard 
Schroedinger equation and the variational  approach discussed here provides 
a powerful method for obtaining approximate solutions of the wave function.

\subsection*{Derivation of Hartree-Fock equations in coordinate space}

Let us denote the ground state energy by $E_0$. According to the
variational principle we have
\[
  E_0 \le E[\Phi] = \int \Phi^*\hat{H}\Phi d\mathbf{\tau}
\]
where $\Phi$ is a trial function which we assume to be normalized
\[
  \int \Phi^*\Phi d\mathbf{\tau} = 1,
\]
where we have used the shorthand $d\mathbf{\tau}=dx_1dx_2\dots dx_A$.

In the Hartree-Fock method the trial function is a Slater
determinant which can be rewritten as 
\[
  \Psi(x_1,x_2,\dots,x_A,\alpha,\beta,\dots,\nu) = \frac{1}{\sqrt{A!}}\sum_{P} (-)^PP\psi_{\alpha}(x_1)
    \psi_{\beta}(x_2)\dots\psi_{\nu}(x_A)=\sqrt{A!}\hat{A}\Phi_H,
\]
where we have introduced the anti-symmetrization operator $\hat{A}$ defined by the 
summation over all possible permutations \emph{p} of two fermions.
It is defined as
\[
  \hat{A} = \frac{1}{A!}\sum_{p} (-)^p\hat{P},
\]
with the the Hartree-function given by the simple product of all possible single-particle function
\[
  \Phi_H(x_1,x_2,\dots,x_A,\alpha,\beta,\dots,\nu) =
  \psi_{\alpha}(x_1)
    \psi_{\beta}(x_2)\dots\psi_{\nu}(x_A).
\]

Our functional is written as
\[
  E[\Phi] = \sum_{\mu=1}^A \int \psi_{\mu}^*(x_i)\hat{h}_0(x_i)\psi_{\mu}(x_i) dx_i 
  + \frac{1}{2}\sum_{\mu=1}^A\sum_{\nu=1}^A
   \left[ \int \psi_{\mu}^*(x_i)\psi_{\nu}^*(x_j)\hat{v}(r_{ij})\psi_{\mu}(x_i)\psi_{\nu}(x_j)dx_idx_j- \int \psi_{\mu}^*(x_i)\psi_{\nu}^*(x_j)
 \hat{v}(r_{ij})\psi_{\nu}(x_i)\psi_{\mu}(x_j)dx_idx_j\right]
\]
The more compact version reads
\[
  E[\Phi] 
  = \sum_{\mu}^A \langle \mu | \hat{h}_0 | \mu\rangle+ \frac{1}{2}\sum_{\mu\nu}^A\left[\langle \mu\nu |\hat{v}|\mu\nu\rangle-\langle \nu\mu |\hat{v}|\mu\nu\rangle\right].
\]

Since the interaction is invariant under the interchange of two particles it means for example that we have
\[
\langle \mu\nu|\hat{v}|\mu\nu\rangle =  \langle \nu\mu|\hat{v}|\nu\mu\rangle,  
\]
or in the more general case
\[
\langle \mu\nu|\hat{v}|\sigma\tau\rangle =  \langle \nu\mu|\hat{v}|\tau\sigma\rangle.  
\]

The direct and exchange matrix elements can be  brought together if we define the antisymmetrized matrix element
\[
\langle \mu\nu|\hat{v}|\mu\nu\rangle_{AS}= \langle \mu\nu|\hat{v}|\mu\nu\rangle-\langle \mu\nu|\hat{v}|\nu\mu\rangle,
\]
or for a general matrix element  
\[
\langle \mu\nu|\hat{v}|\sigma\tau\rangle_{AS}= \langle \mu\nu|\hat{v}|\sigma\tau\rangle-\langle \mu\nu|\hat{v}|\tau\sigma\rangle.
\]
It has the symmetry property
\[
\langle \mu\nu|\hat{v}|\sigma\tau\rangle_{AS}= -\langle \mu\nu|\hat{v}|\tau\sigma\rangle_{AS}=-\langle \nu\mu|\hat{v}|\sigma\tau\rangle_{AS}.
\]
The antisymmetric matrix element is also hermitian, implying 
\[
\langle \mu\nu|\hat{v}|\sigma\tau\rangle_{AS}= \langle \sigma\tau|\hat{v}|\mu\nu\rangle_{AS}.
\]

With these notations we rewrite the Hartree-Fock functional as
\begin{equation}
  \int \Phi^*\hat{H_I}\Phi d\mathbf{\tau} 
  = \frac{1}{2}\sum_{\mu=1}^A\sum_{\nu=1}^A \langle \mu\nu|\hat{v}|\mu\nu\rangle_{AS}. \label{H2Expectation2}
\end{equation}

Adding the contribution from the one-body operator $\hat{H}_0$ to
(\ref{H2Expectation2}) we obtain the energy functional 
\begin{equation}
  E[\Phi] 
  = \sum_{\mu=1}^A \langle \mu | h | \mu \rangle +
  \frac{1}{2}\sum_{{\mu}=1}^A\sum_{{\nu}=1}^A \langle \mu\nu|\hat{v}|\mu\nu\rangle_{AS}. \label{FunctionalEPhi}
\end{equation}
In our coordinate space derivations below we will spell out the Hartree-Fock equations in terms of their integrals.

If we generalize the Euler-Lagrange equations to more variables 
and introduce $N^2$ Lagrange multipliers which we denote by 
$\epsilon_{\mu\nu}$, we can write the variational equation for the functional of $E$
\[
  \delta E - \sum_{\mu\nu}^A \epsilon_{\mu\nu} \delta
  \int \psi_{\mu}^* \psi_{\nu} = 0.
\]
For the orthogonal wave functions $\psi_{i}$ this reduces to
\[
  \delta E - \sum_{\mu=1}^A \epsilon_{\mu} \delta
  \int \psi_{\mu}^* \psi_{\mu} = 0.
\]

Variation with respect to the single-particle wave functions $\psi_{\mu}$ yields then
\[
  \sum_{\mu=1}^A \int \delta\psi_{\mu}^*\hat{h_0}(x_i)\psi_{\mu}
  dx_i  
  + \frac{1}{2}\sum_{{\mu}=1}^A\sum_{{\nu}=1}^A \left[ \int
  \delta\psi_{\mu}^*\psi_{\nu}^*\hat{v}(r_{ij})\psi_{\mu}\psi_{\nu} dx_idx_j- \int
  \delta\psi_{\mu}^*\psi_{\nu}^*\hat{v}(r_{ij})\psi_{\nu}\psi_{\mu}
  dx_idx_j \right]+ 
\]
\[
\sum_{\mu=1}^A \int \psi_{\mu}^*\hat{h_0}(x_i)\delta\psi_{\mu}
  dx_i 
  + \frac{1}{2}\sum_{{\mu}=1}^A\sum_{{\nu}=1}^A \left[ \int
  \psi_{\mu}^*\psi_{\nu}^*\hat{v}(r_{ij})\delta\psi_{\mu}\psi_{\nu} dx_idx_j- \int
  \psi_{\mu}^*\psi_{\nu}^*\hat{v}(r_{ij})\psi_{\nu}\delta\psi_{\mu}
  dx_idx_j \right]-  \sum_{{\mu}=1}^A E_{\mu} \int \delta\psi_{\mu}^*
  \psi_{\mu}dx_i
  -  \sum_{{\mu}=1}^A E_{\mu} \int \psi_{\mu}^*
  \delta\psi_{\mu}dx_i = 0.
\]

Although the variations $\delta\psi$ and $\delta\psi^*$ are not
independent, they may in fact be treated as such, so that the 
terms dependent on either $\delta\psi$ and $\delta\psi^*$ individually 
may be set equal to zero. To see this, simply 
replace the arbitrary variation $\delta\psi$ by $i\delta\psi$, so that
$\delta\psi^*$ is replaced by $-i\delta\psi^*$, and combine the two
equations. We thus arrive at the Hartree-Fock equations
\begin{equation}
\left[ -\frac{1}{2}\nabla_i^2+ \sum_{\nu=1}^A\int \psi_{\nu}^*(x_j)\hat{v}(r_{ij})\psi_{\nu}(x_j)dx_j \right]\psi_{\mu}(x_i) - \left[ \sum_{{\nu}=1}^A \int\psi_{\nu}^*(x_j)\hat{v}(r_{ij})\psi_{\mu}(x_j) dx_j\right] \psi_{\nu}(x_i) = \epsilon_{\mu} \psi_{\mu}(x_i).  \label{eq:hartreefockcoordinatespace}
\end{equation}
Notice that the integration $\int dx_j$ implies an
integration over the spatial coordinates $\mathbf{r_j}$ and a summation
over the spin-coordinate of fermion $j$. We note that the factor of $1/2$ in front of the sum involving the two-body interaction, has been removed. This is due to the fact that we need to vary both $\delta\psi_{\mu}^*$ and
$\delta\psi_{\nu}^*$. Using the symmetry properties of the two-body interaction and interchanging $\mu$ and $\nu$
as summation indices, we obtain two identical terms. 

The two first terms in the last equation are the one-body kinetic energy and the
electron-nucleus potential. The third or \emph{direct} term is the averaged electronic repulsion of the other
electrons. As written, the
term includes the \emph{self-interaction} of 
electrons when $\mu=\nu$. The self-interaction is cancelled in the fourth
term, or the \emph{exchange} term. The exchange term results from our
inclusion of the Pauli principle and the assumed determinantal form of
the wave-function. Equation (\ref{eq:hartreefockcoordinatespace}), in addition to the kinetic energy and the attraction from the atomic nucleus that confines the motion of a single electron,   represents now the motion of a single-particle modified by the two-body interaction. The additional contribution to the Schroedinger equation due to the two-body interaction, represents a mean field set up by all the other bystanding electrons, the latter given by the sum over all single-particle states occupied by $N$ electrons. 

The Hartree-Fock equation is an example of an integro-differential equation. These equations involve repeated calculations of integrals, in addition to the solution of a set of coupled differential equations. 
The Hartree-Fock equations can also be rewritten in terms of an eigenvalue problem. The solution of an eigenvalue problem represents often a more practical algorithm and the  solution of  coupled  integro-differential equations.
This alternative derivation of the Hartree-Fock equations is given below.

\subsection*{Analysis of Hartree-Fock equations in coordinate space}

  A theoretically convenient form of the
Hartree-Fock equation is to regard the direct and exchange operator
defined through 
\begin{equation*}
  V_{\mu}^{d}(x_i) = \int \psi_{\mu}^*(x_j) 
 \hat{v}(r_{ij})\psi_{\mu}(x_j) dx_j
\end{equation*}
and
\begin{equation*}
  V_{\mu}^{ex}(x_i) g(x_i) 
  = \left(\int \psi_{\mu}^*(x_j) 
 \hat{v}(r_{ij})g(x_j) dx_j
  \right)\psi_{\mu}(x_i),
\end{equation*}
respectively. 

The function $g(x_i)$ is an arbitrary function,
and by the substitution $g(x_i) = \psi_{\nu}(x_i)$
we get
\begin{equation*}
  V_{\mu}^{ex}(x_i) \psi_{\nu}(x_i) 
  = \left(\int \psi_{\mu}^*(x_j) 
 \hat{v}(r_{ij})\psi_{\nu}(x_j)
  dx_j\right)\psi_{\mu}(x_i).
\end{equation*}
We may then rewrite the Hartree-Fock equations as
\[
  \hat{h}^{HF}(x_i) \psi_{\nu}(x_i) = \epsilon_{\nu}\psi_{\nu}(x_i),
\]
with
\[
  \hat{h}^{HF}(x_i)= \hat{h}_0(x_i) + \sum_{\mu=1}^AV_{\mu}^{d}(x_i) -
  \sum_{\mu=1}^AV_{\mu}^{ex}(x_i),
\]
and where $\hat{h}_0(i)$ is the one-body part. The latter is normally chosen as a part which yields solutions in closed form. The harmonic oscilltor is a classical problem thereof.
We normally rewrite the last equation as
\[
  \hat{h}^{HF}(x_i)= \hat{h}_0(x_i) + \hat{u}^{HF}(x_i). 
\]

\subsection*{Hartree-Fock by varying the coefficients of a wave function expansion}

Another possibility is to expand the single-particle functions in a known basis  and vary the coefficients, 
that is, the new single-particle wave function is written as a linear expansion
in terms of a fixed chosen orthogonal basis (for example the well-known harmonic oscillator functions or the hydrogen-like functions etc).
We define our new Hartree-Fock single-particle basis by performing a unitary transformation 
on our previous basis (labelled with greek indices) as
\begin{equation}
\psi_p^{HF}  = \sum_{\lambda} C_{p\lambda}\phi_{\lambda}. \label{eq:newbasis}
\end{equation}
In this case we vary the coefficients $C_{p\lambda}$. If the basis has infinitely many solutions, we need
to truncate the above sum.  We assume that the basis $\phi_{\lambda}$ is orthogonal.

It is normal to choose a single-particle basis defined as the eigenfunctions
of parts of the full Hamiltonian. The typical situation consists of the solutions of the one-body part of the Hamiltonian, that is we have
\[
\hat{h}_0\phi_{\lambda}=\epsilon_{\lambda}\phi_{\lambda}.
\]
The single-particle wave functions $\phi_{\lambda}(\mathbf{r})$, defined by the quantum numbers $\lambda$ and $\mathbf{r}$
are defined as the overlap 
\[
   \phi_{\lambda}(\mathbf{r})  = \langle \mathbf{r} | \lambda \rangle .
\]

In deriving the Hartree-Fock equations, we  will expand the single-particle functions in a known basis  and vary the coefficients, 
that is, the new single-particle wave function is written as a linear expansion
in terms of a fixed chosen orthogonal basis (for example the well-known harmonic oscillator functions or the hydrogen-like functions etc).

We stated that a unitary transformation keeps the orthogonality. To see this consider first a basis of vectors $\mathbf{v}_i$,
\[
\mathbf{v}_i = \begin{bmatrix} v_{i1} \\ \dots \\ \dots \\v_{in} \end{bmatrix}
\]
We assume that the basis is orthogonal, that is 
\[
\mathbf{v}_j^T\mathbf{v}_i = \delta_{ij}.
\]
An orthogonal or unitary transformation
\[
\mathbf{w}_i=\mathbf{U}\mathbf{v}_i,
\]
preserves the dot product and orthogonality since
\[
\mathbf{w}_j^T\mathbf{w}_i=(\mathbf{U}\mathbf{v}_j)^T\mathbf{U}\mathbf{v}_i=\mathbf{v}_j^T\mathbf{U}^T\mathbf{U}\mathbf{v}_i= \mathbf{v}_j^T\mathbf{v}_i = \delta_{ij}.
\]

This means that if the coefficients $C_{p\lambda}$ belong to a unitary or orthogonal trasformation (using the Dirac bra-ket notation)
\[
\vert p\rangle  = \sum_{\lambda} C_{p\lambda}\vert\lambda\rangle,
\]
orthogonality is preserved, that is $\langle \alpha \vert \beta\rangle = \delta_{\alpha\beta}$
and $\langle p \vert q\rangle = \delta_{pq}$. 

This propertry is extremely useful when we build up a basis of many-body Stater determinant based states. 

\textbf{Note also that although a basis $\vert \alpha\rangle$ contains an infinity of states, for practical calculations we have always to make some truncations.} 

Before we develop the Hartree-Fock equations, there is another very useful property of determinants that we will use both in connection with Hartree-Fock calculations and later shell-model calculations.  

Consider the following determinant
\[
\left| \begin{array}{cc} \alpha_1b_{11}+\alpha_2sb_{12}& a_{12}\\
                         \alpha_1b_{21}+\alpha_2b_{22}&a_{22}\end{array} \right|=\alpha_1\left|\begin{array}{cc} b_{11}& a_{12}\\
                         b_{21}&a_{22}\end{array} \right|+\alpha_2\left| \begin{array}{cc} b_{12}& a_{12}\\b_{22}&a_{22}\end{array} \right|
\]

We can generalize this to  an $n\times n$ matrix and have 
\[
\left| \begin{array}{cccccc} a_{11}& a_{12} & \dots & \sum_{k=1}^n c_k b_{1k} &\dots & a_{1n}\\
a_{21}& a_{22} & \dots & \sum_{k=1}^n c_k b_{2k} &\dots & a_{2n}\\
\dots & \dots & \dots & \dots & \dots & \dots \\
\dots & \dots & \dots & \dots & \dots & \dots \\
a_{n1}& a_{n2} & \dots & \sum_{k=1}^n c_k b_{nk} &\dots & a_{nn}\end{array} \right|=
\sum_{k=1}^n c_k\left| \begin{array}{cccccc} a_{11}& a_{12} & \dots &  b_{1k} &\dots & a_{1n}\\
a_{21}& a_{22} & \dots &  b_{2k} &\dots & a_{2n}\\
\dots & \dots & \dots & \dots & \dots & \dots\\
\dots & \dots & \dots & \dots & \dots & \dots\\
a_{n1}& a_{n2} & \dots &  b_{nk} &\dots & a_{nn}\end{array} \right| .
\]
This is a property we will use in our Hartree-Fock discussions. 

We can generalize the previous results, now 
with all elements $a_{ij}$  being given as functions of 
linear combinations  of various coefficients $c$ and elements $b_{ij}$,
\[
\left| \begin{array}{cccccc} \sum_{k=1}^n b_{1k}c_{k1}& \sum_{k=1}^n b_{1k}c_{k2} & \dots & \sum_{k=1}^n b_{1k}c_{kj}  &\dots & \sum_{k=1}^n b_{1k}c_{kn}\\
\sum_{k=1}^n b_{2k}c_{k1}& \sum_{k=1}^n b_{2k}c_{k2} & \dots & \sum_{k=1}^n b_{2k}c_{kj} &\dots & \sum_{k=1}^n b_{2k}c_{kn}\\
\dots & \dots & \dots & \dots & \dots & \dots \\
\dots & \dots & \dots & \dots & \dots &\dots \\
\sum_{k=1}^n b_{nk}c_{k1}& \sum_{k=1}^n b_{nk}c_{k2} & \dots & \sum_{k=1}^n b_{nk}c_{kj} &\dots & \sum_{k=1}^n b_{nk}c_{kn}\end{array} \right|=det(\mathbf{C})det(\mathbf{B}),
\]
where $det(\mathbf{C})$ and $det(\mathbf{B})$ are the determinants of $n\times n$ matrices
with elements $c_{ij}$ and $b_{ij}$ respectively.  
This is a property we will use in our Hartree-Fock discussions. Convince yourself about the correctness of the above expression by setting $n=2$. 

With our definition of the new basis in terms of an orthogonal basis we have
\[
\psi_p(x)  = \sum_{\lambda} C_{p\lambda}\phi_{\lambda}(x).
\]
If the coefficients $C_{p\lambda}$ belong to an orthogonal or unitary matrix, the new basis
is also orthogonal. 
Our Slater determinant in the new basis $\psi_p(x)$ is written as
\[
\frac{1}{\sqrt{A!}}
\left| \begin{array}{ccccc} \psi_{p}(x_1)& \psi_{p}(x_2)& \dots & \dots & \psi_{p}(x_A)\\
                            \psi_{q}(x_1)&\psi_{q}(x_2)& \dots & \dots & \psi_{q}(x_A)\\  
                            \dots & \dots & \dots & \dots & \dots \\
                            \dots & \dots & \dots & \dots & \dots \\
                     \psi_{t}(x_1)&\psi_{t}(x_2)& \dots & \dots & \psi_{t}(x_A)\end{array} \right|=\frac{1}{\sqrt{A!}}
\left| \begin{array}{ccccc} \sum_{\lambda} C_{p\lambda}\phi_{\lambda}(x_1)& \sum_{\lambda} C_{p\lambda}\phi_{\lambda}(x_2)& \dots & \dots & \sum_{\lambda} C_{p\lambda}\phi_{\lambda}(x_A)\\
                            \sum_{\lambda} C_{q\lambda}\phi_{\lambda}(x_1)&\sum_{\lambda} C_{q\lambda}\phi_{\lambda}(x_2)& \dots & \dots & \sum_{\lambda} C_{q\lambda}\phi_{\lambda}(x_A)\\  
                            \dots & \dots & \dots & \dots & \dots \\
                            \dots & \dots & \dots & \dots & \dots \\
                     \sum_{\lambda} C_{t\lambda}\phi_{\lambda}(x_1)&\sum_{\lambda} C_{t\lambda}\phi_{\lambda}(x_2)& \dots & \dots & \sum_{\lambda} C_{t\lambda}\phi_{\lambda}(x_A)\end{array} \right|,
\]
which is nothing but $det(\mathbf{C})det(\Phi)$, with $det(\Phi)$ being the determinant given by the basis functions $\phi_{\lambda}(x)$. 

In our discussions hereafter we will use our definitions of single-particle states above and below the Fermi ($F$) level given by the labels
$ijkl\dots \le F$ for so-called single-hole states and $abcd\dots > F$ for so-called particle states.
For general single-particle states we employ the labels $pqrs\dots$. 

In Eq.~(\ref{FunctionalEPhi}), restated here
\[
  E[\Phi] 
  = \sum_{\mu=1}^A \langle \mu | h | \mu \rangle +
  \frac{1}{2}\sum_{{\mu}=1}^A\sum_{{\nu}=1}^A \langle \mu\nu|\hat{v}|\mu\nu\rangle_{AS},
\]
we found the expression for the energy functional in terms of the basis function $\phi_{\lambda}(\mathbf{r})$. We then  varied the above energy functional with respect to the basis functions $|\mu \rangle$. 
Now we are interested in defining a new basis defined in terms of
a chosen basis as defined in Eq.~(\ref{eq:newbasis}). We can then rewrite the energy functional as
\begin{equation}
  E[\Phi^{HF}] 
  = \sum_{i=1}^A \langle i | h | i \rangle +
  \frac{1}{2}\sum_{ij=1}^A\langle ij|\hat{v}|ij\rangle_{AS}, \label{FunctionalEPhi2}
\end{equation}
where $\Phi^{HF}$ is the new Slater determinant defined by the new basis of Eq.~(\ref{eq:newbasis}). 

Using Eq.~(\ref{eq:newbasis}) we can rewrite Eq.~(\ref{FunctionalEPhi2}) as 
\begin{equation}
  E[\Psi] 
  = \sum_{i=1}^A \sum_{\alpha\beta} C^*_{i\alpha}C_{i\beta}\langle \alpha | h | \beta \rangle +
  \frac{1}{2}\sum_{ij=1}^A\sum_{{\alpha\beta\gamma\delta}} C^*_{i\alpha}C^*_{j\beta}C_{i\gamma}C_{j\delta}\langle \alpha\beta|\hat{v}|\gamma\delta\rangle_{AS}. \label{FunctionalEPhi3}
\end{equation}

We wish now to minimize the above functional. We introduce again a set of Lagrange multipliers, noting that
since $\langle i | j \rangle = \delta_{i,j}$ and $\langle \alpha | \beta \rangle = \delta_{\alpha,\beta}$, 
the coefficients $C_{i\gamma}$ obey the relation
\[
 \langle i | j \rangle=\delta_{i,j}=\sum_{\alpha\beta} C^*_{i\alpha}C_{i\beta}\langle \alpha | \beta \rangle=
\sum_{\alpha} C^*_{i\alpha}C_{i\alpha},
\]
which allows us to define a functional to be minimized that reads
\begin{equation}
  F[\Phi^{HF}]=E[\Phi^{HF}] - \sum_{i=1}^A\epsilon_i\sum_{\alpha} C^*_{i\alpha}C_{i\alpha}.
\end{equation}

Minimizing with respect to $C^*_{i\alpha}$, remembering that the equations for $C^*_{i\alpha}$ and $C_{i\alpha}$
can be written as two  independent equations, we obtain
\[
\frac{d}{dC^*_{i\alpha}}\left[  E[\Phi^{HF}] - \sum_{j}\epsilon_j\sum_{\alpha} C^*_{j\alpha}C_{j\alpha}\right]=0,
\]
which yields for every single-particle state $i$ and index $\alpha$ (recalling that the coefficients $C_{i\alpha}$ are matrix elements of a unitary (or orthogonal for a real symmetric matrix) matrix)
the following Hartree-Fock equations
\[
\sum_{\beta} C_{i\beta}\langle \alpha | h | \beta \rangle+
\sum_{j=1}^A\sum_{\beta\gamma\delta} C^*_{j\beta}C_{j\delta}C_{i\gamma}\langle \alpha\beta|\hat{v}|\gamma\delta\rangle_{AS}=\epsilon_i^{HF}C_{i\alpha}.
\]

We can rewrite this equation as (changing dummy variables)
\[
\sum_{\beta} \left\{\langle \alpha | h | \beta \rangle+
\sum_{j}^A\sum_{\gamma\delta} C^*_{j\gamma}C_{j\delta}\langle \alpha\gamma|\hat{v}|\beta\delta\rangle_{AS}\right\}C_{i\beta}=\epsilon_i^{HF}C_{i\alpha}.
\]
Note that the sums over greek indices run over the number of basis set functions (in principle an infinite number).

Defining 
\[
h_{\alpha\beta}^{HF}=\langle \alpha | h | \beta \rangle+
\sum_{j=1}^A\sum_{\gamma\delta} C^*_{j\gamma}C_{j\delta}\langle \alpha\gamma|\hat{v}|\beta\delta\rangle_{AS},
\]
we can rewrite the new equations as 
\begin{equation}
\sum_{\beta}h_{\alpha\beta}^{HF}C_{i\beta}=\epsilon_i^{HF}C_{i\alpha}. \label{eq:newhf}
\end{equation}
The latter is nothing but a standard eigenvalue problem. Compared with Eq.~(\ref{eq:hartreefockcoordinatespace}),
we see that we do not need to compute any integrals in an iterative procedure for solving the equations.
It suffices to tabulate the matrix elements $\langle \alpha | h | \beta \rangle$ and $\langle \alpha\gamma|\hat{v}|\beta\delta\rangle_{AS}$ once and for all. Successive iterations require thus only a look-up in tables over one-body and two-body matrix elements. These details will be discussed below when we solve the Hartree-Fock equations numerical. 

\subsection*{Hartree-Fock algorithm}

Our Hartree-Fock matrix  is thus
\[
\hat{h}_{\alpha\beta}^{HF}=\langle \alpha | \hat{h}_0 | \beta \rangle+
\sum_{j=1}^A\sum_{\gamma\delta} C^*_{j\gamma}C_{j\delta}\langle \alpha\gamma|\hat{v}|\beta\delta\rangle_{AS}.
\]
The Hartree-Fock equations are solved in an iterative waym starting with a guess for the coefficients $C_{j\gamma}=\delta_{j,\gamma}$ and solving the equations by diagonalization till the new single-particle energies
$\epsilon_i^{\mathrm{HF}}$ do not change anymore by a prefixed quantity. 

Normally we assume that the single-particle basis $|\beta\rangle$ forms an eigenbasis for the operator
$\hat{h}_0$, meaning that the Hartree-Fock matrix becomes  
\[
\hat{h}_{\alpha\beta}^{HF}=\epsilon_{\alpha}\delta_{\alpha,\beta}+
\sum_{j=1}^A\sum_{\gamma\delta} C^*_{j\gamma}C_{j\delta}\langle \alpha\gamma|\hat{v}|\beta\delta\rangle_{AS}.
\]
The Hartree-Fock eigenvalue problem
\[
\sum_{\beta}\hat{h}_{\alpha\beta}^{HF}C_{i\beta}=\epsilon_i^{\mathrm{HF}}C_{i\alpha},
\]
can be written out in a more compact form as
\[
\hat{h}^{HF}\hat{C}=\epsilon^{\mathrm{HF}}\hat{C}. 
\]

The Hartree-Fock equations are, in their simplest form, solved in an iterative way, starting with a guess for the
coefficients $C_{i\alpha}$. We label the coefficients as $C_{i\alpha}^{(n)}$, where the subscript $n$ stands for iteration $n$.
To set up the algorithm we can proceed as follows:

\begin{itemize}
 \item We start with a guess $C_{i\alpha}^{(0)}=\delta_{i,\alpha}$. Alternatively, we could have used random starting values as long as the vectors are normalized. Another possibility is to give states below the Fermi level a larger weight.

 \item The Hartree-Fock matrix simplifies then to (assuming that the coefficients $C_{i\alpha} $  are real)
\end{itemize}

\noindent
\[
\hat{h}_{\alpha\beta}^{HF}=\epsilon_{\alpha}\delta_{\alpha,\beta}+
\sum_{j = 1}^A\sum_{\gamma\delta} C_{j\gamma}^{(0)}C_{j\delta}^{(0)}\langle \alpha\gamma|\hat{v}|\beta\delta\rangle_{AS}.
\]

Solving the Hartree-Fock eigenvalue problem yields then new eigenvectors $C_{i\alpha}^{(1)}$ and eigenvalues
$\epsilon_i^{HF(1)}$. 
\begin{itemize}
 \item With the new eigenvalues we can set up a new Hartree-Fock potential 
\end{itemize}

\noindent
\[
\sum_{j = 1}^A\sum_{\gamma\delta} C_{j\gamma}^{(1)}C_{j\delta}^{(1)}\langle \alpha\gamma|\hat{v}|\beta\delta\rangle_{AS}.
\]
The diagonalization with the new Hartree-Fock potential yields new eigenvectors and eigenvalues.
This process is continued till for example
\[
\frac{\sum_{p} |\epsilon_i^{(n)}-\epsilon_i^{(n-1)}|}{m} \le \lambda,  
\]
where $\lambda$ is a user prefixed quantity ($\lambda \sim 10^{-8}$ or smaller) and $p$ runs over all calculated single-particle
energies and $m$ is the number of single-particle states.

\subsection*{Analysis of Hartree-Fock equations and Koopman's theorem}

We can rewrite the ground state energy by adding and subtracting $\hat{u}^{HF}(x_i)$ 
\[
  E_0^{HF} =\langle \Phi_0 | \hat{H} | \Phi_0\rangle = 
\sum_{i\le F}^A \langle i | \hat{h}_0 +\hat{u}^{HF}| j\rangle+ \frac{1}{2}\sum_{i\le F}^A\sum_{j \le F}^A\left[\langle ij |\hat{v}|ij \rangle-\langle ij|\hat{v}|ji\rangle\right]-\sum_{i\le F}^A \langle i |\hat{u}^{HF}| i\rangle,
\]
which results in
\[
  E_0^{HF}
  = \sum_{i\le F}^A \varepsilon_i^{HF} + \frac{1}{2}\sum_{i\le F}^A\sum_{j \le F}^A\left[\langle ij |\hat{v}|ij \rangle-\langle ij|\hat{v}|ji\rangle\right]-\sum_{i\le F}^A \langle i |\hat{u}^{HF}| i\rangle.
\]
Our single-particle states $ijk\dots$ are now single-particle states obtained from the solution of the Hartree-Fock equations.

Using our definition of the Hartree-Fock single-particle energies we obtain then the following expression for the total ground-state energy
\[
  E_0^{HF}
  = \sum_{i\le F}^A \varepsilon_i - \frac{1}{2}\sum_{i\le F}^A\sum_{j \le F}^A\left[\langle ij |\hat{v}|ij \rangle-\langle ij|\hat{v}|ji\rangle\right].
\]
This form will be used in our discussion of Koopman's theorem.

In the   atomic physics case we have 
\[
  E[\Phi^{\mathrm{HF}}(N)] 
  = \sum_{i=1}^H \langle i | \hat{h}_0 | i \rangle +
  \frac{1}{2}\sum_{ij=1}^N\langle ij|\hat{v}|ij\rangle_{AS},
\]
where $\Phi^{\mathrm{HF}}(N)$ is the new Slater determinant defined by the new basis of Eq.~(\ref{eq:newbasis})
for $N$ electrons (same $Z$).  If we assume that the single-particle wave functions in the new basis do not change 
when we remove one electron or add one electron, we can then define the corresponding energy for the $N-1$ systems as 
\[
  E[\Phi^{\mathrm{HF}}(N-1)] 
  = \sum_{i=1; i\ne k}^N \langle i | \hat{h}_0 | i \rangle +
  \frac{1}{2}\sum_{ij=1;i,j\ne k}^N\langle ij|\hat{v}|ij\rangle_{AS},
\]
where we have removed a single-particle state $k\le F$, that is a state below the Fermi level.  

Calculating the difference 
\[
  E[\Phi^{\mathrm{HF}}(N)]-   E[\Phi^{\mathrm{HF}}(N-1)] = \langle k | \hat{h}_0 | k \rangle +
  \frac{1}{2}\sum_{i=1;i\ne k}^N\langle ik|\hat{v}|ik\rangle_{AS} + \frac{1}{2}\sum_{j=1;j\ne k}^N\langle kj|\hat{v}|kj\rangle_{AS},
\]
we obtain
\[
  E[\Phi^{\mathrm{HF}}(N)]-   E[\Phi^{\mathrm{HF}}(N-1)] = \langle k | \hat{h}_0 | k \rangle +\sum_{j=1}^N\langle kj|\hat{v}|kj\rangle_{AS}
\]
which is just our definition of the Hartree-Fock single-particle energy
\[
  E[\Phi^{\mathrm{HF}}(N)]-   E[\Phi^{\mathrm{HF}}(N-1)] = \epsilon_k^{\mathrm{HF}} 
\]

Similarly, we can now compute the difference (we label the single-particle states above the Fermi level as $abcd > F$)
\[
  E[\Phi^{\mathrm{HF}}(N+1)]-   E[\Phi^{\mathrm{HF}}(N)]= \epsilon_a^{\mathrm{HF}}. 
\]
These two equations can thus be used to the electron affinity or ionization energies, respectively. 
Koopman's theorem states that for example the ionization energy of a closed-shell system is given by the energy of the highest occupied single-particle state.  If we assume that changing the number of electrons from $N$ to $N+1$ does not change the Hartree-Fock single-particle energies and eigenfunctions, then Koopman's theorem simply states that the ionization energy of an atom is given by the single-particle energy of the last bound state. In a similar way, we can also define the electron affinities. 

As an example, consider a simple model for atomic sodium, Na. Neutral sodium has eleven electrons, 
with the weakest bound one being confined the $3s$ single-particle quantum numbers. The energy needed to remove an electron from neutral sodium is rather small, 5.1391 eV, a feature which pertains to all alkali metals.
Having performed a  Hartree-Fock calculation for neutral sodium would then allows us to compute the
ionization energy by using the single-particle energy for the $3s$ states, namely $\epsilon_{3s}^{\mathrm{HF}}$. 

From these considerations, we see that Hartree-Fock theory allows us to make a connection between experimental 
observables (here ionization and affinity energies) and the underlying interactions between particles.  
In this sense, we are now linking the dynamics and structure of a many-body system with the laws of motion which govern the system. Our approach is a reductionistic one, meaning that we want to understand the laws of motion 
in terms of the particles or degrees of freedom which we believe are the fundamental ones. Our Slater determinant, being constructed as the product of various single-particle functions, follows this philosophy.

With similar arguments as in atomic physics, we can now use Hartree-Fock theory to make a link
between nuclear forces and separation energies. Changing to nuclear system, we define
\[
  E[\Phi^{\mathrm{HF}}(A)] 
  = \sum_{i=1}^A \langle i | \hat{h}_0 | i \rangle +
  \frac{1}{2}\sum_{ij=1}^A\langle ij|\hat{v}|ij\rangle_{AS},
\]
where $\Phi^{\mathrm{HF}}(A)$ is the new Slater determinant defined by the new basis of Eq.~(\ref{eq:newbasis})
for $A$ nucleons, where $A=N+Z$, with $N$ now being the number of neutrons and $Z$ th enumber of protons.  If we assume again that the single-particle wave functions in the new basis do not change from a nucleus with $A$ nucleons to a nucleus with $A-1$  nucleons, we can then define the corresponding energy for the $A-1$ systems as 
\[
  E[\Phi^{\mathrm{HF}}(A-1)] 
  = \sum_{i=1; i\ne k}^A \langle i | \hat{h}_0 | i \rangle +
  \frac{1}{2}\sum_{ij=1;i,j\ne k}^A\langle ij|\hat{v}|ij\rangle_{AS},
\]
where we have removed a single-particle state $k\le F$, that is a state below the Fermi level.  

Calculating the difference 
\[
  E[\Phi^{\mathrm{HF}}(A)]-   E[\Phi^{\mathrm{HF}}(A-1)] 
  = \langle k | \hat{h}_0 | k \rangle +
  \frac{1}{2}\sum_{i=1;i\ne k}^A\langle ik|\hat{v}|ik\rangle_{AS} + \frac{1}{2}\sum_{j=1;j\ne k}^A\langle kj|\hat{v}|kj\rangle_{AS},
\]
which becomes 
\[
  E[\Phi^{\mathrm{HF}}(A)]-   E[\Phi^{\mathrm{HF}}(A-1)] 
  = \langle k | \hat{h}_0 | k \rangle +\sum_{j=1}^A\langle kj|\hat{v}|kj\rangle_{AS}
\]
which is just our definition of the Hartree-Fock single-particle energy
\[
  E[\Phi^{\mathrm{HF}}(A)]-   E[\Phi^{\mathrm{HF}}(A-1)] 
  = \epsilon_k^{\mathrm{HF}} 
\]

Similarly, we can now compute the difference (recall that the single-particle states $abcd > F$)
\[
  E[\Phi^{\mathrm{HF}}(A+1)]-   E[\Phi^{\mathrm{HF}}(A)]= \epsilon_a^{\mathrm{HF}}. 
\]
If we then recall that the binding energy differences 
\[
BE(A)-BE(A-1) \hspace{0.5cm} \mathrm{and} \hspace{0.5cm} BE(A+1)-BE(A), 
\]
define the separation energies, we see that the Hartree-Fock single-particle energies can be used to
define separation energies. We have thus our first link between nuclear forces (included in the potential energy term) and an observable quantity defined by differences in binding energies. 

We have thus the following interpretations (if the single-particle fields do not change)
\[
BE(A)-BE(A-1)\approx  E[\Phi^{\mathrm{HF}}(A)]-   E[\Phi^{\mathrm{HF}}(A-1)] 
  = \epsilon_k^{\mathrm{HF}}, 
\]
and
\[
BE(A+1)-BE(A)\approx  E[\Phi^{\mathrm{HF}}(A+1)]-   E[\Phi^{\mathrm{HF}}(A)] =  \epsilon_a^{\mathrm{HF}}. 
\]
If  we use $^{16}\mbox{O}$ as our closed-shell nucleus, we could then interpret the separation energy
\[
BE(^{16}\mathrm{O})-BE(^{15}\mathrm{O})\approx \epsilon_{0p^{\nu}_{1/2}}^{\mathrm{HF}}, 
\]
and
\[
BE(^{16}\mathrm{O})-BE(^{15}\mathrm{N})\approx \epsilon_{0p^{\pi}_{1/2}}^{\mathrm{HF}}.
\]

Similalry, we could interpret
\[
BE(^{17}\mathrm{O})-BE(^{16}\mathrm{O})\approx \epsilon_{0d^{\nu}_{5/2}}^{\mathrm{HF}}, 
\]
and 
\[
BE(^{17}\mathrm{F})-BE(^{16}\mathrm{O})\approx\epsilon_{0d^{\pi}_{5/2}}^{\mathrm{HF}}.
\]
We can continue like this for all $A\pm 1$ nuclei where $A$ is a good closed-shell (or subshell closure)
nucleus. Examples are $^{22}\mbox{O}$, $^{24}\mbox{O}$, $^{40}\mbox{Ca}$, $^{48}\mbox{Ca}$, $^{52}\mbox{Ca}$, $^{54}\mbox{Ca}$, $^{56}\mbox{Ni}$, 
$^{68}\mbox{Ni}$, $^{78}\mbox{Ni}$, $^{90}\mbox{Zr}$, $^{88}\mbox{Sr}$, $^{100}\mbox{Sn}$, $^{132}\mbox{Sn}$ and $^{208}\mbox{Pb}$, to mention some possile cases.

We can thus make our first interpretation of the separation energies in terms of the simplest
possible many-body theory. 
If we also recall that the so-called energy gap for neutrons (or protons) is defined as
\[
\Delta S_n= 2BE(N,Z)-BE(N-1,Z)-BE(N+1,Z),
\]
for neutrons and the corresponding gap for protons
\[
\Delta S_p= 2BE(N,Z)-BE(N,Z-1)-BE(N,Z+1),
\]
we can define the neutron and proton energy gaps for $^{16}\mbox{O}$ as
\[
\Delta S_{\nu}=\epsilon_{0d^{\nu}_{5/2}}^{\mathrm{HF}}-\epsilon_{0p^{\nu}_{1/2}}^{\mathrm{HF}}, 
\]
and 
\[
\Delta S_{\pi}=\epsilon_{0d^{\pi}_{5/2}}^{\mathrm{HF}}-\epsilon_{0p^{\pi}_{1/2}}^{\mathrm{HF}}. 
\]

% --- begin exercise ---
\begin{doconceexercise}
\refstepcounter{doconceexercisecounter}

\exercisesection*{Exercise \thedoconceexercisecounter: Derivation of Hartree-Fock equations}
                             

Consider a Slater determinant built up of single-particle orbitals $\psi_{\lambda}$, 
with $\lambda = 1,2,\dots,N$.

The unitary transformation
\[
\psi_a  = \sum_{\lambda} C_{a\lambda}\phi_{\lambda},
\]
brings us into the new basis.  
The new basis has quantum numbers $a=1,2,\dots,N$.

% --- begin subexercise ---
\subex{a)}
Show that the new basis is orthonormal.

% --- end subexercise ---

% --- begin subexercise ---
\subex{b)}
Show that the new Slater determinant constructed from the new single-particle wave functions can be
written as the determinant based on the previous basis and the determinant of the matrix $C$.

% --- end subexercise ---

% --- begin subexercise ---
\subex{c)}
Show that the old and the new Slater determinants are equal up to a complex constant with absolute value unity.

% --- begin hint in exercise ---

\paragraph{Hint.}
Use the fact that $C$ is a unitary matrix.

% --- end hint in exercise ---

% --- end subexercise ---

\end{doconceexercise}
% --- end exercise ---

% --- begin exercise ---
\begin{doconceexercise}
\refstepcounter{doconceexercisecounter}

\exercisesection*{Exercise \thedoconceexercisecounter: Derivation of Hartree-Fock equations}
                             

Consider the  Slater  determinant
\[
\Phi_{0}=\frac{1}{\sqrt{n!}}\sum_{p}(-)^{p}P
\prod_{i=1}^{n}\psi_{\alpha_{i}}(x_{i}).
\]
A small variation in this function is given by
\[
\delta\Phi_{0}=\frac{1}{\sqrt{n!}}\sum_{p}(-)^{p}P
\psi_{\alpha_{1}}(x_{1})\psi_{\alpha_{2}}(x_{2})\dots
\psi_{\alpha_{i-1}}(x_{i-1})(\delta\psi_{\alpha_{i}}(x_{i}))
\psi_{\alpha_{i+1}}(x_{i+1})\dots\psi_{\alpha_{n}}(x_{n}).
\]

% --- begin subexercise ---
\subex{a)}
Show that
\[
\langle \delta\Phi_{0}|\sum_{i=1}^{n}\left\{t(x_{i})+u(x_{i})
\right\}+\frac{1}{2}
\sum_{i\neq j=1}^{n}v(x_{i},x_{j})|\Phi_{0}\rangle=\sum_{i=1}^{n}\langle \delta\psi_{\alpha_{i}}|\hat{t}+\hat{u}
|\phi_{\alpha_{i}}\rangle
+\sum_{i\neq j=1}^{n}\left\{\langle\delta\psi_{\alpha_{i}}
\psi_{\alpha_{j}}|\hat{v}|\psi_{\alpha_{i}}\psi_{\alpha_{j}}\rangle-
\langle\delta\psi_{\alpha_{i}}\psi_{\alpha_{j}}|\hat{v}
|\psi_{\alpha_{j}}\psi_{\alpha_{i}}\rangle\right\}
\]

% --- end subexercise ---

\end{doconceexercise}
% --- end exercise ---

% --- begin exercise ---
\begin{doconceexercise}
\refstepcounter{doconceexercisecounter}

\exercisesection*{Exercise \thedoconceexercisecounter: Developing a  Hartree-Fock program}
                             

Neutron drops are a powerful theoretical laboratory for testing,
validating and improving nuclear structure models. Indeed, all
approaches to nuclear structure, from ab initio theory to shell model
to density functional theory are applicable in such systems. We will,
therefore, use neutron drops as a test system for setting up a
Hartree-Fock code.  This program can later be extended to studies of
the binding energy of nuclei like $^{16}$O or $^{40}$Ca. The
single-particle energies obtained by solving the Hartree-Fock
equations can then be directly related to experimental separation
energies. 
Since Hartree-Fock theory is the starting point for
several many-body techniques (density functional theory, random-phase
approximation, shell-model etc), the aim here is to develop a computer
program to solve the Hartree-Fock equations in a given single-particle basis,
here the harmonic oscillator.

The Hamiltonian for a system of $N$ neutron drops confined in a
harmonic potential reads
\[
\hat{H} = \sum_{i=1}^{N} \frac{\hat{p}_{i}^{2}}{2m}+\sum_{i=1}^{N} \frac{1}{2} m\omega {r}_{i}^{2}+\sum_{i<j} \hat{V}_{ij},
\]
with $\hbar^{2}/2m = 20.73$ fm$^{2}$, $mc^{2} = 938.90590$ MeV, and 
$\hat{V}_{ij}$ is the two-body interaction potential whose 
matrix elements are precalculated
and to be read in by you.

The Hartree-Fock algorithm can be broken down as follows. We recall that  our Hartree-Fock matrix  is 
\[
\hat{h}_{\alpha\beta}^{HF}=\langle \alpha \vert\hat{h}_0 \vert \beta \rangle+
\sum_{j=1}^N\sum_{\gamma\delta} C^*_{j\gamma}C_{j\delta}\langle \alpha\gamma|V|\beta\delta\rangle_{AS}.
\]
Normally we assume that the single-particle basis $\vert\beta\rangle$
forms an eigenbasis for the operator $\hat{h}_0$ (this is our case), meaning that the
Hartree-Fock matrix becomes
\[
\hat{h}_{\alpha\beta}^{HF}=\epsilon_{\alpha}\delta_{\alpha,\beta}+
\sum_{j=1}^N\sum_{\gamma\delta} C^*_{j\gamma}C_{j\delta}\langle \alpha\gamma|V|\beta\delta\rangle_{AS}.
\]
The Hartree-Fock eigenvalue problem
\[
\sum_{\beta}\hat{h}_{\alpha\beta}^{HF}C_{i\beta}=\epsilon_i^{\mathrm{HF}}C_{i\alpha},
\]
can be written out in a more compact form as
\[
\hat{h}^{HF}\hat{C}=\epsilon^{\mathrm{HF}}\hat{C}. 
\]

The equations are often rewritten in terms of a so-called density matrix,
which is defined as 
\begin{equation}
\rho_{\gamma\delta}=\sum_{i=1}^{N}\langle\gamma|i\rangle\langle i|\delta\rangle = \sum_{i=1}^{N}C_{i\gamma}C^*_{i\delta}.
\end{equation}
It means that we can rewrite the Hartree-Fock Hamiltonian as
\[
\hat{h}_{\alpha\beta}^{HF}=\epsilon_{\alpha}\delta_{\alpha,\beta}+
\sum_{\gamma\delta} \rho_{\gamma\delta}\langle \alpha\gamma|V|\beta\delta\rangle_{AS}.
\]
It is convenient to use the density matrix since we can precalculate in every iteration the product of two eigenvector components $C$. 

Note that $\langle \alpha\vert\hat{h}_0\vert\beta \rangle$ denotes the
matrix elements of the one-body part of the starting hamiltonian. For
self-bound nuclei $\langle \alpha\vert\hat{h}_0\vert\beta \rangle$ is the
kinetic energy, whereas for neutron drops, $\langle \alpha \vert \hat{h}_0 \vert \beta \rangle$ represents the harmonic oscillator hamiltonian since
the system is confined in a harmonic trap. If we are working in a
harmonic oscillator basis with the same $\omega$ as the trapping
potential, then $\langle \alpha\vert\hat{h}_0 \vert \beta \rangle$ is
diagonal.

The python
\href{{https://github.com/CompPhysics/ManyBodyMethods/tree/master/doc/src/hfock/Code}}{program}
shows how one can, in a brute force way read in matrix elements in
$m$-scheme and compute the Hartree-Fock single-particle energies for
four major shells. The interaction which has been used is the
so-called N3LO interaction of \href{{http://journals.aps.org/prc/abstract/10.1103/PhysRevC.68.041001}}{Machleidt and
Entem}
using the \href{{http://journals.aps.org/prc/abstract/10.1103/PhysRevC.75.061001}}{Similarity Renormalization
Group}
approach method to renormalize the interaction, using an oscillator
energy $\hbar\omega=10$ MeV.

The nucleon-nucleon two-body matrix elements are in $m$-scheme and are fully anti-symmetrized. The Hartree-Fock programs uses the density matrix discussed above in order to compute the Hartree-Fock matrix.
Here we display the Hartree-Fock part only, assuming that single-particle data and two-body matrix elements have already been read in. 





































































































\begin{minted}[fontsize=\fontsize{9pt}{9pt},linenos=false,mathescape,baselinestretch=1.0,fontfamily=tt,xleftmargin=7mm]{python}
import numpy as np 
from decimal import Decimal
# expectation value for the one body part, Harmonic oscillator in three dimensions
def onebody(i, n, l):
        homega = 10.0
        return homega*(2*n[i] + l[i] + 1.5)

if __name__ == '__main__':
        
    Nparticles = 16
    """ Read quantum numbers from file """
    index = []
    n = []
    l = []
    j = []
    mj = []
    tz = []
    spOrbitals = 0
    with open("nucleispnumbers.dat", "r") as qnumfile:
                for line in qnumfile:
                        nums = line.split()
                        if len(nums) != 0:
                                index.append(int(nums[0]))
                                n.append(int(nums[1]))
                                l.append(int(nums[2]))
                                j.append(int(nums[3]))
                                mj.append(int(nums[4]))
                                tz.append(int(nums[5]))
                                spOrbitals += 1


    """ Read two-nucleon interaction elements (integrals) from file, brute force 4-dim array """
    nninteraction = np.zeros([spOrbitals, spOrbitals, spOrbitals, spOrbitals])
    with open("nucleitwobody.dat", "r") as infile:
        for line in infile:
                number = line.split()
                a = int(number[0]) - 1
                b = int(number[1]) - 1
                c = int(number[2]) - 1
                d = int(number[3]) - 1
                nninteraction[a][b][c][d] = Decimal(number[4])
        """ Set up single-particle integral """
        singleparticleH = np.zeros(spOrbitals)
        for i in range(spOrbitals):
                singleparticleH[i] = Decimal(onebody(i, n, l))
        
        """ Star HF-iterations, preparing variables and density matrix """

        """ Coefficients for setting up density matrix, assuming only one along the diagonals """
        C = np.eye(spOrbitals) # HF coefficients
        DensityMatrix = np.zeros([spOrbitals,spOrbitals])
        for gamma in range(spOrbitals):
            for delta in range(spOrbitals):
                sum = 0.0
                for i in range(Nparticles):
                    sum += C[gamma][i]*C[delta][i]
                DensityMatrix[gamma][delta] = Decimal(sum)
        maxHFiter = 100
        epsilon =  1.0e-5 
        difference = 1.0
        hf_count = 0
        oldenergies = np.zeros(spOrbitals)
        newenergies = np.zeros(spOrbitals)
        while hf_count < maxHFiter and difference > epsilon:
                print("############### Iteration %i ###############" % hf_count)
                HFmatrix = np.zeros([spOrbitals,spOrbitals])            
                for alpha in range(spOrbitals):
                        for beta in range(spOrbitals):
                            """  If tests for three-dimensional systems, including isospin conservation """
                            if l[alpha] != l[beta] and j[alpha] != j[beta] and mj[alpha] != mj[beta] and tz[alpha] != tz[beta]: continue
                            """  Setting up the Fock matrix using the density matrix and antisymmetrized NN interaction in m-scheme """
                            sumFockTerm = 0.0
                            for gamma in range(spOrbitals):
                                for delta in range(spOrbitals):
                                    if (mj[alpha]+mj[gamma]) != (mj[beta]+mj[delta]) and (tz[alpha]+tz[gamma]) != (tz[beta]+tz[delta]): continue
                                    sumFockTerm += DensityMatrix[gamma][delta]*nninteraction[alpha][gamma][beta][delta]
                            HFmatrix[alpha][beta] = Decimal(sumFockTerm)
                            """  Adding the one-body term, here plain harmonic oscillator """
                            if beta == alpha:   HFmatrix[alpha][alpha] += singleparticleH[alpha]
                spenergies, C = np.linalg.eigh(HFmatrix)
                """ Setting up new density matrix in m-scheme """
                DensityMatrix = np.zeros([spOrbitals,spOrbitals])
                for gamma in range(spOrbitals):
                    for delta in range(spOrbitals):
                        sum = 0.0
                        for i in range(Nparticles):
                            sum += C[gamma][i]*C[delta][i]
                        DensityMatrix[gamma][delta] = Decimal(sum)
                newenergies = spenergies
                """ Brute force computation of difference between previous and new sp HF energies """
                sum =0.0
                for i in range(spOrbitals):
                    sum += (abs(newenergies[i]-oldenergies[i]))/spOrbitals
                difference = sum
                oldenergies = newenergies
                print ("Single-particle energies, ordering may have changed ")
                for i in range(spOrbitals):
                    print('{0:4d}  {1:.4f}'.format(i, Decimal(oldenergies[i])))
                hf_count += 1


\end{minted}

Running the program, one finds that the lowest-lying states for a nucleus like $^{16}\mbox{O}$, we see that the nucleon-nucleon force brings a natural spin-orbit splitting for the $0p$ states (or other states except the $s$-states).
Since we are using the $m$-scheme for our calculations, we observe that there are several states with the same
eigenvalues. The number of eigenvalues corresponds to the degeneracy $2j+1$ and is well respected in our calculations, as see from the table here.

The values of the lowest-lying states are ($\pi$ for protons and $\nu$ for neutrons)

\begin{quote}
\begin{tabular}{cc}
\hline
\multicolumn{1}{c}{ Quantum numbers } & \multicolumn{1}{c}{ Energy [MeV] } \\
\hline
$0s_{1/2}^{\pi}$ & -40.4602     \\
$0s_{1/2}^{\pi}$ & -40.4602     \\
$0s_{1/2}^{\nu}$ & -40.6426     \\
$0s_{1/2}^{\nu}$ & -40.6426     \\
$0p_{1/2}^{\pi}$ & -6.7133      \\
$0p_{1/2}^{\pi}$ & -6.7133      \\
$0p_{1/2}^{\nu}$ & -6.8403      \\
$0p_{1/2}^{\nu}$ & -6.8403      \\
$0p_{3/2}^{\pi}$ & -11.5886     \\
$0p_{3/2}^{\pi}$ & -11.5886     \\
$0p_{3/2}^{\pi}$ & -11.5886     \\
$0p_{3/2}^{\pi}$ & -11.5886     \\
$0p_{3/2}^{\nu}$ & -11.7201     \\
$0p_{3/2}^{\nu}$ & -11.7201     \\
$0p_{3/2}^{\nu}$ & -11.7201     \\
$0p_{3/2}^{\nu}$ & -11.7201     \\
$0d_{5/2}^{\pi}$ & 18.7589      \\
$0d_{5/2}^{\nu}$ & 18.8082      \\
\hline
\end{tabular}
\end{quote}

\noindent
We can use these results to attempt our first link with experimental data, namely to compute the shell gap or the separation energies. The shell gap for neutrons is given by
\[
\Delta S_n= 2BE(N,Z)-BE(N-1,Z)-BE(N+1,Z).
\]
For $^{16}\mbox{O}$  we have an experimental value for the  shell gap of $11.51$ MeV for neutrons, while our Hartree-Fock calculations result in $25.65$ MeV. This means that correlations beyond a simple Hartree-Fock calculation with a two-body force play an important role in nuclear physics.
The splitting between the $0p_{3/2}^{\nu}$ and the $0p_{1/2}^{\nu}$ state is 4.88 MeV, while the experimental value for the gap between the ground state $1/2^{-}$ and the first excited $3/2^{-}$ states is 6.08 MeV. The two-nucleon spin-orbit force plays a central role here. In our discussion of nuclear forces we will see how the spin-orbit force comes into play here.

\end{doconceexercise}
% --- end exercise ---

\subsection*{Hartree-Fock in second quantization and stability of HF solution}

We wish now to derive the Hartree-Fock equations using our second-quantized formalism and study the stability of the equations. 
Our ansatz for the ground state of the system is approximated as (this is our representation of a Slater determinant in second quantization)
\[   
|\Phi_0\rangle = |c\rangle = a^{\dagger}_i a^{\dagger}_j \dots a^{\dagger}_l|0\rangle.
\]
We wish to determine $\hat{u}^{HF}$ so that 
$E_0^{HF}= \langle c|\hat{H}| c\rangle$ becomes a local minimum. 

In our analysis here we will need Thouless' theorem, which states that
an arbitrary Slater determinant $|c'\rangle$ which is not orthogonal to a determinant
$| c\rangle ={\displaystyle\prod_{i=1}^{n}}
a_{\alpha_{i}}^{\dagger}|0\rangle$, can be written as
\[
|c'\rangle=exp\left\{\sum_{a>F}\sum_{i\le F}C_{ai}a_{a}^{\dagger}a_{i}\right\}| c\rangle 
\]

Let us give a simple proof of Thouless' theorem. The theorem states that we can make a linear combination av particle-hole excitations  with respect to a given reference state $\vert c\rangle$. With this linear combination, we can make a new Slater determinant $\vert c'\rangle $ which is not orthogonal to 
$\vert c\rangle$, that is
\[
\langle c|c'\rangle \ne 0.
\] 
To show this we need some intermediate steps. The exponential product of two operators  $\exp{\hat{A}}\times\exp{\hat{B}}$ is equal to $\exp{(\hat{A}+\hat{B})}$ only if the two operators commute, that is
\[
[\hat{A},\hat{B}] = 0.
\]

\subsection*{Thouless' theorem}

If the operators do not commute, we need to resort to the \href{{http://www.encyclopediaofmath.org/index.php/Campbell%E2%80%93Hausdorff_formula}}{Baker-Campbell-Hauersdorf}. This relation states that
\[
\exp{\hat{C}}=\exp{\hat{A}}\exp{\hat{B}},
\]
with 
\[
\hat{C}=\hat{A}+\hat{B}+\frac{1}{2}[\hat{A},\hat{B}]+\frac{1}{12}[[\hat{A},\hat{B}],\hat{B}]-\frac{1}{12}[[\hat{A},\hat{B}],\hat{A}]+\dots
\]
From these relations, we note that 
in our expression  for $|c'\rangle$ we have commutators of the type
\[
[a_{a}^{\dagger}a_{i},a_{b}^{\dagger}a_{j}],
\]
and it is easy to convince oneself that these commutators, or higher powers thereof, are all zero. This means that we can write out our new representation of a Slater determinant as
\[
|c'\rangle=exp\left\{\sum_{a>F}\sum_{i\le F}C_{ai}a_{a}^{\dagger}a_{i}\right\}| c\rangle=\prod_{i}\left\{1+\sum_{a>F}C_{ai}a_{a}^{\dagger}a_{i}+\left(\sum_{a>F}C_{ai}a_{a}^{\dagger}a_{i}\right)^2+\dots\right\}| c\rangle
\]

We note that
\[
\prod_{i}\sum_{a>F}C_{ai}a_{a}^{\dagger}a_{i}\sum_{b>F}C_{bi}a_{b}^{\dagger}a_{i}| c\rangle =0,
\]
and all higher-order powers of these combinations of creation and annihilation operators disappear 
due to the fact that $(a_i)^n| c\rangle =0$ when $n > 1$. This allows us to rewrite the expression for $|c'\rangle $ as
\[
|c'\rangle=\prod_{i}\left\{1+\sum_{a>F}C_{ai}a_{a}^{\dagger}a_{i}\right\}| c\rangle,
\]
which we can rewrite as 
\[
|c'\rangle=\prod_{i}\left\{1+\sum_{a>F}C_{ai}a_{a}^{\dagger}a_{i}\right\}| a^{\dagger}_{i_1} a^{\dagger}_{i_2} \dots a^{\dagger}_{i_n}|0\rangle.
\]
The last equation can be written as
\begin{align}
|c'\rangle&=\prod_{i}\left\{1+\sum_{a>F}C_{ai}a_{a}^{\dagger}a_{i}\right\}| a^{\dagger}_{i_1} a^{\dagger}_{i_2} \dots a^{\dagger}_{i_n}|0\rangle=\left(1+\sum_{a>F}C_{ai_1}a_{a}^{\dagger}a_{i_1}\right)a^{\dagger}_{i_1} \\
& \times\left(1+\sum_{a>F}C_{ai_2}a_{a}^{\dagger}a_{i_2}\right)a^{\dagger}_{i_2} \dots |0\rangle=\prod_{i}\left(a^{\dagger}_{i}+\sum_{a>F}C_{ai}a_{a}^{\dagger}\right)|0\rangle.
\end{align}

\subsection*{New operators}

If we define a new creation operator 
\begin{equation}
b^{\dagger}_{i}=a^{\dagger}_{i}+\sum_{a>F}C_{ai}a_{a}^{\dagger}, \label{eq:newb}
\end{equation}
we have 
\[
|c'\rangle=\prod_{i}b^{\dagger}_{i}|0\rangle=\prod_{i}\left(a^{\dagger}_{i}+\sum_{a>F}C_{ai}a_{a}^{\dagger}\right)|0\rangle,
\]
meaning that the new representation of the Slater determinant in second quantization, $|c'\rangle$, looks like our previous ones. However, this representation is not general enough since we have a restriction on the sum over single-particle states in Eq.~(\ref{eq:newb}). The single-particle states have all to be above the Fermi level.
The question then is whether we can construct a general representation of a Slater determinant with a creation operator 
\[
\tilde{b}^{\dagger}_{i}=\sum_{p}f_{ip}a_{p}^{\dagger},
\]
where $f_{ip}$ is a matrix element of a unitary matrix which transforms our creation and annihilation operators
$a^{\dagger}$ and $a$ to $\tilde{b}^{\dagger}$ and $\tilde{b}$. These new operators define a new representation of a Slater determinant as
\[
|\tilde{c}\rangle=\prod_{i}\tilde{b}^{\dagger}_{i}|0\rangle.
\]

\subsection*{Showing that $|\tilde{c}\rangle= |c'\rangle$}

We need to show that $|\tilde{c}\rangle= |c'\rangle$. We need also to assume that the new state
is not orthogonal to $|c\rangle$, that is $\langle c| \tilde{c}\rangle \ne 0$. From this it follows that 
\[
\langle c| \tilde{c}\rangle=\langle 0| a_{i_n}\dots a_{i_1}\left(\sum_{p=i_1}^{i_n}f_{i_1p}a_{p}^{\dagger} \right)\left(\sum_{q=i_1}^{i_n}f_{i_2q}a_{q}^{\dagger} \right)\dots \left(\sum_{t=i_1}^{i_n}f_{i_nt}a_{t}^{\dagger} \right)|0\rangle,
\]
which is nothing but the determinant $det(f_{ip})$ which we can, using the intermediate normalization condition, 
normalize to one, that is
\[
det(f_{ip})=1,
\]
meaning that $f$ has an inverse defined as (since we are dealing with orthogonal, and in our case unitary as well, transformations)
\[
\sum_{k} f_{ik}f^{-1}_{kj} = \delta_{ij},
\]
and 
\[
\sum_{j} f^{-1}_{ij}f_{jk} = \delta_{ik}.
\]

Using these relations we can then define the linear combination of creation (and annihilation as well) 
operators as
\[
\sum_{i}f^{-1}_{ki}\tilde{b}^{\dagger}_{i}=\sum_{i}f^{-1}_{ki}\sum_{p=i_1}^{\infty}f_{ip}a_{p}^{\dagger}=a_{k}^{\dagger}+\sum_{i}\sum_{p=i_{n+1}}^{\infty}f^{-1}_{ki}f_{ip}a_{p}^{\dagger}.
\]
Defining 
\[
c_{kp}=\sum_{i \le F}f^{-1}_{ki}f_{ip},
\]
we can redefine 
\[
a_{k}^{\dagger}+\sum_{i}\sum_{p=i_{n+1}}^{\infty}f^{-1}_{ki}f_{ip}a_{p}^{\dagger}=a_{k}^{\dagger}+\sum_{p=i_{n+1}}^{\infty}c_{kp}a_{p}^{\dagger}=b_k^{\dagger},
\]
our starting point. We have shown that our general representation of a Slater determinant 
\[
|\tilde{c}\rangle=\prod_{i}\tilde{b}^{\dagger}_{i}|0\rangle=|c'\rangle=\prod_{i}b^{\dagger}_{i}|0\rangle,
\]
with 
\[
b_k^{\dagger}=a_{k}^{\dagger}+\sum_{p=i_{n+1}}^{\infty}c_{kp}a_{p}^{\dagger}.
\]

This means that we can actually write an ansatz for the ground state of the system as a linear combination of
terms which contain the ansatz itself $|c\rangle$ with  an admixture from an infinity of one-particle-one-hole states. The latter has important consequences when we wish to interpret the Hartree-Fock equations and their stability. We can rewrite the new representation as 
\[
|c'\rangle = |c\rangle+|\delta c\rangle,
\]
where $|\delta c\rangle$ can now be interpreted as a small variation. If we approximate this term with 
contributions from one-particle-one-hole (\emph{1p-1h}) states only, we arrive at 
\[
|c'\rangle = \left(1+\sum_{ai}\delta C_{ai}a_{a}^{\dagger}a_i\right)|c\rangle.
\]
In our derivation of the Hartree-Fock equations we have shown that 
\[
\langle \delta c| \hat{H} | c\rangle =0,
\]
which means that we have to satisfy
\[
\langle c|\sum_{ai}\delta C_{ai}\left\{a_{a}^{\dagger}a_i\right\} \hat{H} | c\rangle =0.
\]
With this as a background, we are now ready to study the stability of the Hartree-Fock equations.

\subsection*{Hartree-Fock in second quantization and stability of HF solution}

The variational condition for deriving the Hartree-Fock equations guarantees only that the expectation value $\langle c | \hat{H} | c \rangle$ has an extreme value, not necessarily a minimum. To figure out whether the extreme value we have found  is a minimum, we can use second quantization to analyze our results and find a criterion 
for the above expectation value to a local minimum. We will use Thouless' theorem and show that
\[
\frac{\langle c' |\hat{H} | c'\rangle}{\langle c' |c'\rangle} \ge \langle c |\hat{H} | c\rangle= E_0,
\]
with
\[
 {|c'\rangle} = {|c\rangle + |\delta c\rangle}.
\]
Using Thouless' theorem we can write out $|c'\rangle$ as
\begin{align}
 {|c'\rangle}&=\exp\left\{\sum_{a > F}\sum_{i \le F}\delta C_{ai}a_{a}^{\dagger}a_{i}\right\}| c\rangle\\ 
&=\left\{1+\sum_{a > F}\sum_{i \le F}\delta C_{ai}a_{a}^{\dagger}
a_{i}+\frac{1}{2!}\sum_{ab > F}\sum_{ij \le F}\delta C_{ai}\delta C_{bj}a_{a}^{\dagger}a_{i}a_{b}^{\dagger}a_{j}+\dots\right\}
\end{align}
where the amplitudes $\delta C$ are small.

The norm of $|c'\rangle$ is given by (using the intermediate normalization condition $\langle c' |c\rangle=1$) 
\[
\langle c' | c'\rangle = 1+\sum_{a>F}
\sum_{i\le F}|\delta C_{ai}|^2+O(\delta C_{ai}^3).
\]
The expectation value for the energy is now given by (using the Hartree-Fock condition)
\[
\langle c' |\hat{H} | c'\rangle=\langle c |\hat{H} | c\rangle +
\sum_{ab>F}
\sum_{ij\le F}\delta C_{ai}^*\delta C_{bj}\langle c |a_{i}^{\dagger}a_{a}\hat{H}a_{b}^{\dagger}a_{j}|c\rangle+
\]
\[
\frac{1}{2!}\sum_{ab>F}
\sum_{ij\le F}\delta C_{ai}\delta C_{bj}\langle c |\hat{H}a_{a}^{\dagger}a_{i}a_{b}^{\dagger}a_{j}|c\rangle+\frac{1}{2!}\sum_{ab>F}
\sum_{ij\le F}\delta C_{ai}^*\delta C_{bj}^*\langle c|a_{j}^{\dagger}a_{b}a_{i}^{\dagger}a_{a}\hat{H}|c\rangle
+\dots
\] 

We have already calculated the second term on the right-hand side of the previous equation
\begin{align}
\langle c | \left(\{a^\dagger_i a_a\} \hat{H} \{a^\dagger_b a_j\} \right) | c\rangle&=\sum_{pq} \sum_{ijab}\delta C_{ai}^*\delta C_{bj} \langle p|\hat{h}_0 |q\rangle 
            \langle c | \left(\{a^{\dagger}_i a_a\}\{a^{\dagger}_pa_q\} 
             \{a^{\dagger}_b a_j\} \right)| c\rangle\\
& +\frac{1}{4} \sum_{pqrs} \sum_{ijab}\delta C_{ai}^*\delta C_{bj} \langle pq| \hat{v}|rs\rangle 
            \langle c | \left(\{a^\dagger_i a_a\}\{a^{\dagger}_p a^{\dagger}_q a_s  a_r\} \{a^{\dagger}_b a_j\} \right)| c\rangle ,
\end{align}
resulting in
\[
E_0\sum_{ai}|\delta C_{ai}|^2+\sum_{ai}|\delta C_{ai}|^2(\varepsilon_a-\varepsilon_i)-\sum_{ijab} \langle aj|\hat{v}| bi\rangle \delta C_{ai}^*\delta C_{bj}.
\]

\[
\frac{1}{2!}\langle c |\left(\{a^\dagger_j a_b\} \{a^\dagger_i a_a\} \hat{V}_N  \right) | c\rangle  = 
\frac{1}{2!}\langle c |\left( \hat{V}_N \{a^\dagger_a a_i\} \{a^\dagger_b a_j\} \right)^{\dagger} | c\rangle 
\]
which is nothing but
\[
\frac{1}{2!}\langle c |  \left( \hat{V}_N \{a^\dagger_a a_i\} \{a^\dagger_b a_j\} \right) | c\rangle^*
=\frac{1}{2} \sum_{ijab} (\langle ij|\hat{v}|ab\rangle)^*\delta C_{ai}^*\delta C_{bj}^*
\]
or 
\[
\frac{1}{2} \sum_{ijab} (\langle ab|\hat{v}|ij\rangle)\delta C_{ai}^*\delta C_{bj}^*
\]
where we have used the relation
\[ 
\langle a |\hat{A} | b\rangle =  (\langle b |\hat{A}^{\dagger} | a\rangle)^*
\]
due to the hermiticity of $\hat{H}$ and $\hat{V}$.

We define two matrix elements
\[
A_{ai,bj}=-\langle aj|\hat{v} bi\rangle
\]
and
\[
B_{ai,bj}=\langle ab|\hat{v}|ij\rangle
\]
both being anti-symmetrized.

With these definitions we write out the energy as
\begin{align}
\langle c'|H|c'\rangle& = \left(1+\sum_{ai}|\delta C_{ai}|^2\right)\langle c |H|c\rangle+\sum_{ai}|\delta C_{ai}|^2(\varepsilon_a^{HF}-\varepsilon_i^{HF})+\sum_{ijab}A_{ai,bj}\delta C_{ai}^*\delta C_{bj}+\\
&\frac{1}{2} \sum_{ijab} B_{ai,bj}^*\delta C_{ai}\delta C_{bj}+\frac{1}{2} \sum_{ijab} B_{ai,bj}\delta C_{ai}^*\delta C_{bj}^*
+O(\delta C_{ai}^3),
\end{align}
which can be rewritten as
\[
\langle c'|H|c'\rangle = \left(1+\sum_{ai}|\delta C_{ai}|^2\right)\langle c |H|c\rangle+\Delta E+O(\delta C_{ai}^3),
\]
and skipping higher-order terms we arrived
\[
\frac{\langle c' |\hat{H} | c'\rangle}{\langle c' |c'\rangle} =E_0+\frac{\Delta E}{\left(1+\sum_{ai}|\delta C_{ai}|^2\right)}.
\]

We have defined 
\[
\Delta E = \frac{1}{2} \langle \chi | \hat{M}| \chi \rangle
\]
with the vectors 
\[ 
\chi = \left[ \delta C\hspace{0.2cm} \delta C^*\right]^T
\]
and the matrix 
\[
\hat{M}=\left(\begin{array}{cc} \Delta + A & B \\ B^* & \Delta + A^*\end{array}\right),
\]
with $\Delta_{ai,bj} = (\varepsilon_a-\varepsilon_i)\delta_{ab}\delta_{ij}$.

The condition
\[
\Delta E = \frac{1}{2} \langle \chi | \hat{M}| \chi \rangle \ge 0
\]
for an arbitrary  vector 
\[ 
\chi = \left[ \delta C\hspace{0.2cm} \delta C^*\right]^T
\]
means that all eigenvalues of the matrix have to be larger than or equal zero. 
A necessary (but no sufficient) condition is that the matrix elements (for all $ai$ )
\[
(\varepsilon_a-\varepsilon_i)\delta_{ab}\delta_{ij}+A_{ai,bj} \ge 0.
\]
This equation can be used as a first test of the stability of the Hartree-Fock equation.


\clearemptydoublepage

\chapter{Many-body perturbation theory}

We assume here that we are only interested in the ground state of the system and 
expand the exact wave function in term of a series of Slater determinants
\[
\vert \Psi_0\rangle = \vert \Phi_0\rangle + \sum_{m=1}^{\infty}C_m\vert \Phi_m\rangle,
\]
where we have assumed that the true ground state is dominated by the 
solution of the unperturbed problem, that is
\[
\hat{H}_0\vert \Phi_0\rangle= W_0\vert \Phi_0\rangle.
\]
The state $\vert \Psi_0\rangle$ is not normalized, rather we have used an intermediate 
normalization $\langle \Phi_0 \vert \Psi_0\rangle=1$ since we have $\langle \Phi_0\vert \Phi_0\rangle=1$. 

The Schroedinger equation is
\[
\hat{H}\vert \Psi_0\rangle = E\vert \Psi_0\rangle,
\]
and multiplying the latter from the left with $\langle \Phi_0\vert $ gives
\[
\langle \Phi_0\vert \hat{H}\vert \Psi_0\rangle = E\langle \Phi_0\vert \Psi_0\rangle=E,
\]
and subtracting from this equation
\[
\langle \Psi_0\vert \hat{H}_0\vert \Phi_0\rangle= W_0\langle \Psi_0\vert \Phi_0\rangle=W_0,
\]
and using the fact that the both operators $\hat{H}$ and $\hat{H}_0$ are hermitian 
results in
\[
\Delta E=E-W_0=\langle \Phi_0\vert \hat{H}_I\vert \Psi_0\rangle,
\]
which is an exact result. We call this quantity the correlation energy.

This equation forms the starting point for all perturbative derivations. However,
as it stands it represents nothing but a mere formal rewriting of Schroedinger's equation and is not of much practical use. The exact wave function $\vert \Psi_0\rangle$ is unknown. In order to obtain a perturbative expansion, we need to expand the exact wave function in terms of the interaction $\hat{H}_I$. 

Here we have assumed that our model space defined by the operator $\hat{P}$ is one-dimensional, meaning that
\[
\hat{P}= \vert \Phi_0\rangle \langle \Phi_0\vert ,
\]
and
\[
\hat{Q}=\sum_{m=1}^{\infty}\vert \Phi_m\rangle \langle \Phi_m\vert .
\]

We can thus rewrite the exact wave function as
\[
\vert \Psi_0\rangle= (\hat{P}+\hat{Q})\vert \Psi_0\rangle=\vert \Phi_0\rangle+\hat{Q}\vert \Psi_0\rangle.
\]
Going back to the Schr\"odinger equation, we can rewrite it as, adding and a subtracting a term $\omega \vert \Psi_0\rangle$ as
\[
\left(\omega-\hat{H}_0\right)\vert \Psi_0\rangle=\left(\omega-E+\hat{H}_I\right)\vert \Psi_0\rangle,
\]
where $\omega$ is an energy variable to be specified later. 

We assume also that the resolvent of $\left(\omega-\hat{H}_0\right)$ exits, that is
it has an inverse which defined the unperturbed Green's function as
\[
\left(\omega-\hat{H}_0\right)^{-1}=\frac{1}{\left(\omega-\hat{H}_0\right)}.
\]

We can rewrite Schroedinger's equation as
\[
\vert \Psi_0\rangle=\frac{1}{\omega-\hat{H}_0}\left(\omega-E+\hat{H}_I\right)\vert \Psi_0\rangle,
\]
and multiplying from the left with $\hat{Q}$ results in
\[
\hat{Q}\vert \Psi_0\rangle=\frac{\hat{Q}}{\omega-\hat{H}_0}\left(\omega-E+\hat{H}_I\right)\vert \Psi_0\rangle,
\]
which is possible since we have defined the operator $\hat{Q}$ in terms of the eigenfunctions of $\hat{H}$.

These operators commute meaning that
\[
\hat{Q}\frac{1}{\left(\omega-\hat{H}_0\right)}\hat{Q}=\hat{Q}\frac{1}{\left(\omega-\hat{H}_0\right)}=\frac{\hat{Q}}{\left(\omega-\hat{H}_0\right)}.
\]
With these definitions we can in turn define the wave function as 
\[
\vert \Psi_0\rangle=\vert \Phi_0\rangle+\frac{\hat{Q}}{\omega-\hat{H}_0}\left(\omega-E+\hat{H}_I\right)\vert \Psi_0\rangle.
\]
This equation is again nothing but a formal rewrite of Schr\"odinger's equation
and does not represent a practical calculational scheme.  
It is a non-linear equation in two unknown quantities, the energy $E$ and the exact
wave function $\vert \Psi_0\rangle$. We can however start with a guess for $\vert \Psi_0\rangle$ on the right hand side of the last equation.

 The most common choice is to start with the function which is expected to exhibit the largest overlap with the wave function we are searching after, namely $\vert \Phi_0\rangle$. This can again be inserted in the solution for $\vert \Psi_0\rangle$ in an iterative fashion and if we continue along these lines we end up with
\[
\vert \Psi_0\rangle=\sum_{i=0}^{\infty}\left\{\frac{\hat{Q}}{\omega-\hat{H}_0}\left(\omega-E+\hat{H}_I\right)\right\}^i\vert \Phi_0\rangle, 
\]
for the wave function and
\[
\Delta E=\sum_{i=0}^{\infty}\langle \Phi_0\vert \hat{H}_I\left\{\frac{\hat{Q}}{\omega-\hat{H}_0}\left(\omega-E+\hat{H}_I\right)\right\}^i\vert \Phi_0\rangle, 
\]
which is now  a perturbative expansion of the exact energy in terms of the interaction
$\hat{H}_I$ and the unperturbed wave function $\vert \Psi_0\rangle$.

In our equations for $\vert \Psi_0\rangle$ and $\Delta E$ in terms of the unperturbed
solutions $\vert \Phi_i\rangle$  we have still an undetermined parameter $\omega$
and a dependecy on the exact energy $E$. Not much has been gained thus from a practical computational point of view. 

In Brilluoin-Wigner perturbation theory it is customary to set $\omega=E$. This results in the following perturbative expansion for the energy $\Delta E$
\[
\Delta E=\sum_{i=0}^{\infty}\langle \Phi_0\vert \hat{H}_I\left\{\frac{\hat{Q}}{\omega-\hat{H}_0}\left(\omega-E+\hat{H}_I\right)\right\}^i\vert \Phi_0\rangle=
\]
\[
\langle \Phi_0\vert \left(\hat{H}_I+\hat{H}_I\frac{\hat{Q}}{E-\hat{H}_0}\hat{H}_I+
\hat{H}_I\frac{\hat{Q}}{E-\hat{H}_0}\hat{H}_I\frac{\hat{Q}}{E-\hat{H}_0}\hat{H}_I+\dots\right)\vert \Phi_0\rangle. 
\]

\[
\Delta E=\sum_{i=0}^{\infty}\langle \Phi_0\vert \hat{H}_I\left\{\frac{\hat{Q}}{\omega-\hat{H}_0}\left(\omega-E+\hat{H}_I\right)\right\}^i\vert \Phi_0\rangle=\]
\[
\langle \Phi_0\vert \left(\hat{H}_I+\hat{H}_I\frac{\hat{Q}}{E-\hat{H}_0}\hat{H}_I+
\hat{H}_I\frac{\hat{Q}}{E-\hat{H}_0}\hat{H}_I\frac{\hat{Q}}{E-\hat{H}_0}\hat{H}_I+\dots\right)\vert \Phi_0\rangle. 
\]
This expression depends however on the exact energy $E$ and is again not very convenient from a practical point of view. It can obviously be solved iteratively, by starting with a guess for  $E$ and then solve till some kind of self-consistency criterion has been reached. 

Actually, the above expression is nothing but a rewrite again of the full Schr\"odinger equation. 

Defining $e=E-\hat{H}_0$ and recalling that $\hat{H}_0$ commutes with 
$\hat{Q}$ by construction and that $\hat{Q}$ is an idempotent operator
$\hat{Q}^2=\hat{Q}$. 
Using this equation in the above expansion for $\Delta E$ we can write the denominator 
\[
\hat{Q}\frac{1}{\hat{e}-\hat{Q}\hat{H}_I\hat{Q}}=
\]
\[
\hat{Q}\left[\frac{1}{\hat{e}}+\frac{1}{\hat{e}}\hat{Q}\hat{H}_I\hat{Q}
\frac{1}{\hat{e}}+\frac{1}{\hat{e}}\hat{Q}\hat{H}_I\hat{Q}
\frac{1}{\hat{e}}\hat{Q}\hat{H}_I\hat{Q}\frac{1}{\hat{e}}+\dots\right]\hat{Q}.
\]

Inserted in the expression for $\Delta E$ leads to 
\[
\Delta E=
\langle \Phi_0\vert \hat{H}_I+\hat{H}_I\hat{Q}\frac{1}{E-\hat{H}_0-\hat{Q}\hat{H}_I\hat{Q}}\hat{Q}\hat{H}_I\vert \Phi_0\rangle. 
\]
In RS perturbation theory we set $\omega = W_0$ and obtain the following expression for the energy difference
\[
\Delta E=\sum_{i=0}^{\infty}\langle \Phi_0\vert \hat{H}_I\left\{\frac{\hat{Q}}{W_0-\hat{H}_0}\left(\hat{H}_I-\Delta E\right)\right\}^i\vert \Phi_0\rangle=
\]
\[
\langle \Phi_0\vert \left(\hat{H}_I+\hat{H}_I\frac{\hat{Q}}{W_0-\hat{H}_0}(\hat{H}_I-\Delta E)+
\hat{H}_I\frac{\hat{Q}}{W_0-\hat{H}_0}(\hat{H}_I-\Delta E)\frac{\hat{Q}}{W_0-\hat{H}_0}(\hat{H}_I-\Delta E)+\dots\right)\vert \Phi_0\rangle.
\]

Recalling that $\hat{Q}$ commutes with $\hat{H_0}$ and since $\Delta E$ is a constant we obtain that
\[
\hat{Q}\Delta E\vert \Phi_0\rangle = \hat{Q}\Delta E\vert \hat{Q}\Phi_0\rangle = 0.
\]
Inserting this results in the expression for the energy results in
\[
\Delta E=\langle \Phi_0\vert \left(\hat{H}_I+\hat{H}_I\frac{\hat{Q}}{W_0-\hat{H}_0}\hat{H}_I+
\hat{H}_I\frac{\hat{Q}}{W_0-\hat{H}_0}(\hat{H}_I-\Delta E)\frac{\hat{Q}}{W_0-\hat{H}_0}\hat{H}_I+\dots\right)\vert \Phi_0\rangle.
\]

We can now this expression in terms of a perturbative expression in terms
of $\hat{H}_I$ where we iterate the last expression in terms of $\Delta E$
\[
\Delta E=\sum_{i=1}^{\infty}\Delta E^{(i)}.
\]
We get the following expression for $\Delta E^{(i)}$
\[
\Delta E^{(1)}=\langle \Phi_0\vert \hat{H}_I\vert \Phi_0\rangle,
\] 
which is just the contribution to first order in perturbation theory,
\[
\Delta E^{(2)}=\langle\Phi_0\vert \hat{H}_I\frac{\hat{Q}}{W_0-\hat{H}_0}\hat{H}_I\vert \Phi_0\rangle, 
\]
which is the contribution to second order.

\[
\Delta E^{(3)}=\langle \Phi_0\vert \hat{H}_I\frac{\hat{Q}}{W_0-\hat{H}_0}\hat{H}_I\frac{\hat{Q}}{W_0-\hat{H}_0}\hat{H}_I\Phi_0\rangle-
\langle\Phi_0\vert \hat{H}_I\frac{\hat{Q}}{W_0-\hat{H}_0}\langle \Phi_0\vert \hat{H}_I\vert \Phi_0\rangle\frac{\hat{Q}}{W_0-\hat{H}_0}\hat{H}_I\vert \Phi_0\rangle,
\]
being the third-order contribution. 

\subsection*{Interpreting the correlation energy and the wave operator}

In the shell-model lectures we showed that we could rewrite the exact state function for say the ground state, as a linear expansion in terms of all possible Slater determinants. That is, we 
define the ansatz for the ground state as 
\[
|\Phi_0\rangle = \left(\prod_{i\le F}\hat{a}_{i}^{\dagger}\right)|0\rangle,
\]
where the index $i$ defines different single-particle states up to the Fermi level. We have assumed that we have $N$ fermions. 
A given one-particle-one-hole ($1p1h$) state can be written as
\[
|\Phi_i^a\rangle = \hat{a}_{a}^{\dagger}\hat{a}_i|\Phi_0\rangle,
\]
while a $2p2h$ state can be written as
\[
|\Phi_{ij}^{ab}\rangle = \hat{a}_{a}^{\dagger}\hat{a}_{b}^{\dagger}\hat{a}_j\hat{a}_i|\Phi_0\rangle,
\]
and a general $ApAh$ state as 
\[
|\Phi_{ijk\dots}^{abc\dots}\rangle = \hat{a}_{a}^{\dagger}\hat{a}_{b}^{\dagger}\hat{a}_{c}^{\dagger}\dots\hat{a}_k\hat{a}_j\hat{a}_i|\Phi_0\rangle.
\]

We use letters $ijkl\dots$ for states below the Fermi level and $abcd\dots$ for states above the Fermi level. A general single-particle state is given by letters $pqrs\dots$.

We can then expand our exact state function for the ground state 
as
\[
|\Psi_0\rangle=C_0|\Phi_0\rangle+\sum_{ai}C_i^a|\Phi_i^a\rangle+\sum_{abij}C_{ij}^{ab}|\Phi_{ij}^{ab}\rangle+\dots
=(C_0+\hat{C})|\Phi_0\rangle,
\]
where we have introduced the so-called correlation operator 
\[
\hat{C}=\sum_{ai}C_i^a\hat{a}_{a}^{\dagger}\hat{a}_i  +\sum_{abij}C_{ij}^{ab}\hat{a}_{a}^{\dagger}\hat{a}_{b}^{\dagger}\hat{a}_j\hat{a}_i+\dots
\]
Since the normalization of $\Psi_0$ is at our disposal and since $C_0$ is by hypothesis non-zero, we may arbitrarily set $C_0=1$ with 
corresponding proportional changes in all other coefficients. Using this so-called intermediate normalization we have
\[
\langle \Psi_0 | \Phi_0 \rangle = \langle \Phi_0 | \Phi_0 \rangle = 1, 
\]
resulting in 
\[
|\Psi_0\rangle=(1+\hat{C})|\Phi_0\rangle.
\]

In a shell-model calculation, the unknown coefficients in $\hat{C}$ are the 
eigenvectors which result from the diagonalization of the Hamiltonian matrix.

How can we use perturbation theory to determine the same coefficients? Let us study the contributions to second order in the interaction, namely
\[
\Delta E^{(2)}=\langle\Phi_0\vert \hat{H}_I\frac{\hat{Q}}{W_0-\hat{H}_0}\hat{H}_I\vert \Phi_0\rangle.
\]

The intermediate states given by $\hat{Q}$ can at most be of a $2p-2h$ nature if we have a two-body Hamiltonian. This means that second order in the perturbation theory can have $1p-1h$ and $2p-2h$ at most as intermediate states. When we diagonalize, these contributions are included to infinite order. This means that higher-orders in perturbation theory bring in more complicated correlations. 

If we limit the attention to a Hartree-Fock basis, then we have that
$\langle\Phi_0\vert \hat{H}_I \vert 2p-2h\rangle$ is the only contribution and the contribution to the energy reduces to
\[
\Delta E^{(2)}=\frac{1}{4}\sum_{abij}\langle ij\vert \hat{v}\vert ab\rangle \frac{\langle ab\vert \hat{v}\vert ij\rangle}{\epsilon_i+\epsilon_j-\epsilon_a-\epsilon_b}.
\]

If we compare this to the correlation energy obtained from full configuration interaction theory with a Hartree-Fock basis, we found that
\[
E-E_0 =\Delta E=
\sum_{abij}\langle ij | \hat{v}| ab \rangle C_{ij}^{ab},
\]
where the energy $E_0$ is the reference energy and $\Delta E$ defines the so-called correlation energy.

We see that if we set
\[
C_{ij}^{ab} =\frac{1}{4}\frac{\langle ab \vert \hat{v} \vert ij \rangle}{\epsilon_i+\epsilon_j-\epsilon_a-\epsilon_b},
\]
we have a perfect agreement between FCI and MBPT. However, FCI includes such $2p-2h$ correlations to infinite order. In order to make a meaningful comparison we would at least need to sum such correlations to infinite order in perturbation theory. 

Summing up, we can see that
\begin{itemize}
\item MBPT introduces order-by-order specific correlations and we make comparisons with exact calculations like FCI

\item At every order, we can calculate all contributions since they are well-known and either tabulated or calculated on the fly.

\item MBPT is a non-variational theory and there is no guarantee that higher orders will improve the convergence. 

\item However, since FCI calculations are limited by the size of the Hamiltonian matrices to diagonalize (today's most efficient codes can attach dimensionalities of ten billion basis states, MBPT can function as an approximative method which gives a straightforward (but tedious) calculation recipe. 

\item MBPT has been widely used to compute effective interactions for the nuclear shell-model.

\item But there are better methods which sum to infinite order important correlations. Coupled cluster theory is one of these methods. 
\end{itemize}


\clearemptydoublepage

% 
\chapter{Green's function theory}

\subsection{Single-particle Green's functions}

We consider first a particle in free space described by a single particle Hamiltonian $\hat{h}$. Its eigenstates and eigenenergies are

$$
\hat{h}\left|\phi_{n}\right\rangle=\varepsilon_{n}\left|\phi_{n}\right\rangle
$$

In general, if we put the particle in one of its
$\left|\phi_{n}\right\rangle$ orbits, it will remain in the same state
forever.

We prepare now the system in a generic state
$\left|\psi_{\mathrm{T}}\right\rangle$ ($\mathrm{T}$ stands for trial) and then
follow its time evolution. If the trial state is created at time
$t=0$, the wavefunction at a later time $t$ is given by

$$
\begin{aligned}
|\psi(t)\rangle & =e^{-i h_{1} t / \hbar}\left|\psi_{t r}\right\rangle \\
& =\sum_{n}\left|\phi_{n}\right\rangle e^{-i \varepsilon_{n} t / \hbar}\left\langle\phi_{n} \mid \psi_{t r}\right\rangle
\end{aligned}
$$


The above result shows that if one knows the eigentstates
$\left|\phi_{n}\right\rangle$, it is easy to compute the
time evolution. We expand $\left|\psi_{\mathrm{T}}\right\rangle$ in this
basis and let every component propagate independently. Eventually, at
time $t$, we want to know the probability amplitude for a  measurement
where  the particle is at a position $\mathbf{r}$,

$$
\begin{aligned}
\langle\mathbf{r} \mid \psi(t)\rangle & =\left\langle\mathbf{r}\left|e^{-i \hat{h} t / \hbar}\right| \psi_{\mathrm{T}}\right\rangle \\
& =\int d \mathbf{r}^{\prime}\left\langle\mathbf{r}\left|e^{-i \hat{h} t / \hbar}\right| \mathbf{r}^{\prime}\right\rangle\left\langle\mathbf{r}^{\prime} \mid \psi_{\mathrm{T}}\right\rangle
\end{aligned}
$$

$$
\begin{aligned}
& =\int d \mathbf{r}^{\prime} \sum_{n}\left\langle\mathbf{r} \mid \phi_{n}\right\rangle e^{-i \varepsilon_{n} t / \hbar}\left\langle\phi_{n} \mid \mathbf{r}^{\prime}\right\rangle\left\langle\mathbf{r}^{\prime} \mid \psi_{\mathrm{T}}\right\rangle \\
& \equiv \int d \mathbf{r}^{\prime} G\left(\mathbf{r}, \mathbf{r}^{\prime} ; t\right) \psi_{\mathrm{T}}\left(\mathbf{r}^{\prime}\right),
\end{aligned}
$$

which defines the propagator $G$.

 Once $G\left(\mathbf{r}, \mathbf{r}^{\prime} ; t\right)$
 is known it can be used to calculate the evolution of any initial
 state. However,there is more information included in the
 propagator. This is apparent from the expansion in the third line of the last equation.
 First, the braket $\left\langle\phi_{n} \mid
 \mathbf{r}\right\rangle=\left\langle\phi_{n}\left|\psi^{\dagger}(\mathbf{r})\right|
 0\right\rangle$ gives us the probability that putting a particle at
 position $\mathbf{r}$ and mesuring its energy right away, would make
 the system to collapse into the eigenstate
 $\left|\phi_{n}\right\rangle$.

 Second, the time evolution is a
 superposition of waves propagating with different energies and could
 be inverted to find the eigenspectrum. We could think of  an experiment in
 which the particle is put at position $\mathbf{r}$ and picked up at
 $\mathbf{r}^{\prime}$ after some time $t$. If one can do this for
 different positions and elapsed times and with good resolution, then a
 Fourier transform would simply give back the full eigenvalue
 spectrum. This  gives us the complete information about our particle.

We now want to apply the above ideas to see what we can learn by
adding and removing a particle in an environment when many other particles are
present. This can cause the particle to behave in an unxepected way,
induce collective excitations of the full systems, and so
on. Moreover, the role played by the physical vacuum in the above
example, is now taken by a many-body state (usually its ground
state). Thus, it is also possible to probe the system by removing
particles.


In the following we consider the Heisenberg description of the field operators (see discussions earlier on different pictures)),

$$
\psi_{s}^{\dagger}(\mathbf{r}, t)=e^{i H t / \hbar} \psi_{s}^{\dagger}(\mathbf{r}) e^{-i H t / \hbar}
$$

where the subscript $s$ serves to indicate possible internal degrees of freedom (spin, etc...). We omit the superscrips $\mathrm{H}$ (Heisenberg) and $S$ (Schr\"odinger) from the operators since the two pictures distinguish themselves by  the presence of the time variable, which appears only in the first case. Similarly,

$$
\psi_{s}(\mathbf{r}, t)=e^{i H t / \hbar} \psi_{s}(\mathbf{r}) e^{-i H t / \hbar},
$$


For the case of a general single-particle basis
$\left\{u_{\alpha}(\mathbf{r})\right\}$ one uses the following
creation and annihilation operators

$$
\begin{aligned}
& a_{\alpha}^{\dagger}(t)=e^{i H t / \hbar} a_{\alpha}^{\dagger} e^{-i H t / \hbar}, \\
& a_{\alpha}(t)=e^{i H t / \hbar} a_{\alpha} e^{-i H t / \hbar}
\end{aligned}
$$

which are related to $\psi_{s}^{\dagger}(\mathbf{r}, t)$ and
$\psi_{s}(\mathbf{r}, t)$.

{
\subsection{Eigenstates}


  In most applications the Hamiltonian is split in an unperturbed part $\hat{H}_{0}$ and a residual interaction $\hat{H}_I$

$$
\hat{H}=\hat{H}_{0}+\hat{H}_I .
$$

The $N$-body eigenstates  of the full Hamiltonian are $\left|\Psi_{n}^{N}\right\rangle$, while $\left|\Phi_{n}^{N}\right\rangle$ are the corresponding unperturbed ones

$$
\begin{aligned}
H\left|\Psi_{n}^{N}\right\rangle & =E_{n}^{N}\left|\Psi_{n}^{N}\right\rangle \\
H_{0}\left|\Phi_{n}^{N}\right\rangle & =E_{n}^{(0), N}\left|\Phi_{n}^{N}\right\rangle
\end{aligned}
$$

The definitions given in the following are general and do not depend on the type of interaction being used. Thus, most properties of Green's functions result from genaral principles of quantum mechanics and are valid for any system.

The two-points Green's function describes the propagation of one particle or one hole on top of the ground state $\left|\Psi_{0}^{N}\right\rangle$. This is defined by

$$
g_{s s^{\prime}}\left(\mathbf{r}, t ; \mathbf{r}^{\prime}, t^{\prime}\right)=-\frac{i}{\hbar}\left\langle\Psi_{0}^{N}\left|T\left[\psi_{s}(\mathbf{r}, t) \psi_{s^{\prime}}^{\dagger}\left(\mathbf{r}^{\prime}, t^{\prime}\right)\right]\right| \Psi_{0}^{N}\right\rangle
$$

where $T[\cdots]$ is the time ordering operator that imposes a change of sign for each exchange of two fermion operators

$$
T\left[\psi_{s}(\mathbf{r}, t) \psi_{s^{\prime}}^{\dagger}\left(\mathbf{r}^{\prime}, t^{\prime}\right)\right]= \begin{cases}\psi_{s}(\mathbf{r}, t) \psi_{s^{\prime}}^{\dagger}\left(\mathbf{r}^{\prime}, t^{\prime}\right), & t>t^{\prime} \\ \pm \psi_{s^{\prime}}^{\dagger}\left(\mathbf{r}^{\prime}, t^{\prime}\right) \psi_{s}(\mathbf{r}, t), & t^{\prime}>t\end{cases}
$$

where the upper (lower) sign is for bosons (fermions). A similar definition can be given for the non interacting state $\left|\Phi_{0}^{N}\right\rangle$, in this case the Heisenberg operators discussed above must evolve only according to $H_{0}$ and the notation $g^{(0)}$ is used.

\subsection{Fourier transform}

If the Hamiltonian does not depend on time, the propagator defined above
depends only on the difference $t-t^{\prime}$

$$
\begin{aligned}
g_{s s^{\prime}}\left(\mathbf{r}, \mathbf{r}^{\prime} ; t-t^{\prime}\right)= & -\frac{i}{\hbar} \theta\left(t-t^{\prime}\right)\left\langle\Psi_{0}^{N}\left|\psi_{s}(\mathbf{r}) e^{-i\left(H-E_{0}^{N}\right)\left(t-t^{\prime}\right) / \hbar} \psi_{s^{\prime}}^{\dagger}\left(\mathbf{r}^{\prime}\right)\right| \Psi_{0}^{N}\right\rangle \\
& \mp \frac{i}{\hbar} \theta\left(t^{\prime}-t\right)\left\langle\Psi_{0}^{N}\left|\psi_{s^{\prime}}^{\dagger}\left(\mathbf{r}^{\prime}\right) e^{i\left(H-E_{0}^{N}\right)\left(t-t^{\prime}\right) / \hbar} \psi_{s}(\mathbf{r})\right| \Psi_{0}^{N}\right\rangle .
\end{aligned}
$$

In this case it is useful to Fourier transform with respect to time and define

$$
g_{s s^{\prime}}\left(\mathbf{r}, \mathbf{r}^{\prime} ; \omega\right)=\int d \tau e^{i \omega \tau} g_{s s^{\prime}}\left(\mathbf{r}, \mathbf{r}^{\prime} ; \tau\right)
$$


\subsection{Rewrite of propagator}

Using the relation

$$
\theta( \pm \tau)=\mp \lim _{\eta \rightarrow 0^{+}} \frac{1}{2 \pi i} \int_{-\infty}^{+\infty} d \omega \frac{e^{-i \omega \tau}}{\omega \pm i \eta}
$$

we obtain

$$
\begin{aligned}
g_{s s^{\prime}}\left(\mathbf{r}, \mathbf{r}^{\prime} ; \omega\right)= & g_{s s^{\prime}}^{p}\left(\mathbf{r}, \mathbf{r}^{\prime} ; \omega\right)+g_{s s^{\prime}}^{h}\left(\mathbf{r}, \mathbf{r}^{\prime} ; \omega\right) \\
= & \left\langle\Psi_{0}^{N}\left|\psi_{s}(\mathbf{r}) \frac{1}{\hbar \omega-\left(H-E_{0}^{N}\right)+i \eta} \psi_{s^{\prime}}^{\dagger}\left(\mathbf{r}^{\prime}\right)\right| \Psi_{0}^{N}\right\rangle \\
& \mp\left\langle\Psi_{0}^{N}\left|\psi_{s^{\prime}}^{\dagger}\left(\mathbf{r}^{\prime}\right) \frac{1}{\hbar \omega+\left(H-E_{0}^{N}\right)-i \eta} \psi_{s}(\mathbf{r})\right| \Psi_{0}^{N}\right\rangle,
\end{aligned}
$$

In the last equation, $g^{p}$ propagates a particle from
$\mathbf{r}^{\prime}$ to $\mathbf{r}$, while $g^{h}$ propagates a hole
from $\mathbf{r}$ to $\mathbf{r}^{\prime}$. Note that the
interpretation is that a particle is added at $\mathbf{r}^{\prime}$,
and later on some (indistiguishable) particle is removed from
$\mathbf{r}^{\prime}$ (and similarly for holes). In the meantime, it
is the fully correlated $(N \pm 1)$ body system that propagates.

In many cases,  in particular in
the vicinity of the Fermi surface, this motion mantains many
characteristics that are typical of a particle moving in free space,
even if the motion itself could actually be a collective excitation of
many constituents. But since it looks like a single particle state we
may still refer to it as quasiparticle.

{
\subsection{Orthonormal basis set definitions}

The same definitions can be made for any orthonormal basis $\{\alpha\}$, leading to the realtions

$$
g_{\alpha \beta}\left(t, t^{\prime}\right)=-\frac{i}{\hbar}\left\langle\Psi_{0}^{N}\left|T\left[a_{\alpha}(t) a_{\beta}^{\dagger}\left(t^{\prime}\right)\right]\right| \Psi_{0}^{N}\right\rangle
$$

where

$$
g_{s s^{\prime}}\left(\mathbf{r}, t ; \mathbf{r}^{\prime}, t^{\prime}\right)=\sum_{\alpha \beta} u_{\alpha}(\mathbf{r}, s) g_{\alpha \beta}\left(t, t^{\prime}\right) u_{\beta}^{*}\left(\mathbf{r}^{\prime}, s^{\prime}\right)
$$

and

$$
\begin{aligned}
g_{\alpha \beta}(\omega)= & \left\langle\Psi_{0}^{N}\left|a_{\alpha} \frac{1}{\hbar \omega-\left(H-E_{0}^{N}\right)+i \eta} a_{\beta}^{\dagger}\right| \Psi_{0}^{N}\right\rangle \\
& \mp\left\langle\Psi_{0}^{N}\left|a_{\beta}^{\dagger} \frac{1}{\hbar \omega+\left(H-E_{0}^{N}\right)-i \eta} a_{\alpha}\right| \Psi_{0}^{N}\right\rangle .
\end{aligned}
$$

\subsection{Lehmann representation}

As discussed above for the one particle case, the information
contained in the propagators becomes more clear if one Fourier
transforms the time variable and inserts a completness for the
intermediate states. This is so because it makes the spectrum and the
transition amplitudes to apper explicitely. Using the completeness
relations for the $(N \pm 1)$-body systems in the last equation, one has

$$
\begin{aligned}
g_{\alpha \beta}(\omega)= & \sum_{n} \frac{\left\langle\Psi_{0}^{N}\left|a_{\alpha}\right| \Psi_{n}^{N+1}\right\rangle\left\langle\Psi_{n}^{N+1}\left|a_{\beta}^{\dagger}\right| \Psi_{0}^{N}\right\rangle}{\hbar \omega-\left(E_{n}^{N+1}-E_{0}^{N}\right)+i \eta} \\
& \mp \sum_{k} \frac{\left\langle\Psi_{0}^{N}\left|a_{\beta}^{\dagger}\right| \Psi_{k}^{N-1}\right\rangle\left\langle\Psi_{k}^{N-1}\left|a_{\alpha}\right| \Psi_{0}^{N}\right\rangle}{\hbar \omega-\left(E_{0}^{N}-E_{k}^{N-1}\right)-i \eta} .
\end{aligned}
$$

which is known as the Lehmann repressentation of a many-body Green's
function. Here, the first and second terms on the left hand
side describe the propagation of a (quasi)particle and a (quasi)hole
excitations.


The poles in last equation are the energies relatives to the
$\left|\Psi_{0}^{N}\right\rangle$ ground state. Hence they give the
energies actually relased in a capture reaction experiment to a bound
state of $\left|\Psi_{n}^{N+1}\right\rangle$. The residues are
transition amplitudes for the addition of a particle and take the name
of spectroscopic amplitudes. They play the same role as the
$\left\langle\phi_{n} \mid \mathbf{r}\right\rangle$ wave function.
In fact these energies and amplitudes are solutions of a
Schr\"odinger-like equation: the Dyson equation. The hole part of the
propagator gives instead information on the process of particle
emission, the poles being the exact energy absorbed in the
process. For example, in the single particle Green's function of an atome,
the quasiparticle and quasihole poles are respectively the
electron affinities an ionization energies.

We will look at the physical significance of spectroscopic amplitudes
below and derive the Dyson equation (which is the
fundamental equation in many-body Green's function theory) below.

\subsection{Spectral function}

As a last definition, we rewrite the above Lehmann representation  in a form
that can compared more easily to experiments. By using the relation

$$
\frac{1}{x \pm i \eta}=\mathcal{P} \frac{1}{x} \mp i \pi \delta(x)
$$
}
we can extract the one-body spectral function

$$
S_{\alpha \beta}(\omega)=S_{\alpha \beta}^{p}(\omega)+S_{\alpha \beta}^{h}(\omega),
$$

where the particle and hole components are

$$
\begin{aligned}
S_{\alpha \beta}^{p}(\omega) & =-\frac{1}{\pi} \operatorname{Im} g_{\alpha \beta}^{p}(\omega) \\
& =\sum_{n}\left\langle\Psi_{0}^{N}\left|a_{\alpha}\right| \Psi_{n}^{N+1}\right\rangle\left\langle\Psi_{n}^{N+1}\left|a_{\beta}^{\dagger}\right| \Psi_{0}^{N}\right\rangle \delta\left(\hbar \omega-\left(E_{n}^{N+1}-E_{0}^{N}\right)\right) \\
S_{\alpha \beta}^{h}(\omega) & =\frac{1}{\pi} \operatorname{Im} g_{\alpha \beta}^{h}(\omega) \\
& =\mp \sum_{k}\left\langle\Psi_{0}^{N}\left|a_{\beta}^{\dagger}\right| \Psi_{k}^{N-1}\right\rangle\left\langle\Psi_{k}^{N-1}\left|a_{\alpha}\right| \Psi_{0}^{N}\right\rangle \delta\left(\hbar \omega-\left(E_{0}^{N}-E_{k}^{N-1}\right)\right) .
\end{aligned}
$$


\subsection{Dispersion relation}

The diagonal part of the spectral function is interpreted as the
probability of adding $\left[S_{\alpha \alpha}^{p}(\omega)\right]$ or
removing $\left[S_{\alpha \alpha}^{h}(\omega)\right]$ one particle in
the state $\alpha$ leaving the residual system in a state of energy
$\omega$.

By comparing the last two equations to the Lehmann representation, we
see that the propagator is completely constrained by its imaginary
part. We have

$$
g_{\alpha \beta}(\omega)=\int d \omega^{\prime} \frac{S_{\alpha \beta}^{p}\left(\omega^{\prime}\right)}{\omega-\omega^{\prime}+i \eta}+\int d \omega^{\prime} \frac{S_{\alpha \beta}^{h}\left(\omega^{\prime}\right)}{\omega-\omega^{\prime}-i \eta} .
$$

In general the single particle propagator of a finite system has
isolated poles in correspondence to the bound eigenstates of the
$(N+1)$-body system. For larger enegies, where
$\left|\Psi_{n}^{N+1}\right\rangle$ are states in the continuum, it
develops a branch cut. The particle propagator $g^{p}(\omega)$ is
analytic in the upper half of the complex plane, and so is the full
propagator for $\omega \geq
E_{0}^{N+1}-E_{0}^{N}$. Analogously, the hole propagator has poles for
$\omega \leq E_{0}^{N}-E_{0}^{N-1}$ and is analytic in the lower
complex plane. Note that high excitation energies in the (N-1)body
system correspond to negative values of the poles
$E_{0}^{N}-E_{k}^{N-1}$, so $g^{h}(\omega)$ develops a branch cut for
large negative energies.


\subsection{Expectation values}

The one-body density matrix can be obtained from the one-body
propagator. One simply chooses the appropriate time ordering

$$
\rho_{\alpha \beta}=\left\langle\Psi_{0}^{N}\left|a_{\beta}^{\dagger} a_{\alpha}\right| \Psi_{0}^{N}\right\rangle= \pm i \hbar \lim _{t^{\prime} \rightarrow t^{+}} g_{\alpha \beta}\left(t, t^{\prime}\right)
$$

(where the upper sign is for bosons and the lower one is for fermions). Alternatively, the hole spectral function can be used

$$
\rho_{\alpha \beta}=\mp \int d \omega S_{\alpha \beta}^{h}(\omega) \text {. }
$$


\subsection{Single-particle Green's functions}

Thus, the expectation value of a one-body operator for the
ground states $\left|\Psi_{0}^{N}\right\rangle$ is usually as

$$
\begin{aligned}
\left\langle\Psi_{0}^{N}|O| \Psi_{0}^{N}\right\rangle & =\mp \sum_{\alpha \beta} \int d \omega o_{\alpha \beta} S_{\beta \alpha}^{h}(\omega) \\
& = \pm i \hbar \lim _{t^{\prime} \rightarrow t^{+}} \sum_{\alpha \beta} o_{\alpha \beta} g_{\beta \alpha}\left(t, t^{\prime}\right)
\end{aligned}
$$

and both terms are equivalent.


From the particle spectral function, one can extract the quantity

$$
d_{\alpha \beta}=\left\langle\Psi_{0}^{N}\left|a_{\alpha} a_{\beta}^{\dagger}\right| \Psi_{0}^{N}\right\rangle=\int d \omega S_{\beta \alpha}^{p}(\omega)
$$

which leads to the following sum rule

$$
\int d \omega S_{\alpha \beta}(\omega)=d_{\alpha \beta} \mp \rho_{\alpha \beta}=\left\langle\Psi_{0}^{N}\left|\left[a_{\alpha}, a_{\beta}^{\dagger}\right]_{\mp}\right| \Psi_{0}^{N}\right\rangle=\delta_{\alpha \beta} .
$$


For the case of an Hamiltonian containing up to two-body interactions only we have

$$
\begin{aligned}
H & =U+V \\
& =\sum_{\alpha \beta} t_{\alpha \beta} a_{\alpha}^{\dagger} a_{\beta}+\frac{1}{4} \sum_{\alpha \beta \gamma \delta} v_{\alpha \beta, \gamma \delta} a_{\alpha}^{\dagger} a_{\beta}^{\dagger} a_{\delta} a_{\gamma},
\end{aligned}
$$

and we can derive an important sum rule that relates the total energy
of the state $\left|\Psi_{0}^{N}\right\rangle$ to its one-body Green's
function. To derive this, one makes use of the equation of motion for
Heisenberg operators, which gives

$$
i \hbar \frac{d}{d t} a_{\alpha}(t)=e^{i H t / \hbar}\left[a_{\alpha}, H\right] e^{-i H t / \hbar},
$$

with

$$
\left[a_{\alpha}, H\right]=\sum_{\beta} t_{\alpha \beta} a_{\beta}+\frac{1}{2} \sum_{\beta \gamma \delta} v_{\alpha \beta \gamma \delta} a_{\beta}^{\dagger} a_{\delta} a_{\gamma}
$$

which is valid for both fermions and bosons.



If one uses the last equation we can derive the propagator as function of time through

\[
\begin{aligned}
i \hbar \frac{\partial}{\partial t} g_{\alpha \beta}\left(t-t^{\prime}\right)= & \delta\left(t-t^{\prime}\right) \delta_{\alpha \beta}+\sum_{\gamma} t_{\alpha \gamma} g_{\gamma \beta}\left(t-t^{\prime}\right) \\
& -\frac{i}{\hbar} \sum_{\eta \gamma \zeta} \frac{1}{2} v_{\alpha \eta, \gamma \zeta}\left\langle\Psi_{0}^{N}\left|T\left[a_{\eta}^{\dagger}(t) a_{\zeta}(t) a_{\gamma}(t) a_{\beta}^{\dagger}\left(t^{\prime}\right)\right]\right| \Psi_{0}^{N}\right\rangle
\end{aligned}
\]


The braket in the last line contains the four-point Green's function to be discussed below.
The four-point Green's function can describe the simultaneous propagation of
two particles. Thus, one sees that applying the equation of motion to
a propagator leads to relations which contain Green's functions of
higher order. This result is particularly important because it shows
there exist a hierarchy between propagators, so that the exact
equations that determine the one-body function will depend on the
two-body one, the two-body function will contain contributions from
three-body propagators, and so on.

For the moment we just want to select a particular order of the
operators in order to extract the one- and two-body
density matrices. To do this, we chose $t^{\prime}$ to be a later time
than $t$ and take its limit to the latter from above. This yields

$$
\pm i \hbar \lim _{t^{\prime} \rightarrow t^{+}} \sum_{\alpha} \frac{\partial}{\partial t} g_{\alpha \alpha}\left(t-t^{\prime}\right)=\langle T\rangle+2\langle V\rangle
$$

(note that for $t \neq t^{\prime}$, the term
$\delta\left(t-t^{\prime}\right)=0$ and it does not contribute to the
limit). This result can also be expressed in energy representation by
inverting the Fourier transformation. We have then

$$
\lim _{\tau \rightarrow 0^{-}} \frac{\partial}{\partial \tau} g_{\alpha \beta}(\tau)=-\int d \omega \omega S_{\alpha \beta}^{h}(\omega)
$$


\subsection{Expectation values of the energy}

By combining the above results we arrive at

$$
\begin{aligned}
\langle H\rangle=\langle U\rangle+\langle V\rangle & = \pm i \hbar \frac{1}{2} \lim _{t^{\prime} \rightarrow t^{+}} \sum_{\alpha \beta}\left\{\delta_{\alpha \beta} \frac{\partial}{\partial t}+t_{\alpha \beta}\right\} g_{\beta \alpha}\left(t-t^{\prime}\right) \\
& =\mp \frac{1}{2} \sum_{\alpha \beta} \int d \omega\left\{\delta_{\alpha \beta} \omega+t_{\alpha \beta}\right\} S_{\beta \alpha}^{h}(\omega)
\end{aligned}
$$

We used here the relation $[A, B C]_{-}=[A, B] C-B[C, A]=\{A, B\}
C-B\{C, A\}$ which is valid for both commutators and anticommutators


Surprisingly, for an Hamiltonian containing only two-body forces it is
possible to extract the ground state energy by knowing only the
one-body propagator. 

When interactions among three or more particles are present, this
relation has to be augmented to include additional terms. In these
cases higher order Green's functions will appear explicitly.


\subsection{Higher order  Green's functions}

The definition of the one-body Green's function  can be extended to Green's functions for the
propagation of more than one particle. In general, for each additional
particle it will be necessary to introduce one additional creation and
one annihilation operator. Thus a $2 n$-points Green's function will
propagate a maximum of $n$ quasiparticles. The explicit definition of
the four-point propagator is

$$
g_{\alpha \beta, \gamma \delta}^{4-p t}\left(t_{1}, t_{2} ; t_{1}^{\prime}, t_{2}^{\prime}\right)=-\frac{i}{\hbar}\left\langle\Psi_{0}^{N}\left|T\left[a_{\beta}\left(t_{2}\right) a_{\alpha}\left(t_{1}\right) a_{\gamma}^{\dagger}\left(t_{1}^{\prime}\right) a_{\delta}^{\dagger}\left(t_{2}^{\prime}\right)\right]\right| \Psi_{0}^{N}\right\rangle
$$

while the six-point case is

$$
\begin{aligned}
& g_{\alpha \beta \gamma, \mu \nu \lambda}^{6-p t}\left(t_{1}, t_{2}, t_{3} ; t_{1}^{\prime}, t_{2}^{\prime}, t_{3}^{\prime}\right)= \\
& \quad-\frac{i}{\hbar}\left\langle\Psi_{0}^{N}\left|T\left[a_{\gamma}\left(t_{3}\right) a_{\beta}\left(t_{2}\right) a_{\alpha}\left(t_{1}\right) a_{\mu}^{\dagger}\left(t_{1}^{\prime}\right) a_{\nu}^{\dagger}\left(t_{2}^{\prime}\right) a_{\lambda}^{\dagger}\left(t_{3}^{\prime}\right)\right]\right| \Psi_{0}^{N}\right\rangle,
\end{aligned}
$$


\subsection{Interpretations}

It should be noted that the actual number of particles that are
propagated by these objects depends on the ordering of the time
variables. Therefore the information on transitions between
eigenstates of the systems with $N, N \pm 1$ and $N \pm 2$ bodies are
all encoded in the four-point propagator, while additional states of $N \pm 3$-body
states are included in the six-point propagator. Obviously, the presence of so many
time variables makes the use of these functions extremely difficult
(and even impossible, in many cases). However, it is still useful to
consider only certain time orderings which allow to extract the
information not included in the two-point propagator.


\subsection{Two-particle-two-hole propagator}

The two-particle-two-hole propagator is a two-times Green's function
defined as

$$
g_{\alpha \beta, \gamma \delta}^{I I}\left(t, t^{\prime}\right)=-\frac{i}{\hbar}\left\langle\Psi_{0}^{N}\left|T\left[a_{\beta}(t) a_{\alpha}(t) a_{\gamma}^{\dagger}\left(t^{\prime}\right) a_{\delta}^{\dagger}\left(t^{\prime}\right)\right]\right| \Psi_{0}^{N}\right\rangle
$$

which corresponds to the limit $t_{1}^{\prime}=t_{2}^{\prime+}$ and $t_{2}=t_{1}^{+}$of $g^{4-p t}$.


As for the case of $g_{\alpha \beta}\left(t, t^{\prime}\right)$, if
the Hamiltonian is time-independent, the last equation is a function of the
time difference only. Therefore it has a Lehmann representation
containing the exact spectrum of the $(N \pm 2)$-body systems

$$
\begin{aligned}
g_{\alpha \beta, \gamma \delta}^{I I}(\omega) & =\sum_{n} \frac{\left\langle\Psi_{0}^{N}\left|a_{\beta} a_{\alpha}\right| \Psi_{n}^{N+2}\right\rangle\left\langle\Psi^{N+2}\left|a_{\gamma}^{\dagger} a_{\delta}^{\dagger}\right| \Psi_{0}^{N}\right\rangle}{\omega-\left(E_{n}^{N+2}-E_{0}^{N}\right)+i \eta} \\
& -\sum_{k} \frac{\left\langle\Psi_{0}^{N}\left|a_{\gamma}^{\dagger} a_{\delta}^{\dagger}\right| \Psi_{k}^{N-2}\right\rangle\left\langle\Psi_{k}^{N-2}\left|a_{\beta} a_{\alpha}\right| \Psi_{0}^{N}\right\rangle}{\omega-\left(E_{0}^{N}-E_{k}^{N-2}\right)-i \eta}
\end{aligned}
$$

\subsection{Spectral functions}

Similarly one defines the two-particle and two-hole spectral functions

$$
S_{\alpha \beta}^{I I}, \gamma \delta(\omega)=S_{\alpha \beta, \gamma \delta}^{p p}(\omega)+S_{\alpha \beta, \gamma \delta}^{h h}(\omega)
$$

and

$$
\begin{aligned}
S_{\alpha \beta, \gamma \delta}^{p p}(\omega) & =-\frac{1}{\pi} \operatorname{Im} g_{\alpha \beta, \gamma \delta}^{p p}(\omega) \\
& =\sum_{n}\left\langle\Psi_{0}^{N}\left|a_{\beta} a_{\alpha}\right| \Psi_{n}^{N+2}\right\rangle\left\langle\Psi_{n}^{N+2}\left|a_{\gamma}^{\dagger} a_{\delta}^{\dagger}\right| \Psi_{0}^{N}\right\rangle \delta\left(\hbar \omega-\left(E_{n}^{N+2}-E_{0}^{N}\right)\right), \\
S_{\alpha \beta, \gamma \delta}^{h h}(\omega) & =\frac{1}{\pi} \operatorname{Im} g_{\alpha \beta, \gamma \delta}^{h h}(\omega) \\
& =-\sum_{k}\left\langle\Psi_{0}^{N}\left|a_{\gamma}^{\dagger} a_{\delta}^{\dagger}\right| \Psi_{k}^{N-2}\right\rangle\left\langle\Psi_{k}^{N-2}\left|a_{\beta} a_{\alpha}\right| \Psi_{0}^{N}\right\rangle \delta\left(\hbar \omega-\left(E_{0}^{N}-E_{k}^{N-2}\right)\right) .
\end{aligned}
$$

Based on these equations it is easy to obtain relations for the two-body density matrix

$$
\Gamma_{\alpha \beta, \gamma \delta}=\left\langle\Psi^{N}\left|a_{\gamma}^{\dagger} a_{\delta}^{\dagger} a_{\beta} a_{\alpha}\right| \Psi^{N}\right\rangle=-\int d \omega S_{\alpha \beta, \gamma \delta}^{h h}(\omega)
$$

and, hence, for the expectation value of any two-body operator

$$
\begin{aligned}
\left\langle\Psi_{0}^{N}|V| \Psi_{0}^{N}\right\rangle & =-\sum_{\alpha \beta \gamma \delta} \int d \omega v_{\alpha \beta, \gamma \delta} S_{\gamma \delta, \alpha \beta}^{h}(\omega) \\
& =+i \hbar \lim _{t^{\prime} \rightarrow t^{+}} \frac{1}{4} \sum_{\alpha \beta \gamma \delta} v_{\alpha \beta, \gamma \delta} g_{\gamma \delta, \alpha \beta}^{I I}\left(t, t^{\prime}\right) .
\end{aligned}
$$

\subsection{Polarization propagator}

The polarization propagator $\Pi_{\alpha \beta, \gamma \delta}$
corresponds to the time ordering of $g^{4-p t}$ in which a
particle-hole excitation is created at one single time. Therefore, no
process involving particle transfer in included. However it describes
transition to the excitations of the system, as long as they can be
reached with a one-body operator. For example, this includes
collective modes of a nucleus. This is defined as

$$
\begin{aligned}
\Pi_{\alpha \beta, \gamma \delta}\left(t, t^{\prime}\right)=- & \frac{i}{\hbar}\left\langle\Psi_{0}^{N}\left|T\left[a_{\beta}^{\dagger}(t) a_{\alpha}(t) a_{\gamma}^{\dagger}\left(t^{\prime}\right) a_{\delta}\left(t^{\prime}\right)\right]\right| \Psi_{0}^{N}\right\rangle \\
& +\frac{i}{\hbar}\left\langle\Psi_{0}^{N}\left|a_{\beta}^{\dagger} a_{\alpha}\right| \Psi_{0}^{N}\right\rangle\left\langle\Psi_{0}^{N}\left|a_{\gamma}^{\dagger} a_{\delta}\right| \Psi_{0}^{N}\right\rangle
\end{aligned}
$$


  After including a completeness of $\left|\Psi_{n}^{N}\right\rangle$
states, the contribution of to the ground states (at zero
energy) is cancelled by the last term in the equation. Thus one can
Fourier transform to the Lehmann representation

$$
\begin{aligned}
\Pi_{\alpha \beta, \gamma \delta}(\omega) & =\sum_{n \neq 0} \frac{\left\langle\Psi_{0}^{N}\left|a_{\beta}^{\dagger} a_{\alpha}\right| \Psi_{n}^{N}\right\rangle\left\langle\Psi_{n}^{N}\left|a_{\gamma}^{\dagger} a_{\delta}\right| \Psi_{0}^{N}\right\rangle}{\omega-\left(E_{n}^{N}-E_{0}^{N}\right)+i \eta} \\
& -\sum_{n \neq 0} \frac{\left\langle\Psi_{0}^{N}\left|a_{\gamma}^{\dagger} a_{\delta}\right| \Psi_{n}^{N}\right\rangle\left\langle\Psi_{n}^{N}\left|a_{\beta}^{\dagger} a_{\alpha}\right| \Psi_{0}^{N}\right\rangle}{\omega+\left(E_{n}^{N}-E_{0}^{N}\right)-i \eta}
\end{aligned}
$$


\subsection{Transition matrix elements}

Note that $\Pi_{\alpha \beta, \gamma \delta}(\omega)=\Pi_{\delta
  \gamma, \beta \alpha}(-\omega)$ due to time reversal symmetry. Also
the forward and backward parts carry the same information.

Once again, the residues of the propagator can be used to
calculate expectation values. In this case, given a one-body operator
we obtain the transition matrix elements to any excited state

$$
\left\langle\Psi_{n}^{N}|O| \Psi_{0}^{N}\right\rangle=\sum_{\alpha \beta} o_{\beta \alpha}\left\langle\Psi_{n}^{N}\left|a_{\beta}^{\dagger} a_{\alpha}\right| \Psi_{0}^{N}\right\rangle
$$


\subsection{Coupling to experiment}

Here we explore the connection between the information contained in
various propagators and experimental data. The focus is on the
experimental properties that are probed by the removal of
particles. Also, from now on, we will only consider fermionic systems.

An important case is when the spectrum for the $N \pm 1$-particle
system near the Fermi energy involves discrete bound states. This
happens in finite system like nuclei or molecules. In these cases the
main quantity of interest is the overlap wave function

$$
\begin{aligned}
\psi_{k}^{\text {overlap }}(\mathbf{r})= & \left\langle\Psi_{k}^{N-1}\left|\psi_{s}(\mathbf{r})\right| \Psi_{0}^{N}\right\rangle \\
= & \sqrt{N} \int d \mathbf{r}_{2} \int d \mathbf{r}_{3} \cdots \int d \mathbf{r}_{N} \\
& \quad \times\left[\Psi_{k}^{N-1}\left(\mathbf{r}_{2}, \mathbf{r}_{3}, \ldots \mathbf{r}_{N}\right)\right]^{*} \Psi_{0}^{N}\left(\mathbf{r}, \mathbf{r}_{2}, \mathbf{r}_{3}, \ldots \mathbf{r}_{N}\right) .
\end{aligned}
$$


This integral comes out in the description of most
particle knock-out processes because it represents the matrix element
between the initial and final states, in the case when the emitted
particle is ejected with energy large enough the it interacts only
weakly with the residual system. The quantity of interest here is the
so called spectroscopic factor to the final state $k$,

$$
S_{k}=\int d \mathbf{r}\left|\psi_{k}^{\text {overlap }}(\mathbf{r})\right|^{2}
$$

When the system is made of completely non interacting particles,
$S_{k}$ is unity. In real cases however, correlations among the
constituents reduce this value. The possibility of extracting this
quantity from experimental data gives us information on the spectral
function and therefore on the structure of the correlated system.


\subsection{Spectroscopic strength from particle emission}

In order to make the connection with experimental data obtained from
knockout reactions, it is useful to consider the response of a system
to a weak probe. The hole spectral function
can be substantially "observed" these reactions. The general idea is
to transfer a large amount of momentum and energy to a particle of a
bound system in the ground state. This is then ejected from the
system, and one ends up with a fast-moving particle and a bound
$(N-1)$-particle system. By observing the momentum of the ejected
particle it is then possible the reconstruct the spectral function of
the system, provided that the interaction between the ejected particle
and the remainder is sufficiently weak or treated in a controlled
fashion, for example by constraining this treatment with information from
other experimental data.

We assume that the $N$-particle system is initially in its ground state,

$$
\left|\Psi_{i}\right\rangle=\left|\Psi_{0}^{N}\right\rangle
$$

and makes a transition to a final $N$-particle eigenstate

$$
\left|\Psi_{f}\right\rangle=a_{p}^{\dagger}\left|\Psi_{n}^{N-1}\right\rangle
$$

composed of a bound $(N-1)$-particle eigenstate, $\left|\Psi_{n}^{N-1}\right\rangle$, and a particle with momentum $\boldsymbol{p}$.


For simplicity we consider the transition matrix elements for a scalar external probe

$$
\rho(\boldsymbol{q})=\sum_{j=1}^{N} \exp \left(i \boldsymbol{q} \cdot \boldsymbol{r}_{j}\right)
$$

which transfers momentum $\boldsymbol{q}$ to a particle. Suppressing other possible sp quantum numbers, like e.g. spin, the second-quantized form of this operator is given by

$$
\hat{\rho}(\boldsymbol{q})=\sum_{\boldsymbol{p}, \boldsymbol{p}^{\prime}}\left\langle\boldsymbol{p}|\exp (i \boldsymbol{q} \cdot \boldsymbol{r})| \boldsymbol{p}^{\prime}\right\rangle a_{\boldsymbol{p}}^{\dagger} a_{\boldsymbol{p}^{\prime}}=\sum_{\boldsymbol{p}} a_{\boldsymbol{p}}^{\dagger} a_{\boldsymbol{p}-\boldsymbol{q}}
$$


\subsection{Transition matrix element}

The transition matrix element now becomes

$$
\begin{aligned}
\left\langle\Psi_{f}|\hat{\rho}(\boldsymbol{q})| \Psi_{i}\right\rangle & =\sum_{\boldsymbol{p}^{\prime}}\left\langle\Psi_{n}^{N-1}\left|a_{\boldsymbol{p}} a_{\boldsymbol{p}^{\prime}}^{\dagger} a_{\boldsymbol{p}^{\prime}-\boldsymbol{q}}\right| \Psi_{0}^{N}\right\rangle \\
& =\sum_{\boldsymbol{p}^{\prime}}\left\langle\Psi_{n}^{N-1}\left|\delta_{\boldsymbol{p}^{\prime}, \boldsymbol{p}} a_{\boldsymbol{p}^{\prime}-\boldsymbol{q}}+a_{\boldsymbol{p}^{\prime}}^{\dagger} a_{\boldsymbol{p}^{\prime}-\boldsymbol{q}} a_{\boldsymbol{p}}\right| \Psi_{0}^{N}\right\rangle \\
& \approx\left\langle\Psi_{n}^{N-1}\left|a_{\boldsymbol{p}-\boldsymbol{q}}\right| \Psi_{0}^{N}\right\rangle .
\end{aligned}
$$

    
The last line is obtained in the so-called Impulse Approximation (or
Sudden Approximation), where it is assumed that the ejected particle
is the one that has absorbed the momentum from the external
field. This is a very good approximation whenever the momentum
$\boldsymbol{p}$ of the ejectile is much larger than typical momenta
for the particles in the bound states; the neglected term 
is then very small, as it involves the removal of a particle with
momentum $\boldsymbol{p}$ from $\left|\Psi_{0}^{N}\right\rangle$.

There is one other assumption in the derivation: the fact that the
final eigenstate of the $N$-particle system was written in the form of
a plane-wave state for the ejectile on top of an $(N-1)$-particle
eigenstate. This is again a good approximation if the ejectile
momentum is large enough, as can be understood by rewriting the
Hamiltonian in the $N$-particle system as

$$
H_{N}=\sum_{i=1}^{N} \frac{\boldsymbol{p}_{i}^{2}}{2 m}+\sum_{i<j=1}^{N} V(i, j)=H_{N-1}+\frac{\boldsymbol{p}_{N}^{2}}{2 m}+\sum_{i=1}^{N-1} V(i, N)
$$


The last term in the above equation represents the Final State Interaction, or
the interaction between the ejected particle $N$ and the other
particles $1 \ldots N-1$. If the relative momentum between particle
$N$ and the others is large enough their mutual interaction can be
neglected, and $H_{N} \approx H_{N-1}+\boldsymbol{p}_{N}^{2} / 2
m$. The results above are called the Plane Wave Impulse
Approximation or PWIA knock-out amplitude, for obvious reasons, and is
precisely a removal amplitude (in the momentum representation)
appearing in the Lehmann representation of the sp propagator.

The cross section of the knock-out reaction, where the momentum and
energy of the ejected particle and the probe are either measured or
known, is according to Fermi's golden rule proportional to

$$
d \sigma \sim \sum_{n} \delta\left(\omega+E_{i}-E_{f}\right)\left|\left\langle\Psi_{f}|\hat{\rho}(\boldsymbol{q})| \Psi_{i}\right\rangle\right|^{2}
$$

where the energy-conserving $\delta$-function contains the energy
transfer $\omega$ of the probe, and the initial and final energies of
the system are $E_{i}=E_{0}^{N}$ and
$E_{f}=E_{n}^{N-1}+\boldsymbol{p}^{2} / 2 m$, respectively. Note that
the internal state of the residual $N-1$ system is not measured, hence
the summation over $n$. Defining the missing momentum
$\boldsymbol{p}_{\text {miss }}$ and missing energy $E_{\text {miss
}}$ of the knock-out reaction as ${ }^{1}$

$$
\boldsymbol{p}_{m i s s}=\boldsymbol{p}-\boldsymbol{q}
$$

and


\[
E_{\text {miss }}=\boldsymbol{p}^{2} / 2 m-\omega=E_{0}^{N}-E_{n}^{N-1}.
\]
We  neglect here the recoil of the residual $N-1$ system, i.e. we assume the mass of the $N$ and $N-1$ system to be much
heavier than the mass $m$ of the ejected particle.

The PWIA knock-out cross section can be rewritten as

\[
\begin{aligned}
d \sigma & \sim \sum_{n} \delta\left(E_{\text {miss }}-E_{0}^{N}+E_{n}^{N-1}\right)\left|\left\langle\Psi_{n}^{N-1}\left|a_{\boldsymbol{p}_{\text {miss }}}\right| \Psi_{0}^{N}\right\rangle\right|^{2} \\
& =S^{h}\left(\boldsymbol{p}_{\text {miss }}, E_{\text {miss }}\right) .
\end{aligned}
\]

The PWIA cross section is therefore exactly proportional to the
diagonal part of the hole spectral function. This is of course only
true in the PWIA, but when the deviations of the impulse approximation
and the effects of the final state interaction are under control, it
is possible to obtain precise experimental information on the hole
spectral function of the system under study.

\part{Monte Carlo Methods}

\chapter{Variational Monte Carlo methods}

\subsection*{Quantum Monte Carlo Motivation}

We start with the variational principle.
Given a hamiltonian $H$ and a trial wave function $\Psi_T$, the variational principle states that the expectation value of $\langle H \rangle$, defined through 
\[
   E[H]= \langle H \rangle =
   \frac{\int d\bm{R}\Psi^{\ast}_T(\bm{R})H(\bm{R})\Psi_T(\bm{R})}
        {\int d\bm{R}\Psi^{\ast}_T(\bm{R})\Psi_T(\bm{R})},
\]
is an upper bound to the ground state energy $E_0$ of the hamiltonian $H$, that is 
\[
    E_0 \le \langle H \rangle .
\]
In general, the integrals involved in the calculation of various  expectation values  are multi-dimensional ones. Traditional integration methods such as the Gauss-Legendre will not be adequate for say the  computation of the energy of a many-body system.

The trial wave function can be expanded in the eigenstates of the hamiltonian since they form a complete set, viz.,
\[
   \Psi_T(\bm{R})=\sum_i a_i\Psi_i(\bm{R}),
\]
and assuming the set of eigenfunctions to be normalized one obtains 
\[
     \frac{\sum_{nm}a^*_ma_n \int d\bm{R}\Psi^{\ast}_m(\bm{R})H(\bm{R})\Psi_n(\bm{R})}
        {\sum_{nm}a^*_ma_n \int d\bm{R}\Psi^{\ast}_m(\bm{R})\Psi_n(\bm{R})} =\frac{\sum_{n}a^2_n E_n}
        {\sum_{n}a^2_n} \ge E_0,
\]
where we used that $H(\bm{R})\Psi_n(\bm{R})=E_n\Psi_n(\bm{R})$.
In general, the integrals involved in the calculation of various  expectation
values  are multi-dimensional ones. 
The variational principle yields the lowest state of a given symmetry.

In most cases, a wave function has only small values in large parts of 
configuration space, and a straightforward procedure which uses
homogenously distributed random points in configuration space 
will most likely lead to poor results. This may suggest that some kind
of importance sampling combined with e.g., the Metropolis algorithm 
may be  a more efficient way of obtaining the ground state energy.
The hope is then that those regions of configurations space where
the wave function assumes appreciable values are sampled more 
efficiently. 

The tedious part in a VMC calculation is the search for the variational
minimum. A good knowledge of the system is required in order to carry out
reasonable VMC calculations. This is not always the case, 
and often VMC calculations 
serve rather as the starting
point for so-called diffusion Monte Carlo calculations (DMC). DMC is a way of
solving exactly the many-body Schroedinger equation by means of 
a stochastic procedure. A good guess on the binding energy
and its wave function is however necessary. 
A carefully performed VMC calculation can aid in this context. 

The basic recipe in a VMC calculation consists of the following elements:

\begin{itemize}
\item Construct first a trial wave function $\psi_T(\bm{R},\bm{\alpha})$,  for a many-body system consisting of $N$ particles located at positions  $\bm{R}=(\bm{R}_1,\dots ,\bm{R}_N)$. The trial wave function depends on $\alpha$ variational parameters $\bm{\alpha}=(\alpha_1,\dots ,\alpha_M)$.

\item Then we evaluate the expectation value of the hamiltonian $H$ 
\end{itemize}

\noindent
\[
   E[H]=\langle H \rangle =
   \frac{\int d\bm{R}\Psi^{\ast}_{T}(\bm{R},\bm{\alpha})H(\bm{R})\Psi_{T}(\bm{R},\bm{\alpha})}
        {\int d\bm{R}\Psi^{\ast}_{T}(\bm{R},\bm{\alpha})\Psi_{T}(\bm{R},\bm{\alpha})}.
\]
\begin{itemize}
\item Thereafter we vary $\alpha$ according to some minimization algorithm and return to the first step.
\end{itemize}

\noindent
With a trial wave function $\psi_T(\bm{R})$ we can in turn construct the quantum mechanical probability distribution
\[
   P(\bm{R})= \frac{\left|\psi_T(\bm{R})\right|^2}{\int \left|\psi_T(\bm{R})\right|^2d\bm{R}}.
\]
This is our new probability distribution function  (PDF).
The approximation to the expectation value of the Hamiltonian is now 
\[
   E[H(\bm{\alpha})] = 
   \frac{\int d\bm{R}\Psi^{\ast}_T(\bm{R},\bm{\alpha})H(\bm{R})\Psi_T(\bm{R},\bm{\alpha})}
        {\int d\bm{R}\Psi^{\ast}_T(\bm{R},\bm{\alpha})\Psi_T(\bm{R},\bm{\alpha})}.
\]

Define a new quantity
\[
   E_L(\bm{R},\bm{\alpha})=\frac{1}{\psi_T(\bm{R},\bm{\alpha})}H\psi_T(\bm{R},\bm{\alpha}),
   \label{eq:locale1}
\]
called the local energy, which, together with our trial PDF yields
\[
  E[H(\bm{\alpha})]=\int P(\bm{R})E_L(\bm{R}) d\bm{R}\approx \frac{1}{N}\sum_{i=1}^NP(\bm{R_i},\bm{\alpha})E_L(\bm{R_i},\bm{\alpha})
  \label{eq:vmc1}
\]
with $N$ being the number of Monte Carlo samples.

The Algorithm for performing a variational Monte Carlo calculations runs thus as this

\begin{itemize}
   \item Initialisation: Fix the number of Monte Carlo steps. Choose an initial $\bm{R}$ and variational parameters $\alpha$ and calculate $\left|\psi_T^{\alpha}(\bm{R})\right|^2$. 

   \item Initialise the energy and the variance and start the Monte Carlo calculation.
\begin{itemize}

      \item Calculate  a trial position  $\bm{R}_p=\bm{R}+r*step$ where $r$ is a random variable $r \in [0,1]$.

      \item Metropolis algorithm to accept or reject this move  $w = P(\bm{R}_p)/P(\bm{R})$.

      \item If the step is accepted, then we set $\bm{R}=\bm{R}_p$. 

      \item Update averages

\end{itemize}

\noindent
   \item Finish and compute final averages.
\end{itemize}

\noindent
Observe that the jumping in space is governed by the variable \emph{step}. This is Called brute-force sampling.
Need importance sampling to get more relevant sampling, see lectures below.

\paragraph{Quantum Monte Carlo: hydrogen atom.}
The radial Schroedinger equation for the hydrogen atom can be
written as
\[
-\frac{\hbar^2}{2m}\frac{\partial^2 u(r)}{\partial r^2}-
\left(\frac{ke^2}{r}-\frac{\hbar^2l(l+1)}{2mr^2}\right)u(r)=Eu(r),
\]
or with dimensionless variables
\[
-\frac{1}{2}\frac{\partial^2 u(\rho)}{\partial \rho^2}-
\frac{u(\rho)}{\rho}+\frac{l(l+1)}{2\rho^2}u(\rho)-\lambda u(\rho)=0,
\label{eq:hydrodimless1}
\]
with the hamiltonian
\[
H=-\frac{1}{2}\frac{\partial^2 }{\partial \rho^2}-
\frac{1}{\rho}+\frac{l(l+1)}{2\rho^2}.
\]
Use variational parameter $\alpha$ in the trial
wave function 
\[
   u_T^{\alpha}(\rho)=\alpha\rho e^{-\alpha\rho}. 
   \label{eq:trialhydrogen}
\]

Inserting this wave function into the expression for the
local energy $E_L$ gives
\[
   E_L(\rho)=-\frac{1}{\rho}-
              \frac{\alpha}{2}\left(\alpha-\frac{2}{\rho}\right).
\]
A simple variational Monte Carlo calculation results in

\begin{quote}
\begin{tabular}{cccc}
\hline
\multicolumn{1}{c}{ $\alpha$ } & \multicolumn{1}{c}{ $\langle H \rangle $ } & \multicolumn{1}{c}{ $\sigma^2$ } & \multicolumn{1}{c}{ $\sigma/\sqrt{N}$ } \\
\hline
7.00000E-01 & -4.57759E-01         & 4.51201E-02 & 6.71715E-04       \\
8.00000E-01 & -4.81461E-01         & 3.05736E-02 & 5.52934E-04       \\
9.00000E-01 & -4.95899E-01         & 8.20497E-03 & 2.86443E-04       \\
1.00000E-00 & -5.00000E-01         & 0.00000E+00 & 0.00000E+00       \\
1.10000E+00 & -4.93738E-01         & 1.16989E-02 & 3.42036E-04       \\
1.20000E+00 & -4.75563E-01         & 8.85899E-02 & 9.41222E-04       \\
1.30000E+00 & -4.54341E-01         & 1.45171E-01 & 1.20487E-03       \\
\hline
\end{tabular}
\end{quote}

\noindent
We note that at $\alpha=1$ we obtain the exact
result, and the variance is zero, as it should. The reason is that 
we then have the exact wave function, and the action of the hamiltionan
on the wave function
\[
   H\psi = \mathrm{constant}\times \psi,
\]
yields just a constant. The integral which defines various 
expectation values involving moments of the hamiltonian becomes then
\[
   \langle H^n \rangle =
   \frac{\int d\bm{R}\Psi^{\ast}_T(\bm{R})H^n(\bm{R})\Psi_T(\bm{R})}
        {\int d\bm{R}\Psi^{\ast}_T(\bm{R})\Psi_T(\bm{R})}=
\mathrm{constant}\times\frac{\int d\bm{R}\Psi^{\ast}_T(\bm{R})\Psi_T(\bm{R})}
        {\int d\bm{R}\Psi^{\ast}_T(\bm{R})\Psi_T(\bm{R})}=\mathrm{constant}.
\]
\textbf{This gives an important information: the exact wave function leads to zero variance!}
Variation is then performed by minimizing both the energy and the variance.

For bosons in a harmonic oscillator-like  trap we will use is a spherical (S)
 or an elliptical (E) harmonic trap in one, two and finally three
 dimensions, with the latter given by
 \begin{equation}
 V_{ext}(\mathbf{r}) = \Bigg\{
 \begin{array}{ll}
	 \frac{1}{2}m\omega_{ho}^2r^2 & (S)\\
 \strut
	 \frac{1}{2}m[\omega_{ho}^2(x^2+y^2) + \omega_z^2z^2] & (E)
 \label{trap_eqn}
 \end{array}
 \end{equation}
where (S) stands for symmetric and 
\begin{equation}
     \hat{H} = \sum_i^N \left(
	 \frac{-\hbar^2}{2m}
	 { \bigtriangledown }_{i}^2 +
	 V_{ext}({\bf{r}}_i)\right)  +
	 \sum_{i<j}^{N} V_{int}({\bf{r}}_i,{\bf{r}}_j),
\end{equation}
as the two-body Hamiltonian of the system.  

 We will represent the inter-boson interaction by a pairwise, repulsive potential
\begin{equation}
 V_{int}(|\mathbf{r}_i-\mathbf{r}_j|) =  \Bigg\{
 \begin{array}{ll}
	 \infty & {|\mathbf{r}_i-\mathbf{r}_j|} \leq {a}\\
	 0 & {|\mathbf{r}_i-\mathbf{r}_j|} > {a}
 \end{array}
 \end{equation}
 where $a$ is the so-called hard-core diameter of the bosons.
 Clearly, $V_{int}(|\mathbf{r}_i-\mathbf{r}_j|)$ is zero if the bosons are
 separated by a distance $|\mathbf{r}_i-\mathbf{r}_j|$ greater than $a$ but
 infinite if they attempt to come within a distance $|\mathbf{r}_i-\mathbf{r}_j| \leq a$.

 Our trial wave function for the ground state with $N$ atoms is given by
 \begin{equation}
 \Psi_T(\mathbf{R})=\Psi_T(\mathbf{r}_1, \mathbf{r}_2, \dots \mathbf{r}_N,\alpha,\beta)=\prod_i g(\alpha,\beta,\mathbf{r}_i)\prod_{i<j}f(a,|\mathbf{r}_i-\mathbf{r}_j|),
 \label{eq:trialwf}
 \end{equation}
 where $\alpha$ and $\beta$ are variational parameters. The
 single-particle wave function is proportional to the harmonic
 oscillator function for the ground state
\begin{equation}
    g(\alpha,\beta,\mathbf{r}_i)= \exp{[-\alpha(x_i^2+y_i^2+\beta z_i^2)]}.
 \end{equation}

For spherical traps we have $\beta = 1$ and for non-interacting
bosons ($a=0$) we have $\alpha = 1/2a_{ho}^2$.  The correlation wave
 function is
 \begin{equation}
    f(a,|\mathbf{r}_i-\mathbf{r}_j|)=\Bigg\{
 \begin{array}{ll}
	 0 & {|\mathbf{r}_i-\mathbf{r}_j|} \leq {a}\\
	 (1-\frac{a}{|\mathbf{r}_i-\mathbf{r}_j|}) & {|\mathbf{r}_i-\mathbf{r}_j|} > {a}.
 \end{array}
 \end{equation}  

\paragraph{A simple Python code that solves the two-boson or two-fermion case in two-dimensions.}













































































































\begin{minted}[fontsize=\fontsize{9pt}{9pt},linenos=false,mathescape,baselinestretch=1.0,fontfamily=tt,xleftmargin=7mm]{python}
# Importing various packages
from math import exp, sqrt
from random import random, seed
import numpy as np
import matplotlib.pyplot as plt
from mpl_toolkits.mplot3d import Axes3D
from matplotlib import cm
from matplotlib.ticker import LinearLocator, FormatStrFormatter
import sys

#Trial wave function for quantum dots in two dims
def WaveFunction(r,alpha,beta):
    r1 = r[0,0]**2 + r[0,1]**2
    r2 = r[1,0]**2 + r[1,1]**2
    r12 = sqrt((r[0,0]-r[1,0])**2 + (r[0,1]-r[1,1])**2)
    deno = r12/(1+beta*r12)
    return exp(-0.5*alpha*(r1+r2)+deno)

#Local energy  for quantum dots in two dims, using analytical local energy
def LocalEnergy(r,alpha,beta):
    
    r1 = (r[0,0]**2 + r[0,1]**2)
    r2 = (r[1,0]**2 + r[1,1]**2)
    r12 = sqrt((r[0,0]-r[1,0])**2 + (r[0,1]-r[1,1])**2)
    deno = 1.0/(1+beta*r12)
    deno2 = deno*deno
    return 0.5*(1-alpha*alpha)*(r1 + r2) +2.0*alpha + 1.0/r12+deno2*(alpha*r12-deno2+2*beta*deno-1.0/r12)

# The Monte Carlo sampling with the Metropolis algo
def MonteCarloSampling():

    NumberMCcycles= 100000
    StepSize = 1.0
    # positions
    PositionOld = np.zeros((NumberParticles,Dimension), np.double)
    PositionNew = np.zeros((NumberParticles,Dimension), np.double)
    # seed for rng generator
    seed()
    # start variational parameter
    alpha = 0.9
    for ia in range(MaxVariations):
        alpha += .025
        AlphaValues[ia] = alpha
        beta = 0.2 
        for jb in range(MaxVariations):
            beta += .01
            BetaValues[jb] = beta
            energy = energy2 = 0.0
            DeltaE = 0.0
            #Initial position
            for i in range(NumberParticles):
                for j in range(Dimension):
                    PositionOld[i,j] = StepSize * (random() - .5)
            wfold = WaveFunction(PositionOld,alpha,beta)

            #Loop over MC MCcycles
            for MCcycle in range(NumberMCcycles):
                #Trial position
                for i in range(NumberParticles):
                    for j in range(Dimension):
                        PositionNew[i,j] = PositionOld[i,j] + StepSize * (random() - .5)
                wfnew = WaveFunction(PositionNew,alpha,beta)

                #Metropolis test to see whether we accept the move
                if random() < wfnew**2 / wfold**2:
                   PositionOld = PositionNew.copy()
                   wfold = wfnew
                   DeltaE = LocalEnergy(PositionOld,alpha,beta)
                energy += DeltaE
                energy2 += DeltaE**2

            #We calculate mean, variance and error ...
            energy /= NumberMCcycles
            energy2 /= NumberMCcycles
            variance = energy2 - energy**2
            error = sqrt(variance/NumberMCcycles)
            Energies[ia,jb] = energy    
    return Energies, AlphaValues, BetaValues


#Here starts the main program with variable declarations
NumberParticles = 2
Dimension = 2
MaxVariations = 10
Energies = np.zeros((MaxVariations,MaxVariations))
AlphaValues = np.zeros(MaxVariations)
BetaValues = np.zeros(MaxVariations)
(Energies, AlphaValues, BetaValues) = MonteCarloSampling()

# Prepare for plots
fig = plt.figure()
ax = fig.gca(projection='3d')
# Plot the surface.
X, Y = np.meshgrid(AlphaValues, BetaValues)
surf = ax.plot_surface(X, Y, Energies,cmap=cm.coolwarm,linewidth=0, antialiased=False)
# Customize the z axis.
zmin = np.matrix(Energies).min()
zmax = np.matrix(Energies).max()
ax.set_zlim(zmin, zmax)
ax.set_xlabel(r'$\alpha$')
ax.set_ylabel(r'$\beta$')
ax.set_zlabel(r'$\langle E \rangle$')
ax.zaxis.set_major_locator(LinearLocator(10))
ax.zaxis.set_major_formatter(FormatStrFormatter('%.02f'))
# Add a color bar which maps values to colors.
fig.colorbar(surf, shrink=0.5, aspect=5)
plt.show()


\end{minted}


\subsection*{Quantum Monte Carlo: the helium atom}

The helium atom consists of two electrons and a nucleus with
charge $Z=2$. 
The contribution  
to the potential energy due to the attraction from the nucleus is
\[
   -\frac{2ke^2}{r_1}-\frac{2ke^2}{r_2},
\] 
and if we add the repulsion arising from the two 
interacting electrons, we obtain the potential energy
\[
 V(r_1, r_2)=-\frac{2ke^2}{r_1}-\frac{2ke^2}{r_2}+
               \frac{ke^2}{r_{12}},
\]
with the electrons separated at a distance 
$r_{12}=|\bm{r}_1-\bm{r}_2|$.

The hamiltonian becomes then
\[
   \hat{H}=-\frac{\hbar^2\nabla_1^2}{2m}-\frac{\hbar^2\nabla_2^2}{2m}
          -\frac{2ke^2}{r_1}-\frac{2ke^2}{r_2}+
               \frac{ke^2}{r_{12}},
\]
and  Schroedingers equation reads
\[
   \hat{H}\psi=E\psi.
\]
All observables are evaluated with respect to the probability distribution
\[
   P(\bm{R})= \frac{\left|\psi_T(\bm{R})\right|^2}{\int \left|\psi_T(\bm{R})\right|^2d\bm{R}}.
\]
generated by the trial wave function.   
The trial wave function must approximate an exact 
eigenstate in order that accurate results are to be obtained. 

Choice of trial wave function for Helium:
Assume $r_1 \rightarrow 0$.
\[
   E_L(\bm{R})=\frac{1}{\psi_T(\bm{R})}H\psi_T(\bm{R})=
     \frac{1}{\psi_T(\bm{R})}\left(-\frac{1}{2}\nabla^2_1
     -\frac{Z}{r_1}\right)\psi_T(\bm{R}) + \mathrm{finite \hspace{0.1cm}terms}.
\]
\[ 
    E_L(R)=
    \frac{1}{\mathbf{R}_T(r_1)}\left(-\frac{1}{2}\frac{d^2}{dr_1^2}-
     \frac{1}{r_1}\frac{d}{dr_1}
     -\frac{Z}{r_1}\right)\mathbf{R}_T(r_1) + \mathrm{finite\hspace{0.1cm} terms}
\]
For small values of $r_1$, the terms which dominate are
\[ 
    \lim_{r_1 \rightarrow 0}E_L(R)=
    \frac{1}{\mathbf{R}_T(r_1)}\left(-
     \frac{1}{r_1}\frac{d}{dr_1}
     -\frac{Z}{r_1}\right)\mathbf{R}_T(r_1),
\]
since the second derivative does not diverge due to the finiteness of  $\Psi$ at the origin.

This results in
\[
     \frac{1}{\mathbf{R}_T(r_1)}\frac{d \mathbf{R}_T(r_1)}{dr_1}=-Z,
\]
and
\[
   \mathbf{R}_T(r_1)\propto e^{-Zr_1}.
\]
A similar condition applies to electron 2 as well. 
For orbital momenta $l > 0$ we have 
\[
     \frac{1}{\mathbf{R}_T(r)}\frac{d \mathbf{R}_T(r)}{dr}=-\frac{Z}{l+1}.
\]
Similarly, studying the case $r_{12}\rightarrow 0$ we can write 
a possible trial wave function as
\[
   \psi_T(\bm{R})=e^{-\alpha(r_1+r_2)}e^{\beta r_{12}}.
    \label{eq:wavehelium2}
\]
The last equation can be generalized to
\[
   \psi_T(\bm{R})=\phi(\bm{r}_1)\phi(\bm{r}_2)\dots\phi(\bm{r}_N)
                   \prod_{i < j}f(r_{ij}),
\]
for a system with $N$ electrons or particles. 

During the development of our code we need to make several checks. It is also very instructive to compute a closed form expression for the local energy. Since our wave function is rather simple  it is straightforward
to find an analytic expressions.  Consider first the case of the simple helium function 
\[
   \Psi_T(\bm{r}_1,\bm{r}_2) = e^{-\alpha(r_1+r_2)}
\]
The local energy is for this case 
\[ 
E_{L1} = \left(\alpha-Z\right)\left(\frac{1}{r_1}+\frac{1}{r_2}\right)+\frac{1}{r_{12}}-\alpha^2
\]
which gives an expectation value for the local energy given by
\[
\langle E_{L1} \rangle = \alpha^2-2\alpha\left(Z-\frac{5}{16}\right)
\]

With closed form formulae we  can speed up the computation of the correlation. In our case
we write it as 
\[
\Psi_C= \exp{\left\{\sum_{i < j}\frac{ar_{ij}}{1+\beta r_{ij}}\right\}},
\]
which means that the gradient needed for the so-called quantum force and local energy 
can be calculated analytically.
This will speed up your code since the computation of the correlation part and the Slater determinant are the most 
time consuming parts in your code.  

We will refer to this correlation function as $\Psi_C$ or the \emph{linear Pade-Jastrow}.

We can test this by computing the local energy for our helium wave function
\[
   \psi_{T}(\bm{r}_1,\bm{r}_2) = 
   \exp{\left(-\alpha(r_1+r_2)\right)}
   \exp{\left(\frac{r_{12}}{2(1+\beta r_{12})}\right)}, 
\]
with $\alpha$ and $\beta$ as variational parameters.

The local energy is for this case 
\[ 
E_{L2} = E_{L1}+\frac{1}{2(1+\beta r_{12})^2}\left\{\frac{\alpha(r_1+r_2)}{r_{12}}(1-\frac{\bm{r}_1\bm{r}_2}{r_1r_2})-\frac{1}{2(1+\beta r_{12})^2}-\frac{2}{r_{12}}+\frac{2\beta}{1+\beta r_{12}}\right\}
\]
It is very useful to test your code against these expressions. It means also that you don't need to
compute a derivative numerically as discussed in the code example below. 

For the computation of various derivatives with different types of wave functions, you will find it useful to use python with symbolic python, that is sympy, see \href{{http://docs.sympy.org/latest/index.html}}{online manual}.  Using sympy allows you autogenerate both Latex code as well c++, python or Fortran codes. Here you will find some simple examples. We choose 
the $2s$ hydrogen-orbital  (not normalized) as an example
\[
 \phi_{2s}(\bm{r}) = (Zr - 2)\exp{-(\frac{1}{2}Zr)},
\]
with $ r^2 = x^2 + y^2 + z^2$.









\begin{minted}[fontsize=\fontsize{9pt}{9pt},linenos=false,mathescape,baselinestretch=1.0,fontfamily=tt,xleftmargin=7mm]{python}
from sympy import symbols, diff, exp, sqrt
x, y, z, Z = symbols('x y z Z')
r = sqrt(x*x + y*y + z*z)
r
phi = (Z*r - 2)*exp(-Z*r/2)
phi
diff(phi, x)

\end{minted}

This doesn't look very nice, but sympy provides several functions that allow for improving and simplifying the output.

We can improve our output by factorizing and substituting expressions









\begin{minted}[fontsize=\fontsize{9pt}{9pt},linenos=false,mathescape,baselinestretch=1.0,fontfamily=tt,xleftmargin=7mm]{python}
from sympy import symbols, diff, exp, sqrt, factor, Symbol, printing
x, y, z, Z = symbols('x y z Z')
r = sqrt(x*x + y*y + z*z)
phi = (Z*r - 2)*exp(-Z*r/2)
R = Symbol('r') #Creates a symbolic equivalent of r
#print latex and c++ code
print printing.latex(diff(phi, x).factor().subs(r, R))
print printing.ccode(diff(phi, x).factor().subs(r, R))

\end{minted}


We can in turn look at second derivatives












\begin{minted}[fontsize=\fontsize{9pt}{9pt},linenos=false,mathescape,baselinestretch=1.0,fontfamily=tt,xleftmargin=7mm]{python}
from sympy import symbols, diff, exp, sqrt, factor, Symbol, printing
x, y, z, Z = symbols('x y z Z')
r = sqrt(x*x + y*y + z*z)
phi = (Z*r - 2)*exp(-Z*r/2)
R = Symbol('r') #Creates a symbolic equivalent of r
(diff(diff(phi, x), x) + diff(diff(phi, y), y) + diff(diff(phi, z), z)).factor().subs(r, R)
# Collect the Z values
(diff(diff(phi, x), x) + diff(diff(phi, y), y) +diff(diff(phi, z), z)).factor().collect(Z).subs(r, R)
# Factorize also the r**2 terms
(diff(diff(phi, x), x) + diff(diff(phi, y), y) + diff(diff(phi, z), z)).factor().collect(Z).subs(r, R).subs(r**2, R**2).factor()
print printing.ccode((diff(diff(phi, x), x) + diff(diff(phi, y), y) + diff(diff(phi, z), z)).factor().collect(Z).subs(r, R).subs(r**2, R**2).factor())

\end{minted}

With some practice this allows one to be able to check one's own calculation and translate automatically into code lines.

\subsection*{The Metropolis algorithm}

The Metropolis algorithm , see \href{{http://scitation.aip.org/content/aip/journal/jcp/21/6/10.1063/1.1699114}}{the original article} was invented by Metropolis et. al
and is often simply called the Metropolis algorithm.
It is a method to sample a normalized probability
distribution by a stochastic process. We define $\mathbf{P}_i^{(n)}$ to
be the probability for finding the system in the state $i$ at step $n$.
The algorithm is then

\begin{itemize}
\item Sample a possible new state $j$ with some probability $T_{i\rightarrow j}$.

\item Accept the new state $j$ with probability $A_{i \rightarrow j}$ and use it as the next sample. With probability $1-A_{i\rightarrow j}$ the move is rejected and the original state $i$ is used again as a sample.
\end{itemize}

\noindent
We wish to derive the required properties of $T$ and $A$ such that
$\mathbf{P}_i^{(n\rightarrow \infty)} \rightarrow p_i$ so that starting
from any distribution, the method converges to the correct distribution.
Note that the description here is for a discrete probability distribution.
Replacing probabilities $p_i$ with expressions like $p(x_i)dx_i$ will
take all of these over to the corresponding continuum expressions.

The dynamical equation for $\mathbf{P}_i^{(n)}$ can be written directly from
the description above. The probability of being in the state $i$ at step $n$
is given by the probability of being in any state $j$ at the previous step,
and making an accepted transition to $i$ added to the probability of
being in the state $i$, making a transition to any state $j$ and
rejecting the move:
\[
\mathbf{P}^{(n)}_i = \sum_j \left [
\mathbf{P}^{(n-1)}_jT_{j\rightarrow i} A_{j\rightarrow i} 
+\mathbf{P}^{(n-1)}_iT_{i\rightarrow j}\left ( 1- A_{i\rightarrow j} \right)
\right ] \,.
\]
Since the probability of making some transition must be 1,
$\sum_j T_{i\rightarrow j} = 1$, and the above equation becomes
\[
\mathbf{P}^{(n)}_i = \mathbf{P}^{(n-1)}_i +
 \sum_j \left [
\mathbf{P}^{(n-1)}_jT_{j\rightarrow i} A_{j\rightarrow i} 
-\mathbf{P}^{(n-1)}_iT_{i\rightarrow j}A_{i\rightarrow j}
\right ] \,.
\]

For large $n$ we require that $\mathbf{P}^{(n\rightarrow \infty)}_i = p_i$,
the desired probability distribution. Taking this limit, gives the
balance requirement
\[
 \sum_j \left [
p_jT_{j\rightarrow i} A_{j\rightarrow i}
-p_iT_{i\rightarrow j}A_{i\rightarrow j}
\right ] = 0 \,.
\]
The balance requirement is very weak. Typically the much stronger detailed
balance requirement is enforced, that is rather than the sum being
set to zero, we set each term separately to zero and use this
to determine the acceptance probabilities. Rearranging, the result is
\[
\frac{ A_{j\rightarrow i}}{A_{i\rightarrow j}}
= \frac{p_iT_{i\rightarrow j}}{ p_jT_{j\rightarrow i}} \,.
\]

The Metropolis choice is to maximize the $A$ values, that is
\[
A_{j \rightarrow i} = \min \left ( 1,
\frac{p_iT_{i\rightarrow j}}{ p_jT_{j\rightarrow i}}\right ).
\]
Other choices are possible, but they all correspond to multilplying
$A_{i\rightarrow j}$ and $A_{j\rightarrow i}$ by the same constant
smaller than unity.\footnote{The penalty function method uses just such
a factor to compensate for $p_i$ that are evaluated stochastically
and are therefore noisy.}

Having chosen the acceptance probabilities, we have guaranteed that
if the  $\mathbf{P}_i^{(n)}$ has equilibrated, that is if it is equal to $p_i$,
it will remain equilibrated. Next we need to find the circumstances for
convergence to equilibrium.

The dynamical equation can be written as
\[
\mathbf{P}^{(n)}_i = \sum_j M_{ij}\mathbf{P}^{(n-1)}_j
\]
with the matrix $M$ given by
\[
M_{ij} = \delta_{ij}\left [ 1 -\sum_k T_{i\rightarrow k} A_{i \rightarrow k}
\right ] + T_{j\rightarrow i} A_{j\rightarrow i} \,.
\]
Summing over $i$ shows that $\sum_i M_{ij} = 1$, and since
$\sum_k T_{i\rightarrow k} = 1$, and $A_{i \rightarrow k} \leq 1$, the
elements of the matrix satisfy $M_{ij} \geq 0$. The matrix $M$ is therefore
a stochastic matrix.

The Metropolis method is simply the power method for computing the
right eigenvector of $M$ with the largest magnitude eigenvalue.
By construction, the correct probability distribution is a right eigenvector
with eigenvalue 1. Therefore, for the Metropolis method to converge
to this result, we must show that $M$ has only one eigenvalue with this
magnitude, and all other eigenvalues are smaller.

\subsection*{Importance sampling}

We need to replace the brute force
Metropolis algorithm with a walk in coordinate space biased by the trial wave function.
This approach is based on the Fokker-Planck equation and the Langevin equation for generating a trajectory in coordinate space.  The link between the Fokker-Planck equation and the Langevin equations are explained, only partly, in the slides below.
An excellent reference on topics like Brownian motion, Markov chains, the Fokker-Planck equation and the Langevin equation is the text by  \href{{http://www.elsevier.com/books/stochastic-processes-in-physics-and-chemistry/van-kampen/978-0-444-52965-7}}{Van Kampen}
Here we will focus first on the implementation part first.

For a diffusion process characterized by a time-dependent probability density $P(x,t)$ in one dimension the Fokker-Planck
equation reads (for one particle /walker) 
\[
   \frac{\partial P}{\partial t} = D\frac{\partial }{\partial x}\left(\frac{\partial }{\partial x} -F\right)P(x,t),
\]
where $F$ is a drift term and $D$ is the diffusion coefficient. 

The new positions in coordinate space are given as the solutions of the Langevin equation using Euler's method, namely,
we go from the Langevin equation
\[ 
   \frac{\partial x(t)}{\partial t} = DF(x(t)) +\eta,
\]
with $\eta$ a random variable,
yielding a new position 
\[
   y = x+DF(x)\Delta t +\xi\sqrt{\Delta t},
\]
where $\xi$ is gaussian random variable and $\Delta t$ is a chosen time step. 
The quantity $D$ is, in atomic units, equal to $1/2$ and comes from the factor $1/2$ in the kinetic energy operator. Note that $\Delta t$ is to be viewed as a parameter. Values of $\Delta t \in [0.001,0.01]$ yield in general rather stable values of the ground state energy.  

The process of isotropic diffusion characterized by a time-dependent probability density $P(\mathbf{x},t)$ obeys (as an approximation) the so-called Fokker-Planck equation 
\[
   \frac{\partial P}{\partial t} = \sum_i D\frac{\partial }{\partial \mathbf{x_i}}\left(\frac{\partial }{\partial \mathbf{x_i}} -\mathbf{F_i}\right)P(\mathbf{x},t),
\]
where $\mathbf{F_i}$ is the $i^{th}$ component of the drift term (drift velocity) caused by an external potential, and $D$ is the diffusion coefficient. The convergence to a stationary probability density can be obtained by setting the left hand side to zero. The resulting equation will be satisfied if and only if all the terms of the sum are equal zero,
\[
\frac{\partial^2 P}{\partial {\mathbf{x_i}^2}} = P\frac{\partial}{\partial {\mathbf{x_i}}}\mathbf{F_i} + \mathbf{F_i}\frac{\partial}{\partial {\mathbf{x_i}}}P.
\]

The drift vector should be of the form $\mathbf{F} = g(\mathbf{x}) \frac{\partial P}{\partial \mathbf{x}}$. Then,
\[
\frac{\partial^2 P}{\partial {\mathbf{x_i}^2}} = P\frac{\partial g}{\partial P}\left( \frac{\partial P}{\partial {\mathbf{x}_i}}  \right)^2 + P g \frac{\partial ^2 P}{\partial {\mathbf{x}_i^2}}  + g \left( \frac{\partial P}{\partial {\mathbf{x}_i}}  \right)^2.
\]
The condition of stationary density means that the left hand side equals zero. In other words, the terms containing first and second derivatives have to cancel each other. It is possible only if $g = \frac{1}{P}$, which yields
\[
\mathbf{F} = 2\frac{1}{\Psi_T}\nabla\Psi_T,
\]
which is known as the so-called \emph{quantum force}. This term is responsible for pushing the walker towards regions of configuration space where the trial wave function is large, increasing the efficiency of the simulation in contrast to the Metropolis algorithm where the walker has the same probability of moving in every direction.

The Fokker-Planck equation yields a (the solution to the equation) transition probability given by the Green's function
\[
  G(y,x,\Delta t) = \frac{1}{(4\pi D\Delta t)^{3N/2}} \exp{\left(-(y-x-D\Delta t F(x))^2/4D\Delta t\right)}
\]
which in turn means that our brute force Metropolis algorithm
\[ 
    A(y,x) = \mathrm{min}(1,q(y,x))),
\]
with $q(y,x) = |\Psi_T(y)|^2/|\Psi_T(x)|^2$ is now replaced by the \href{{http://scitation.aip.org/content/aip/journal/jcp/21/6/10.1063/1.1699114}}{Metropolis-Hastings algorithm} as well as \href{{http://biomet.oxfordjournals.org/content/57/1/97.abstract}}{Hasting's article}, 
\[
q(y,x) = \frac{G(x,y,\Delta t)|\Psi_T(y)|^2}{G(y,x,\Delta t)|\Psi_T(x)|^2}
\]

\subsection*{Importance sampling, program elements}

The general derivative formula of the Jastrow factor is (the subscript $C$ stands for Correlation)
\[
\frac{1}{\Psi_C}\frac{\partial \Psi_C}{\partial x_k} =
\sum_{i=1}^{k-1}\frac{\partial g_{ik}}{\partial x_k}
+
\sum_{i=k+1}^{N}\frac{\partial g_{ki}}{\partial x_k}
\]
However, 
with our written in way which can be reused later as
\[
\Psi_C=\prod_{i< j}g(r_{ij})= \exp{\left\{\sum_{i<j}f(r_{ij})\right\}},
\]
the gradient needed for the quantum force and local energy is easy to compute.  
The function $f(r_{ij})$ will depends on the system under study. In the equations below we will keep this general form.

In the Metropolis/Hasting algorithm, the \emph{acceptance ratio} determines the probability for a particle  to be accepted at a new position. The ratio of the trial wave functions evaluated at the new and current positions is given by ($OB$ for the onebody  part)
\[
R \equiv \frac{\Psi_{T}^{new}}{\Psi_{T}^{old}} = 
\frac{\Psi_{OB}^{new}}{\Psi_{OB}^{old}}\frac{\Psi_{C}^{new}}{\Psi_{C}^{old}}
\]
Here $\Psi_{OB}$ is our onebody part (Slater determinant or product of boson single-particle states)  while $\Psi_{C}$ is our correlation function, or Jastrow factor. 
We need to optimize the $\nabla \Psi_T / \Psi_T$ ratio and the second derivative as well, that is
the $\mathbf{\nabla}^2 \Psi_T/\Psi_T$ ratio. The first is needed when we compute the so-called quantum force in importance sampling.
The second is needed when we compute the kinetic energy term of the local energy.
\[
\frac{\mathbf{\mathbf{\nabla}}  \Psi}{\Psi}  = \frac{\mathbf{\nabla}  (\Psi_{OB} \, \Psi_{C})}{\Psi_{OB} \, \Psi_{C}}  =  \frac{ \Psi_C \mathbf{\nabla}  \Psi_{OB} + \Psi_{OB} \mathbf{\nabla}  \Psi_{C}}{\Psi_{OB} \Psi_{C}} = \frac{\mathbf{\nabla}  \Psi_{OB}}{\Psi_{OB}} + \frac{\mathbf{\nabla}   \Psi_C}{ \Psi_C}
\]

The expectation value of the kinetic energy expressed in atomic units for electron $i$ is 
\[
 \langle \hat{K}_i \rangle = -\frac{1}{2}\frac{\langle\Psi|\mathbf{\nabla}_{i}^2|\Psi \rangle}{\langle\Psi|\Psi \rangle},
\]
\[
\hat{K}_i = -\frac{1}{2}\frac{\mathbf{\nabla}_{i}^{2} \Psi}{\Psi}.
\]

The second derivative which enters the definition of the local energy is 
\[
\frac{\mathbf{\nabla}^2 \Psi}{\Psi}=\frac{\mathbf{\nabla}^2 \Psi_{OB}}{\Psi_{OB}} + \frac{\mathbf{\nabla}^2  \Psi_C}{ \Psi_C} + 2 \frac{\mathbf{\nabla}  \Psi_{OB}}{\Psi_{OB}}\cdot\frac{\mathbf{\nabla}   \Psi_C}{ \Psi_C}
\]
We discuss here how to calculate these quantities in an optimal way,

We have defined the correlated function as
\[
\Psi_C=\prod_{i< j}g(r_{ij})=\prod_{i< j}^Ng(r_{ij})= \prod_{i=1}^N\prod_{j=i+1}^Ng(r_{ij}),
\]
with 
$r_{ij}=|\mathbf{r}_i-\mathbf{r}_j|=\sqrt{(x_i-x_j)^2+(y_i-y_j)^2+(z_i-z_j)^2}$ in three dimensions or
$r_{ij}=|\mathbf{r}_i-\mathbf{r}_j|=\sqrt{(x_i-x_j)^2+(y_i-y_j)^2}$ if we work with two-dimensional systems.

In our particular case we have
\[
\Psi_C=\prod_{i< j}g(r_{ij})=\exp{\left\{\sum_{i<j}f(r_{ij})\right\}}.
\]

The total number of different relative distances $r_{ij}$ is $N(N-1)/2$. In a matrix storage format, the relative distances  form a strictly upper triangular matrix
\[
 \mathbf{r} \equiv \begin{pmatrix}
  0 & r_{1,2} & r_{1,3} & \cdots & r_{1,N} \\
  \vdots & 0       & r_{2,3} & \cdots & r_{2,N} \\
  \vdots & \vdots  & 0  & \ddots & \vdots  \\
  \vdots & \vdots  & \vdots  & \ddots  & r_{N-1,N} \\
  0 & 0  & 0  & \cdots  & 0
 \end{pmatrix}.
\]
This applies to  $\mathbf{g} = \mathbf{g}(r_{ij})$ as well. 

In our algorithm we will move one particle  at the time, say the $kth$-particle.  This sampling will be seen to be particularly efficient when we are going to compute a Slater determinant. 

We have that the ratio between Jastrow factors $R_C$ is given by
\[
R_{C} = \frac{\Psi_{C}^\mathrm{new}}{\Psi_{C}^\mathrm{cur}} =
\prod_{i=1}^{k-1}\frac{g_{ik}^\mathrm{new}}{g_{ik}^\mathrm{cur}}
\prod_{i=k+1}^{N}\frac{ g_{ki}^\mathrm{new}} {g_{ki}^\mathrm{cur}}.
\]
For the Pade-Jastrow form
\[
 R_{C} = \frac{\Psi_{C}^\mathrm{new}}{\Psi_{C}^\mathrm{cur}} = 
\frac{\exp{U_{new}}}{\exp{U_{cur}}} = \exp{\Delta U},
\]
where
\[
\Delta U =
\sum_{i=1}^{k-1}\big(f_{ik}^\mathrm{new}-f_{ik}^\mathrm{cur}\big)
+
\sum_{i=k+1}^{N}\big(f_{ki}^\mathrm{new}-f_{ki}^\mathrm{cur}\big)
\]

One needs to develop a special algorithm 
that runs only through the elements of the upper triangular
matrix $\mathbf{g}$ and have $k$ as an index. 

The expression to be derived in the following is of interest when computing the quantum force and the kinetic energy. It has the form
\[
\frac{\mathbf{\nabla}_i\Psi_C}{\Psi_C} = \frac{1}{\Psi_C}\frac{\partial \Psi_C}{\partial x_i},
\]
for all dimensions and with $i$ running over all particles.

For the first derivative only $N-1$ terms survive the ratio because the $g$-terms that are not differentiated cancel with their corresponding ones in the denominator. Then,
\[
\frac{1}{\Psi_C}\frac{\partial \Psi_C}{\partial x_k} =
\sum_{i=1}^{k-1}\frac{1}{g_{ik}}\frac{\partial g_{ik}}{\partial x_k}
+
\sum_{i=k+1}^{N}\frac{1}{g_{ki}}\frac{\partial g_{ki}}{\partial x_k}.
\]
An equivalent equation is obtained for the exponential form after replacing $g_{ij}$ by $\exp(f_{ij})$, yielding:
\[
\frac{1}{\Psi_C}\frac{\partial \Psi_C}{\partial x_k} =
\sum_{i=1}^{k-1}\frac{\partial g_{ik}}{\partial x_k}
+
\sum_{i=k+1}^{N}\frac{\partial g_{ki}}{\partial x_k},
\]
with both expressions scaling as $\mathcal{O}(N)$.

Using the identity 
\[
\frac{\partial}{\partial x_i}g_{ij} = -\frac{\partial}{\partial x_j}g_{ij},
\]
we get expressions where all the derivatives acting on the particle  are represented by the \emph{second} index of $g$:
\[
\frac{1}{\Psi_C}\frac{\partial \Psi_C}{\partial x_k} =
\sum_{i=1}^{k-1}\frac{1}{g_{ik}}\frac{\partial g_{ik}}{\partial x_k}
-\sum_{i=k+1}^{N}\frac{1}{g_{ki}}\frac{\partial g_{ki}}{\partial x_i},
\]
and for the exponential case:
\[
\frac{1}{\Psi_C}\frac{\partial \Psi_C}{\partial x_k} =
\sum_{i=1}^{k-1}\frac{\partial g_{ik}}{\partial x_k}
-\sum_{i=k+1}^{N}\frac{\partial g_{ki}}{\partial x_i}.
\]

For correlation forms depending only on the scalar distances $r_{ij}$ we can use the chain rule. Noting that 
\[
\frac{\partial g_{ij}}{\partial x_j} = \frac{\partial g_{ij}}{\partial r_{ij}} \frac{\partial r_{ij}}{\partial x_j} = \frac{x_j - x_i}{r_{ij}} \frac{\partial g_{ij}}{\partial r_{ij}},
\]
we arrive at
\[
\frac{1}{\Psi_C}\frac{\partial \Psi_C}{\partial x_k} = 
\sum_{i=1}^{k-1}\frac{1}{g_{ik}} \frac{\mathbf{r_{ik}}}{r_{ik}} \frac{\partial g_{ik}}{\partial r_{ik}}
-\sum_{i=k+1}^{N}\frac{1}{g_{ki}}\frac{\mathbf{r_{ki}}}{r_{ki}}\frac{\partial g_{ki}}{\partial r_{ki}}.
\]

Note that for the Pade-Jastrow form we can set $g_{ij} \equiv g(r_{ij}) = e^{f(r_{ij})} = e^{f_{ij}}$ and 
\[
\frac{\partial g_{ij}}{\partial r_{ij}} = g_{ij} \frac{\partial f_{ij}}{\partial r_{ij}}.
\]
Therefore, 
\[
\frac{1}{\Psi_{C}}\frac{\partial \Psi_{C}}{\partial x_k} =
\sum_{i=1}^{k-1}\frac{\mathbf{r_{ik}}}{r_{ik}}\frac{\partial f_{ik}}{\partial r_{ik}}
-\sum_{i=k+1}^{N}\frac{\mathbf{r_{ki}}}{r_{ki}}\frac{\partial f_{ki}}{\partial r_{ki}},
\]
where 
\[
 \mathbf{r}_{ij} = |\mathbf{r}_j - \mathbf{r}_i| = (x_j - x_i)\mathbf{e}_1 + (y_j - y_i)\mathbf{e}_2 + (z_j - z_i)\mathbf{e}_3
\]
is the relative distance. 

The second derivative of the Jastrow factor divided by the Jastrow factor (the way it enters the kinetic energy) is
\[
\left[\frac{\mathbf{\nabla}^2 \Psi_C}{\Psi_C}\right]_x =\  
2\sum_{k=1}^{N}
\sum_{i=1}^{k-1}\frac{\partial^2 g_{ik}}{\partial x_k^2}\ +\ 
\sum_{k=1}^N
\left(
\sum_{i=1}^{k-1}\frac{\partial g_{ik}}{\partial x_k} -
\sum_{i=k+1}^{N}\frac{\partial g_{ki}}{\partial x_i}
\right)^2
\]

But we have a simple form for the function, namely
\[
\Psi_{C}=\prod_{i< j}\exp{f(r_{ij})},
\]
and it is easy to see that for particle  $k$
we have
\[
  \frac{\mathbf{\nabla}^2_k \Psi_C}{\Psi_C }=
\sum_{ij\ne k}\frac{(\mathbf{r}_k-\mathbf{r}_i)(\mathbf{r}_k-\mathbf{r}_j)}{r_{ki}r_{kj}}f'(r_{ki})f'(r_{kj})+
\sum_{j\ne k}\left( f''(r_{kj})+\frac{2}{r_{kj}}f'(r_{kj})\right)
\]

\subsection*{Importance sampling, Fokker-Planck and Langevin equations}

A stochastic process is simply a function of two variables, one is the time,
the other is a stochastic variable $X$, defined by specifying
\begin{itemize}
\item the set $\left\{x\right\}$ of possible values for $X$;

\item the probability distribution, $w_X(x)$,  over this set, or briefly $w(x)$
\end{itemize}

\noindent
The set of values $\left\{x\right\}$ for $X$ 
may be discrete, or continuous. If the set of
values is continuous, then $w_X (x)$ is a probability density so that 
$w_X (x)dx$
is the probability that one finds the stochastic variable $X$ to have values
in the range $[x, x + dx]$ .

     An arbitrary number of other stochastic variables may be derived from
$X$. For example, any $Y$ given by a mapping of $X$, is also a stochastic
variable. The mapping may also be time-dependent, that is, the mapping
depends on an additional variable $t$
\[
                              Y_X (t) = f (X, t) .
\]
The quantity $Y_X (t)$ is called a random function, or, since $t$ often is time,
a stochastic process. A stochastic process is a function of two variables,
one is the time, the other is a stochastic variable $X$. Let $x$ be one of the
possible values of $X$ then
\[
                               y(t) = f (x, t),
\]
is a function of $t$, called a sample function or realization of the process.
In physics one considers the stochastic process to be an ensemble of such
sample functions.

     For many physical systems initial distributions of a stochastic 
variable $y$ tend to equilibrium distributions: $w(y, t)\rightarrow w_0(y)$ 
as $t\rightarrow\infty$. In
equilibrium detailed balance constrains the transition rates
\[
     W(y\rightarrow y')w(y ) = W(y'\rightarrow y)w_0 (y),
\]
where $W(y'\rightarrow y)$ 
is the probability, per unit time, that the system changes
from a state $|y\rangle$ , characterized by the value $y$ 
for the stochastic variable $Y$ , to a state $|y'\rangle$.

Note that for a system in equilibrium the transition rate 
$W(y'\rightarrow y)$ and
the reverse $W(y\rightarrow y')$ may be very different. 

Consider, for instance, a simple
system that has only two energy levels $\epsilon_0 = 0$ and 
$\epsilon_1 = \Delta E$. 

For a system governed by the Boltzmann distribution we find (the partition function has been taken out)
\[
     W(0\rightarrow 1)\exp{-(\epsilon_0/kT)} = W(1\rightarrow 0)\exp{-(\epsilon_1/kT)}
\]
We get then
\[
     \frac{W(1\rightarrow 0)}{W(0 \rightarrow 1)}=\exp{-(\Delta E/kT)},
\]
which goes to zero when $T$ tends to zero.

If we assume a discrete set of events,
our initial probability
distribution function can be  given by 
\[
   w_i(0) = \delta_{i,0},
\]
and its time-development after a given time step $\Delta t=\epsilon$ is
\[ 
   w_i(t) = \sum_{j}W(j\rightarrow i)w_j(t=0).
\] 
The continuous analog to $w_i(0)$ is
\[
   w(\mathbf{x})\rightarrow \delta(\mathbf{x}),
\]
where we now have generalized the one-dimensional position $x$ to a generic-dimensional  
vector $\mathbf{x}$. The Kroenecker $\delta$ function is replaced by the $\delta$ distribution
function $\delta(\mathbf{x})$ at  $t=0$.  

The transition from a state $j$ to a state $i$ is now replaced by a transition
to a state with position $\mathbf{y}$ from a state with position $\mathbf{x}$. 
The discrete sum of transition probabilities can then be replaced by an integral
and we obtain the new distribution at a time $t+\Delta t$ as 
\[
   w(\mathbf{y},t+\Delta t)= \int W(\mathbf{y},t+\Delta t| \mathbf{x},t)w(\mathbf{x},t)d\mathbf{x},
\]
and after $m$ time steps we have
\[
   w(\mathbf{y},t+m\Delta t)= \int W(\mathbf{y},t+m\Delta t| \mathbf{x},t)w(\mathbf{x},t)d\mathbf{x}.
\]
When equilibrium is reached we have
\[
   w(\mathbf{y})= \int W(\mathbf{y}|\mathbf{x}, t)w(\mathbf{x})d\mathbf{x},
\]
that is no time-dependence. Note our change of notation for $W$

We can solve the equation for $w(\mathbf{y},t)$ by making a Fourier transform to
momentum space. 
The PDF $w(\mathbf{x},t)$ is related to its Fourier transform
$\tilde{w}(\mathbf{k},t)$ through
\[
   w(\mathbf{x},t) = \int_{-\infty}^{\infty}d\mathbf{k} \exp{(i\mathbf{kx})}\tilde{w}(\mathbf{k},t),
\]
and using the definition of the 
$\delta$-function 
\[
   \delta(\mathbf{x}) = \frac{1}{2\pi} \int_{-\infty}^{\infty}d\mathbf{k} \exp{(i\mathbf{kx})},
\]
 we see that
\[
   \tilde{w}(\mathbf{k},0)=1/2\pi.
\]

We can then use the Fourier-transformed diffusion equation 
\[
    \frac{\partial \tilde{w}(\mathbf{k},t)}{\partial t} = -D\mathbf{k}^2\tilde{w}(\mathbf{k},t),
\]
with the obvious solution
\[
   \tilde{w}(\mathbf{k},t)=\tilde{w}(\mathbf{k},0)\exp{\left[-(D\mathbf{k}^2t)\right)}=
    \frac{1}{2\pi}\exp{\left[-(D\mathbf{k}^2t)\right]}. 
\]

With the Fourier transform we obtain 
\[
   w(\mathbf{x},t)=\int_{-\infty}^{\infty}d\mathbf{k} \exp{\left[i\mathbf{kx}\right]}\frac{1}{2\pi}\exp{\left[-(D\mathbf{k}^2t)\right]}=
    \frac{1}{\sqrt{4\pi Dt}}\exp{\left[-(\mathbf{x}^2/4Dt)\right]}, 
\]
with the normalization condition
\[
   \int_{-\infty}^{\infty}w(\mathbf{x},t)d\mathbf{x}=1.
\]

The solution represents the probability of finding
our random walker at position $\mathbf{x}$ at time $t$ if the initial distribution 
was placed at $\mathbf{x}=0$ at $t=0$. 

There is another interesting feature worth observing. The discrete transition probability $W$
itself is given by a binomial distribution.
The results from the central limit theorem state that 
transition probability in the limit $n\rightarrow \infty$ converges to the normal 
distribution. It is then possible to show that
\[
    W(il-jl,n\epsilon)\rightarrow W(\mathbf{y},t+\Delta t|\mathbf{x},t)=
    \frac{1}{\sqrt{4\pi D\Delta t}}\exp{\left[-((\mathbf{y}-\mathbf{x})^2/4D\Delta t)\right]},
\]
and that it satisfies the normalization condition and is itself a solution
to the diffusion equation.

Let us now assume that we have three PDFs for times $t_0 < t' < t$, that is
$w(\mathbf{x}_0,t_0)$, $w(\mathbf{x}',t')$ and $w(\mathbf{x},t)$.
We have then  
\[
   w(\mathbf{x},t)= \int_{-\infty}^{\infty} W(\mathbf{x}.t|\mathbf{x}'.t')w(\mathbf{x}',t')d\mathbf{x}',
\]
and
\[
   w(\mathbf{x},t)= \int_{-\infty}^{\infty} W(\mathbf{x}.t|\mathbf{x}_0.t_0)w(\mathbf{x}_0,t_0)d\mathbf{x}_0,
\]
and
\[
   w(\mathbf{x}',t')= \int_{-\infty}^{\infty} W(\mathbf{x}'.t'|\mathbf{x}_0,t_0)w(\mathbf{x}_0,t_0)d\mathbf{x}_0.
\]

We can combine these equations and arrive at the famous Einstein-Smoluchenski-Kolmogorov-Chapman (ESKC) relation
\[
 W(\mathbf{x}t|\mathbf{x}_0t_0)  = \int_{-\infty}^{\infty} W(\mathbf{x},t|\mathbf{x}',t')W(\mathbf{x}',t'|\mathbf{x}_0,t_0)d\mathbf{x}'.
\]
We can replace the spatial dependence with a dependence upon say the velocity
(or momentum), that is we have
\[
 W(\mathbf{v},t|\mathbf{v}_0,t_0)  = \int_{-\infty}^{\infty} W(\mathbf{v},t|\mathbf{v}',t')W(\mathbf{v}',t'|\mathbf{v}_0,t_0)d\mathbf{x}'.
\]

We will now derive the Fokker-Planck equation. 
We start from the ESKC equation
\[
 W(\mathbf{x},t|\mathbf{x}_0,t_0)  = \int_{-\infty}^{\infty} W(\mathbf{x},t|\mathbf{x}',t')W(\mathbf{x}',t'|\mathbf{x}_0,t_0)d\mathbf{x}'.
\]
Define $s=t'-t_0$, $\tau=t-t'$ and $t-t_0=s+\tau$. We have then
\[
 W(\mathbf{x},s+\tau|\mathbf{x}_0)  = \int_{-\infty}^{\infty} W(\mathbf{x},\tau|\mathbf{x}')W(\mathbf{x}',s|\mathbf{x}_0)d\mathbf{x}'.
\]

Assume now that $\tau$ is very small so that we can make an expansion in terms of a small step $xi$, with $\mathbf{x}'=\mathbf{x}-\xi$, that is
\[
 W(\mathbf{x},s|\mathbf{x}_0)+\frac{\partial W}{\partial s}\tau +O(\tau^2) = \int_{-\infty}^{\infty} W(\mathbf{x},\tau|\mathbf{x}-\xi)W(\mathbf{x}-\xi,s|\mathbf{x}_0)d\mathbf{x}'.
\]
We assume that $W(\mathbf{x},\tau|\mathbf{x}-\xi)$ takes non-negligible values only when $\xi$ is small. This is just another way of stating the Master equation!!

We say thus that $\mathbf{x}$ changes only by a small amount in the time interval $\tau$. 
This means that we can make a Taylor expansion in terms of $\xi$, that is we
expand
\[
W(\mathbf{x},\tau|\mathbf{x}-\xi)W(\mathbf{x}-\xi,s|\mathbf{x}_0) =
\sum_{n=0}^{\infty}\frac{(-\xi)^n}{n!}\frac{\partial^n}{\partial x^n}\left[W(\mathbf{x}+\xi,\tau|\mathbf{x})W(\mathbf{x},s|\mathbf{x}_0)
\right].
\]

We can then rewrite the ESKC equation as 
\[
\frac{\partial W}{\partial s}\tau=-W(\mathbf{x},s|\mathbf{x}_0)+
\sum_{n=0}^{\infty}\frac{(-\xi)^n}{n!}\frac{\partial^n}{\partial x^n}
\left[W(\mathbf{x},s|\mathbf{x}_0)\int_{-\infty}^{\infty} \xi^nW(\mathbf{x}+\xi,\tau|\mathbf{x})d\xi\right].
\]
We have neglected higher powers of $\tau$ and have used that for $n=0$ 
we get simply $W(\mathbf{x},s|\mathbf{x}_0)$ due to normalization.

We say thus that $\mathbf{x}$ changes only by a small amount in the time interval $\tau$. 
This means that we can make a Taylor expansion in terms of $\xi$, that is we
expand
\[
W(\mathbf{x},\tau|\mathbf{x}-\xi)W(\mathbf{x}-\xi,s|\mathbf{x}_0) =
\sum_{n=0}^{\infty}\frac{(-\xi)^n}{n!}\frac{\partial^n}{\partial x^n}\left[W(\mathbf{x}+\xi,\tau|\mathbf{x})W(\mathbf{x},s|\mathbf{x}_0)
\right].
\]

We can then rewrite the ESKC equation as 
\[
\frac{\partial W(\mathbf{x},s|\mathbf{x}_0)}{\partial s}\tau=-W(\mathbf{x},s|\mathbf{x}_0)+
\sum_{n=0}^{\infty}\frac{(-\xi)^n}{n!}\frac{\partial^n}{\partial x^n}
\left[W(\mathbf{x},s|\mathbf{x}_0)\int_{-\infty}^{\infty} \xi^nW(\mathbf{x}+\xi,\tau|\mathbf{x})d\xi\right].
\]
We have neglected higher powers of $\tau$ and have used that for $n=0$ 
we get simply $W(\mathbf{x},s|\mathbf{x}_0)$ due to normalization.

We simplify the above by introducing the moments 
\[
M_n=\frac{1}{\tau}\int_{-\infty}^{\infty} \xi^nW(\mathbf{x}+\xi,\tau|\mathbf{x})d\xi=
\frac{\langle [\Delta x(\tau)]^n\rangle}{\tau},
\]
resulting in
\[
\frac{\partial W(\mathbf{x},s|\mathbf{x}_0)}{\partial s}=
\sum_{n=1}^{\infty}\frac{(-\xi)^n}{n!}\frac{\partial^n}{\partial x^n}
\left[W(\mathbf{x},s|\mathbf{x}_0)M_n\right].
\]

When $\tau \rightarrow 0$ we assume that $\langle [\Delta x(\tau)]^n\rangle \rightarrow 0$ more rapidly than $\tau$ itself if $n > 2$. 
When $\tau$ is much larger than the standard correlation time of 
system then $M_n$ for $n > 2$ can normally be neglected.
This means that fluctuations become negligible at large time scales.

If we neglect such terms we can rewrite the ESKC equation as 
\[
\frac{\partial W(\mathbf{x},s|\mathbf{x}_0)}{\partial s}=
-\frac{\partial M_1W(\mathbf{x},s|\mathbf{x}_0)}{\partial x}+
\frac{1}{2}\frac{\partial^2 M_2W(\mathbf{x},s|\mathbf{x}_0)}{\partial x^2}.
\]

In a more compact form we have
\[
\frac{\partial W}{\partial s}=
-\frac{\partial M_1W}{\partial x}+
\frac{1}{2}\frac{\partial^2 M_2W}{\partial x^2},
\]
which is the Fokker-Planck equation!  It is trivial to replace 
position with velocity (momentum).

Consider a particle  suspended in a liquid. On its path through the liquid it will continuously collide with the liquid molecules. Because on average the particle  will collide more often on the front side than on the back side, it will experience a systematic force proportional with its velocity, and directed opposite to its velocity. Besides this systematic force the particle  will experience a stochastic force  $\mathbf{F}(t)$. 
The equations of motion are 
\begin{itemize}
\item $\frac{d\mathbf{r}}{dt}=\mathbf{v}$ and 

\item $\frac{d\mathbf{v}}{dt}=-\xi \mathbf{v}+\mathbf{F}$.
\end{itemize}

\noindent
From hydrodynamics  we know that the friction constant  $\xi$ is given by
\[
\xi =6\pi \eta a/m 
\]
where $\eta$ is the viscosity  of the solvent and a is the radius of the particle .

Solving the second equation in the previous slide we get 
\[
\mathbf{v}(t)=\mathbf{v}_{0}e^{-\xi t}+\int_{0}^{t}d\tau e^{-\xi (t-\tau )}\mathbf{F }(\tau ). 
\]

If we want to get some useful information out of this, we have to average over all possible realizations of 
$\mathbf{F}(t)$, with the initial velocity as a condition. A useful quantity for example is
\[ 
\langle \mathbf{v}(t)\cdot \mathbf{v}(t)\rangle_{\mathbf{v}_{0}}=v_{0}^{-\xi 2t}
+2\int_{0}^{t}d\tau e^{-\xi (2t-\tau)}\mathbf{v}_{0}\cdot \langle \mathbf{F}(\tau )\rangle_{\mathbf{v}_{0}}
\]
\[  	  	
 +\int_{0}^{t}d\tau ^{\prime }\int_{0}^{t}d\tau e^{-\xi (2t-\tau -\tau ^{\prime })}
\langle \mathbf{F}(\tau )\cdot \mathbf{F}(\tau ^{\prime })\rangle_{ \mathbf{v}_{0}}.
\]

In order to continue we have to make some assumptions about the conditional averages of the stochastic forces. 
In view of the chaotic character of the stochastic forces the following 
assumptions seem to be appropriate
\[ 
\langle \mathbf{F}(t)\rangle=0, 
\]
and
\[
\langle \mathbf{F}(t)\cdot \mathbf{F}(t^{\prime })\rangle_{\mathbf{v}_{0}}=  C_{\mathbf{v}_{0}}\delta (t-t^{\prime }).
\] 	

We omit the subscript $\mathbf{v}_{0}$, when the quantity of interest turns out to be independent of $\mathbf{v}_{0}$. Using the last three equations we get
 \[
\langle \mathbf{v}(t)\cdot \mathbf{v}(t)\rangle_{\mathbf{v}_{0}}=v_{0}^{2}e^{-2\xi t}+\frac{C_{\mathbf{v}_{0}}}{2\xi }(1-e^{-2\xi t}).
\]
For large t this should be equal to 3kT/m, from which it follows that
\[
\langle \mathbf{F}(t)\cdot \mathbf{F}(t^{\prime })\rangle =6\frac{kT}{m}\xi \delta (t-t^{\prime }). 
\]
This result is called the fluctuation-dissipation theorem .

Integrating 
 \[ 
\mathbf{v}(t)=\mathbf{v}_{0}e^{-\xi t}+\int_{0}^{t}d\tau e^{-\xi (t-\tau )}\mathbf{F }(\tau ), 
\] 
we get
\[
\mathbf{r}(t)=\mathbf{r}_{0}+\mathbf{v}_{0}\frac{1}{\xi }(1-e^{-\xi t})+
\int_0^td\tau \int_0^{\tau}\tau ^{\prime } e^{-\xi (\tau -\tau ^{\prime })}\mathbf{F}(\tau ^{\prime }), 
\]
from which we calculate the mean square displacement 
\[
\langle ( \mathbf{r}(t)-\mathbf{r}_{0})^{2}\rangle _{\mathbf{v}_{0}}=\frac{v_0^2}{\xi}(1-e^{-\xi t})^{2}+\frac{3kT}{m\xi ^{2}}(2\xi t-3+4e^{-\xi t}-e^{-2\xi t}). 
\]

For very large $t$ this becomes
\[
\langle (\mathbf{r}(t)-\mathbf{r}_{0})^{2}\rangle =\frac{6kT}{m\xi }t 
\] 
from which we get the Einstein relation  
 \[ 
D= \frac{kT}{m\xi } 
\] 	
where we have used $\langle (\mathbf{r}(t)-\mathbf{r}_{0})^{2}\rangle =6Dt$.

\subsection*{Code example for two electrons in a quantum dots}
























































































































































\begin{minted}[fontsize=\fontsize{9pt}{9pt},linenos=false,mathescape,baselinestretch=1.0,fontfamily=tt,xleftmargin=7mm]{python}
# 2-electron VMC code for 2dim quantum dot with importance sampling
# Using gaussian rng for new positions and Metropolis- Hastings 
# No energy minimization
from math import exp, sqrt
from random import random, seed, normalvariate
import numpy as np
import matplotlib.pyplot as plt
from mpl_toolkits.mplot3d import Axes3D
from matplotlib import cm
from matplotlib.ticker import LinearLocator, FormatStrFormatter
import sys
from numba import jit,njit


#Read name of output file from command line
if len(sys.argv) == 2:
    outfilename = sys.argv[1]
else:
    print('\nError: Name of output file must be given as command line argument.\n')
outfile = open(outfilename,'w')

# Trial wave function for the 2-electron quantum dot in two dims
def WaveFunction(r,alpha,beta):
    r1 = r[0,0]**2 + r[0,1]**2
    r2 = r[1,0]**2 + r[1,1]**2
    r12 = sqrt((r[0,0]-r[1,0])**2 + (r[0,1]-r[1,1])**2)
    deno = r12/(1+beta*r12)
    return exp(-0.5*alpha*(r1+r2)+deno)

# Local energy  for the 2-electron quantum dot in two dims, using analytical local energy
def LocalEnergy(r,alpha,beta):
    
    r1 = (r[0,0]**2 + r[0,1]**2)
    r2 = (r[1,0]**2 + r[1,1]**2)
    r12 = sqrt((r[0,0]-r[1,0])**2 + (r[0,1]-r[1,1])**2)
    deno = 1.0/(1+beta*r12)
    deno2 = deno*deno
    return 0.5*(1-alpha*alpha)*(r1 + r2) +2.0*alpha + 1.0/r12+deno2*(alpha*r12-deno2+2*beta*deno-1.0/r12)

# Setting up the quantum force for the two-electron quantum dot, recall that it is a vector
def QuantumForce(r,alpha,beta):

    qforce = np.zeros((NumberParticles,Dimension), np.double)
    r12 = sqrt((r[0,0]-r[1,0])**2 + (r[0,1]-r[1,1])**2)
    deno = 1.0/(1+beta*r12)
    qforce[0,:] = -2*r[0,:]*alpha*(r[0,:]-r[1,:])*deno*deno/r12
    qforce[1,:] = -2*r[1,:]*alpha*(r[1,:]-r[0,:])*deno*deno/r12
    return qforce
    
# The Monte Carlo sampling with the Metropolis algo
# jit decorator tells Numba to compile this function.
# The argument types will be inferred by Numba when function is called.
@jit()
def MonteCarloSampling():

    NumberMCcycles= 100000
    # Parameters in the Fokker-Planck simulation of the quantum force
    D = 0.5
    TimeStep = 0.05
    # positions
    PositionOld = np.zeros((NumberParticles,Dimension), np.double)
    PositionNew = np.zeros((NumberParticles,Dimension), np.double)
    # Quantum force
    QuantumForceOld = np.zeros((NumberParticles,Dimension), np.double)
    QuantumForceNew = np.zeros((NumberParticles,Dimension), np.double)

    # seed for rng generator 
    seed()
    # start variational parameter  loops, two parameters here
    alpha = 0.9
    for ia in range(MaxVariations):
        alpha += .025
        AlphaValues[ia] = alpha
        beta = 0.2 
        for jb in range(MaxVariations):
            beta += .01
            BetaValues[jb] = beta
            energy = energy2 = 0.0
            DeltaE = 0.0
            #Initial position
            for i in range(NumberParticles):
                for j in range(Dimension):
                    PositionOld[i,j] = normalvariate(0.0,1.0)*sqrt(TimeStep)
            wfold = WaveFunction(PositionOld,alpha,beta)
            QuantumForceOld = QuantumForce(PositionOld,alpha, beta)

            #Loop over MC MCcycles
            for MCcycle in range(NumberMCcycles):
                #Trial position moving one particle at the time
                for i in range(NumberParticles):
                    for j in range(Dimension):
                        PositionNew[i,j] = PositionOld[i,j]+normalvariate(0.0,1.0)*sqrt(TimeStep)+\
                                           QuantumForceOld[i,j]*TimeStep*D
                    wfnew = WaveFunction(PositionNew,alpha,beta)
                    QuantumForceNew = QuantumForce(PositionNew,alpha, beta)
                    GreensFunction = 0.0
                    for j in range(Dimension):
                        GreensFunction += 0.5*(QuantumForceOld[i,j]+QuantumForceNew[i,j])*\
	                              (D*TimeStep*0.5*(QuantumForceOld[i,j]-QuantumForceNew[i,j])-\
                                      PositionNew[i,j]+PositionOld[i,j])
      
                    GreensFunction = exp(GreensFunction)
                    ProbabilityRatio = GreensFunction*wfnew**2/wfold**2
                    #Metropolis-Hastings test to see whether we accept the move
                    if random() <= ProbabilityRatio:
                       for j in range(Dimension):
                           PositionOld[i,j] = PositionNew[i,j]
                           QuantumForceOld[i,j] = QuantumForceNew[i,j]
                       wfold = wfnew
                DeltaE = LocalEnergy(PositionOld,alpha,beta)
                energy += DeltaE
                energy2 += DeltaE**2
            # We calculate mean, variance and error (no blocking applied)
            energy /= NumberMCcycles
            energy2 /= NumberMCcycles
            variance = energy2 - energy**2
            error = sqrt(variance/NumberMCcycles)
            Energies[ia,jb] = energy    
            outfile.write('%f %f %f %f %f\n' %(alpha,beta,energy,variance,error))
    return Energies, AlphaValues, BetaValues


#Here starts the main program with variable declarations
NumberParticles = 2
Dimension = 2
MaxVariations = 10
Energies = np.zeros((MaxVariations,MaxVariations))
AlphaValues = np.zeros(MaxVariations)
BetaValues = np.zeros(MaxVariations)
(Energies, AlphaValues, BetaValues) = MonteCarloSampling()
outfile.close()
# Prepare for plots
fig = plt.figure()
ax = fig.gca(projection='3d')
# Plot the surface.
X, Y = np.meshgrid(AlphaValues, BetaValues)
surf = ax.plot_surface(X, Y, Energies,cmap=cm.coolwarm,linewidth=0, antialiased=False)
# Customize the z axis.
zmin = np.matrix(Energies).min()
zmax = np.matrix(Energies).max()
ax.set_zlim(zmin, zmax)
ax.set_xlabel(r'$\alpha$')
ax.set_ylabel(r'$\beta$')
ax.set_zlabel(r'$\langle E \rangle$')
ax.zaxis.set_major_locator(LinearLocator(10))
ax.zaxis.set_major_formatter(FormatStrFormatter('%.02f'))
# Add a color bar which maps values to colors.
fig.colorbar(surf, shrink=0.5, aspect=5)
plt.show()



\end{minted}


\paragraph{Bringing the gradient optmization.}
The simple one-particle case in a harmonic oscillator trap


















\begin{minted}[fontsize=\fontsize{9pt}{9pt},linenos=false,mathescape,baselinestretch=1.0,fontfamily=tt,xleftmargin=7mm]{python}
# Gradient descent stepping with analytical derivative
import numpy as np
from scipy.optimize import minimize
def DerivativeE(x):
    return x-1.0/(4*x*x*x);

def Energy(x):
   return x*x*0.5+1.0/(8*x*x);
x0 = 1.0
eta = 0.1
Niterations = 100

for iter in range(Niterations):
    gradients = DerivativeE(x0)
    x0 -= eta*gradients

print(x0)

\end{minted}




































































































































\begin{minted}[fontsize=\fontsize{9pt}{9pt},linenos=false,mathescape,baselinestretch=1.0,fontfamily=tt,xleftmargin=7mm]{python}
# 2-electron VMC code for 2dim quantum dot with importance sampling
# Using gaussian rng for new positions and Metropolis- Hastings 
from math import exp, sqrt
from random import random, seed, normalvariate
import numpy as np
import matplotlib.pyplot as plt
from mpl_toolkits.mplot3d import Axes3D
from matplotlib import cm
from matplotlib.ticker import LinearLocator, FormatStrFormatter
import sys
from numba import jit


# Trial wave function for the 2-electron quantum dot in two dims
def WaveFunction(r,alpha):
    r1 = r[0,0]**2 + r[0,1]**2
    r2 = r[1,0]**2 + r[1,1]**2
    return exp(-0.5*alpha*(r1+r2))

# Local energy  for the 2-electron quantum dot in two dims, using analytical local energy
def LocalEnergy(r,alpha):
    
    r1 = (r[0,0]**2 + r[0,1]**2)
    r2 = (r[1,0]**2 + r[1,1]**2)
    return 0.5*(1-alpha*alpha)*(r1 + r2) +2.0*alpha

# Derivate of wave function ansatz as function of variational parameters
def DerivativeWFansatz(r,alpha):
    
    r1 = (r[0,0]**2 + r[0,1]**2)
    r2 = (r[1,0]**2 + r[1,1]**2)
    WfDer = -(r1+r2)
    return  WfDer

# Setting up the quantum force for the two-electron quantum dot, recall that it is a vector
def QuantumForce(r,alpha):

    qforce = np.zeros((NumberParticles,Dimension), np.double)
    qforce[0,:] = -2*r[0,:]*alpha
    qforce[1,:] = -2*r[1,:]*alpha
    return qforce
    
# Computing the derivative of the energy and the energy 
# jit decorator tells Numba to compile this function.
# The argument types will be inferred by Numba when function is called.
@jit
def EnergyMinimization(alpha):

    NumberMCcycles= 1000
    # Parameters in the Fokker-Planck simulation of the quantum force
    D = 0.5
    TimeStep = 0.05
    # positions
    PositionOld = np.zeros((NumberParticles,Dimension), np.double)
    PositionNew = np.zeros((NumberParticles,Dimension), np.double)
    # Quantum force
    QuantumForceOld = np.zeros((NumberParticles,Dimension), np.double)
    QuantumForceNew = np.zeros((NumberParticles,Dimension), np.double)

    # seed for rng generator 
    seed()
    energy = 0.0
    DeltaE = 0.0
    EnergyDer = 0.0
    DeltaPsi = 0.0
    DerivativePsiE = 0.0
    #Initial position
    for i in range(NumberParticles):
        for j in range(Dimension):
            PositionOld[i,j] = normalvariate(0.0,1.0)*sqrt(TimeStep)
    wfold = WaveFunction(PositionOld,alpha)
    QuantumForceOld = QuantumForce(PositionOld,alpha)

    #Loop over MC MCcycles
    for MCcycle in range(NumberMCcycles):
        #Trial position moving one particle at the time
        for i in range(NumberParticles):
            for j in range(Dimension):
                PositionNew[i,j] = PositionOld[i,j]+normalvariate(0.0,1.0)*sqrt(TimeStep)+\
                                       QuantumForceOld[i,j]*TimeStep*D
            wfnew = WaveFunction(PositionNew,alpha)
            QuantumForceNew = QuantumForce(PositionNew,alpha)
            GreensFunction = 0.0
            for j in range(Dimension):
                GreensFunction += 0.5*(QuantumForceOld[i,j]+QuantumForceNew[i,j])*\
	                              (D*TimeStep*0.5*(QuantumForceOld[i,j]-QuantumForceNew[i,j])-\
                                      PositionNew[i,j]+PositionOld[i,j])
      
            GreensFunction = exp(GreensFunction)
            ProbabilityRatio = GreensFunction*wfnew**2/wfold**2
            #Metropolis-Hastings test to see whether we accept the move
            if random() <= ProbabilityRatio:
                for j in range(Dimension):
                    PositionOld[i,j] = PositionNew[i,j]
                    QuantumForceOld[i,j] = QuantumForceNew[i,j]
                wfold = wfnew
        DeltaE = LocalEnergy(PositionOld,alpha)
        DeltaPsi = DerivativeWFansatz(PositionOld,alpha)
        energy += DeltaE
        DerivativePsiE += DeltaPsi*DeltaE
            
    # We calculate mean, variance and error (no blocking applied)
    energy /= NumberMCcycles
    DerivativePsiE /= NumberMCcycles
    DeltaPsi /= NumberMCcycles
    EnergyDer  = 2*(DerivativePsiE-DeltaPsi*energy)
    return energy, EnergyDer


#Here starts the main program with variable declarations
NumberParticles = 2
Dimension = 2
# guess for variational parameters
x0 = 1.5
# Set up iteration using stochastic gradient method
Energy =0 ; EnergyDer = 0
Energy, EnergyDer = EnergyMinimization(x0)
print(Energy, EnergyDer)

eta = 0.01
Niterations = 100

for iter in range(Niterations):
    gradients = EnergyDer
    x0 -= eta*gradients
    Energy, EnergyDer = EnergyMinimization(x0)

print(x0)


\end{minted}


\subsection*{VMC for fermions: Efficient calculation of Slater determinants}
The potentially most time-consuming part is the
evaluation of the gradient and the Laplacian of an $N$-particle  Slater
determinant. 

We have to differentiate the determinant with respect to
all spatial coordinates of all particles. A brute force
differentiation would involve $N\cdot d$ evaluations of the entire
determinant which would even worsen the already undesirable time
scaling, making it $Nd\cdot O(N^3)\sim O(d\cdot N^4)$.

This poses serious hindrances to the overall efficiency of our code.

The efficiency can be improved however if we move only one electron at the time.
The Slater determinant matrix $\hat{D}$ is defined by the matrix elements
\[
d_{ij}=\phi_j(x_i)
\]
where $\phi_j(\mathbf{r}_i)$ is a single particle  wave function.
The columns correspond to the position of a given particle 
while the rows stand for the various quantum numbers.

What we need to realize is that when differentiating a Slater
determinant with respect to some given coordinate, only one row of the
corresponding Slater matrix is changed. 

Therefore, by recalculating
the whole determinant we risk producing redundant information. The
solution turns out to be an algorithm that requires to keep track of
the \emph{inverse} of the Slater matrix.

Let the current position in phase space be represented by the $(N\cdot d)$-element 
vector $\mathbf{r}^{\mathrm{old}}$ and the new suggested
position by the vector $\mathbf{r}^{\mathrm{new}}$.

The inverse of $\hat{D}$ can be expressed in terms of its
cofactors $C_{ij}$ and its determinant (this our notation for a determinant) $\vert\hat{D}\vert$:
\begin{equation}
d_{ij}^{-1} = \frac{C_{ji}}{\vert\hat{D}\vert}
\label{eq:inverse_cofactor}
\end{equation}
Notice that the interchanged indices indicate that the matrix of cofactors is to be transposed.

If $\hat{D}$ is invertible, then we must obviously have $\hat{D}^{-1}\hat{D}= \mathbf{1}$, or explicitly in terms of the individual
elements of $\hat{D}$ and $\hat{D}^{-1}$:
\begin{equation}
\sum_{k=1}^N d_{ik}^{\phantom X}d_{kj}^{-1} = \delta_{ij}^{\phantom X}
\label{eq:unity_explicitely}
\end{equation}

Consider the ratio, which we shall call $R$, between $\vert\hat{D}(\mathbf{r}^{\mathrm{new}})\vert$ and $\vert\hat{D}(\mathbf{r}^{\mathrm{old}})\vert$. 
By definition, each of these determinants can
individually be expressed in terms of the \emph{i}-th row of its cofactor
matrix
\begin{equation}
R\equiv\frac{\vert\hat{D}(\mathbf{r}^{\mathrm{new}})\vert}
{\vert\hat{D}(\mathbf{r}^{\mathrm{old}})\vert} =
\frac{\sum_{j=1}^N d_{ij}(\mathbf{r}^{\mathrm{new}})\,
C_{ij}(\mathbf{r}^{\mathrm{new}})}
{\sum_{j=1}^N d_{ij}(\mathbf{r}^{\mathrm{old}})\,
C_{ij}(\mathbf{r}^{\mathrm{old}})}
\label{eq:detratio_cofactors}
\end{equation}

Suppose now that we move only one particle  at a time, meaning that
$\mathbf{r}^{\mathrm{new}}$ differs from $\mathbf{r}^{\mathrm{old}}$ by the
position of only one, say the \emph{i}-th, particle . This means that $\hat{D}(\mathbf{r}^{\mathrm{new}})$ and $\hat{D}(\mathbf{r}^{\mathrm{old}})$ differ
only by the entries of the \emph{i}-th row.  Recall also that the \emph{i}-th row
of a cofactor matrix $\hat{C}$ is independent of the entries of the
\emph{i}-th row of its corresponding matrix $\hat{D}$. In this particular
case we therefore get that the \emph{i}-th row of $\hat{C}(\mathbf{r}^{\mathrm{new}})$ 
and $\hat{C}(\mathbf{r}^{\mathrm{old}})$ must be
equal. Explicitly, we have:
\begin{equation}
C_{ij}(\mathbf{r}^{\mathrm{new}}) = C_{ij}(\mathbf{r}^{\mathrm{old}})\quad
\forall\ j\in\{1,\dots,N\}
\end{equation}

Inserting this into the numerator of eq.~(\ref{eq:detratio_cofactors})
and using eq.~(\ref{eq:inverse_cofactor}) to substitute the cofactors
with the elements of the inverse matrix, we get:
\begin{equation}
R =\frac{\sum_{j=1}^N d_{ij}(\mathbf{r}^{\mathrm{new}})\,
C_{ij}(\mathbf{r}^{\mathrm{old}})}
{\sum_{j=1}^N d_{ij}(\mathbf{r}^{\mathrm{old}})\,
C_{ij}(\mathbf{r}^{\mathrm{old}})} =
\frac{\sum_{j=1}^N d_{ij}(\mathbf{r}^{\mathrm{new}})\,
d_{ji}^{-1}(\mathbf{r}^{\mathrm{old}})}
{\sum_{j=1}^N d_{ij}(\mathbf{r}^{\mathrm{old}})\,
d_{ji}^{-1}(\mathbf{r}^{\mathrm{old}})}
\end{equation}

Now by eq.~(\ref{eq:unity_explicitely}) the denominator of the rightmost
expression must be unity, so that we finally arrive at:
\begin{equation}
R =
\sum_{j=1}^N d_{ij}(\mathbf{r}^{\mathrm{new}})\,
d_{ji}^{-1}(\mathbf{r}^{\mathrm{old}}) = 
\sum_{j=1}^N \phi_j(\mathbf{r}_i^{\mathrm{new}})\,
d_{ji}^{-1}(\mathbf{r}^{\mathrm{old}})
\label{eq:detratio_inverse}
\end{equation}
What this means is that in order to get the ratio when only the \emph{i}-th
particle  has been moved, we only need to calculate the dot
product of the vector $\left(\phi_1(\mathbf{r}_i^\mathrm{new}),\,\dots,\,\phi_N(\mathbf{r}_i^\mathrm{new})\right)$ of single particle  wave functions
evaluated at this new position with the \emph{i}-th column of the inverse
matrix $\hat{D}^{-1}$ evaluated at the original position. Such
an operation has a time scaling of $O(N)$. The only extra thing we
need to do is to maintain the inverse matrix $\hat{D}^{-1}(\mathbf{x}^{\mathrm{old}})$.

If the new position $\mathbf{r}^{\mathrm{new}}$ is accepted, then the
inverse matrix can by suitably updated by an algorithm having a time
scaling of $O(N^2)$.  This algorithm goes as
follows. First we update all but the \emph{i}-th column of $\hat{D}^{-1}$. For each column $j\neq i$, we first calculate the quantity:
\begin{equation}
S_j =
(\hat{D}(\mathbf{r}^{\mathrm{new}})\times
\hat{D}^{-1}(\mathbf{r}^{\mathrm{old}}))_{ij} =
\sum_{l=1}^N d_{il}(\mathbf{r}^{\mathrm{new}})\,
d^{-1}_{lj}(\mathbf{r}^{\mathrm{old}})
\label{eq:inverse_update_1}
\end{equation}

The new elements of the \emph{j}-th column of $\hat{D}^{-1}$ are then given
by:
\begin{equation}
d_{kj}^{-1}(\mathbf{r}^{\mathrm{new}}) =
d_{kj}^{-1}(\mathbf{r}^{\mathrm{old}}) -
\frac{S_j}{R}\,d_{ki}^{-1}(\mathbf{r}^{\mathrm{old}})\quad
\begin{array}{ll}
\forall\ \ k\in\{1,\dots,N\}\\j\neq i
\end{array}
\label{eq:inverse_update_2}
\end{equation}

Finally the elements of the \emph{i}-th column of $\hat{D}^{-1}$ are updated
simply as follows:
\begin{equation}
d_{ki}^{-1}(\mathbf{r}^{\mathrm{new}}) =
\frac{1}{R}\,d_{ki}^{-1}(\mathbf{r}^{\mathrm{old}})\quad
\forall\ \ k\in\{1,\dots,N\}
\label{eq:inverse_update_3}
\end{equation}
We see from these formulas that the time scaling of an update of
$\hat{D}^{-1}$ after changing one row of $\hat{D}$ is $O(N^2)$.

The scheme is also applicable for the calculation of the ratios
involving derivatives. It turns
out that differentiating the Slater determinant with respect
to the coordinates of a single particle  $\mathbf{r}_i$ changes only the
\emph{i}-th row of the corresponding Slater matrix. 

\paragraph{The gradient and the Laplacian.}
The gradient and the Laplacian can therefore be calculated as follows:
\[
\frac{\vec\nabla_i\vert\hat{D}(\mathbf{r})\vert}{\vert\hat{D}(\mathbf{r})\vert} =
\sum_{j=1}^N \vec\nabla_i d_{ij}(\mathbf{r})d_{ji}^{-1}(\mathbf{r}) =
\sum_{j=1}^N \vec\nabla_i \phi_j(\mathbf{r}_i)d_{ji}^{-1}(\mathbf{r})
\]
and
\[
\frac{\nabla^2_i\vert\hat{D}(\mathbf{r})\vert}{\vert\hat{D}(\mathbf{r})\vert} =
\sum_{j=1}^N \nabla^2_i d_{ij}(\mathbf{r})d_{ji}^{-1}(\mathbf{r}) =
\sum_{j=1}^N \nabla^2_i \phi_j(\mathbf{r}_i)\,d_{ji}^{-1}(\mathbf{r})
\]

Thus, to calculate all the derivatives of the Slater determinant, we
only need the derivatives of the single particle  wave functions
($\vec\nabla_i \phi_j(\mathbf{r}_i)$ and $\nabla^2_i \phi_j(\mathbf{r}_i)$)
and the elements of the corresponding inverse Slater matrix ($\hat{D}^{-1}(\mathbf{r}_i)$). A calculation of a single derivative is by the
above result an $O(N)$ operation. Since there are $d\cdot N$
derivatives, the time scaling of the total evaluation becomes
$O(d\cdot N^2)$. With an $O(N^2)$ updating algorithm for the
inverse matrix, the total scaling is no worse, which is far better
than the brute force approach yielding $O(d\cdot N^4)$.

\textbf{Important note}: In most cases you end with closed form expressions for the single-particle  wave functions. It is then useful to calculate the various derivatives and make separate functions
for them.

The Slater determinant takes the form  
\[
   \Phi(\mathbf{r}_1,\mathbf{r}_2,,\mathbf{r}_3,\mathbf{r}_4, \alpha,\beta,\gamma,\delta)=\frac{1}{\sqrt{4!}}
\left| \begin{array}{cccc} \psi_{100\uparrow}(\mathbf{r}_1)& \psi_{100\uparrow}(\mathbf{r}_2)& \psi_{100\uparrow}(\mathbf{r}_3)&\psi_{100\uparrow}(\mathbf{r}_4) \\
\psi_{100\downarrow}(\mathbf{r}_1)& \psi_{100\downarrow}(\mathbf{r}_2)& \psi_{100\downarrow}(\mathbf{r}_3)&\psi_{100\downarrow}(\mathbf{r}_4) \\
\psi_{200\uparrow}(\mathbf{r}_1)& \psi_{200\uparrow}(\mathbf{r}_2)& \psi_{200\uparrow}(\mathbf{r}_3)&\psi_{200\uparrow}(\mathbf{r}_4) \\
\psi_{200\downarrow}(\mathbf{r}_1)& \psi_{200\downarrow}(\mathbf{r}_2)& \psi_{200\downarrow}(\mathbf{r}_3)&\psi_{200\downarrow}(\mathbf{r}_4) \end{array} \right|.
\]
The Slater determinant as written is zero since the spatial wave functions for the spin up and spin down 
states are equal.  
But we can rewrite it as the product of two Slater determinants, one for spin up and one for spin down.

We can rewrite it as 
\[
   \Phi(\mathbf{r}_1,\mathbf{r}_2,,\mathbf{r}_3,\mathbf{r}_4, \alpha,\beta,\gamma,\delta)=\det\uparrow(1,2)\det\downarrow(3,4)-\det\uparrow(1,3)\det\downarrow(2,4)
\]
\[
-\det\uparrow(1,4)\det\downarrow(3,2)+\det\uparrow(2,3)\det\downarrow(1,4)-\det\uparrow(2,4)\det\downarrow(1,3)
\]
\[
+\det\uparrow(3,4)\det\downarrow(1,2),
\]
where we have defined
\[
\det\uparrow(1,2)=\frac{1}{\sqrt{2}}\left| \begin{array}{cc} \psi_{100\uparrow}(\mathbf{r}_1)& \psi_{100\uparrow}(\mathbf{r}_2)\\
\psi_{200\uparrow}(\mathbf{r}_1)& \psi_{200\uparrow}(\mathbf{r}_2) \end{array} \right|,
\]
and 
\[
\det\downarrow(3,4)=\frac{1}{\sqrt{2}}\left| \begin{array}{cc} \psi_{100\downarrow}(\mathbf{r}_3)& \psi_{100\downarrow}(\mathbf{r}_4)\\
\psi_{200\downarrow}(\mathbf{r}_3)& \psi_{200\downarrow}(\mathbf{r}_4) \end{array} \right|.
\]

We want to avoid to sum over spin variables, in particular when the interaction does not depend on spin.

It can be shown, see for example Moskowitz and Kalos, \href{{http://onlinelibrary.wiley.com/doi/10.1002/qua.560200508/abstract}}{Int.~J.~Quantum Chem. \textbf{20} 1107 (1981)}, that for the variational energy
we can approximate the Slater determinant as  
\[
   \Phi(\mathbf{r}_1,\mathbf{r}_2,,\mathbf{r}_3,\mathbf{r}_4, \alpha,\beta,\gamma,\delta) \propto \det\uparrow(1,2)\det\downarrow(3,4),
\]
or more generally as 
\[
   \Phi(\mathbf{r}_1,\mathbf{r}_2,\dots \mathbf{r}_N) \propto \det\uparrow \det\downarrow,
\]
where we have the Slater determinant as the product of a spin up part involving the number of electrons with spin up only (2 for beryllium and 5 for neon) and a spin down part involving the electrons with spin down.

This ansatz is not antisymmetric under the exchange of electrons with  opposite spins but it can be shown (show this) that it gives the same
expectation value for the energy as the full Slater determinant.

As long as the Hamiltonian is spin independent, the above is correct. It is rather straightforward to see this if you go back to the equations for the energy discussed earlier  this semester.

We will thus
factorize the full determinant $\vert\hat{D}\vert$ into two smaller ones, where 
each can be identified with $\uparrow$ and $\downarrow$
respectively:
\[
\vert\hat{D}\vert = \vert\hat{D}\vert_\uparrow\cdot \vert\hat{D}\vert_\downarrow
\]

The combined dimensionality of the two smaller determinants equals the
dimensionality of the full determinant. Such a factorization is
advantageous in that it makes it possible to perform the calculation
of the ratio $R$ and the updating of the inverse matrix separately for
$\vert\hat{D}\vert_\uparrow$ and $\vert\hat{D}\vert_\downarrow$:
\[
\frac{\vert\hat{D}\vert^\mathrm{new}}{\vert\hat{D}\vert^\mathrm{old}} =
\frac{\vert\hat{D}\vert^\mathrm{new}_\uparrow}
{\vert\hat{D}\vert^\mathrm{old}_\uparrow}\cdot
\frac{\vert\hat{D}\vert^\mathrm{new}_\downarrow
}{\vert\hat{D}\vert^\mathrm{old}_\downarrow}
\]

This reduces the calculation time by a constant factor. The maximal
time reduction happens in a system of equal numbers of $\uparrow$ and
$\downarrow$ particles, so that the two factorized determinants are
half the size of the original one.

Consider the case of moving only one particle  at a time which
originally had the following time scaling for one transition:
\[
O_R(N)+O_\mathrm{inverse}(N^2)
\]
For the factorized determinants one of the two determinants is
obviously unaffected by the change so that it cancels from the ratio
$R$. 

Therefore, only one determinant of size $N/2$ is involved in each
calculation of $R$ and update of the inverse matrix. The scaling of
each transition then becomes:
\[
O_R(N/2)+O_\mathrm{inverse}(N^2/4)
\]
and the time scaling when the transitions for all $N$ particles are
put together:
\[
O_R(N^2/2)+O_\mathrm{inverse}(N^3/4)
\]
which gives the same reduction as in the case of moving all particles
at once.

Computing the ratios discussed above requires that we maintain 
the inverse of the Slater matrix evaluated at the current position. 
Each time a trial position is accepted, the row number $i$ of the Slater 
matrix changes and updating its inverse has to be carried out. 
Getting the inverse of an $N \times N$ matrix by Gaussian elimination has a 
complexity of order of $\mathcal{O}(N^3)$ operations, a luxury that we 
cannot afford for each time a particle  move is accepted.
We will use the expression
\begin{equation}
\label{updatingInverse}
d^{-1}_{kj}(\mathbf{x^{new}}) = \left\{\begin{array}{l l}
  d^{-1}_{kj}(\mathbf{x^{old}}) - \frac{d^{-1}_{ki}(\mathbf{x^{old}})}{R} \sum_{l=1}^{N} d_{il}(\mathbf{x^{new}})  d^{-1}_{lj}(\mathbf{x^{old}}) & \mbox{if $j \neq i$}\nonumber \\ \\
 \frac{d^{-1}_{ki}(\mathbf{x^{old}})}{R} \sum_{l=1}^{N} d_{il}(\mathbf{x^{old}}) d^{-1}_{lj}(\mathbf{x^{old}}) & \mbox{if $j=i$}
\end{array} \right.
\end{equation}

This equation scales as $O(N^2)$.
The evaluation of the determinant of an $N \times N$ matrix by standard Gaussian elimination 
requires $\mathbf{O}(N^3)$
calculations. 
As there are $Nd$ independent coordinates we need to evaluate $Nd$ Slater determinants 
for the gradient (quantum force) and $Nd$ for the Laplacian (kinetic energy). 
With the updating algorithm we need only to invert the Slater 
determinant matrix once. This can be done by standard LU decomposition methods.

\paragraph{Expectation value of the kinetic energy.}
The expectation value of the kinetic energy expressed in atomic units for electron $i$ is 
\[
 \langle \hat{K}_i \rangle = -\frac{1}{2}\frac{\langle\Psi|\nabla_{i}^2|\Psi \rangle}{\langle\Psi|\Psi \rangle},
\]
\begin{equation}
\label{kineticE}
K_i = -\frac{1}{2}\frac{\nabla_{i}^{2} \Psi}{\Psi}.
\end{equation}
\begin{align}
\frac{\nabla^2 \Psi}{\Psi} & =  \frac{\nabla^2 ({\Psi_{D} \,  \Psi_C})}{\Psi_{D} \,  \Psi_C} = \frac{\nabla  \cdot [\nabla  {(\Psi_{D} \,  \Psi_C)}]}{\Psi_{D} \,  \Psi_C} = \frac{\nabla  \cdot [ \Psi_C \nabla  \Psi_{D} + \Psi_{D} \nabla   \Psi_C]}{\Psi_{D} \,  \Psi_C}\nonumber\\
&  =  \frac{\nabla   \Psi_C \cdot \nabla  \Psi_{D} +  \Psi_C \nabla^2 \Psi_{D} + \nabla  \Psi_{D} \cdot \nabla   \Psi_C + \Psi_{D} \nabla^2  \Psi_C}{\Psi_{D} \,  \Psi_C}\nonumber\\
\end{align}
\begin{align}
\frac{\nabla^2 \Psi}{\Psi}
& =  \frac{\nabla^2 \Psi_{D}}{\Psi_{D}} + \frac{\nabla^2  \Psi_C}{ \Psi_C} + 2 \frac{\nabla  \Psi_{D}}{\Psi_{D}}\cdot\frac{\nabla   \Psi_C}{ \Psi_C}
\end{align}

The second derivative of the Jastrow factor divided by the Jastrow factor (the way it enters the kinetic energy) is
\[
\left[\frac{\nabla^2 \Psi_C}{\Psi_C}\right]_x =\  
2\sum_{k=1}^{N}
\sum_{i=1}^{k-1}\frac{\partial^2 g_{ik}}{\partial x_k^2}\ +\ 
\sum_{k=1}^N
\left(
\sum_{i=1}^{k-1}\frac{\partial g_{ik}}{\partial x_k} -
\sum_{i=k+1}^{N}\frac{\partial g_{ki}}{\partial x_i}
\right)^2
\]

But we have a simple form for the function, namely
\[
\Psi_{C}=\prod_{i< j}\exp{f(r_{ij})}= \exp{\left\{\sum_{i<j}\frac{ar_{ij}}{1+\beta r_{ij}}\right\}},
\]
and it is easy to see that for particle  $k$
we have
\[
  \frac{\nabla^2_k \Psi_C}{\Psi_C }=
\sum_{ij\ne k}\frac{(\mathbf{r}_k-\mathbf{r}_i)(\mathbf{r}_k-\mathbf{r}_j)}{r_{ki}r_{kj}}f'(r_{ki})f'(r_{kj})+
\sum_{j\ne k}\left( f''(r_{kj})+\frac{2}{r_{kj}}f'(r_{kj})\right)
\]

Using 
\[
f(r_{ij})= \frac{ar_{ij}}{1+\beta r_{ij}},
\]
and $g'(r_{kj})=dg(r_{kj})/dr_{kj}$ and 
$g''(r_{kj})=d^2g(r_{kj})/dr_{kj}^2$  we find that for particle  $k$
we have
\[
  \frac{\nabla^2_k \Psi_C}{\Psi_C }=
\sum_{ij\ne k}\frac{(\mathbf{r}_k-\mathbf{r}_i)(\mathbf{r}_k-\mathbf{r}_j)}{r_{ki}r_{kj}}\frac{a}{(1+\beta r_{ki})^2}
\frac{a}{(1+\beta r_{kj})^2}+
\sum_{j\ne k}\left(\frac{2a}{r_{kj}(1+\beta r_{kj})^2}-\frac{2a\beta}{(1+\beta r_{kj})^3}\right)
\]

The gradient and
Laplacian can be calculated as follows:
\[
\frac{\mathbf{\nabla}_i\vert\hat{D}(\mathbf{r})\vert}
{\vert\hat{D}(\mathbf{r})\vert} =
\sum_{j=1}^N \vec\nabla_i d_{ij}(\mathbf{r})\,
d_{ji}^{-1}(\mathbf{r}) =
\sum_{j=1}^N \vec\nabla_i \phi_j(\mathbf{r}_i)\,
d_{ji}^{-1}(\mathbf{r})
\]
and
\[
\frac{\nabla^2_i\vert\hat{D}(\mathbf{r})\vert}
{\vert\hat{D}(\mathbf{r})\vert} =
\sum_{j=1}^N \nabla^2_i d_{ij}(\mathbf{r})\,
d_{ji}^{-1}(\mathbf{r}) =
\sum_{j=1}^N \nabla^2_i \phi_j(\mathbf{r}_i)\,
d_{ji}^{-1}(\mathbf{r})
\]

The gradient for the determinant is 
\[
\frac{\mathbf{\nabla}_i\vert\hat{D}(\mathbf{r})\vert}
{\vert\hat{D}(\mathbf{r})\vert} =
\sum_{j=1}^N \mathbf{\nabla}_i d_{ij}(\mathbf{r})\,
d_{ji}^{-1}(\mathbf{r}) =
\sum_{j=1}^N \mathbf{\nabla}_i \phi_j(\mathbf{r}_i)\,
d_{ji}^{-1}(\mathbf{r}).
\]

We have
\[
\Psi_C=\prod_{i< j}g(r_{ij})= \exp{\left\{\sum_{i<j}\frac{ar_{ij}}{1+\beta r_{ij}}\right\}},
\]
the gradient needed for the quantum force and local energy is easy to compute.  
We get for particle  $k$
\[
\frac{ \nabla_k \Psi_C}{ \Psi_C }= \sum_{j\ne k}\frac{\mathbf{r}_{kj}}{r_{kj}}\frac{a}{(1+\beta r_{kj})^2},
\]
which is rather easy to code.  Remember to sum over all particles  when you compute the local energy.

We need to compute the ratio between wave functions, in particular  for the Slater determinants.
\[
R =\sum_{j=1}^N d_{ij}(\mathbf{r}^{\mathrm{new}})\,
d_{ji}^{-1}(\mathbf{r}^{\mathrm{old}}) = 
\sum_{j=1}^N \phi_j(\mathbf{r}_i^{\mathrm{new}})\,
d_{ji}^{-1}(\mathbf{r}^{\mathrm{old}})
\]
What this means is that in order to get the ratio when only the \emph{i}-th
particle  has been moved, we only need to calculate the dot
product of the vector $\left(\phi_1(\mathbf{r}_i^\mathrm{new}),\,\dots,\,
\phi_N(\mathbf{r}_i^\mathrm{new})\right)$ of single particle  wave functions
evaluated at this new position with the \emph{i}-th column of the inverse
matrix $\hat{D}^{-1}$ evaluated at the original position. Such
an operation has a time scaling of $O(N)$. The only extra thing we
need to do is to maintain the inverse matrix 
$\hat{D}^{-1}(\mathbf{x}^{\mathrm{old}})$.


 \clearemptydoublepage
%%
%% Automatically generated file from DocOnce source
%% (https://github.com/doconce/doconce/)
%% doconce format latex gradientmethods.do.txt --minted_latex_style=trac --latex_admon=paragraph --no_mako
%%


%-------------------- begin preamble ----------------------

\documentclass[%
oneside,                 % oneside: electronic viewing, twoside: printing
final,                   % draft: marks overfull hboxes, figures with paths
10pt]{article}

\listfiles               %  print all files needed to compile this document

\usepackage{relsize,makeidx,color,setspace,amsmath,amsfonts,amssymb}
\usepackage[table]{xcolor}
\usepackage{bm,ltablex,microtype}

\usepackage[pdftex]{graphicx}

\usepackage{fancyvrb} % packages needed for verbatim environments
\usepackage{minted}
\usemintedstyle{default}

\usepackage[T1]{fontenc}
%\usepackage[latin1]{inputenc}
\usepackage{ucs}
\usepackage[utf8x]{inputenc}

\usepackage{lmodern}         % Latin Modern fonts derived from Computer Modern

% Hyperlinks in PDF:
\definecolor{linkcolor}{rgb}{0,0,0.4}
\usepackage{hyperref}
\hypersetup{
    breaklinks=true,
    colorlinks=true,
    linkcolor=linkcolor,
    urlcolor=linkcolor,
    citecolor=black,
    filecolor=black,
    %filecolor=blue,
    pdfmenubar=true,
    pdftoolbar=true,
    bookmarksdepth=3   % Uncomment (and tweak) for PDF bookmarks with more levels than the TOC
    }
%\hyperbaseurl{}   % hyperlinks are relative to this root

\setcounter{tocdepth}{2}  % levels in table of contents

\usepackage[framemethod=TikZ]{mdframed}

% --- begin definitions of admonition environments ---

% --- end of definitions of admonition environments ---

% prevent orhpans and widows
\clubpenalty = 10000
\widowpenalty = 10000

\newenvironment{doconceexercise}{}{}
\newcounter{doconceexercisecounter}


% ------ header in subexercises ------
%\newcommand{\subex}[1]{\paragraph{#1}}
%\newcommand{\subex}[1]{\par\vspace{1.7mm}\noindent{\bf #1}\ \ }
\makeatletter
% 1.5ex is the spacing above the header, 0.5em the spacing after subex title
\newcommand\subex{\@startsection*{paragraph}{4}{\z@}%
                  {1.5ex\@plus1ex \@minus.2ex}%
                  {-0.5em}%
                  {\normalfont\normalsize\bfseries}}
\makeatother


% --- end of standard preamble for documents ---


% insert custom LaTeX commands...

\raggedbottom
\makeindex
\usepackage[totoc]{idxlayout}   % for index in the toc
\usepackage[nottoc]{tocbibind}  % for references/bibliography in the toc

%-------------------- end preamble ----------------------

\begin{document}

% matching end for #ifdef PREAMBLE

\newcommand{\exercisesection}[1]{\subsection*{#1}}


% ------------------- main content ----------------------

\section*{Gradient Methods}

\subsection*{Top-down start}

\begin{itemize}
\item We will start with a top-down view, with a simple harmonic oscillator problem in one dimension as case.

\item Thereafter we continue with implementing the simplest possible steepest descent approach to our two-electron problem with an electrostatic (Coulomb) interaction. Our code includes also importance sampling. The simple Python code here illustrates the basic elements which need to be included in our own code.

\item Then we move on to the mathematical description of various gradient methods.
\end{itemize}

\noindent
\subsection*{Motivation}

% --- begin paragraph admon ---
\paragraph{}
Our aim with this part of the project is to be able to
\begin{itemize}
\item find an optimal value for the variational parameters using only some few Monte Carlo cycles

\item use these optimal values for the variational parameters to perform a large-scale Monte Carlo calculation
\end{itemize}

\noindent
To achieve this will look at methods like \emph{Steepest descent} and the \emph{conjugate gradient method}. Both these methods allow us to find
the minima of a multivariable  function like our energy (function of several variational parameters). 
Alternatively, you can always use Newton's method. In particular, since we will normally have one variational parameter,
Newton's method can be easily used in finding the minimum of the local energy.
% --- end paragraph admon ---



\subsection*{Simple example and demonstration}

Let us illustrate what is needed in our calculations using a simple example, the harmonic oscillator in one dimension.
For the harmonic oscillator in one-dimension we have a  trial wave function and probability
\[
\psi_T(x;\alpha) = \exp{-(\frac{1}{2}\alpha^2x^2)},
\]
which results in a local energy 
\[
\frac{1}{2}\left(\alpha^2+x^2(1-\alpha^4)\right).
\]
We can compare our numerically calculated energies with the exact energy as function of $\alpha$
\[
\overline{E}[\alpha] = \frac{1}{4}\left(\alpha^2+\frac{1}{\alpha^2}\right).
\]

\subsection*{Simple example and demonstration}

% --- begin paragraph admon ---
\paragraph{}
The derivative of the energy with respect to $\alpha$ gives
\begin{equation*}
\frac{d\langle  E_L[\alpha]\rangle}{d\alpha} = \frac{1}{2}\alpha-\frac{1}{2\alpha^3}
\end{equation*}
and a second derivative which is always positive (meaning that we find a minimum)
\begin{equation*}
\frac{d^2\langle  E_L[\alpha]\rangle}{d\alpha^2} = \frac{1}{2}+\frac{3}{2\alpha^4}
\end{equation*}
The condition
\begin{equation*}
\frac{d\langle  E_L[\alpha]\rangle}{d\alpha} = 0,
\end{equation*}
gives the optimal $\alpha=1$, as expected.
% --- end paragraph admon ---



% --- begin exercise ---
\begin{doconceexercise}
\refstepcounter{doconceexercisecounter}

\exercisesection*{Exercise \thedoconceexercisecounter: Find the local energy for the harmonic oscillator}
                             

% --- begin subexercise ---
\subex{a)}
Derive the local energy for the harmonic oscillator in one dimension and find its expectation value.

% --- end subexercise ---

% --- begin subexercise ---
\subex{b)}
Show also that the optimal value of optimal $\alpha=1$

% --- end subexercise ---

% --- begin subexercise ---
\subex{c)}
Repeat the above steps in two dimensions for $N$ bosons or electrons. What is the optimal value of $\alpha$?

% --- end subexercise ---

\end{doconceexercise}
% --- end exercise ---

\subsection*{Variance in the simple model}

% --- begin paragraph admon ---
\paragraph{}
We can also minimize the variance. In our simple model the variance is

\[
\sigma^2[\alpha]=\frac{1}{4}\left(1+(1-\alpha^4)^2\frac{3}{4\alpha^4}\right)-\overline{E}^2.
\]
which yields a second derivative which is always positive.
% --- end paragraph admon ---



\subsection*{Computing the derivatives}

% --- begin paragraph admon ---
\paragraph{}

In general we end up computing the expectation value of the energy in terms 
of some parameters $\alpha_0,\alpha_1,\dots,\alpha_n$
and we search for a minimum in this multi-variable parameter space.  
This leads to an energy minimization problem \emph{where we need the derivative of the energy as a function of the variational parameters}.

In the above example this was easy and we were able to find the expression for the derivative by simple derivations. 
However, in our actual calculations the energy is represented by a multi-dimensional integral with several variational parameters.
How can we can then obtain the derivatives of the energy with respect to the variational parameters without having 
to resort to expensive numerical derivations?
% --- end paragraph admon ---



\subsection*{Expressions for finding the derivatives of the local energy}

% --- begin paragraph admon ---
\paragraph{}

To find the derivatives of the local energy expectation value as function of the variational parameters, we can use the chain rule and the hermiticity of the Hamiltonian.  

Let us define 
\[
\bar{E}_{\alpha}=\frac{d\langle  E_L[\alpha]\rangle}{d\alpha}.
\]
as the derivative of the energy with respect to the variational parameter $\alpha$ (we limit ourselves to one parameter only).
In the above example this was easy and we obtain a simple expression for the derivative.
We define also the derivative of the trial function (skipping the subindex $T$) as 
\[
\bar{\psi}_{\alpha}=\frac{d\psi[\alpha]\rangle}{d\alpha}.
\]
% --- end paragraph admon ---



\subsection*{Derivatives of the local energy}

% --- begin paragraph admon ---
\paragraph{}
The elements of the gradient of the local energy are then (using the chain rule and the hermiticity of the Hamiltonian)
\[
\bar{E}_{\alpha} = 2\left( \langle \frac{\bar{\psi}_{\alpha}}{\psi[\alpha]}E_L[\alpha]\rangle -\langle \frac{\bar{\psi}_{\alpha}}{\psi[\alpha]}\rangle\langle E_L[\alpha] \rangle\right).
\]
From a computational point of view it means that you need to compute the expectation values of 
\[
\langle \frac{\bar{\psi}_{\alpha}}{\psi[\alpha]}E_L[\alpha]\rangle,
\]
and
\[
\langle \frac{\bar{\psi}_{\alpha}}{\psi[\alpha]}\rangle\langle E_L[\alpha]\rangle
\]
% --- end paragraph admon ---



% --- begin exercise ---
\begin{doconceexercise}
\refstepcounter{doconceexercisecounter}

\exercisesection*{Exercise \thedoconceexercisecounter: General expression for the derivative of the energy}
                             

% --- begin subexercise ---
\subex{a)}
Show that 
\[
\bar{E}_{\alpha} = 2\left( \langle \frac{\bar{\psi}_{\alpha}}{\psi[\alpha]}E_L[\alpha]\rangle -\langle \frac{\bar{\psi}_{\alpha}}{\psi[\alpha]}\rangle\langle E_L[\alpha] \rangle\right).
\]

% --- end subexercise ---

% --- begin subexercise ---
\subex{b)}
Find the corresponding expression for the variance.

% --- end subexercise ---

\end{doconceexercise}
% --- end exercise ---

\subsection*{Python program for 2-electrons in 2 dimensions}



















































































































































\begin{minted}[fontsize=\fontsize{9pt}{9pt},linenos=false,mathescape,baselinestretch=1.0,fontfamily=tt,xleftmargin=7mm]{python}
# 2-electron VMC code for 2dim quantum dot with importance sampling
# Using gaussian rng for new positions and Metropolis- Hastings 
# Added energy minimization with gradient descent using fixed step size
# To do: replace with optimization codes from scipy and/or use stochastic gradient descent
from math import exp, sqrt
from random import random, seed, normalvariate
import numpy as np
import matplotlib.pyplot as plt
from mpl_toolkits.mplot3d import Axes3D
from matplotlib import cm
from matplotlib.ticker import LinearLocator, FormatStrFormatter
import sys



# Trial wave function for the 2-electron quantum dot in two dims
def WaveFunction(r,alpha,beta):
    r1 = r[0,0]**2 + r[0,1]**2
    r2 = r[1,0]**2 + r[1,1]**2
    r12 = sqrt((r[0,0]-r[1,0])**2 + (r[0,1]-r[1,1])**2)
    deno = r12/(1+beta*r12)
    return exp(-0.5*alpha*(r1+r2)+deno)

# Local energy  for the 2-electron quantum dot in two dims, using analytical local energy
def LocalEnergy(r,alpha,beta):
    
    r1 = (r[0,0]**2 + r[0,1]**2)
    r2 = (r[1,0]**2 + r[1,1]**2)
    r12 = sqrt((r[0,0]-r[1,0])**2 + (r[0,1]-r[1,1])**2)
    deno = 1.0/(1+beta*r12)
    deno2 = deno*deno
    return 0.5*(1-alpha*alpha)*(r1 + r2) +2.0*alpha + 1.0/r12+deno2*(alpha*r12-deno2+2*beta*deno-1.0/r12)

# Derivate of wave function ansatz as function of variational parameters
def DerivativeWFansatz(r,alpha,beta):
    
    WfDer  = np.zeros((2), np.double)
    r1 = (r[0,0]**2 + r[0,1]**2)
    r2 = (r[1,0]**2 + r[1,1]**2)
    r12 = sqrt((r[0,0]-r[1,0])**2 + (r[0,1]-r[1,1])**2)
    deno = 1.0/(1+beta*r12)
    deno2 = deno*deno
    WfDer[0] = -0.5*(r1+r2)
    WfDer[1] = -r12*r12*deno2
    return  WfDer

# Setting up the quantum force for the two-electron quantum dot, recall that it is a vector
def QuantumForce(r,alpha,beta):

    qforce = np.zeros((NumberParticles,Dimension), np.double)
    r12 = sqrt((r[0,0]-r[1,0])**2 + (r[0,1]-r[1,1])**2)
    deno = 1.0/(1+beta*r12)
    qforce[0,:] = -2*r[0,:]*alpha*(r[0,:]-r[1,:])*deno*deno/r12
    qforce[1,:] = -2*r[1,:]*alpha*(r[1,:]-r[0,:])*deno*deno/r12
    return qforce
    

# Computing the derivative of the energy and the energy 
def EnergyMinimization(alpha, beta):

    NumberMCcycles= 10000
    # Parameters in the Fokker-Planck simulation of the quantum force
    D = 0.5
    TimeStep = 0.05
    # positions
    PositionOld = np.zeros((NumberParticles,Dimension), np.double)
    PositionNew = np.zeros((NumberParticles,Dimension), np.double)
    # Quantum force
    QuantumForceOld = np.zeros((NumberParticles,Dimension), np.double)
    QuantumForceNew = np.zeros((NumberParticles,Dimension), np.double)

    # seed for rng generator 
    seed()
    energy = 0.0
    DeltaE = 0.0
    EnergyDer = np.zeros((2), np.double)
    DeltaPsi = np.zeros((2), np.double)
    DerivativePsiE = np.zeros((2), np.double)
    #Initial position
    for i in range(NumberParticles):
        for j in range(Dimension):
            PositionOld[i,j] = normalvariate(0.0,1.0)*sqrt(TimeStep)
    wfold = WaveFunction(PositionOld,alpha,beta)
    QuantumForceOld = QuantumForce(PositionOld,alpha, beta)

    #Loop over MC MCcycles
    for MCcycle in range(NumberMCcycles):
        #Trial position moving one particle at the time
        for i in range(NumberParticles):
            for j in range(Dimension):
                PositionNew[i,j] = PositionOld[i,j]+normalvariate(0.0,1.0)*sqrt(TimeStep)+\
                                       QuantumForceOld[i,j]*TimeStep*D
            wfnew = WaveFunction(PositionNew,alpha,beta)
            QuantumForceNew = QuantumForce(PositionNew,alpha, beta)
            GreensFunction = 0.0
            for j in range(Dimension):
                GreensFunction += 0.5*(QuantumForceOld[i,j]+QuantumForceNew[i,j])*\
	                              (D*TimeStep*0.5*(QuantumForceOld[i,j]-QuantumForceNew[i,j])-\
                                      PositionNew[i,j]+PositionOld[i,j])
      
            GreensFunction = exp(GreensFunction)
            ProbabilityRatio = GreensFunction*wfnew**2/wfold**2
            #Metropolis-Hastings test to see whether we accept the move
            if random() <= ProbabilityRatio:
                for j in range(Dimension):
                    PositionOld[i,j] = PositionNew[i,j]
                    QuantumForceOld[i,j] = QuantumForceNew[i,j]
                wfold = wfnew
        DeltaE = LocalEnergy(PositionOld,alpha,beta)
        DerPsi = DerivativeWFansatz(PositionOld,alpha,beta)
        DeltaPsi += DerPsi
        energy += DeltaE
        DerivativePsiE += DerPsi*DeltaE
            
    # We calculate mean values
    energy /= NumberMCcycles
    DerivativePsiE /= NumberMCcycles
    DeltaPsi /= NumberMCcycles
    EnergyDer  = 2*(DerivativePsiE-DeltaPsi*energy)
    return energy, EnergyDer


#Here starts the main program with variable declarations
NumberParticles = 2
Dimension = 2
# guess for variational parameters
alpha = 0.9
beta = 0.2
# Set up iteration using gradient descent method
Energy = 0
EDerivative = np.zeros((2), np.double)
eta = 0.01
Niterations = 50
# 
for iter in range(Niterations):
    Energy, EDerivative = EnergyMinimization(alpha,beta)
    alphagradient = EDerivative[0]
    betagradient = EDerivative[1]
    alpha -= eta*alphagradient
    beta -= eta*betagradient 

print(alpha, beta)
print(Energy, EDerivative[0], EDerivative[1])




\end{minted}


\subsection*{Using Broyden's algorithm in scipy}
The following function uses the above described BFGS algorithm. Here we have defined a function which calculates the energy and a function which computes the first derivative.






























































































































































































\begin{minted}[fontsize=\fontsize{9pt}{9pt},linenos=false,mathescape,baselinestretch=1.0,fontfamily=tt,xleftmargin=7mm]{python}
# 2-electron VMC code for 2dim quantum dot with importance sampling
# Using gaussian rng for new positions and Metropolis- Hastings 
# Added energy minimization using the BFGS algorithm, see p. 136 of https://www.springer.com/it/book/9780387303031
from math import exp, sqrt
from random import random, seed, normalvariate
import numpy as np
import matplotlib.pyplot as plt
from mpl_toolkits.mplot3d import Axes3D
from matplotlib import cm
from matplotlib.ticker import LinearLocator, FormatStrFormatter
from scipy.optimize import minimize
import sys


# Trial wave function for the 2-electron quantum dot in two dims
def WaveFunction(r,alpha,beta):
    r1 = r[0,0]**2 + r[0,1]**2
    r2 = r[1,0]**2 + r[1,1]**2
    r12 = sqrt((r[0,0]-r[1,0])**2 + (r[0,1]-r[1,1])**2)
    deno = r12/(1+beta*r12)
    return exp(-0.5*alpha*(r1+r2)+deno)

# Local energy  for the 2-electron quantum dot in two dims, using analytical local energy
def LocalEnergy(r,alpha,beta):
    
    r1 = (r[0,0]**2 + r[0,1]**2)
    r2 = (r[1,0]**2 + r[1,1]**2)
    r12 = sqrt((r[0,0]-r[1,0])**2 + (r[0,1]-r[1,1])**2)
    deno = 1.0/(1+beta*r12)
    deno2 = deno*deno
    return 0.5*(1-alpha*alpha)*(r1 + r2) +2.0*alpha + 1.0/r12+deno2*(alpha*r12-deno2+2*beta*deno-1.0/r12)

# Derivate of wave function ansatz as function of variational parameters
def DerivativeWFansatz(r,alpha,beta):
    
    WfDer  = np.zeros((2), np.double)
    r1 = (r[0,0]**2 + r[0,1]**2)
    r2 = (r[1,0]**2 + r[1,1]**2)
    r12 = sqrt((r[0,0]-r[1,0])**2 + (r[0,1]-r[1,1])**2)
    deno = 1.0/(1+beta*r12)
    deno2 = deno*deno
    WfDer[0] = -0.5*(r1+r2)
    WfDer[1] = -r12*r12*deno2
    return  WfDer

# Setting up the quantum force for the two-electron quantum dot, recall that it is a vector
def QuantumForce(r,alpha,beta):

    qforce = np.zeros((NumberParticles,Dimension), np.double)
    r12 = sqrt((r[0,0]-r[1,0])**2 + (r[0,1]-r[1,1])**2)
    deno = 1.0/(1+beta*r12)
    qforce[0,:] = -2*r[0,:]*alpha*(r[0,:]-r[1,:])*deno*deno/r12
    qforce[1,:] = -2*r[1,:]*alpha*(r[1,:]-r[0,:])*deno*deno/r12
    return qforce
    

# Computing the derivative of the energy and the energy 
def EnergyDerivative(x0):

    
    # Parameters in the Fokker-Planck simulation of the quantum force
    D = 0.5
    TimeStep = 0.05
    NumberMCcycles= 10000
    # positions
    PositionOld = np.zeros((NumberParticles,Dimension), np.double)
    PositionNew = np.zeros((NumberParticles,Dimension), np.double)
    # Quantum force
    QuantumForceOld = np.zeros((NumberParticles,Dimension), np.double)
    QuantumForceNew = np.zeros((NumberParticles,Dimension), np.double)

    energy = 0.0
    DeltaE = 0.0
    alpha = x0[0]
    beta = x0[1]
    EnergyDer = 0.0
    DeltaPsi = 0.0
    DerivativePsiE = 0.0 
    #Initial position
    for i in range(NumberParticles):
        for j in range(Dimension):
            PositionOld[i,j] = normalvariate(0.0,1.0)*sqrt(TimeStep)
    wfold = WaveFunction(PositionOld,alpha,beta)
    QuantumForceOld = QuantumForce(PositionOld,alpha, beta)

    #Loop over MC MCcycles
    for MCcycle in range(NumberMCcycles):
        #Trial position moving one particle at the time
        for i in range(NumberParticles):
            for j in range(Dimension):
                PositionNew[i,j] = PositionOld[i,j]+normalvariate(0.0,1.0)*sqrt(TimeStep)+\
                                       QuantumForceOld[i,j]*TimeStep*D
            wfnew = WaveFunction(PositionNew,alpha,beta)
            QuantumForceNew = QuantumForce(PositionNew,alpha, beta)
            GreensFunction = 0.0
            for j in range(Dimension):
                GreensFunction += 0.5*(QuantumForceOld[i,j]+QuantumForceNew[i,j])*\
	                              (D*TimeStep*0.5*(QuantumForceOld[i,j]-QuantumForceNew[i,j])-\
                                      PositionNew[i,j]+PositionOld[i,j])
      
            GreensFunction = exp(GreensFunction)
            ProbabilityRatio = GreensFunction*wfnew**2/wfold**2
            #Metropolis-Hastings test to see whether we accept the move
            if random() <= ProbabilityRatio:
                for j in range(Dimension):
                    PositionOld[i,j] = PositionNew[i,j]
                    QuantumForceOld[i,j] = QuantumForceNew[i,j]
                wfold = wfnew
        DeltaE = LocalEnergy(PositionOld,alpha,beta)
        DerPsi = DerivativeWFansatz(PositionOld,alpha,beta)
        DeltaPsi += DerPsi
        energy += DeltaE
        DerivativePsiE += DerPsi*DeltaE
            
    # We calculate mean values
    energy /= NumberMCcycles
    DerivativePsiE /= NumberMCcycles
    DeltaPsi /= NumberMCcycles
    EnergyDer  = 2*(DerivativePsiE-DeltaPsi*energy)
    return EnergyDer


# Computing the expectation value of the local energy 
def Energy(x0):
    # Parameters in the Fokker-Planck simulation of the quantum force
    D = 0.5
    TimeStep = 0.05
    # positions
    PositionOld = np.zeros((NumberParticles,Dimension), np.double)
    PositionNew = np.zeros((NumberParticles,Dimension), np.double)
    # Quantum force
    QuantumForceOld = np.zeros((NumberParticles,Dimension), np.double)
    QuantumForceNew = np.zeros((NumberParticles,Dimension), np.double)

    energy = 0.0
    DeltaE = 0.0
    alpha = x0[0]
    beta = x0[1]
    NumberMCcycles= 10000
    #Initial position
    for i in range(NumberParticles):
        for j in range(Dimension):
            PositionOld[i,j] = normalvariate(0.0,1.0)*sqrt(TimeStep)
    wfold = WaveFunction(PositionOld,alpha,beta)
    QuantumForceOld = QuantumForce(PositionOld,alpha, beta)

    #Loop over MC MCcycles
    for MCcycle in range(NumberMCcycles):
        #Trial position moving one particle at the time
        for i in range(NumberParticles):
            for j in range(Dimension):
                PositionNew[i,j] = PositionOld[i,j]+normalvariate(0.0,1.0)*sqrt(TimeStep)+\
                                       QuantumForceOld[i,j]*TimeStep*D
            wfnew = WaveFunction(PositionNew,alpha,beta)
            QuantumForceNew = QuantumForce(PositionNew,alpha, beta)
            GreensFunction = 0.0
            for j in range(Dimension):
                GreensFunction += 0.5*(QuantumForceOld[i,j]+QuantumForceNew[i,j])*\
	                              (D*TimeStep*0.5*(QuantumForceOld[i,j]-QuantumForceNew[i,j])-\
                                      PositionNew[i,j]+PositionOld[i,j])
      
            GreensFunction = exp(GreensFunction)
            ProbabilityRatio = GreensFunction*wfnew**2/wfold**2
            #Metropolis-Hastings test to see whether we accept the move
            if random() <= ProbabilityRatio:
                for j in range(Dimension):
                    PositionOld[i,j] = PositionNew[i,j]
                    QuantumForceOld[i,j] = QuantumForceNew[i,j]
                wfold = wfnew
        DeltaE = LocalEnergy(PositionOld,alpha,beta)
        energy += DeltaE
            
    # We calculate mean values
    energy /= NumberMCcycles
    return energy




#Here starts the main program with variable declarations
NumberParticles = 2
Dimension = 2
# seed for rng generator 
seed()
# guess for variational parameters
x0 = np.array([0.9,0.2])
# Using Broydens method
res = minimize(Energy, x0, method='BFGS', jac=EnergyDerivative, options={'gtol': 1e-4,'disp': True})
print(res.x)

\end{minted}

Note that the \textbf{minimize} function returns the finale values for the variable $\alpha=x0[0]$ and $\beta=x0[1]$ in the array $x$. 

\subsection*{Brief reminder on Newton-Raphson's method}

Let us quickly remind ourselves how we derive the above method.

Perhaps the most celebrated of all one-dimensional root-finding
routines is Newton's method, also called the Newton-Raphson
method. This method  requires the evaluation of both the
function $f$ and its derivative $f'$ at arbitrary points. 
If you can only calculate the derivative
numerically and/or your function is not of the smooth type, we
normally discourage the use of this method.

\subsection*{The equations}

The Newton-Raphson formula consists geometrically of extending the
tangent line at a current point until it crosses zero, then setting
the next guess to the abscissa of that zero-crossing.  The mathematics
behind this method is rather simple. Employing a Taylor expansion for
$x$ sufficiently close to the solution $s$, we have

\[
    f(s)=0=f(x)+(s-x)f'(x)+\frac{(s-x)^2}{2}f''(x) +\dots.
    \label{eq:taylornr}
\]

For small enough values of the function and for well-behaved
functions, the terms beyond linear are unimportant, hence we obtain

\[
   f(x)+(s-x)f'(x)\approx 0,
\]
yielding
\[
   s\approx x-\frac{f(x)}{f'(x)}.
\]

Having in mind an iterative procedure, it is natural to start iterating with
\[
   x_{n+1}=x_n-\frac{f(x_n)}{f'(x_n)}.
\]

\subsection*{Simple geometric interpretation}

The above is Newton-Raphson's method. It has a simple geometric
interpretation, namely $x_{n+1}$ is the point where the tangent from
$(x_n,f(x_n))$ crosses the $x$-axis.  Close to the solution,
Newton-Raphson converges fast to the desired result. However, if we
are far from a root, where the higher-order terms in the series are
important, the Newton-Raphson formula can give grossly inaccurate
results. For instance, the initial guess for the root might be so far
from the true root as to let the search interval include a local
maximum or minimum of the function.  If an iteration places a trial
guess near such a local extremum, so that the first derivative nearly
vanishes, then Newton-Raphson may fail totally

\subsection*{Extending to more than one variable}

Newton's method can be generalized to systems of several non-linear equations
and variables. Consider the case with two equations
\[
   \begin{array}{cc} f_1(x_1,x_2) &=0\\
                     f_2(x_1,x_2) &=0,\end{array}
\]
which we Taylor expand to obtain

\[
   \begin{array}{cc} 0=f_1(x_1+h_1,x_2+h_2)=&f_1(x_1,x_2)+h_1
                     \partial f_1/\partial x_1+h_2
                     \partial f_1/\partial x_2+\dots\\
                     0=f_2(x_1+h_1,x_2+h_2)=&f_2(x_1,x_2)+h_1
                     \partial f_2/\partial x_1+h_2
                     \partial f_2/\partial x_2+\dots
                       \end{array}.
\]
Defining the Jacobian matrix $\hat{J}$ we have
\[
 \hat{J}=\left( \begin{array}{cc}
                         \partial f_1/\partial x_1  & \partial f_1/\partial x_2 \\
                          \partial f_2/\partial x_1     &\partial f_2/\partial x_2
             \end{array} \right),
\]
we can rephrase Newton's method as
\[
\left(\begin{array}{c} x_1^{n+1} \\ x_2^{n+1} \end{array} \right)=
\left(\begin{array}{c} x_1^{n} \\ x_2^{n} \end{array} \right)+
\left(\begin{array}{c} h_1^{n} \\ h_2^{n} \end{array} \right),
\]
where we have defined
\[
   \left(\begin{array}{c} h_1^{n} \\ h_2^{n} \end{array} \right)=
   -{\bf \hat{J}}^{-1}
   \left(\begin{array}{c} f_1(x_1^{n},x_2^{n}) \\ f_2(x_1^{n},x_2^{n}) \end{array} \right).
\]
We need thus to compute the inverse of the Jacobian matrix and it
is to understand that difficulties  may
arise in case $\hat{J}$ is nearly singular.

It is rather straightforward to extend the above scheme to systems of
more than two non-linear equations. In our case, the Jacobian matrix is given by the Hessian that represents the second derivative of cost function. 

\subsection*{Steepest descent}

The basic idea of gradient descent is
that a function $F(\mathbf{x})$, 
$\mathbf{x} \equiv (x_1,\cdots,x_n)$, decreases fastest if one goes from $\bf {x}$ in the
direction of the negative gradient $-\nabla F(\mathbf{x})$.

It can be shown that if 
\[
\mathbf{x}_{k+1} = \mathbf{x}_k - \gamma_k \nabla F(\mathbf{x}_k),
\]
with $\gamma_k > 0$.

For $\gamma_k$ small enough, then $F(\mathbf{x}_{k+1}) \leq
F(\mathbf{x}_k)$. This means that for a sufficiently small $\gamma_k$
we are always moving towards smaller function values, i.e a minimum.

\subsection*{More on Steepest descent}

The previous observation is the basis of the method of steepest
descent, which is also referred to as just gradient descent (GD). One
starts with an initial guess $\mathbf{x}_0$ for a minimum of $F$ and
computes new approximations according to

\[
\mathbf{x}_{k+1} = \mathbf{x}_k - \gamma_k \nabla F(\mathbf{x}_k), \ \ k \geq 0.
\]

The parameter $\gamma_k$ is often referred to as the step length or
the learning rate within the context of Machine Learning.

\subsection*{The ideal}

Ideally the sequence $\{\mathbf{x}_k \}_{k=0}$ converges to a global
minimum of the function $F$. In general we do not know if we are in a
global or local minimum. In the special case when $F$ is a convex
function, all local minima are also global minima, so in this case
gradient descent can converge to the global solution. The advantage of
this scheme is that it is conceptually simple and straightforward to
implement. However the method in this form has some severe
limitations:

In machine learing we are often faced with non-convex high dimensional
cost functions with many local minima. Since GD is deterministic we
will get stuck in a local minimum, if the method converges, unless we
have a very good intial guess. This also implies that the scheme is
sensitive to the chosen initial condition.

Note that the gradient is a function of $\mathbf{x} =
(x_1,\cdots,x_n)$ which makes it expensive to compute numerically.

\subsection*{The sensitiveness of the gradient descent}

The gradient descent method 
is sensitive to the choice of learning rate $\gamma_k$. This is due
to the fact that we are only guaranteed that $F(\mathbf{x}_{k+1}) \leq
F(\mathbf{x}_k)$ for sufficiently small $\gamma_k$. The problem is to
determine an optimal learning rate. If the learning rate is chosen too
small the method will take a long time to converge and if it is too
large we can experience erratic behavior.

Many of these shortcomings can be alleviated by introducing
randomness. One such method is that of Stochastic Gradient Descent
(SGD), see below.

\subsection*{Convex functions}

Ideally we want our cost/loss function to be convex(concave).

First we give the definition of a convex set: A set $C$ in
$\mathbb{R}^n$ is said to be convex if, for all $x$ and $y$ in $C$ and
all $t \in (0,1)$ , the point $(1 − t)x + ty$ also belongs to
C. Geometrically this means that every point on the line segment
connecting $x$ and $y$ is in $C$ as discussed below.

The convex subsets of $\mathbb{R}$ are the intervals of
$\mathbb{R}$. Examples of convex sets of $\mathbb{R}^2$ are the
regular polygons (triangles, rectangles, pentagons, etc...).

\subsection*{Convex function}

\textbf{Convex function}: Let $X \subset \mathbb{R}^n$ be a convex set. Assume that the function $f: X \rightarrow \mathbb{R}$ is continuous, then $f$ is said to be convex if $$f(tx_1 + (1-t)x_2) \leq tf(x_1) + (1-t)f(x_2) $$ for all $x_1, x_2 \in X$ and for all $t \in [0,1]$. If $\leq$ is replaced with a strict inequaltiy in the definition, we demand $x_1 \neq x_2$ and $t\in(0,1)$ then $f$ is said to be strictly convex. For a single variable function, convexity means that if you draw a straight line connecting $f(x_1)$ and $f(x_2)$, the value of the function on the interval $[x_1,x_2]$ is always below the line as illustrated below.

\subsection*{Conditions on convex functions}

In the following we state first and second-order conditions which
ensures convexity of a function $f$. We write $D_f$ to denote the
domain of $f$, i.e the subset of $R^n$ where $f$ is defined. For more
details and proofs we refer to: \href{{http://stanford.edu/boyd/cvxbook/, 2004}}{S. Boyd and L. Vandenberghe. Convex Optimization. Cambridge University Press}.


% --- begin paragraph admon ---
\paragraph{First order condition.}
Suppose $f$ is differentiable (i.e $\nabla f(x)$ is well defined for
all $x$ in the domain of $f$). Then $f$ is convex if and only if $D_f$
is a convex set and $$f(y) \geq f(x) + \nabla f(x)^T (y-x) $$ holds
for all $x,y \in D_f$. This condition means that for a convex function
the first order Taylor expansion (right hand side above) at any point
a global under estimator of the function. To convince yourself you can
make a drawing of $f(x) = x^2+1$ and draw the tangent line to $f(x)$ and
note that it is always below the graph.
% --- end paragraph admon ---




% --- begin paragraph admon ---
\paragraph{Second order condition.}
Assume that $f$ is twice
differentiable, i.e the Hessian matrix exists at each point in
$D_f$. Then $f$ is convex if and only if $D_f$ is a convex set and its
Hessian is positive semi-definite for all $x\in D_f$. For a
single-variable function this reduces to $f''(x) \geq 0$. Geometrically this means that $f$ has nonnegative curvature
everywhere.
% --- end paragraph admon ---



This condition is particularly useful since it gives us an procedure for determining if the function under consideration is convex, apart from using the definition.

\subsection*{More on convex functions}

The next result is of great importance to us and the reason why we are
going on about convex functions. In machine learning we frequently
have to minimize a loss/cost function in order to find the best
parameters for the model we are considering. 

Ideally we want the
global minimum (for high-dimensional models it is hard to know
if we have local or global minimum). However, if the cost/loss function
is convex the following result provides invaluable information:


% --- begin paragraph admon ---
\paragraph{Any minimum is global for convex functions.}
Consider the problem of finding $x \in \mathbb{R}^n$ such that $f(x)$
is minimal, where $f$ is convex and differentiable. Then, any point
$x^*$ that satisfies $\nabla f(x^*) = 0$ is a global minimum.
% --- end paragraph admon ---



This result means that if we know that the cost/loss function is convex and we are able to find a minimum, we are guaranteed that it is a global minimum.

\subsection*{Some simple problems}

\begin{enumerate}
\item Show that $f(x)=x^2$ is convex for $x \in \mathbb{R}$ using the definition of convexity. Hint: If you re-write the definition, $f$ is convex if the following holds for all $x,y \in D_f$ and any $\lambda \in [0,1]$ $\lambda f(x)+(1-\lambda)f(y)-f(\lambda x + (1-\lambda) y ) \geq 0$.

\item Using the second order condition show that the following functions are convex on the specified domain.
\begin{itemize}

 \item $f(x) = e^x$ is convex for $x \in \mathbb{R}$.

 \item $g(x) = -\ln(x)$ is convex for $x \in (0,\infty)$.

\end{itemize}

\noindent
\item Let $f(x) = x^2$ and $g(x) = e^x$. Show that $f(g(x))$ and $g(f(x))$ is convex for $x \in \mathbb{R}$. Also show that if $f(x)$ is any convex function than $h(x) = e^{f(x)}$ is convex.

\item A norm is any function that satisfy the following properties
\begin{itemize}

 \item $f(\alpha x) = |\alpha| f(x)$ for all $\alpha \in \mathbb{R}$.

 \item $f(x+y) \leq f(x) + f(y)$

 \item $f(x) \leq 0$ for all $x \in \mathbb{R}^n$ with equality if and only if $x = 0$
\end{itemize}

\noindent
\end{enumerate}

\noindent
Using the definition of convexity, try to show that a function satisfying the properties above is convex (the third condition is not needed to show this).

\subsection*{Standard steepest descent}

Before we proceed, we would like to discuss the approach called the
\textbf{standard Steepest descent}, which again leads to us having to be able
to compute a matrix. It belongs to the class of Conjugate Gradient methods (CG).

\href{{https://www.cs.cmu.edu/~quake-papers/painless-conjugate-gradient.pdf}}{The success of the CG method}
for finding solutions of non-linear problems is based on the theory
of conjugate gradients for linear systems of equations. It belongs to
the class of iterative methods for solving problems from linear
algebra of the type 
\begin{equation*} 
\hat{A}\hat{x} = \hat{b}.
\end{equation*} 

In the iterative process we end up with a problem like

\begin{equation*}
  \hat{r}= \hat{b}-\hat{A}\hat{x},
\end{equation*}
where $\hat{r}$ is the so-called residual or error in the iterative process.

When we have found the exact solution, $\hat{r}=0$.

\subsection*{Gradient method}

The residual is zero when we reach the minimum of the quadratic equation
\begin{equation*}
  P(\hat{x})=\frac{1}{2}\hat{x}^T\hat{A}\hat{x} - \hat{x}^T\hat{b},
\end{equation*}

with the constraint that the matrix $\hat{A}$ is positive definite and
symmetric.  This defines also the Hessian and we want it to be  positive definite.  

\subsection*{Steepest descent  method}

We denote the initial guess for $\hat{x}$ as $\hat{x}_0$. 
We can assume without loss of generality that
\begin{equation*}
\hat{x}_0=0,
\end{equation*}
or consider the system
\begin{equation*}
\hat{A}\hat{z} = \hat{b}-\hat{A}\hat{x}_0,
\end{equation*}
instead.

\subsection*{Steepest descent  method}

% --- begin paragraph admon ---
\paragraph{}
One can show that the solution $\hat{x}$ is also the unique minimizer of the quadratic form
\begin{equation*}
  f(\hat{x}) = \frac{1}{2}\hat{x}^T\hat{A}\hat{x} - \hat{x}^T \hat{x} , \quad \hat{x}\in\mathbf{R}^n. 
\end{equation*}
This suggests taking the first basis vector $\hat{r}_1$ (see below for definition) 
to be the gradient of $f$ at $\hat{x}=\hat{x}_0$, 
which equals
\begin{equation*}
\hat{A}\hat{x}_0-\hat{b},
\end{equation*}
and 
$\hat{x}_0=0$ it is equal $-\hat{b}$.
% --- end paragraph admon ---



\subsection*{Final expressions}

% --- begin paragraph admon ---
\paragraph{}
We can compute the residual iteratively as
\begin{equation*}
\hat{r}_{k+1}=\hat{b}-\hat{A}\hat{x}_{k+1},
 \end{equation*}
which equals
\begin{equation*}
\hat{b}-\hat{A}(\hat{x}_k+\alpha_k\hat{r}_k),
 \end{equation*}
or
\begin{equation*}
(\hat{b}-\hat{A}\hat{x}_k)-\alpha_k\hat{A}\hat{r}_k,
 \end{equation*}
which gives

\[
\alpha_k = \frac{\hat{r}_k^T\hat{r}_k}{\hat{r}_k^T\hat{A}\hat{r}_k}
\]
leading to the iterative scheme
\begin{equation*}
\hat{x}_{k+1}=\hat{x}_k-\alpha_k\hat{r}_{k},
 \end{equation*}
% --- end paragraph admon ---



\subsection*{Steepest descent example}






















\begin{minted}[fontsize=\fontsize{9pt}{9pt},linenos=false,mathescape,baselinestretch=1.0,fontfamily=tt,xleftmargin=7mm]{python}
import numpy as np
import numpy.linalg as la

import scipy.optimize as sopt

import matplotlib.pyplot as pt
from mpl_toolkits.mplot3d import axes3d

def f(x):
    return 0.5*x[0]**2 + 2.5*x[1]**2

def df(x):
    return np.array([x[0], 5*x[1]])

fig = pt.figure()
ax = fig.gca(projection="3d")

xmesh, ymesh = np.mgrid[-2:2:50j,-2:2:50j]
fmesh = f(np.array([xmesh, ymesh]))
ax.plot_surface(xmesh, ymesh, fmesh)

\end{minted}

And then as countor plot




\begin{minted}[fontsize=\fontsize{9pt}{9pt},linenos=false,mathescape,baselinestretch=1.0,fontfamily=tt,xleftmargin=7mm]{python}
pt.axis("equal")
pt.contour(xmesh, ymesh, fmesh)
guesses = [np.array([2, 2./5])]

\end{minted}

Find guesses



\begin{minted}[fontsize=\fontsize{9pt}{9pt},linenos=false,mathescape,baselinestretch=1.0,fontfamily=tt,xleftmargin=7mm]{python}
x = guesses[-1]
s = -df(x)

\end{minted}

Run it!








\begin{minted}[fontsize=\fontsize{9pt}{9pt},linenos=false,mathescape,baselinestretch=1.0,fontfamily=tt,xleftmargin=7mm]{python}
def f1d(alpha):
    return f(x + alpha*s)

alpha_opt = sopt.golden(f1d)
next_guess = x + alpha_opt * s
guesses.append(next_guess)
print(next_guess)

\end{minted}

What happened?





\begin{minted}[fontsize=\fontsize{9pt}{9pt},linenos=false,mathescape,baselinestretch=1.0,fontfamily=tt,xleftmargin=7mm]{python}
pt.axis("equal")
pt.contour(xmesh, ymesh, fmesh, 50)
it_array = np.array(guesses)
pt.plot(it_array.T[0], it_array.T[1], "x-")

\end{minted}


\subsection*{Conjugate gradient method}

% --- begin paragraph admon ---
\paragraph{}
In the CG method we define so-called conjugate directions and two vectors 
$\hat{s}$ and $\hat{t}$
are said to be
conjugate if
\begin{equation*}
\hat{s}^T\hat{A}\hat{t}= 0.
\end{equation*}
The philosophy of the CG method is to perform searches in various conjugate directions
of our vectors $\hat{x}_i$ obeying the above criterion, namely
\begin{equation*}
\hat{x}_i^T\hat{A}\hat{x}_j= 0.
\end{equation*}
Two vectors are conjugate if they are orthogonal with respect to 
this inner product. Being conjugate is a symmetric relation: if $\hat{s}$ is conjugate to $\hat{t}$, then $\hat{t}$ is conjugate to $\hat{s}$.
% --- end paragraph admon ---



\subsection*{Conjugate gradient method}

% --- begin paragraph admon ---
\paragraph{}
An example is given by the eigenvectors of the matrix
\begin{equation*}
\hat{v}_i^T\hat{A}\hat{v}_j= \lambda\hat{v}_i^T\hat{v}_j,
\end{equation*}
which is zero unless $i=j$.
% --- end paragraph admon ---



\subsection*{Conjugate gradient method}

% --- begin paragraph admon ---
\paragraph{}
Assume now that we have a symmetric positive-definite matrix $\hat{A}$ of size
$n\times n$. At each iteration $i+1$ we obtain the conjugate direction of a vector
\begin{equation*}
\hat{x}_{i+1}=\hat{x}_{i}+\alpha_i\hat{p}_{i}. 
\end{equation*}
We assume that $\hat{p}_{i}$ is a sequence of $n$ mutually conjugate directions. 
Then the $\hat{p}_{i}$  form a basis of $R^n$ and we can expand the solution 
$  \hat{A}\hat{x} = \hat{b}$ in this basis, namely

\begin{equation*}
  \hat{x}  = \sum^{n}_{i=1} \alpha_i \hat{p}_i.
\end{equation*}
% --- end paragraph admon ---



\subsection*{Conjugate gradient method}

% --- begin paragraph admon ---
\paragraph{}
The coefficients are given by
\begin{equation*}
    \mathbf{A}\mathbf{x} = \sum^{n}_{i=1} \alpha_i \mathbf{A} \mathbf{p}_i = \mathbf{b}.
\end{equation*}
Multiplying with $\hat{p}_k^T$  from the left gives

\begin{equation*}
  \hat{p}_k^T \hat{A}\hat{x} = \sum^{n}_{i=1} \alpha_i\hat{p}_k^T \hat{A}\hat{p}_i= \hat{p}_k^T \hat{b},
\end{equation*}
and we can define the coefficients $\alpha_k$ as

\begin{equation*}
    \alpha_k = \frac{\hat{p}_k^T \hat{b}}{\hat{p}_k^T \hat{A} \hat{p}_k}
\end{equation*}
% --- end paragraph admon ---



\subsection*{Conjugate gradient method and iterations}

% --- begin paragraph admon ---
\paragraph{}

If we choose the conjugate vectors $\hat{p}_k$ carefully, 
then we may not need all of them to obtain a good approximation to the solution 
$\hat{x}$. 
We want to regard the conjugate gradient method as an iterative method. 
This will us to solve systems where $n$ is so large that the direct 
method would take too much time.

We denote the initial guess for $\hat{x}$ as $\hat{x}_0$. 
We can assume without loss of generality that
\begin{equation*}
\hat{x}_0=0,
\end{equation*}
or consider the system
\begin{equation*}
\hat{A}\hat{z} = \hat{b}-\hat{A}\hat{x}_0,
\end{equation*}
instead.
% --- end paragraph admon ---



\subsection*{Conjugate gradient method}

% --- begin paragraph admon ---
\paragraph{}
One can show that the solution $\hat{x}$ is also the unique minimizer of the quadratic form
\begin{equation*}
  f(\hat{x}) = \frac{1}{2}\hat{x}^T\hat{A}\hat{x} - \hat{x}^T \hat{x} , \quad \hat{x}\in\mathbf{R}^n. 
\end{equation*}
This suggests taking the first basis vector $\hat{p}_1$ 
to be the gradient of $f$ at $\hat{x}=\hat{x}_0$, 
which equals
\begin{equation*}
\hat{A}\hat{x}_0-\hat{b},
\end{equation*}
and 
$\hat{x}_0=0$ it is equal $-\hat{b}$.
The other vectors in the basis will be conjugate to the gradient, 
hence the name conjugate gradient method.
% --- end paragraph admon ---



\subsection*{Conjugate gradient method}

% --- begin paragraph admon ---
\paragraph{}
Let  $\hat{r}_k$ be the residual at the $k$-th step:
\begin{equation*}
\hat{r}_k=\hat{b}-\hat{A}\hat{x}_k.
\end{equation*}
Note that $\hat{r}_k$ is the negative gradient of $f$ at 
$\hat{x}=\hat{x}_k$, 
so the gradient descent method would be to move in the direction $\hat{r}_k$. 
Here, we insist that the directions $\hat{p}_k$ are conjugate to each other, 
so we take the direction closest to the gradient $\hat{r}_k$  
under the conjugacy constraint. 
This gives the following expression
\begin{equation*}
\hat{p}_{k+1}=\hat{r}_k-\frac{\hat{p}_k^T \hat{A}\hat{r}_k}{\hat{p}_k^T\hat{A}\hat{p}_k} \hat{p}_k.
\end{equation*}
% --- end paragraph admon ---



\subsection*{Conjugate gradient method}

% --- begin paragraph admon ---
\paragraph{}
We can also  compute the residual iteratively as
\begin{equation*}
\hat{r}_{k+1}=\hat{b}-\hat{A}\hat{x}_{k+1},
 \end{equation*}
which equals
\begin{equation*}
\hat{b}-\hat{A}(\hat{x}_k+\alpha_k\hat{p}_k),
 \end{equation*}
or
\begin{equation*}
(\hat{b}-\hat{A}\hat{x}_k)-\alpha_k\hat{A}\hat{p}_k,
 \end{equation*}
which gives

\begin{equation*}
\hat{r}_{k+1}=\hat{r}_k-\hat{A}\hat{p}_{k},
 \end{equation*}
% --- end paragraph admon ---



\subsection*{Broyden–Fletcher–Goldfarb–Shanno algorithm}

% --- begin paragraph admon ---
\paragraph{}
The optimization problem is to minimize $f(\mathbf {x} )$ where $\mathbf {x}$  is a vector in $R^{n}$, and $f$ is a differentiable scalar function. There are no constraints on the values that  $\mathbf {x}$  can take.

The algorithm begins at an initial estimate for the optimal value $\mathbf {x}_{0}$ and proceeds iteratively to get a better estimate at each stage.

The search direction $p_k$ at stage $k$ is given by the solution of the analogue of the Newton equation
\[
B_{k}\mathbf {p} _{k}=-\nabla f(\mathbf {x}_{k}),
\]

where $B_{k}$ is an approximation to the Hessian matrix, which is
updated iteratively at each stage, and $\nabla f(\mathbf {x} _{k})$
is the gradient of the function
evaluated at $x_k$. 
A line search in the direction $p_k$ is then used to
find the next point $x_{k+1}$ by minimising 
\[
f(\mathbf {x}_{k}+\alpha \mathbf {p}_{k}),
\]
over the scalar $\alpha > 0$.
% --- end paragraph admon ---



\subsection*{Stochastic Gradient Descent}

Stochastic gradient descent (SGD) and variants thereof address some of
the shortcomings of the Gradient descent method discussed above.

The underlying idea of SGD comes from the observation that a given 
function, which we want to minimize, can almost always be written as a
sum over $n$ data points $\{\mathbf{x}_i\}_{i=1}^n$,
\[
C(\mathbf{\beta}) = \sum_{i=1}^n c_i(\mathbf{x}_i,
\mathbf{\beta}). 
\]

\subsection*{Computation of gradients}

This in turn means that the gradient can be
computed as a sum over $i$-gradients 
\[
\nabla_\beta C(\mathbf{\beta}) = \sum_i^n \nabla_\beta c_i(\mathbf{x}_i,
\mathbf{\beta}).
\]

Stochasticity/randomness is introduced by only taking the
gradient on a subset of the data called minibatches.  If there are $n$
data points and the size of each minibatch is $M$, there will be $n/M$
minibatches. We denote these minibatches by $B_k$ where
$k=1,\cdots,n/M$.

\subsection*{SGD example}
As an example, suppose we have $10$ data points $(\mathbf{x}_1,\cdots, \mathbf{x}_{10})$ 
and we choose to have $M=5$ minibathces,
then each minibatch contains two data points. In particular we have
$B_1 = (\mathbf{x}_1,\mathbf{x}_2), \cdots, B_5 =
(\mathbf{x}_9,\mathbf{x}_{10})$. Note that if you choose $M=1$ you
have only a single batch with all data points and on the other extreme,
you may choose $M=n$ resulting in a minibatch for each datapoint, i.e
$B_k = \mathbf{x}_k$.

The idea is now to approximate the gradient by replacing the sum over
all data points with a sum over the data points in one the minibatches
picked at random in each gradient descent step 
\[
\nabla_{\beta}
C(\mathbf{\beta}) = \sum_{i=1}^n \nabla_\beta c_i(\mathbf{x}_i,
\mathbf{\beta}) \rightarrow \sum_{i \in B_k}^n \nabla_\beta
c_i(\mathbf{x}_i, \mathbf{\beta}).
\]

\subsection*{The gradient step}

Thus a gradient descent step now looks like 
\[
\beta_{j+1} = \beta_j - \gamma_j \sum_{i \in B_k}^n \nabla_\beta c_i(\mathbf{x}_i,
\mathbf{\beta})
\]

where $k$ is picked at random with equal
probability from $[1,n/M]$. An iteration over the number of
minibathces (n/M) is commonly referred to as an epoch. Thus it is
typical to choose a number of epochs and for each epoch iterate over
the number of minibatches, as exemplified in the code below.

\subsection*{Simple example code}
















\begin{minted}[fontsize=\fontsize{9pt}{9pt},linenos=false,mathescape,baselinestretch=1.0,fontfamily=tt,xleftmargin=7mm]{python}
import numpy as np 

n = 100 #100 datapoints 
M = 5   #size of each minibatch
m = int(n/M) #number of minibatches
n_epochs = 10 #number of epochs

j = 0
for epoch in range(1,n_epochs+1):
    for i in range(m):
        k = np.random.randint(m) #Pick the k-th minibatch at random
        #Compute the gradient using the data in minibatch Bk
        #Compute new suggestion for 
        j += 1

\end{minted}


Taking the gradient only on a subset of the data has two important
benefits. First, it introduces randomness which decreases the chance
that our opmization scheme gets stuck in a local minima. Second, if
the size of the minibatches are small relative to the number of
datapoints ($M <  n$), the computation of the gradient is much
cheaper since we sum over the datapoints in the $k-th$ minibatch and not
all $n$ datapoints.

\subsection*{When do we stop?}

A natural question is when do we stop the search for a new minimum?
One possibility is to compute the full gradient after a given number
of epochs and check if the norm of the gradient is smaller than some
threshold and stop if true. However, the condition that the gradient
is zero is valid also for local minima, so this would only tell us
that we are close to a local/global minimum. However, we could also
evaluate the cost function at this point, store the result and
continue the search. If the test kicks in at a later stage we can
compare the values of the cost function and keep the $\beta$ that
gave the lowest value.

\subsection*{Slightly different approach}

Another approach is to let the step length $\gamma_j$ depend on the
number of epochs in such a way that it becomes very small after a
reasonable time such that we do not move at all.

As an example, let $e = 0,1,2,3,\cdots$ denote the current epoch and let $t_0, t_1 > 0$ be two fixed numbers. Furthermore, let $t = e \cdot m + i$ where $m$ is the number of minibatches and $i=0,\cdots,m-1$. Then the function $$\gamma_j(t; t_0, t_1) = \frac{t_0}{t+t_1} $$ goes to zero as the number of epochs gets large. I.e. we start with a step length $\gamma_j (0; t_0, t_1) = t_0/t_1$ which decays in \emph{time} $t$.

In this way we can fix the number of epochs, compute $\beta$ and
evaluate the cost function at the end. Repeating the computation will
give a different result since the scheme is random by design. Then we
pick the final $\beta$ that gives the lowest value of the cost
function.


























\begin{minted}[fontsize=\fontsize{9pt}{9pt},linenos=false,mathescape,baselinestretch=1.0,fontfamily=tt,xleftmargin=7mm]{python}
import numpy as np 

def step_length(t,t0,t1):
    return t0/(t+t1)

n = 100 #100 datapoints 
M = 5   #size of each minibatch
m = int(n/M) #number of minibatches
n_epochs = 500 #number of epochs
t0 = 1.0
t1 = 10

gamma_j = t0/t1
j = 0
for epoch in range(1,n_epochs+1):
    for i in range(m):
        k = np.random.randint(m) #Pick the k-th minibatch at random
        #Compute the gradient using the data in minibatch Bk
        #Compute new suggestion for beta
        t = epoch*m+i
        gamma_j = step_length(t,t0,t1)
        j += 1

print("gamma_j after %d epochs: %g" % (n_epochs,gamma_j))

\end{minted}


\subsection*{Program for stochastic gradient}





































































\begin{minted}[fontsize=\fontsize{9pt}{9pt},linenos=false,mathescape,baselinestretch=1.0,fontfamily=tt,xleftmargin=7mm]{python}
# Importing various packages
from math import exp, sqrt
from random import random, seed
import numpy as np
import matplotlib.pyplot as plt
from sklearn.linear_model import SGDRegressor

x = 2*np.random.rand(100,1)
y = 4+3*x+np.random.randn(100,1)

xb = np.c_[np.ones((100,1)), x]
theta_linreg = np.linalg.inv(xb.T.dot(xb)).dot(xb.T).dot(y)
print("Own inversion")
print(theta_linreg)
sgdreg = SGDRegressor(n_iter = 50, penalty=None, eta0=0.1)
sgdreg.fit(x,y.ravel())
print("sgdreg from scikit")
print(sgdreg.intercept_, sgdreg.coef_)


theta = np.random.randn(2,1)

eta = 0.1
Niterations = 1000
m = 100

for iter in range(Niterations):
    gradients = 2.0/m*xb.T.dot(xb.dot(theta)-y)
    theta -= eta*gradients
print("theta frm own gd")
print(theta)

xnew = np.array([[0],[2]])
xbnew = np.c_[np.ones((2,1)), xnew]
ypredict = xbnew.dot(theta)
ypredict2 = xbnew.dot(theta_linreg)


n_epochs = 50
t0, t1 = 5, 50
m = 100
def learning_schedule(t):
    return t0/(t+t1)

theta = np.random.randn(2,1)

for epoch in range(n_epochs):
    for i in range(m):
        random_index = np.random.randint(m)
        xi = xb[random_index:random_index+1]
        yi = y[random_index:random_index+1]
        gradients = 2 * xi.T.dot(xi.dot(theta)-yi)
        eta = learning_schedule(epoch*m+i)
        theta = theta - eta*gradients
print("theta from own sdg")
print(theta)


plt.plot(xnew, ypredict, "r-")
plt.plot(xnew, ypredict2, "b-")
plt.plot(x, y ,'ro')
plt.axis([0,2.0,0, 15.0])
plt.xlabel(r'$x$')
plt.ylabel(r'$y$')
plt.title(r'Random numbers ')
plt.show()


\end{minted}


\subsection*{Using gradient descent methods, limitations}

\begin{itemize}
\item \textbf{Gradient descent (GD) finds local minima of our function}. Since the GD algorithm is deterministic, if it converges, it will converge to a local minimum of our energy function. Because in ML we are often dealing with extremely rugged landscapes with many local minima, this can lead to poor performance.

\item \textbf{GD is sensitive to initial conditions}. One consequence of the local nature of GD is that initial conditions matter. Depending on where one starts, one will end up at a different local minima. Therefore, it is very important to think about how one initializes the training process. This is true for GD as well as more complicated variants of GD.

\item \textbf{Gradients are computationally expensive to calculate for large datasets}. In many cases in statistics and ML, the energy function is a sum of terms, with one term for each data point. For example, in linear regression, $E \propto \sum_{i=1}^n (y_i - \mathbf{w}^T\cdot\mathbf{x}_i)^2$; for logistic regression, the square error is replaced by the cross entropy. To calculate the gradient we have to sum over \emph{all} $n$ data points. Doing this at every GD step becomes extremely computationally expensive. An ingenious solution to this, is to calculate the gradients using small subsets of the data called ``mini batches''. This has the added benefit of introducing stochasticity into our algorithm.

\item \textbf{GD is very sensitive to choices of learning rates}. GD is extremely sensitive to the choice of learning rates. If the learning rate is very small, the training process take an extremely long time. For larger learning rates, GD can diverge and give poor results. Furthermore, depending on what the local landscape looks like, we have to modify the learning rates to ensure convergence. Ideally, we would \emph{adaptively} choose the learning rates to match the landscape.

\item \textbf{GD treats all directions in parameter space uniformly.} Another major drawback of GD is that unlike Newton's method, the learning rate for GD is the same in all directions in parameter space. For this reason, the maximum learning rate is set by the behavior of the steepest direction and this can significantly slow down training. Ideally, we would like to take large steps in flat directions and small steps in steep directions. Since we are exploring rugged landscapes where curvatures change, this requires us to keep track of not only the gradient but second derivatives. The ideal scenario would be to calculate the Hessian but this proves to be too computationally expensive. 

\item GD can take exponential time to escape saddle points, even with random initialization. As we mentioned, GD is extremely sensitive to initial condition since it determines the particular local minimum GD would eventually reach. However, even with a good initialization scheme, through the introduction of randomness, GD can still take exponential time to escape saddle points.
\end{itemize}

\noindent
\subsection*{Codes from numerical recipes}

% --- begin paragraph admon ---
\paragraph{}
You can however use codes we have adapted from the text \href{{http://www.nr.com/}}{Numerical Recipes in C++}, see chapter 10.7.  
Here we present a program, which you also can find at the webpage of the course we use the functions \textbf{dfpmin} and \textbf{lnsrch}.  This is a variant of the Broyden et al algorithm discussed in the previous slide.

\begin{itemize}
\item The program uses the harmonic oscillator in one dimensions as example.

\item The program does not use armadillo to handle vectors and matrices, but employs rather my own vector-matrix class. These auxiliary functions, and the main program \emph{model.cpp} can all be found under the \href{{https://github.com/CompPhysics/ComputationalPhysics2/tree/gh-pages/doc/pub/cg/programs/c%2B%2B}}{program link here}.
\end{itemize}

\noindent
Below we show only excerpts from the main program. For the full program, see the above link.
% --- end paragraph admon ---



\subsection*{Finding the minimum of the harmonic oscillator model in one dimension}

% --- begin paragraph admon ---
\paragraph{}























\begin{minted}[fontsize=\fontsize{9pt}{9pt},linenos=false,mathescape,baselinestretch=1.0,fontfamily=tt,xleftmargin=7mm]{c++}
//   Main function begins here
int main()
{
     int n, iter;
     double gtol, fret;
     double alpha;
     n = 1;
//   reserve space in memory for vectors containing the variational
//   parameters
     Vector g(n), p(n);
     cout << "Read in guess for alpha" << endl;
     cin >> alpha;
     gtol = 1.0e-5;
//   now call dfmin and compute the minimum
     p(0) = alpha;
     dfpmin(p, n, gtol, &iter, &fret, Efunction, dEfunction);
     cout << "Value of energy minimum = " << fret << endl;
     cout << "Number of iterations = " << iter << endl;
     cout << "Value of alpha at minimum = " << p(0) << endl;
      return 0;
}  // end of main program


\end{minted}
% --- end paragraph admon ---



\subsection*{Functions to observe}

% --- begin paragraph admon ---
\paragraph{}
The functions \textbf{Efunction} and \textbf{dEfunction} compute the expectation value of the energy and its derivative.
They use the the quasi-Newton method of \href{{https://www.springer.com/it/book/9780387303031}}{Broyden, Fletcher, Goldfarb, and Shanno (BFGS)}
It uses the first derivatives only. The BFGS algorithm has proven good performance even for non-smooth optimizations. 
These functions need to be changed when you want to your own derivatives.













\begin{minted}[fontsize=\fontsize{9pt}{9pt},linenos=false,mathescape,baselinestretch=1.0,fontfamily=tt,xleftmargin=7mm]{c++}
//  this function defines the expectation value of the local energy
double Efunction(Vector  &x)
{
  double value = x(0)*x(0)*0.5+1.0/(8*x(0)*x(0));
  return value;
} // end of function to evaluate

//  this function defines the derivative of the energy 
void dEfunction(Vector &x, Vector &g)
{
  g(0) = x(0)-1.0/(4*x(0)*x(0)*x(0));
} // end of function to evaluate

\end{minted}

You need to change these functions in order to compute the local energy for your system. I used 1000
cycles per call to get a new value of $\langle E_L[\alpha]\rangle$.
When I compute the local energy I also compute its derivative.
After roughly 10-20 iterations I got a converged result in terms of $\alpha$.
% --- end paragraph admon ---




% ------------------- end of main content ---------------

\end{document}


 \clearemptydoublepage
\chapter{Resampling Techniques, Bootstrap and Blocking}

\subsection*{Why resampling methods ?}

% --- begin paragraph admon ---
\paragraph{Statistical analysis.}
\begin{itemize}
    \item Our simulations can be treated as \emph{computer experiments}. This is particularly the case for Monte Carlo methods

    \item The results can be analysed with the same statistical tools as we would use analysing experimental data.

    \item As in all experiments, we are looking for expectation values and an estimate of how accurate they are, i.e., possible sources for errors.
\end{itemize}

\noindent
% --- end paragraph admon ---

    

\subsection*{Statistical analysis}

% --- begin paragraph admon ---
\paragraph{}
\begin{itemize}
    \item As in other experiments, many numerical  experiments have two classes of errors:
\begin{itemize}

      \item Statistical errors

      \item Systematical errors

\end{itemize}

\noindent
    \item Statistical errors can be estimated using standard tools from statistics

    \item Systematical errors are method specific and must be treated differently from case to case. 
\end{itemize}

\noindent
% --- end paragraph admon ---

    

\subsection*{Statistics, wrapping up from last week}

% --- begin paragraph admon ---
\paragraph{}
Let us analyze the problem by splitting up the correlation term into
partial sums of the form:
\[
f_d = \frac{1}{n-d}\sum_{k=1}^{n-d}(x_k - \bar x_n)(x_{k+d} - \bar x_n)
\]
The correlation term of the error can now be rewritten in terms of
$f_d$
\[
\frac{2}{n}\sum_{k<l} (x_k - \bar x_n)(x_l - \bar x_n) =
2\sum_{d=1}^{n-1} f_d
\]
The value of $f_d$ reflects the correlation between measurements
separated by the distance $d$ in the sample samples.  Notice that for
$d=0$, $f$ is just the sample variance, $\mathrm{var}(x)$. If we divide $f_d$
by $\mathrm{var}(x)$, we arrive at the so called \emph{autocorrelation function}
\[
\kappa_d = \frac{f_d}{\mathrm{var}(x)}
\]
which gives us a useful measure of pairwise correlations
starting always at $1$ for $d=0$.
% --- end paragraph admon ---



\subsection*{Statistics, final expression}

% --- begin paragraph admon ---
\paragraph{}
The sample error can now be
written in terms of the autocorrelation function:

\begin{align}
\mathrm{err}_X^2 &=
\frac{1}{n}\mathrm{var}(x)+\frac{2}{n}\cdot\mathrm{var}(x)\sum_{d=1}^{n-1}
\frac{f_d}{\mathrm{var}(x)}\nonumber\\ &=&
\left(1+2\sum_{d=1}^{n-1}\kappa_d\right)\frac{1}{n}\mathrm{var}(x)\nonumber\\
&=\frac{\tau}{n}\cdot\mathrm{var}(x)
\end{align}

and we see that $\mathrm{err}_X$ can be expressed in terms the
uncorrelated sample variance times a correction factor $\tau$ which
accounts for the correlation between measurements. We call this
correction factor the \emph{autocorrelation time}:
\begin{equation}
\tau = 1+2\sum_{d=1}^{n-1}\kappa_d
\label{eq:autocorrelation_time}
\end{equation}
% --- end paragraph admon ---



\subsection*{Statistics, effective number of correlations}

% --- begin paragraph admon ---
\paragraph{}
For a correlation free experiment, $\tau$
equals 1.

We can interpret a sequential
correlation as an effective reduction of the number of measurements by
a factor $\tau$. The effective number of measurements becomes:
\[
n_\mathrm{eff} = \frac{n}{\tau}
\]
To neglect the autocorrelation time $\tau$ will always cause our
simple uncorrelated estimate of $\mathrm{err}_X^2\approx \mathrm{var}(x)/n$ to
be less than the true sample error. The estimate of the error will be
too \emph{good}. On the other hand, the calculation of the full
autocorrelation time poses an efficiency problem if the set of
measurements is very large.
% --- end paragraph admon ---



\subsection*{Can we understand this? Time Auto-correlation Function}

% --- begin paragraph admon ---
\paragraph{}

The so-called time-displacement autocorrelation $\phi(t)$ for a quantity $\mathbf{M}$ is given by
\[
\phi(t) = \int dt' \left[\mathbf{M}(t')-\langle \mathbf{M} \rangle\right]\left[\mathbf{M}(t'+t)-\langle \mathbf{M} \rangle\right],
\]
which can be rewritten as 
\[
\phi(t) = \int dt' \left[\mathbf{M}(t')\mathbf{M}(t'+t)-\langle \mathbf{M} \rangle^2\right],
\]
where $\langle \mathbf{M} \rangle$ is the average value and
$\mathbf{M}(t)$ its instantaneous value. We can discretize this function as follows, where we used our
set of computed values $\mathbf{M}(t)$ for a set of discretized times (our Monte Carlo cycles corresponding to moving all electrons?)
\[
\phi(t)  = \frac{1}{t_{\mathrm{max}}-t}\sum_{t'=0}^{t_{\mathrm{max}}-t}\mathbf{M}(t')\mathbf{M}(t'+t)
-\frac{1}{t_{\mathrm{max}}-t}\sum_{t'=0}^{t_{\mathrm{max}}-t}\mathbf{M}(t')\times
\frac{1}{t_{\mathrm{max}}-t}\sum_{t'=0}^{t_{\mathrm{max}}-t}\mathbf{M}(t'+t).
\label{eq:phitf}
\]
% --- end paragraph admon ---



\subsection*{Time Auto-correlation Function}

% --- begin paragraph admon ---
\paragraph{}

One should be careful with times close to $t_{\mathrm{max}}$, the upper limit of the sums 
becomes small and we end up integrating over a rather small time interval. This means that the statistical
error in $\phi(t)$ due to the random nature of the fluctuations in $\mathbf{M}(t)$ can become large.

One should therefore choose $t \ll t_{\mathrm{max}}$.

Note that the variable $\mathbf{M}$ can be any expectation values of interest.

The time-correlation function gives a measure of the correlation between the various values of the variable 
at a time $t'$ and a time $t'+t$. If we multiply the values of $\mathbf{M}$ at these two different times,
we will get a positive contribution if they are fluctuating in the same direction, or a negative value
if they fluctuate in the opposite direction. If we then integrate over time, or use the discretized version of, the time correlation function $\phi(t)$ should take a non-zero value if the fluctuations are 
correlated, else it should gradually go to zero. For times a long way apart 
the different values of $\mathbf{M}$  are most likely 
uncorrelated and $\phi(t)$ should be zero.
% --- end paragraph admon ---



\subsection*{Time Auto-correlation Function}

% --- begin paragraph admon ---
\paragraph{}
We can derive the correlation time by observing that our Metropolis algorithm is based on a random
walk in the space of all  possible spin configurations. 
Our probability 
distribution function $\mathbf{\hat{w}}(t)$ after a given number of time steps $t$ could be written as
\[
   \mathbf{\hat{w}}(t) = \mathbf{\hat{W}^t\hat{w}}(0),
\]
with $\mathbf{\hat{w}}(0)$ the distribution at $t=0$ and $\mathbf{\hat{W}}$ representing the 
transition probability matrix. 
We can always expand $\mathbf{\hat{w}}(0)$ in terms of the right eigenvectors of 
$\mathbf{\hat{v}}$ of $\mathbf{\hat{W}}$ as 
\[
    \mathbf{\hat{w}}(0)  = \sum_i\alpha_i\mathbf{\hat{v}}_i,
\]
resulting in 
\[
   \mathbf{\hat{w}}(t) = \mathbf{\hat{W}}^t\mathbf{\hat{w}}(0)=\mathbf{\hat{W}}^t\sum_i\alpha_i\mathbf{\hat{v}}_i=
\sum_i\lambda_i^t\alpha_i\mathbf{\hat{v}}_i,
\]
with $\lambda_i$ the $i^{\mathrm{th}}$ eigenvalue corresponding to  
the eigenvector $\mathbf{\hat{v}}_i$.
% --- end paragraph admon ---



\subsection*{Time Auto-correlation Function}

% --- begin paragraph admon ---
\paragraph{}
If we assume that $\lambda_0$ is the largest eigenvector we see that in the limit $t\rightarrow \infty$,
$\mathbf{\hat{w}}(t)$ becomes proportional to the corresponding eigenvector 
$\mathbf{\hat{v}}_0$. This is our steady state or final distribution. 

We can relate this property to an observable like the mean energy.
With the probabilty $\mathbf{\hat{w}}(t)$ (which in our case is the squared trial wave function) we
can write the expectation values as 
\[
 \langle \mathbf{M}(t) \rangle  = \sum_{\mu} \mathbf{\hat{w}}(t)_{\mu}\mathbf{M}_{\mu},
\] 
or as the scalar of a  vector product
 \[
 \langle \mathbf{M}(t) \rangle  = \mathbf{\hat{w}}(t)\mathbf{m},
\] 
with $\mathbf{m}$ being the vector whose elements are the values of $\mathbf{M}_{\mu}$ in its 
various microstates $\mu$.
% --- end paragraph admon ---



\subsection*{Time Auto-correlation Function}


% --- begin paragraph admon ---
\paragraph{}

We rewrite this relation  as
 \[
 \langle \mathbf{M}(t) \rangle  = \mathbf{\hat{w}}(t)\mathbf{m}=\sum_i\lambda_i^t\alpha_i\mathbf{\hat{v}}_i\mathbf{m}_i.
\] 
If we define $m_i=\mathbf{\hat{v}}_i\mathbf{m}_i$ as the expectation value of
$\mathbf{M}$ in the $i^{\mathrm{th}}$ eigenstate we can rewrite the last equation as
 \[
 \langle \mathbf{M}(t) \rangle  = \sum_i\lambda_i^t\alpha_im_i.
\] 
Since we have that in the limit $t\rightarrow \infty$ the mean value is dominated by the 
the largest eigenvalue $\lambda_0$, we can rewrite the last equation as
 \[
 \langle \mathbf{M}(t) \rangle  = \langle \mathbf{M}(\infty) \rangle+\sum_{i\ne 0}\lambda_i^t\alpha_im_i.
\] 
We define the quantity
\[
   \tau_i=-\frac{1}{log\lambda_i},
\]
and rewrite the last expectation value as
 \[
 \langle \mathbf{M}(t) \rangle  = \langle \mathbf{M}(\infty) \rangle+\sum_{i\ne 0}\alpha_im_ie^{-t/\tau_i}.
\label{eq:finalmeanm}
\]
% --- end paragraph admon ---



\subsection*{Time Auto-correlation Function}

% --- begin paragraph admon ---
\paragraph{}

The quantities $\tau_i$ are the correlation times for the system. They control also the auto-correlation function 
discussed above.  The longest correlation time is obviously given by the second largest
eigenvalue $\tau_1$, which normally defines the correlation time discussed above. For large times, this is the 
only correlation time that survives. If higher eigenvalues of the transition matrix are well separated from 
$\lambda_1$ and we simulate long enough,  $\tau_1$ may well define the correlation time. 
In other cases we may not be able to extract a reliable result for $\tau_1$. 
Coming back to the time correlation function $\phi(t)$ we can present a more general definition in terms
of the mean magnetizations $ \langle \mathbf{M}(t) \rangle$. Recalling that the mean value is equal 
to $ \langle \mathbf{M}(\infty) \rangle$ we arrive at the expectation values
\[
\phi(t) =\langle \mathbf{M}(0)-\mathbf{M}(\infty)\rangle \langle \mathbf{M}(t)-\mathbf{M}(\infty)\rangle,
\]
resulting in
\[
\phi(t) =\sum_{i,j\ne 0}m_i\alpha_im_j\alpha_je^{-t/\tau_i},
\]
which is appropriate for all times.
% --- end paragraph admon ---



\subsection*{Correlation Time}

% --- begin paragraph admon ---
\paragraph{}

If the correlation function decays exponentially
\[ \phi (t) \sim \exp{(-t/\tau)}\]
then the exponential correlation time can be computed as the average
\[   \tau_{\mathrm{exp}}  =  -\langle  \frac{t}{log|\frac{\phi(t)}{\phi(0)}|} \rangle. \]
If the decay is exponential, then
\[  \int_0^{\infty} dt \phi(t)  = \int_0^{\infty} dt \phi(0)\exp{(-t/\tau)}  = \tau \phi(0),\] 
which  suggests another measure of correlation
\[   \tau_{\mathrm{int}} = \sum_k \frac{\phi(k)}{\phi(0)}, \]
called the integrated correlation time.
% --- end paragraph admon ---



\subsection*{Resampling methods: Jackknife and Bootstrap}

Two famous
resampling methods are the \textbf{independent bootstrap} and \textbf{the jackknife}. 

The jackknife is a special case of the independent bootstrap. Still, the jackknife was made
popular prior to the independent bootstrap. And as the popularity of
the independent bootstrap soared, new variants, such as \textbf{the dependent bootstrap}.

The Jackknife and independent bootstrap work for
independent, identically distributed random variables.
If these conditions are not
satisfied, the methods will fail.  Yet, it should be said that if the data are
independent, identically distributed, and we only want to estimate the
variance of $\overline{X}$ (which often is the case), then there is no
need for bootstrapping. 

\subsection*{Resampling methods: Jackknife}

The Jackknife works by making many replicas of the estimator $\widehat{\theta}$. 
The jackknife is a resampling method, we explained that this happens by scrambling the data in some way. When using the jackknife, this is done by systematically leaving out one observation from the vector of observed values $\hat{x} = (x_1,x_2,\cdots,X_n)$. 
Let $\hat{x}_i$ denote the vector
\[
\hat{x}_i = (x_1,x_2,\cdots,x_{i-1},x_{i+1},\cdots,x_n),
\]

which equals the vector $\hat{x}$ with the exception that observation
number $i$ is left out. Using this notation, define
$\widehat{\theta}_i$ to be the estimator
$\widehat{\theta}$ computed using $\vec{X}_i$. 

\subsection*{Resampling methods: Jackknife estimator}

To get an estimate for the bias and
standard error of $\widehat{\theta}$, use the following
estimators for each component of $\widehat{\theta}$

\[
\widehat{\mathrm{Bias}}(\widehat \theta,\theta) = (n-1)\left( - \widehat{\theta} + \frac{1}{n}\sum_{i=1}^{n} \widehat \theta_i \right) \qquad \text{and} \qquad \widehat{\sigma}^2_{\widehat{\theta} } = \frac{n-1}{n}\sum_{i=1}^{n}( \widehat{\theta}_i - \frac{1}{n}\sum_{j=1}^{n}\widehat \theta_j )^2.
\]

\subsection*{Jackknife code example}






























\begin{minted}[fontsize=\fontsize{9pt}{9pt},linenos=false,mathescape,baselinestretch=1.0,fontfamily=tt,xleftmargin=7mm]{python}
from numpy import *
from numpy.random import randint, randn
from time import time

def jackknife(data, stat):
    n = len(data);t = zeros(n); inds = arange(n); t0 = time()
    ## 'jackknifing' by leaving out an observation for each i                                                                                                                      
    for i in range(n):
        t[i] = stat(delete(data,i) )

    # analysis                                                                                                                                                                     
    print("Runtime: %g sec" % (time()-t0)); print("Jackknife Statistics :")
    print("original           bias      std. error")
    print("%8g %14g %15g" % (stat(data),(n-1)*mean(t)/n, (n*var(t))**.5))

    return t


# Returns mean of data samples                                                                                                                                                     
def stat(data):
    return mean(data)


mu, sigma = 100, 15
datapoints = 10000
x = mu + sigma*random.randn(datapoints)
# jackknife returns the data sample                                                                                                                                                
t = jackknife(x, stat)


\end{minted}


\subsection*{Resampling methods: Bootstrap}

% --- begin paragraph admon ---
\paragraph{}
Bootstrapping is a nonparametric approach to statistical inference
that substitutes computation for more traditional distributional
assumptions and asymptotic results. Bootstrapping offers a number of
advantages: 
\begin{enumerate}
\item The bootstrap is quite general, although there are some cases in which it fails.  

\item Because it does not require distributional assumptions (such as normally distributed errors), the bootstrap can provide more accurate inferences when the data are not well behaved or when the sample size is small.  

\item It is possible to apply the bootstrap to statistics with sampling distributions that are difficult to derive, even asymptotically. 

\item It is relatively simple to apply the bootstrap to complex data-collection plans (such as stratified and clustered samples).
\end{enumerate}

\noindent
% --- end paragraph admon ---



\subsection*{Resampling methods: Bootstrap background}

Since $\widehat{\theta} = \widehat{\theta}(\hat{X})$ is a function of random variables,
$\widehat{\theta}$ itself must be a random variable. Thus it has
a pdf, call this function $p(\hat{t})$. The aim of the bootstrap is to
estimate $p(\hat{t})$ by the relative frequency of
$\widehat{\theta}$. You can think of this as using a histogram
in the place of $p(\hat{t})$. If the relative frequency closely
resembles $p(\vec{t})$, then using numerics, it is straight forward to
estimate all the interesting parameters of $p(\hat{t})$ using point
estimators.  

\subsection*{Resampling methods: More Bootstrap background}

In the case that $\widehat{\theta}$ has
more than one component, and the components are independent, we use the
same estimator on each component separately.  If the probability
density function of $X_i$, $p(x)$, had been known, then it would have
been straight forward to do this by: 
\begin{enumerate}
\item Drawing lots of numbers from $p(x)$, suppose we call one such set of numbers $(X_1^*, X_2^*, \cdots, X_n^*)$. 

\item Then using these numbers, we could compute a replica of $\widehat{\theta}$ called $\widehat{\theta}^*$. 
\end{enumerate}

\noindent
By repeated use of (1) and (2), many
estimates of $\widehat{\theta}$ could have been obtained. The
idea is to use the relative frequency of $\widehat{\theta}^*$
(think of a histogram) as an estimate of $p(\hat{t})$.

\subsection*{Resampling methods: Bootstrap approach}

But
unless there is enough information available about the process that
generated $X_1,X_2,\cdots,X_n$, $p(x)$ is in general
unknown. Therefore, \href{{https://projecteuclid.org/euclid.aos/1176344552}}{Efron in 1979}  asked the
question: What if we replace $p(x)$ by the relative frequency
of the observation $X_i$; if we draw observations in accordance with
the relative frequency of the observations, will we obtain the same
result in some asymptotic sense? The answer is yes.

Instead of generating the histogram for the relative
frequency of the observation $X_i$, just draw the values
$(X_1^*,X_2^*,\cdots,X_n^*)$ with replacement from the vector
$\hat{X}$. 

\subsection*{Resampling methods: Bootstrap steps}

The independent bootstrap works like this: 

\begin{enumerate}
\item Draw with replacement $n$ numbers for the observed variables $\hat{x} = (x_1,x_2,\cdots,x_n)$. 

\item Define a vector $\hat{x}^*$ containing the values which were drawn from $\hat{x}$. 

\item Using the vector $\hat{x}^*$ compute $\widehat{\theta}^*$ by evaluating $\widehat \theta$ under the observations $\hat{x}^*$. 

\item Repeat this process $k$ times. 
\end{enumerate}

\noindent
When you are done, you can draw a histogram of the relative frequency of $\widehat \theta^*$. This is your estimate of the probability distribution $p(t)$. Using this probability distribution you can estimate any statistics thereof. In principle you never draw the histogram of the relative frequency of $\widehat{\theta}^*$. Instead you use the estimators corresponding to the statistic of interest. For example, if you are interested in estimating the variance of $\widehat \theta$, apply the etsimator $\widehat \sigma^2$ to the values $\widehat \theta ^*$.

\subsection*{Code example for the Bootstrap method}

The following code starts with a Gaussian distribution with mean value $\mu =100$ and variance $\sigma=15$. We use this to generate the data used in the bootstrap analysis. The bootstrap analysis returns a data set after a given number of bootstrap operations (as many as we have data points). This data set consists of estimated mean values for each bootstrap operation. The histogram generated by the bootstrap method shows that the distribution for these mean values is also a Gaussian, centered around the mean value $\mu=100$ but with standard deviation $\sigma/\sqrt{n}$, where $n$ is the number of bootstrap samples (in this case the same as the number of original data points). The value of the standard deviation is what we expect from the central limit theorem. 


















































\begin{minted}[fontsize=\fontsize{9pt}{9pt},linenos=false,mathescape,baselinestretch=1.0,fontfamily=tt,xleftmargin=7mm]{python}

%matplotlib inline

from numpy import *
from numpy.random import randint, randn
from time import time
from scipy.stats import norm
import matplotlib.pyplot as plt

# Returns mean of bootstrap samples                                                                                                                                                
def stat(data):
    return mean(data)

# Bootstrap algorithm                                                                                                                                                              
def bootstrap(data, statistic, R):
    t = zeros(R); n = len(data); inds = arange(n); t0 = time()

    # non-parametric bootstrap                                                                                                                                                     
    for i in range(R):
        t[i] = statistic(data[randint(0,n,n)])

    # analysis                                                                                                                                                                     
    print("Runtime: %g sec" % (time()-t0)); print("Bootstrap Statistics :")
    print("original           bias      std. error")
    print("%8g %8g %14g %15g" % (statistic(data), std(data),\
                             mean(t), \
                             std(t)))
    return t


mu, sigma = 100, 15
datapoints = 10000
x = mu + sigma*random.randn(datapoints)
# bootstrap returns the data sample                                                                                                          t = bootstrap(x, stat, datapoints)
# the histogram of the bootstrapped  data  
t = bootstrap(x, stat, datapoints)
# the histogram of the bootstrapped  data                                            
n, binsboot, patches = plt.hist(t, bins=50, density='true',histtype='bar', color='red', alpha=0.75)

# add a 'best fit' line                                                                                                                                                          
y = norm.pdf( binsboot, mean(t), std(t))
lt = plt.plot(binsboot, y, 'r--', linewidth=1)
plt.xlabel('Smarts')
plt.ylabel('Probability')
plt.axis([99.5, 100.6, 0, 3.0])
plt.grid(True)

plt.show()


\end{minted}


\subsection*{Resampling methods: Blocking}

The blocking method was made popular by \href{{https://aip.scitation.org/doi/10.1063/1.457480}}{Flyvbjerg and Pedersen (1989)}
and has become one of the standard ways to estimate
$V(\widehat{\theta})$ for exactly one $\widehat{\theta}$, namely
$\widehat{\theta} = \overline{X}$. 

Assume $n = 2^d$ for some integer $d>1$ and $X_1,X_2,\cdots, X_n$ is a stationary time series to begin with. 
Moreover, assume that the time series is asymptotically uncorrelated. We switch to vector notation by arranging $X_1,X_2,\cdots,X_n$ in an $n$-tuple. Define:
\begin{align*}
\hat{X} = (X_1,X_2,\cdots,X_n).
\end{align*}

The strength of the blocking method is when the number of
observations, $n$ is large. For large $n$, the complexity of dependent
bootstrapping scales poorly, but the blocking method does not,
moreover, it becomes more accurate the larger $n$ is.

\subsection*{Blocking Transformations}
 We now define
blocking transformations. The idea is to take the mean of subsequent
pair of elements from $\vec{X}$ and form a new vector
$\vec{X}_1$. Continuing in the same way by taking the mean of
subsequent pairs of elements of $\vec{X}_1$ we obtain $\vec{X}_2$, and
so on. 
Define $\vec{X}_i$ recursively by:

\begin{align} 
(\vec{X}_0)_k &\equiv (\vec{X})_k \nonumber \\
(\vec{X}_{i+1})_k &\equiv \frac{1}{2}\Big( (\vec{X}_i)_{2k-1} +
(\vec{X}_i)_{2k} \Big) \qquad \text{for all} \qquad 1 \leq i \leq d-1
\end{align} 

The quantity $\vec{X}_k$ is
subject to $k$ \textbf{blocking transformations}.  We now have $d$ vectors
$\vec{X}_0, \vec{X}_1,\cdots,\vec X_{d-1}$ containing the subsequent
averages of observations. It turns out that if the components of
$\vec{X}$ is a stationary time series, then the components of
$\vec{X}_i$ is a stationary time series for all $0 \leq i \leq d-1$

We can then compute the autocovariance, the variance, sample mean, and
number of observations for each $i$. 
Let $\gamma_i, \sigma_i^2,
\overline{X}_i$ denote the autocovariance, variance and average of the
elements of $\vec{X}_i$ and let $n_i$ be the number of elements of
$\vec{X}_i$. It follows by induction that $n_i = n/2^i$. 

\subsection*{Blocking Transformations}

Using the
definition of the blocking transformation and the distributive
property of the covariance, it is clear that since $h =|i-j|$
we can define
\begin{align}
\gamma_{k+1}(h) &= cov\left( ({X}_{k+1})_{i}, ({X}_{k+1})_{j} \right) \nonumber \\
&=  \frac{1}{4}cov\left( ({X}_{k})_{2i-1} + ({X}_{k})_{2i}, ({X}_{k})_{2j-1} + ({X}_{k})_{2j} \right) \nonumber \\
&=  \frac{1}{2}\gamma_{k}(2h) + \frac{1}{2}\gamma_k(2h+1) \hspace{0.1cm} \mathrm{h = 0} \\
&=\frac{1}{4}\gamma_k(2h-1) + \frac{1}{2}\gamma_k(2h) + \frac{1}{4}\gamma_k(2h+1) \quad \mathrm{else}
\end{align}

The quantity $\hat{X}$ is asymptotic uncorrelated by assumption, $\hat{X}_k$ is also asymptotic uncorrelated. Let's turn our attention to the variance of the sample mean $V(\overline{X})$. 

\subsection*{Blocking Transformations, getting there}
We have
\begin{align}
V(\overline{X}_k) = \frac{\sigma_k^2}{n_k} + \underbrace{\frac{2}{n_k} \sum_{h=1}^{n_k-1}\left( 1 - \frac{h}{n_k} \right)\gamma_k(h)}_{\equiv e_k} = \frac{\sigma^2_k}{n_k} + e_k \quad \text{if} \quad \gamma_k(0) = \sigma_k^2. 
\end{align}
The term $e_k$ is called the \textbf{truncation error}: 
\begin{equation}
e_k = \frac{2}{n_k} \sum_{h=1}^{n_k-1}\left( 1 - \frac{h}{n_k} \right)\gamma_k(h). 
\end{equation}
We can show that $V(\overline{X}_i) = V(\overline{X}_j)$ for all $0 \leq i \leq d-1$ and $0 \leq j \leq d-1$. 

\subsection*{Blocking Transformations, final expressions}

We can then wrap up
\begin{align}
n_{j+1} \overline{X}_{j+1}  &= \sum_{i=1}^{n_{j+1}} (\hat{X}_{j+1})_i =  \frac{1}{2}\sum_{i=1}^{n_{j}/2} (\hat{X}_{j})_{2i-1} + (\hat{X}_{j})_{2i} \nonumber \\
&= \frac{1}{2}\left[ (\hat{X}_j)_1 + (\hat{X}_j)_2 + \cdots + (\hat{X}_j)_{n_j} \right] = \underbrace{\frac{n_j}{2}}_{=n_{j+1}} \overline{X}_j = n_{j+1}\overline{X}_j. 
\end{align}
By repeated use of this equation we get $V(\overline{X}_i) = V(\overline{X}_0) = V(\overline{X})$ for all $0 \leq i \leq d-1$. This has the consequence that
\begin{align}
V(\overline{X}) = \frac{\sigma_k^2}{n_k} + e_k \qquad \text{for all} \qquad 0 \leq k \leq d-1. \label{eq:convergence}
\end{align}

Flyvbjerg and Petersen demonstrated that the sequence
$\{e_k\}_{k=0}^{d-1}$ is decreasing, and conjecture that the term
$e_k$ can be made as small as we would like by making $k$ (and hence
$d$) sufficiently large. The sequence is decreasing (Master of Science thesis by Marius Jonsson, UiO 2018).
It means we can apply blocking transformations until
$e_k$ is sufficiently small, and then estimate $V(\overline{X})$ by
$\widehat{\sigma}^2_k/n_k$. 

For an elegant solution and proof of the blocking method, see the recent article of \href{{https://journals.aps.org/pre/abstract/10.1103/PhysRevE.98.043304}}{Marius Jonsson (former MSc student of the Computational Physics group)}.

 \clearemptydoublepage
%%
%% Automatically generated file from DocOnce source
%% (https://github.com/doconce/doconce/)
%% doconce format latex vectorization.do.txt --minted_latex_style=trac --latex_admon=paragraph --no_mako
%%


%-------------------- begin preamble ----------------------

\documentclass[%
oneside,                 % oneside: electronic viewing, twoside: printing
final,                   % draft: marks overfull hboxes, figures with paths
10pt]{article}

\listfiles               %  print all files needed to compile this document

\usepackage{relsize,makeidx,color,setspace,amsmath,amsfonts,amssymb}
\usepackage[table]{xcolor}
\usepackage{bm,ltablex,microtype}

\usepackage[pdftex]{graphicx}

\usepackage{fancyvrb} % packages needed for verbatim environments
\usepackage{minted}
\usemintedstyle{default}

\usepackage[T1]{fontenc}
%\usepackage[latin1]{inputenc}
\usepackage{ucs}
\usepackage[utf8x]{inputenc}

\usepackage{lmodern}         % Latin Modern fonts derived from Computer Modern

% Hyperlinks in PDF:
\definecolor{linkcolor}{rgb}{0,0,0.4}
\usepackage{hyperref}
\hypersetup{
    breaklinks=true,
    colorlinks=true,
    linkcolor=linkcolor,
    urlcolor=linkcolor,
    citecolor=black,
    filecolor=black,
    %filecolor=blue,
    pdfmenubar=true,
    pdftoolbar=true,
    bookmarksdepth=3   % Uncomment (and tweak) for PDF bookmarks with more levels than the TOC
    }
%\hyperbaseurl{}   % hyperlinks are relative to this root

\setcounter{tocdepth}{2}  % levels in table of contents

\usepackage[framemethod=TikZ]{mdframed}

% --- begin definitions of admonition environments ---

% --- end of definitions of admonition environments ---

% prevent orhpans and widows
\clubpenalty = 10000
\widowpenalty = 10000

% --- end of standard preamble for documents ---


% insert custom LaTeX commands...

\raggedbottom
\makeindex
\usepackage[totoc]{idxlayout}   % for index in the toc
\usepackage[nottoc]{tocbibind}  % for references/bibliography in the toc

%-------------------- end preamble ----------------------

\begin{document}

% matching end for #ifdef PREAMBLE

\newcommand{\exercisesection}[1]{\subsection*{#1}}


% ------------------- main content ----------------------

\section*{Optimization and Vectorization}

\subsection*{Optimization and profiling}

% --- begin paragraph admon ---
\paragraph{}

Till now we have not paid much attention to speed and possible optimization possibilities
inherent in the various compilers. We have compiled and linked as



\begin{minted}[fontsize=\fontsize{9pt}{9pt},linenos=false,mathescape,baselinestretch=1.0,fontfamily=tt,xleftmargin=7mm]{c++}
c++  -c  mycode.cpp
c++  -o  mycode.exe  mycode.o

\end{minted}

For Fortran replace with for example \textbf{gfortran} or \textbf{ifort}.
This is what we call a flat compiler option and should be used when we develop the code.
It produces normally a very large and slow code when translated to machine instructions.
We use this option for debugging and for establishing the correct program output because
every operation is done precisely as the user specified it.

It is instructive to look up the compiler manual for further instructions by writing


\begin{minted}[fontsize=\fontsize{9pt}{9pt},linenos=false,mathescape,baselinestretch=1.0,fontfamily=tt,xleftmargin=7mm]{c++}
man c++

\end{minted}
% --- end paragraph admon ---



\subsection*{More on optimization}

% --- begin paragraph admon ---
\paragraph{}
We have additional compiler options for optimization. These may include procedure inlining where 
performance may be improved, moving constants inside loops outside the loop, 
identify potential parallelism, include automatic vectorization or replace a division with a reciprocal
and a multiplication if this speeds up the code.



\begin{minted}[fontsize=\fontsize{9pt}{9pt},linenos=false,mathescape,baselinestretch=1.0,fontfamily=tt,xleftmargin=7mm]{c++}
c++  -O3 -c  mycode.cpp
c++  -O3 -o  mycode.exe  mycode.o

\end{minted}

This (other options are -O2 or -Ofast) is the recommended option.
% --- end paragraph admon ---



\subsection*{Optimization and profiling}

% --- begin paragraph admon ---
\paragraph{}
It is also useful to profile your program under the development stage.
You would then compile with 



\begin{minted}[fontsize=\fontsize{9pt}{9pt},linenos=false,mathescape,baselinestretch=1.0,fontfamily=tt,xleftmargin=7mm]{c++}
c++  -pg -O3 -c  mycode.cpp
c++  -pg -O3 -o  mycode.exe  mycode.o

\end{minted}

After you have run the code you can obtain the profiling information via


\begin{minted}[fontsize=\fontsize{9pt}{9pt},linenos=false,mathescape,baselinestretch=1.0,fontfamily=tt,xleftmargin=7mm]{c++}
gprof mycode.exe >  ProfileOutput

\end{minted}

When you have profiled properly your code, you must take out this option as it 
slows down performance.
For memory tests use \href{{http://www.valgrind.org}}{valgrind}. An excellent environment for all these aspects, and much  more, is  Qt creator.
% --- end paragraph admon ---



\subsection*{Optimization and debugging}

% --- begin paragraph admon ---
\paragraph{}
Adding debugging options is a very useful alternative under the development stage of a program.
You would then compile with 



\begin{minted}[fontsize=\fontsize{9pt}{9pt},linenos=false,mathescape,baselinestretch=1.0,fontfamily=tt,xleftmargin=7mm]{c++}
c++  -g -O0 -c  mycode.cpp
c++  -g -O0 -o  mycode.exe  mycode.o

\end{minted}

This option generates debugging information allowing you to trace for example if an array is properly allocated. Some compilers work best with the no optimization option \textbf{-O0}.
% --- end paragraph admon ---



% --- begin paragraph admon ---
\paragraph{Other optimization flags.}
Depending on the compiler, one can add flags which generate code that catches integer overflow errors. 
The flag \textbf{-ftrapv} does this for the CLANG compiler on OS X operating systems.
% --- end paragraph admon ---



\subsection*{Other hints}

% --- begin paragraph admon ---
\paragraph{}
In general, irrespective of compiler options, it is useful to
\begin{itemize}
\item avoid if tests or call to functions inside loops, if possible. 

\item avoid multiplication with constants inside loops if possible
\end{itemize}

\noindent
Here is an example of a part of a program where specific operations lead to a slower code






\begin{minted}[fontsize=\fontsize{9pt}{9pt},linenos=false,mathescape,baselinestretch=1.0,fontfamily=tt,xleftmargin=7mm]{c++}
k = n-1;
for (i = 0; i < n; i++){
    a[i] = b[i] +c*d;
    e = g[k];
}

\end{minted}

A better code is






\begin{minted}[fontsize=\fontsize{9pt}{9pt},linenos=false,mathescape,baselinestretch=1.0,fontfamily=tt,xleftmargin=7mm]{c++}
temp = c*d;
for (i = 0; i < n; i++){
    a[i] = b[i] + temp;
}
e = g[n-1];

\end{minted}

Here we avoid a repeated multiplication inside a loop. 
Most compilers, depending on compiler flags, identify and optimize such bottlenecks on their own, without requiring any particular action by the programmer. However, it is always useful to single out and avoid code examples like the first one discussed here.
% --- end paragraph admon ---



\subsection*{Vectorization and the basic idea behind parallel computing}

% --- begin paragraph admon ---
\paragraph{}
Present CPUs are highly parallel processors with varying levels of parallelism. The typical situation can be described via the following three statements.
\begin{itemize}
\item Pursuit of shorter computation time and larger simulation size gives rise to parallel computing.

\item Multiple processors are involved to solve a global problem.

\item The essence is to divide the entire computation evenly among collaborative processors.  Divide and conquer.
\end{itemize}

\noindent
Before we proceed with a more detailed discussion of topics like vectorization and parallelization, we need to remind ourselves about some basic features of different hardware models.
% --- end paragraph admon ---



\subsection*{A rough classification of hardware models}

% --- begin paragraph admon ---
\paragraph{}

\begin{itemize}
\item Conventional single-processor computers are named SISD (single-instruction-single-data) machines.

\item SIMD (single-instruction-multiple-data) machines incorporate the idea of parallel processing, using a large number of processing units to execute the same instruction on different data.

\item Modern parallel computers are so-called MIMD (multiple-instruction-multiple-data) machines and can execute different instruction streams in parallel on different data.
\end{itemize}

\noindent
% --- end paragraph admon ---



\subsection*{Shared memory and distributed memory}

% --- begin paragraph admon ---
\paragraph{}
One way of categorizing modern parallel computers is to look at the memory configuration.
\begin{itemize}
\item In shared memory systems the CPUs share the same address space. Any CPU can access any data in the global memory.

\item In distributed memory systems each CPU has its own memory.
\end{itemize}

\noindent
The CPUs are connected by some network and may exchange messages.
% --- end paragraph admon ---



\subsection*{Different parallel programming paradigms}

% --- begin paragraph admon ---
\paragraph{}

\begin{itemize}
\item \textbf{Task parallelism}:  the work of a global problem can be divided into a number of independent tasks, which rarely need to synchronize.  Monte Carlo simulations represent a typical situation. Integration is another. However this paradigm is of limited use.

\item \textbf{Data parallelism}:  use of multiple threads (e.g.~one or more threads per processor) to dissect loops over arrays etc.  Communication and synchronization between processors are often hidden, thus easy to program. However, the user surrenders much control to a specialized compiler. Examples of data parallelism are compiler-based parallelization and OpenMP directives. 
\end{itemize}

\noindent
% --- end paragraph admon ---



\subsection*{Different parallel programming paradigms}

% --- begin paragraph admon ---
\paragraph{}

\begin{itemize}
\item \textbf{Message passing}:  all involved processors have an independent memory address space. The user is responsible for  partitioning the data/work of a global problem and distributing the  subproblems to the processors. Collaboration between processors is achieved by explicit message passing, which is used for data transfer plus synchronization.

\item This paradigm is the most general one where the user has full control. Better parallel efficiency is usually achieved by explicit message passing. However, message-passing programming is more difficult.
\end{itemize}

\noindent
% --- end paragraph admon ---



\subsection*{What is vectorization?}
Vectorization is a special
case of \textbf{Single Instructions Multiple Data} (SIMD) to denote a single
instruction stream capable of operating on multiple data elements in
parallel. 
We can think of vectorization as the unrolling of loops accompanied with SIMD instructions.

Vectorization is the process of converting an algorithm that performs scalar operations
(typically one operation at the time) to vector operations where a single operation can refer to many simultaneous operations.
Consider the following example




\begin{minted}[fontsize=\fontsize{9pt}{9pt},linenos=false,mathescape,baselinestretch=1.0,fontfamily=tt,xleftmargin=7mm]{c++}
for (i = 0; i < n; i++){
    a[i] = b[i] + c[i];
}

\end{minted}

If the code is not vectorized, the compiler will simply start with the first element and 
then perform subsequent additions operating on one address in memory at the time. 

\subsection*{Number of elements that can acted upon}
A SIMD instruction can operate  on multiple data elements in one single instruction.
It uses the so-called 128-bit SIMD floating-point register. 
In this sense, vectorization adds some form of parallelism since one instruction is applied  
to many parts of say a vector.

The number of elements which can be operated on in parallel
range from four single-precision floating point data elements in so-called 
Streaming SIMD Extensions and two double-precision floating-point data
elements in Streaming SIMD Extensions 2 to sixteen byte operations in
a 128-bit register in Streaming SIMD Extensions 2. Thus, vector-length
ranges from 2 to 16, depending on the instruction extensions used and
on the data type. 

IN summary, our instructions  operate on 128 bit (16 byte) operands
\begin{itemize}
\item 4 floats or ints

\item 2 doubles

\item Data paths 128 bits vide for vector unit
\end{itemize}

\noindent
\subsection*{Number of elements that can acted upon, examples}
We start with the simple scalar operations given by




\begin{minted}[fontsize=\fontsize{9pt}{9pt},linenos=false,mathescape,baselinestretch=1.0,fontfamily=tt,xleftmargin=7mm]{c++}
for (i = 0; i < n; i++){
    a[i] = b[i] + c[i];
}

\end{minted}

If the code is not vectorized  and we have a 128-bit register to store a 32 bits floating point number,
it means that we have $3\times 32$ bits that are not used. For the first element we have


\begin{quote}
\begin{tabular}{cccc}
\hline
\multicolumn{1}{c}{ 0 } & \multicolumn{1}{c}{ 1 } & \multicolumn{1}{c}{ 2 } & \multicolumn{1}{c}{ 3 } \\
\hline
a[0]= & not used & not used & not used \\
\hline
b[0]+ & not used & not used & not used \\
\hline
c[0]  & not used & not used & not used \\
\hline
\end{tabular}
\end{quote}

\noindent
We have thus unused space in our SIMD registers. These registers could hold three additional integers.

\subsection*{Operation counts for scalar operation}
The code




\begin{minted}[fontsize=\fontsize{9pt}{9pt},linenos=false,mathescape,baselinestretch=1.0,fontfamily=tt,xleftmargin=7mm]{c++}
for (i = 0; i < n; i++){
    a[i] = b[i] + c[i];
}

\end{minted}

has for $n$ repeats
\begin{enumerate}
\item one load for $c[i]$ in address 1

\item one load for $b[i]$ in address 2

\item add $c[i]$ and $b[i]$ to give $a[i]$

\item store $a[i]$ in address 2
\end{enumerate}

\noindent
\subsection*{Number of elements that can acted upon, examples}
If we vectorize the code, we can perform, with a 128-bit register four simultaneous operations, that is
we have







\begin{minted}[fontsize=\fontsize{9pt}{9pt},linenos=false,mathescape,baselinestretch=1.0,fontfamily=tt,xleftmargin=7mm]{c++}
for (i = 0; i < n; i+=4){
    a[i] = b[i] + c[i];
    a[i+1] = b[i+1] + c[i+1];
    a[i+2] = b[i+2] + c[i+2];
    a[i+3] = b[i+3] + c[i+3];
}

\end{minted}

displayed here as


\begin{quote}
\begin{tabular}{cccc}
\hline
\multicolumn{1}{c}{ 0 } & \multicolumn{1}{c}{ 1 } & \multicolumn{1}{c}{ 2 } & \multicolumn{1}{c}{ 3 } \\
\hline
a[0]= & a[1]= & a[2]= & a[3]= \\
\hline
b[0]+ & b[1]+ & b[2]+ & b[3]+ \\
\hline
c[0]  & c[1]  & c[2]  & c[3]  \\
\hline
\end{tabular}
\end{quote}

\noindent
Four additions are now done in a single step.

\subsection*{Number of operations when vectorized}
For $n/4$ repeats assuming floats or integers
\begin{enumerate}
\item one vector load for $c[i]$ in address 1

\item one load for $b[i]$ in address 2

\item add $c[i]$ and $b[i]$ to give $a[i]$

\item store $a[i]$ in address 2
\end{enumerate}

\noindent
\subsection*{\href{{https://github.com/CompPhysics/ComputationalPhysicsMSU/blob/master/doc/Programs/LecturePrograms/programs/Classes/cpp/program7.cpp}}{A simple test case with and without vectorization}}
We implement these operations in a simple c++ program that computes at the end the norm of a vector.




















































\begin{Verbatim}[numbers=none,fontsize=\fontsize{9pt}{9pt},baselinestretch=0.95]
#include <cstdlib>
#include <iostream>
#include <cmath>
#include <iomanip>
#include "time.h"

using namespace std; // note use of namespace
int main (int argc, char* argv[])
{
  // read in dimension of square matrix
  int n = atoi(argv[1]);
  double s = 1.0/sqrt( (double) n);
  double *a, *b, *c;
  // Start timing
  clock_t start, finish;
  start = clock();
// Allocate space for the vectors to be used
    a = new double [n]; b = new double [n]; c = new double [n];
  // Define parallel region
  // Set up values for vectors  a and b
  for (int i = 0; i < n; i++){
    double angle = 2.0*M_PI*i/ (( double ) n);
    a[i] = s*(sin(angle) + cos(angle));
    b[i] =  s*sin(2.0*angle);
    c[i] = 0.0;
  }
  // Then perform the vector addition
  for (int i = 0; i < n; i++){
    c[i] += a[i]+b[i];
  }
  // Compute now the norm-2
  double Norm2 = 0.0;
  for (int i = 0; i < n; i++){
    Norm2  += c[i]*c[i];
  }
  finish = clock();
  double timeused = (double) (finish - start)/(CLOCKS_PER_SEC );
  cout << setiosflags(ios::showpoint | ios::uppercase);
  cout << setprecision(10) << setw(20) << "Time used  for norm computation=" << timeused  << endl;
  cout << "  Norm-2  = " << Norm2 << endl;
  // Free up space
  delete[] a;
  delete[] b;
  delete[] c;
  return 0;
}





\end{Verbatim}


\subsection*{Compiling with and without vectorization}
We can compile and link without vectorization using the clang c++ compiler


\begin{minted}[fontsize=\fontsize{9pt}{9pt},linenos=false,mathescape,baselinestretch=1.0,fontfamily=tt,xleftmargin=7mm]{c++}
clang -o novec.x vecexample.cpp

\end{minted}

and with vectorization (and additional optimizations)


\begin{minted}[fontsize=\fontsize{9pt}{9pt},linenos=false,mathescape,baselinestretch=1.0,fontfamily=tt,xleftmargin=7mm]{c++}
clang++ -O3 -Rpass=loop-vectorize -o  vec.x vecexample.cpp 

\end{minted}

The speedup depends on the size of the vectors. In the example here we have run with $10^7$ elements.
The example here was run on an IMac17.1 with OSX El Capitan (10.11.4) as operating system and an Intel i5 3.3 GHz CPU.  





\begin{minted}[fontsize=\fontsize{9pt}{9pt},linenos=false,mathescape,baselinestretch=1.0,fontfamily=tt,xleftmargin=7mm]{c++}
Compphys:~ hjensen$ ./vec.x 10000000
Time used  for norm computation=0.04720500000
Compphys:~ hjensen$ ./novec.x 10000000
Time used  for norm computation=0.03311700000

\end{minted}

This particular C++ compiler speeds up the above loop operations with a factor of 1.5 
Performing the same operations for $10^9$ elements results in a smaller speedup since reading from main memory is required. The non-vectorized code is seemingly faster. 





\begin{minted}[fontsize=\fontsize{9pt}{9pt},linenos=false,mathescape,baselinestretch=1.0,fontfamily=tt,xleftmargin=7mm]{c++}
Compphys:~ hjensen$ ./vec.x 1000000000
Time used  for norm computation=58.41391100
Compphys:~ hjensen$ ./novec.x 1000000000
Time used  for norm computation=46.51295300

\end{minted}

We will discuss these issues further in the next slides.  

\subsection*{Compiling with and without vectorization using clang}
We can compile and link without vectorization with clang compiler


\begin{minted}[fontsize=\fontsize{9pt}{9pt},linenos=false,mathescape,baselinestretch=1.0,fontfamily=tt,xleftmargin=7mm]{c++}
clang++ -o -fno-vectorize novec.x vecexample.cpp

\end{minted}

and with vectorization


\begin{minted}[fontsize=\fontsize{9pt}{9pt},linenos=false,mathescape,baselinestretch=1.0,fontfamily=tt,xleftmargin=7mm]{c++}
clang++ -O3 -Rpass=loop-vectorize -o  vec.x vecexample.cpp 

\end{minted}

We can also add vectorization analysis, see for example


\begin{minted}[fontsize=\fontsize{9pt}{9pt},linenos=false,mathescape,baselinestretch=1.0,fontfamily=tt,xleftmargin=7mm]{c++}
clang++ -O3 -Rpass-analysis=loop-vectorize -o  vec.x vecexample.cpp 

\end{minted}

or figure out if vectorization was missed


\begin{minted}[fontsize=\fontsize{9pt}{9pt},linenos=false,mathescape,baselinestretch=1.0,fontfamily=tt,xleftmargin=7mm]{c++}
clang++ -O3 -Rpass-missed=loop-vectorize -o  vec.x vecexample.cpp 

\end{minted}


\subsection*{Automatic vectorization and vectorization inhibitors, criteria}

Not all loops can be vectorized, as discussed in \href{{https://software.intel.com/en-us/articles/a-guide-to-auto-vectorization-with-intel-c-compilers}}{Intel's guide to vectorization}

An important criteria is that the loop counter $n$ is known at the entry of the loop.




\begin{minted}[fontsize=\fontsize{9pt}{9pt},linenos=false,mathescape,baselinestretch=1.0,fontfamily=tt,xleftmargin=7mm]{c++}
  for (int j = 0; j < n; j++) {
    a[j] = cos(j*1.0);
  }

\end{minted}

The variable $n$ does need to be known at compile time. However, this variable must stay the same for the entire duration of the loop. It implies that an exit statement inside the loop cannot be data dependent.

\subsection*{Automatic vectorization and vectorization inhibitors, exit criteria}

An exit statement should in general be avoided. 
If the exit statement contains data-dependent conditions, the loop cannot be vectorized. 
The following is an example of a non-vectorizable loop





\begin{minted}[fontsize=\fontsize{9pt}{9pt},linenos=false,mathescape,baselinestretch=1.0,fontfamily=tt,xleftmargin=7mm]{c++}
  for (int j = 0; j < n; j++) {
    a[j] = cos(j*1.0);
    if (a[j] < 0 ) break;
  }

\end{minted}

Avoid loop termination conditions and opt for a single entry loop variable $n$. The lower and upper bounds have to be kept fixed within the loop. 

\subsection*{Automatic vectorization and vectorization inhibitors, straight-line code}

SIMD instructions perform the same type of operations multiple times. 
A \textbf{switch} statement leads thus to a non-vectorizable loop since different statemens cannot branch.
The following code can however be vectorized since the \textbf{if} statement is implemented as a masked assignment.










\begin{minted}[fontsize=\fontsize{9pt}{9pt},linenos=false,mathescape,baselinestretch=1.0,fontfamily=tt,xleftmargin=7mm]{c++}
  for (int j = 0; j < n; j++) {
    double x  = cos(j*1.0);
    if (x > 0 ) {
       a[j] =  x*sin(j*2.0); 
    }
    else {
       a[j] = 0.0;
    }
  }

\end{minted}

These operations can be performed for all data elements but only those elements which the mask evaluates as true are stored. In general, one should avoid branches such as \textbf{switch}, \textbf{go to}, or \textbf{return} statements or \textbf{if} constructs that cannot be treated as masked assignments. 

\subsection*{Automatic vectorization and vectorization inhibitors, nested loops}

Only the innermost loop of the following example is vectorized






\begin{minted}[fontsize=\fontsize{9pt}{9pt},linenos=false,mathescape,baselinestretch=1.0,fontfamily=tt,xleftmargin=7mm]{c++}
  for (int i = 0; i < n; i++) {
      for (int j = 0; j < n; j++) {
           a[i][j] += b[i][j];
      }  
  }

\end{minted}

The exception is if an original outer loop is transformed into an inner loop as the result of compiler optimizations.

\subsection*{Automatic vectorization and vectorization inhibitors, function calls}

Calls to programmer defined functions ruin vectorization. However, calls to intrinsic functions like
$\sin{x}$, $\cos{x}$, $\exp{x}$ etc are allowed since they are normally efficiently vectorized. 
The following example is fully vectorizable




\begin{minted}[fontsize=\fontsize{9pt}{9pt},linenos=false,mathescape,baselinestretch=1.0,fontfamily=tt,xleftmargin=7mm]{c++}
  for (int i = 0; i < n; i++) {
      a[i] = log10(i)*cos(i);
  }

\end{minted}

Similarly, \textbf{inline} functions defined by the programmer, allow for vectorization since the function statements are glued into the actual place where the function is called. 

\subsection*{Automatic vectorization and vectorization inhibitors, data dependencies}

One has to keep in mind that vectorization changes the order of operations inside a loop. A so-called
read-after-write statement with an explicit flow dependency cannot be vectorized. The following code





\begin{minted}[fontsize=\fontsize{9pt}{9pt},linenos=false,mathescape,baselinestretch=1.0,fontfamily=tt,xleftmargin=7mm]{c++}
  double b = 15.;
  for (int i = 1; i < n; i++) {
      a[i] = a[i-1] + b;
  }

\end{minted}

is an example of flow dependency and results in wrong numerical results if vectorized. For a scalar operation, the value $a[i-1]$ computed during the iteration is loaded into the right-hand side and the results are fine. In vector mode however, with a vector length of four, the values $a[0]$, $a[1]$, $a[2]$ and $a[3]$ from the previous loop will be loaded into the right-hand side and produce wrong results. That is, we have





\begin{minted}[fontsize=\fontsize{9pt}{9pt},linenos=false,mathescape,baselinestretch=1.0,fontfamily=tt,xleftmargin=7mm]{c++}
   a[1] = a[0] + b;
   a[2] = a[1] + b;
   a[3] = a[2] + b;
   a[4] = a[3] + b;

\end{minted}

and if the two first iterations are  executed at the same by the SIMD instruction, the value of say $a[1]$ could be used by the second iteration before it has been calculated by the first iteration, leading thereby to wrong results.

\subsection*{Automatic vectorization and vectorization inhibitors, more data dependencies}

On the other hand,  a so-called 
write-after-read statement can be vectorized. The following code





\begin{minted}[fontsize=\fontsize{9pt}{9pt},linenos=false,mathescape,baselinestretch=1.0,fontfamily=tt,xleftmargin=7mm]{c++}
  double b = 15.;
  for (int i = 1; i < n; i++) {
      a[i-1] = a[i] + b;
  }

\end{minted}

is an example of flow dependency that can be vectorized since no iteration with a higher value of $i$
can complete before an iteration with a lower value of $i$. However, such code leads to problems with parallelization.

\subsection*{Automatic vectorization and vectorization inhibitors, memory stride}

For C++ programmers  it is also worth keeping in mind that an array notation is preferred to the more compact use of pointers to access array elements. The compiler can often not tell if it is safe to vectorize the code. 

When dealing with arrays, you should also avoid memory stride, since this slows down considerably vectorization. When you access array element, write for example the inner loop to vectorize using unit stride, that is, access successively the next array element in memory, as shown here






\begin{minted}[fontsize=\fontsize{9pt}{9pt},linenos=false,mathescape,baselinestretch=1.0,fontfamily=tt,xleftmargin=7mm]{c++}
  for (int i = 0; i < n; i++) {
      for (int j = 0; j < n; j++) {
           a[i][j] += b[i][j];
      }  
  }

\end{minted}


\subsection*{Memory management}
The main memory contains the program data
\begin{enumerate}
\item Cache memory contains a copy of the main memory data

\item Cache is faster but consumes more space and power. It is normally assumed to be much faster than main memory

\item Registers contain working data only
\begin{itemize}

 \item Modern CPUs perform most or all operations only on data in register

\end{itemize}

\noindent
\item Multiple Cache memories contain a copy of the main memory data
\begin{itemize}

 \item Cache items accessed by their address in main memory

 \item L1 cache is the fastest but has the least capacity

 \item L2, L3 provide intermediate performance/size tradeoffs
\end{itemize}

\noindent
\end{enumerate}

\noindent
Loads and stores to memory can be as important as floating point operations when we measure performance.

\subsection*{Memory and communication}

\begin{enumerate}
\item Most communication in a computer is carried out in chunks, blocks of bytes of data that move together

\item In the memory hierarchy, data moves between memory and cache, and between different levels of cache, in groups called lines
\begin{itemize}

 \item Lines are typically 64-128 bytes, or 8-16 double precision words

 \item Even if you do not use the data, it is moved and occupies space in the cache
\end{itemize}

\noindent
\end{enumerate}

\noindent
Many of these  performance features are not captured in most programming languages.

\subsection*{Measuring performance}

How do we measure performance? What is wrong with this code to time a loop?








\begin{minted}[fontsize=\fontsize{9pt}{9pt},linenos=false,mathescape,baselinestretch=1.0,fontfamily=tt,xleftmargin=7mm]{text}
  clock_t start, finish;
  start = clock();
  for (int j = 0; j < i; j++) {
    a[j] = b[j]+b[j]*c[j];
  }
  finish = clock();
  double timeused = (double) (finish - start)/(CLOCKS_PER_SEC );

\end{minted}


\subsection*{Problems with measuring time}
\begin{enumerate}
\item Timers are not infinitely accurate

\item All clocks have a granularity, the minimum time that they can measure

\item The error in a time measurement, even if everything is perfect, may be the size of this granularity (sometimes called a clock tick)

\item Always know what your clock granularity is

\item Ensure that your measurement is for a long enough duration (say 100 times the \textbf{tick})
\end{enumerate}

\noindent
\subsection*{Problems with cold start}

What happens when the code is executed? The assumption is that the code is ready to
execute. But
\begin{enumerate}
\item Code may still be on disk, and not even read into memory.

\item Data may be in slow memory rather than fast (which may be wrong or right for what you are measuring)

\item Multiple tests often necessary to ensure that cold start effects are not present

\item Special effort often required to ensure data in the intended part of the memory hierarchy.
\end{enumerate}

\noindent
\subsection*{Problems with smart compilers}

\begin{enumerate}
\item If the result of the computation is not used, the compiler may eliminate the code

\item Performance will look impossibly fantastic

\item Even worse, eliminate some of the code so the performance looks plausible

\item Ensure that the results are (or may be) used.
\end{enumerate}

\noindent
\subsection*{Problems with interference}
\begin{enumerate}
\item Other activities are sharing your processor
\begin{itemize}

  \item Operating system, system demons, other users

  \item Some parts of the hardware do not always perform with exactly the same performance

\end{itemize}

\noindent
\item Make multiple tests and report

\item Easy choices include
\begin{itemize}

  \item Average tests represent what users might observe over time
\end{itemize}

\noindent
\end{enumerate}

\noindent
\subsection*{Problems with measuring performance}
\begin{enumerate}
\item Accurate, reproducible performance measurement is hard

\item Think carefully about your experiment:

\item What is it, precisely, that you want to measure?

\item How representative is your test to the situation that you are trying to measure?
\end{enumerate}

\noindent
\subsection*{Thomas algorithm for tridiagonal linear algebra equations}

% --- begin paragraph admon ---
\paragraph{}
\[
\left( \begin{array}{ccccc}
        b_0 & c_0 &        &         &         \\
	a_0 &  b_1 &  c_1    &         &         \\
	   &    & \ddots  &         &         \\
	      &	    & a_{m-3} & b_{m-2} & c_{m-2} \\
	         &    &         & a_{m-2} & b_{m-1}
   \end{array} \right)
\left( \begin{array}{c}
       x_0     \\
       x_1     \\
       \vdots  \\
       x_{m-2} \\
       x_{m-1}
   \end{array} \right)=\left( \begin{array}{c}
       f_0     \\
       f_1     \\
       \vdots  \\
       f_{m-2} \\
       f_{m-1} \\
   \end{array} \right)
\]
% --- end paragraph admon ---



\subsection*{Thomas algorithm, forward substitution}

% --- begin paragraph admon ---
\paragraph{}
The first step is to multiply the first row by $a_0/b_0$ and subtract it from the second row.  This is known as the forward substitution step. We obtain then
\[
	a_i = 0,
\]

\[                                 
	b_i = b_i - \frac{a_{i-1}}{b_{i-1}}c_{i-1},
\]
and
\[
	f_i = f_i - \frac{a_{i-1}}{b_{i-1}}f_{i-1}.
\]
At this point the simplified equation, with only an upper triangular matrix takes the form
\[
\left( \begin{array}{ccccc}
    b_0 & c_0 &        &         &         \\
       & b_1 &  c_1    &         &         \\
          &    & \ddots &         &         \\
	     &     &        & b_{m-2} & c_{m-2} \\
	        &    &        &         & b_{m-1}
   \end{array} \right)\left( \begin{array}{c}
       x_0     \\
       x_1     \\
       \vdots  \\
       x_{m-2} \\
       x_{m-1}
   \end{array} \right)=\left( \begin{array}{c}
       f_0     \\
       f_1     \\
       \vdots  \\
       f_{m-2} \\
       f_{m-1} \\
   \end{array} \right)
\]
% --- end paragraph admon ---



\subsection*{Thomas algorithm, backward substitution}

% --- begin paragraph admon ---
\paragraph{}
The next step is  the backward substitution step.  The last row is multiplied by $c_{N-3}/b_{N-2}$ and subtracted from the second to last row, thus eliminating $c_{N-3}$ from the last row.  The general backward substitution procedure is 
\[
	c_i = 0, 
\]
and 
\[
	f_{i-1} = f_{i-1} - \frac{c_{i-1}}{b_i}f_i
\]
All that ramains to be computed is the solution, which is the very straight forward process of
\[
x_i = \frac{f_i}{b_i}
\]
% --- end paragraph admon ---



\subsection*{Thomas algorithm and counting of operations (floating point and memory)}

% --- begin paragraph admon ---
\paragraph{}


\begin{quote}
\begin{tabular}{cc}
\hline
\multicolumn{1}{c}{ Operation } & \multicolumn{1}{c}{ Floating Point } \\
\hline
Memory Reads    & $14(N-2)$      \\
Memory Writes   & $4(N-2)$       \\
Subtractions    & $3(N-2)$       \\
Multiplications & $3(N-2)$       \\
Divisions       & $4(N-2)$       \\
\hline
\end{tabular}
\end{quote}

\noindent
% --- end paragraph admon ---




% --- begin paragraph admon ---
\paragraph{}













\begin{minted}[fontsize=\fontsize{9pt}{9pt},linenos=false,mathescape,baselinestretch=1.0,fontfamily=tt,xleftmargin=7mm]{c++}
// Forward substitution    
// Note that we can simplify by precalculating a[i-1]/b[i-1]
  for (int i=1; i < n; i++) {
     b[i] = b[i] - (a[i-1]*c[i-1])/b[i-1];
     f[i] = g[i] - (a[i-1]*f[i-1])/b[i-1];
  }
  x[n-1] = f[n-1] / b[n-1];
  // Backwards substitution                                                           
  for (int i = n-2; i >= 0; i--) {
     f[i] = f[i] - c[i]*f[i+1]/b[i+1];
     x[i] = f[i]/b[i];
  }

\end{minted}
% --- end paragraph admon ---



\subsection*{\href{{https://github.com/CompPhysics/ComputationalPhysicsMSU/blob/master/doc/Programs/LecturePrograms/programs/Classes/cpp/program8.cpp}}{Example: Transpose of a matrix}}


















































\begin{Verbatim}[numbers=none,fontsize=\fontsize{9pt}{9pt},baselinestretch=0.95]
#include <cstdlib>
#include <iostream>
#include <cmath>
#include <iomanip>
#include "time.h"

using namespace std; // note use of namespace
int main (int argc, char* argv[])
{
  // read in dimension of square matrix
  int n = atoi(argv[1]);
  double **A, **B;
  // Allocate space for the two matrices
  A = new double*[n]; B = new double*[n];
  for (int i = 0; i < n; i++){
    A[i] = new double[n];
    B[i] = new double[n];
  }
  // Set up values for matrix A
  for (int i = 0; i < n; i++){
    for (int j = 0; j < n; j++) {
      A[i][j] =  cos(i*1.0)*sin(j*3.0);
    }
  }
  clock_t start, finish;
  start = clock();
  // Then compute the transpose
  for (int i = 0; i < n; i++){
    for (int j = 0; j < n; j++) {
      B[i][j]= A[j][i];
    }
  }

  finish = clock();
  double timeused = (double) (finish - start)/(CLOCKS_PER_SEC );
  cout << setiosflags(ios::showpoint | ios::uppercase);
  cout << setprecision(10) << setw(20) << "Time used  for setting up transpose of matrix=" << timeused  << endl;

  // Free up space
  for (int i = 0; i < n; i++){
    delete[] A[i];
    delete[] B[i];
  }
  delete[] A;
  delete[] B;
  return 0;
}


\end{Verbatim}


\subsection*{\href{{https://github.com/CompPhysics/ComputationalPhysicsMSU/blob/master/doc/Programs/LecturePrograms/programs/Classes/cpp/program9.cpp}}{Matrix-matrix multiplication}}
This the matrix-matrix multiplication code with plain c++ memory allocation. It computes at the end the Frobenius norm.



































































\begin{minted}[fontsize=\fontsize{9pt}{9pt},linenos=false,mathescape,baselinestretch=1.0,fontfamily=tt,xleftmargin=7mm]{text}
#include <cstdlib>
#include <iostream>
#include <cmath>
#include <iomanip>
#include "time.h"

using namespace std; // note use of namespace
int main (int argc, char* argv[])
{
  // read in dimension of square matrix
  int n = atoi(argv[1]);
  double s = 1.0/sqrt( (double) n);
  double **A, **B, **C;
  // Start timing
  clock_t start, finish;
  start = clock();
  // Allocate space for the two matrices
  A = new double*[n]; B = new double*[n]; C = new double*[n];
  for (int i = 0; i < n; i++){
    A[i] = new double[n];
    B[i] = new double[n];
    C[i] = new double[n];
  }
  // Set up values for matrix A and B and zero matrix C
  for (int i = 0; i < n; i++){
    for (int j = 0; j < n; j++) {
      double angle = 2.0*M_PI*i*j/ (( double ) n);
      A[i][j] = s * ( sin ( angle ) + cos ( angle ) );
      B[j][i] =  A[i][j];
    }
  }
  // Then perform the matrix-matrix multiplication
  for (int i = 0; i < n; i++){
    for (int j = 0; j < n; j++) {
      double sum = 0.0;
       for (int k = 0; k < n; k++) {
           sum += B[i][k]*A[k][j];
       }
       C[i][j] = sum;
    }
  }
  // Compute now the Frobenius norm
  double Fsum = 0.0;
  for (int i = 0; i < n; i++){
    for (int j = 0; j < n; j++) {
      Fsum += C[i][j]*C[i][j];
    }
  }
  Fsum = sqrt(Fsum);
  finish = clock();
  double timeused = (double) (finish - start)/(CLOCKS_PER_SEC );
  cout << setiosflags(ios::showpoint | ios::uppercase);
  cout << setprecision(10) << setw(20) << "Time used  for matrix-matrix multiplication=" << timeused  << endl;
  cout << "  Frobenius norm  = " << Fsum << endl;
  // Free up space
  for (int i = 0; i < n; i++){
    delete[] A[i];
    delete[] B[i];
    delete[] C[i];
  }
  delete[] A;
  delete[] B;
  delete[] C;
  return 0;
}

\end{minted}


\subsection*{How do we define speedup? Simplest form}

% --- begin paragraph admon ---
\paragraph{}
\begin{itemize}
\item Speedup measures the ratio of performance between two objects

\item Versions of same code, with different number of processors

\item Serial and vector versions

\item Try different programing languages, c++ and Fortran

\item Two algorithms computing the \textbf{same} result 
\end{itemize}

\noindent
% --- end paragraph admon ---



\subsection*{How do we define speedup? Correct baseline}

% --- begin paragraph admon ---
\paragraph{}
The key is choosing the correct baseline for comparison
\begin{itemize}
\item For our serial vs.~vectorization examples, using compiler-provided vectorization, the baseline is simple; the same code, with vectorization turned off
\begin{itemize}

 \item For parallel applications, this is much harder:
\begin{itemize}

  \item Choice of algorithm, decomposition, performance of baseline case etc.
\end{itemize}

\noindent
\end{itemize}

\noindent
\end{itemize}

\noindent
% --- end paragraph admon ---



\subsection*{Parallel  speedup}

% --- begin paragraph admon ---
\paragraph{}
For parallel applications, speedup  is typically defined as
\begin{itemize}
\item Speedup $=T_1/T_p$
\end{itemize}

\noindent
Here  $T_1$ is the time on one processor and $T_p$ is the time using $p$ processors.
\begin{itemize}
 \item Can the speedup become larger than  $p$? That means using $p$ processors is more than $p$ times faster than using one processor.
\end{itemize}

\noindent
% --- end paragraph admon ---



\subsection*{Speedup and memory}

% --- begin paragraph admon ---
\paragraph{}
The speedup on $p$ processors can
be greater than $p$ if memory usage is optimal!
Consider the case of a memorybound computation with $M$ words of memory
\begin{itemize}
 \item If $M/p$ fits into cache while $M$ does not, the time to access memory will be different in the two cases:

 \item $T_1$ uses the main memory bandwidth

 \item $T_p$ uses the appropriate cache bandwidth 
\end{itemize}

\noindent
% --- end paragraph admon ---



\subsection*{Upper bounds on speedup}

% --- begin paragraph admon ---
\paragraph{}
Assume that almost all parts of a code are perfectly
parallelizable (fraction $f$). The remainder,
fraction $(1-f)$ cannot be parallelized at all.

That is, there is work that takes time $W$ on one process; a fraction $f$ of that work will take
time $Wf/p$ on $p$ processors. 
\begin{itemize}
\item What is the maximum possible speedup as a function of $f$? 
\end{itemize}

\noindent
% --- end paragraph admon ---



\subsection*{Amdahl's law}

% --- begin paragraph admon ---
\paragraph{}
On one processor we have 
\[
T_1 = (1-f)W + fW = W
\]
On $p$ processors we have
\[
T_p = (1-f)W + \frac{fW}{p},
\]
resulting in a speedup of 
\[
\frac{T_1}{T_p} = \frac{W}{(1-f)W+fW/p}
\]

As $p$ goes to infinity, $fW/p$ goes to zero, and the maximum speedup is
\[
\frac{1}{1-f},
\]
meaning that if 
if $f = 0.99$ (all but $1\%$ parallelizable), the maximum speedup
is $1/(1-.99)=100$!
% --- end paragraph admon ---




% ------------------- end of main content ---------------

\end{document}


 \clearemptydoublepage
%%
%% Automatically generated file from DocOnce source
%% (https://github.com/doconce/doconce/)
%% doconce format latex parallelization.do.txt --minted_latex_style=trac --latex_admon=paragraph --no_mako
%%


%-------------------- begin preamble ----------------------

\documentclass[%
oneside,                 % oneside: electronic viewing, twoside: printing
final,                   % draft: marks overfull hboxes, figures with paths
10pt]{article}

\listfiles               %  print all files needed to compile this document

\usepackage{relsize,makeidx,color,setspace,amsmath,amsfonts,amssymb}
\usepackage[table]{xcolor}
\usepackage{bm,ltablex,microtype}

\usepackage[pdftex]{graphicx}

\usepackage{fancyvrb} % packages needed for verbatim environments
\usepackage{minted}
\usemintedstyle{default}

\usepackage[T1]{fontenc}
%\usepackage[latin1]{inputenc}
\usepackage{ucs}
\usepackage[utf8x]{inputenc}

\usepackage{lmodern}         % Latin Modern fonts derived from Computer Modern

% Hyperlinks in PDF:
\definecolor{linkcolor}{rgb}{0,0,0.4}
\usepackage{hyperref}
\hypersetup{
    breaklinks=true,
    colorlinks=true,
    linkcolor=linkcolor,
    urlcolor=linkcolor,
    citecolor=black,
    filecolor=black,
    %filecolor=blue,
    pdfmenubar=true,
    pdftoolbar=true,
    bookmarksdepth=3   % Uncomment (and tweak) for PDF bookmarks with more levels than the TOC
    }
%\hyperbaseurl{}   % hyperlinks are relative to this root

\setcounter{tocdepth}{2}  % levels in table of contents

\usepackage[framemethod=TikZ]{mdframed}

% --- begin definitions of admonition environments ---

% --- end of definitions of admonition environments ---

% prevent orhpans and widows
\clubpenalty = 10000
\widowpenalty = 10000

\newenvironment{doconceexercise}{}{}
\newcounter{doconceexercisecounter}


% ------ header in subexercises ------
%\newcommand{\subex}[1]{\paragraph{#1}}
%\newcommand{\subex}[1]{\par\vspace{1.7mm}\noindent{\bf #1}\ \ }
\makeatletter
% 1.5ex is the spacing above the header, 0.5em the spacing after subex title
\newcommand\subex{\@startsection*{paragraph}{4}{\z@}%
                  {1.5ex\@plus1ex \@minus.2ex}%
                  {-0.5em}%
                  {\normalfont\normalsize\bfseries}}
\makeatother


% --- end of standard preamble for documents ---


% insert custom LaTeX commands...

\raggedbottom
\makeindex
\usepackage[totoc]{idxlayout}   % for index in the toc
\usepackage[nottoc]{tocbibind}  % for references/bibliography in the toc

%-------------------- end preamble ----------------------

\begin{document}

% matching end for #ifdef PREAMBLE

\newcommand{\exercisesection}[1]{\subsection*{#1}}


% ------------------- main content ----------------------

\section*{Parallelization with MPI and OpenMPI}

\subsection*{How much is parallelizable}

% --- begin paragraph admon ---
\paragraph{}
If any non-parallel code slips into the
application, the parallel
performance is limited. 

In many simulations, however, the fraction of non-parallelizable work
is $10^{-6}$ or less due to large arrays or objects that are perfectly parallelizable.
% --- end paragraph admon ---



\subsection*{Today's situation of parallel computing}

% --- begin paragraph admon ---
\paragraph{}

\begin{itemize}
\item Distributed memory is the dominant hardware configuration. There is a large diversity in these machines, from  MPP (massively parallel processing) systems to clusters of off-the-shelf PCs, which are very cost-effective.

\item Message-passing is a mature programming paradigm and widely accepted. It often provides an efficient match to the hardware. It is primarily used for the distributed memory systems, but can also be used on shared memory systems.

\item Modern nodes have nowadays several cores, which makes it interesting to use both shared memory (the given node) and distributed memory (several nodes with communication). This leads often to codes which use both MPI and OpenMP.
\end{itemize}

\noindent
Our lectures will focus on both MPI and OpenMP.
% --- end paragraph admon ---



\subsection*{Overhead present in parallel computing}

% --- begin paragraph admon ---
\paragraph{}

\begin{itemize}
\item \textbf{Uneven load balance}:  not all the processors can perform useful work at all time.

\item \textbf{Overhead of synchronization}

\item \textbf{Overhead of communication}

\item \textbf{Extra computation due to parallelization}
\end{itemize}

\noindent
Due to the above overhead and that certain parts of a sequential
algorithm cannot be parallelized we may not achieve an optimal parallelization.
% --- end paragraph admon ---



\subsection*{Parallelizing a sequential algorithm}

% --- begin paragraph admon ---
\paragraph{}

\begin{itemize}
\item Identify the part(s) of a sequential algorithm that can be  executed in parallel. This is the difficult part,

\item Distribute the global work and data among $P$ processors.
\end{itemize}

\noindent
% --- end paragraph admon ---



\subsection*{Strategies}

% --- begin paragraph admon ---
\paragraph{}
\begin{itemize}
\item Develop codes locally, run with some few processes and test your codes.  Do benchmarking, timing and so forth on local nodes, for example your laptop or PC. 

\item When you are convinced that your codes run correctly, you can start your production runs on available supercomputers.
\end{itemize}

\noindent
% --- end paragraph admon ---



\subsection*{How do I run MPI on a PC/Laptop? MPI}

% --- begin paragraph admon ---
\paragraph{}
To install MPI is rather easy on hardware running unix/linux as operating systems, follow simply the instructions from the \href{{https://www.open-mpi.org/}}{OpenMPI website}. See also subsequent slides.
When you have made sure you have installed MPI on your PC/laptop, 
\begin{itemize}
\item Compile with mpicxx/mpic++ or mpif90
\end{itemize}

\noindent





\begin{minted}[fontsize=\fontsize{9pt}{9pt},linenos=false,mathescape,baselinestretch=1.0,fontfamily=tt,xleftmargin=7mm]{c++}
  # Compile and link
  mpic++ -O3 -o nameofprog.x nameofprog.cpp
  #  run code with for example 8 processes using mpirun/mpiexec
  mpiexec -n 8 ./nameofprog.x

\end{minted}
% --- end paragraph admon ---



\subsection*{Can I do it on my own PC/laptop? OpenMP installation}

% --- begin paragraph admon ---
\paragraph{}
If you wish to install MPI and OpenMP 
on your laptop/PC, we recommend the following:

\begin{itemize}
\item For OpenMP, the compile option \textbf{-fopenmp} is included automatically in recent versions of the C++ compiler and Fortran compilers. For users of different Linux distributions, simply use the available C++ or Fortran compilers and add the above compiler instructions, see also code examples below.

\item For OS X users however, install \textbf{libomp}
\end{itemize}

\noindent


\begin{minted}[fontsize=\fontsize{9pt}{9pt},linenos=false,mathescape,baselinestretch=1.0,fontfamily=tt,xleftmargin=7mm]{c++}
  brew install libomp

\end{minted}

and compile and link as


\begin{minted}[fontsize=\fontsize{9pt}{9pt},linenos=false,mathescape,baselinestretch=1.0,fontfamily=tt,xleftmargin=7mm]{c++}
c++ -o <name executable> <name program.cpp>  -lomp

\end{minted}
% --- end paragraph admon ---



\subsection*{Installing MPI}

% --- begin paragraph admon ---
\paragraph{}
For linux/ubuntu users, you need to install two packages (alternatively use the synaptic package manager)



\begin{minted}[fontsize=\fontsize{9pt}{9pt},linenos=false,mathescape,baselinestretch=1.0,fontfamily=tt,xleftmargin=7mm]{c++}
  sudo apt-get install libopenmpi-dev
  sudo apt-get install openmpi-bin

\end{minted}

For OS X users, install brew (after having installed xcode and gcc, needed for the 
gfortran compiler of openmpi) and then install with brew


\begin{minted}[fontsize=\fontsize{9pt}{9pt},linenos=false,mathescape,baselinestretch=1.0,fontfamily=tt,xleftmargin=7mm]{c++}
   brew install openmpi

\end{minted}

When running an executable (code.x), run as


\begin{minted}[fontsize=\fontsize{9pt}{9pt},linenos=false,mathescape,baselinestretch=1.0,fontfamily=tt,xleftmargin=7mm]{c++}
  mpirun -n 10 ./code.x

\end{minted}

where we indicate that we want  the number of processes to be 10.
% --- end paragraph admon ---



\subsection*{Installing MPI and using Qt}

% --- begin paragraph admon ---
\paragraph{}
With openmpi installed, when using Qt, add to your .pro file the instructions \href{{http://dragly.org/2012/03/14/developing-mpi-applications-in-qt-creator/}}{here}

You may need to tell Qt where openmpi is stored.
% --- end paragraph admon ---



\subsection*{What is Message Passing Interface (MPI)?}

% --- begin paragraph admon ---
\paragraph{}

\textbf{MPI} is a library, not a language. It specifies the names, calling sequences and results of functions
or subroutines to be called from C/C++ or Fortran programs, and the classes and methods that make up the MPI C++
library. The programs that users write in Fortran, C or C++ are compiled with ordinary compilers and linked
with the MPI library.

MPI programs should be able to run
on all possible machines and run all MPI implementetations without change.

An MPI computation is a collection of processes communicating with messages.
% --- end paragraph admon ---



\subsection*{Going Parallel with MPI}

% --- begin paragraph admon ---
\paragraph{}
\textbf{Task parallelism}: the work of a global problem can be divided
into a number of independent tasks, which rarely need to synchronize. 
Monte Carlo simulations or numerical integration are examples of this.

MPI is a message-passing library where all the routines
have corresponding C/C++-binding


\begin{minted}[fontsize=\fontsize{9pt}{9pt},linenos=false,mathescape,baselinestretch=1.0,fontfamily=tt,xleftmargin=7mm]{c++}
   MPI_Command_name

\end{minted}

and Fortran-binding (routine names are in uppercase, but can also be in lower case)


\begin{Verbatim}[numbers=none,fontsize=\fontsize{9pt}{9pt},baselinestretch=0.95]
   MPI_COMMAND_NAME

\end{Verbatim}
% --- end paragraph admon ---



\subsection*{MPI is a library}

% --- begin paragraph admon ---
\paragraph{}
MPI is a library specification for the message passing interface,
proposed as a standard.

\begin{itemize}
\item independent of hardware;

\item not a language or compiler specification;

\item not a specific implementation or product.
\end{itemize}

\noindent
A message passing standard for portability and ease-of-use. 
Designed for high performance.

Insert communication and synchronization functions where necessary.
% --- end paragraph admon ---



\subsection*{Bindings to MPI routines}

% --- begin paragraph admon ---
\paragraph{}

MPI is a message-passing library where all the routines
have corresponding C/C++-binding


\begin{minted}[fontsize=\fontsize{9pt}{9pt},linenos=false,mathescape,baselinestretch=1.0,fontfamily=tt,xleftmargin=7mm]{c++}
   MPI_Command_name

\end{minted}

and Fortran-binding (routine names are in uppercase, but can also be in lower case)


\begin{Verbatim}[numbers=none,fontsize=\fontsize{9pt}{9pt},baselinestretch=0.95]
   MPI_COMMAND_NAME

\end{Verbatim}

The discussion in these slides focuses on the C++ binding.
% --- end paragraph admon ---



\subsection*{Communicator}

% --- begin paragraph admon ---
\paragraph{}
\begin{itemize}
\item A group of MPI processes with a name (context).

\item Any process is identified by its rank. The rank is only meaningful within a particular communicator.

\item By default the communicator contains all the MPI processes.
\end{itemize}

\noindent


\begin{minted}[fontsize=\fontsize{9pt}{9pt},linenos=false,mathescape,baselinestretch=1.0,fontfamily=tt,xleftmargin=7mm]{c++}
  MPI_COMM_WORLD 

\end{minted}

\begin{itemize}
\item Mechanism to identify subset of processes.

\item Promotes modular design of parallel libraries.
\end{itemize}

\noindent
% --- end paragraph admon ---



\subsection*{Some of the most  important MPI functions}

% --- begin paragraph admon ---
\paragraph{}

\begin{itemize}
\item $MPI\_Init$ - initiate an MPI computation

\item $MPI\_Finalize$ - terminate the MPI computation and clean up

\item $MPI\_Comm\_size$ - how many processes participate in a given MPI communicator?

\item $MPI\_Comm\_rank$ - which one am I? (A number between 0 and size-1.)

\item $MPI\_Send$ - send a message to a particular process within an MPI communicator

\item $MPI\_Recv$ - receive a message from a particular process within an MPI communicator

\item $MPI\_reduce$  or $MPI\_Allreduce$, send and receive messages
\end{itemize}

\noindent
% --- end paragraph admon ---



\subsection*{\href{{https://github.com/CompPhysics/ComputationalPhysics2/blob/gh-pages/doc/Programs/LecturePrograms/programs/MPI/chapter07/program2.cpp}}{The first MPI C/C++ program}}

% --- begin paragraph admon ---
\paragraph{}

Let every process write "Hello world" (oh not this program again!!) on the standard output. 















\begin{minted}[fontsize=\fontsize{9pt}{9pt},linenos=false,mathescape,baselinestretch=1.0,fontfamily=tt,xleftmargin=7mm]{c++}
using namespace std;
#include <mpi.h>
#include <iostream>
int main (int nargs, char* args[])
{
int numprocs, my_rank;
//   MPI initializations
MPI_Init (&nargs, &args);
MPI_Comm_size (MPI_COMM_WORLD, &numprocs);
MPI_Comm_rank (MPI_COMM_WORLD, &my_rank);
cout << "Hello world, I have  rank " << my_rank << " out of " 
     << numprocs << endl;
//  End MPI
MPI_Finalize ();

\end{minted}
% --- end paragraph admon ---



\subsection*{The Fortran program}

% --- begin paragraph admon ---
\paragraph{}












\begin{Verbatim}[numbers=none,fontsize=\fontsize{9pt}{9pt},baselinestretch=0.95]
PROGRAM hello
INCLUDE "mpif.h"
INTEGER:: size, my_rank, ierr

CALL  MPI_INIT(ierr)
CALL MPI_COMM_SIZE(MPI_COMM_WORLD, size, ierr)
CALL MPI_COMM_RANK(MPI_COMM_WORLD, my_rank, ierr)
WRITE(*,*)"Hello world, I've rank ",my_rank," out of ",size
CALL MPI_FINALIZE(ierr)

END PROGRAM hello

\end{Verbatim}
% --- end paragraph admon ---



\subsection*{Note 1}

% --- begin paragraph admon ---
\paragraph{}

\begin{itemize}
\item The output to screen is not ordered since all processes are trying to write  to screen simultaneously.

\item It is the operating system which opts for an ordering.  

\item If we wish to have an organized output, starting from the first process, we may rewrite our program as in the next example.
\end{itemize}

\noindent
% --- end paragraph admon ---



\subsection*{\href{{https://github.com/CompPhysics/ComputationalPhysics2/blob/gh-pages/doc/Programs/LecturePrograms/programs/MPI/chapter07/program3.cpp}}{Ordered output with MPIBarrier}}

% --- begin paragraph admon ---
\paragraph{}














\begin{minted}[fontsize=\fontsize{9pt}{9pt},linenos=false,mathescape,baselinestretch=1.0,fontfamily=tt,xleftmargin=7mm]{c++}
int main (int nargs, char* args[])
{
 int numprocs, my_rank, i;
 MPI_Init (&nargs, &args);
 MPI_Comm_size (MPI_COMM_WORLD, &numprocs);
 MPI_Comm_rank (MPI_COMM_WORLD, &my_rank);
 for (i = 0; i < numprocs; i++) {}
 MPI_Barrier (MPI_COMM_WORLD);
 if (i == my_rank) {
 cout << "Hello world, I have  rank " << my_rank << 
        " out of " << numprocs << endl;}
      MPI_Finalize ();

\end{minted}
% --- end paragraph admon ---



\subsection*{Note 2}

% --- begin paragraph admon ---
\paragraph{}
\begin{itemize}
\item Here we have used the $MPI\_Barrier$ function to ensure that that every process has completed  its set of instructions in  a particular order.

\item A barrier is a special collective operation that does not allow the processes to continue until all processes in the communicator (here $MPI\_COMM\_WORLD$) have called $MPI\_Barrier$. 

\item The barriers make sure that all processes have reached the same point in the code. Many of the collective operations like $MPI\_ALLREDUCE$ to be discussed later, have the same property; that is, no process can exit the operation until all processes have started. 
\end{itemize}

\noindent
However, this is slightly more time-consuming since the processes synchronize between themselves as many times as there
are processes.  In the next Hello world example we use the send and receive functions in order to a have a synchronized
action.
% --- end paragraph admon ---



\subsection*{\href{{https://github.com/CompPhysics/ComputationalPhysics2/blob/gh-pages/doc/Programs/LecturePrograms/programs/MPI/chapter07/program4.cpp}}{Ordered output}}

% --- begin paragraph admon ---
\paragraph{}

















\begin{Verbatim}[numbers=none,fontsize=\fontsize{9pt}{9pt},baselinestretch=0.95]
.....
int numprocs, my_rank, flag;
MPI_Status status;
MPI_Init (&nargs, &args);
MPI_Comm_size (MPI_COMM_WORLD, &numprocs);
MPI_Comm_rank (MPI_COMM_WORLD, &my_rank);
if (my_rank > 0)
MPI_Recv (&flag, 1, MPI_INT, my_rank-1, 100, 
           MPI_COMM_WORLD, &status);
cout << "Hello world, I have  rank " << my_rank << " out of " 
<< numprocs << endl;
if (my_rank < numprocs-1)
MPI_Send (&my_rank, 1, MPI_INT, my_rank+1, 
          100, MPI_COMM_WORLD);
MPI_Finalize ();

\end{Verbatim}
% --- end paragraph admon ---



\subsection*{Note 3}

% --- begin paragraph admon ---
\paragraph{}

The basic sending of messages is given by the function $MPI\_SEND$, which in C/C++
is defined as 




\begin{minted}[fontsize=\fontsize{9pt}{9pt},linenos=false,mathescape,baselinestretch=1.0,fontfamily=tt,xleftmargin=7mm]{c++}
int MPI_Send(void *buf, int count, 
             MPI_Datatype datatype, 
             int dest, int tag, MPI_Comm comm)}

\end{minted}

This single command allows the passing of any kind of variable, even a large array, to any group of tasks. 
The variable \textbf{buf} is the variable we wish to send while \textbf{count}
is the  number of variables we are passing. If we are passing only a single value, this should be 1. 

If we transfer an array, it is  the overall size of the array. 
For example, if we want to send a 10 by 10 array, count would be $10\times 10=100$ 
since we are  actually passing 100 values.
% --- end paragraph admon ---



\subsection*{Note 4}

% --- begin paragraph admon ---
\paragraph{}

Once you have  sent a message, you must receive it on another task. The function $MPI\_RECV$
is similar to the send call.




\begin{minted}[fontsize=\fontsize{9pt}{9pt},linenos=false,mathescape,baselinestretch=1.0,fontfamily=tt,xleftmargin=7mm]{c++}
int MPI_Recv( void *buf, int count, MPI_Datatype datatype, 
            int source, 
            int tag, MPI_Comm comm, MPI_Status *status )

\end{minted}


The arguments that are different from those in MPI\_SEND are
\textbf{buf} which  is the name of the variable where you will  be storing the received data, 
\textbf{source} which  replaces the destination in the send command. This is the return ID of the sender.

Finally,  we have used  $MPI\_Status\_status$,  
where one can check if the receive was completed.

The output of this code is the same as the previous example, but now
process 0 sends a message to process 1, which forwards it further
to process 2, and so forth.
% --- end paragraph admon ---



\subsection*{\href{{https://github.com/CompPhysics/ComputationalPhysics2/blob/gh-pages/doc/Programs/LecturePrograms/programs/MPI/chapter07/program6.cpp}}{Numerical integration in parallel}}

% --- begin paragraph admon ---
\paragraph{Integrating $\pi$.}

\begin{itemize}
\item The code example computes $\pi$ using the trapezoidal rules.

\item The trapezoidal rule
\end{itemize}

\noindent
\[
   I=\int_a^bf(x) dx\approx h\left(f(a)/2 + f(a+h) +f(a+2h)+\dots +f(b-h)+ f(b)/2\right).
\]
Click \href{{https://github.com/CompPhysics/ComputationalPhysics2/blob/gh-pages/doc/Programs/LecturePrograms/programs/MPI/chapter07/program6.cpp}}{on this link} for the full program.
% --- end paragraph admon ---



\subsection*{Dissection of trapezoidal rule with $MPI\_reduce$}

% --- begin paragraph admon ---
\paragraph{}


















\begin{minted}[fontsize=\fontsize{9pt}{9pt},linenos=false,mathescape,baselinestretch=1.0,fontfamily=tt,xleftmargin=7mm]{c++}
//    Trapezoidal rule and numerical integration usign MPI
using namespace std;
#include <mpi.h>
#include <iostream>

//     Here we define various functions called by the main program

double int_function(double );
double trapezoidal_rule(double , double , int , double (*)(double));

//   Main function begins here
int main (int nargs, char* args[])
{
  int n, local_n, numprocs, my_rank; 
  double a, b, h, local_a, local_b, total_sum, local_sum;   
  double  time_start, time_end, total_time;

\end{minted}
% --- end paragraph admon ---



\subsection*{Dissection of trapezoidal rule}

% --- begin paragraph admon ---
\paragraph{}
















\begin{minted}[fontsize=\fontsize{9pt}{9pt},linenos=false,mathescape,baselinestretch=1.0,fontfamily=tt,xleftmargin=7mm]{c++}
  //  MPI initializations
  MPI_Init (&nargs, &args);
  MPI_Comm_size (MPI_COMM_WORLD, &numprocs);
  MPI_Comm_rank (MPI_COMM_WORLD, &my_rank);
  time_start = MPI_Wtime();
  //  Fixed values for a, b and n 
  a = 0.0 ; b = 1.0;  n = 1000;
  h = (b-a)/n;    // h is the same for all processes 
  local_n = n/numprocs;  
  // make sure n > numprocs, else integer division gives zero
  // Length of each process' interval of
  // integration = local_n*h.  
  local_a = a + my_rank*local_n*h;
  local_b = local_a + local_n*h;

\end{minted}
% --- end paragraph admon ---



\subsection*{Integrating with \textbf{MPI}}

% --- begin paragraph admon ---
\paragraph{}


















\begin{minted}[fontsize=\fontsize{9pt}{9pt},linenos=false,mathescape,baselinestretch=1.0,fontfamily=tt,xleftmargin=7mm]{c++}
  total_sum = 0.0;
  local_sum = trapezoidal_rule(local_a, local_b, local_n, 
                               &int_function); 
  MPI_Reduce(&local_sum, &total_sum, 1, MPI_DOUBLE, 
              MPI_SUM, 0, MPI_COMM_WORLD);
  time_end = MPI_Wtime();
  total_time = time_end-time_start;
  if ( my_rank == 0) {
    cout << "Trapezoidal rule = " <<  total_sum << endl;
    cout << "Time = " <<  total_time  
         << " on number of processors: "  << numprocs  << endl;
  }
  // End MPI
  MPI_Finalize ();  
  return 0;
}  // end of main program

\end{minted}
% --- end paragraph admon ---



\subsection*{How do I use $MPI\_reduce$?}

% --- begin paragraph admon ---
\paragraph{}

Here we have used



\begin{minted}[fontsize=\fontsize{9pt}{9pt},linenos=false,mathescape,baselinestretch=1.0,fontfamily=tt,xleftmargin=7mm]{c++}
MPI_reduce( void *senddata, void* resultdata, int count, 
     MPI_Datatype datatype, MPI_Op, int root, MPI_Comm comm)

\end{minted}


The two variables $senddata$ and $resultdata$ are obvious, besides the fact that one sends the address
of the variable or the first element of an array.  If they are arrays they need to have the same size. 
The variable $count$ represents the total dimensionality, 1 in case of just one variable, 
while $MPI\_Datatype$ 
defines the type of variable which is sent and received.  

The new feature is $MPI\_Op$. It defines the type
of operation we want to do.
% --- end paragraph admon ---



\subsection*{More on $MPI\_Reduce$}

% --- begin paragraph admon ---
\paragraph{}
In our case, since we are summing
the rectangle  contributions from every process we define  $MPI\_Op = MPI\_SUM$.
If we have an array or matrix we can search for the largest og smallest element by sending either $MPI\_MAX$ or 
$MPI\_MIN$.  If we want the location as well (which array element) we simply transfer 
$MPI\_MAXLOC$ or $MPI\_MINOC$. If we want the product we write $MPI\_PROD$. 

$MPI\_Allreduce$ is defined as



\begin{minted}[fontsize=\fontsize{9pt}{9pt},linenos=false,mathescape,baselinestretch=1.0,fontfamily=tt,xleftmargin=7mm]{c++}
MPI_Allreduce( void *senddata, void* resultdata, int count, 
          MPI_Datatype datatype, MPI_Op, MPI_Comm comm)        

\end{minted}
% --- end paragraph admon ---



\subsection*{Dissection of trapezoidal rule}

% --- begin paragraph admon ---
\paragraph{}

We use $MPI\_reduce$ to collect data from each process. Note also the use of the function 
$MPI\_Wtime$. 








\begin{minted}[fontsize=\fontsize{9pt}{9pt},linenos=false,mathescape,baselinestretch=1.0,fontfamily=tt,xleftmargin=7mm]{c++}
//  this function defines the function to integrate
double int_function(double x)
{
  double value = 4./(1.+x*x);
  return value;
} // end of function to evaluate


\end{minted}
% --- end paragraph admon ---



\subsection*{Dissection of trapezoidal rule}

% --- begin paragraph admon ---
\paragraph{}



















\begin{minted}[fontsize=\fontsize{9pt}{9pt},linenos=false,mathescape,baselinestretch=1.0,fontfamily=tt,xleftmargin=7mm]{c++}
//  this function defines the trapezoidal rule
double trapezoidal_rule(double a, double b, int n, 
                         double (*func)(double))
{
  double trapez_sum;
  double fa, fb, x, step;
  int    j;
  step=(b-a)/((double) n);
  fa=(*func)(a)/2. ;
  fb=(*func)(b)/2. ;
  trapez_sum=0.;
  for (j=1; j <= n-1; j++){
    x=j*step+a;
    trapez_sum+=(*func)(x);
  }
  trapez_sum=(trapez_sum+fb+fa)*step;
  return trapez_sum;
}  // end trapezoidal_rule 

\end{minted}
% --- end paragraph admon ---



\subsection*{\href{{https://github.com/CompPhysics/ComputationalPhysics2/blob/master/doc/Programs/ParallelizationMPI/MPIvmcqdot.cpp}}{The quantum dot program for two electrons}}

% --- begin paragraph admon ---
\paragraph{}










































































































































































































































































































































































































































































\begin{minted}[fontsize=\fontsize{9pt}{9pt},linenos=false,mathescape,baselinestretch=1.0,fontfamily=tt,xleftmargin=7mm]{c++}
// Variational Monte Carlo for atoms with importance sampling, slater det
// Test case for 2-electron quantum dot, no classes using Mersenne-Twister RNG
#include "mpi.h"
#include <cmath>
#include <random>
#include <string>
#include <iostream>
#include <fstream>
#include <iomanip>
#include "vectormatrixclass.h"

using namespace  std;
// output file as global variable
ofstream ofile;  
// the step length and its squared inverse for the second derivative 
//  Here we define global variables  used in various functions
//  These can be changed by using classes
int Dimension = 2; 
int NumberParticles  = 2;  //  we fix also the number of electrons to be 2

// declaration of functions 

// The Mc sampling for the variational Monte Carlo 
void  MonteCarloSampling(int, double &, double &, Vector &);

// The variational wave function
double  WaveFunction(Matrix &, Vector &);

// The local energy 
double  LocalEnergy(Matrix &, Vector &);

// The quantum force
void  QuantumForce(Matrix &, Matrix &, Vector &);


// inline function for single-particle wave function
inline double SPwavefunction(double r, double alpha) { 
   return exp(-alpha*r*0.5);
}

// inline function for derivative of single-particle wave function
inline double DerivativeSPwavefunction(double r, double alpha) { 
  return -r*alpha;
}

// function for absolute value of relative distance
double RelativeDistance(Matrix &r, int i, int j) { 
      double r_ij = 0;  
      for (int k = 0; k < Dimension; k++) { 
	r_ij += (r(i,k)-r(j,k))*(r(i,k)-r(j,k));
      }
      return sqrt(r_ij); 
}

// inline function for derivative of Jastrow factor
inline double JastrowDerivative(Matrix &r, double beta, int i, int j, int k){
  return (r(i,k)-r(j,k))/(RelativeDistance(r, i, j)*pow(1.0+beta*RelativeDistance(r, i, j),2));
}

// function for square of position of single particle
double singleparticle_pos2(Matrix &r, int i) { 
    double r_single_particle = 0;
    for (int j = 0; j < Dimension; j++) { 
      r_single_particle  += r(i,j)*r(i,j);
    }
    return r_single_particle;
}

void lnsrch(int n, Vector &xold, double fold, Vector &g, Vector &p, Vector &x,
		 double *f, double stpmax, int *check, double (*func)(Vector &p));

void dfpmin(Vector &p, int n, double gtol, int *iter, double *fret,
	    double(*func)(Vector &p), void (*dfunc)(Vector &p, Vector &g));

static double sqrarg;
#define SQR(a) ((sqrarg=(a)) == 0.0 ? 0.0 : sqrarg*sqrarg)


static double maxarg1,maxarg2;
#define FMAX(a,b) (maxarg1=(a),maxarg2=(b),(maxarg1) > (maxarg2) ?\
        (maxarg1) : (maxarg2))


// Begin of main program   

int main(int argc, char* argv[])
{

  //  MPI initializations
  int NumberProcesses, MyRank, NumberMCsamples;
  MPI_Init (&argc, &argv);
  MPI_Comm_size (MPI_COMM_WORLD, &NumberProcesses);
  MPI_Comm_rank (MPI_COMM_WORLD, &MyRank);
  double StartTime = MPI_Wtime();
  if (MyRank == 0 && argc <= 1) {
    cout << "Bad Usage: " << argv[0] << 
      " Read also output file on same line and number of Monte Carlo cycles" << endl;
  }
  // Read filename and number of Monte Carlo cycles from the command line
  if (MyRank == 0 && argc > 2) {
    string filename = argv[1]; // first command line argument after name of program
    NumberMCsamples  = atoi(argv[2]);
    string fileout = filename;
    string argument = to_string(NumberMCsamples);
    // Final filename as filename+NumberMCsamples
    fileout.append(argument);
    ofile.open(fileout);
  }
  // broadcast the number of  Monte Carlo samples
  MPI_Bcast (&NumberMCsamples, 1, MPI_INT, 0, MPI_COMM_WORLD);
  // Two variational parameters only
  Vector VariationalParameters(2);
  int TotalNumberMCsamples = NumberMCsamples*NumberProcesses; 
  // Loop over variational parameters
  for (double alpha = 0.5; alpha <= 1.5; alpha +=0.1){
    for (double beta = 0.1; beta <= 0.5; beta +=0.05){
      VariationalParameters(0) = alpha;  // value of alpha
      VariationalParameters(1) = beta;  // value of beta
      //  Do the mc sampling  and accumulate data with MPI_Reduce
      double TotalEnergy, TotalEnergySquared, LocalProcessEnergy, LocalProcessEnergy2;
      LocalProcessEnergy = LocalProcessEnergy2 = 0.0;
      MonteCarloSampling(NumberMCsamples, LocalProcessEnergy, LocalProcessEnergy2, VariationalParameters);
      //  Collect data in total averages
      MPI_Reduce(&LocalProcessEnergy, &TotalEnergy, 1, MPI_DOUBLE, MPI_SUM, 0, MPI_COMM_WORLD);
      MPI_Reduce(&LocalProcessEnergy2, &TotalEnergySquared, 1, MPI_DOUBLE, MPI_SUM, 0, MPI_COMM_WORLD);
      // Print out results  in case of Master node, set to MyRank = 0
      if ( MyRank == 0) {
	double Energy = TotalEnergy/( (double)NumberProcesses);
	double Variance = TotalEnergySquared/( (double)NumberProcesses)-Energy*Energy;
	double StandardDeviation = sqrt(Variance/((double)TotalNumberMCsamples)); // over optimistic error
	ofile << setiosflags(ios::showpoint | ios::uppercase);
	ofile << setw(15) << setprecision(8) << VariationalParameters(0);
	ofile << setw(15) << setprecision(8) << VariationalParameters(1);
	ofile << setw(15) << setprecision(8) << Energy;
	ofile << setw(15) << setprecision(8) << Variance;
	ofile << setw(15) << setprecision(8) << StandardDeviation << endl;
      }
    }
  }
  double EndTime = MPI_Wtime();
  double TotalTime = EndTime-StartTime;
  if ( MyRank == 0 )  cout << "Time = " <<  TotalTime  << " on number of processors: "  << NumberProcesses  << endl;
  if (MyRank == 0)  ofile.close();  // close output file
  // End MPI
  MPI_Finalize ();  
  return 0;
}  //  end of main function


// Monte Carlo sampling with the Metropolis algorithm  

void MonteCarloSampling(int NumberMCsamples, double &cumulative_e, double &cumulative_e2, Vector &VariationalParameters)
{

 // Initialize the seed and call the Mersienne algo
  std::random_device rd;
  std::mt19937_64 gen(rd());
  // Set up the uniform distribution for x \in [[0, 1]
  std::uniform_real_distribution<double> UniformNumberGenerator(0.0,1.0);
  std::normal_distribution<double> Normaldistribution(0.0,1.0);
  // diffusion constant from Schroedinger equation
  double D = 0.5; 
  double timestep = 0.05;  //  we fix the time step  for the gaussian deviate
  // allocate matrices which contain the position of the particles  
  Matrix OldPosition( NumberParticles, Dimension), NewPosition( NumberParticles, Dimension);
  Matrix OldQuantumForce(NumberParticles, Dimension), NewQuantumForce(NumberParticles, Dimension);
  double Energy = 0.0; double EnergySquared = 0.0; double DeltaE = 0.0;
  //  initial trial positions
  for (int i = 0; i < NumberParticles; i++) { 
    for (int j = 0; j < Dimension; j++) {
      OldPosition(i,j) = Normaldistribution(gen)*sqrt(timestep);
    }
  }
  double OldWaveFunction = WaveFunction(OldPosition, VariationalParameters);
  QuantumForce(OldPosition, OldQuantumForce, VariationalParameters);
  // loop over monte carlo cycles 
  for (int cycles = 1; cycles <= NumberMCsamples; cycles++){ 
    // new position 
    for (int i = 0; i < NumberParticles; i++) { 
      for (int j = 0; j < Dimension; j++) {
	// gaussian deviate to compute new positions using a given timestep
	NewPosition(i,j) = OldPosition(i,j) + Normaldistribution(gen)*sqrt(timestep)+OldQuantumForce(i,j)*timestep*D;
	//	NewPosition(i,j) = OldPosition(i,j) + gaussian_deviate(&idum)*sqrt(timestep)+OldQuantumForce(i,j)*timestep*D;
      }  
      //  for the other particles we need to set the position to the old position since
      //  we move only one particle at the time
      for (int k = 0; k < NumberParticles; k++) {
	if ( k != i) {
	  for (int j = 0; j < Dimension; j++) {
	    NewPosition(k,j) = OldPosition(k,j);
	  }
	} 
      }
      double NewWaveFunction = WaveFunction(NewPosition, VariationalParameters); 
      QuantumForce(NewPosition, NewQuantumForce, VariationalParameters);
      //  we compute the log of the ratio of the greens functions to be used in the 
      //  Metropolis-Hastings algorithm
      double GreensFunction = 0.0;            
      for (int j = 0; j < Dimension; j++) {
	GreensFunction += 0.5*(OldQuantumForce(i,j)+NewQuantumForce(i,j))*
	  (D*timestep*0.5*(OldQuantumForce(i,j)-NewQuantumForce(i,j))-NewPosition(i,j)+OldPosition(i,j));
      }
      GreensFunction = exp(GreensFunction);
      // The Metropolis test is performed by moving one particle at the time
      if(UniformNumberGenerator(gen) <= GreensFunction*NewWaveFunction*NewWaveFunction/OldWaveFunction/OldWaveFunction ) { 
	for (int  j = 0; j < Dimension; j++) {
	  OldPosition(i,j) = NewPosition(i,j);
	  OldQuantumForce(i,j) = NewQuantumForce(i,j);
	}
	OldWaveFunction = NewWaveFunction;
      }
    }  //  end of loop over particles
    // compute local energy  
    double DeltaE = LocalEnergy(OldPosition, VariationalParameters);
    // update energies
    Energy += DeltaE;
    EnergySquared += DeltaE*DeltaE;
  }   // end of loop over MC trials   
  // update the energy average and its squared 
  cumulative_e = Energy/NumberMCsamples;
  cumulative_e2 = EnergySquared/NumberMCsamples;
}   // end MonteCarloSampling function  


// Function to compute the squared wave function and the quantum force

double  WaveFunction(Matrix &r, Vector &VariationalParameters)
{
  double wf = 0.0;
  // full Slater determinant for two particles, replace with Slater det for more particles 
  wf  = SPwavefunction(singleparticle_pos2(r, 0), VariationalParameters(0))*SPwavefunction(singleparticle_pos2(r, 1),VariationalParameters(0));
  // contribution from Jastrow factor
  for (int i = 0; i < NumberParticles-1; i++) { 
    for (int j = i+1; j < NumberParticles; j++) {
      wf *= exp(RelativeDistance(r, i, j)/((1.0+VariationalParameters(1)*RelativeDistance(r, i, j))));
    }
  }
  return wf;
}

// Function to calculate the local energy without numerical derivation of kinetic energy

double  LocalEnergy(Matrix &r, Vector &VariationalParameters)
{

  // compute the kinetic and potential energy from the single-particle part
  // for a many-electron system this has to be replaced by a Slater determinant
  // The absolute value of the interparticle length
  Matrix length( NumberParticles, NumberParticles);
  // Set up interparticle distance
  for (int i = 0; i < NumberParticles-1; i++) { 
    for(int j = i+1; j < NumberParticles; j++){
      length(i,j) = RelativeDistance(r, i, j);
      length(j,i) =  length(i,j);
    }
  }
  double KineticEnergy = 0.0;
  // Set up kinetic energy from Slater and Jastrow terms
  for (int i = 0; i < NumberParticles; i++) { 
    for (int k = 0; k < Dimension; k++) {
      double sum1 = 0.0; 
      for(int j = 0; j < NumberParticles; j++){
	if ( j != i) {
	  sum1 += JastrowDerivative(r, VariationalParameters(1), i, j, k);
	}
      }
      KineticEnergy += (sum1+DerivativeSPwavefunction(r(i,k),VariationalParameters(0)))*(sum1+DerivativeSPwavefunction(r(i,k),VariationalParameters(0)));
    }
  }
  KineticEnergy += -2*VariationalParameters(0)*NumberParticles;
  for (int i = 0; i < NumberParticles-1; i++) {
      for (int j = i+1; j < NumberParticles; j++) {
        KineticEnergy += 2.0/(pow(1.0 + VariationalParameters(1)*length(i,j),2))*(1.0/length(i,j)-2*VariationalParameters(1)/(1+VariationalParameters(1)*length(i,j)) );
      }
  }
  KineticEnergy *= -0.5;
  // Set up potential energy, external potential + eventual electron-electron repulsion
  double PotentialEnergy = 0;
  for (int i = 0; i < NumberParticles; i++) { 
    double DistanceSquared = singleparticle_pos2(r, i);
    PotentialEnergy += 0.5*DistanceSquared;  // sp energy HO part, note it has the oscillator frequency set to 1!
  }
  // Add the electron-electron repulsion
  for (int i = 0; i < NumberParticles-1; i++) { 
    for (int j = i+1; j < NumberParticles; j++) {
      PotentialEnergy += 1.0/length(i,j);          
    }
  }
  double LocalE = KineticEnergy+PotentialEnergy;
  return LocalE;
}

// Compute the analytical expression for the quantum force
void  QuantumForce(Matrix &r, Matrix &qforce, Vector &VariationalParameters)
{
  // compute the first derivative 
  for (int i = 0; i < NumberParticles; i++) {
    for (int k = 0; k < Dimension; k++) {
      // single-particle part, replace with Slater det for larger systems
      double sppart = DerivativeSPwavefunction(r(i,k),VariationalParameters(0));
      //  Jastrow factor contribution
      double Jsum = 0.0;
      for (int j = 0; j < NumberParticles; j++) {
	if ( j != i) {
	  Jsum += JastrowDerivative(r, VariationalParameters(1), i, j, k);
	}
      }
      qforce(i,k) = 2.0*(Jsum+sppart);
    }
  }
} // end of QuantumForce function


#define ITMAX 200
#define EPS 3.0e-8
#define TOLX (4*EPS)
#define STPMX 100.0

void dfpmin(Vector &p, int n, double gtol, int *iter, double *fret,
	    double(*func)(Vector &p), void (*dfunc)(Vector &p, Vector &g))
{

  int check,i,its,j;
  double den,fac,fad,fae,fp,stpmax,sum=0.0,sumdg,sumxi,temp,test;
  Vector dg(n), g(n), hdg(n), pnew(n), xi(n);
  Matrix hessian(n,n);

  fp=(*func)(p);
  (*dfunc)(p,g);
  for (i = 0;i < n;i++) {
    for (j = 0; j< n;j++) hessian(i,j)=0.0;
    hessian(i,i)=1.0;
    xi(i) = -g(i);
    sum += p(i)*p(i);
  }
  stpmax=STPMX*FMAX(sqrt(sum),(double)n);
  for (its=1;its<=ITMAX;its++) {
    *iter=its;
    lnsrch(n,p,fp,g,xi,pnew,fret,stpmax,&check,func);
    fp = *fret;
    for (i = 0; i< n;i++) {
      xi(i)=pnew(i)-p(i);
      p(i)=pnew(i);
    }
    test=0.0;
    for (i = 0;i< n;i++) {
      temp=fabs(xi(i))/FMAX(fabs(p(i)),1.0);
      if (temp > test) test=temp;
    }
    if (test < TOLX) {
      return;
    }
    for (i=0;i<n;i++) dg(i)=g(i);
    (*dfunc)(p,g);
    test=0.0;
    den=FMAX(*fret,1.0);
    for (i=0;i<n;i++) {
      temp=fabs(g(i))*FMAX(fabs(p(i)),1.0)/den;
      if (temp > test) test=temp;
    }
    if (test < gtol) {
      return;
    }
    for (i=0;i<n;i++) dg(i)=g(i)-dg(i);
    for (i=0;i<n;i++) {
      hdg(i)=0.0;
      for (j=0;j<n;j++) hdg(i) += hessian(i,j)*dg(j);
    }
    fac=fae=sumdg=sumxi=0.0;
    for (i=0;i<n;i++) {
      fac += dg(i)*xi(i);
      fae += dg(i)*hdg(i);
      sumdg += SQR(dg(i));
      sumxi += SQR(xi(i));
    }
    if (fac*fac > EPS*sumdg*sumxi) {
      fac=1.0/fac;
      fad=1.0/fae;
      for (i=0;i<n;i++) dg(i)=fac*xi(i)-fad*hdg(i);
      for (i=0;i<n;i++) {
	for (j=0;j<n;j++) {
	  hessian(i,j) += fac*xi(i)*xi(j)
	    -fad*hdg(i)*hdg(j)+fae*dg(i)*dg(j);
	}
      }
    }
    for (i=0;i<n;i++) {
      xi(i)=0.0;
      for (j=0;j<n;j++) xi(i) -= hessian(i,j)*g(j);
    }
  }
  cout << "too many iterations in dfpmin" << endl;
}
#undef ITMAX
#undef EPS
#undef TOLX
#undef STPMX

#define ALF 1.0e-4
#define TOLX 1.0e-7

void lnsrch(int n, Vector &xold, double fold, Vector &g, Vector &p, Vector &x,
	    double *f, double stpmax, int *check, double (*func)(Vector &p))
{
  int i;
  double a,alam,alam2,alamin,b,disc,f2,fold2,rhs1,rhs2,slope,sum,temp,
    test,tmplam;

  *check=0;
  for (sum=0.0,i=0;i<n;i++) sum += p(i)*p(i);
  sum=sqrt(sum);
  if (sum > stpmax)
    for (i=0;i<n;i++) p(i) *= stpmax/sum;
  for (slope=0.0,i=0;i<n;i++)
    slope += g(i)*p(i);
  test=0.0;
  for (i=0;i<n;i++) {
    temp=fabs(p(i))/FMAX(fabs(xold(i)),1.0);
    if (temp > test) test=temp;
  }
  alamin=TOLX/test;
  alam=1.0;
  for (;;) {
    for (i=0;i<n;i++) x(i)=xold(i)+alam*p(i);
    *f=(*func)(x);
    if (alam < alamin) {
      for (i=0;i<n;i++) x(i)=xold(i);
      *check=1;
      return;
    } else if (*f <= fold+ALF*alam*slope) return;
    else {
      if (alam == 1.0)
	tmplam = -slope/(2.0*(*f-fold-slope));
      else {
	rhs1 = *f-fold-alam*slope;
	rhs2=f2-fold2-alam2*slope;
	a=(rhs1/(alam*alam)-rhs2/(alam2*alam2))/(alam-alam2);
	b=(-alam2*rhs1/(alam*alam)+alam*rhs2/(alam2*alam2))/(alam-alam2);
	if (a == 0.0) tmplam = -slope/(2.0*b);
	else {
	  disc=b*b-3.0*a*slope;
	  if (disc<0.0) cout << "Roundoff problem in lnsrch." << endl;
	  else tmplam=(-b+sqrt(disc))/(3.0*a);
	}
	if (tmplam>0.5*alam)
	  tmplam=0.5*alam;
      }
    }
    alam2=alam;
    f2 = *f;
    fold2=fold;
    alam=FMAX(tmplam,0.1*alam);
  }
}
#undef ALF
#undef TOLX


\end{minted}
% --- end paragraph admon ---



\subsection*{What is OpenMP}

% --- begin paragraph admon ---
\paragraph{}
\begin{itemize}
\item OpenMP provides high-level thread programming

\item Multiple cooperating threads are allowed to run simultaneously

\item Threads are created and destroyed dynamically in a fork-join pattern
\begin{itemize}

   \item An OpenMP program consists of a number of parallel regions

   \item Between two parallel regions there is only one master thread

   \item In the beginning of a parallel region, a team of new threads is spawned

\end{itemize}

\noindent
  \item The newly spawned threads work simultaneously with the master thread

  \item At the end of a parallel region, the new threads are destroyed
\end{itemize}

\noindent
Many good tutorials online and excellent textbook
\begin{enumerate}
\item \href{{http://mitpress.mit.edu/books/using-openmp}}{Using OpenMP, by B. Chapman, G. Jost, and A. van der Pas}

\item Many tutorials online like \href{{http://www.openmp.org}}{OpenMP official site}
\end{enumerate}

\noindent
% --- end paragraph admon ---



\subsection*{Getting started, things to remember}

% --- begin paragraph admon ---
\paragraph{}
\begin{itemize}
 \item Remember the header file 
\end{itemize}

\noindent


\begin{minted}[fontsize=\fontsize{9pt}{9pt},linenos=false,mathescape,baselinestretch=1.0,fontfamily=tt,xleftmargin=7mm]{c++}
#include <omp.h>

\end{minted}

\begin{itemize}
 \item Insert compiler directives in C++ syntax as 
\end{itemize}

\noindent


\begin{minted}[fontsize=\fontsize{9pt}{9pt},linenos=false,mathescape,baselinestretch=1.0,fontfamily=tt,xleftmargin=7mm]{c++}
#pragma omp...

\end{minted}

\begin{itemize}
\item Compile with for example \emph{c++ -fopenmp code.cpp}

\item Execute
\begin{itemize}

  \item Remember to assign the environment variable \textbf{OMP NUM THREADS}

  \item It specifies the total number of threads inside a parallel region, if not otherwise overwritten
\end{itemize}

\noindent
\end{itemize}

\noindent
% --- end paragraph admon ---



\subsection*{OpenMP syntax}
\begin{itemize}
\item Mostly directives
\end{itemize}

\noindent


\begin{minted}[fontsize=\fontsize{9pt}{9pt},linenos=false,mathescape,baselinestretch=1.0,fontfamily=tt,xleftmargin=7mm]{c++}
#pragma omp construct [ clause ...]

\end{minted}

\begin{itemize}
 \item Some functions and types 
\end{itemize}

\noindent


\begin{minted}[fontsize=\fontsize{9pt}{9pt},linenos=false,mathescape,baselinestretch=1.0,fontfamily=tt,xleftmargin=7mm]{c++}
#include <omp.h>

\end{minted}

\begin{itemize}
 \item Most apply to a block of code

 \item Specifically, a \textbf{structured block}

 \item Enter at top, exit at bottom only, exit(), abort() permitted
\end{itemize}

\noindent
\subsection*{Different OpenMP styles of parallelism}
OpenMP supports several different ways to specify thread parallelism

\begin{itemize}
\item General parallel regions: All threads execute the code, roughly as if you made a routine of that region and created a thread to run that code

\item Parallel loops: Special case for loops, simplifies data parallel code

\item Task parallelism, new in OpenMP 3

\item Several ways to manage thread coordination, including Master regions and Locks

\item Memory model for shared data
\end{itemize}

\noindent
\subsection*{General code structure}

% --- begin paragraph admon ---
\paragraph{}




















\begin{minted}[fontsize=\fontsize{9pt}{9pt},linenos=false,mathescape,baselinestretch=1.0,fontfamily=tt,xleftmargin=7mm]{c++}
#include <omp.h>
main ()
{
int var1, var2, var3;
/* serial code */
/* ... */
/* start of a parallel region */
#pragma omp parallel private(var1, var2) shared(var3)
{
/* ... */
}
/* more serial code */
/* ... */
/* another parallel region */
#pragma omp parallel
{
/* ... */
}
}

\end{minted}
% --- end paragraph admon ---



\subsection*{Parallel region}

% --- begin paragraph admon ---
\paragraph{}
\begin{itemize}
\item A parallel region is a block of code that is executed by a team of threads

\item The following compiler directive creates a parallel region
\end{itemize}

\noindent


\begin{minted}[fontsize=\fontsize{9pt}{9pt},linenos=false,mathescape,baselinestretch=1.0,fontfamily=tt,xleftmargin=7mm]{c++}
#pragma omp parallel { ... }

\end{minted}

\begin{itemize}
\item Clauses can be added at the end of the directive

\item Most often used clauses:
\begin{itemize}

 \item \textbf{default(shared)} or \textbf{default(none)}

 \item \textbf{public(list of variables)}

 \item \textbf{private(list of variables)}
\end{itemize}

\noindent
\end{itemize}

\noindent
% --- end paragraph admon ---



\subsection*{Hello world, not again, please!}

% --- begin paragraph admon ---
\paragraph{}


















\begin{minted}[fontsize=\fontsize{9pt}{9pt},linenos=false,mathescape,baselinestretch=1.0,fontfamily=tt,xleftmargin=7mm]{c++}
#include <omp.h>
#include <cstdio>
int main (int argc, char *argv[])
{
int th_id, nthreads;
#pragma omp parallel private(th_id) shared(nthreads)
{
th_id = omp_get_thread_num();
printf("Hello World from thread %d\n", th_id);
#pragma omp barrier
if ( th_id == 0 ) {
nthreads = omp_get_num_threads();
printf("There are %d threads\n",nthreads);
}
}
return 0;
}

\end{minted}
% --- end paragraph admon ---



\subsection*{Hello world, yet another variant}

% --- begin paragraph admon ---
\paragraph{}














\begin{minted}[fontsize=\fontsize{9pt}{9pt},linenos=false,mathescape,baselinestretch=1.0,fontfamily=tt,xleftmargin=7mm]{c++}
#include <cstdio>
#include <omp.h>
int main(int argc, char *argv[]) 
{
 omp_set_num_threads(4); 
#pragma omp parallel
 {
   int id = omp_get_thread_num();
   int nproc = omp_get_num_threads(); 
   cout << "Hello world with id number and processes " <<  id <<  nproc << endl;
 } 
return 0;
}

\end{minted}

Variables declared outside of the parallel region are shared by all threads
If a variable like \textbf{id} is  declared outside of the 


\begin{minted}[fontsize=\fontsize{9pt}{9pt},linenos=false,mathescape,baselinestretch=1.0,fontfamily=tt,xleftmargin=7mm]{c++}
#pragma omp parallel, 

\end{minted}

it would have been shared by various the threads, possibly causing erroneous output
\begin{itemize}
 \item Why? What would go wrong? Why do we add  possibly?
\end{itemize}

\noindent
% --- end paragraph admon ---



\subsection*{Important OpenMP library routines}

% --- begin paragraph admon ---
\paragraph{}

\begin{itemize}
\item \textbf{int omp get num threads ()}, returns the number of threads inside a parallel region

\item \textbf{int omp get thread num ()},  returns the  a thread for each thread inside a parallel region

\item \textbf{void omp set num threads (int)}, sets the number of threads to be used

\item \textbf{void omp set nested (int)},  turns nested parallelism on/off
\end{itemize}

\noindent
% --- end paragraph admon ---



\subsection*{Private variables}

% --- begin paragraph admon ---
\paragraph{}
Private clause can be used to make thread- private versions of such variables: 






\begin{minted}[fontsize=\fontsize{9pt}{9pt},linenos=false,mathescape,baselinestretch=1.0,fontfamily=tt,xleftmargin=7mm]{c++}
#pragma omp parallel private(id)
{
 int id = omp_get_thread_num();
 cout << "My thread num" << id << endl; 
}

\end{minted}

\begin{itemize}
\item What is their value on entry? Exit?

\item OpenMP provides ways to control that

\item Can use default(none) to require the sharing of each variable to be described
\end{itemize}

\noindent
% --- end paragraph admon ---



\subsection*{Master region}

% --- begin paragraph admon ---
\paragraph{}
It is often useful to have only one thread execute some of the code in a parallel region. I/O statements are a common example









\begin{minted}[fontsize=\fontsize{9pt}{9pt},linenos=false,mathescape,baselinestretch=1.0,fontfamily=tt,xleftmargin=7mm]{c++}
#pragma omp parallel 
{
  #pragma omp master
   {
      int id = omp_get_thread_num();
      cout << "My thread num" << id << endl; 
   } 
}

\end{minted}
% --- end paragraph admon ---



\subsection*{Parallel for loop}

% --- begin paragraph admon ---
\paragraph{}
\begin{itemize}
 \item Inside a parallel region, the following compiler directive can be used to parallelize a for-loop:
\end{itemize}

\noindent


\begin{minted}[fontsize=\fontsize{9pt}{9pt},linenos=false,mathescape,baselinestretch=1.0,fontfamily=tt,xleftmargin=7mm]{c++}
#pragma omp for

\end{minted}

\begin{itemize}
\item Clauses can be added, such as
\begin{itemize}

  \item \textbf{schedule(static, chunk size)}

  \item \textbf{schedule(dynamic, chunk size)} 

  \item \textbf{schedule(guided, chunk size)} (non-deterministic allocation)

  \item \textbf{schedule(runtime)}

  \item \textbf{private(list of variables)}

  \item \textbf{reduction(operator:variable)}

  \item \textbf{nowait}
\end{itemize}

\noindent
\end{itemize}

\noindent
% --- end paragraph admon ---



\subsection*{Parallel computations and loops}


% --- begin paragraph admon ---
\paragraph{}
OpenMP provides an easy way to parallelize a loop



\begin{minted}[fontsize=\fontsize{9pt}{9pt},linenos=false,mathescape,baselinestretch=1.0,fontfamily=tt,xleftmargin=7mm]{c++}
#pragma omp parallel for
  for (i=0; i<n; i++) c[i] = a[i];

\end{minted}

OpenMP handles index variable (no need to declare in for loop or make private)

Which thread does which values?  Several options.
% --- end paragraph admon ---



\subsection*{Scheduling of  loop computations}


% --- begin paragraph admon ---
\paragraph{}
We can let  the OpenMP runtime decide. The decision is about how the loop iterates are scheduled
and  OpenMP defines three choices of loop scheduling:
\begin{enumerate}
\item Static: Predefined at compile time. Lowest overhead, predictable

\item Dynamic: Selection made at runtime 

\item Guided: Special case of dynamic; attempts to reduce overhead
\end{enumerate}

\noindent
% --- end paragraph admon ---



\subsection*{Example code for loop scheduling}

% --- begin paragraph admon ---
\paragraph{}
















\begin{minted}[fontsize=\fontsize{9pt}{9pt},linenos=false,mathescape,baselinestretch=1.0,fontfamily=tt,xleftmargin=7mm]{c++}
#include <omp.h>
#define CHUNKSIZE 100
#define N 1000
int main (int argc, char *argv[])
{
int i, chunk;
float a[N], b[N], c[N];
for (i=0; i < N; i++) a[i] = b[i] = i * 1.0;
chunk = CHUNKSIZE;
#pragma omp parallel shared(a,b,c,chunk) private(i)
{
#pragma omp for schedule(dynamic,chunk)
for (i=0; i < N; i++) c[i] = a[i] + b[i];
} /* end of parallel region */
}

\end{minted}
% --- end paragraph admon ---



\subsection*{Example code for loop scheduling, guided instead of dynamic}

% --- begin paragraph admon ---
\paragraph{}
















\begin{minted}[fontsize=\fontsize{9pt}{9pt},linenos=false,mathescape,baselinestretch=1.0,fontfamily=tt,xleftmargin=7mm]{c++}
#include <omp.h>
#define CHUNKSIZE 100
#define N 1000
int main (int argc, char *argv[])
{
int i, chunk;
float a[N], b[N], c[N];
for (i=0; i < N; i++) a[i] = b[i] = i * 1.0;
chunk = CHUNKSIZE;
#pragma omp parallel shared(a,b,c,chunk) private(i)
{
#pragma omp for schedule(guided,chunk)
for (i=0; i < N; i++) c[i] = a[i] + b[i];
} /* end of parallel region */
}

\end{minted}
% --- end paragraph admon ---



\subsection*{More on Parallel for loop}

% --- begin paragraph admon ---
\paragraph{}
\begin{itemize}
\item The number of loop iterations cannot be non-deterministic; break, return, exit, goto not allowed inside the for-loop

\item The loop index is private to each thread

\item A reduction variable is special
\begin{itemize}

  \item During the for-loop there is a local private copy in each thread

  \item At the end of the for-loop, all the local copies are combined together by the reduction operation

\end{itemize}

\noindent
\item Unless the nowait clause is used, an implicit barrier synchronization will be added at the end by the compiler
\end{itemize}

\noindent


\begin{minted}[fontsize=\fontsize{9pt}{9pt},linenos=false,mathescape,baselinestretch=1.0,fontfamily=tt,xleftmargin=7mm]{c++}
// #pragma omp parallel and #pragma omp for

\end{minted}

can be combined into


\begin{minted}[fontsize=\fontsize{9pt}{9pt},linenos=false,mathescape,baselinestretch=1.0,fontfamily=tt,xleftmargin=7mm]{c++}
#pragma omp parallel for

\end{minted}
% --- end paragraph admon ---



\subsection*{What can happen with this loop?}


% --- begin paragraph admon ---
\paragraph{}
What happens with code like this 



\begin{minted}[fontsize=\fontsize{9pt}{9pt},linenos=false,mathescape,baselinestretch=1.0,fontfamily=tt,xleftmargin=7mm]{c++}
#pragma omp parallel for
for (i=0; i<n; i++) sum += a[i]*a[i];

\end{minted}

All threads can access the \textbf{sum} variable, but the addition is not atomic! It is important to avoid race between threads. So-called reductions in OpenMP are thus important for performance and for obtaining correct results.  OpenMP lets us indicate that a variable is used for a reduction with a particular operator. The above code becomes




\begin{minted}[fontsize=\fontsize{9pt}{9pt},linenos=false,mathescape,baselinestretch=1.0,fontfamily=tt,xleftmargin=7mm]{c++}
sum = 0.0;
#pragma omp parallel for reduction(+:sum)
for (i=0; i<n; i++) sum += a[i]*a[i];

\end{minted}
% --- end paragraph admon ---



\subsection*{Inner product}

% --- begin paragraph admon ---
\paragraph{}
\[
\sum_{i=0}^{n-1} a_ib_i
\]








\begin{minted}[fontsize=\fontsize{9pt}{9pt},linenos=false,mathescape,baselinestretch=1.0,fontfamily=tt,xleftmargin=7mm]{c++}
int i;
double sum = 0.;
/* allocating and initializing arrays */
/* ... */
#pragma omp parallel for default(shared) private(i) reduction(+:sum)
 for (i=0; i<N; i++) sum += a[i]*b[i];
}

\end{minted}
% --- end paragraph admon ---



\subsection*{Different threads do different tasks}

% --- begin paragraph admon ---
\paragraph{}

Different threads do different tasks independently, each section is executed by one thread.













\begin{minted}[fontsize=\fontsize{9pt}{9pt},linenos=false,mathescape,baselinestretch=1.0,fontfamily=tt,xleftmargin=7mm]{c++}
#pragma omp parallel
{
#pragma omp sections
{
#pragma omp section
funcA ();
#pragma omp section
funcB ();
#pragma omp section
funcC ();
}
}

\end{minted}
% --- end paragraph admon ---



\subsection*{Single execution}

% --- begin paragraph admon ---
\paragraph{}


\begin{minted}[fontsize=\fontsize{9pt}{9pt},linenos=false,mathescape,baselinestretch=1.0,fontfamily=tt,xleftmargin=7mm]{c++}
#pragma omp single { ... }

\end{minted}

The code is executed by one thread only, no guarantee which thread

Can introduce an implicit barrier at the end


\begin{minted}[fontsize=\fontsize{9pt}{9pt},linenos=false,mathescape,baselinestretch=1.0,fontfamily=tt,xleftmargin=7mm]{c++}
#pragma omp master { ... }

\end{minted}

Code executed by the master thread, guaranteed and no implicit barrier at the end.
% --- end paragraph admon ---



\subsection*{Coordination and synchronization}

% --- begin paragraph admon ---
\paragraph{}


\begin{minted}[fontsize=\fontsize{9pt}{9pt},linenos=false,mathescape,baselinestretch=1.0,fontfamily=tt,xleftmargin=7mm]{c++}
#pragma omp barrier

\end{minted}

Synchronization, must be encountered by all threads in a team (or none)


\begin{minted}[fontsize=\fontsize{9pt}{9pt},linenos=false,mathescape,baselinestretch=1.0,fontfamily=tt,xleftmargin=7mm]{c++}
#pragma omp ordered { a block of codes }

\end{minted}

is another form of synchronization (in sequential order).
The form


\begin{minted}[fontsize=\fontsize{9pt}{9pt},linenos=false,mathescape,baselinestretch=1.0,fontfamily=tt,xleftmargin=7mm]{c++}
#pragma omp critical { a block of codes }

\end{minted}

and 


\begin{minted}[fontsize=\fontsize{9pt}{9pt},linenos=false,mathescape,baselinestretch=1.0,fontfamily=tt,xleftmargin=7mm]{c++}
#pragma omp atomic { single assignment statement }

\end{minted}

is  more efficient than 


\begin{minted}[fontsize=\fontsize{9pt}{9pt},linenos=false,mathescape,baselinestretch=1.0,fontfamily=tt,xleftmargin=7mm]{c++}
#pragma omp critical

\end{minted}
% --- end paragraph admon ---



\subsection*{Data scope}

% --- begin paragraph admon ---
\paragraph{}
\begin{itemize}
\item OpenMP data scope attribute clauses:
\begin{itemize}

 \item \textbf{shared}

 \item \textbf{private}

 \item \textbf{firstprivate}

 \item \textbf{lastprivate}

 \item \textbf{reduction}
\end{itemize}

\noindent
\end{itemize}

\noindent
What are the purposes of these attributes
\begin{itemize}
\item define how and which variables are transferred to a parallel region (and back)

\item define which variables are visible to all threads in a parallel region, and which variables are privately allocated to each thread
\end{itemize}

\noindent
% --- end paragraph admon ---



\subsection*{Some remarks}

% --- begin paragraph admon ---
\paragraph{}

\begin{itemize}
\item When entering a parallel region, the \textbf{private} clause ensures each thread having its own new variable instances. The new variables are assumed to be uninitialized.

\item A shared variable exists in only one memory location and all threads can read and write to that address. It is the programmer's responsibility to ensure that multiple threads properly access a shared variable.

\item The \textbf{firstprivate} clause combines the behavior of the private clause with automatic initialization.

\item The \textbf{lastprivate} clause combines the behavior of the private clause with a copy back (from the last loop iteration or section) to the original variable outside the parallel region.
\end{itemize}

\noindent
% --- end paragraph admon ---



\subsection*{Parallelizing nested for-loops}

% --- begin paragraph admon ---
\paragraph{}

\begin{itemize}
 \item Serial code
\end{itemize}

\noindent






\begin{minted}[fontsize=\fontsize{9pt}{9pt},linenos=false,mathescape,baselinestretch=1.0,fontfamily=tt,xleftmargin=7mm]{c++}
for (i=0; i<100; i++)
    for (j=0; j<100; j++)
        a[i][j] = b[i][j] + c[i][j];
    }
}

\end{minted}


\begin{itemize}
\item Parallelization
\end{itemize}

\noindent







\begin{minted}[fontsize=\fontsize{9pt}{9pt},linenos=false,mathescape,baselinestretch=1.0,fontfamily=tt,xleftmargin=7mm]{c++}
#pragma omp parallel for private(j)
for (i=0; i<100; i++)
    for (j=0; j<100; j++)
       a[i][j] = b[i][j] + c[i][j];
    }
}

\end{minted}


\begin{itemize}
\item Why not parallelize the inner loop? to save overhead of repeated thread forks-joins

\item Why must \textbf{j} be private? To avoid race condition among the threads
\end{itemize}

\noindent
% --- end paragraph admon ---



\subsection*{Nested parallelism}

% --- begin paragraph admon ---
\paragraph{}
When a thread in a parallel region encounters another parallel construct, it
may create a new team of threads and become the master of the new
team.









\begin{minted}[fontsize=\fontsize{9pt}{9pt},linenos=false,mathescape,baselinestretch=1.0,fontfamily=tt,xleftmargin=7mm]{c++}
#pragma omp parallel num_threads(4)
{
/* .... */
#pragma omp parallel num_threads(2)
{
//  
}
}

\end{minted}
% --- end paragraph admon ---



\subsection*{Parallel tasks}

% --- begin paragraph admon ---
\paragraph{}











\begin{minted}[fontsize=\fontsize{9pt}{9pt},linenos=false,mathescape,baselinestretch=1.0,fontfamily=tt,xleftmargin=7mm]{c++}
#pragma omp task 
#pragma omp parallel shared(p_vec) private(i)
{
#pragma omp single
{
for (i=0; i<N; i++) {
  double r = random_number();
  if (p_vec[i] > r) {
#pragma omp task
   do_work (p_vec[i]);

\end{minted}
% --- end paragraph admon ---



\subsection*{Common mistakes}

% --- begin paragraph admon ---
\paragraph{}
Race condition






\begin{minted}[fontsize=\fontsize{9pt}{9pt},linenos=false,mathescape,baselinestretch=1.0,fontfamily=tt,xleftmargin=7mm]{c++}
int nthreads;
#pragma omp parallel shared(nthreads)
{
nthreads = omp_get_num_threads();
}

\end{minted}

Deadlock










\begin{minted}[fontsize=\fontsize{9pt}{9pt},linenos=false,mathescape,baselinestretch=1.0,fontfamily=tt,xleftmargin=7mm]{c++}
#pragma omp parallel
{
...
#pragma omp critical
{
...
#pragma omp barrier
}
}

\end{minted}
% --- end paragraph admon ---



\subsection*{Not all computations are simple}

% --- begin paragraph admon ---
\paragraph{}
Not all computations are simple loops where the data can be evenly 
divided among threads without any dependencies between threads

An example is finding the location and value of the largest element in an array







\begin{minted}[fontsize=\fontsize{9pt}{9pt},linenos=false,mathescape,baselinestretch=1.0,fontfamily=tt,xleftmargin=7mm]{c++}
for (i=0; i<n; i++) { 
   if (x[i] > maxval) {
      maxval = x[i];
      maxloc = i; 
   }
}

\end{minted}
% --- end paragraph admon ---



\subsection*{Not all computations are simple, competing threads}

% --- begin paragraph admon ---
\paragraph{}
All threads are potentially accessing and changing the same values, \textbf{maxloc} and \textbf{maxval}.
\begin{enumerate}
\item OpenMP provides several ways to coordinate access to shared values
\end{enumerate}

\noindent


\begin{minted}[fontsize=\fontsize{9pt}{9pt},linenos=false,mathescape,baselinestretch=1.0,fontfamily=tt,xleftmargin=7mm]{c++}
#pragma omp atomic

\end{minted}

\begin{enumerate}
\item Only one thread at a time can execute the following statement (not block). We can use the critical option
\end{enumerate}

\noindent


\begin{minted}[fontsize=\fontsize{9pt}{9pt},linenos=false,mathescape,baselinestretch=1.0,fontfamily=tt,xleftmargin=7mm]{c++}
#pragma omp critical

\end{minted}

\begin{enumerate}
\item Only one thread at a time can execute the following block
\end{enumerate}

\noindent
Atomic may be faster than critical but depends on hardware
% --- end paragraph admon ---



\subsection*{How to find the max value using OpenMP}

% --- begin paragraph admon ---
\paragraph{}
Write down the simplest algorithm and look carefully for race conditions. How would you handle them? 
The first step would be to parallelize as 








\begin{minted}[fontsize=\fontsize{9pt}{9pt},linenos=false,mathescape,baselinestretch=1.0,fontfamily=tt,xleftmargin=7mm]{c++}
#pragma omp parallel for
 for (i=0; i<n; i++) {
    if (x[i] > maxval) {
      maxval = x[i];
      maxloc = i; 
    }
}

\end{minted}
% --- end paragraph admon ---



\subsection*{Then deal with the race conditions}

% --- begin paragraph admon ---
\paragraph{}
Write down the simplest algorithm and look carefully for race conditions. How would you handle them? 
The first step would be to parallelize as 











\begin{minted}[fontsize=\fontsize{9pt}{9pt},linenos=false,mathescape,baselinestretch=1.0,fontfamily=tt,xleftmargin=7mm]{c++}
#pragma omp parallel for
 for (i=0; i<n; i++) {
#pragma omp critical
  {
     if (x[i] > maxval) {
       maxval = x[i];
       maxloc = i; 
     }
  }
} 

\end{minted}


Exercise: write a code which implements this and give an estimate on performance. Perform several runs,
with a serial code only with and without vectorization and compare the serial code with the one that  uses OpenMP. Run on different archictectures if you can.
% --- end paragraph admon ---



\subsection*{What can slow down OpenMP performance?}
Give it a thought!

\subsection*{What can slow down OpenMP performance?}

% --- begin paragraph admon ---
\paragraph{}
Performance poor because we insisted on keeping track of the maxval and location during the execution of the loop.
\begin{itemize}
 \item We do not care about the value during the execution of the loop, just the value at the end.
\end{itemize}

\noindent
This is a common source of performance issues, namely the description of the method used to compute a value imposes additional, unnecessary requirements or properties

\textbf{Idea: Have each thread find the maxloc in its own data, then combine and use temporary arrays indexed by thread number to hold the values found by each thread}
% --- end paragraph admon ---



\subsection*{Find the max location for each thread}

% --- begin paragraph admon ---
\paragraph{}















\begin{minted}[fontsize=\fontsize{9pt}{9pt},linenos=false,mathescape,baselinestretch=1.0,fontfamily=tt,xleftmargin=7mm]{c++}
int maxloc[MAX_THREADS], mloc;
double maxval[MAX_THREADS], mval; 
#pragma omp parallel shared(maxval,maxloc)
{
  int id = omp_get_thread_num(); 
  maxval[id] = -1.0e30;
#pragma omp for
   for (int i=0; i<n; i++) {
       if (x[i] > maxval[id]) { 
           maxloc[id] = i;
           maxval[id] = x[i]; 
       }
    }
}

\end{minted}
% --- end paragraph admon ---



\subsection*{Combine the values from each thread}

% --- begin paragraph admon ---
\paragraph{}














\begin{minted}[fontsize=\fontsize{9pt}{9pt},linenos=false,mathescape,baselinestretch=1.0,fontfamily=tt,xleftmargin=7mm]{c++}
#pragma omp flush (maxloc,maxval)
#pragma omp master
  {
    int nt = omp_get_num_threads(); 
    mloc = maxloc[0]; 
    mval = maxval[0]; 
    for (int i=1; i<nt; i++) {
        if (maxval[i] > mval) { 
           mval = maxval[i]; 
           mloc = maxloc[i];
        } 
     }
   }

\end{minted}

Note that we let the master process perform the last operation.
% --- end paragraph admon ---



\subsection*{\href{{https://github.com/CompPhysics/ComputationalPhysicsMSU/blob/master/doc/Programs/ParallelizationOpenMP/OpenMPvectornorm.cpp}}{Matrix-matrix multiplication}}
This code computes the norm of a vector using OpenMp


























































\begin{minted}[fontsize=\fontsize{9pt}{9pt},linenos=false,mathescape,baselinestretch=1.0,fontfamily=tt,xleftmargin=7mm]{text}
//  OpenMP program to compute vector norm by adding two other vectors
#include <cstdlib>
#include <iostream>
#include <cmath>
#include <iomanip>
#include  <omp.h>
# include <ctime>

using namespace std; // note use of namespace
int main (int argc, char* argv[])
{
  // read in dimension of vector
  int n = atoi(argv[1]);
  double *a, *b, *c;
  int i;
  int thread_num;
  double wtime, Norm2, s, angle;
  cout << "  Perform addition of two vectors and compute the norm-2." << endl;
  omp_set_num_threads(4);
  thread_num = omp_get_max_threads ();
  cout << "  The number of processors available = " << omp_get_num_procs () << endl ;
  cout << "  The number of threads available    = " << thread_num <<  endl;
  cout << "  The matrix order n                 = " << n << endl;

  s = 1.0/sqrt( (double) n);
  wtime = omp_get_wtime ( );
  // Allocate space for the vectors to be used
  a = new double [n]; b = new double [n]; c = new double [n];
  // Define parallel region
# pragma omp parallel for default(shared) private (angle, i) reduction(+:Norm2)
  // Set up values for vectors  a and b
  for (i = 0; i < n; i++){
      angle = 2.0*M_PI*i/ (( double ) n);
      a[i] = s*(sin(angle) + cos(angle));
      b[i] =  s*sin(2.0*angle);
      c[i] = 0.0;
  }
  // Then perform the vector addition
  for (i = 0; i < n; i++){
     c[i] += a[i]+b[i];
  }
  // Compute now the norm-2
  Norm2 = 0.0;
  for (i = 0; i < n; i++){
     Norm2  += c[i]*c[i];
  }
// end parallel region
  wtime = omp_get_wtime ( ) - wtime;
  cout << setiosflags(ios::showpoint | ios::uppercase);
  cout << setprecision(10) << setw(20) << "Time used  for norm-2 computation=" << wtime  << endl;
  cout << " Norm-2  = " << Norm2 << endl;
  // Free up space
  delete[] a;
  delete[] b;
  delete[] c;
  return 0;
}

\end{minted}


\subsection*{\href{{https://github.com/CompPhysics/ComputationalPhysicsMSU/blob/master/doc/Programs/ParallelizationOpenMP/OpenMPmatrixmatrixmult.cpp}}{Matrix-matrix multiplication}}
This the matrix-matrix multiplication code with plain c++ memory allocation using OpenMP
















































































\begin{minted}[fontsize=\fontsize{9pt}{9pt},linenos=false,mathescape,baselinestretch=1.0,fontfamily=tt,xleftmargin=7mm]{text}
//  Matrix-matrix multiplication and Frobenius norm of a matrix with OpenMP
#include <cstdlib>
#include <iostream>
#include <cmath>
#include <iomanip>
#include  <omp.h>
# include <ctime>

using namespace std; // note use of namespace
int main (int argc, char* argv[])
{
  // read in dimension of square matrix
  int n = atoi(argv[1]);
  double **A, **B, **C;
  int i, j, k;
  int thread_num;
  double wtime, Fsum, s, angle;
  cout << "  Compute matrix product C = A * B and Frobenius norm." << endl;
  omp_set_num_threads(4);
  thread_num = omp_get_max_threads ();
  cout << "  The number of processors available = " << omp_get_num_procs () << endl ;
  cout << "  The number of threads available    = " << thread_num <<  endl;
  cout << "  The matrix order n                 = " << n << endl;

  s = 1.0/sqrt( (double) n);
  wtime = omp_get_wtime ( );
  // Allocate space for the two matrices
  A = new double*[n]; B = new double*[n]; C = new double*[n];
  for (i = 0; i < n; i++){
    A[i] = new double[n];
    B[i] = new double[n];
    C[i] = new double[n];
  }
  // Define parallel region
# pragma omp parallel for default(shared) private (angle, i, j, k) reduction(+:Fsum)
  // Set up values for matrix A and B and zero matrix C
  for (i = 0; i < n; i++){
    for (j = 0; j < n; j++) {
      angle = 2.0*M_PI*i*j/ (( double ) n);
      A[i][j] = s * ( sin ( angle ) + cos ( angle ) );
      B[j][i] =  A[i][j];
    }
  }
  // Then perform the matrix-matrix multiplication
  for (i = 0; i < n; i++){
    for (j = 0; j < n; j++) {
       C[i][j] =  0.0;    
       for (k = 0; k < n; k++) {
            C[i][j] += A[i][k]*B[k][j];
       }
    }
  }
  // Compute now the Frobenius norm
  Fsum = 0.0;
  for (i = 0; i < n; i++){
    for (j = 0; j < n; j++) {
      Fsum += C[i][j]*C[i][j];
    }
  }
  Fsum = sqrt(Fsum);
// end parallel region and letting only one thread perform I/O
  wtime = omp_get_wtime ( ) - wtime;
  cout << setiosflags(ios::showpoint | ios::uppercase);
  cout << setprecision(10) << setw(20) << "Time used  for matrix-matrix multiplication=" << wtime  << endl;
  cout << "  Frobenius norm  = " << Fsum << endl;
  // Free up space
  for (int i = 0; i < n; i++){
    delete[] A[i];
    delete[] B[i];
    delete[] C[i];
  }
  delete[] A;
  delete[] B;
  delete[] C;
  return 0;
}



\end{minted}



% ------------------- end of main content ---------------

\end{document}



\part{Machine Learning}

\chapter{Neural Networks}

\subsection*{Introduction}

Artificial neural networks are computational systems that can learn to
perform tasks by considering examples, generally without being
programmed with any task-specific rules. It is supposed to mimic a
biological system, wherein neurons interact by sending signals in the
form of mathematical functions between layers. All layers can contain
an arbitrary number of neurons, and each connection is represented by
a weight variable.

The field of artificial neural networks has a long history of
development, and is closely connected with the advancement of computer
science and computers in general. A model of artificial neurons was
first developed by McCulloch and Pitts in 1943 to study signal
processing in the brain and has later been refined by others. The
general idea is to mimic neural networks in the human brain, which is
composed of billions of neurons that communicate with each other by
sending electrical signals.  Each neuron accumulates its incoming
signals, which must exceed an activation threshold to yield an
output. If the threshold is not overcome, the neuron remains inactive,
i.e.~has zero output.

This behaviour has inspired a simple mathematical model for an artificial neuron.

\begin{equation}
 y = f\left(\sum_{i=1}^n w_ix_i\right) = f(u)
 \label{artificialNeuron}
\end{equation}
Here, the output $y$ of the neuron is the value of its activation function, which have as input
a weighted sum of signals $x_i, \dots ,x_n$ received by $n$ other neurons.

Conceptually, it is helpful to divide neural networks into four
categories:
\begin{enumerate}
\item general purpose neural networks for supervised learning,

\item neural networks designed specifically for image processing, the most prominent example of this class being Convolutional Neural Networks (CNNs),

\item neural networks for sequential data such as Recurrent Neural Networks (RNNs), and

\item neural networks for unsupervised learning such as Deep Boltzmann Machines.
\end{enumerate}

\noindent
In natural science, DNNs and CNNs have already found numerous
applications. In statistical physics, they have been applied to detect
phase transitions in 2D Ising and Potts models, lattice gauge
theories, and different phases of polymers, or solving the
Navier-Stokes equation in weather forecasting.  Deep learning has also
found interesting applications in quantum physics. Various quantum
phase transitions can be detected and studied using DNNs and CNNs,
topological phases, and even non-equilibrium many-body
localization. Representing quantum states as DNNs quantum state
tomography are among some of the impressive achievements to reveal the
potential of DNNs to facilitate the study of quantum systems.

In quantum information theory, it has been shown that one can perform
gate decompositions with the help of neural. 

The applications are not limited to the natural sciences. There is a
plethora of applications in essentially all disciplines, from the
humanities to life science and medicine.

An artificial neural network (ANN), is a computational model that
consists of layers of connected neurons, or nodes or units.  We will
refer to these interchangeably as units or nodes, and sometimes as
neurons.

It is supposed to mimic a biological nervous system by letting each
neuron interact with other neurons by sending signals in the form of
mathematical functions between layers.  A wide variety of different
ANNs have been developed, but most of them consist of an input layer,
an output layer and eventual layers in-between, called \emph{hidden
layers}. All layers can contain an arbitrary number of nodes, and each
connection between two nodes is associated with a weight variable.

Neural networks (also called neural nets) are neural-inspired
nonlinear models for supervised learning.  As we will see, neural nets
can be viewed as natural, more powerful extensions of supervised
learning methods such as linear and logistic regression and soft-max
methods we discussed earlier.

\paragraph{Feed-forward neural networks.}
The feed-forward neural network (FFNN) was the first and simplest type
of ANNs that were devised. In this network, the information moves in
only one direction: forward through the layers.

Nodes are represented by circles, while the arrows display the
connections between the nodes, including the direction of information
flow. Additionally, each arrow corresponds to a weight variable
(figure to come).  We observe that each node in a layer is connected
to \emph{all} nodes in the subsequent layer, making this a so-called
\emph{fully-connected} FFNN.

\paragraph{Convolutional Neural Network.}
A different variant of FFNNs are \emph{convolutional neural networks}
(CNNs), which have a connectivity pattern inspired by the animal
visual cortex. Individual neurons in the visual cortex only respond to
stimuli from small sub-regions of the visual field, called a receptive
field. This makes the neurons well-suited to exploit the strong
spatially local correlation present in natural images. The response of
each neuron can be approximated mathematically as a convolution
operation.  (figure to come)

Convolutional neural networks emulate the behaviour of neurons in the
visual cortex by enforcing a \emph{local} connectivity pattern between
nodes of adjacent layers: Each node in a convolutional layer is
connected only to a subset of the nodes in the previous layer, in
contrast to the fully-connected FFNN.  Often, CNNs consist of several
convolutional layers that learn local features of the input, with a
fully-connected layer at the end, which gathers all the local data and
produces the outputs. They have wide applications in image and video
recognition.

\paragraph{Recurrent neural networks.}
So far we have only mentioned ANNs where information flows in one
direction: forward. \emph{Recurrent neural networks} on the other hand,
have connections between nodes that form directed \emph{cycles}. This
creates a form of internal memory which are able to capture
information on what has been calculated before; the output is
dependent on the previous computations. Recurrent NNs make use of
sequential information by performing the same task for every element
in a sequence, where each element depends on previous elements. An
example of such information is sentences, making recurrent NNs
especially well-suited for handwriting and speech recognition.

\paragraph{Other types of networks.}
There are many other kinds of ANNs that have been developed. One type
that is specifically designed for interpolation in multidimensional
space is the radial basis function (RBF) network. RBFs are typically
made up of three layers: an input layer, a hidden layer with
non-linear radial symmetric activation functions and a linear output
layer (''linear'' here means that each node in the output layer has a
linear activation function). The layers are normally fully-connected
and there are no cycles, thus RBFs can be viewed as a type of
fully-connected FFNN. They are however usually treated as a separate
type of NN due the unusual activation functions.

\subsection*{Multilayer perceptrons}

One uses often so-called fully-connected feed-forward neural networks
with three or more layers (an input layer, one or more hidden layers
and an output layer) consisting of neurons that have non-linear
activation functions.

Such networks are often called \emph{multilayer perceptrons} (MLPs).

According to the \emph{Universal approximation theorem}, a feed-forward
neural network with just a single hidden layer containing a finite
number of neurons can approximate a continuous multidimensional
function to arbitrary accuracy, assuming the activation function for
the hidden layer is a \textbf{non-constant, bounded and
monotonically-increasing continuous function}.

Note that the requirements on the activation function only applies to
the hidden layer, the output nodes are always assumed to be linear, so
as to not restrict the range of output values.

The output $y$ is produced via the activation function $f$
\[
 y = f\left(\sum_{i=1}^n w_ix_i + b_i\right) = f(z),
\]
This function receives $x_i$ as inputs.
Here the activation $z=(\sum_{i=1}^n w_ix_i+b_i)$. 
In an FFNN of such neurons, the \emph{inputs} $x_i$ are the \emph{outputs} of
the neurons in the preceding layer. Furthermore, an MLP is
fully-connected, which means that each neuron receives a weighted sum
of the outputs of \emph{all} neurons in the previous layer.

First, for each node $i$ in the first hidden layer, we calculate a weighted sum $z_i^1$ of the input coordinates $x_j$,

\begin{equation} z_i^1 = \sum_{j=1}^{M} w_{ij}^1 x_j + b_i^1
\end{equation}

Here $b_i$ is the so-called bias which is normally needed in
case of zero activation weights or inputs. How to fix the biases and
the weights will be discussed below.  The value of $z_i^1$ is the
argument to the activation function $f_i$ of each node $i$, The
variable $M$ stands for all possible inputs to a given node $i$ in the
first layer.  We define  the output $y_i^1$ of all neurons in layer 1 as

\begin{equation}
 y_i^1 = f(z_i^1) = f\left(\sum_{j=1}^M w_{ij}^1 x_j  + b_i^1\right)
 \label{outputLayer1}
\end{equation}

where we assume that all nodes in the same layer have identical
activation functions, hence the notation $f$. In general, we could assume in the more general case that different layers have different activation functions.
In this case we would identify these functions with a superscript $l$ for the $l$-th layer,

\begin{equation}
 y_i^l = f^l(u_i^l) = f^l\left(\sum_{j=1}^{N_{l-1}} w_{ij}^l y_j^{l-1} + b_i^l\right)
 \label{generalLayer}
\end{equation}

where $N_l$ is the number of nodes in layer $l$. When the output of
all the nodes in the first hidden layer are computed, the values of
the subsequent layer can be calculated and so forth until the output
is obtained.

The output of neuron $i$ in layer 2 is thus,

\begin{align}
 y_i^2 &= f^2\left(\sum_{j=1}^N w_{ij}^2 y_j^1 + b_i^2\right) \\
 &= f^2\left[\sum_{j=1}^N w_{ij}^2f^1\left(\sum_{k=1}^M w_{jk}^1 x_k + b_j^1\right) + b_i^2\right]
 \label{outputLayer2}
\end{align}
where we have substituted $y_k^1$ with the inputs $x_k$. Finally, the ANN output reads

\begin{align}
 y_i^3 &= f^3\left(\sum_{j=1}^N w_{ij}^3 y_j^2 + b_i^3\right) \\
 &= f_3\left[\sum_{j} w_{ij}^3 f^2\left(\sum_{k} w_{jk}^2 f^1\left(\sum_{m} w_{km}^1 x_m + b_k^1\right) + b_j^2\right)
  + b_1^3\right]
\end{align}

We can generalize this expression to an MLP with $l$ hidden
layers. The complete functional form is,

\begin{align}
&y^{l+1}_i = f^{l+1}\left[\!\sum_{j=1}^{N_l} w_{ij}^3 f^l\left(\sum_{k=1}^{N_{l-1}}w_{jk}^{l-1}\left(\dots f^1\left(\sum_{n=1}^{N_0} w_{mn}^1 x_n+ b_m^1\right)\dots\right)+b_k^2\right)+b_1^3\right] &&
 \label{completeNN}
\end{align}

which illustrates a basic property of MLPs: The only independent
variables are the input values $x_n$.

This confirms that an MLP, despite its quite convoluted mathematical
form, is nothing more than an analytic function, specifically a
mapping of real-valued vectors $\hat{x} \in \mathbb{R}^n \rightarrow
\hat{y} \in \mathbb{R}^m$.

Furthermore, the flexibility and universality of an MLP can be
illustrated by realizing that the expression is essentially a nested
sum of scaled activation functions of the form

\begin{equation}
 f(x) = c_1 f(c_2 x + c_3) + c_4
\end{equation}

where the parameters $c_i$ are weights and biases. By adjusting these
parameters, the activation functions can be shifted up and down or
left and right, change slope or be rescaled which is the key to the
flexibility of a neural network.

We can introduce a more convenient notation for the activations in an A NN. 

Additionally, we can represent the biases and activations
as layer-wise column vectors $\hat{b}_l$ and $\hat{y}_l$, so that the $i$-th element of each vector 
is the bias $b_i^l$ and activation $y_i^l$ of node $i$ in layer $l$ respectively. 

We have that $\mathrm{W}_l$ is an $N_{l-1} \times N_l$ matrix, while $\hat{b}_l$ and $\hat{y}_l$ are $N_l \times 1$ column vectors. 
With this notation, the sum becomes a matrix-vector multiplication, and we can write
the equation for the activations of hidden layer 2 (assuming three nodes for simplicity) as
\begin{equation}
 \hat{y}_2 = f_2(\mathrm{W}_2 \hat{y}_{1} + \hat{b}_{2}) = 
 f_2\left(\left[\begin{array}{ccc}
    w^2_{11} &w^2_{12} &w^2_{13} \\
    w^2_{21} &w^2_{22} &w^2_{23} \\
    w^2_{31} &w^2_{32} &w^2_{33} \\
    \end{array} \right] \cdot
    \left[\begin{array}{c}
           y^1_1 \\
           y^1_2 \\
           y^1_3 \\
          \end{array}\right] + 
    \left[\begin{array}{c}
           b^2_1 \\
           b^2_2 \\
           b^2_3 \\
          \end{array}\right]\right).
\end{equation}

The activation of node $i$ in layer 2 is

\begin{equation}
 y^2_i = f_2\Bigr(w^2_{i1}y^1_1 + w^2_{i2}y^1_2 + w^2_{i3}y^1_3 + b^2_i\Bigr) = 
 f_2\left(\sum_{j=1}^3 w^2_{ij} y_j^1 + b^2_i\right).
\end{equation}

This is not just a convenient and compact notation, but also a useful
and intuitive way to think about MLPs: The output is calculated by a
series of matrix-vector multiplications and vector additions that are
used as input to the activation functions. For each operation
$\mathrm{W}_l \hat{y}_{l-1}$ we move forward one layer.

\paragraph{Activation functions.}
A property that characterizes a neural network, other than its
connectivity, is the choice of activation function(s).  As described
in, the following restrictions are imposed on an activation function
for a FFNN to fulfill the universal approximation theorem

\begin{itemize}
  \item Non-constant

  \item Bounded

  \item Monotonically-increasing

  \item Continuous
\end{itemize}

\noindent
The second requirement excludes all linear functions. Furthermore, in
a MLP with only linear activation functions, each layer simply
performs a linear transformation of its inputs.

Regardless of the number of layers, the output of the NN will be
nothing but a linear function of the inputs. Thus we need to introduce
some kind of non-linearity to the NN to be able to fit non-linear
functions Typical examples are the logistic \emph{Sigmoid}

\[
 f(x) = \frac{1}{1 + e^{-x}},
\]
and the \emph{hyperbolic tangent} function
\[
 f(x) = \tanh(x)
\]

The \emph{sigmoid} function are more biologically plausible because the
output of inactive neurons are zero. Such activation function are
called \emph{one-sided}. However, it has been shown that the hyperbolic
tangent performs better than the sigmoid for training MLPs.  has
become the most popular for \emph{deep neural networks}









































































\begin{minted}[fontsize=\fontsize{9pt}{9pt},linenos=false,mathescape,baselinestretch=1.0,fontfamily=tt,xleftmargin=7mm]{python}
"""The sigmoid function (or the logistic curve) is a 
function that takes any real number, z, and outputs a number (0,1).
It is useful in neural networks for assigning weights on a relative scale.
The value z is the weighted sum of parameters involved in the learning algorithm."""

import numpy
import matplotlib.pyplot as plt
import math as mt

z = numpy.arange(-5, 5, .1)
sigma_fn = numpy.vectorize(lambda z: 1/(1+numpy.exp(-z)))
sigma = sigma_fn(z)

fig = plt.figure()
ax = fig.add_subplot(111)
ax.plot(z, sigma)
ax.set_ylim([-0.1, 1.1])
ax.set_xlim([-5,5])
ax.grid(True)
ax.set_xlabel('z')
ax.set_title('sigmoid function')

plt.show()

"""Step Function"""
z = numpy.arange(-5, 5, .02)
step_fn = numpy.vectorize(lambda z: 1.0 if z >= 0.0 else 0.0)
step = step_fn(z)

fig = plt.figure()
ax = fig.add_subplot(111)
ax.plot(z, step)
ax.set_ylim([-0.5, 1.5])
ax.set_xlim([-5,5])
ax.grid(True)
ax.set_xlabel('z')
ax.set_title('step function')

plt.show()

"""Sine Function"""
z = numpy.arange(-2*mt.pi, 2*mt.pi, 0.1)
t = numpy.sin(z)

fig = plt.figure()
ax = fig.add_subplot(111)
ax.plot(z, t)
ax.set_ylim([-1.0, 1.0])
ax.set_xlim([-2*mt.pi,2*mt.pi])
ax.grid(True)
ax.set_xlabel('z')
ax.set_title('sine function')

plt.show()

"""Plots a graph of the squashing function used by a rectified linear
unit"""
z = numpy.arange(-2, 2, .1)
zero = numpy.zeros(len(z))
y = numpy.max([zero, z], axis=0)

fig = plt.figure()
ax = fig.add_subplot(111)
ax.plot(z, y)
ax.set_ylim([-2.0, 2.0])
ax.set_xlim([-2.0, 2.0])
ax.grid(True)
ax.set_xlabel('z')
ax.set_title('Rectified linear unit')

plt.show()

\end{minted}


\subsection*{The multilayer  perceptron (MLP)}

The multilayer perceptron is a very popular, and easy to implement approach, to deep learning. It consists of
\begin{enumerate}
\item A neural network with one or more layers of nodes between the input and the output nodes.

\item The multilayer network structure, or architecture, or topology, consists of an input layer, one or more hidden layers, and one output layer.

\item The input nodes pass values to the first hidden layer, its nodes pass the information on to the second and so on till we reach the output layer.
\end{enumerate}

\noindent
As a convention it is normal to call  a  network with one layer of input units, one layer of hidden
units and one layer of output units as  a two-layer network. A network with two layers of hidden units is called a three-layer network etc etc.

For an MLP network there is no direct connection between the output nodes/neurons/units and  the input nodes/neurons/units.
Hereafter we will call the various entities of a layer for nodes.
There are also no connections within a single layer.

The number of input nodes does not need to equal the number of output
nodes. This applies also to the hidden layers. Each layer may have its
own number of nodes and activation functions.

The hidden layers have their name from the fact that they are not
linked to observables and as we will see below when we define the
so-called activation $\hat{z}$, we can think of this as a basis
expansion of the original inputs $\hat{x}$. The difference however
between neural networks and say linear regression is that now these
basis functions (which will correspond to the weights in the network)
are learned from data. This results in  an important difference between
neural networks and deep learning approaches on one side and methods
like logistic regression or linear regression and their modifications on the other side.

\paragraph{From one to many layers, the universal approximation theorem.}
A neural network with only one layer, what we called the simple
perceptron, is best suited if we have a standard binary model with
clear (linear) boundaries between the outcomes. As such it could
equally well be replaced by standard linear regression or logistic
regression. Networks with one or more hidden layers approximate
systems with more complex boundaries.

As stated earlier, 
an important theorem in studies of neural networks, restated without
proof here, is the \href{{http://citeseerx.ist.psu.edu/viewdoc/download?doi=10.1.1.441.7873&rep=rep1&type=pdf}}{universal approximation
theorem}.

It states that a feed-forward network with a single hidden layer
containing a finite number of neurons can approximate continuous
functions on compact subsets of real functions. The theorem thus
states that simple neural networks can represent a wide variety of
interesting functions when given appropriate parameters. It is the
multilayer feedforward architecture itself which gives neural networks
the potential of being universal approximators.

\subsection*{Deriving the back propagation code for a multilayer perceptron model}

As we have seen now in a feed forward network, we can express the final output of our network in terms of basic matrix-vector multiplications.
The unknowwn quantities are our weights $w_{ij}$ and we need to find an algorithm for changing them so that our errors are as small as possible.
This leads us to the famous \href{{https://www.nature.com/articles/323533a0}}{back propagation algorithm}.

The questions we want to ask are how do changes in the biases and the
weights in our network change the cost function and how can we use the
final output to modify the weights?

To derive these equations let us start with a plain regression problem
and define our cost function as

\[
{\cal C}(\hat{W})  =  \frac{1}{2}\sum_{i=1}^n\left(y_i - t_i\right)^2, 
\]

where the $t_i$s are our $n$ targets (the values we want to
reproduce), while the outputs of the network after having propagated
all inputs $\hat{x}$ are given by $y_i$.  Below we will demonstrate
how the basic equations arising from the back propagation algorithm
can be modified in order to study classification problems with $K$
classes.

With our definition of the targets $\hat{t}$, the outputs of the
network $\hat{y}$ and the inputs $\hat{x}$ we
define now the activation $z_j^l$ of node/neuron/unit $j$ of the
$l$-th layer as a function of the bias, the weights which add up from
the previous layer $l-1$ and the forward passes/outputs
$\hat{a}^{l-1}$ from the previous layer as

\[
z_j^l = \sum_{i=1}^{M_{l-1}}w_{ij}^la_i^{l-1}+b_j^l,
\]

where $b_k^l$ are the biases from layer $l$.  Here $M_{l-1}$
represents the total number of nodes/neurons/units of layer $l-1$. The
figure here illustrates this equation.  We can rewrite this in a more
compact form as the matrix-vector products we discussed earlier,

\[
\hat{z}^l = \left(\hat{W}^l\right)^T\hat{a}^{l-1}+\hat{b}^l.
\]

With the activation values $\hat{z}^l$ we can in turn define the
output of layer $l$ as $\hat{a}^l = f(\hat{z}^l)$ where $f$ is our
activation function. In the examples here we will use the sigmoid
function discussed in our logistic regression lectures. We will also use the same activation function $f$ for all layers
and their nodes.  It means we have

\[
a_j^l = f(z_j^l) = \frac{1}{1+\exp{-(z_j^l)}}.
\]

From the definition of the activation $z_j^l$ we have
\[
\frac{\partial z_j^l}{\partial w_{ij}^l} = a_i^{l-1},
\]
and
\[
\frac{\partial z_j^l}{\partial a_i^{l-1}} = w_{ji}^l. 
\]

With our definition of the activation function we have that (note that this function depends only on $z_j^l$)
\[
\frac{\partial a_j^l}{\partial z_j^{l}} = a_j^l(1-a_j^l)=f(z_j^l)(1-f(z_j^l)). 
\]

With these definitions we can now compute the derivative of the cost function in terms of the weights.

Let us specialize to the output layer $l=L$. Our cost function is
\[
{\cal C}(\hat{W^L})  =  \frac{1}{2}\sum_{i=1}^n\left(y_i - t_i\right)^2=\frac{1}{2}\sum_{i=1}^n\left(a_i^L - t_i\right)^2, 
\]
The derivative of this function with respect to the weights is

\[
\frac{\partial{\cal C}(\hat{W^L})}{\partial w_{jk}^L}  =  \left(a_j^L - t_j\right)\frac{\partial a_j^L}{\partial w_{jk}^{L}}, 
\]
The last partial derivative can easily be computed and reads (by applying the chain rule)
\[
\frac{\partial a_j^L}{\partial w_{jk}^{L}} = \frac{\partial a_j^L}{\partial z_{j}^{L}}\frac{\partial z_j^L}{\partial w_{jk}^{L}}=a_j^L(1-a_j^L)a_k^{L-1},  
\]

We have thus
\[
\frac{\partial{\cal C}(\hat{W^L})}{\partial w_{jk}^L}  =  \left(a_j^L - t_j\right)a_j^L(1-a_j^L)a_k^{L-1}, 
\]

Defining
\[
\delta_j^L = a_j^L(1-a_j^L)\left(a_j^L - t_j\right) = f'(z_j^L)\frac{\partial {\cal C}}{\partial (a_j^L)},
\]
and using the Hadamard product of two vectors we can write this as
\[
\hat{\delta}^L = f'(\hat{z}^L)\circ\frac{\partial {\cal C}}{\partial (\hat{a}^L)}.
\]

This is an important expression. The second term on the right handside
measures how fast the cost function is changing as a function of the $j$th
output activation.  If, for example, the cost function doesn't depend
much on a particular output node $j$, then $\delta_j^L$ will be small,
which is what we would expect. The first term on the right, measures
how fast the activation function $f$ is changing at a given activation
value $z_j^L$.

Notice that everything in the above equations is easily computed.  In
particular, we compute $z_j^L$ while computing the behaviour of the
network, and it is only a small additional overhead to compute
$f'(z^L_j)$.  The exact form of the derivative with respect to the
output depends on the form of the cost function.
However, provided the cost function is known there should be little
trouble in calculating

\[
\frac{\partial {\cal C}}{\partial (a_j^L)}
\]

With the definition of $\delta_j^L$ we have a more compact definition of the derivative of the cost function in terms of the weights, namely
\[
\frac{\partial{\cal C}(\hat{W^L})}{\partial w_{jk}^L}  =  \delta_j^La_k^{L-1}.
\]

It is also easy to see that our previous equation can be written as

\[
\delta_j^L =\frac{\partial {\cal C}}{\partial z_j^L}= \frac{\partial {\cal C}}{\partial a_j^L}\frac{\partial a_j^L}{\partial z_j^L},
\]
which can also be interpreted as the partial derivative of the cost function with respect to the biases $b_j^L$, namely
\[
\delta_j^L = \frac{\partial {\cal C}}{\partial b_j^L}\frac{\partial b_j^L}{\partial z_j^L}=\frac{\partial {\cal C}}{\partial b_j^L},
\]
That is, the error $\delta_j^L$ is exactly equal to the rate of change of the cost function as a function of the bias. 

We have now three equations that are essential for the computations of the derivatives of the cost function at the output layer. These equations are needed to start the algorithm and they are


% --- begin paragraph admon ---
\paragraph{The starting equations.}

\begin{equation}
\frac{\partial{\cal C}(\hat{W^L})}{\partial w_{jk}^L}  =  \delta_j^La_k^{L-1},
\end{equation}
and
\begin{equation}
\delta_j^L = f'(z_j^L)\frac{\partial {\cal C}}{\partial (a_j^L)},
\end{equation}
and

\begin{equation}
\delta_j^L = \frac{\partial {\cal C}}{\partial b_j^L},
\end{equation}
% --- end paragraph admon ---



An interesting consequence of the above equations is that when the
activation $a_k^{L-1}$ is small, the gradient term, that is the
derivative of the cost function with respect to the weights, will also
tend to be small. We say then that the weight learns slowly, meaning
that it changes slowly when we minimize the weights via say gradient
descent. In this case we say the system learns slowly.

Another interesting feature is that is when the activation function,
represented by the sigmoid function here, is rather flat when we move towards
its end values $0$ and $1$ (see the above Python codes). In these
cases, the derivatives of the activation function will also be close
to zero, meaning again that the gradients will be small and the
network learns slowly again.

We need a fourth equation and we are set. We are going to propagate
backwards in order to the determine the weights and biases. In order
to do so we need to represent the error in the layer before the final
one $L-1$ in terms of the errors in the final output layer.

We have that (replacing $L$ with a general layer $l$)
\[
\delta_j^l =\frac{\partial {\cal C}}{\partial z_j^l}.
\]
We want to express this in terms of the equations for layer $l+1$. Using the chain rule and summing over all $k$ entries we have

\[
\delta_j^l =\sum_k \frac{\partial {\cal C}}{\partial z_k^{l+1}}\frac{\partial z_k^{l+1}}{\partial z_j^{l}}=\sum_k \delta_k^{l+1}\frac{\partial z_k^{l+1}}{\partial z_j^{l}},
\]
and recalling that
\[
z_j^{l+1} = \sum_{i=1}^{M_{l}}w_{ij}^{l+1}a_i^{l}+b_j^{l+1},
\]
with $M_l$ being the number of nodes in layer $l$, we obtain
\[
\delta_j^l =\sum_k \delta_k^{l+1}w_{kj}^{l+1}f'(z_j^l),
\]
This is our final equation.

We are now ready to set up the algorithm for back propagation and learning the weights and biases.

\subsection*{Setting up the Back propagation algorithm}

The four equations  provide us with a way of computing the gradient of the cost function. Let us write this out in the form of an algorithm.


% --- begin paragraph admon ---
\paragraph{}
First, we set up the input data $\hat{x}$ and the activations
$\hat{z}_1$ of the input layer and compute the activation function and
the pertinent outputs $\hat{a}^1$.
% --- end paragraph admon ---




% --- begin paragraph admon ---
\paragraph{}
Secondly, we perform then the feed forward till we reach the output
layer and compute all $\hat{z}_l$ of the input layer and compute the
activation function and the pertinent outputs $\hat{a}^l$ for
$l=2,3,\dots,L$.
% --- end paragraph admon ---




% --- begin paragraph admon ---
\paragraph{}
Thereafter we compute the ouput error $\hat{\delta}^L$ by computing all
\[
\delta_j^L = f'(z_j^L)\frac{\partial {\cal C}}{\partial (a_j^L)}.
\]
% --- end paragraph admon ---




% --- begin paragraph admon ---
\paragraph{}
Then we compute the back propagate error for each $l=L-1,L-2,\dots,2$ as
\[
\delta_j^l = \sum_k \delta_k^{l+1}w_{kj}^{l+1}f'(z_j^l).
\]
% --- end paragraph admon ---




% --- begin paragraph admon ---
\paragraph{}
Finally, we update the weights and the biases using gradient descent for each $l=L-1,L-2,\dots,2$ and update the weights and biases according to the rules
\[
w_{jk}^l\leftarrow  = w_{jk}^l- \eta \delta_j^la_k^{l-1},
\]

\[
b_j^l \leftarrow b_j^l-\eta \frac{\partial {\cal C}}{\partial b_j^l}=b_j^l-\eta \delta_j^l,
\]
% --- end paragraph admon ---



The parameter $\eta$ is the learning parameter discussed in connection with the gradient descent methods.
Here it is convenient to use stochastic gradient descent (see the examples below) with mini-batches with an outer loop that steps through multiple epochs of training.

\subsection*{Setting up a Multi-layer perceptron model for classification}

We are now gong to develop an example based on the MNIST data
base. This is a classification problem and we need to use our
cross-entropy function we discussed in connection with logistic
regression. The cross-entropy defines our cost function for the
classificaton problems with neural networks.

In binary classification with two classes $(0, 1)$ we define the
logistic/sigmoid function as the probability that a particular input
is in class $0$ or $1$.  This is possible because the logistic
function takes any input from the real numbers and inputs a number
between 0 and 1, and can therefore be interpreted as a probability. It
also has other nice properties, such as a derivative that is simple to
calculate.

For an input $\boldsymbol{a}$ from the hidden layer, the probability that the input $\boldsymbol{x}$
is in class 0 or 1 is just. We let $\theta$ represent the unknown weights and biases to be adjusted by our equations). The variable $x$
represents our activation values $z$. We have
\[
P(y = 0 \mid \hat{x}, \hat{\theta}) = \frac{1}{1 + \exp{(- \hat{x}})} ,
\]
and
\[
P(y = 1 \mid \hat{x}, \hat{\theta}) = 1 - P(y = 0 \mid \hat{x}, \hat{\theta}) ,
\]

where $y \in \{0, 1\}$  and $\hat{\theta}$ represents the weights and biases
of our network.


Our cost function is given as (see the Logistic regression lectures)
\[
\mathcal{C}(\hat{\theta}) = - \ln P(\mathcal{D} \mid \hat{\theta}) = - \sum_{i=1}^n
y_i \ln[P(y_i = 0)] + (1 - y_i) \ln [1 - P(y_i = 0)] = \sum_{i=1}^n \mathcal{L}_i(\hat{\theta}) .
\]

This last equality means that we can interpret our \emph{cost} function as a sum over the \emph{loss} function
for each point in the dataset $\mathcal{L}_i(\hat{\theta})$.  
The negative sign is just so that we can think about our algorithm as minimizing a positive number, rather
than maximizing a negative number.  

In \emph{multiclass} classification it is common to treat each integer label as a so called \emph{one-hot} vector:  

$y = 5 \quad \rightarrow \quad \hat{y} = (0, 0, 0, 0, 0, 1, 0, 0, 0, 0) ,$ and

$y = 1 \quad \rightarrow \quad \hat{y} = (0, 1, 0, 0, 0, 0, 0, 0, 0, 0) ,$ 


i.e.~a binary bit string of length $C$, where $C = 10$ is the number of classes in the MNIST dataset (numbers from $0$ to $9$)..  

If $\hat{x}_i$ is the $i$-th input (image), $y_{ic}$ refers to the $c$-th component of the $i$-th
output vector $\hat{y}_i$.  
The probability of $\hat{x}_i$ being in class $c$ will be given by the softmax function:  

\[
P(y_{ic} = 1 \mid \hat{x}_i, \hat{\theta}) = \frac{\exp{((\hat{a}_i^{hidden})^T \hat{w}_c)}}
{\sum_{c'=0}^{C-1} \exp{((\hat{a}_i^{hidden})^T \hat{w}_{c'})}} ,
\]

which reduces to the logistic function in the binary case.  
The likelihood of this $C$-class classifier
is now given as:  

\[
P(\mathcal{D} \mid \hat{\theta}) = \prod_{i=1}^n \prod_{c=0}^{C-1} [P(y_{ic} = 1)]^{y_{ic}} .
\]
Again we take the negative log-likelihood to define our cost function:  

\[
\mathcal{C}(\hat{\theta}) = - \log{P(\mathcal{D} \mid \hat{\theta})}.
\]
See the logistic regression lectures for a full definition of the cost function.

The back propagation equations need now only a small change, namely the definition of a new cost function. We are thus ready to use the same equations as before!

\paragraph{Example: binary classification problem.}
As an example of the above, relevant for project 2 as well, let us consider a binary class. As discussed in our logistic regression lectures, we defined a cost function in terms of the parameters $\beta$ as
\[
\mathcal{C}(\hat{\beta}) = - \sum_{i=1}^n \left(y_i\log{p(y_i \vert x_i,\hat{\beta})}+(1-y_i)\log{1-p(y_i \vert x_i,\hat{\beta})}\right),
\]
where we had defined the logistic (sigmoid) function
\[
p(y_i =1\vert x_i,\hat{\beta})=\frac{\exp{(\beta_0+\beta_1 x_i)}}{1+\exp{(\beta_0+\beta_1 x_i)}},
\]
and
\[
p(y_i =0\vert x_i,\hat{\beta})=1-p(y_i =1\vert x_i,\hat{\beta}).
\]
The parameters $\hat{\beta}$ were defined using a minimization method like gradient descent or Newton-Raphson's method. 

Now we replace $x_i$ with the activation $z_i^l$ for a given layer $l$ and the outputs as $y_i=a_i^l=f(z_i^l)$, with $z_i^l$ now being a function of the weights $w_{ij}^l$ and biases $b_i^l$. 
We have then
\[
a_i^l = y_i = \frac{\exp{(z_i^l)}}{1+\exp{(z_i^l)}},
\]
with 
\[
z_i^l = \sum_{j}w_{ij}^l a_j^{l-1}+b_i^l,
\]
where the superscript $l-1$ indicates that these are the outputs from layer $l-1$.
Our cost function at the final layer $l=L$ is now
\[
\mathcal{C}(\hat{W}) = - \sum_{i=1}^n \left(t_i\log{a_i^L}+(1-t_i)\log{(1-a_i^L)}\right),
\]
where we have defined the targets $t_i$. The derivatives of the cost function with respect to the output $a_i^L$ are then easily calculated and we get
\[
\frac{\partial \mathcal{C}(\hat{W})}{\partial a_i^L} = \frac{a_i^L-t_i}{a_i^L(1-a_i^L)}. 
\]
In case we use another activation function than the logistic one, we need to evaluate other derivatives. 

\paragraph{The Softmax function.}
In case we employ the more general case given by the Softmax equation, we need to evaluate the derivative of the activation function with respect to the activation $z_i^l$, that is we need
\[
\frac{\partial f(z_i^l)}{\partial w_{jk}^l} =
\frac{\partial f(z_i^l)}{\partial z_j^l} \frac{\partial z_j^l}{\partial w_{jk}^l}= \frac{\partial f(z_i^l)}{\partial z_j^l}a_k^{l-1}.
\]
For the Softmax function we have
\[
f(z_i^l) = \frac{\exp{(z_i^l)}}{\sum_{m=1}^K\exp{(z_m^l)}}.
\]
Its derivative with respect to $z_j^l$ gives 
\[
\frac{\partial f(z_i^l)}{\partial z_j^l}= f(z_i^l)\left(\delta_{ij}-f(z_j^l)\right), 
\]
which in case of the simply binary model reduces to  having $i=j$. 

\paragraph{Developing a code for doing neural networks with back propagation.}
One can identify a set of key steps when using neural networks to solve supervised learning problems:  

\begin{enumerate}
\item Collect and pre-process data  

\item Define model and architecture  

\item Choose cost function and optimizer  

\item Train the model  

\item Evaluate model performance on test data  

\item Adjust hyperparameters (if necessary, network architecture)
\end{enumerate}

\noindent
\paragraph{Collect and pre-process data.}
Here we will be using the MNIST dataset, which is readily available through the \textbf{scikit-learn}
package. You may also find it for example \href{{http://yann.lecun.com/exdb/mnist/}}{here}.  
The \emph{MNIST} (Modified National Institute of Standards and Technology) database is a large database
of handwritten digits that is commonly used for training various image processing systems.  
The MNIST dataset consists of 70 000 images of size $28\times 28$ pixels, each labeled from 0 to 9.  
The scikit-learn dataset we will use consists of a selection of 1797 images of size $8\times 8$ collected and processed from this database.  

To feed data into a feed-forward neural network we need to represent
the inputs as a design/feature matrix $X = (n_{inputs}, n_{features})$.  Each
row represents an \emph{input}, in this case a handwritten digit, and
each column represents a \emph{feature}, in this case a pixel.  The
correct answers, also known as \emph{labels} or \emph{targets} are
represented as a 1D array of integers 
$Y = (n_{inputs}) = (5, 3, 1, 8,...)$.

As an example, say we want to build a neural network using supervised learning to predict Body-Mass Index (BMI) from
measurements of height (in m)  
and weight (in kg). If we have measurements of 5 people the design/feature matrix could be for example:  

$$ X = \begin{bmatrix}
1.85 {\&} 81\\
1.71 {\&} 65\\
1.95 {\&} 103\\
1.55 {\&} 42\\
1.63 {\&} 56
\end{bmatrix} ,$$  

and the targets would be:  

$$ Y = (23.7, 22.2, 27.1, 17.5, 21.1) $$  

Since each input image is a 2D matrix, we need to flatten the image
(i.e. "unravel" the 2D matrix into a 1D array) to turn the data into a
design/feature matrix. This means we lose all spatial information in the
image, such as locality and translational invariance. More complicated
architectures such as Convolutional Neural Networks can take advantage
of such information, and are most commonly applied when analyzing
images.












































\begin{minted}[fontsize=\fontsize{9pt}{9pt},linenos=false,mathescape,baselinestretch=1.0,fontfamily=tt,xleftmargin=7mm]{python}
# import necessary packages
import numpy as np
import matplotlib.pyplot as plt
from sklearn import datasets


# ensure the same random numbers appear every time
np.random.seed(0)

# display images in notebook
%matplotlib inline
plt.rcParams['figure.figsize'] = (12,12)


# download MNIST dataset
digits = datasets.load_digits()

# define inputs and labels
inputs = digits.images
labels = digits.target

print("inputs = (n_inputs, pixel_width, pixel_height) = " + str(inputs.shape))
print("labels = (n_inputs) = " + str(labels.shape))


# flatten the image
# the value -1 means dimension is inferred from the remaining dimensions: 8x8 = 64
n_inputs = len(inputs)
inputs = inputs.reshape(n_inputs, -1)
print("X = (n_inputs, n_features) = " + str(inputs.shape))


# choose some random images to display
indices = np.arange(n_inputs)
random_indices = np.random.choice(indices, size=5)

for i, image in enumerate(digits.images[random_indices]):
    plt.subplot(1, 5, i+1)
    plt.axis('off')
    plt.imshow(image, cmap=plt.cm.gray_r, interpolation='nearest')
    plt.title("Label: %d" % digits.target[random_indices[i]])
plt.show()

\end{minted}


\paragraph{Train and test datasets.}
Performing analysis before partitioning the dataset is a major error, that can lead to incorrect conclusions.  

We will reserve $80 \%$ of our dataset for training and $20 \%$ for testing.  

It is important that the train and test datasets are drawn randomly from our dataset, to ensure
no bias in the sampling.  
Say you are taking measurements of weather data to predict the weather in the coming 5 days.
You don't want to train your model on measurements taken from the hours 00.00 to 12.00, and then test it on data
collected from 12.00 to 24.00.





























\begin{minted}[fontsize=\fontsize{9pt}{9pt},linenos=false,mathescape,baselinestretch=1.0,fontfamily=tt,xleftmargin=7mm]{python}
from sklearn.model_selection import train_test_split

# one-liner from scikit-learn library
train_size = 0.8
test_size = 1 - train_size
X_train, X_test, Y_train, Y_test = train_test_split(inputs, labels, train_size=train_size,
                                                    test_size=test_size)

# equivalently in numpy
def train_test_split_numpy(inputs, labels, train_size, test_size):
    n_inputs = len(inputs)
    inputs_shuffled = inputs.copy()
    labels_shuffled = labels.copy()
    
    np.random.shuffle(inputs_shuffled)
    np.random.shuffle(labels_shuffled)
    
    train_end = int(n_inputs*train_size)
    X_train, X_test = inputs_shuffled[:train_end], inputs_shuffled[train_end:]
    Y_train, Y_test = labels_shuffled[:train_end], labels_shuffled[train_end:]
    
    return X_train, X_test, Y_train, Y_test

#X_train, X_test, Y_train, Y_test = train_test_split_numpy(inputs, labels, train_size, test_size)

print("Number of training images: " + str(len(X_train)))
print("Number of test images: " + str(len(X_test)))

\end{minted}


\paragraph{Define model and architecture.}
Our simple feed-forward neural network will consist of an \emph{input} layer, a single \emph{hidden} layer and an \emph{output} layer. The activation $y$ of each neuron is a weighted sum of inputs, passed through an activation function. In case of the simple perceptron model we have 

$$ z = \sum_{i=1}^n w_i a_i ,$$

$$ y = f(z) ,$$

where $f$ is the activation function, $a_i$ represents input from neuron $i$ in the preceding layer
and $w_i$ is the weight to input $i$.  
The activation of the neurons in the input layer is just the features (e.g.~a pixel value).  

The simplest activation function for a neuron is the \emph{Heaviside} function:

$$ f(z) = 
\begin{cases}
1,  &  z > 0\\
0,  & \text{otherwise}
\end{cases}
$$

A feed-forward neural network with this activation is known as a \emph{perceptron}.  
For a binary classifier (i.e.~two classes, 0 or 1, dog or not-dog) we can also use this in our output layer.  
This activation can be generalized to $k$ classes (using e.g.~the \emph{one-against-all} strategy), 
and we call these architectures \emph{multiclass perceptrons}.  

However, it is now common to use the terms Single Layer Perceptron (SLP) (1 hidden layer) and  
Multilayer Perceptron (MLP) (2 or more hidden layers) to refer to feed-forward neural networks with any activation function.  

Typical choices for activation functions include the sigmoid function, hyperbolic tangent, and Rectified Linear Unit (ReLU).  
We will be using the sigmoid function $\sigma(x)$:  

$$ f(x) = \sigma(x) = \frac{1}{1 + e^{-x}} ,$$

which is inspired by probability theory (see logistic regression) and was most commonly used until about 2011. See the discussion below concerning other activation functions.

\paragraph{Layers.}
\begin{itemize}
\item Input 
\end{itemize}

\noindent
Since each input image has 8x8 = 64 pixels or features, we have an input layer of 64 neurons.  

\begin{itemize}
\item Hidden layer
\end{itemize}

\noindent
We will use 50 neurons in the hidden layer receiving input from the neurons in the input layer.  
Since each neuron in the hidden layer is connected to the 64 inputs we have 64x50 = 3200 weights to the hidden layer.  

\begin{itemize}
\item Output
\end{itemize}

\noindent
If we were building a binary classifier, it would be sufficient with a single neuron in the output layer,
which could output 0 or 1 according to the Heaviside function. This would be an example of a \emph{hard} classifier, meaning it outputs the class of the input directly. However, if we are dealing with noisy data it is often beneficial to use a \emph{soft} classifier, which outputs the probability of being in class 0 or 1.  

For a soft binary classifier, we could use a single neuron and interpret the output as either being the probability of being in class 0 or the probability of being in class 1. Alternatively we could use 2 neurons, and interpret each neuron as the probability of being in each class.  

Since we are doing multiclass classification, with 10 categories, it is natural to use 10 neurons in the output layer. We number the neurons $j = 0,1,...,9$. The activation of each output neuron $j$ will be according to the \emph{softmax} function:  

$$ P(\text{class $j$} \mid \text{input $\hat{a}$}) = \frac{\exp{(\hat{a}^T \hat{w}_j)}}
{\sum_{c=0}^{9} \exp{(\hat{a}^T \hat{w}_c)}} ,$$  

i.e.~each neuron $j$ outputs the probability of being in class $j$ given an input from the hidden layer $\hat{a}$, with $\hat{w}_j$ the weights of neuron $j$ to the inputs.  
The denominator is a normalization factor to ensure the outputs (probabilities) sum up to 1.  
The exponent is just the weighted sum of inputs as before:  

$$ z_j = \sum_{i=1}^n w_ {ij} a_i+b_j.$$  

Since each neuron in the output layer is connected to the 50 inputs from the hidden layer we have 50x10 = 500
weights to the output layer.

\paragraph{Weights and biases.}
Typically weights are initialized with small values distributed around zero, drawn from a uniform
or normal distribution. Setting all weights to zero means all neurons give the same output, making the network useless.  

Adding a bias value to the weighted sum of inputs allows the neural network to represent a greater range
of values. Without it, any input with the value 0 will be mapped to zero (before being passed through the activation). The bias unit has an output of 1, and a weight to each neuron $j$, $b_j$:  

$$ z_j = \sum_{i=1}^n w_ {ij} a_i + b_j.$$  

The bias weights $\hat{b}$ are often initialized to zero, but a small value like $0.01$ ensures all neurons have some output which can be backpropagated in the first training cycle.
















\begin{minted}[fontsize=\fontsize{9pt}{9pt},linenos=false,mathescape,baselinestretch=1.0,fontfamily=tt,xleftmargin=7mm]{python}
# building our neural network

n_inputs, n_features = X_train.shape
n_hidden_neurons = 50
n_categories = 10

# we make the weights normally distributed using numpy.random.randn

# weights and bias in the hidden layer
hidden_weights = np.random.randn(n_features, n_hidden_neurons)
hidden_bias = np.zeros(n_hidden_neurons) + 0.01

# weights and bias in the output layer
output_weights = np.random.randn(n_hidden_neurons, n_categories)
output_bias = np.zeros(n_categories) + 0.01

\end{minted}


\paragraph{Feed-forward pass.}
Denote $F$ the number of features, $H$ the number of hidden neurons and $C$ the number of categories.  
For each input image we calculate a weighted sum of input features (pixel values) to each neuron $j$ in the hidden layer $l$:  

$$ z_{j}^{l} = \sum_{i=1}^{F} w_{ij}^{l} x_i + b_{j}^{l},$$

this is then passed through our activation function  

$$ a_{j}^{l} = f(z_{j}^{l}) .$$  

We calculate a weighted sum of inputs (activations in the hidden layer) to each neuron $j$ in the output layer:  

$$ z_{j}^{L} = \sum_{i=1}^{H} w_{ij}^{L} a_{i}^{l} + b_{j}^{L}.$$  

Finally we calculate the output of neuron $j$ in the output layer using the softmax function:  

$$ a_{j}^{L} = \frac{\exp{(z_j^{L})}}
{\sum_{c=0}^{C-1} \exp{(z_c^{L})}} .$$  

\paragraph{Matrix multiplications.}
Since our data has the dimensions $X = (n_{inputs}, n_{features})$ and our weights to the hidden
layer have the dimensions  
$W_{hidden} = (n_{features}, n_{hidden})$,
we can easily feed the network all our training data in one go by taking the matrix product  

$$ X W^{h} = (n_{inputs}, n_{hidden}),$$ 

and obtain a matrix that holds the weighted sum of inputs to the hidden layer
for each input image and each hidden neuron.    
We also add the bias to obtain a matrix of weighted sums to the hidden layer $Z^{h}$:  

$$ \hat{z}^{l} = \hat{X} \hat{W}^{l} + \hat{b}^{l} ,$$

meaning the same bias (1D array with size equal number of hidden neurons) is added to each input image.  
This is then passed through the activation:  

$$ \hat{a}^{l} = f(\hat{z}^l) .$$  

This is fed to the output layer:  

$$ \hat{z}^{L} = \hat{a}^{L} \hat{W}^{L} + \hat{b}^{L} .$$

Finally we receive our output values for each image and each category by passing it through the softmax function:  

$$ output = softmax (\hat{z}^{L}) = (n_{inputs}, n_{categories}) .$$





































\begin{minted}[fontsize=\fontsize{9pt}{9pt},linenos=false,mathescape,baselinestretch=1.0,fontfamily=tt,xleftmargin=7mm]{python}
# setup the feed-forward pass, subscript h = hidden layer

def sigmoid(x):
    return 1/(1 + np.exp(-x))

def feed_forward(X):
    # weighted sum of inputs to the hidden layer
    z_h = np.matmul(X, hidden_weights) + hidden_bias
    # activation in the hidden layer
    a_h = sigmoid(z_h)
    
    # weighted sum of inputs to the output layer
    z_o = np.matmul(a_h, output_weights) + output_bias
    # softmax output
    # axis 0 holds each input and axis 1 the probabilities of each category
    exp_term = np.exp(z_o)
    probabilities = exp_term / np.sum(exp_term, axis=1, keepdims=True)
    
    return probabilities

probabilities = feed_forward(X_train)
print("probabilities = (n_inputs, n_categories) = " + str(probabilities.shape))
print("probability that image 0 is in category 0,1,2,...,9 = \n" + str(probabilities[0]))
print("probabilities sum up to: " + str(probabilities[0].sum()))
print()

# we obtain a prediction by taking the class with the highest likelihood
def predict(X):
    probabilities = feed_forward(X)
    return np.argmax(probabilities, axis=1)

predictions = predict(X_train)
print("predictions = (n_inputs) = " + str(predictions.shape))
print("prediction for image 0: " + str(predictions[0]))
print("correct label for image 0: " + str(Y_train[0]))

\end{minted}


\paragraph{Choose cost function and optimizer.}
To measure how well our neural network is doing we need to introduce a cost function.  
We will call the function that gives the error of a single sample output the \emph{loss} function, and the function
that gives the total error of our network across all samples the \emph{cost} function.
A typical choice for multiclass classification is the \emph{cross-entropy} loss, also known as the negative log likelihood.  

In \emph{multiclass} classification it is common to treat each integer label as a so called \emph{one-hot} vector:  

$$ y = 5 \quad \rightarrow \quad \hat{y} = (0, 0, 0, 0, 0, 1, 0, 0, 0, 0) ,$$  

$$ y = 1 \quad \rightarrow \quad \hat{y} = (0, 1, 0, 0, 0, 0, 0, 0, 0, 0) ,$$  


i.e.~a binary bit string of length $C$, where $C = 10$ is the number of classes in the MNIST dataset.  

Let $y_{ic}$ denote the $c$-th component of the $i$-th one-hot vector.  
We define the cost function $\mathcal{C}$ as a sum over the cross-entropy loss for each point $\hat{x}_i$ in the dataset.

In the one-hot representation only one of the terms in the loss function is non-zero, namely the
probability of the correct category $c'$  
(i.e.~the category $c'$ such that $y_{ic'} = 1$). This means that the cross entropy loss only punishes you for how wrong
you got the correct label. The probability of category $c$ is given by the softmax function. The vector $\hat{\theta}$ represents the parameters of our network, i.e.~all the weights and biases.  



\paragraph{Optimizing the cost function.}
The network is trained by finding the weights and biases that minimize the cost function. One of the most widely used classes of methods is \emph{gradient descent} and its generalizations. The idea behind gradient descent
is simply to adjust the weights in the direction where the gradient of the cost function is large and negative. This ensures we flow toward a \emph{local} minimum of the cost function.  
Each parameter $\theta$ is iteratively adjusted according to the rule  

$$ \theta_{i+1} = \theta_i - \eta \nabla \mathcal{C}(\theta_i) ,$$

where $\eta$ is known as the \emph{learning rate}, which controls how big a step we take towards the minimum.  
This update can be repeated for any number of iterations, or until we are satisfied with the result.  

A simple and effective improvement is a variant called \emph{Batch Gradient Descent}.  
Instead of calculating the gradient on the whole dataset, we calculate an approximation of the gradient
on a subset of the data called a \emph{minibatch}.  
If there are $N$ data points and we have a minibatch size of $M$, the total number of batches
is $N/M$.  
We denote each minibatch $B_k$, with $k = 1, 2,...,N/M$. The gradient then becomes:  

$$ \nabla \mathcal{C}(\theta) = \frac{1}{N} \sum_{i=1}^N \nabla \mathcal{L}_i(\theta) \quad \rightarrow \quad
\frac{1}{M} \sum_{i \in B_k} \nabla \mathcal{L}_i(\theta) ,$$

i.e.~instead of averaging the loss over the entire dataset, we average over a minibatch.  

This has two important benefits:  
\begin{enumerate}
\item Introducing stochasticity decreases the chance that the algorithm becomes stuck in a local minima.  

\item It significantly speeds up the calculation, since we do not have to use the entire dataset to calculate the gradient.  
\end{enumerate}

\noindent
The various optmization  methods, with codes and algorithms,  are discussed in our lectures on \href{{https://compphysics.github.io/MachineLearning/doc/pub/Splines/html/Splines-bs.html}}{Gradient descent approaches}.

\paragraph{Regularization.}
It is common to add an extra term to the cost function, proportional
to the size of the weights.  This is equivalent to constraining the
size of the weights, so that they do not grow out of control.
Constraining the size of the weights means that the weights cannot
grow arbitrarily large to fit the training data, and in this way
reduces \emph{overfitting}.

We will measure the size of the weights using the so called \emph{L2-norm}, meaning our cost function becomes:  

$$  \mathcal{C}(\theta) = \frac{1}{N} \sum_{i=1}^N \mathcal{L}_i(\theta) \quad \rightarrow \quad
\frac{1}{N} \sum_{i=1}^N  \mathcal{L}_i(\theta) + \lambda \lvert \lvert \hat{w} \rvert \rvert_2^2 
= \frac{1}{N} \sum_{i=1}^N  \mathcal{L}(\theta) + \lambda \sum_{ij} w_{ij}^2,$$  

i.e.~we sum up all the weights squared. The factor $\lambda$ is known as a regularization parameter.

In order to train the model, we need to calculate the derivative of
the cost function with respect to every bias and weight in the
network.  In total our network has $(64 + 1)\times 50=3250$ weights in
the hidden layer and $(50 + 1)\times 10=510$ weights to the output
layer ($+1$ for the bias), and the gradient must be calculated for
every parameter.  We use the \emph{backpropagation} algorithm discussed
above. This is a clever use of the chain rule that allows us to
calculate the gradient efficently. 


\paragraph{Matrix  multiplication.}
To more efficently train our network these equations are implemented using matrix operations.  
The error in the output layer is calculated simply as, with $\hat{t}$ being our targets,  

$$ \delta_L = \hat{t} - \hat{y} = (n_{inputs}, n_{categories}) .$$  

The gradient for the output weights is calculated as  

$$ \nabla W_{L} = \hat{a}^T \delta_L   = (n_{hidden}, n_{categories}) ,$$

where $\hat{a} = (n_{inputs}, n_{hidden})$. This simply means that we are summing up the gradients for each input.  
Since we are going backwards we have to transpose the activation matrix.  

The gradient with respect to the output bias is then  

$$ \nabla \hat{b}_{L} = \sum_{i=1}^{n_{inputs}} \delta_L = (n_{categories}) .$$  

The error in the hidden layer is  

$$ \Delta_h = \delta_L W_{L}^T \circ f'(z_{h}) = \delta_L W_{L}^T \circ a_{h} \circ (1 - a_{h}) = (n_{inputs}, n_{hidden}) ,$$  

where $f'(a_{h})$ is the derivative of the activation in the hidden layer. The matrix products mean
that we are summing up the products for each neuron in the output layer. The symbol $\circ$ denotes
the \emph{Hadamard product}, meaning element-wise multiplication.  

This again gives us the gradients in the hidden layer:  

$$ \nabla W_{h} = X^T \delta_h = (n_{features}, n_{hidden}) ,$$  

$$ \nabla b_{h} = \sum_{i=1}^{n_{inputs}} \delta_h = (n_{hidden}) .$$






































































\begin{minted}[fontsize=\fontsize{9pt}{9pt},linenos=false,mathescape,baselinestretch=1.0,fontfamily=tt,xleftmargin=7mm]{python}
# to categorical turns our integer vector into a onehot representation
from sklearn.metrics import accuracy_score

# one-hot in numpy
def to_categorical_numpy(integer_vector):
    n_inputs = len(integer_vector)
    n_categories = np.max(integer_vector) + 1
    onehot_vector = np.zeros((n_inputs, n_categories))
    onehot_vector[range(n_inputs), integer_vector] = 1
    
    return onehot_vector

#Y_train_onehot, Y_test_onehot = to_categorical(Y_train), to_categorical(Y_test)
Y_train_onehot, Y_test_onehot = to_categorical_numpy(Y_train), to_categorical_numpy(Y_test)

def feed_forward_train(X):
    # weighted sum of inputs to the hidden layer
    z_h = np.matmul(X, hidden_weights) + hidden_bias
    # activation in the hidden layer
    a_h = sigmoid(z_h)
    
    # weighted sum of inputs to the output layer
    z_o = np.matmul(a_h, output_weights) + output_bias
    # softmax output
    # axis 0 holds each input and axis 1 the probabilities of each category
    exp_term = np.exp(z_o)
    probabilities = exp_term / np.sum(exp_term, axis=1, keepdims=True)
    
    # for backpropagation need activations in hidden and output layers
    return a_h, probabilities

def backpropagation(X, Y):
    a_h, probabilities = feed_forward_train(X)
    
    # error in the output layer
    error_output = probabilities - Y
    # error in the hidden layer
    error_hidden = np.matmul(error_output, output_weights.T) * a_h * (1 - a_h)
    
    # gradients for the output layer
    output_weights_gradient = np.matmul(a_h.T, error_output)
    output_bias_gradient = np.sum(error_output, axis=0)
    
    # gradient for the hidden layer
    hidden_weights_gradient = np.matmul(X.T, error_hidden)
    hidden_bias_gradient = np.sum(error_hidden, axis=0)

    return output_weights_gradient, output_bias_gradient, hidden_weights_gradient, hidden_bias_gradient

print("Old accuracy on training data: " + str(accuracy_score(predict(X_train), Y_train)))

eta = 0.01
lmbd = 0.01
for i in range(1000):
    # calculate gradients
    dWo, dBo, dWh, dBh = backpropagation(X_train, Y_train_onehot)
    
    # regularization term gradients
    dWo += lmbd * output_weights
    dWh += lmbd * hidden_weights
    
    # update weights and biases
    output_weights -= eta * dWo
    output_bias -= eta * dBo
    hidden_weights -= eta * dWh
    hidden_bias -= eta * dBh

print("New accuracy on training data: " + str(accuracy_score(predict(X_train), Y_train)))

\end{minted}


\paragraph{Improving performance.}
As we can see the network does not seem to be learning at all. It seems to be just guessing the label for each image.  
In order to obtain a network that does something useful, we will have to do a bit more work.  

The choice of \emph{hyperparameters} such as learning rate and regularization parameter is hugely influential for the performance of the network. Typically a \emph{grid-search} is performed, wherein we test different hyperparameters separated by orders of magnitude. For example we could test the learning rates $\eta = 10^{-6}, 10^{-5},...,10^{-1}$ with different regularization parameters $\lambda = 10^{-6},...,10^{-0}$.  

Next, we haven't implemented minibatching yet, which introduces stochasticity and is though to act as an important regularizer on the weights. We call a feed-forward + backward pass with a minibatch an \emph{iteration}, and a full training period
going through the entire dataset ($n/M$ batches) an \emph{epoch}.

If this does not improve network performance, you may want to consider altering the network architecture, adding more neurons or hidden layers.  
Andrew Ng goes through some of these considerations in this \href{{https://youtu.be/F1ka6a13S9I}}{video}. You can find a summary of the video \href{{https://kevinzakka.github.io/2016/09/26/applying-deep-learning/}}{here}.  


\paragraph{Full object-oriented implementation.}
It is very natural to think of the network as an object, with specific instances of the network
being realizations of this object with different hyperparameters. An implementation using Python classes provides a clean structure and interface, and the full implementation of our neural network is given below.





































































































\begin{minted}[fontsize=\fontsize{9pt}{9pt},linenos=false,mathescape,baselinestretch=1.0,fontfamily=tt,xleftmargin=7mm]{python}
class NeuralNetwork:
    def __init__(
            self,
            X_data,
            Y_data,
            n_hidden_neurons=50,
            n_categories=10,
            epochs=10,
            batch_size=100,
            eta=0.1,
            lmbd=0.0):

        self.X_data_full = X_data
        self.Y_data_full = Y_data

        self.n_inputs = X_data.shape[0]
        self.n_features = X_data.shape[1]
        self.n_hidden_neurons = n_hidden_neurons
        self.n_categories = n_categories

        self.epochs = epochs
        self.batch_size = batch_size
        self.iterations = self.n_inputs // self.batch_size
        self.eta = eta
        self.lmbd = lmbd

        self.create_biases_and_weights()

    def create_biases_and_weights(self):
        self.hidden_weights = np.random.randn(self.n_features, self.n_hidden_neurons)
        self.hidden_bias = np.zeros(self.n_hidden_neurons) + 0.01

        self.output_weights = np.random.randn(self.n_hidden_neurons, self.n_categories)
        self.output_bias = np.zeros(self.n_categories) + 0.01

    def feed_forward(self):
        # feed-forward for training
        self.z_h = np.matmul(self.X_data, self.hidden_weights) + self.hidden_bias
        self.a_h = sigmoid(self.z_h)

        self.z_o = np.matmul(self.a_h, self.output_weights) + self.output_bias

        exp_term = np.exp(self.z_o)
        self.probabilities = exp_term / np.sum(exp_term, axis=1, keepdims=True)

    def feed_forward_out(self, X):
        # feed-forward for output
        z_h = np.matmul(X, self.hidden_weights) + self.hidden_bias
        a_h = sigmoid(z_h)

        z_o = np.matmul(a_h, self.output_weights) + self.output_bias
        
        exp_term = np.exp(z_o)
        probabilities = exp_term / np.sum(exp_term, axis=1, keepdims=True)
        return probabilities

    def backpropagation(self):
        error_output = self.probabilities - self.Y_data
        error_hidden = np.matmul(error_output, self.output_weights.T) * self.a_h * (1 - self.a_h)

        self.output_weights_gradient = np.matmul(self.a_h.T, error_output)
        self.output_bias_gradient = np.sum(error_output, axis=0)

        self.hidden_weights_gradient = np.matmul(self.X_data.T, error_hidden)
        self.hidden_bias_gradient = np.sum(error_hidden, axis=0)

        if self.lmbd > 0.0:
            self.output_weights_gradient += self.lmbd * self.output_weights
            self.hidden_weights_gradient += self.lmbd * self.hidden_weights

        self.output_weights -= self.eta * self.output_weights_gradient
        self.output_bias -= self.eta * self.output_bias_gradient
        self.hidden_weights -= self.eta * self.hidden_weights_gradient
        self.hidden_bias -= self.eta * self.hidden_bias_gradient

    def predict(self, X):
        probabilities = self.feed_forward_out(X)
        return np.argmax(probabilities, axis=1)

    def predict_probabilities(self, X):
        probabilities = self.feed_forward_out(X)
        return probabilities

    def train(self):
        data_indices = np.arange(self.n_inputs)

        for i in range(self.epochs):
            for j in range(self.iterations):
                # pick datapoints with replacement
                chosen_datapoints = np.random.choice(
                    data_indices, size=self.batch_size, replace=False
                )

                # minibatch training data
                self.X_data = self.X_data_full[chosen_datapoints]
                self.Y_data = self.Y_data_full[chosen_datapoints]

                self.feed_forward()
                self.backpropagation()

\end{minted}


\paragraph{Evaluate model performance on test data.}
To measure the performance of our network we evaluate how well it does it data it has never seen before, i.e.~the test data.  
We measure the performance of the network using the \emph{accuracy} score.  
The accuracy is as you would expect just the number of images correctly labeled divided by the total number of images. A perfect classifier will have an accuracy score of $1$.  

$$ \text{Accuracy} = \frac{\sum_{i=1}^n I(\hat{y}_i = y_i)}{n} ,$$  

where $I$ is the indicator function, $1$ if $\hat{y}_i = y_i$ and $0$ otherwise.


















\begin{minted}[fontsize=\fontsize{9pt}{9pt},linenos=false,mathescape,baselinestretch=1.0,fontfamily=tt,xleftmargin=7mm]{python}
epochs = 100
batch_size = 100

dnn = NeuralNetwork(X_train, Y_train_onehot, eta=eta, lmbd=lmbd, epochs=epochs, batch_size=batch_size,
                    n_hidden_neurons=n_hidden_neurons, n_categories=n_categories)
dnn.train()
test_predict = dnn.predict(X_test)

# accuracy score from scikit library
print("Accuracy score on test set: ", accuracy_score(Y_test, test_predict))

# equivalent in numpy
def accuracy_score_numpy(Y_test, Y_pred):
    return np.sum(Y_test == Y_pred) / len(Y_test)

#print("Accuracy score on test set: ", accuracy_score_numpy(Y_test, test_predict))

\end{minted}


\paragraph{Adjust hyperparameters.}
We now perform a grid search to find the optimal hyperparameters for the network.  
Note that we are only using 1 layer with 50 neurons, and human performance is estimated to be around $98\%$ ($2\%$ error rate).






















\begin{minted}[fontsize=\fontsize{9pt}{9pt},linenos=false,mathescape,baselinestretch=1.0,fontfamily=tt,xleftmargin=7mm]{python}
eta_vals = np.logspace(-5, 1, 7)
lmbd_vals = np.logspace(-5, 1, 7)
# store the models for later use
DNN_numpy = np.zeros((len(eta_vals), len(lmbd_vals)), dtype=object)

# grid search
for i, eta in enumerate(eta_vals):
    for j, lmbd in enumerate(lmbd_vals):
        dnn = NeuralNetwork(X_train, Y_train_onehot, eta=eta, lmbd=lmbd, epochs=epochs, batch_size=batch_size,
                            n_hidden_neurons=n_hidden_neurons, n_categories=n_categories)
        dnn.train()
        
        DNN_numpy[i][j] = dnn
        
        test_predict = dnn.predict(X_test)
        
        print("Learning rate  = ", eta)
        print("Lambda = ", lmbd)
        print("Accuracy score on test set: ", accuracy_score(Y_test, test_predict))
        print()

\end{minted}


\paragraph{Visualization.}


































\begin{minted}[fontsize=\fontsize{9pt}{9pt},linenos=false,mathescape,baselinestretch=1.0,fontfamily=tt,xleftmargin=7mm]{python}
# visual representation of grid search
# uses seaborn heatmap, you can also do this with matplotlib imshow
import seaborn as sns

sns.set()

train_accuracy = np.zeros((len(eta_vals), len(lmbd_vals)))
test_accuracy = np.zeros((len(eta_vals), len(lmbd_vals)))

for i in range(len(eta_vals)):
    for j in range(len(lmbd_vals)):
        dnn = DNN_numpy[i][j]
        
        train_pred = dnn.predict(X_train) 
        test_pred = dnn.predict(X_test)

        train_accuracy[i][j] = accuracy_score(Y_train, train_pred)
        test_accuracy[i][j] = accuracy_score(Y_test, test_pred)

        
fig, ax = plt.subplots(figsize = (10, 10))
sns.heatmap(train_accuracy, annot=True, ax=ax, cmap="viridis")
ax.set_title("Training Accuracy")
ax.set_ylabel("$\eta$")
ax.set_xlabel("$\lambda$")
plt.show()

fig, ax = plt.subplots(figsize = (10, 10))
sns.heatmap(test_accuracy, annot=True, ax=ax, cmap="viridis")
ax.set_title("Test Accuracy")
ax.set_ylabel("$\eta$")
ax.set_xlabel("$\lambda$")
plt.show()

\end{minted}


\paragraph{scikit-learn implementation.}
\textbf{scikit-learn} focuses more
on traditional machine learning methods, such as regression,
clustering, decision trees, etc. As such, it has only two types of
neural networks: Multi Layer Perceptron outputting continuous values,
\emph{MPLRegressor}, and Multi Layer Perceptron outputting labels,
\emph{MLPClassifier}. We will see how simple it is to use these classes.

\textbf{scikit-learn} implements a few improvements from our neural network,
such as early stopping, a varying learning rate, different
optimization methods, etc. We would therefore expect a better
performance overall.


















\begin{minted}[fontsize=\fontsize{9pt}{9pt},linenos=false,mathescape,baselinestretch=1.0,fontfamily=tt,xleftmargin=7mm]{python}
from sklearn.neural_network import MLPClassifier
# store models for later use
DNN_scikit = np.zeros((len(eta_vals), len(lmbd_vals)), dtype=object)

for i, eta in enumerate(eta_vals):
    for j, lmbd in enumerate(lmbd_vals):
        dnn = MLPClassifier(hidden_layer_sizes=(n_hidden_neurons), activation='logistic',
                            alpha=lmbd, learning_rate_init=eta, max_iter=epochs)
        dnn.fit(X_train, Y_train)
        
        DNN_scikit[i][j] = dnn
        
        print("Learning rate  = ", eta)
        print("Lambda = ", lmbd)
        print("Accuracy score on test set: ", dnn.score(X_test, Y_test))
        print()

\end{minted}


\paragraph{Visualization.}



































\begin{minted}[fontsize=\fontsize{9pt}{9pt},linenos=false,mathescape,baselinestretch=1.0,fontfamily=tt,xleftmargin=7mm]{python}
# optional
# visual representation of grid search
# uses seaborn heatmap, could probably do this in matplotlib
import seaborn as sns

sns.set()

train_accuracy = np.zeros((len(eta_vals), len(lmbd_vals)))
test_accuracy = np.zeros((len(eta_vals), len(lmbd_vals)))

for i in range(len(eta_vals)):
    for j in range(len(lmbd_vals)):
        dnn = DNN_scikit[i][j]
        
        train_pred = dnn.predict(X_train) 
        test_pred = dnn.predict(X_test)

        train_accuracy[i][j] = accuracy_score(Y_train, train_pred)
        test_accuracy[i][j] = accuracy_score(Y_test, test_pred)

        
fig, ax = plt.subplots(figsize = (10, 10))
sns.heatmap(train_accuracy, annot=True, ax=ax, cmap="viridis")
ax.set_title("Training Accuracy")
ax.set_ylabel("$\eta$")
ax.set_xlabel("$\lambda$")
plt.show()

fig, ax = plt.subplots(figsize = (10, 10))
sns.heatmap(test_accuracy, annot=True, ax=ax, cmap="viridis")
ax.set_title("Test Accuracy")
ax.set_ylabel("$\eta$")
ax.set_xlabel("$\lambda$")
plt.show()

\end{minted}


\subsection*{Building neural networks in Tensorflow and Keras}

Now we want  to build on the experience gained from our neural network implementation in NumPy and scikit-learn
and use it to construct a neural network in Tensorflow. Once we have constructed a neural network in NumPy
and Tensorflow, building one in Keras is really quite trivial, though the performance may suffer.  

In our previous example we used only one hidden layer, and in this we will use two. From this it should be quite
clear how to build one using an arbitrary number of hidden layers, using data structures such as Python lists or
NumPy arrays.

Tensorflow is an open source library machine learning library
developed by the Google Brain team for internal use. It was released
under the Apache 2.0 open source license in November 9, 2015.

Tensorflow is a computational framework that allows you to construct
machine learning models at different levels of abstraction, from
high-level, object-oriented APIs like Keras, down to the C++ kernels
that Tensorflow is built upon. The higher levels of abstraction are
simpler to use, but less flexible, and our choice of implementation
should reflect the problems we are trying to solve.

\href{{https://www.tensorflow.org/guide/graphs}}{Tensorflow uses} so-called graphs to represent your computation
in terms of the dependencies between individual operations, such that you first build a Tensorflow \emph{graph}
to represent your model, and then create a Tensorflow \emph{session} to run the graph.

In this guide we will analyze the same data as we did in our NumPy and
scikit-learn tutorial, gathered from the MNIST database of images. We
will give an introduction to the lower level Python Application
Program Interfaces (APIs), and see how we use them to build our graph.
Then we will build (effectively) the same graph in Keras, to see just
how simple solving a machine learning problem can be.

To install tensorflow on Unix/Linux systems, use pip as


\begin{minted}[fontsize=\fontsize{9pt}{9pt},linenos=false,mathescape,baselinestretch=1.0,fontfamily=tt,xleftmargin=7mm]{python}
pip3 install tensorflow

\end{minted}

and/or if you use \textbf{anaconda}, just write (or install from the graphical user interface)
(current release of CPU-only TensorFlow)



\begin{minted}[fontsize=\fontsize{9pt}{9pt},linenos=false,mathescape,baselinestretch=1.0,fontfamily=tt,xleftmargin=7mm]{python}
conda create -n tf tensorflow
conda activate tf

\end{minted}

To install the current release of GPU TensorFlow



\begin{minted}[fontsize=\fontsize{9pt}{9pt},linenos=false,mathescape,baselinestretch=1.0,fontfamily=tt,xleftmargin=7mm]{python}
conda create -n tf-gpu tensorflow-gpu
conda activate tf-gpu

\end{minted}


Keras is a high level \href{{https://en.wikipedia.org/wiki/Application_programming_interface}}{neural network}
that supports Tensorflow, CTNK and Theano as backends.  
If you have Anaconda installed you may run the following command


\begin{minted}[fontsize=\fontsize{9pt}{9pt},linenos=false,mathescape,baselinestretch=1.0,fontfamily=tt,xleftmargin=7mm]{python}
conda install keras

\end{minted}

You can look up the \href{{https://keras.io/}}{instructions here} for more information.

We will to a large extent use \textbf{keras} in this course. 

Let us look again at the MINST data set.













































\begin{minted}[fontsize=\fontsize{9pt}{9pt},linenos=false,mathescape,baselinestretch=1.0,fontfamily=tt,xleftmargin=7mm]{python}
# import necessary packages
import numpy as np
import matplotlib.pyplot as plt
import tensorflow as tf
from sklearn import datasets


# ensure the same random numbers appear every time
np.random.seed(0)

# display images in notebook
%matplotlib inline
plt.rcParams['figure.figsize'] = (12,12)


# download MNIST dataset
digits = datasets.load_digits()

# define inputs and labels
inputs = digits.images
labels = digits.target

print("inputs = (n_inputs, pixel_width, pixel_height) = " + str(inputs.shape))
print("labels = (n_inputs) = " + str(labels.shape))


# flatten the image
# the value -1 means dimension is inferred from the remaining dimensions: 8x8 = 64
n_inputs = len(inputs)
inputs = inputs.reshape(n_inputs, -1)
print("X = (n_inputs, n_features) = " + str(inputs.shape))


# choose some random images to display
indices = np.arange(n_inputs)
random_indices = np.random.choice(indices, size=5)

for i, image in enumerate(digits.images[random_indices]):
    plt.subplot(1, 5, i+1)
    plt.axis('off')
    plt.imshow(image, cmap=plt.cm.gray_r, interpolation='nearest')
    plt.title("Label: %d" % digits.target[random_indices[i]])
plt.show()

\end{minted}




















\begin{minted}[fontsize=\fontsize{9pt}{9pt},linenos=false,mathescape,baselinestretch=1.0,fontfamily=tt,xleftmargin=7mm]{python}
from tensorflow.keras.layers import Input
from tensorflow.keras.models import Sequential      #This allows appending layers to existing models
from tensorflow.keras.layers import Dense           #This allows defining the characteristics of a particular layer
from tensorflow.keras import optimizers             #This allows using whichever optimiser we want (sgd,adam,RMSprop)
from tensorflow.keras import regularizers           #This allows using whichever regularizer we want (l1,l2,l1_l2)
from tensorflow.keras.utils import to_categorical   #This allows using categorical cross entropy as the cost function

from sklearn.model_selection import train_test_split

# one-hot representation of labels
labels = to_categorical(labels)

# split into train and test data
train_size = 0.8
test_size = 1 - train_size
X_train, X_test, Y_train, Y_test = train_test_split(inputs, labels, train_size=train_size,
                                                    test_size=test_size)

\end{minted}





















\begin{minted}[fontsize=\fontsize{9pt}{9pt},linenos=false,mathescape,baselinestretch=1.0,fontfamily=tt,xleftmargin=7mm]{python}

epochs = 100
batch_size = 100
n_neurons_layer1 = 100
n_neurons_layer2 = 50
n_categories = 10
eta_vals = np.logspace(-5, 1, 7)
lmbd_vals = np.logspace(-5, 1, 7)
def create_neural_network_keras(n_neurons_layer1, n_neurons_layer2, n_categories, eta, lmbd):
    model = Sequential()
    model.add(Dense(n_neurons_layer1, activation='sigmoid', kernel_regularizer=regularizers.l2(lmbd)))
    model.add(Dense(n_neurons_layer2, activation='sigmoid', kernel_regularizer=regularizers.l2(lmbd)))
    model.add(Dense(n_categories, activation='softmax'))
    
    sgd = optimizers.SGD(lr=eta)
    model.compile(loss='categorical_crossentropy', optimizer=sgd, metrics=['accuracy'])
    
    return model

\end{minted}


















\begin{minted}[fontsize=\fontsize{9pt}{9pt},linenos=false,mathescape,baselinestretch=1.0,fontfamily=tt,xleftmargin=7mm]{python}
DNN_keras = np.zeros((len(eta_vals), len(lmbd_vals)), dtype=object)
        
for i, eta in enumerate(eta_vals):
    for j, lmbd in enumerate(lmbd_vals):
        DNN = create_neural_network_keras(n_neurons_layer1, n_neurons_layer2, n_categories,
                                         eta=eta, lmbd=lmbd)
        DNN.fit(X_train, Y_train, epochs=epochs, batch_size=batch_size, verbose=0)
        scores = DNN.evaluate(X_test, Y_test)
        
        DNN_keras[i][j] = DNN
        
        print("Learning rate = ", eta)
        print("Lambda = ", lmbd)
        print("Test accuracy: %.3f" % scores[1])
        print()

\end{minted}


































\begin{minted}[fontsize=\fontsize{9pt}{9pt},linenos=false,mathescape,baselinestretch=1.0,fontfamily=tt,xleftmargin=7mm]{python}
# optional
# visual representation of grid search
# uses seaborn heatmap, could probably do this in matplotlib
import seaborn as sns

sns.set()

train_accuracy = np.zeros((len(eta_vals), len(lmbd_vals)))
test_accuracy = np.zeros((len(eta_vals), len(lmbd_vals)))

for i in range(len(eta_vals)):
    for j in range(len(lmbd_vals)):
        DNN = DNN_keras[i][j]

        train_accuracy[i][j] = DNN.evaluate(X_train, Y_train)[1]
        test_accuracy[i][j] = DNN.evaluate(X_test, Y_test)[1]

        
fig, ax = plt.subplots(figsize = (10, 10))
sns.heatmap(train_accuracy, annot=True, ax=ax, cmap="viridis")
ax.set_title("Training Accuracy")
ax.set_ylabel("$\eta$")
ax.set_xlabel("$\lambda$")
plt.show()

fig, ax = plt.subplots(figsize = (10, 10))
sns.heatmap(test_accuracy, annot=True, ax=ax, cmap="viridis")
ax.set_title("Test Accuracy")
ax.set_ylabel("$\eta$")
ax.set_xlabel("$\lambda$")
plt.show()

\end{minted}


\paragraph{The Breast Cancer Data, now with Keras.}








































































































































































\begin{minted}[fontsize=\fontsize{9pt}{9pt},linenos=false,mathescape,baselinestretch=1.0,fontfamily=tt,xleftmargin=7mm]{python}

import tensorflow as tf
from tensorflow.keras.layers import Input
from tensorflow.keras.models import Sequential      #This allows appending layers to existing models
from tensorflow.keras.layers import Dense           #This allows defining the characteristics of a particular layer
from tensorflow.keras import optimizers             #This allows using whichever optimiser we want (sgd,adam,RMSprop)
from tensorflow.keras import regularizers           #This allows using whichever regularizer we want (l1,l2,l1_l2)
from tensorflow.keras.utils import to_categorical   #This allows using categorical cross entropy as the cost function
import numpy as np
import matplotlib.pyplot as plt
import seaborn as sns
from sklearn.model_selection import train_test_split as splitter
from sklearn.datasets import load_breast_cancer
import pickle
import os 


"""Load breast cancer dataset"""

np.random.seed(0)        #create same seed for random number every time

cancer=load_breast_cancer()      #Download breast cancer dataset

inputs=cancer.data                     #Feature matrix of 569 rows (samples) and 30 columns (parameters)
outputs=cancer.target                  #Label array of 569 rows (0 for benign and 1 for malignant)
labels=cancer.feature_names[0:30]

print('The content of the breast cancer dataset is:')      #Print information about the datasets
print(labels)
print('-------------------------')
print("inputs =  " + str(inputs.shape))
print("outputs =  " + str(outputs.shape))
print("labels =  "+ str(labels.shape))

x=inputs      #Reassign the Feature and Label matrices to other variables
y=outputs

#%% 

# Visualisation of dataset (for correlation analysis)

plt.figure()
plt.scatter(x[:,0],x[:,2],s=40,c=y,cmap=plt.cm.Spectral)
plt.xlabel('Mean radius',fontweight='bold')
plt.ylabel('Mean perimeter',fontweight='bold')
plt.show()

plt.figure()
plt.scatter(x[:,5],x[:,6],s=40,c=y, cmap=plt.cm.Spectral)
plt.xlabel('Mean compactness',fontweight='bold')
plt.ylabel('Mean concavity',fontweight='bold')
plt.show()


plt.figure()
plt.scatter(x[:,0],x[:,1],s=40,c=y,cmap=plt.cm.Spectral)
plt.xlabel('Mean radius',fontweight='bold')
plt.ylabel('Mean texture',fontweight='bold')
plt.show()

plt.figure()
plt.scatter(x[:,2],x[:,1],s=40,c=y,cmap=plt.cm.Spectral)
plt.xlabel('Mean perimeter',fontweight='bold')
plt.ylabel('Mean compactness',fontweight='bold')
plt.show()


# Generate training and testing datasets

#Select features relevant to classification (texture,perimeter,compactness and symmetery) 
#and add to input matrix

temp1=np.reshape(x[:,1],(len(x[:,1]),1))
temp2=np.reshape(x[:,2],(len(x[:,2]),1))
X=np.hstack((temp1,temp2))      
temp=np.reshape(x[:,5],(len(x[:,5]),1))
X=np.hstack((X,temp))       
temp=np.reshape(x[:,8],(len(x[:,8]),1))
X=np.hstack((X,temp))       

X_train,X_test,y_train,y_test=splitter(X,y,test_size=0.1)   #Split datasets into training and testing

y_train=to_categorical(y_train)     #Convert labels to categorical when using categorical cross entropy
y_test=to_categorical(y_test)

del temp1,temp2,temp

# %%

# Define tunable parameters"

eta=np.logspace(-3,-1,3)                    #Define vector of learning rates (parameter to SGD optimiser)
lamda=0.01                                  #Define hyperparameter
n_layers=2                                  #Define number of hidden layers in the model
n_neuron=np.logspace(0,3,4,dtype=int)       #Define number of neurons per layer
epochs=100                                   #Number of reiterations over the input data
batch_size=100                              #Number of samples per gradient update

# %%

"""Define function to return Deep Neural Network model"""

def NN_model(inputsize,n_layers,n_neuron,eta,lamda):
    model=Sequential()      
    for i in range(n_layers):       #Run loop to add hidden layers to the model
        if (i==0):                  #First layer requires input dimensions
            model.add(Dense(n_neuron,activation='relu',kernel_regularizer=regularizers.l2(lamda),input_dim=inputsize))
        else:                       #Subsequent layers are capable of automatic shape inferencing
            model.add(Dense(n_neuron,activation='relu',kernel_regularizer=regularizers.l2(lamda)))
    model.add(Dense(2,activation='softmax'))  #2 outputs - ordered and disordered (softmax for prob)
    sgd=optimizers.SGD(lr=eta)
    model.compile(loss='categorical_crossentropy',optimizer=sgd,metrics=['accuracy'])
    return model

    
Train_accuracy=np.zeros((len(n_neuron),len(eta)))      #Define matrices to store accuracy scores as a function
Test_accuracy=np.zeros((len(n_neuron),len(eta)))       #of learning rate and number of hidden neurons for 

for i in range(len(n_neuron)):     #run loops over hidden neurons and learning rates to calculate 
    for j in range(len(eta)):      #accuracy scores 
        DNN_model=NN_model(X_train.shape[1],n_layers,n_neuron[i],eta[j],lamda)
        DNN_model.fit(X_train,y_train,epochs=epochs,batch_size=batch_size,verbose=1)
        Train_accuracy[i,j]=DNN_model.evaluate(X_train,y_train)[1]
        Test_accuracy[i,j]=DNN_model.evaluate(X_test,y_test)[1]
               

def plot_data(x,y,data,title=None):

    # plot results
    fontsize=16


    fig = plt.figure()
    ax = fig.add_subplot(111)
    cax = ax.matshow(data, interpolation='nearest', vmin=0, vmax=1)
    
    cbar=fig.colorbar(cax)
    cbar.ax.set_ylabel('accuracy (%)',rotation=90,fontsize=fontsize)
    cbar.set_ticks([0,.2,.4,0.6,0.8,1.0])
    cbar.set_ticklabels(['0%','20%','40%','60%','80%','100%'])

    # put text on matrix elements
    for i, x_val in enumerate(np.arange(len(x))):
        for j, y_val in enumerate(np.arange(len(y))):
            c = "${0:.1f}\\%$".format( 100*data[j,i])  
            ax.text(x_val, y_val, c, va='center', ha='center')

    # convert axis vaues to to string labels
    x=[str(i) for i in x]
    y=[str(i) for i in y]


    ax.set_xticklabels(['']+x)
    ax.set_yticklabels(['']+y)

    ax.set_xlabel('$\\mathrm{learning\\ rate}$',fontsize=fontsize)
    ax.set_ylabel('$\\mathrm{hidden\\ neurons}$',fontsize=fontsize)
    if title is not None:
        ax.set_title(title)

    plt.tight_layout()

    plt.show()
    
plot_data(eta,n_neuron,Train_accuracy, 'training')
plot_data(eta,n_neuron,Test_accuracy, 'testing')


\end{minted}


\subsection*{Fine-tuning neural network hyperparameters}

The flexibility of neural networks is also one of their main
drawbacks: there are many hyperparameters to tweak. Not only can you
use any imaginable network topology (how neurons/nodes are interconnected),
but even in a simple FFNN you can change the number of layers, the
number of neurons per layer, the type of activation function to use in
each layer, the weight initialization logic, the stochastic gradient optmized and much more. How do you
know what combination of hyperparameters is the best for your task?

\begin{itemize}
\item You can use grid search with cross-validation to find the right hyperparameters.
\end{itemize}

\noindent
However,since there are many hyperparameters to tune, and since
training a neural network on a large dataset takes a lot of time, you
will only be able to explore a tiny part of the hyperparameter space.

\begin{itemize}
\item You can use randomized search.

\item Or use tools like \href{{http://oscar.calldesk.ai/}}{Oscar}, which implements more complex algorithms to help you find a good set of hyperparameters quickly.  
\end{itemize}

\noindent
\paragraph{Hidden layers.}
For many problems you can start with just one or two hidden layers and it will work just fine.
For the MNIST data set you ca easily get a high accuracy using just one hidden layer with a
few hundred neurons.
You can reach for this data set above 98\% accuracy using two hidden layers with the same total amount of
neurons, in roughly the same amount of training time. 

For more complex problems, you can gradually
ramp up the number of hidden layers, until you start overfitting the training set. Very complex tasks, such
as large image classification or speech recognition, typically require networks with dozens of layers
and they need a huge amount
of training data. However, you will rarely have to train such networks from scratch: it is much more
common to reuse parts of a pretrained state-of-the-art network that performs a similar task.

\paragraph{Which activation function should I use?}
The Back propagation algorithm we derived above works by going from
the output layer to the input layer, propagating the error gradient on
the way. Once the algorithm has computed the gradient of the cost
function with regards to each parameter in the network, it uses these
gradients to update each parameter with a Gradient Descent (GD) step.

Unfortunately for us, the gradients often get smaller and smaller as the
algorithm progresses down to the first hidden layers. As a result, the
GD update leaves the lower layer connection weights
virtually unchanged, and training never converges to a good
solution. This is known in the literature as 
\textbf{the vanishing gradients problem}. 

In other cases, the opposite can happen, namely the the gradients can grow bigger and
bigger. The result is that many of the layers get large updates of the 
weights the
algorithm diverges. This is the \textbf{exploding gradients problem}, which is
mostly encountered in recurrent neural networks. More generally, deep
neural networks suffer from unstable gradients, different layers may
learn at widely different speeds

\paragraph{Is the Logistic activation function (Sigmoid)  our choice?}
Although this unfortunate behavior has been empirically observed for
quite a while (it was one of the reasons why deep neural networks were
mostly abandoned for a long time), it is only around 2010 that
significant progress was made in understanding it.

A paper titled \href{{http://proceedings.mlr.press/v9/glorot10a.html}}{Understanding the Difficulty of Training Deep
Feedforward Neural Networks by Xavier Glorot and Yoshua Bengio} found that
the problems with the popular logistic
sigmoid activation function and the weight initialization technique
that was most popular at the time, namely random initialization using
a normal distribution with a mean of 0 and a standard deviation of
1. 

They showed that with this activation function and this
initialization scheme, the variance of the outputs of each layer is
much greater than the variance of its inputs. Going forward in the
network, the variance keeps increasing after each layer until the
activation function saturates at the top layers. This is actually made
worse by the fact that the logistic function has a mean of 0.5, not 0
(the hyperbolic tangent function has a mean of 0 and behaves slightly
better than the logistic function in deep networks).

\paragraph{The derivative of the Logistic funtion.}
Looking at the logistic activation function, when inputs become large
(negative or positive), the function saturates at 0 or 1, with a
derivative extremely close to 0. Thus when backpropagation kicks in,
it has virtually no gradient to propagate back through the network,
and what little gradient exists keeps getting diluted as
backpropagation progresses down through the top layers, so there is
really nothing left for the lower layers.

In their paper, Glorot and Bengio propose a way to significantly
alleviate this problem. We need the signal to flow properly in both
directions: in the forward direction when making predictions, and in
the reverse direction when backpropagating gradients. We don’t want
the signal to die out, nor do we want it to explode and saturate. For
the signal to flow properly, the authors argue that we need the
variance of the outputs of each layer to be equal to the variance of
its inputs, and we also need the gradients to have equal variance
before and after flowing through a layer in the reverse direction.

One of the insights in the 2010 paper by Glorot and Bengio was that
the vanishing/exploding gradients problems were in part due to a poor
choice of activation function. Until then most people had assumed that
if Nature had chosen to use roughly sigmoid activation functions in
biological neurons, they must be an excellent choice. But it turns out
that other activation functions behave much better in deep neural
networks, in particular the ReLU activation function, mostly because
it does not saturate for positive values (and also because it is quite
fast to compute).

\paragraph{The RELU function family.}
The ReLU activation function suffers from a problem known as the dying
ReLUs: during training, some neurons effectively die, meaning they
stop outputting anything other than 0.

In some cases, you may find that half of your network’s neurons are
dead, especially if you used a large learning rate. During training,
if a neuron’s weights get updated such that the weighted sum of the
neuron’s inputs is negative, it will start outputting 0. When this
happen, the neuron is unlikely to come back to life since the gradient
of the ReLU function is 0 when its input is negative.

To solve this problem, nowadays practitioners use a  variant of the ReLU
function, such as the leaky ReLU discussed above or the so-called
exponential linear unit (ELU) function

\[
ELU(z) = \left\{\begin{array}{cc} \alpha\left( \exp{(z)}-1\right) & z < 0,\\  z & z \ge 0.\end{array}\right. 
\]

\paragraph{Which activation function should we use?}
In general it seems that the ELU activation function is better than
the leaky ReLU function (and its variants), which is better than
ReLU. ReLU performs better than $\tanh$ which in turn performs better
than the logistic function. 

If runtime
performance is an issue, then you may opt for the  leaky ReLU function  over the 
ELU function If you don’t
want to tweak yet another hyperparameter, you may just use the default
$\alpha$ of $0.01$ for the leaky ReLU, and $1$ for ELU. If you have
spare time and computing power, you can use cross-validation or
bootstrap to evaluate other activation functions.

In most cases you can use the ReLU activation function in the hidden layers (or one of its variants).

It is a bit faster to compute than other activation functions, and the gradient descent optimization does in general not get stuck.

\textbf{For the output layer:}

\begin{itemize}
\item For classification the softmax activation function is generally a good choice for classification tasks (when the classes are mutually exclusive).

\item For regression tasks, you can simply use no activation function at all.
\end{itemize}

\noindent
\paragraph{Batch Normalization.}
Batch Normalization
aims to address the vanishing/exploding gradients problems, and more generally the problem that the
distribution of each layer’s inputs changes during training, as the parameters of the previous layers change.

The technique consists of adding an operation in the model just before the activation function of each
layer, simply zero-centering and normalizing the inputs, then scaling and shifting the result using two new
parameters per layer (one for scaling, the other for shifting). In other words, this operation lets the model
learn the optimal scale and mean of the inputs for each layer.
In order to zero-center and normalize the inputs, the algorithm needs to estimate the inputs’ mean and
standard deviation. It does so by evaluating the mean and standard deviation of the inputs over the current
mini-batch, from this the name batch normalization.

\paragraph{Dropout.}
It is a fairly simple algorithm: at every training step, every neuron (including the input neurons but
excluding the output neurons) has a probability $p$ of being temporarily dropped out, meaning it will be
entirely ignored during this training step, but it may be active during the next step.

The
hyperparameter $p$ is called the dropout rate, and it is typically set to 50\%. After training, the neurons are not dropped anymore.
 It is viewed as one of the most popular regularization techniques.

\paragraph{Gradient Clipping.}
A popular technique to lessen the exploding gradients problem is to simply clip the gradients during
backpropagation so that they never exceed some threshold (this is mostly useful for recurrent neural
networks).

This technique is called Gradient Clipping.

In general however, Batch
Normalization is preferred.

\subsection*{A top-down perspective on Neural networks}

The first thing we would like to do is divide the data into two or three
parts. A training set, a validation or dev (development) set, and a
test set. The test set is the data on which we want to make
predictions. The dev set is a subset of the training data we use to
check how well we are doing out-of-sample, after training the model on
the training dataset. We use the validation error as a proxy for the
test error in order to make tweaks to our model. It is crucial that we
do not use any of the test data to train the algorithm. This is a
cardinal sin in ML. Then:

\begin{itemize}
\item Estimate optimal error rate

\item Minimize underfitting (bias) on training data set.

\item Make sure you are not overfitting.
\end{itemize}

\noindent
If the validation and test sets are drawn from the same distributions,
then a good performance on the validation set should lead to similarly
good performance on the test set. 

However, sometimes
the training data and test data differ in subtle ways because, for
example, they are collected using slightly different methods, or
because it is cheaper to collect data in one way versus another. In
this case, there can be a mismatch between the training and test
data. This can lead to the neural network overfitting these small
differences between the test and training sets, and a poor performance
on the test set despite having a good performance on the validation
set. To rectify this, Andrew Ng suggests making two validation or dev
sets, one constructed from the training data and one constructed from
the test data. The difference between the performance of the algorithm
on these two validation sets quantifies the train-test mismatch. This
can serve as another important diagnostic when using DNNs for
supervised learning.

\paragraph{Limitations of supervised learning with deep networks.}
Like all statistical methods, supervised learning using neural
networks has important limitations. This is especially important when
one seeks to apply these methods, especially to physics problems. Like
all tools, DNNs are not a universal solution. Often, the same or
better performance on a task can be achieved by using a few
hand-engineered features (or even a collection of random
features). 

Here we list some of the important limitations of supervised neural network based models. 

\begin{itemize}
\item \textbf{Need labeled data}. All supervised learning methods, DNNs for supervised learning require labeled data. Often, labeled data is harder to acquire than unlabeled data (e.g.~one must pay for human experts to label images).

\item \textbf{Supervised neural networks are extremely data intensive.} DNNs are data hungry. They perform best when data is plentiful. This is doubly so for supervised methods where the data must also be labeled. The utility of DNNs is extremely limited if data is hard to acquire or the datasets are small (hundreds to a few thousand samples). In this case, the performance of other methods that utilize hand-engineered features can exceed that of DNNs.

\item \textbf{Homogeneous data.} Almost all DNNs deal with homogeneous data of one type. It is very hard to design architectures that mix and match data types (i.e.~some continuous variables, some discrete variables, some time series). In applications beyond images, video, and language, this is often what is required. In contrast, ensemble models like random forests or gradient-boosted trees have no difficulty handling mixed data types.

\item \textbf{Many problems are not about prediction.} In natural science we are often interested in learning something about the underlying distribution that generates the data. In this case, it is often difficult to cast these ideas in a supervised learning setting. While the problems are related, it is possible to make good predictions with a \emph{wrong} model. The model might or might not be useful for understanding the underlying science.
\end{itemize}

\noindent
Some of these remarks are particular to DNNs, others are shared by all supervised learning methods. This motivates the use of unsupervised methods which in part circumvent these problems.


%
\chapter{Boltzmann Machines}

Why use a generative model rather than the more well known discriminative deep neural networks (DNN)? 

\begin{itemize}
\item Discriminitave methods have several limitations: They are mainly supervised learning methods, thus requiring labeled data. And there are tasks they cannot accomplish, like drawing new examples from an unknown probability distribution.

\item A generative model can learn to represent and sample from a probability distribution. The core idea is to learn a parametric model of the probability distribution from which the training data was drawn. As an example
\begin{enumerate}

 \item A model for images could learn to draw new examples of cats and dogs, given a training dataset of images of cats and dogs.

 \item Generate a sample of an ordered or disordered Ising model phase, having been given samples of such phases.

 \item Model the trial function for Monte Carlo calculations

\end{enumerate}

\noindent
\item Both use gradient-descent based learning procedures for minimizing cost functions

\item Energy based models don't use backpropagation and automatic differentiation for computing gradients, instead turning to Markov Chain Monte Carlo methods.

\item DNNs often have several hidden layers. A restricted Boltzmann machine has only one hidden layer, however several RBMs can be stacked to make up Deep Belief Networks, of which they constitute the building blocks.
\end{itemize}

\noindent
History: The RBM was developed by amongst others Geoffrey Hinton, called by some the "Godfather of Deep Learning", working with the University of Toronto and Google.


% --- begin paragraph admon ---
\paragraph{}
A BM is what we would call an undirected probabilistic graphical model
with stochastic continuous or discrete units.
% --- end paragraph admon ---



% --- begin paragraph admon ---
\paragraph{}
It is interpreted as a stochastic recurrent neural network where the
state of each unit(neurons/nodes) depends on the units it is connected
to. The weights in the network represent thus the strength of the
interaction between various units/nodes.
% --- end paragraph admon ---



% --- begin paragraph admon ---
\paragraph{}
It turns into a Hopfield network if we choose deterministic rather
than stochastic units. In contrast to a Hopfield network, a BM is a
so-called generative model. It allows us to generate new samples from
the learned distribution.
% --- end paragraph admon ---




% --- begin paragraph admon ---
\paragraph{}
A standard BM network is divided into a set of observable and visible units $\hat{x}$ and a set of unknown hidden units/nodes $\hat{h}$.
% --- end paragraph admon ---




% --- begin paragraph admon ---
\paragraph{}
Additionally there can be bias nodes for the hidden and visible layers. These biases are normally set to $1$.
% --- end paragraph admon ---




% --- begin paragraph admon ---
\paragraph{}
BMs are stackable, meaning they cwe can train a BM which serves as input to another BM. We can construct deep networks for learning complex PDFs. The layers can be trained one after another, a feature which makes them popular in deep learning
% --- end paragraph admon ---



However, they are often hard to train. This leads to the introduction of so-called restricted BMs, or RBMS.
Here we take away all lateral connections between nodes in the visible layer as well as connections between nodes in the hidden layer. The network is illustrated in the figure below.

% FIGURE: [figures/RBM.pdf, width=800 frac=1.0]

\subsection*{The network}

\textbf{The network layers}:
\begin{enumerate}
 \item A function $\mathbf{x}$ that represents the visible layer, a vector of $M$ elements (nodes). This layer represents both what the RBM might be given as training input, and what we want it to be able to reconstruct. This might for example be the pixels of an image, the spin values of the Ising model, or coefficients representing speech.

 \item The function $\mathbf{h}$ represents the hidden, or latent, layer. A vector of $N$ elements (nodes). Also called "feature detectors".
\end{enumerate}

\noindent
The goal of the hidden layer is to increase the model's expressive power. We encode complex interactions between visible variables by introducing additional, hidden variables that interact with visible degrees of freedom in a simple manner, yet still reproduce the complex correlations between visible degrees in the data once marginalized over (integrated out).

Examples of this trick being employed in physics: 
\begin{enumerate}
 \item The Hubbard-Stratonovich transformation

 \item The introduction of ghost fields in gauge theory

 \item Shadow wave functions in Quantum Monte Carlo simulations
\end{enumerate}

\noindent
\textbf{The network parameters, to be optimized/learned}:
\begin{enumerate}
 \item $\mathbf{a}$ represents the visible bias, a vector of same length as $\mathbf{x}$.

 \item $\mathbf{b}$ represents the hidden bias, a vector of same lenght as $\mathbf{h}$.

 \item $W$ represents the interaction weights, a matrix of size $M\times N$.
\end{enumerate}

\noindent
\paragraph{Joint distribution.}
The restricted Boltzmann machine is described by a Bolztmann distribution
\begin{align}
	P_{rbm}(\mathbf{x},\mathbf{h}) = \frac{1}{Z} e^{-\frac{1}{T_0}E(\mathbf{x},\mathbf{h})},
\end{align}
where $Z$ is the normalization constant or partition function, defined as 
\begin{align}
	Z = \int \int e^{-\frac{1}{T_0}E(\mathbf{x},\mathbf{h})} d\mathbf{x} d\mathbf{h}.
\end{align}
It is common to ignore $T_0$ by setting it to one. 

\paragraph{Network Elements, the energy function.}
The function $E(\mathbf{x},\mathbf{h})$ gives the \textbf{energy} of a
configuration (pair of vectors) $(\mathbf{x}, \mathbf{h})$. The lower
the energy of a configuration, the higher the probability of it. This
function also depends on the parameters $\mathbf{a}$, $\mathbf{b}$ and
$W$. Thus, when we adjust them during the learning procedure, we are
adjusting the energy function to best fit our problem.

\paragraph{Defining different types of RBMs.}
There are different variants of RBMs, and the differences lie in the types of visible and hidden units we choose as well as in the implementation of the energy function $E(\mathbf{x},\mathbf{h})$. The connection between the nodes in the two layers is given by the weights $w_{ij}$. 


% --- begin paragraph admon ---
\paragraph{Binary-Binary RBM:}

RBMs were first developed using binary units in both the visible and hidden layer. The corresponding energy function is defined as follows:
\begin{align}
	E(\mathbf{x}, \mathbf{h}) = - \sum_i^M x_i a_i- \sum_j^N b_j h_j - \sum_{i,j}^{M,N} x_i w_{ij} h_j,
\end{align}
where the binary values taken on by the nodes are most commonly 0 and 1.
% --- end paragraph admon ---



% --- begin paragraph admon ---
\paragraph{Gaussian-Binary RBM:}

Another varient is the RBM where the visible units are Gaussian while the hidden units remain binary:
\begin{align}
	E(\mathbf{x}, \mathbf{h}) = \sum_i^M \frac{(x_i - a_i)^2}{2\sigma_i^2} - \sum_j^N b_j h_j - \sum_{i,j}^{M,N} \frac{x_i w_{ij} h_j}{\sigma_i^2}. 
\end{align}
% --- end paragraph admon ---



\begin{enumerate}
\item RBMs are Useful when we model continuous data (i.e., we wish $\mathbf{x}$ to be continuous)

\item Requires a smaller learning rate, since there's no upper bound to the value a component might take in the reconstruction
\end{enumerate}

\noindent
Other types of units include:
\begin{enumerate}
\item Softmax and multinomial units

\item Gaussian visible and hidden units

\item Binomial units

\item Rectified linear units
\end{enumerate}

\noindent
\paragraph{Cost function.}
When working with a training dataset, the most common training approach is maximizing the log-likelihood of the training data. The log likelihood characterizes the log-probability of generating the observed data using our generative model. Using this method our cost function is chosen as the negative log-likelihood. The learning then consists of trying to find parameters that maximize the probability of the dataset, and is known as Maximum Likelihood Estimation (MLE).
Denoting the parameters as $\bm{\theta} = a_1,...,a_M,b_1,...,b_N,w_{11},...,w_{MN}$, the log-likelihood is given by
\begin{align}
	\mathcal{L}(\{ \theta_i \}) &= \langle \text{log} P_\theta(\bm{x}) \rangle_{data} \\
	&= - \langle E(\bm{x}; \{ \theta_i\}) \rangle_{data} - \text{log} Z(\{ \theta_i\}),
\end{align}
where we used that the normalization constant does not depend on the data, $\langle \text{log} Z(\{ \theta_i\}) \rangle = \text{log} Z(\{ \theta_i\})$
Our cost function is the negative log-likelihood, $\mathcal{C}(\{ \theta_i \}) = - \mathcal{L}(\{ \theta_i \})$

\paragraph{Optimization / Training.}
The training procedure of choice often is Stochastic Gradient Descent (SGD). It consists of a series of iterations where we update the parameters according to the equation
\begin{align}
	\bm{\theta}_{k+1} = \bm{\theta}_k - \eta \nabla \mathcal{C} (\bm{\theta}_k)
\end{align}
at each $k$-th iteration. There are a range of variants of the algorithm which aim at making the learning rate $\eta$ more adaptive so the method might be more efficient while remaining stable.

We now need the gradient of the cost function in order to minimize it. We find that
\begin{align}
	\frac{\partial \mathcal{C}(\{ \theta_i\})}{\partial \theta_i}
	&= \langle \frac{\partial E(\bm{x}; \theta_i)}{\partial \theta_i} \rangle_{data}
	+ \frac{\partial \text{log} Z(\{ \theta_i\})}{\partial \theta_i} \\
	&= \langle O_i(\bm{x}) \rangle_{data} - \langle O_i(\bm{x}) \rangle_{model},
\end{align}
where in order to simplify notation we defined the "operator"
\begin{align}
	O_i(\bm{x}) = \frac{\partial E(\bm{x}; \theta_i)}{\partial \theta_i}, 
\end{align}
and used the statistical mechanics relationship between expectation values and the log-partition function:
\begin{align}
	\langle O_i(\bm{x}) \rangle_{model} = \text{Tr} P_\theta(\bm{x})O_i(\bm{x}) = - \frac{\partial \text{log} Z(\{ \theta_i\})}{\partial \theta_i}.
\end{align}

The data-dependent term in the gradient is known as the positive phase
of the gradient, while the model-dependent term is known as the
negative phase of the gradient. The aim of the training is to lower
the energy of configurations that are near observed data points
(increasing their probability), and raising the energy of
configurations that are far from observed data points (decreasing
their probability).

The gradient of the negative log-likelihood cost function of a Binary-Binary RBM is then
\begin{align}
	\frac{\partial \mathcal{C} (w_{ij}, a_i, b_j)}{\partial w_{ij}} =& \langle x_i h_j \rangle_{data} - \langle x_i h_j \rangle_{model} \\
	\frac{\partial \mathcal{C} (w_{ij}, a_i, b_j)}{\partial a_{ij}} =& \langle x_i \rangle_{data} - \langle x_i \rangle_{model} \\
	\frac{\partial \mathcal{C} (w_{ij}, a_i, b_j)}{\partial b_{ij}} =& \langle h_i \rangle_{data} - \langle h_i \rangle_{model}. \\
\end{align}
To get the expectation values with respect to the \emph{data}, we set the visible units to each of the observed samples in the training data, then update the hidden units according to the conditional probability found before. We then average over all samples in the training data to calculate expectation values with respect to the data. 

\paragraph{Kullback-Leibler relative entropy.}
When the goal of the training is to approximate a probability
distribution, as it is in generative modeling, another relevant
measure is the \textbf{Kullback-Leibler divergence}, also known as the
relative entropy or Shannon entropy. It is a non-symmetric measure of the
dissimilarity between two probability density functions $p$ and
$q$. If $p$ is the unkown probability which we approximate with $q$,
we can measure the difference by
\begin{align}
	\text{KL}(p||q) = \int_{-\infty}^{\infty} p (\bm{x}) \log \frac{p(\bm{x})}{q(\bm{x})}  d\bm{x}.
\end{align}

Thus, the Kullback-Leibler divergence between the distribution of the
training data $f(\bm{x})$ and the model distribution $p(\bm{x}|
\bm{\theta})$ is

\begin{align}
	\text{KL} (f(\bm{x})|| p(\bm{x}| \bm{\theta})) =& \int_{-\infty}^{\infty}
	f (\bm{x}) \log \frac{f(\bm{x})}{p(\bm{x}| \bm{\theta})} d\bm{x} \\
	=& \int_{-\infty}^{\infty} f(\bm{x}) \log f(\bm{x}) d\bm{x} - \int_{-\infty}^{\infty} f(\bm{x}) \log
	p(\bm{x}| \bm{\theta}) d\bm{x} \\
	%=& \mathbb{E}_{f(\bm{x})} (\log f(\bm{x})) - \mathbb{E}_{f(\bm{x})} (\log p(\bm{x}| \bm{\theta}))
	=& \langle \log f(\bm{x}) \rangle_{f(\bm{x})} - \langle \log p(\bm{x}| \bm{\theta}) \rangle_{f(\bm{x})} \\
	=& \langle \log f(\bm{x}) \rangle_{data} + \langle E(\bm{x}) \rangle_{data} + \log Z \\
	=& \langle \log f(\bm{x}) \rangle_{data} + \mathcal{C}_{LL} .
\end{align}

The first term is constant with respect to $\bm{\theta}$ since $f(\bm{x})$ is independent of $\bm{\theta}$. Thus the Kullback-Leibler Divergence is minimal when the second term is minimal. The second term is the log-likelihood cost function, hence minimizing the Kullback-Leibler divergence is equivalent to maximizing the log-likelihood.

To further understand generative models it is useful to study the
gradient of the cost function which is needed in order to minimize it
using methods like stochastic gradient descent. 

The partition function is the generating function of
expectation values, in particular there are mathematical relationships
between expectation values and the log-partition function. In this
case we have
\begin{align}
	\langle \frac{ \partial E(\bm{x}; \theta_i) } { \partial \theta_i} \rangle_{model}
	= \int p(\bm{x}| \bm{\theta}) \frac{ \partial E(\bm{x}; \theta_i) } { \partial \theta_i} d\bm{x} 
	= -\frac{\partial \log Z(\theta_i)}{ \partial  \theta_i} .
\end{align}

Here $\langle \cdot \rangle_{model}$ is the expectation value over the model probability distribution $p(\bm{x}| \bm{\theta})$.

\subsection*{Setting up for gradient descent calculations}

Using the previous relationship we can express the gradient of the cost function as

\begin{align}
	\frac{\partial \mathcal{C}_{LL}}{\partial \theta_i}
	=& \langle \frac{ \partial E(\bm{x}; \theta_i) } { \partial \theta_i} \rangle_{data} + \frac{\partial \log Z(\theta_i)}{ \partial  \theta_i} \\
	=& \langle \frac{ \partial E(\bm{x}; \theta_i) } { \partial \theta_i} \rangle_{data} - \langle \frac{ \partial E(\bm{x}; \theta_i) } { \partial \theta_i} \rangle_{model} \\
	%=& \langle O_i(\bm{x}) \rangle_{data} - \langle O_i(\bm{x}) \rangle_{model}
\end{align}

This expression shows that the gradient of the log-likelihood cost
function is a \textbf{difference of moments}, with one calculated from
the data and one calculated from the model. The data-dependent term is
called the \textbf{positive phase} and the model-dependent term is
called the \textbf{negative phase} of the gradient. We see now that
minimizing the cost function results in lowering the energy of
configurations $\bm{x}$ near points in the training data and
increasing the energy of configurations not observed in the training
data. That means we increase the model's probability of configurations
similar to those in the training data.

The gradient of the cost function also demonstrates why gradients of
unsupervised, generative models must be computed differently from for
those of for example FNNs. While the data-dependent expectation value
is easily calculated based on the samples $\bm{x}_i$ in the training
data, we must sample from the model in order to generate samples from
which to caclulate the model-dependent term. We sample from the model
by using MCMC-based methods. We can not sample from the model directly
because the partition function $Z$ is generally intractable.

As in supervised machine learning problems, the goal is also here to
perform well on \textbf{unseen} data, that is to have good
generalization from the training data. The distribution $f(x)$ we
approximate is not the \textbf{true} distribution we wish to estimate,
it is limited to the training data. Hence, in unsupervised training as
well it is important to prevent overfitting to the training data. Thus
it is common to add regularizers to the cost function in the same
manner as we discussed for say linear regression.

\subsection*{RBMs for the quantum many body problem}

The idea of applying RBMs to quantum many body problems was presented by G. Carleo and M. Troyer, working with ETH Zurich and Microsoft Research.

Some of their motivation included

\begin{itemize}
\item The wave function $\Psi$ is a monolithic mathematical quantity that contains all the information on a quantum state, be it a single particle or a complex molecule. In principle, an exponential amount of information is needed to fully encode a generic many-body quantum state.

\item There are still interesting open problems, including fundamental questions ranging from the dynamical properties of high-dimensional systems to the exact ground-state properties of strongly interacting fermions.

\item The difficulty lies in finding a general strategy to reduce the exponential complexity of the full many-body wave function down to its most essential features. That is
\begin{enumerate}

\item Dimensional reduction

\item Feature extraction

\end{enumerate}

\noindent
\item Among the most successful techniques to attack these challenges, artifical neural networks play a prominent role.

\item Want to understand whether an artifical neural network may adapt to describe a quantum system.
\end{itemize}

\noindent
Carleo and Troyer applied the RBM to the quantum mechanical spin lattice systems of the Ising model and Heisenberg model, with encouraging results. Our goal is to test the method on systems of moving particles. For the spin lattice systems it was natural to use a binary-binary RBM, with the nodes taking values of 1 and -1. For moving particles, on the other hand, we want the visible nodes to be continuous, representing position coordinates. Thus, we start by choosing a Gaussian-binary RBM, where the visible nodes are continuous and hidden nodes take on values of 0 or 1. If eventually we would like the hidden nodes to be continuous as well the rectified linear units seem like the most relevant choice.

\subsection*{Representing the wave function}

The wavefunction should be a probability amplitude depending on
 $\bm{x}$. The RBM model is given by the joint distribution of
 $\bm{x}$ and $\bm{h}$

\begin{align}
        F_{rbm}(\mathbf{x},\mathbf{h}) = \frac{1}{Z} e^{-\frac{1}{T_0}E(\mathbf{x},\mathbf{h})}.
\end{align}

To find the marginal distribution of $\bm{x}$ we set:

\begin{align}
        F_{rbm}(\mathbf{x}) &= \sum_\mathbf{h} F_{rbm}(\mathbf{x}, \mathbf{h}) \\
                                &= \frac{1}{Z}\sum_\mathbf{h} e^{-E(\mathbf{x}, \mathbf{h})}.
\end{align}

Now this is what we use to represent the wave function, calling it a neural-network quantum state (NQS)
\begin{align}
        \Psi (\mathbf{X}) &= F_{rbm}(\mathbf{x}) \\
        &= \frac{1}{Z}\sum_{\bm{h}} e^{-E(\mathbf{x}, \mathbf{h})} \\
        &= \frac{1}{Z} \sum_{\{h_j\}} e^{-\sum_i^M \frac{(x_i - a_i)^2}{2\sigma^2} + \sum_j^N b_j h_j + \sum_\
{i,j}^{M,N} \frac{x_i w_{ij} h_j}{\sigma^2}} \\
        &= \frac{1}{Z} e^{-\sum_i^M \frac{(x_i - a_i)^2}{2\sigma^2}} \prod_j^N (1 + e^{b_j + \sum_i^M \frac{x\
_i w_{ij}}{\sigma^2}}). \\
\end{align}

\subsection*{Choose the cost function}

Now we don't necessarily have training data (unless we generate it by using some other method). However, what we do have is the variational principle which allows us to obtain the ground state wave function by minimizing the expectation value of the energy of a trial wavefunction (corresponding to the untrained NQS). Similarly to the traditional variational Monte Carlo method then, it is the local energy we wish to minimize. The gradient to use for the stochastic gradient descent procedure is

\begin{align}
	C_i = \frac{\partial \langle E_L \rangle}{\partial \theta_i}
	= 2(\langle E_L \frac{1}{\Psi}\frac{\partial \Psi}{\partial \theta_i} \rangle - \langle E_L \rangle \langle \frac{1}{\Psi}\frac{\partial \Psi}{\partial \theta_i} \rangle ),
\end{align}
where the local energy is given by
\begin{align}
	E_L = \frac{1}{\Psi} \hat{\mathbf{H}} \Psi.
\end{align}

\paragraph{Mathematical details.}
Because we are restricted to potential functions which are positive it
is convenient to express them as exponentials, so that

\begin{align}
	\phi_C (\bm{x}_C) = e^{-E_C(\bm{x}_C)}
\end{align}

where $E(\bm{x}_C)$ is called an \emph{energy function}, and the
exponential representation is the \emph{Boltzmann distribution}. The
joint distribution is defined as the product of potentials.

The joint distribution of the random variables is then

\begin{align}
	p(\bm{x}) =& \frac{1}{Z} \prod_C \phi_C (\bm{x}_C) \nonumber \\
	=& \frac{1}{Z} \prod_C e^{-E_C(\bm{x}_C)} \nonumber \\
	=& \frac{1}{Z} e^{-\sum_C E_C(\bm{x}_C)} \nonumber \\
	=& \frac{1}{Z} e^{-E(\bm{x})}.
\end{align} 
\begin{align}
	p_{BM}(\bm{x}, \bm{h}) = \frac{1}{Z_{BM}} e^{-\frac{1}{T}E_{BM}(\bm{x}, \bm{h})} ,
\end{align}

with the partition function 
\begin{align}
	Z_{BM} = \int \int e^{-\frac{1}{T} E_{BM}(\tilde{\bm{x}}, \tilde{\bm{h}})} d\tilde{\bm{x}} d\tilde{\bm{h}} .
\end{align}

$T$ is a physics-inspired parameter named temperature and will be assumed to be 1 unless otherwise stated. The energy function of the Boltzmann machine determines the interactions between the nodes and is defined  

\begin{align}
	E_{BM}(\bm{x}, \bm{h}) =& - \sum_{i, k}^{M, K} a_i^k \alpha_i^k (x_i)
	- \sum_{j, l}^{N, L} b_j^l \beta_j^l (h_j) 
	- \sum_{i,j,k,l}^{M,N,K,L} \alpha_i^k (x_i) w_{ij}^{kl} \beta_j^l (h_j) \nonumber \\
	&- \sum_{i, m=i+1, k}^{M, M, K} \alpha_i^k (x_i) v_{im}^k \alpha_m^k (x_m)
	- \sum_{j,n=j+1,l}^{N,N,L} \beta_j^l (h_j) u_{jn}^l \beta_n^l (h_n).
\end{align}

Here $\alpha_i^k (x_i)$ and $\beta_j^l (h_j)$ are one-dimensional
transfer functions or mappings from the given input value to the
desired feature value. They can be arbitrary functions of the input
variables and are independent of the parameterization (parameters
referring to weight and biases), meaning they are not affected by
training of the model. The indices $k$ and $l$ indicate that there can
be multiple transfer functions per variable.  Furthermore, $a_i^k$ and
$b_j^l$ are the visible and hidden bias. $w_{ij}^{kl}$ are weights of
the \textbf{inter-layer} connection terms which connect visible and
hidden units. $ v_{im}^k$ and $u_{jn}^l$ are weights of the
\textbf{intra-layer} connection terms which connect the visible units
to each other and the hidden units to each other, respectively.

We remove the intra-layer connections by setting $v_{im}$ and $u_{jn}$
to zero. The expression for the energy of the RBM is then

\begin{align}
	E_{RBM}(\bm{x}, \bm{h}) = - \sum_{i, k}^{M, K} a_i^k \alpha_i^k (x_i)
	- \sum_{j, l}^{N, L} b_j^l \beta_j^l (h_j) 
	- \sum_{i,j,k,l}^{M,N,K,L} \alpha_i^k (x_i) w_{ij}^{kl} \beta_j^l (h_j). 
\end{align}
resulting in 
\begin{align}
	P_{RBM} (\bm{x}) =& \int P_{RBM} (\bm{x}, \tilde{\bm{h}})  d \tilde{\bm{h}} \nonumber \\
	=& \frac{1}{Z_{RBM}} \int e^{-E_{RBM} (\bm{x}, \tilde{\bm{h}}) } d\tilde{\bm{h}} \nonumber \\
	=& \frac{1}{Z_{RBM}} \int e^{\sum_{i, k} a_i^k \alpha_i^k (x_i)
	+ \sum_{j, l} b_j^l \beta_j^l (\tilde{h}_j) 
	+ \sum_{i,j,k,l} \alpha_i^k (x_i) w_{ij}^{kl} \beta_j^l (\tilde{h}_j)} 
	d\tilde{\bm{h}} \nonumber \\
	=& \frac{1}{Z_{RBM}} e^{\sum_{i, k} a_i^k \alpha_i^k (x_i)}
	\int \prod_j^N e^{\sum_l b_j^l \beta_j^l (\tilde{h}_j) 
	+ \sum_{i,k,l} \alpha_i^k (x_i) w_{ij}^{kl} \beta_j^l (\tilde{h}_j)} d\tilde{\bm{h}} \nonumber \\
	=& \frac{1}{Z_{RBM}} e^{\sum_{i, k} a_i^k \alpha_i^k (x_i)}
	\biggl( \int e^{\sum_l b_1^l \beta_1^l (\tilde{h}_1) + \sum_{i,k,l} \alpha_i^k (x_i) w_{i1}^{kl} \beta_1^l (\tilde{h}_1)} d \tilde{h}_1 \nonumber \\
	& \times \int e^{\sum_l b_2^l \beta_2^l (\tilde{h}_2) + \sum_{i,k,l} \alpha_i^k (x_i) w_{i2}^{kl} \beta_2^l (\tilde{h}_2)} d \tilde{h}_2 \nonumber \\
	& \times ... \nonumber \\
	& \times \int e^{\sum_l b_N^l \beta_N^l (\tilde{h}_N) + \sum_{i,k,l} \alpha_i^k (x_i) w_{iN}^{kl} \beta_N^l (\tilde{h}_N)} d \tilde{h}_N \biggr) \nonumber \\
	=& \frac{1}{Z_{RBM}} e^{\sum_{i, k} a_i^k \alpha_i^k (x_i)}
	\prod_j^N \int e^{\sum_l b_j^l \beta_j^l (\tilde{h}_j) + \sum_{i,k,l} \alpha_i^k (x_i) w_{ij}^{kl} \beta_j^l (\tilde{h}_j)}  d\tilde{h}_j
\end{align}

Similarly

\begin{align}
	P_{RBM} (\bm{h}) =& \frac{1}{Z_{RBM}} \int e^{-E_{RBM} (\tilde{\bm{x}}, \bm{h})} d\tilde{\bm{x}} \nonumber \\
	=& \frac{1}{Z_{RBM}} e^{\sum_{j, l} b_j^l \beta_j^l (h_j)}
	\prod_i^M \int e^{\sum_k a_i^k \alpha_i^k (\tilde{x}_i)
	+ \sum_{j,k,l} \alpha_i^k (\tilde{x}_i) w_{ij}^{kl} \beta_j^l (h_j)} d\tilde{x}_i
\end{align}

Using Bayes theorem

\begin{align}
	P_{RBM} (\bm{h}|\bm{x}) =& \frac{P_{RBM} (\bm{x}, \bm{h})}{P_{RBM} (\bm{x})} \nonumber \\
	=& \frac{\frac{1}{Z_{RBM}} e^{\sum_{i, k} a_i^k \alpha_i^k (x_i)
	+ \sum_{j, l} b_j^l \beta_j^l (h_j) 
	+ \sum_{i,j,k,l} \alpha_i^k (x_i) w_{ij}^{kl} \beta_j^l (h_j)}}
	{\frac{1}{Z_{RBM}} e^{\sum_{i, k} a_i^k \alpha_i^k (x_i)}
	\prod_j^N \int e^{\sum_l b_j^l \beta_j^l (\tilde{h}_j) + \sum_{i,k,l} \alpha_i^k (x_i) w_{ij}^{kl} \beta_j^l (\tilde{h}_j)}  d\tilde{h}_j} \nonumber \\
	=& \prod_j^N \frac{e^{\sum_l b_j^l \beta_j^l (h_j) + \sum_{i,k,l} \alpha_i^k (x_i) w_{ij}^{kl} \beta_j^l (h_j)} }
	{\int e^{\sum_l b_j^l \beta_j^l (\tilde{h}_j) + \sum_{i,k,l} \alpha_i^k (x_i) w_{ij}^{kl} \beta_j^l (\tilde{h}_j)}  d\tilde{h}_j}
\end{align}

Similarly

\begin{align}
	P_{RBM} (\bm{x}|\bm{h}) =&  \frac{P_{RBM} (\bm{x}, \bm{h})}{P_{RBM} (\bm{h})} \nonumber \\
	=& \prod_i^M \frac{e^{\sum_k a_i^k \alpha_i^k (x_i)
	+ \sum_{j,k,l} \alpha_i^k (x_i) w_{ij}^{kl} \beta_j^l (h_j)}}
	{\int e^{\sum_k a_i^k \alpha_i^k (\tilde{x}_i)
	+ \sum_{j,k,l} \alpha_i^k (\tilde{x}_i) w_{ij}^{kl} \beta_j^l (h_j)} d\tilde{x}_i}
\end{align}

The original RBM had binary visible and hidden nodes. They were
showned to be universal approximators of discrete distributions.
It was also shown that adding hidden units yields
strictly improved modelling power. The common choice of binary values
are 0 and 1. However, in some physics applications, -1 and 1 might be
a more natural choice. We will here use 0 and 1.

\begin{align}
	E_{BB}(\bm{x}, \mathbf{h}) = - \sum_i^M x_i a_i- \sum_j^N b_j h_j - \sum_{i,j}^{M,N} x_i w_{ij} h_j.
\end{align}

\begin{align}
	p_{BB}(\bm{x}, \bm{h}) =& \frac{1}{Z_{BB}} e^{\sum_i^M a_i x_i + \sum_j^N b_j h_j + \sum_{ij}^{M,N} x_i w_{ij} h_j} \\
	=& \frac{1}{Z_{BB}} e^{\bm{x}^T \bm{a} + \bm{b}^T \bm{h} + \bm{x}^T \bm{W} \bm{h}}
\end{align}

with the partition function

\begin{align}
	Z_{BB} = \sum_{\bm{x}, \bm{h}} e^{\bm{x}^T \bm{a} + \bm{b}^T \bm{h} + \bm{x}^T \bm{W} \bm{h}} .
\end{align}

\paragraph{Marginal Probability Density Functions.}
In order to find the probability of any configuration of the visible units we derive the marginal probability density function.

\begin{align}
	p_{BB} (\bm{x}) =& \sum_{\bm{h}} p_{BB} (\bm{x}, \bm{h}) \\
	=& \frac{1}{Z_{BB}} \sum_{\bm{h}} e^{\bm{x}^T \bm{a} + \bm{b}^T \bm{h} + \bm{x}^T \bm{W} \bm{h}} \nonumber \\
	=& \frac{1}{Z_{BB}} e^{\bm{x}^T \bm{a}} \sum_{\bm{h}} e^{\sum_j^N (b_j + \bm{x}^T \bm{w}_{\ast j})h_j} \nonumber \\
	=& \frac{1}{Z_{BB}} e^{\bm{x}^T \bm{a}} \sum_{\bm{h}} \prod_j^N e^{ (b_j + \bm{x}^T \bm{w}_{\ast j})h_j} \nonumber \\
	=& \frac{1}{Z_{BB}} e^{\bm{x}^T \bm{a}} \bigg ( \sum_{h_1} e^{(b_1 + \bm{x}^T \bm{w}_{\ast 1})h_1}
	\times \sum_{h_2} e^{(b_2 + \bm{x}^T \bm{w}_{\ast 2})h_2} \times \nonumber \\
	& ... \times \sum_{h_2} e^{(b_N + \bm{x}^T \bm{w}_{\ast N})h_N} \bigg ) \nonumber \\
	=& \frac{1}{Z_{BB}} e^{\bm{x}^T \bm{a}} \prod_j^N \sum_{h_j} e^{(b_j + \bm{x}^T \bm{w}_{\ast j}) h_j} \nonumber \\
	=& \frac{1}{Z_{BB}} e^{\bm{x}^T \bm{a}} \prod_j^N (1 + e^{b_j + \bm{x}^T \bm{w}_{\ast j}}) .
\end{align}

A similar derivation yields the marginal probability of the hidden units

\begin{align}
	p_{BB} (\bm{h}) = \frac{1}{Z_{BB}} e^{\bm{b}^T \bm{h}} \prod_i^M (1 + e^{a_i + \bm{w}_{i\ast}^T \bm{h}}) .
\end{align}

\paragraph{Conditional Probability Density Functions.}
We derive the probability of the hidden units given the visible units using Bayes' rule

\begin{align}
	p_{BB} (\bm{h}|\bm{x}) =& \frac{p_{BB} (\bm{x}, \bm{h})}{p_{BB} (\bm{x})} \nonumber \\
	=& \frac{ \frac{1}{Z_{BB}}  e^{\bm{x}^T \bm{a} + \bm{b}^T \bm{h} + \bm{x}^T \bm{W} \bm{h}} }
	        {\frac{1}{Z_{BB}} e^{\bm{x}^T \bm{a}} \prod_j^N (1 + e^{b_j + \bm{x}^T \bm{w}_{\ast j}})} \nonumber \\
	=& \frac{  e^{\bm{x}^T \bm{a}} e^{ \sum_j^N (b_j + \bm{x}^T \bm{w}_{\ast j} ) h_j} }
	        { e^{\bm{x}^T \bm{a}} \prod_j^N (1 + e^{b_j + \bm{x}^T \bm{w}_{\ast j}})} \nonumber \\
	=& \prod_j^N \frac{ e^{(b_j + \bm{x}^T \bm{w}_{\ast j} ) h_j}  }
	{1 + e^{b_j + \bm{x}^T \bm{w}_{\ast j}}} \nonumber \\
	=& \prod_j^N p_{BB} (h_j| \bm{x}) .
\end{align}

From this we find the probability of a hidden unit being "on" or "off":

\begin{align}
	p_{BB} (h_j=1 | \bm{x}) =&   \frac{ e^{(b_j + \bm{x}^T \bm{w}_{\ast j} ) h_j}  }
	{1 + e^{b_j + \bm{x}^T \bm{w}_{\ast j}}} \\
	=&  \frac{ e^{(b_j + \bm{x}^T \bm{w}_{\ast j} )}  }
	{1 + e^{b_j + \bm{x}^T \bm{w}_{\ast j}}} \\
	=&  \frac{ 1 }{1 + e^{-(b_j + \bm{x}^T \bm{w}_{\ast j})} } ,
\end{align}
and

\begin{align}
	p_{BB} (h_j=0 | \bm{x}) =\frac{ 1 }{1 + e^{b_j + \bm{x}^T \bm{w}_{\ast j}} } .
\end{align}

Similarly we have that the conditional probability of the visible units given the hidden are

\begin{align}
	p_{BB} (\bm{x}|\bm{h}) =& \prod_i^M \frac{ e^{ (a_i + \bm{w}_{i\ast}^T \bm{h}) x_i} }{ 1 + e^{a_i + \bm{w}_{i\ast}^T \bm{h}} } \\
	&= \prod_i^M p_{BB} (x_i | \bm{h}) .
\end{align}

\begin{align}
	p_{BB} (x_i=1 | \bm{h}) =& \frac{1}{1 + e^{-(a_i + \bm{w}_{i\ast}^T \bm{h} )}} \\
	p_{BB} (x_i=0 | \bm{h}) =& \frac{1}{1 + e^{a_i + \bm{w}_{i\ast}^T \bm{h} }} .
\end{align}

\paragraph{Gaussian-Binary Restricted Boltzmann Machines.}
Inserting into the expression for $E_{RBM}(\bm{x},\bm{h})$ in equation  results in the energy

\begin{align}
	E_{GB}(\bm{x}, \bm{h}) =& \sum_i^M \frac{(x_i - a_i)^2}{2\sigma_i^2}
	- \sum_j^N b_j h_j 
	-\sum_{ij}^{M,N} \frac{x_i w_{ij} h_j}{\sigma_i^2} \nonumber \\
	=& \vert\vert\frac{\bm{x} -\bm{a}}{2\bm{\sigma}}\vert\vert^2 - \bm{b}^T \bm{h} 
	- (\frac{\bm{x}}{\bm{\sigma}^2})^T \bm{W}\bm{h} . 
\end{align}

\paragraph{Joint Probability Density Function.}
\begin{align}
	p_{GB} (\bm{x}, \bm{h}) =& \frac{1}{Z_{GB}} e^{-\vert\vert\frac{\bm{x} -\bm{a}}{2\bm{\sigma}}\vert\vert^2 + \bm{b}^T \bm{h} 
	+ (\frac{\bm{x}}{\bm{\sigma}^2})^T \bm{W}\bm{h}} \nonumber \\
	=& \frac{1}{Z_{GB}} e^{- \sum_i^M \frac{(x_i - a_i)^2}{2\sigma_i^2}
	+ \sum_j^N b_j h_j 
	+\sum_{ij}^{M,N} \frac{x_i w_{ij} h_j}{\sigma_i^2}} \nonumber \\
	=& \frac{1}{Z_{GB}} \prod_{ij}^{M,N} e^{-\frac{(x_i - a_i)^2}{2\sigma_i^2}
	+ b_j h_j 
	+\frac{x_i w_{ij} h_j}{\sigma_i^2}} ,
\end{align}

with the partition function given by

\begin{align}
	Z_{GB} =& \int \sum_{\tilde{\bm{h}}}^{\tilde{\bm{H}}} e^{-\vert\vert\frac{\tilde{\bm{x}} -\bm{a}}{2\bm{\sigma}}\vert\vert^2 + \bm{b}^T \tilde{\bm{h}} 
	+ (\frac{\tilde{\bm{x}}}{\bm{\sigma}^2})^T \bm{W}\tilde{\bm{h}}} d\tilde{\bm{x}} .
\end{align}

\paragraph{Marginal Probability Density Functions.}
We proceed to find the marginal probability densitites of the
Gaussian-binary RBM. We first marginalize over the binary hidden units
to find $p_{GB} (\bm{x})$

\begin{align}
	p_{GB} (\bm{x}) =& \sum_{\tilde{\bm{h}}}^{\tilde{\bm{H}}} p_{GB} (\bm{x}, \tilde{\bm{h}}) \nonumber \\
	=& \frac{1}{Z_{GB}} \sum_{\tilde{\bm{h}}}^{\tilde{\bm{H}}} 
	e^{-\vert\vert\frac{\bm{x} -\bm{a}}{2\bm{\sigma}}\vert\vert^2 + \bm{b}^T \tilde{\bm{h}} 
	+ (\frac{\bm{x}}{\bm{\sigma}^2})^T \bm{W}\tilde{\bm{h}}} \nonumber \\
	=& \frac{1}{Z_{GB}} e^{-\vert\vert\frac{\bm{x} -\bm{a}}{2\bm{\sigma}}\vert\vert^2}
	\prod_j^N (1 + e^{b_j + (\frac{\bm{x}}{\bm{\sigma}^2})^T \bm{w}_{\ast j}} ) .
\end{align}

We next marginalize over the visible units. This is the first time we
marginalize over continuous values. We rewrite the exponential factor
dependent on $\bm{x}$ as a Gaussian function before we integrate in
the last step.

\begin{align}
	p_{GB} (\bm{h}) =& \int p_{GB} (\tilde{\bm{x}}, \bm{h}) d\tilde{\bm{x}} \nonumber \\
	=& \frac{1}{Z_{GB}} \int e^{-\vert\vert\frac{\tilde{\bm{x}} -\bm{a}}{2\bm{\sigma}}\vert\vert^2 + \bm{b}^T \bm{h} 
	+ (\frac{\tilde{\bm{x}}}{\bm{\sigma}^2})^T \bm{W}\bm{h}} d\tilde{\bm{x}} \nonumber \\
	=& \frac{1}{Z_{GB}} e^{\bm{b}^T \bm{h} } \int \prod_i^M
	e^{- \frac{(\tilde{x}_i - a_i)^2}{2\sigma_i^2} + \frac{\tilde{x}_i \bm{w}_{i\ast}^T \bm{h}}{\sigma_i^2} } d\tilde{\bm{x}} \nonumber \\
	=& \frac{1}{Z_{GB}} e^{\bm{b}^T \bm{h} }
	\biggl( \int e^{- \frac{(\tilde{x}_1 - a_1)^2}{2\sigma_1^2} + \frac{\tilde{x}_1 \bm{w}_{1\ast}^T \bm{h}}{\sigma_1^2} } d\tilde{x}_1 \nonumber \\
	& \times \int e^{- \frac{(\tilde{x}_2 - a_2)^2}{2\sigma_2^2} + \frac{\tilde{x}_2 \bm{w}_{2\ast}^T \bm{h}}{\sigma_2^2} } d\tilde{x}_2 \nonumber \\
	& \times ... \nonumber \\
	&\times \int e^{- \frac{(\tilde{x}_M - a_M)^2}{2\sigma_M^2} + \frac{\tilde{x}_M \bm{w}_{M\ast}^T \bm{h}}{\sigma_M^2} } d\tilde{x}_M \biggr) \nonumber \\
	=& \frac{1}{Z_{GB}} e^{\bm{b}^T \bm{h}} \prod_i^M
	\int e^{- \frac{(\tilde{x}_i - a_i)^2 - 2\tilde{x}_i \bm{w}_{i\ast}^T \bm{h}}{2\sigma_i^2} } d\tilde{x}_i \nonumber \\
	=& \frac{1}{Z_{GB}} e^{\bm{b}^T \bm{h}} \prod_i^M
	\int e^{- \frac{\tilde{x}_i^2 - 2\tilde{x}_i(a_i + \tilde{x}_i \bm{w}_{i\ast}^T \bm{h}) + a_i^2}{2\sigma_i^2} } d\tilde{x}_i \nonumber \\
	=& \frac{1}{Z_{GB}} e^{\bm{b}^T \bm{h}} \prod_i^M
	\int e^{- \frac{\tilde{x}_i^2 - 2\tilde{x}_i(a_i + \bm{w}_{i\ast}^T \bm{h}) + (a_i + \bm{w}_{i\ast}^T \bm{h})^2 - (a_i + \bm{w}_{i\ast}^T \bm{h})^2 + a_i^2}{2\sigma_i^2} } d\tilde{x}_i \nonumber \\
	=& \frac{1}{Z_{GB}} e^{\bm{b}^T \bm{h}} \prod_i^M
	\int e^{- \frac{(\tilde{x}_i - (a_i + \bm{w}_{i\ast}^T \bm{h}))^2 - a_i^2 -2a_i \bm{w}_{i\ast}^T \bm{h} - (\bm{w}_{i\ast}^T \bm{h})^2 + a_i^2}{2\sigma_i^2} } d\tilde{x}_i \nonumber \\
	=& \frac{1}{Z_{GB}} e^{\bm{b}^T \bm{h}} \prod_i^M
	e^{\frac{2a_i \bm{w}_{i\ast}^T \bm{h} +(\bm{w}_{i\ast}^T \bm{h})^2 }{2\sigma_i^2}}
	\int e^{- \frac{(\tilde{x}_i - a_i - \bm{w}_{i\ast}^T \bm{h})^2}{2\sigma_i^2}}
	d\tilde{x}_i \nonumber \\
	=& \frac{1}{Z_{GB}} e^{\bm{b}^T \bm{h}} \prod_i^M
	\sqrt{2\pi \sigma_i^2}
	e^{\frac{2a_i \bm{w}_{i\ast}^T \bm{h} +(\bm{w}_{i\ast}^T \bm{h})^2 }{2\sigma_i^2}} .
\end{align}

\paragraph{Conditional Probability Density Functions.}
We finish by deriving the conditional probabilities.
\begin{align}
	p_{GB} (\bm{h}| \bm{x}) =& \frac{p_{GB} (\bm{x}, \bm{h})}{p_{GB} (\bm{x})} \nonumber \\
	=& \frac{\frac{1}{Z_{GB}} e^{-\vert\vert\frac{\bm{x} -\bm{a}}{2\bm{\sigma}}\vert\vert^2 + \bm{b}^T \bm{h} 
	+ (\frac{\bm{x}}{\bm{\sigma}^2})^T \bm{W}\bm{h}}}
	{\frac{1}{Z_{GB}} e^{-\vert\vert\frac{\bm{x} -\bm{a}}{2\bm{\sigma}}\vert\vert^2}
	\prod_j^N (1 + e^{b_j + (\frac{\bm{x}}{\bm{\sigma}^2})^T \bm{w}_{\ast j}} ) }
	\nonumber \\
	=& \prod_j^N \frac{e^{(b_j + (\frac{\bm{x}}{\bm{\sigma}^2})^T \bm{w}_{\ast j})h_j } }
	{1 + e^{b_j + (\frac{\bm{x}}{\bm{\sigma}^2})^T \bm{w}_{\ast j}}} \nonumber \\
	=& \prod_j^N p_{GB} (h_j|\bm{x}).
\end{align}
The conditional probability of a binary hidden unit $h_j$ being on or off again takes the form of a sigmoid function

\begin{align}
	p_{GB} (h_j =1 | \bm{x}) =& \frac{e^{b_j + (\frac{\bm{x}}{\bm{\sigma}^2})^T \bm{w}_{\ast j} } }
	{1 + e^{b_j + (\frac{\bm{x}}{\bm{\sigma}^2})^T \bm{w}_{\ast j}}} \nonumber \\
	=& \frac{1}{1 + e^{-b_j - (\frac{\bm{x}}{\bm{\sigma}^2})^T \bm{w}_{\ast j}}} \\
	p_{GB} (h_j =0 | \bm{x}) =&
	\frac{1}{1 + e^{b_j +(\frac{\bm{x}}{\bm{\sigma}^2})^T \bm{w}_{\ast j}}} .
\end{align}

The conditional probability of the continuous $\bm{x}$ now has another form, however.

\begin{align}
	p_{GB} (\bm{x}|\bm{h})
	=& \frac{p_{GB} (\bm{x}, \bm{h})}{p_{GB} (\bm{h})} \nonumber \\
	=& \frac{\frac{1}{Z_{GB}} e^{-\vert\vert\frac{\bm{x} -\bm{a}}{2\bm{\sigma}}\vert\vert^2 + \bm{b}^T \bm{h} 
	+ (\frac{\bm{x}}{\bm{\sigma}^2})^T \bm{W}\bm{h}}}
	{\frac{1}{Z_{GB}} e^{\bm{b}^T \bm{h}} \prod_i^M
	\sqrt{2\pi \sigma_i^2}
	e^{\frac{2a_i \bm{w}_{i\ast}^T \bm{h} +(\bm{w}_{i\ast}^T \bm{h})^2 }{2\sigma_i^2}}}
	\nonumber \\
	=& \prod_i^M \frac{1}{\sqrt{2\pi \sigma_i^2}}
	\frac{e^{- \frac{(x_i - a_i)^2}{2\sigma_i^2} + \frac{x_i \bm{w}_{i\ast}^T \bm{h}}{2\sigma_i^2} }}
	{e^{\frac{2a_i \bm{w}_{i\ast}^T \bm{h} +(\bm{w}_{i\ast}^T \bm{h})^2 }{2\sigma_i^2}}}
	\nonumber \\
	=& \prod_i^M \frac{1}{\sqrt{2\pi \sigma_i^2}}
	\frac{e^{-\frac{x_i^2 - 2a_i x_i + a_i^2 - 2x_i \bm{w}_{i\ast}^T\bm{h} }{2\sigma_i^2} } }
	{e^{\frac{2a_i \bm{w}_{i\ast}^T \bm{h} +(\bm{w}_{i\ast}^T \bm{h})^2 }{2\sigma_i^2}}}
	\nonumber \\
	=& \prod_i^M \frac{1}{\sqrt{2\pi \sigma_i^2}}
	e^{- \frac{x_i^2 - 2a_i x_i + a_i^2 - 2x_i \bm{w}_{i\ast}^T\bm{h}
	+ 2a_i \bm{w}_{i\ast}^T \bm{h} +(\bm{w}_{i\ast}^T \bm{h})^2}
	{2\sigma_i^2} }
	\nonumber \\
	=& \prod_i^M \frac{1}{\sqrt{2\pi \sigma_i^2}}
	e^{ - \frac{(x_i - b_i - \bm{w}_{i\ast}^T \bm{h})^2}{2\sigma_i^2}} \nonumber \\
	=& \prod_i^M \mathcal{N}
	(x_i | b_i + \bm{w}_{i\ast}^T \bm{h}, \sigma_i^2) \\
	\Rightarrow p_{GB} (x_i|\bm{h}) =& \mathcal{N}
	(x_i | b_i + \bm{w}_{i\ast}^T \bm{h}, \sigma_i^2) .
\end{align}

The form of these conditional probabilities explains the name
"Gaussian" and the form of the Gaussian-binary energy function. We see
that the conditional probability of $x_i$ given $\bm{h}$ is a normal
distribution with mean $b_i + \bm{w}_{i\ast}^T \bm{h}$ and variance
$\sigma_i^2$.

\subsection*{Neural Quantum States}

The wavefunction should be a probability amplitude depending on $\bm{x}$. The RBM model is given by the joint distribution of $\bm{x}$ and $\bm{h}$
\begin{align}
	F_{rbm}(\bm{x},\mathbf{h}) = \frac{1}{Z} e^{-\frac{1}{T_0}E(\bm{x},\mathbf{h})}
\end{align}
To find the marginal distribution of $\bm{x}$ we set:

\begin{align}
	F_{rbm}(\mathbf{x}) &= \sum_\mathbf{h} F_{rbm}(\mathbf{x}, \mathbf{h}) \\
				&= \frac{1}{Z}\sum_\mathbf{h} e^{-E(\mathbf{x}, \mathbf{h})}
\end{align}

Now this is what we use to represent the wave function, calling it a neural-network quantum state (NQS)

\begin{align}
	\Psi (\mathbf{X}) &= F_{rbm}(\mathbf{x}) \\
	&= \frac{1}{Z}\sum_{\bm{h}} e^{-E(\mathbf{x}, \mathbf{h})} \\
	&= \frac{1}{Z} \sum_{\{h_j\}} e^{-\sum_i^M \frac{(x_i - a_i)^2}{2\sigma^2} + \sum_j^N b_j h_j + \sum_{i,j}^{M,N} \frac{x_i w_{ij} h_j}{\sigma^2}} \\
	&= \frac{1}{Z} e^{-\sum_i^M \frac{(x_i - a_i)^2}{2\sigma^2}} \prod_j^N (1 + e^{b_j + \sum_i^M \frac{x_i w_{ij}}{\sigma^2}}) \\
\end{align}

The above wavefunction is the most general one because it allows for
complex valued wavefunctions. However it fundamentally changes the
probabilistic foundation of the RBM, because what is usually a
probability in the RBM framework is now a an amplitude. This means
that a lot of the theoretical framework usually used to interpret the
model, i.e.~graphical models, conditional probabilities, and Markov
random fields, breaks down. If we assume the wavefunction to be
postive definite, however, we can use the RBM to represent the squared
wavefunction, and thereby a probability. This also makes it possible
to sample from the model using Gibbs sampling, because we can obtain
the conditional probabilities.

\begin{align}
	|\Psi (\mathbf{X})|^2 &= F_{rbm}(\mathbf{X}) \\
	\Rightarrow \Psi (\mathbf{X}) &= \sqrt{F_{rbm}(\mathbf{X})} \\
	&= \frac{1}{\sqrt{Z}}\sqrt{\sum_{\{h_j\}} e^{-E(\mathbf{X}, \mathbf{h})}} \\
	&= \frac{1}{\sqrt{Z}} \sqrt{\sum_{\{h_j\}} e^{-\sum_i^M \frac{(X_i - a_i)^2}{2\sigma^2} + \sum_j^N b_j h_j + \sum_{i,j}^{M,N} \frac{X_i w_{ij} h_j}{\sigma^2}} }\\
	&= \frac{1}{\sqrt{Z}} e^{-\sum_i^M \frac{(X_i - a_i)^2}{4\sigma^2}} \sqrt{\sum_{\{h_j\}} \prod_j^N e^{b_j h_j + \sum_i^M \frac{X_i w_{ij} h_j}{\sigma^2}}} \\
	&= \frac{1}{\sqrt{Z}} e^{-\sum_i^M \frac{(X_i - a_i)^2}{4\sigma^2}} \sqrt{\prod_j^N \sum_{h_j}  e^{b_j h_j + \sum_i^M \frac{X_i w_{ij} h_j}{\sigma^2}}} \\
	&= \frac{1}{\sqrt{Z}} e^{-\sum_i^M \frac{(X_i - a_i)^2}{4\sigma^2}} \prod_j^N \sqrt{e^0 + e^{b_j + \sum_i^M \frac{X_i w_{ij}}{\sigma^2}}} \\
	&= \frac{1}{\sqrt{Z}} e^{-\sum_i^M \frac{(X_i - a_i)^2}{4\sigma^2}} \prod_j^N \sqrt{1 + e^{b_j + \sum_i^M \frac{X_i w_{ij}}{\sigma^2}}} \\
\end{align}

\paragraph{Cost function.}
This is where we deviate from what is common in machine
learning. Rather than defining a cost function based on some dataset,
our cost function is the energy of the quantum mechanical system. From
the variational principle we know that minizing this energy should
lead to the ground state wavefunction. As stated previously the local
energy is given by

\begin{align}
	E_L = \frac{1}{\Psi} \hat{\mathbf{H}} \Psi,
\end{align}
and the gradient is

\begin{align}
	G_i = \frac{\partial \langle E_L \rangle}{\partial \alpha_i}
	= 2(\langle E_L \frac{1}{\Psi}\frac{\partial \Psi}{\partial \alpha_i} \rangle - \langle E_L \rangle \langle \frac{1}{\Psi}\frac{\partial \Psi}{\partial \alpha_i} \rangle ),
\end{align}
where $\alpha_i = a_1,...,a_M,b_1,...,b_N,w_{11},...,w_{MN}$.

We use that $\frac{1}{\Psi}\frac{\partial \Psi}{\partial \alpha_i} 
	= \frac{\partial \ln{\Psi}}{\partial \alpha_i}$,
and find

\begin{align}
	\ln{\Psi({\mathbf{X}})} &= -\ln{Z} - \sum_m^M \frac{(X_m - a_m)^2}{2\sigma^2}
	+ \sum_n^N \ln({1 + e^{b_n + \sum_i^M \frac{X_i w_{in}}{\sigma^2}})}.
\end{align}

This gives

\begin{align}
	\frac{\partial }{\partial a_m} \ln\Psi
	&= 	\frac{1}{\sigma^2} (X_m - a_m) \\
	\frac{\partial }{\partial b_n} \ln\Psi
	&=
	\frac{1}{e^{-b_n-\frac{1}{\sigma^2}\sum_i^M X_i w_{in}} + 1} \\
	\frac{\partial }{\partial w_{mn}} \ln\Psi
	&= \frac{X_m}{\sigma^2(e^{-b_n-\frac{1}{\sigma^2}\sum_i^M X_i w_{in}} + 1)}.
\end{align}

If $\Psi = \sqrt{F_{rbm}}$ we have

\begin{align}
	\ln{\Psi({\mathbf{X}})} &= -\frac{1}{2}\ln{Z} - \sum_m^M \frac{(X_m - a_m)^2}{4\sigma^2}
	+ \frac{1}{2}\sum_n^N \ln({1 + e^{b_n + \sum_i^M \frac{X_i w_{in}}{\sigma^2}})},
\end{align}
which results in

\begin{align}
	\frac{\partial }{\partial a_m} \ln\Psi
	&= 	\frac{1}{2\sigma^2} (X_m - a_m) \\
	\frac{\partial }{\partial b_n} \ln\Psi
	&=
	\frac{1}{2(e^{-b_n-\frac{1}{\sigma^2}\sum_i^M X_i w_{in}} + 1)} \\
	\frac{\partial }{\partial w_{mn}} \ln\Psi
	&= \frac{X_m}{2\sigma^2(e^{-b_n-\frac{1}{\sigma^2}\sum_i^M X_i w_{in}} + 1)}.
\end{align}

Let us assume again that our Hamiltonian is 
\begin{align}
	\hat{\mathbf{H}} = \sum_p^P (-\frac{1}{2}\nabla_p^2 + \frac{1}{2}\omega^2 r_p^2 ) + \sum_{p<q} \frac{1}{r_{pq}},
\end{align}

where the first summation term represents the standard harmonic
oscillator part and the latter the repulsive interaction between two
electrons. Natural units ($\hbar=c=e=m_e=1$) are used, and $P$ is the
number of particles. This gives us the following expression for the
local energy ($D$ being the number of dimensions)

\begin{align}
	E_L &= \frac{1}{\Psi} \mathbf{H} \Psi \\
	&= \frac{1}{\Psi} (\sum_p^P (-\frac{1}{2}\nabla_p^2 + \frac{1}{2}\omega^2 r_p^2 ) + \sum_{p<q} \frac{1}{r_{pq}}) \Psi \\
	&= -\frac{1}{2}\frac{1}{\Psi} \sum_p^P \nabla_p^2 \Psi 
	+ \frac{1}{2}\omega^2 \sum_p^P  r_p^2  + \sum_{p<q} \frac{1}{r_{pq}} \\
	&= -\frac{1}{2}\frac{1}{\Psi} \sum_p^P \sum_d^D \frac{\partial^2 \Psi}{\partial x_{pd}^2} + \frac{1}{2}\omega^2 \sum_p^P  r_p^2  + \sum_{p<q} \frac{1}{r_{pq}} \\
	&= \frac{1}{2} \sum_p^P \sum_d^D (-(\frac{\partial}{\partial x_{pd}} \ln\Psi)^2 -\frac{\partial^2}{\partial x_{pd}^2} \ln\Psi + \omega^2 x_{pd}^2)  + \sum_{p<q} \frac{1}{r_{pq}}. \\
\end{align}

Letting each visible node in the Boltzmann machine 
represent one coordinate of one particle, we obtain

\begin{align}
	E_L &=
	\frac{1}{2} \sum_m^M (-(\frac{\partial}{\partial v_m} \ln\Psi)^2 -\frac{\partial^2}{\partial v_m^2} \ln\Psi + \omega^2 v_m^2)  + \sum_{p<q} \frac{1}{r_{pq}},
\end{align}
where we have that

\begin{align}
	\frac{\partial}{\partial x_m} \ln\Psi
	&= - \frac{1}{\sigma^2}(x_m - a_m) + \frac{1}{\sigma^2} \sum_n^N \frac{w_{mn}}{e^{-b_n - \frac{1}{\sigma^2}\sum_i^M x_i w_{in}} + 1} \\
	\frac{\partial^2}{\partial x_m^2} \ln\Psi
	&= - \frac{1}{\sigma^2} + \frac{1}{\sigma^4}\sum_n^N \omega_{mn}^2 \frac{e^{b_n + \frac{1}{\sigma^2}\sum_i^M x_i w_{in}}}{(e^{b_n + \frac{1}{\sigma^2}\sum_i^M x_i w_{in}} + 1)^2}.
\end{align}

We now have all the expressions neeeded to calculate the gradient of
the expected local energy with respect to the RBM parameters
$\frac{\partial \langle E_L \rangle}{\partial \alpha_i}$.

If we use $\Psi = \sqrt{F_{rbm}}$ we obtain
\begin{align}
	\frac{\partial}{\partial x_m} \ln\Psi
	&= - \frac{1}{2\sigma^2}(x_m - a_m) + \frac{1}{2\sigma^2} \sum_n^N
 	\frac{w_{mn}}{e^{-b_n-\frac{1}{\sigma^2}\sum_i^M x_i w_{in}} + 1}
	\\
	\frac{\partial^2}{\partial x_m^2} \ln\Psi
	&= - \frac{1}{2\sigma^2} + \frac{1}{2\sigma^4}\sum_n^N \omega_{mn}^2 \frac{e^{b_n + \frac{1}{\sigma^2}\sum_i^M x_i w_{in}}}{(e^{b_n + \frac{1}{\sigma^2}\sum_i^M x_i w_{in}} + 1)^2}.
\end{align}
The difference between this equation and the previous one is that we multiply by a factor $1/2$.

\subsection*{Python version for the two non-interacting particles}


































































































































































































































































\begin{minted}[fontsize=\fontsize{9pt}{9pt},linenos=false,mathescape,baselinestretch=1.0,fontfamily=tt,xleftmargin=7mm]{python}
# 2-electron VMC code for 2dim quantum dot with importance sampling
# Using gaussian rng for new positions and Metropolis- Hastings 
# Added restricted boltzmann machine method for dealing with the wavefunction
# RBM code based heavily off of:
# https://github.com/CompPhysics/ComputationalPhysics2/tree/gh-pages/doc/Programs/BoltzmannMachines/MLcpp/src/CppCode/ob
from math import exp, sqrt
from random import random, seed, normalvariate
import numpy as np
import matplotlib.pyplot as plt
from mpl_toolkits.mplot3d import Axes3D
from matplotlib import cm
from matplotlib.ticker import LinearLocator, FormatStrFormatter
import sys



# Trial wave function for the 2-electron quantum dot in two dims
def WaveFunction(r,a,b,w):
    sigma=1.0
    sig2 = sigma**2
    Psi1 = 0.0
    Psi2 = 1.0
    Q = Qfac(r,b,w)
    
    for iq in range(NumberParticles):
        for ix in range(Dimension):
            Psi1 += (r[iq,ix]-a[iq,ix])**2
            
    for ih in range(NumberHidden):
        Psi2 *= (1.0 + np.exp(Q[ih]))
        
    Psi1 = np.exp(-Psi1/(2*sig2))

    return Psi1*Psi2

# Local energy  for the 2-electron quantum dot in two dims, using analytical local energy
def LocalEnergy(r,a,b,w):
    sigma=1.0
    sig2 = sigma**2
    locenergy = 0.0
    
    Q = Qfac(r,b,w)

    for iq in range(NumberParticles):
        for ix in range(Dimension):
            sum1 = 0.0
            sum2 = 0.0
            for ih in range(NumberHidden):
                sum1 += w[iq,ix,ih]/(1+np.exp(-Q[ih]))
                sum2 += w[iq,ix,ih]**2 * np.exp(Q[ih]) / (1.0 + np.exp(Q[ih]))**2
    
            dlnpsi1 = -(r[iq,ix] - a[iq,ix]) /sig2 + sum1/sig2
            dlnpsi2 = -1/sig2 + sum2/sig2**2
            locenergy += 0.5*(-dlnpsi1*dlnpsi1 - dlnpsi2 + r[iq,ix]**2)
            
    if(interaction==True):
        for iq1 in range(NumberParticles):
            for iq2 in range(iq1):
                distance = 0.0
                for ix in range(Dimension):
                    distance += (r[iq1,ix] - r[iq2,ix])**2
                    
                locenergy += 1/sqrt(distance)
                
    return locenergy

# Derivate of wave function ansatz as function of variational parameters
def DerivativeWFansatz(r,a,b,w):
    
    sigma=1.0
    sig2 = sigma**2
    
    Q = Qfac(r,b,w)
    
    WfDer = np.empty((3,),dtype=object)
    WfDer = [np.copy(a),np.copy(b),np.copy(w)]
    
    WfDer[0] = (r-a)/sig2
    WfDer[1] = 1 / (1 + np.exp(-Q))
    
    for ih in range(NumberHidden):
        WfDer[2][:,:,ih] = w[:,:,ih] / (sig2*(1+np.exp(-Q[ih])))
            
    return  WfDer

# Setting up the quantum force for the two-electron quantum dot, recall that it is a vector
def QuantumForce(r,a,b,w):

    sigma=1.0
    sig2 = sigma**2
    
    qforce = np.zeros((NumberParticles,Dimension), np.double)
    sum1 = np.zeros((NumberParticles,Dimension), np.double)
    
    Q = Qfac(r,b,w)
    
    for ih in range(NumberHidden):
        sum1 += w[:,:,ih]/(1+np.exp(-Q[ih]))
    
    qforce = 2*(-(r-a)/sig2 + sum1/sig2)
    
    return qforce
    
def Qfac(r,b,w):
    Q = np.zeros((NumberHidden), np.double)
    temp = np.zeros((NumberHidden), np.double)
    
    for ih in range(NumberHidden):
        temp[ih] = (r*w[:,:,ih]).sum()
        
    Q = b + temp
    
    return Q
    
# Computing the derivative of the energy and the energy 
def EnergyMinimization(a,b,w):

    NumberMCcycles= 10000
    # Parameters in the Fokker-Planck simulation of the quantum force
    D = 0.5
    TimeStep = 0.05
    # positions
    PositionOld = np.zeros((NumberParticles,Dimension), np.double)
    PositionNew = np.zeros((NumberParticles,Dimension), np.double)
    # Quantum force
    QuantumForceOld = np.zeros((NumberParticles,Dimension), np.double)
    QuantumForceNew = np.zeros((NumberParticles,Dimension), np.double)

    # seed for rng generator 
    seed()
    energy = 0.0
    DeltaE = 0.0

    EnergyDer = np.empty((3,),dtype=object)
    DeltaPsi = np.empty((3,),dtype=object)
    DerivativePsiE = np.empty((3,),dtype=object)
    EnergyDer = [np.copy(a),np.copy(b),np.copy(w)]
    DeltaPsi = [np.copy(a),np.copy(b),np.copy(w)]
    DerivativePsiE = [np.copy(a),np.copy(b),np.copy(w)]
    for i in range(3): EnergyDer[i].fill(0.0)
    for i in range(3): DeltaPsi[i].fill(0.0)
    for i in range(3): DerivativePsiE[i].fill(0.0)

    
    #Initial position
    for i in range(NumberParticles):
        for j in range(Dimension):
            PositionOld[i,j] = normalvariate(0.0,1.0)*sqrt(TimeStep)
    wfold = WaveFunction(PositionOld,a,b,w)
    QuantumForceOld = QuantumForce(PositionOld,a,b,w)

    #Loop over MC MCcycles
    for MCcycle in range(NumberMCcycles):
        #Trial position moving one particle at the time
        for i in range(NumberParticles):
            for j in range(Dimension):
                PositionNew[i,j] = PositionOld[i,j]+normalvariate(0.0,1.0)*sqrt(TimeStep)+\
                                       QuantumForceOld[i,j]*TimeStep*D
            wfnew = WaveFunction(PositionNew,a,b,w)
            QuantumForceNew = QuantumForce(PositionNew,a,b,w)
            
            GreensFunction = 0.0
            for j in range(Dimension):
                GreensFunction += 0.5*(QuantumForceOld[i,j]+QuantumForceNew[i,j])*\
                                      (D*TimeStep*0.5*(QuantumForceOld[i,j]-QuantumForceNew[i,j])-\
                                      PositionNew[i,j]+PositionOld[i,j])
      
            GreensFunction = exp(GreensFunction)
            ProbabilityRatio = GreensFunction*wfnew**2/wfold**2
            #Metropolis-Hastings test to see whether we accept the move
            if random() <= ProbabilityRatio:
                for j in range(Dimension):
                    PositionOld[i,j] = PositionNew[i,j]
                    QuantumForceOld[i,j] = QuantumForceNew[i,j]
                wfold = wfnew
        #print("wf new:        ", wfnew)
        #print("force on 1 new:", QuantumForceNew[0,:])
        #print("pos of 1 new:  ", PositionNew[0,:])
        #print("force on 2 new:", QuantumForceNew[1,:])
        #print("pos of 2 new:  ", PositionNew[1,:])
        DeltaE = LocalEnergy(PositionOld,a,b,w)
        DerPsi = DerivativeWFansatz(PositionOld,a,b,w)
        
        DeltaPsi[0] += DerPsi[0]
        DeltaPsi[1] += DerPsi[1]
        DeltaPsi[2] += DerPsi[2]
        
        energy += DeltaE

        DerivativePsiE[0] += DerPsi[0]*DeltaE
        DerivativePsiE[1] += DerPsi[1]*DeltaE
        DerivativePsiE[2] += DerPsi[2]*DeltaE
            
    # We calculate mean values
    energy /= NumberMCcycles
    DerivativePsiE[0] /= NumberMCcycles
    DerivativePsiE[1] /= NumberMCcycles
    DerivativePsiE[2] /= NumberMCcycles
    DeltaPsi[0] /= NumberMCcycles
    DeltaPsi[1] /= NumberMCcycles
    DeltaPsi[2] /= NumberMCcycles
    EnergyDer[0]  = 2*(DerivativePsiE[0]-DeltaPsi[0]*energy)
    EnergyDer[1]  = 2*(DerivativePsiE[1]-DeltaPsi[1]*energy)
    EnergyDer[2]  = 2*(DerivativePsiE[2]-DeltaPsi[2]*energy)
    return energy, EnergyDer


#Here starts the main program with variable declarations
NumberParticles = 2
Dimension = 2
NumberHidden = 2

interaction=False

# guess for parameters
a=np.random.normal(loc=0.0, scale=0.001, size=(NumberParticles,Dimension))
b=np.random.normal(loc=0.0, scale=0.001, size=(NumberHidden))
w=np.random.normal(loc=0.0, scale=0.001, size=(NumberParticles,Dimension,NumberHidden))
# Set up iteration using stochastic gradient method
Energy = 0
EDerivative = np.empty((3,),dtype=object)
EDerivative = [np.copy(a),np.copy(b),np.copy(w)]
# Learning rate eta, max iterations, need to change to adaptive learning rate
eta = 0.001
MaxIterations = 50
iter = 0
np.seterr(invalid='raise')
Energies = np.zeros(MaxIterations)
EnergyDerivatives1 = np.zeros(MaxIterations)
EnergyDerivatives2 = np.zeros(MaxIterations)

while iter < MaxIterations:
    Energy, EDerivative = EnergyMinimization(a,b,w)
    agradient = EDerivative[0]
    bgradient = EDerivative[1]
    wgradient = EDerivative[2]
    a -= eta*agradient
    b -= eta*bgradient 
    w -= eta*wgradient 
    Energies[iter] = Energy
    print("Energy:",Energy)
    #EnergyDerivatives1[iter] = EDerivative[0] 
    #EnergyDerivatives2[iter] = EDerivative[1]
    #EnergyDerivatives3[iter] = EDerivative[2] 


    iter += 1

#nice printout with Pandas
import pandas as pd
from pandas import DataFrame
pd.set_option('max_columns', 6)
data ={'Energy':Energies}#,'A Derivative':EnergyDerivatives1,'B Derivative':EnergyDerivatives2,'Weights Derivative':EnergyDerivatives3}

frame = pd.DataFrame(data)
print(frame)

\end{minted}

\part{Quantum Computing}   

 \bibliographystyle{unsrt}

\backmatter%%%%%%%%%%%%%%%%%%%%%%%%%%%%%%%%%%%%%%%%%%%%%%%%%%%%%%%
%\include{glossary}
%\include{solutions}
\printindex



\end{document}





