%%
%% Automatically generated file from DocOnce source
%% (https://github.com/doconce/doconce/)
%% doconce format latex basicquantumcomputing.do.txt --no_abort
%%


%-------------------- begin preamble ----------------------

\documentclass[%
oneside,                 % oneside: electronic viewing, twoside: printing
final,                   % draft: marks overfull hboxes, figures with paths
10pt]{article}

\listfiles               %  print all files needed to compile this document

\usepackage{relsize,makeidx,color,setspace,amsmath,amsfonts,amssymb}
\usepackage[table]{xcolor}
\usepackage{bm,ltablex,microtype}

\usepackage[pdftex]{graphicx}

\usepackage[T1]{fontenc}
%\usepackage[latin1]{inputenc}
\usepackage{ucs}
\usepackage[utf8x]{inputenc}

\usepackage{lmodern}         % Latin Modern fonts derived from Computer Modern

% Hyperlinks in PDF:
\definecolor{linkcolor}{rgb}{0,0,0.4}
\usepackage{hyperref}
\hypersetup{
    breaklinks=true,
    colorlinks=true,
    linkcolor=linkcolor,
    urlcolor=linkcolor,
    citecolor=black,
    filecolor=black,
    %filecolor=blue,
    pdfmenubar=true,
    pdftoolbar=true,
    bookmarksdepth=3   % Uncomment (and tweak) for PDF bookmarks with more levels than the TOC
    }
%\hyperbaseurl{}   % hyperlinks are relative to this root

\setcounter{tocdepth}{2}  % levels in table of contents

% prevent orhpans and widows
\clubpenalty = 10000
\widowpenalty = 10000

% --- end of standard preamble for documents ---


% insert custom LaTeX commands...

\raggedbottom
\makeindex
\usepackage[totoc]{idxlayout}   % for index in the toc
\usepackage[nottoc]{tocbibind}  % for references/bibliography in the toc

%-------------------- end preamble ----------------------

\begin{document}

% matching end for #ifdef PREAMBLE

\newcommand{\exercisesection}[1]{\subsection*{#1}}


% ------------------- main content ----------------------

\section{Quantum Computing}

\textbf{text to come}

\begin{equation}
   \Psi(x,t)=\Psi(x,t)_a+\Psi(x,t)_b,
\end{equation}

\begin{equation}
    |q\rangle = \alpha \vert 0 \rangle + \beta |1\rangle,
\end{equation}

\begin{equation}
    |q\rangle = \frac{1}{\sqrt{2}}(\vert 0 \rangle + |1\rangle),
\end{equation}

\begin{equation}
    |\psi\rangle = a_0|\psi\rangle_0+a_1|\psi\rangle_1+a_2|\psi\rangle_2+a_3|\psi\rangle_3+a_4|\psi\rangle_4+a_5|\psi\rangle_5+a_6|\psi\rangle_6+a_7|\psi\rangle_7,
\label{eq:dataord}
\end{equation}

\begin{equation}
    \hat{H}\vert 0 \rangle=
    \frac{1}{\sqrt{2}}(\vert 0 \rangle +|1\rangle),
\end{equation}

\begin{equation}
    \hat{H}|1\rangle=
    \frac{1}{\sqrt{2}}(\vert 0 \rangle -|1\rangle).
\end{equation}

\begin{equation}
    \hat{H}\hat{H}\vert 0 \rangle=
    \hat{H}\frac{1}{\sqrt{2}}(\vert 0 \rangle +|1\rangle)
    =\vert 0 \rangle,
\end{equation}

\begin{equation}
    \hat{H}\hat{H}|1\rangle=
    \hat{H}\frac{1}{\sqrt{2}}(\vert 0 \rangle -|1\rangle)
    =|1\rangle.
\end{equation}

\begin{equation}
   \vert 0 \rangle=\left(\begin{array}{c} 1 \\ 0\end{array}   \right),
\end{equation}

\begin{equation}
   |1\rangle=\left(\begin{array}{c} 0 \\ 1\end{array}   \right),
\end{equation}
og

\begin{equation}
   \hat{H}=\frac{1}{\sqrt{2}}
   \left(\begin{array}{cc} 1 & 1 \\ 1& -1\end{array}   \right).
\end{equation}

\begin{equation}
   \hat{H}_{\mathrm{NOT}}=   
   \left(\begin{array}{cc} 0 & 1 \\ 1& 0\end{array}   \right).
\end{equation}

\begin{equation}
   \hat{H}_{\mathrm{NOT}}\vert 0 \rangle=   
   \left(\begin{array}{cc} 0 & 1 \\ 1& 0\end{array}\right)
   \left(\begin{array}{c} 1 \\ 0\end{array}   \right)=
   \left(\begin{array}{c} 0 \\ 1\end{array}   \right)=|1\rangle,
\end{equation}

\begin{equation}
   \hat{H}_{\mathrm{NOT}}|1\rangle=   
   \left(\begin{array}{cc} 0 & 1 \\ 1& 0\end{array}\right)
   \left(\begin{array}{c} 0 \\ 1\end{array}   \right)=
   \left(\begin{array}{c} 1 \\ 0\end{array}   \right)=\vert 0 \rangle.
\end{equation}

\begin{equation}
   \hat{H}_{\Phi}\vert 0 \rangle=e^{i\phi}\vert 0 \rangle,
\end{equation}
og

\begin{equation}
   \hat{H}_{\Phi}|1\rangle=|1\rangle,
\end{equation}

\begin{equation}
   \hat{H}_{\Phi}=   
   \left(\begin{array}{cc} e^{i\phi} & 0 \\ 0& 1\end{array}   \right).
\end{equation}

\begin{equation}
    |q\rangle_{in} = \alpha \vert 0 \rangle_{in} + \beta |1\rangle_{in},
\end{equation}

\begin{equation}
    |q\rangle_{ut}=\hat{H}|q\rangle_{in} = 
    \frac{1}{\sqrt{2}}((\alpha+\beta)\vert 0 \rangle_{ut} + 
    (\alpha-\beta)|1\rangle_{ut}),
\end{equation}

\begin{equation} 
   |q\rangle_{ut}=\hat{H}\hat{H}|q\rangle_{in}=|q\rangle_{in}.
\end{equation}

\begin{equation} 
   \hat{H}\hat{\Phi}\hat{H}\vert 0 \rangle,
\end{equation}

\begin{equation}
    \frac{1}{\sqrt{2}}((e^{i\phi}+1)\vert 0 \rangle + 
    (e^{i\phi}-1)|1\rangle).
\end{equation}

\begin{equation}
   \frac{\vert 0 \rangle_1 |1\rangle_2 + |1\rangle_1\vert 0 \rangle_2}{\sqrt{2}},
   \label{eq:superent}
\end{equation}

\begin{equation}
   \frac{|+\rangle_1 |-\rangle_2 - |-\rangle_1|+\rangle_2}{\sqrt{2}},
   \label{eq:superent1}
\end{equation}

\begin{equation}
    |q\rangle_{1} = \alpha_1 \vert 0 \rangle_{1} + \beta_1 |1\rangle_{1},
\end{equation}

\begin{equation}
     |q\rangle_{12}= |q\rangle_{1} |q\rangle_{2}
\end{equation}
eller

\begin{equation}
   |q\rangle_{12}=\alpha_1 \alpha_2\vert 0 \rangle_1\vert 0 \rangle_2 +
                  \alpha_1 \beta_2\vert 0 \rangle_1|1\rangle_2 +
                  \beta_1 \alpha_2|1\rangle_1\vert 0 \rangle_2 +
                  \beta_1 \beta_2|1\rangle_1|1\rangle_2,
\end{equation}

\begin{equation} 
   |q\rangle_{ut}=\hat{H}|q\rangle_{in},
\end{equation}
$\hat{H}^{\dagger}=\hat{H}$ og $\hat{H}^{\dagger}\hat{H}=1$, 

\begin{equation} 
   \hat{H}^{\dagger}|q\rangle_{ut}=\hat{H}^{\dagger}\hat{H}|q\rangle_{in}=
   |q\rangle_{in}.
\end{equation}

\begin{equation}
   \vert 0 \rangle_{c}  \vert 0 \rangle_{t}\rightarrow
   \vert 0 \rangle_{c}  \vert 0 \rangle_{t},
\end{equation}

\begin{equation}
   \vert 0 \rangle_{c}  |1\rangle_{t}\rightarrow
   \vert 0 \rangle_{c}  |1\rangle_{t},
\end{equation}

\begin{equation}
   |1\rangle_{c}  \vert 0 \rangle_{t}\rightarrow
   |1\rangle_{c}  |1\rangle_{t},
\end{equation}

\begin{equation}
   |1\rangle_{c}  |1\rangle_{t}\rightarrow
   |1\rangle_{c}  \vert 0 \rangle_{t}.
\end{equation}

\begin{equation}
   \hat{H}_{\mathrm{CNOT}}=
    \left(\begin{array}{cccc} 1 & 0 & 0 &0  \\ 
                              0 & 1 & 0 &0  \\
                              0& 0 & 0 &1  \\
                              0 & 0 & 1 &0\end{array}\right),
\end{equation}
$\vert 0 \rangle_c\vert 0 \rangle_t$, $\vert 0 \rangle_c|1\rangle_t$, $|1\rangle_c\vert 0 \rangle_t$ 
$|1\rangle_c|1\rangle_t$.

\begin{equation}
   \vert 0 \rangle_c\vert 0 \rangle_t=\left(\begin{array}{c} 1  \\ 
                                                 0  \\
                                                 0  \\
                                                 0  \end{array}\right),
\end{equation}

\begin{equation}
   \vert 0 \rangle_c|1\rangle_t=\left(\begin{array}{c} 0  \\ 
                                                 1  \\
                                                 0  \\
                                                 0  \end{array}\right),
\end{equation}

\begin{equation}
   |1\rangle_c\vert 0 \rangle_t=\left(\begin{array}{c} 0  \\ 
                                                 0  \\
                                                 1  \\
                                                 0  \end{array}\right),
\end{equation}
og

\begin{equation}
   |1\rangle_c|1\rangle_t=\left(\begin{array}{c} 0  \\ 
                                                 0  \\
                                                 0  \\
                                                 1  \end{array}\right).
\end{equation}

\begin{equation}
   \hat{H}_{\mathrm{CNOT}}|1\rangle_c|1\rangle_t=
    \left(\begin{array}{cccc} 1 & 0 & 0 &0  \\ 
                              0 & 1 & 0 &0  \\
                              0& 0 & 0 &1  \\
                              0 & 0 & 1 &0\end{array}\right)
                              \left(\begin{array}{c} 0  \\ 
                                                 0  \\
                                                 0  \\
                                                 1  \end{array}\right)=
\left(\begin{array}{c} 0  \\ 
                                                 0  \\
                                                 1  \\
                                                 0  \end{array}\right),
\end{equation}

$|1\rangle_c\vert 0 \rangle_t$, dvs.~target-biten forandrer verdi fra bit '1'
til bit '0' når kontroll-biten har verdi bit '1'.

\subsection{Hamiltonians}

A general two-body Hamiltonian for fermionic system 
can be written as

\begin{equation}
\label{eq:twobodyH}
H = E_0 + \sum_{ij=1} E_{ij} a^\dag_i a_j
+\sum_{ijkl = 1} V_{ijkl} a^\dag_i a^\dag_j a_l a_k,
\end{equation}
where $E_0$ is a constant energy term, $E_{ij}$ represent all the
one-particle terms, allowing for non-diagonal terms as well. The
one-body term can represent 
a chosen single-particle potential, the kinetic energy or other more
specialized terms such as  
those discussed in connection with the Hubbard model or the pairing Hamiltonian  
discussed below.
The two-body interaction part is given by $V_{ijkl}$ and can be any
two-body interaction, from 
Coulomb interaction to the interaction between nucleons.  
The sums run over all possible single-particle levels $N$. 
Note that
this model includes particle numbers from zero to the number of
available quantum levels, $n$. To simulate states with fixed numbers
of fermions one would have to either rewrite the Hamiltonian or
generate specialized input states in the simulation.

The algorithm which we will develop in this section and in
However, 
in our demonstrations of the quantum computing algorithm, we will limit ourselves to
two simple models, which however capture much of the important physics
in quantum mechanical 
many-body systems. We will also limit ourselves to spin $j=1/2$
systems, although our algorithm 
can also simulate  higher $j$-values, such as those which occur in nuclear, atomic and
molecular physics, it simply uses one qubit for every available
quantum state.   
These simple models are the Hubbard model and a pairing
Hamiltonian.
We start with the spin $1/2$ Hubbard model, described by the following Hamiltonian

\begin{align}
H_H &&= \epsilon \sum_{i, \sigma} a_{i\sigma}^\dag a_{i\sigma} 
-t \sum_{i, \sigma} \left(a^\dag_{i+1, \sigma}a_{i, \sigma}
+a^\dag_{i, \sigma}a_{i+1, \sigma} \right) \notag \\
&& + U \sum_{i=1} a_{i+}^\dag a_{i-}^\dag a_{i-}a_{i+},
\label{eq:hubbard}
\end{align}
where $a^{\dagger}$ and $a$ are fermion creation and annihilation operators, respectively.
This is a chain of sites where each site has room for one spin up
fermion and one spin down fermion. 
The number of sites is $N$, and the sums over $\sigma$ are sums over
spin up and down only.
Each site has a single-particle
energy $\epsilon$. There is a repulsive term $U$ if there is a pair
of particles at the same site. It is energetically favourable to tunnel to
neighbouring sites, described by 
the hopping terms with coupling constant $-t$.

The second model-Hamiltonian is the simple pairing Hamiltonian
\begin{equation}
   H_P=\sum_i \varepsilon_i a^{\dagger}_i a_i -\frac{1}{2} g\sum_{ij>0}
           a^{\dagger}_{i}
     a^{\dagger}_{\bar{\imath}}a_{\bar{\jmath}}a_{j},
     \label{eq:pairing}
\end{equation}
 The indices $i$ and $j$ run over the number of levels $N$, and the label $\bar{\imath}$ 
stands for a time-reversed state. The parameter $g$ is the strength of the pairing force 
while $\varepsilon_i$ is the single-particle energy of level $i$. 
In our case
we assume that the single-particle levels are equidistant (or
degenerate) with a fixed spacing $d$. 
Moreover, in our simple model, the degeneracy of the single-particle
levels is set to $2j+1=2$, with $j=1/2$  
being the spin of the particle. This gives a set of single-particle
states with the same spin projections as 
for the Hubbard model.  Whereas in the Hubbard model we operate with
different sites with  
spin up or spin down particles, our pairing models deals thus with
levels  with double degeneracy. 
Introducing the pair-creation operator 
$S^+_i=a^{\dagger}_{im}a^{\dagger}_{i-m}$,
one can rewrite the Hamiltonian in 
Eq.~(\ref{eq:pairing}) as

\begin{equation*}
   H_P=d\sum_iiN_i+
     \frac{1}{2} G\sum_{ij>0}S^+_iS^-_j,
\end{equation*}
where  $N_i=a^{\dagger}_i a_i$
is the number operator, and 
$\varepsilon_i = id$ so that the single-particle orbitals 
are equally spaced at intervals $d$. The latter commutes with the 
Hamiltonian $H$. In this model, quantum numbers like seniority 
$\cal{S}$ are good quantum numbers, and the eigenvalue problem 
can be rewritten in terms of blocks with good seniority. 
Loosely 
speaking, the seniority quantum number $\cal{S}$ is equal to 
the number of unpaired particles.
Furthermore, in  a series of papers, Richardson
obtained the exact solution of the pairing Hamiltonian, with 
semi-analytic (since there is still the need for a numerical solution) 
expressions for the eigenvalues and eigenvectors. The exact solutions
have had important consequences for several fields, from Bose condensates to
nuclear superconductivity and is currently a very active field of studies, see for example
Finally, for particle numbers up to $P \sim 20$, the above model can be 
solved exactly through numerical diagonalization and one can obtain all eigenvalues.
It serves therefore also as an excellent ground for comparison with our algorithm based
on models from quantum computing.

\subsection{Basic quantum  gates}

Benioff showed that one could make a quantum mechanical Turing machine
by using various  unitary operations on a quantum system.
Benioff  demonstrated 
that a quantum computer can calculate anything a
classical computer can. To do this one needs a quantum system and
basic operations that can approximate all unitary operations
on the chosen many-body system. We describe in this subsection the basic ingredients entering 
our algorithms.

\paragraph{Qubits, gates and circuits.}
In this article we will use the standard model of quantum information,
where
the basic unit of information is the qubit, the quantum bit. 
As mentioned in the introduction, any
suitable 
two-level quantum system can be a qubit, 
it is the smallest system there is with the
least complex dynamics.
Qubits are both abstract measures of information and physical objects.
Actual physical qubits can be ions trapped in magnetic fields where
lasers can access only two energy levels or  the nuclear spins of some of
the atoms in molecules accessed and manipulated by an NMR machine.
Several other ideas have been proposed and some tested.

The computational basis for one qubit is ${\ensuremath{\vert 0 \rangle}}$ (representing for example bit $0$) 
for the first state
and ${\ensuremath{|1\rangle}}$ (representing bit $1$) for the second, and for a set of qubits  
the tensor products of
these basis states for each qubit form a product basis. Below we write out the different
basis states for a system of $n$ qubits.

\begin{align}
\label{eq:compBasis}
&{\ensuremath{\vert 0 \rangle}} \equiv {\ensuremath{\vert 00\cdots 0 \rangle}} =
      {\ensuremath{\vert 0 \rangle}} \otimes {\ensuremath{| 0 \rangle}} \otimes
          \cdots 
\otimes {\ensuremath{\vert 0 \rangle}} 
\notag \\
&{\ensuremath{|1\rangle}} \equiv {\ensuremath{\vert 00\cdots 1\rangle}} =
    {\ensuremath{\vert 0 \rangle}} \otimes {\ensuremath{| 0 \rangle}} \otimes
        \cdots 
\otimes {\ensuremath{|1\rangle}} 
\notag \\
&\vdots \notag \\
&{\ensuremath{|2^n-1\rangle}} \equiv {\ensuremath{|11\cdots 1\rangle}} =
    {\ensuremath{|1\rangle}} \otimes {\ensuremath{| 1\rangle}} \otimes 
\cdots \otimes {\ensuremath{|1 \rangle}}.
\notag \\
\end{align}
This is a $2^n$-dimensional system and we number the different basis
states using binary numbers corresponding to the order in which they appear in the
tensor product.

Quantum computing means to  manipulate and measure  qubits in such a
way that the results from a measurement yield the solutions to  a given problem. 
The quantum operations we need to be able to perform our simulations are 
a small set of elementary single-qubit
operations, or single-qubit gates, and one universal two-qubit gate,
in our case the so-called CNOT gate defined below.

To represent quantum computer algorithms graphically we use circuit
diagrams. In a circuit diagram each qubit is represented by a line,
and operations on the different qubits are represented by boxes.

\subsection{Number of work qubits versus number of simulation qubits}

The largest possible amount of different eigenvalues is $2^s$, where
$s$ is the number of simulation qubits. The resolution in the energy
spectrum we get from measuring upon the work qubits is $2^w$, with $w$ the number of
work qubits.
Therefore the resolution per eigenvalue in a non-degenerate system is
$2^{w-s}$. The higher the degeneracy the less work qubits are needed. 

\subsection{Number of operations}

Counting the number of single-qubit and $\sigma_z\sigma_z$ operations
for different sizes of systems simulated gives us an indication of the
decoherence time needed for different physical realizations of a
quantum simulator or computer. The decoherence time is an average time
in which the state of the qubits will be destroyed by noise, also called 
decoherence, while the operation time is the average time an operation takes
to perform on the given system. Their fraction is the number of
operations possible to perform before decoherence destroys the
computation. In table  we have listed the number of
gates used for the pairing model, $H_P$, and the Hubbard model, $H_H$,
for different number of simulation qubits. 


\begin{quote}
\begin{tabular}{lcccccc}
\hline
\multicolumn{1}{c}{  } & \multicolumn{1}{c}{ $s=2$ } & \multicolumn{1}{c}{ $s=4$ } & \multicolumn{1}{c}{ $s=6$ } & \multicolumn{1}{c}{ $s=8$ } & \multicolumn{1}{c}{ $s=10$ } & \multicolumn{1}{c}{ $s=12$ } \\
\hline
$H_P$ & 9     & 119   & 333   & 651   & 1073   & 1598   \\
$H_H$ & 9     & 51    & 93    & 135   & 177    & 219    \\
\hline
\end{tabular}
\end{quote}

\noindent
Number of two-qubit gates used in simulating the time
  evolution operator of the pairing model, $H_P$, and the Hubbard
  model, $H_H$, for different number of simulation qubits $s$.

We list here some useful relations involving different $\sigma$ matrices,

\begin{equation}
\sigma_x \sigma_z = -i\sigma_y, \quad
\sigma_z \sigma_x = i\sigma_y, \quad [\sigma_x, \sigma_z]=-2i\sigma_y,
\end{equation}

\begin{equation}
\sigma_x \sigma_y = i\sigma_z, \quad
\sigma_y \sigma_x = -i\sigma_z, \quad [\sigma_x, \sigma_y]=2i\sigma_z,
\end{equation}
and

\begin{equation}
\sigma_y \sigma_z = i\sigma_x, \quad
\sigma_z \sigma_y = -i\sigma_x, \quad [\sigma_y, \sigma_z]=2i\sigma_x.
\end{equation}

For any two non-equal $\sigma$-matrices $a$ and $b$ we have

\begin{equation}
aba = -b.
\end{equation}

The Hermitian $\sigma$-matrices $\sigma_x$, $\sigma_y$ and $\sigma_z$
result in the identity matrix when squared

\begin{equation}
\sigma_x^2 = _1_,\quad 
\sigma_y^2 = _1_,\quad 
\sigma_z^2 = _1_,\quad 
\end{equation}
which can be used to obtain  simplified expressions for exponential functions involving $\sigma$-matrices

\begin{equation}
e^{\pm i\alpha \sigma}=\cos(\alpha) _1_ \pm i \sin(\alpha) \sigma. 
\end{equation}

The equations we list below are necessary for the relation between  a general unitary
transformation on a set of qubits with  a product of two-qubit unitary
transformations. We have the general equation for $a,b \in \{\sigma_x,\sigma_y, \sigma_z\}$, where $a\neq b$.

\begin{align}
  e^{-i\pi/4a} b e^{i\pi/4a} &&= \frac{1}{2} (_1_ -ia) b ( _1_ + ia)
  \notag\\ 
  &&=  \frac{1}{2} (b + aba + i[b,a]) \notag\\
  &&= \frac{i}{2}[b,a].
\end{align}

The more specialized equations read
\begin{align}
  \label{eq:rotations1}
  &&e^{-i\pi/4 \sigma_x} \sigma_z e^{i\pi/4 \sigma_x} = -\sigma_y, \\
  \label{eq:rotations2}
  &&e^{-i\pi/4 \sigma_y} \sigma_z e^{i\pi/4 \sigma_y} = \sigma_x, \\
  \label{eq:rotations3}
  &&e^{-i\pi/4 \sigma_z} \sigma_x e^{i\pi/4 \sigma_z} = \sigma_y, \\
  \label{eq:rotations4}
  &&e^{-i\pi/4 \sigma_z} \sigma_y e^{i\pi/4 \sigma_z} = -\sigma_x. 
\end{align}

We need also different products of the operator$\sigma_z$  with the  raising and lowering operators

\begin{align}
\label{eq:pmzs}
  &&\sigma_+ \sigma_z = -\sigma_+ \\
  &&\sigma_z \sigma_+ = \sigma_+, \\
  &&\sigma_- \sigma_z = \sigma_-, \\
  &&\sigma_z \sigma_- = -\sigma_-. \\
\end{align}

} \nonumber \\
&=\bra{\psi}\left[\left(\bigotimes_{i=1}^nR_{\sigma_i}^{\dagger}\right)Z^{\otimes n}\left(\bigotimes_{i=1}^nR_{\sigma_i}\right)\right]\ket{\psi} \nonumber \\
&=\bra{\bigotimes_{i=1}^nR_{\sigma_i}\psi}Z^{\otimes n}\ket{\bigotimes_{i=1}^nR_{\sigma_i}\psi} \nonumber \\
&=\sum_{x\in\{0,1\}^n}(-1)^{h(x)}P\left(\bigotimes_{i=1}^nR_{\sigma_i}\ket{\psi}=\ket{x}^{\otimes n}\right)
\end{align}
since the expectation value of a string of $Z$'s is calculated to be
\begin{align}
E_{\psi}(Z^{\otimes n})
&=\bra{\psi}Z^{\otimes n}\ket{\psi} \nonumber \\
&=\bra{\psi}\left(\ket{0}\bra{0}-\ket{1}\bra{1}\right)^{\otimes n}\ket{\psi} \nonumber \\
&=\bra{\psi}\left(\sum_{x\in\{0,1\}^n}(-1)^{h(x)}\ket{x}\bra{x}\right)\ket{\psi} \nonumber \\
&=\sum_{x\in\{0,1\}^n}(-1)^{h(x)}\left|\bra{x}\ket{\psi}\right|^2 \nonumber \\
&=\sum_{x\in\{0,1\}^n}(-1)^{h(x)}P(\ket{\psi}=\ket{x})
\end{align}
where $h(x)$ denotes the Hamming weight of bit string $x$ (the number of $1$'s in $x$).

The technique is finally extended to sum of tensor strings $\bigotimes_{i=1}^n\sigma_i$ to include the identity ($\sigma_i\in\{I,X,Y,Z\}$ for all $i$.)
\begin{align}
&E_{\psi}\left(\sum_{j=1}^m\bigotimes_{i=1}^n\sigma_{ij}\right) \\
=&
\sum_{j=1}^mE_{\psi}\left(\bigotimes_{i=1}^n\sigma_{ij}\right) \\
=&\sum_{j=1}^m\sum_{x\in\{0,1\}^n}(-1)^{h(x')}P\left(\bigotimes_{i=1}^nR_{\sigma_{ij}}\ket{\psi}=\ket{x}^{\otimes n}\right)
\end{align}
where $x'$ is defined to be $x$, except that all of the qubits that had $I$ applied to them, are set to zero and $R(I)=I$.

The probabilities are estimated from measurements.

% ------------------- end of main content ---------------

\end{document}

