\chapter{Parallelization with MPI and OpenMPI}

\subsection*{How much is parallelizable}

% --- begin paragraph admon ---
\paragraph{}
If any non-parallel code slips into the
application, the parallel
performance is limited. 

In many simulations, however, the fraction of non-parallelizable work
is $10^{-6}$ or less due to large arrays or objects that are perfectly parallelizable.
% --- end paragraph admon ---



\subsection*{Today's situation of parallel computing}

% --- begin paragraph admon ---
\paragraph{}

\begin{itemize}
\item Distributed memory is the dominant hardware configuration. There is a large diversity in these machines, from  MPP (massively parallel processing) systems to clusters of off-the-shelf PCs, which are very cost-effective.

\item Message-passing is a mature programming paradigm and widely accepted. It often provides an efficient match to the hardware. It is primarily used for the distributed memory systems, but can also be used on shared memory systems.

\item Modern nodes have nowadays several cores, which makes it interesting to use both shared memory (the given node) and distributed memory (several nodes with communication). This leads often to codes which use both MPI and OpenMP.
\end{itemize}

\noindent
Our lectures will focus on both MPI and OpenMP.
% --- end paragraph admon ---



\subsection*{Overhead present in parallel computing}

% --- begin paragraph admon ---
\paragraph{}

\begin{itemize}
\item \textbf{Uneven load balance}:  not all the processors can perform useful work at all time.

\item \textbf{Overhead of synchronization}

\item \textbf{Overhead of communication}

\item \textbf{Extra computation due to parallelization}
\end{itemize}

\noindent
Due to the above overhead and that certain parts of a sequential
algorithm cannot be parallelized we may not achieve an optimal parallelization.
% --- end paragraph admon ---



\subsection*{Parallelizing a sequential algorithm}

% --- begin paragraph admon ---
\paragraph{}

\begin{itemize}
\item Identify the part(s) of a sequential algorithm that can be  executed in parallel. This is the difficult part,

\item Distribute the global work and data among $P$ processors.
\end{itemize}

\noindent
% --- end paragraph admon ---



\subsection*{Strategies}

% --- begin paragraph admon ---
\paragraph{}
\begin{itemize}
\item Develop codes locally, run with some few processes and test your codes.  Do benchmarking, timing and so forth on local nodes, for example your laptop or PC. 

\item When you are convinced that your codes run correctly, you can start your production runs on available supercomputers.
\end{itemize}

\noindent
% --- end paragraph admon ---



\subsection*{How do I run MPI on a PC/Laptop? MPI}

% --- begin paragraph admon ---
\paragraph{}
To install MPI is rather easy on hardware running unix/linux as operating systems, follow simply the instructions from the \href{{https://www.open-mpi.org/}}{OpenMPI website}. See also subsequent slides.
When you have made sure you have installed MPI on your PC/laptop, 
\begin{itemize}
\item Compile with mpicxx/mpic++ or mpif90
\end{itemize}

\noindent





\begin{minted}[fontsize=\fontsize{9pt}{9pt},linenos=false,mathescape,baselinestretch=1.0,fontfamily=tt,xleftmargin=7mm]{c++}
  # Compile and link
  mpic++ -O3 -o nameofprog.x nameofprog.cpp
  #  run code with for example 8 processes using mpirun/mpiexec
  mpiexec -n 8 ./nameofprog.x

\end{minted}
% --- end paragraph admon ---



\subsection*{Can I do it on my own PC/laptop? OpenMP installation}

% --- begin paragraph admon ---
\paragraph{}
If you wish to install MPI and OpenMP 
on your laptop/PC, we recommend the following:

\begin{itemize}
\item For OpenMP, the compile option \textbf{-fopenmp} is included automatically in recent versions of the C++ compiler and Fortran compilers. For users of different Linux distributions, simply use the available C++ or Fortran compilers and add the above compiler instructions, see also code examples below.

\item For OS X users however, install \textbf{libomp}
\end{itemize}

\noindent


\begin{minted}[fontsize=\fontsize{9pt}{9pt},linenos=false,mathescape,baselinestretch=1.0,fontfamily=tt,xleftmargin=7mm]{c++}
  brew install libomp

\end{minted}

and compile and link as


\begin{minted}[fontsize=\fontsize{9pt}{9pt},linenos=false,mathescape,baselinestretch=1.0,fontfamily=tt,xleftmargin=7mm]{c++}
c++ -o <name executable> <name program.cpp>  -lomp

\end{minted}
% --- end paragraph admon ---



\subsection*{Installing MPI}

% --- begin paragraph admon ---
\paragraph{}
For linux/ubuntu users, you need to install two packages (alternatively use the synaptic package manager)



\begin{minted}[fontsize=\fontsize{9pt}{9pt},linenos=false,mathescape,baselinestretch=1.0,fontfamily=tt,xleftmargin=7mm]{c++}
  sudo apt-get install libopenmpi-dev
  sudo apt-get install openmpi-bin

\end{minted}

For OS X users, install brew (after having installed xcode and gcc, needed for the 
gfortran compiler of openmpi) and then install with brew


\begin{minted}[fontsize=\fontsize{9pt}{9pt},linenos=false,mathescape,baselinestretch=1.0,fontfamily=tt,xleftmargin=7mm]{c++}
   brew install openmpi

\end{minted}

When running an executable (code.x), run as


\begin{minted}[fontsize=\fontsize{9pt}{9pt},linenos=false,mathescape,baselinestretch=1.0,fontfamily=tt,xleftmargin=7mm]{c++}
  mpirun -n 10 ./code.x

\end{minted}

where we indicate that we want  the number of processes to be 10.
% --- end paragraph admon ---



\subsection*{Installing MPI and using Qt}

% --- begin paragraph admon ---
\paragraph{}
With openmpi installed, when using Qt, add to your .pro file the instructions \href{{http://dragly.org/2012/03/14/developing-mpi-applications-in-qt-creator/}}{here}

You may need to tell Qt where openmpi is stored.
% --- end paragraph admon ---



\subsection*{What is Message Passing Interface (MPI)?}

% --- begin paragraph admon ---
\paragraph{}

\textbf{MPI} is a library, not a language. It specifies the names, calling sequences and results of functions
or subroutines to be called from C/C++ or Fortran programs, and the classes and methods that make up the MPI C++
library. The programs that users write in Fortran, C or C++ are compiled with ordinary compilers and linked
with the MPI library.

MPI programs should be able to run
on all possible machines and run all MPI implementetations without change.

An MPI computation is a collection of processes communicating with messages.
% --- end paragraph admon ---



\subsection*{Going Parallel with MPI}

% --- begin paragraph admon ---
\paragraph{}
\textbf{Task parallelism}: the work of a global problem can be divided
into a number of independent tasks, which rarely need to synchronize. 
Monte Carlo simulations or numerical integration are examples of this.

MPI is a message-passing library where all the routines
have corresponding C/C++-binding


\begin{minted}[fontsize=\fontsize{9pt}{9pt},linenos=false,mathescape,baselinestretch=1.0,fontfamily=tt,xleftmargin=7mm]{c++}
   MPI_Command_name

\end{minted}

and Fortran-binding (routine names are in uppercase, but can also be in lower case)


\begin{Verbatim}[numbers=none,fontsize=\fontsize{9pt}{9pt},baselinestretch=0.95]
   MPI_COMMAND_NAME

\end{Verbatim}
% --- end paragraph admon ---



\subsection*{MPI is a library}

% --- begin paragraph admon ---
\paragraph{}
MPI is a library specification for the message passing interface,
proposed as a standard.

\begin{itemize}
\item independent of hardware;

\item not a language or compiler specification;

\item not a specific implementation or product.
\end{itemize}

\noindent
A message passing standard for portability and ease-of-use. 
Designed for high performance.

Insert communication and synchronization functions where necessary.
% --- end paragraph admon ---



\subsection*{Bindings to MPI routines}

% --- begin paragraph admon ---
\paragraph{}

MPI is a message-passing library where all the routines
have corresponding C/C++-binding


\begin{minted}[fontsize=\fontsize{9pt}{9pt},linenos=false,mathescape,baselinestretch=1.0,fontfamily=tt,xleftmargin=7mm]{c++}
   MPI_Command_name

\end{minted}

and Fortran-binding (routine names are in uppercase, but can also be in lower case)


\begin{Verbatim}[numbers=none,fontsize=\fontsize{9pt}{9pt},baselinestretch=0.95]
   MPI_COMMAND_NAME

\end{Verbatim}

The discussion in these slides focuses on the C++ binding.
% --- end paragraph admon ---



\subsection*{Communicator}

% --- begin paragraph admon ---
\paragraph{}
\begin{itemize}
\item A group of MPI processes with a name (context).

\item Any process is identified by its rank. The rank is only meaningful within a particular communicator.

\item By default the communicator contains all the MPI processes.
\end{itemize}

\noindent


\begin{minted}[fontsize=\fontsize{9pt}{9pt},linenos=false,mathescape,baselinestretch=1.0,fontfamily=tt,xleftmargin=7mm]{c++}
  MPI_COMM_WORLD 

\end{minted}

\begin{itemize}
\item Mechanism to identify subset of processes.

\item Promotes modular design of parallel libraries.
\end{itemize}

\noindent
% --- end paragraph admon ---



\subsection*{Some of the most  important MPI functions}

% --- begin paragraph admon ---
\paragraph{}

\begin{itemize}
\item $MPI\_Init$ - initiate an MPI computation

\item $MPI\_Finalize$ - terminate the MPI computation and clean up

\item $MPI\_Comm\_size$ - how many processes participate in a given MPI communicator?

\item $MPI\_Comm\_rank$ - which one am I? (A number between 0 and size-1.)

\item $MPI\_Send$ - send a message to a particular process within an MPI communicator

\item $MPI\_Recv$ - receive a message from a particular process within an MPI communicator

\item $MPI\_reduce$  or $MPI\_Allreduce$, send and receive messages
\end{itemize}

\noindent
% --- end paragraph admon ---



\subsection*{\href{{https://github.com/CompPhysics/ComputationalPhysics2/blob/gh-pages/doc/Programs/LecturePrograms/programs/MPI/chapter07/program2.cpp}}{The first MPI C/C++ program}}

% --- begin paragraph admon ---
\paragraph{}

Let every process write "Hello world" (oh not this program again!!) on the standard output. 















\begin{minted}[fontsize=\fontsize{9pt}{9pt},linenos=false,mathescape,baselinestretch=1.0,fontfamily=tt,xleftmargin=7mm]{c++}
using namespace std;
#include <mpi.h>
#include <iostream>
int main (int nargs, char* args[])
{
int numprocs, my_rank;
//   MPI initializations
MPI_Init (&nargs, &args);
MPI_Comm_size (MPI_COMM_WORLD, &numprocs);
MPI_Comm_rank (MPI_COMM_WORLD, &my_rank);
cout << "Hello world, I have  rank " << my_rank << " out of " 
     << numprocs << endl;
//  End MPI
MPI_Finalize ();

\end{minted}
% --- end paragraph admon ---



\subsection*{The Fortran program}

% --- begin paragraph admon ---
\paragraph{}












\begin{Verbatim}[numbers=none,fontsize=\fontsize{9pt}{9pt},baselinestretch=0.95]
PROGRAM hello
INCLUDE "mpif.h"
INTEGER:: size, my_rank, ierr

CALL  MPI_INIT(ierr)
CALL MPI_COMM_SIZE(MPI_COMM_WORLD, size, ierr)
CALL MPI_COMM_RANK(MPI_COMM_WORLD, my_rank, ierr)
WRITE(*,*)"Hello world, I've rank ",my_rank," out of ",size
CALL MPI_FINALIZE(ierr)

END PROGRAM hello

\end{Verbatim}
% --- end paragraph admon ---



\subsection*{Note 1}

% --- begin paragraph admon ---
\paragraph{}

\begin{itemize}
\item The output to screen is not ordered since all processes are trying to write  to screen simultaneously.

\item It is the operating system which opts for an ordering.  

\item If we wish to have an organized output, starting from the first process, we may rewrite our program as in the next example.
\end{itemize}

\noindent
% --- end paragraph admon ---



\subsection*{\href{{https://github.com/CompPhysics/ComputationalPhysics2/blob/gh-pages/doc/Programs/LecturePrograms/programs/MPI/chapter07/program3.cpp}}{Ordered output with MPIBarrier}}

% --- begin paragraph admon ---
\paragraph{}














\begin{minted}[fontsize=\fontsize{9pt}{9pt},linenos=false,mathescape,baselinestretch=1.0,fontfamily=tt,xleftmargin=7mm]{c++}
int main (int nargs, char* args[])
{
 int numprocs, my_rank, i;
 MPI_Init (&nargs, &args);
 MPI_Comm_size (MPI_COMM_WORLD, &numprocs);
 MPI_Comm_rank (MPI_COMM_WORLD, &my_rank);
 for (i = 0; i < numprocs; i++) {}
 MPI_Barrier (MPI_COMM_WORLD);
 if (i == my_rank) {
 cout << "Hello world, I have  rank " << my_rank << 
        " out of " << numprocs << endl;}
      MPI_Finalize ();

\end{minted}
% --- end paragraph admon ---



\subsection*{Note 2}

% --- begin paragraph admon ---
\paragraph{}
\begin{itemize}
\item Here we have used the $MPI\_Barrier$ function to ensure that that every process has completed  its set of instructions in  a particular order.

\item A barrier is a special collective operation that does not allow the processes to continue until all processes in the communicator (here $MPI\_COMM\_WORLD$) have called $MPI\_Barrier$. 

\item The barriers make sure that all processes have reached the same point in the code. Many of the collective operations like $MPI\_ALLREDUCE$ to be discussed later, have the same property; that is, no process can exit the operation until all processes have started. 
\end{itemize}

\noindent
However, this is slightly more time-consuming since the processes synchronize between themselves as many times as there
are processes.  In the next Hello world example we use the send and receive functions in order to a have a synchronized
action.
% --- end paragraph admon ---



\subsection*{\href{{https://github.com/CompPhysics/ComputationalPhysics2/blob/gh-pages/doc/Programs/LecturePrograms/programs/MPI/chapter07/program4.cpp}}{Ordered output}}

% --- begin paragraph admon ---
\paragraph{}

















\begin{Verbatim}[numbers=none,fontsize=\fontsize{9pt}{9pt},baselinestretch=0.95]
.....
int numprocs, my_rank, flag;
MPI_Status status;
MPI_Init (&nargs, &args);
MPI_Comm_size (MPI_COMM_WORLD, &numprocs);
MPI_Comm_rank (MPI_COMM_WORLD, &my_rank);
if (my_rank > 0)
MPI_Recv (&flag, 1, MPI_INT, my_rank-1, 100, 
           MPI_COMM_WORLD, &status);
cout << "Hello world, I have  rank " << my_rank << " out of " 
<< numprocs << endl;
if (my_rank < numprocs-1)
MPI_Send (&my_rank, 1, MPI_INT, my_rank+1, 
          100, MPI_COMM_WORLD);
MPI_Finalize ();

\end{Verbatim}
% --- end paragraph admon ---



\subsection*{Note 3}

% --- begin paragraph admon ---
\paragraph{}

The basic sending of messages is given by the function $MPI\_SEND$, which in C/C++
is defined as 




\begin{minted}[fontsize=\fontsize{9pt}{9pt},linenos=false,mathescape,baselinestretch=1.0,fontfamily=tt,xleftmargin=7mm]{c++}
int MPI_Send(void *buf, int count, 
             MPI_Datatype datatype, 
             int dest, int tag, MPI_Comm comm)}

\end{minted}

This single command allows the passing of any kind of variable, even a large array, to any group of tasks. 
The variable \textbf{buf} is the variable we wish to send while \textbf{count}
is the  number of variables we are passing. If we are passing only a single value, this should be 1. 

If we transfer an array, it is  the overall size of the array. 
For example, if we want to send a 10 by 10 array, count would be $10\times 10=100$ 
since we are  actually passing 100 values.
% --- end paragraph admon ---



\subsection*{Note 4}

% --- begin paragraph admon ---
\paragraph{}

Once you have  sent a message, you must receive it on another task. The function $MPI\_RECV$
is similar to the send call.




\begin{minted}[fontsize=\fontsize{9pt}{9pt},linenos=false,mathescape,baselinestretch=1.0,fontfamily=tt,xleftmargin=7mm]{c++}
int MPI_Recv( void *buf, int count, MPI_Datatype datatype, 
            int source, 
            int tag, MPI_Comm comm, MPI_Status *status )

\end{minted}


The arguments that are different from those in MPI\_SEND are
\textbf{buf} which  is the name of the variable where you will  be storing the received data, 
\textbf{source} which  replaces the destination in the send command. This is the return ID of the sender.

Finally,  we have used  $MPI\_Status\_status$,  
where one can check if the receive was completed.

The output of this code is the same as the previous example, but now
process 0 sends a message to process 1, which forwards it further
to process 2, and so forth.
% --- end paragraph admon ---



\subsection*{\href{{https://github.com/CompPhysics/ComputationalPhysics2/blob/gh-pages/doc/Programs/LecturePrograms/programs/MPI/chapter07/program6.cpp}}{Numerical integration in parallel}}

% --- begin paragraph admon ---
\paragraph{Integrating $\pi$.}

\begin{itemize}
\item The code example computes $\pi$ using the trapezoidal rules.

\item The trapezoidal rule
\end{itemize}

\noindent
\[
   I=\int_a^bf(x) dx\approx h\left(f(a)/2 + f(a+h) +f(a+2h)+\dots +f(b-h)+ f(b)/2\right).
\]
Click \href{{https://github.com/CompPhysics/ComputationalPhysics2/blob/gh-pages/doc/Programs/LecturePrograms/programs/MPI/chapter07/program6.cpp}}{on this link} for the full program.
% --- end paragraph admon ---



\subsection*{Dissection of trapezoidal rule with $MPI\_reduce$}

% --- begin paragraph admon ---
\paragraph{}


















\begin{minted}[fontsize=\fontsize{9pt}{9pt},linenos=false,mathescape,baselinestretch=1.0,fontfamily=tt,xleftmargin=7mm]{c++}
//    Trapezoidal rule and numerical integration usign MPI
using namespace std;
#include <mpi.h>
#include <iostream>

//     Here we define various functions called by the main program

double int_function(double );
double trapezoidal_rule(double , double , int , double (*)(double));

//   Main function begins here
int main (int nargs, char* args[])
{
  int n, local_n, numprocs, my_rank; 
  double a, b, h, local_a, local_b, total_sum, local_sum;   
  double  time_start, time_end, total_time;

\end{minted}
% --- end paragraph admon ---



\subsection*{Dissection of trapezoidal rule}

% --- begin paragraph admon ---
\paragraph{}
















\begin{minted}[fontsize=\fontsize{9pt}{9pt},linenos=false,mathescape,baselinestretch=1.0,fontfamily=tt,xleftmargin=7mm]{c++}
  //  MPI initializations
  MPI_Init (&nargs, &args);
  MPI_Comm_size (MPI_COMM_WORLD, &numprocs);
  MPI_Comm_rank (MPI_COMM_WORLD, &my_rank);
  time_start = MPI_Wtime();
  //  Fixed values for a, b and n 
  a = 0.0 ; b = 1.0;  n = 1000;
  h = (b-a)/n;    // h is the same for all processes 
  local_n = n/numprocs;  
  // make sure n > numprocs, else integer division gives zero
  // Length of each process' interval of
  // integration = local_n*h.  
  local_a = a + my_rank*local_n*h;
  local_b = local_a + local_n*h;

\end{minted}
% --- end paragraph admon ---



\subsection*{Integrating with \textbf{MPI}}

% --- begin paragraph admon ---
\paragraph{}


















\begin{minted}[fontsize=\fontsize{9pt}{9pt},linenos=false,mathescape,baselinestretch=1.0,fontfamily=tt,xleftmargin=7mm]{c++}
  total_sum = 0.0;
  local_sum = trapezoidal_rule(local_a, local_b, local_n, 
                               &int_function); 
  MPI_Reduce(&local_sum, &total_sum, 1, MPI_DOUBLE, 
              MPI_SUM, 0, MPI_COMM_WORLD);
  time_end = MPI_Wtime();
  total_time = time_end-time_start;
  if ( my_rank == 0) {
    cout << "Trapezoidal rule = " <<  total_sum << endl;
    cout << "Time = " <<  total_time  
         << " on number of processors: "  << numprocs  << endl;
  }
  // End MPI
  MPI_Finalize ();  
  return 0;
}  // end of main program

\end{minted}
% --- end paragraph admon ---



\subsection*{How do I use $MPI\_reduce$?}

% --- begin paragraph admon ---
\paragraph{}

Here we have used



\begin{minted}[fontsize=\fontsize{9pt}{9pt},linenos=false,mathescape,baselinestretch=1.0,fontfamily=tt,xleftmargin=7mm]{c++}
MPI_reduce( void *senddata, void* resultdata, int count, 
     MPI_Datatype datatype, MPI_Op, int root, MPI_Comm comm)

\end{minted}


The two variables $senddata$ and $resultdata$ are obvious, besides the fact that one sends the address
of the variable or the first element of an array.  If they are arrays they need to have the same size. 
The variable $count$ represents the total dimensionality, 1 in case of just one variable, 
while $MPI\_Datatype$ 
defines the type of variable which is sent and received.  

The new feature is $MPI\_Op$. It defines the type
of operation we want to do.
% --- end paragraph admon ---



\subsection*{More on $MPI\_Reduce$}

% --- begin paragraph admon ---
\paragraph{}
In our case, since we are summing
the rectangle  contributions from every process we define  $MPI\_Op = MPI\_SUM$.
If we have an array or matrix we can search for the largest og smallest element by sending either $MPI\_MAX$ or 
$MPI\_MIN$.  If we want the location as well (which array element) we simply transfer 
$MPI\_MAXLOC$ or $MPI\_MINOC$. If we want the product we write $MPI\_PROD$. 

$MPI\_Allreduce$ is defined as



\begin{minted}[fontsize=\fontsize{9pt}{9pt},linenos=false,mathescape,baselinestretch=1.0,fontfamily=tt,xleftmargin=7mm]{c++}
MPI_Allreduce( void *senddata, void* resultdata, int count, 
          MPI_Datatype datatype, MPI_Op, MPI_Comm comm)        

\end{minted}
% --- end paragraph admon ---



\subsection*{Dissection of trapezoidal rule}

% --- begin paragraph admon ---
\paragraph{}

We use $MPI\_reduce$ to collect data from each process. Note also the use of the function 
$MPI\_Wtime$. 








\begin{minted}[fontsize=\fontsize{9pt}{9pt},linenos=false,mathescape,baselinestretch=1.0,fontfamily=tt,xleftmargin=7mm]{c++}
//  this function defines the function to integrate
double int_function(double x)
{
  double value = 4./(1.+x*x);
  return value;
} // end of function to evaluate


\end{minted}
% --- end paragraph admon ---



\subsection*{Dissection of trapezoidal rule}

% --- begin paragraph admon ---
\paragraph{}



















\begin{minted}[fontsize=\fontsize{9pt}{9pt},linenos=false,mathescape,baselinestretch=1.0,fontfamily=tt,xleftmargin=7mm]{c++}
//  this function defines the trapezoidal rule
double trapezoidal_rule(double a, double b, int n, 
                         double (*func)(double))
{
  double trapez_sum;
  double fa, fb, x, step;
  int    j;
  step=(b-a)/((double) n);
  fa=(*func)(a)/2. ;
  fb=(*func)(b)/2. ;
  trapez_sum=0.;
  for (j=1; j <= n-1; j++){
    x=j*step+a;
    trapez_sum+=(*func)(x);
  }
  trapez_sum=(trapez_sum+fb+fa)*step;
  return trapez_sum;
}  // end trapezoidal_rule 

\end{minted}
% --- end paragraph admon ---



\subsection*{\href{{https://github.com/CompPhysics/ComputationalPhysics2/blob/master/doc/Programs/ParallelizationMPI/MPIvmcqdot.cpp}}{The quantum dot program for two electrons}}

% --- begin paragraph admon ---
\paragraph{}










































































































































































































































































































































































































































































\begin{minted}[fontsize=\fontsize{9pt}{9pt},linenos=false,mathescape,baselinestretch=1.0,fontfamily=tt,xleftmargin=7mm]{c++}
// Variational Monte Carlo for atoms with importance sampling, slater det
// Test case for 2-electron quantum dot, no classes using Mersenne-Twister RNG
#include "mpi.h"
#include <cmath>
#include <random>
#include <string>
#include <iostream>
#include <fstream>
#include <iomanip>
#include "vectormatrixclass.h"

using namespace  std;
// output file as global variable
ofstream ofile;  
// the step length and its squared inverse for the second derivative 
//  Here we define global variables  used in various functions
//  These can be changed by using classes
int Dimension = 2; 
int NumberParticles  = 2;  //  we fix also the number of electrons to be 2

// declaration of functions 

// The Mc sampling for the variational Monte Carlo 
void  MonteCarloSampling(int, double &, double &, Vector &);

// The variational wave function
double  WaveFunction(Matrix &, Vector &);

// The local energy 
double  LocalEnergy(Matrix &, Vector &);

// The quantum force
void  QuantumForce(Matrix &, Matrix &, Vector &);


// inline function for single-particle wave function
inline double SPwavefunction(double r, double alpha) { 
   return exp(-alpha*r*0.5);
}

// inline function for derivative of single-particle wave function
inline double DerivativeSPwavefunction(double r, double alpha) { 
  return -r*alpha;
}

// function for absolute value of relative distance
double RelativeDistance(Matrix &r, int i, int j) { 
      double r_ij = 0;  
      for (int k = 0; k < Dimension; k++) { 
	r_ij += (r(i,k)-r(j,k))*(r(i,k)-r(j,k));
      }
      return sqrt(r_ij); 
}

// inline function for derivative of Jastrow factor
inline double JastrowDerivative(Matrix &r, double beta, int i, int j, int k){
  return (r(i,k)-r(j,k))/(RelativeDistance(r, i, j)*pow(1.0+beta*RelativeDistance(r, i, j),2));
}

// function for square of position of single particle
double singleparticle_pos2(Matrix &r, int i) { 
    double r_single_particle = 0;
    for (int j = 0; j < Dimension; j++) { 
      r_single_particle  += r(i,j)*r(i,j);
    }
    return r_single_particle;
}

void lnsrch(int n, Vector &xold, double fold, Vector &g, Vector &p, Vector &x,
		 double *f, double stpmax, int *check, double (*func)(Vector &p));

void dfpmin(Vector &p, int n, double gtol, int *iter, double *fret,
	    double(*func)(Vector &p), void (*dfunc)(Vector &p, Vector &g));

static double sqrarg;
#define SQR(a) ((sqrarg=(a)) == 0.0 ? 0.0 : sqrarg*sqrarg)


static double maxarg1,maxarg2;
#define FMAX(a,b) (maxarg1=(a),maxarg2=(b),(maxarg1) > (maxarg2) ?\
        (maxarg1) : (maxarg2))


// Begin of main program   

int main(int argc, char* argv[])
{

  //  MPI initializations
  int NumberProcesses, MyRank, NumberMCsamples;
  MPI_Init (&argc, &argv);
  MPI_Comm_size (MPI_COMM_WORLD, &NumberProcesses);
  MPI_Comm_rank (MPI_COMM_WORLD, &MyRank);
  double StartTime = MPI_Wtime();
  if (MyRank == 0 && argc <= 1) {
    cout << "Bad Usage: " << argv[0] << 
      " Read also output file on same line and number of Monte Carlo cycles" << endl;
  }
  // Read filename and number of Monte Carlo cycles from the command line
  if (MyRank == 0 && argc > 2) {
    string filename = argv[1]; // first command line argument after name of program
    NumberMCsamples  = atoi(argv[2]);
    string fileout = filename;
    string argument = to_string(NumberMCsamples);
    // Final filename as filename+NumberMCsamples
    fileout.append(argument);
    ofile.open(fileout);
  }
  // broadcast the number of  Monte Carlo samples
  MPI_Bcast (&NumberMCsamples, 1, MPI_INT, 0, MPI_COMM_WORLD);
  // Two variational parameters only
  Vector VariationalParameters(2);
  int TotalNumberMCsamples = NumberMCsamples*NumberProcesses; 
  // Loop over variational parameters
  for (double alpha = 0.5; alpha <= 1.5; alpha +=0.1){
    for (double beta = 0.1; beta <= 0.5; beta +=0.05){
      VariationalParameters(0) = alpha;  // value of alpha
      VariationalParameters(1) = beta;  // value of beta
      //  Do the mc sampling  and accumulate data with MPI_Reduce
      double TotalEnergy, TotalEnergySquared, LocalProcessEnergy, LocalProcessEnergy2;
      LocalProcessEnergy = LocalProcessEnergy2 = 0.0;
      MonteCarloSampling(NumberMCsamples, LocalProcessEnergy, LocalProcessEnergy2, VariationalParameters);
      //  Collect data in total averages
      MPI_Reduce(&LocalProcessEnergy, &TotalEnergy, 1, MPI_DOUBLE, MPI_SUM, 0, MPI_COMM_WORLD);
      MPI_Reduce(&LocalProcessEnergy2, &TotalEnergySquared, 1, MPI_DOUBLE, MPI_SUM, 0, MPI_COMM_WORLD);
      // Print out results  in case of Master node, set to MyRank = 0
      if ( MyRank == 0) {
	double Energy = TotalEnergy/( (double)NumberProcesses);
	double Variance = TotalEnergySquared/( (double)NumberProcesses)-Energy*Energy;
	double StandardDeviation = sqrt(Variance/((double)TotalNumberMCsamples)); // over optimistic error
	ofile << setiosflags(ios::showpoint | ios::uppercase);
	ofile << setw(15) << setprecision(8) << VariationalParameters(0);
	ofile << setw(15) << setprecision(8) << VariationalParameters(1);
	ofile << setw(15) << setprecision(8) << Energy;
	ofile << setw(15) << setprecision(8) << Variance;
	ofile << setw(15) << setprecision(8) << StandardDeviation << endl;
      }
    }
  }
  double EndTime = MPI_Wtime();
  double TotalTime = EndTime-StartTime;
  if ( MyRank == 0 )  cout << "Time = " <<  TotalTime  << " on number of processors: "  << NumberProcesses  << endl;
  if (MyRank == 0)  ofile.close();  // close output file
  // End MPI
  MPI_Finalize ();  
  return 0;
}  //  end of main function


// Monte Carlo sampling with the Metropolis algorithm  

void MonteCarloSampling(int NumberMCsamples, double &cumulative_e, double &cumulative_e2, Vector &VariationalParameters)
{

 // Initialize the seed and call the Mersienne algo
  std::random_device rd;
  std::mt19937_64 gen(rd());
  // Set up the uniform distribution for x \in [[0, 1]
  std::uniform_real_distribution<double> UniformNumberGenerator(0.0,1.0);
  std::normal_distribution<double> Normaldistribution(0.0,1.0);
  // diffusion constant from Schroedinger equation
  double D = 0.5; 
  double timestep = 0.05;  //  we fix the time step  for the gaussian deviate
  // allocate matrices which contain the position of the particles  
  Matrix OldPosition( NumberParticles, Dimension), NewPosition( NumberParticles, Dimension);
  Matrix OldQuantumForce(NumberParticles, Dimension), NewQuantumForce(NumberParticles, Dimension);
  double Energy = 0.0; double EnergySquared = 0.0; double DeltaE = 0.0;
  //  initial trial positions
  for (int i = 0; i < NumberParticles; i++) { 
    for (int j = 0; j < Dimension; j++) {
      OldPosition(i,j) = Normaldistribution(gen)*sqrt(timestep);
    }
  }
  double OldWaveFunction = WaveFunction(OldPosition, VariationalParameters);
  QuantumForce(OldPosition, OldQuantumForce, VariationalParameters);
  // loop over monte carlo cycles 
  for (int cycles = 1; cycles <= NumberMCsamples; cycles++){ 
    // new position 
    for (int i = 0; i < NumberParticles; i++) { 
      for (int j = 0; j < Dimension; j++) {
	// gaussian deviate to compute new positions using a given timestep
	NewPosition(i,j) = OldPosition(i,j) + Normaldistribution(gen)*sqrt(timestep)+OldQuantumForce(i,j)*timestep*D;
	//	NewPosition(i,j) = OldPosition(i,j) + gaussian_deviate(&idum)*sqrt(timestep)+OldQuantumForce(i,j)*timestep*D;
      }  
      //  for the other particles we need to set the position to the old position since
      //  we move only one particle at the time
      for (int k = 0; k < NumberParticles; k++) {
	if ( k != i) {
	  for (int j = 0; j < Dimension; j++) {
	    NewPosition(k,j) = OldPosition(k,j);
	  }
	} 
      }
      double NewWaveFunction = WaveFunction(NewPosition, VariationalParameters); 
      QuantumForce(NewPosition, NewQuantumForce, VariationalParameters);
      //  we compute the log of the ratio of the greens functions to be used in the 
      //  Metropolis-Hastings algorithm
      double GreensFunction = 0.0;            
      for (int j = 0; j < Dimension; j++) {
	GreensFunction += 0.5*(OldQuantumForce(i,j)+NewQuantumForce(i,j))*
	  (D*timestep*0.5*(OldQuantumForce(i,j)-NewQuantumForce(i,j))-NewPosition(i,j)+OldPosition(i,j));
      }
      GreensFunction = exp(GreensFunction);
      // The Metropolis test is performed by moving one particle at the time
      if(UniformNumberGenerator(gen) <= GreensFunction*NewWaveFunction*NewWaveFunction/OldWaveFunction/OldWaveFunction ) { 
	for (int  j = 0; j < Dimension; j++) {
	  OldPosition(i,j) = NewPosition(i,j);
	  OldQuantumForce(i,j) = NewQuantumForce(i,j);
	}
	OldWaveFunction = NewWaveFunction;
      }
    }  //  end of loop over particles
    // compute local energy  
    double DeltaE = LocalEnergy(OldPosition, VariationalParameters);
    // update energies
    Energy += DeltaE;
    EnergySquared += DeltaE*DeltaE;
  }   // end of loop over MC trials   
  // update the energy average and its squared 
  cumulative_e = Energy/NumberMCsamples;
  cumulative_e2 = EnergySquared/NumberMCsamples;
}   // end MonteCarloSampling function  


// Function to compute the squared wave function and the quantum force

double  WaveFunction(Matrix &r, Vector &VariationalParameters)
{
  double wf = 0.0;
  // full Slater determinant for two particles, replace with Slater det for more particles 
  wf  = SPwavefunction(singleparticle_pos2(r, 0), VariationalParameters(0))*SPwavefunction(singleparticle_pos2(r, 1),VariationalParameters(0));
  // contribution from Jastrow factor
  for (int i = 0; i < NumberParticles-1; i++) { 
    for (int j = i+1; j < NumberParticles; j++) {
      wf *= exp(RelativeDistance(r, i, j)/((1.0+VariationalParameters(1)*RelativeDistance(r, i, j))));
    }
  }
  return wf;
}

// Function to calculate the local energy without numerical derivation of kinetic energy

double  LocalEnergy(Matrix &r, Vector &VariationalParameters)
{

  // compute the kinetic and potential energy from the single-particle part
  // for a many-electron system this has to be replaced by a Slater determinant
  // The absolute value of the interparticle length
  Matrix length( NumberParticles, NumberParticles);
  // Set up interparticle distance
  for (int i = 0; i < NumberParticles-1; i++) { 
    for(int j = i+1; j < NumberParticles; j++){
      length(i,j) = RelativeDistance(r, i, j);
      length(j,i) =  length(i,j);
    }
  }
  double KineticEnergy = 0.0;
  // Set up kinetic energy from Slater and Jastrow terms
  for (int i = 0; i < NumberParticles; i++) { 
    for (int k = 0; k < Dimension; k++) {
      double sum1 = 0.0; 
      for(int j = 0; j < NumberParticles; j++){
	if ( j != i) {
	  sum1 += JastrowDerivative(r, VariationalParameters(1), i, j, k);
	}
      }
      KineticEnergy += (sum1+DerivativeSPwavefunction(r(i,k),VariationalParameters(0)))*(sum1+DerivativeSPwavefunction(r(i,k),VariationalParameters(0)));
    }
  }
  KineticEnergy += -2*VariationalParameters(0)*NumberParticles;
  for (int i = 0; i < NumberParticles-1; i++) {
      for (int j = i+1; j < NumberParticles; j++) {
        KineticEnergy += 2.0/(pow(1.0 + VariationalParameters(1)*length(i,j),2))*(1.0/length(i,j)-2*VariationalParameters(1)/(1+VariationalParameters(1)*length(i,j)) );
      }
  }
  KineticEnergy *= -0.5;
  // Set up potential energy, external potential + eventual electron-electron repulsion
  double PotentialEnergy = 0;
  for (int i = 0; i < NumberParticles; i++) { 
    double DistanceSquared = singleparticle_pos2(r, i);
    PotentialEnergy += 0.5*DistanceSquared;  // sp energy HO part, note it has the oscillator frequency set to 1!
  }
  // Add the electron-electron repulsion
  for (int i = 0; i < NumberParticles-1; i++) { 
    for (int j = i+1; j < NumberParticles; j++) {
      PotentialEnergy += 1.0/length(i,j);          
    }
  }
  double LocalE = KineticEnergy+PotentialEnergy;
  return LocalE;
}

// Compute the analytical expression for the quantum force
void  QuantumForce(Matrix &r, Matrix &qforce, Vector &VariationalParameters)
{
  // compute the first derivative 
  for (int i = 0; i < NumberParticles; i++) {
    for (int k = 0; k < Dimension; k++) {
      // single-particle part, replace with Slater det for larger systems
      double sppart = DerivativeSPwavefunction(r(i,k),VariationalParameters(0));
      //  Jastrow factor contribution
      double Jsum = 0.0;
      for (int j = 0; j < NumberParticles; j++) {
	if ( j != i) {
	  Jsum += JastrowDerivative(r, VariationalParameters(1), i, j, k);
	}
      }
      qforce(i,k) = 2.0*(Jsum+sppart);
    }
  }
} // end of QuantumForce function


#define ITMAX 200
#define EPS 3.0e-8
#define TOLX (4*EPS)
#define STPMX 100.0

void dfpmin(Vector &p, int n, double gtol, int *iter, double *fret,
	    double(*func)(Vector &p), void (*dfunc)(Vector &p, Vector &g))
{

  int check,i,its,j;
  double den,fac,fad,fae,fp,stpmax,sum=0.0,sumdg,sumxi,temp,test;
  Vector dg(n), g(n), hdg(n), pnew(n), xi(n);
  Matrix hessian(n,n);

  fp=(*func)(p);
  (*dfunc)(p,g);
  for (i = 0;i < n;i++) {
    for (j = 0; j< n;j++) hessian(i,j)=0.0;
    hessian(i,i)=1.0;
    xi(i) = -g(i);
    sum += p(i)*p(i);
  }
  stpmax=STPMX*FMAX(sqrt(sum),(double)n);
  for (its=1;its<=ITMAX;its++) {
    *iter=its;
    lnsrch(n,p,fp,g,xi,pnew,fret,stpmax,&check,func);
    fp = *fret;
    for (i = 0; i< n;i++) {
      xi(i)=pnew(i)-p(i);
      p(i)=pnew(i);
    }
    test=0.0;
    for (i = 0;i< n;i++) {
      temp=fabs(xi(i))/FMAX(fabs(p(i)),1.0);
      if (temp > test) test=temp;
    }
    if (test < TOLX) {
      return;
    }
    for (i=0;i<n;i++) dg(i)=g(i);
    (*dfunc)(p,g);
    test=0.0;
    den=FMAX(*fret,1.0);
    for (i=0;i<n;i++) {
      temp=fabs(g(i))*FMAX(fabs(p(i)),1.0)/den;
      if (temp > test) test=temp;
    }
    if (test < gtol) {
      return;
    }
    for (i=0;i<n;i++) dg(i)=g(i)-dg(i);
    for (i=0;i<n;i++) {
      hdg(i)=0.0;
      for (j=0;j<n;j++) hdg(i) += hessian(i,j)*dg(j);
    }
    fac=fae=sumdg=sumxi=0.0;
    for (i=0;i<n;i++) {
      fac += dg(i)*xi(i);
      fae += dg(i)*hdg(i);
      sumdg += SQR(dg(i));
      sumxi += SQR(xi(i));
    }
    if (fac*fac > EPS*sumdg*sumxi) {
      fac=1.0/fac;
      fad=1.0/fae;
      for (i=0;i<n;i++) dg(i)=fac*xi(i)-fad*hdg(i);
      for (i=0;i<n;i++) {
	for (j=0;j<n;j++) {
	  hessian(i,j) += fac*xi(i)*xi(j)
	    -fad*hdg(i)*hdg(j)+fae*dg(i)*dg(j);
	}
      }
    }
    for (i=0;i<n;i++) {
      xi(i)=0.0;
      for (j=0;j<n;j++) xi(i) -= hessian(i,j)*g(j);
    }
  }
  cout << "too many iterations in dfpmin" << endl;
}
#undef ITMAX
#undef EPS
#undef TOLX
#undef STPMX

#define ALF 1.0e-4
#define TOLX 1.0e-7

void lnsrch(int n, Vector &xold, double fold, Vector &g, Vector &p, Vector &x,
	    double *f, double stpmax, int *check, double (*func)(Vector &p))
{
  int i;
  double a,alam,alam2,alamin,b,disc,f2,fold2,rhs1,rhs2,slope,sum,temp,
    test,tmplam;

  *check=0;
  for (sum=0.0,i=0;i<n;i++) sum += p(i)*p(i);
  sum=sqrt(sum);
  if (sum > stpmax)
    for (i=0;i<n;i++) p(i) *= stpmax/sum;
  for (slope=0.0,i=0;i<n;i++)
    slope += g(i)*p(i);
  test=0.0;
  for (i=0;i<n;i++) {
    temp=fabs(p(i))/FMAX(fabs(xold(i)),1.0);
    if (temp > test) test=temp;
  }
  alamin=TOLX/test;
  alam=1.0;
  for (;;) {
    for (i=0;i<n;i++) x(i)=xold(i)+alam*p(i);
    *f=(*func)(x);
    if (alam < alamin) {
      for (i=0;i<n;i++) x(i)=xold(i);
      *check=1;
      return;
    } else if (*f <= fold+ALF*alam*slope) return;
    else {
      if (alam == 1.0)
	tmplam = -slope/(2.0*(*f-fold-slope));
      else {
	rhs1 = *f-fold-alam*slope;
	rhs2=f2-fold2-alam2*slope;
	a=(rhs1/(alam*alam)-rhs2/(alam2*alam2))/(alam-alam2);
	b=(-alam2*rhs1/(alam*alam)+alam*rhs2/(alam2*alam2))/(alam-alam2);
	if (a == 0.0) tmplam = -slope/(2.0*b);
	else {
	  disc=b*b-3.0*a*slope;
	  if (disc<0.0) cout << "Roundoff problem in lnsrch." << endl;
	  else tmplam=(-b+sqrt(disc))/(3.0*a);
	}
	if (tmplam>0.5*alam)
	  tmplam=0.5*alam;
      }
    }
    alam2=alam;
    f2 = *f;
    fold2=fold;
    alam=FMAX(tmplam,0.1*alam);
  }
}
#undef ALF
#undef TOLX


\end{minted}
% --- end paragraph admon ---



\subsection*{What is OpenMP}

% --- begin paragraph admon ---
\paragraph{}
\begin{itemize}
\item OpenMP provides high-level thread programming

\item Multiple cooperating threads are allowed to run simultaneously

\item Threads are created and destroyed dynamically in a fork-join pattern
\begin{itemize}

   \item An OpenMP program consists of a number of parallel regions

   \item Between two parallel regions there is only one master thread

   \item In the beginning of a parallel region, a team of new threads is spawned

\end{itemize}

\noindent
  \item The newly spawned threads work simultaneously with the master thread

  \item At the end of a parallel region, the new threads are destroyed
\end{itemize}

\noindent
Many good tutorials online and excellent textbook
\begin{enumerate}
\item \href{{http://mitpress.mit.edu/books/using-openmp}}{Using OpenMP, by B. Chapman, G. Jost, and A. van der Pas}

\item Many tutorials online like \href{{http://www.openmp.org}}{OpenMP official site}
\end{enumerate}

\noindent
% --- end paragraph admon ---



\subsection*{Getting started, things to remember}

% --- begin paragraph admon ---
\paragraph{}
\begin{itemize}
 \item Remember the header file 
\end{itemize}

\noindent


\begin{minted}[fontsize=\fontsize{9pt}{9pt},linenos=false,mathescape,baselinestretch=1.0,fontfamily=tt,xleftmargin=7mm]{c++}
#include <omp.h>

\end{minted}

\begin{itemize}
 \item Insert compiler directives in C++ syntax as 
\end{itemize}

\noindent


\begin{minted}[fontsize=\fontsize{9pt}{9pt},linenos=false,mathescape,baselinestretch=1.0,fontfamily=tt,xleftmargin=7mm]{c++}
#pragma omp...

\end{minted}

\begin{itemize}
\item Compile with for example \emph{c++ -fopenmp code.cpp}

\item Execute
\begin{itemize}

  \item Remember to assign the environment variable \textbf{OMP NUM THREADS}

  \item It specifies the total number of threads inside a parallel region, if not otherwise overwritten
\end{itemize}

\noindent
\end{itemize}

\noindent
% --- end paragraph admon ---



\subsection*{OpenMP syntax}
\begin{itemize}
\item Mostly directives
\end{itemize}

\noindent


\begin{minted}[fontsize=\fontsize{9pt}{9pt},linenos=false,mathescape,baselinestretch=1.0,fontfamily=tt,xleftmargin=7mm]{c++}
#pragma omp construct [ clause ...]

\end{minted}

\begin{itemize}
 \item Some functions and types 
\end{itemize}

\noindent


\begin{minted}[fontsize=\fontsize{9pt}{9pt},linenos=false,mathescape,baselinestretch=1.0,fontfamily=tt,xleftmargin=7mm]{c++}
#include <omp.h>

\end{minted}

\begin{itemize}
 \item Most apply to a block of code

 \item Specifically, a \textbf{structured block}

 \item Enter at top, exit at bottom only, exit(), abort() permitted
\end{itemize}

\noindent
\subsection*{Different OpenMP styles of parallelism}
OpenMP supports several different ways to specify thread parallelism

\begin{itemize}
\item General parallel regions: All threads execute the code, roughly as if you made a routine of that region and created a thread to run that code

\item Parallel loops: Special case for loops, simplifies data parallel code

\item Task parallelism, new in OpenMP 3

\item Several ways to manage thread coordination, including Master regions and Locks

\item Memory model for shared data
\end{itemize}

\noindent
\subsection*{General code structure}

% --- begin paragraph admon ---
\paragraph{}




















\begin{minted}[fontsize=\fontsize{9pt}{9pt},linenos=false,mathescape,baselinestretch=1.0,fontfamily=tt,xleftmargin=7mm]{c++}
#include <omp.h>
main ()
{
int var1, var2, var3;
/* serial code */
/* ... */
/* start of a parallel region */
#pragma omp parallel private(var1, var2) shared(var3)
{
/* ... */
}
/* more serial code */
/* ... */
/* another parallel region */
#pragma omp parallel
{
/* ... */
}
}

\end{minted}
% --- end paragraph admon ---



\subsection*{Parallel region}

% --- begin paragraph admon ---
\paragraph{}
\begin{itemize}
\item A parallel region is a block of code that is executed by a team of threads

\item The following compiler directive creates a parallel region
\end{itemize}

\noindent


\begin{minted}[fontsize=\fontsize{9pt}{9pt},linenos=false,mathescape,baselinestretch=1.0,fontfamily=tt,xleftmargin=7mm]{c++}
#pragma omp parallel { ... }

\end{minted}

\begin{itemize}
\item Clauses can be added at the end of the directive

\item Most often used clauses:
\begin{itemize}

 \item \textbf{default(shared)} or \textbf{default(none)}

 \item \textbf{public(list of variables)}

 \item \textbf{private(list of variables)}
\end{itemize}

\noindent
\end{itemize}

\noindent
% --- end paragraph admon ---



\subsection*{Hello world, not again, please!}

% --- begin paragraph admon ---
\paragraph{}


















\begin{minted}[fontsize=\fontsize{9pt}{9pt},linenos=false,mathescape,baselinestretch=1.0,fontfamily=tt,xleftmargin=7mm]{c++}
#include <omp.h>
#include <cstdio>
int main (int argc, char *argv[])
{
int th_id, nthreads;
#pragma omp parallel private(th_id) shared(nthreads)
{
th_id = omp_get_thread_num();
printf("Hello World from thread %d\n", th_id);
#pragma omp barrier
if ( th_id == 0 ) {
nthreads = omp_get_num_threads();
printf("There are %d threads\n",nthreads);
}
}
return 0;
}

\end{minted}
% --- end paragraph admon ---



\subsection*{Hello world, yet another variant}

% --- begin paragraph admon ---
\paragraph{}














\begin{minted}[fontsize=\fontsize{9pt}{9pt},linenos=false,mathescape,baselinestretch=1.0,fontfamily=tt,xleftmargin=7mm]{c++}
#include <cstdio>
#include <omp.h>
int main(int argc, char *argv[]) 
{
 omp_set_num_threads(4); 
#pragma omp parallel
 {
   int id = omp_get_thread_num();
   int nproc = omp_get_num_threads(); 
   cout << "Hello world with id number and processes " <<  id <<  nproc << endl;
 } 
return 0;
}

\end{minted}

Variables declared outside of the parallel region are shared by all threads
If a variable like \textbf{id} is  declared outside of the 


\begin{minted}[fontsize=\fontsize{9pt}{9pt},linenos=false,mathescape,baselinestretch=1.0,fontfamily=tt,xleftmargin=7mm]{c++}
#pragma omp parallel, 

\end{minted}

it would have been shared by various the threads, possibly causing erroneous output
\begin{itemize}
 \item Why? What would go wrong? Why do we add  possibly?
\end{itemize}

\noindent
% --- end paragraph admon ---



\subsection*{Important OpenMP library routines}

% --- begin paragraph admon ---
\paragraph{}

\begin{itemize}
\item \textbf{int omp get num threads ()}, returns the number of threads inside a parallel region

\item \textbf{int omp get thread num ()},  returns the  a thread for each thread inside a parallel region

\item \textbf{void omp set num threads (int)}, sets the number of threads to be used

\item \textbf{void omp set nested (int)},  turns nested parallelism on/off
\end{itemize}

\noindent
% --- end paragraph admon ---



\subsection*{Private variables}

% --- begin paragraph admon ---
\paragraph{}
Private clause can be used to make thread- private versions of such variables: 






\begin{minted}[fontsize=\fontsize{9pt}{9pt},linenos=false,mathescape,baselinestretch=1.0,fontfamily=tt,xleftmargin=7mm]{c++}
#pragma omp parallel private(id)
{
 int id = omp_get_thread_num();
 cout << "My thread num" << id << endl; 
}

\end{minted}

\begin{itemize}
\item What is their value on entry? Exit?

\item OpenMP provides ways to control that

\item Can use default(none) to require the sharing of each variable to be described
\end{itemize}

\noindent
% --- end paragraph admon ---



\subsection*{Master region}

% --- begin paragraph admon ---
\paragraph{}
It is often useful to have only one thread execute some of the code in a parallel region. I/O statements are a common example









\begin{minted}[fontsize=\fontsize{9pt}{9pt},linenos=false,mathescape,baselinestretch=1.0,fontfamily=tt,xleftmargin=7mm]{c++}
#pragma omp parallel 
{
  #pragma omp master
   {
      int id = omp_get_thread_num();
      cout << "My thread num" << id << endl; 
   } 
}

\end{minted}
% --- end paragraph admon ---



\subsection*{Parallel for loop}

% --- begin paragraph admon ---
\paragraph{}
\begin{itemize}
 \item Inside a parallel region, the following compiler directive can be used to parallelize a for-loop:
\end{itemize}

\noindent


\begin{minted}[fontsize=\fontsize{9pt}{9pt},linenos=false,mathescape,baselinestretch=1.0,fontfamily=tt,xleftmargin=7mm]{c++}
#pragma omp for

\end{minted}

\begin{itemize}
\item Clauses can be added, such as
\begin{itemize}

  \item \textbf{schedule(static, chunk size)}

  \item \textbf{schedule(dynamic, chunk size)} 

  \item \textbf{schedule(guided, chunk size)} (non-deterministic allocation)

  \item \textbf{schedule(runtime)}

  \item \textbf{private(list of variables)}

  \item \textbf{reduction(operator:variable)}

  \item \textbf{nowait}
\end{itemize}

\noindent
\end{itemize}

\noindent
% --- end paragraph admon ---



\subsection*{Parallel computations and loops}


% --- begin paragraph admon ---
\paragraph{}
OpenMP provides an easy way to parallelize a loop



\begin{minted}[fontsize=\fontsize{9pt}{9pt},linenos=false,mathescape,baselinestretch=1.0,fontfamily=tt,xleftmargin=7mm]{c++}
#pragma omp parallel for
  for (i=0; i<n; i++) c[i] = a[i];

\end{minted}

OpenMP handles index variable (no need to declare in for loop or make private)

Which thread does which values?  Several options.
% --- end paragraph admon ---



\subsection*{Scheduling of  loop computations}


% --- begin paragraph admon ---
\paragraph{}
We can let  the OpenMP runtime decide. The decision is about how the loop iterates are scheduled
and  OpenMP defines three choices of loop scheduling:
\begin{enumerate}
\item Static: Predefined at compile time. Lowest overhead, predictable

\item Dynamic: Selection made at runtime 

\item Guided: Special case of dynamic; attempts to reduce overhead
\end{enumerate}

\noindent
% --- end paragraph admon ---



\subsection*{Example code for loop scheduling}

% --- begin paragraph admon ---
\paragraph{}
















\begin{minted}[fontsize=\fontsize{9pt}{9pt},linenos=false,mathescape,baselinestretch=1.0,fontfamily=tt,xleftmargin=7mm]{c++}
#include <omp.h>
#define CHUNKSIZE 100
#define N 1000
int main (int argc, char *argv[])
{
int i, chunk;
float a[N], b[N], c[N];
for (i=0; i < N; i++) a[i] = b[i] = i * 1.0;
chunk = CHUNKSIZE;
#pragma omp parallel shared(a,b,c,chunk) private(i)
{
#pragma omp for schedule(dynamic,chunk)
for (i=0; i < N; i++) c[i] = a[i] + b[i];
} /* end of parallel region */
}

\end{minted}
% --- end paragraph admon ---



\subsection*{Example code for loop scheduling, guided instead of dynamic}

% --- begin paragraph admon ---
\paragraph{}
















\begin{minted}[fontsize=\fontsize{9pt}{9pt},linenos=false,mathescape,baselinestretch=1.0,fontfamily=tt,xleftmargin=7mm]{c++}
#include <omp.h>
#define CHUNKSIZE 100
#define N 1000
int main (int argc, char *argv[])
{
int i, chunk;
float a[N], b[N], c[N];
for (i=0; i < N; i++) a[i] = b[i] = i * 1.0;
chunk = CHUNKSIZE;
#pragma omp parallel shared(a,b,c,chunk) private(i)
{
#pragma omp for schedule(guided,chunk)
for (i=0; i < N; i++) c[i] = a[i] + b[i];
} /* end of parallel region */
}

\end{minted}
% --- end paragraph admon ---



\subsection*{More on Parallel for loop}

% --- begin paragraph admon ---
\paragraph{}
\begin{itemize}
\item The number of loop iterations cannot be non-deterministic; break, return, exit, goto not allowed inside the for-loop

\item The loop index is private to each thread

\item A reduction variable is special
\begin{itemize}

  \item During the for-loop there is a local private copy in each thread

  \item At the end of the for-loop, all the local copies are combined together by the reduction operation

\end{itemize}

\noindent
\item Unless the nowait clause is used, an implicit barrier synchronization will be added at the end by the compiler
\end{itemize}

\noindent


\begin{minted}[fontsize=\fontsize{9pt}{9pt},linenos=false,mathescape,baselinestretch=1.0,fontfamily=tt,xleftmargin=7mm]{c++}
// #pragma omp parallel and #pragma omp for

\end{minted}

can be combined into


\begin{minted}[fontsize=\fontsize{9pt}{9pt},linenos=false,mathescape,baselinestretch=1.0,fontfamily=tt,xleftmargin=7mm]{c++}
#pragma omp parallel for

\end{minted}
% --- end paragraph admon ---



\subsection*{What can happen with this loop?}


% --- begin paragraph admon ---
\paragraph{}
What happens with code like this 



\begin{minted}[fontsize=\fontsize{9pt}{9pt},linenos=false,mathescape,baselinestretch=1.0,fontfamily=tt,xleftmargin=7mm]{c++}
#pragma omp parallel for
for (i=0; i<n; i++) sum += a[i]*a[i];

\end{minted}

All threads can access the \textbf{sum} variable, but the addition is not atomic! It is important to avoid race between threads. So-called reductions in OpenMP are thus important for performance and for obtaining correct results.  OpenMP lets us indicate that a variable is used for a reduction with a particular operator. The above code becomes




\begin{minted}[fontsize=\fontsize{9pt}{9pt},linenos=false,mathescape,baselinestretch=1.0,fontfamily=tt,xleftmargin=7mm]{c++}
sum = 0.0;
#pragma omp parallel for reduction(+:sum)
for (i=0; i<n; i++) sum += a[i]*a[i];

\end{minted}
% --- end paragraph admon ---



\subsection*{Inner product}

% --- begin paragraph admon ---
\paragraph{}
\[
\sum_{i=0}^{n-1} a_ib_i
\]








\begin{minted}[fontsize=\fontsize{9pt}{9pt},linenos=false,mathescape,baselinestretch=1.0,fontfamily=tt,xleftmargin=7mm]{c++}
int i;
double sum = 0.;
/* allocating and initializing arrays */
/* ... */
#pragma omp parallel for default(shared) private(i) reduction(+:sum)
 for (i=0; i<N; i++) sum += a[i]*b[i];
}

\end{minted}
% --- end paragraph admon ---



\subsection*{Different threads do different tasks}

% --- begin paragraph admon ---
\paragraph{}

Different threads do different tasks independently, each section is executed by one thread.













\begin{minted}[fontsize=\fontsize{9pt}{9pt},linenos=false,mathescape,baselinestretch=1.0,fontfamily=tt,xleftmargin=7mm]{c++}
#pragma omp parallel
{
#pragma omp sections
{
#pragma omp section
funcA ();
#pragma omp section
funcB ();
#pragma omp section
funcC ();
}
}

\end{minted}
% --- end paragraph admon ---



\subsection*{Single execution}

% --- begin paragraph admon ---
\paragraph{}


\begin{minted}[fontsize=\fontsize{9pt}{9pt},linenos=false,mathescape,baselinestretch=1.0,fontfamily=tt,xleftmargin=7mm]{c++}
#pragma omp single { ... }

\end{minted}

The code is executed by one thread only, no guarantee which thread

Can introduce an implicit barrier at the end


\begin{minted}[fontsize=\fontsize{9pt}{9pt},linenos=false,mathescape,baselinestretch=1.0,fontfamily=tt,xleftmargin=7mm]{c++}
#pragma omp master { ... }

\end{minted}

Code executed by the master thread, guaranteed and no implicit barrier at the end.
% --- end paragraph admon ---



\subsection*{Coordination and synchronization}

% --- begin paragraph admon ---
\paragraph{}


\begin{minted}[fontsize=\fontsize{9pt}{9pt},linenos=false,mathescape,baselinestretch=1.0,fontfamily=tt,xleftmargin=7mm]{c++}
#pragma omp barrier

\end{minted}

Synchronization, must be encountered by all threads in a team (or none)


\begin{minted}[fontsize=\fontsize{9pt}{9pt},linenos=false,mathescape,baselinestretch=1.0,fontfamily=tt,xleftmargin=7mm]{c++}
#pragma omp ordered { a block of codes }

\end{minted}

is another form of synchronization (in sequential order).
The form


\begin{minted}[fontsize=\fontsize{9pt}{9pt},linenos=false,mathescape,baselinestretch=1.0,fontfamily=tt,xleftmargin=7mm]{c++}
#pragma omp critical { a block of codes }

\end{minted}

and 


\begin{minted}[fontsize=\fontsize{9pt}{9pt},linenos=false,mathescape,baselinestretch=1.0,fontfamily=tt,xleftmargin=7mm]{c++}
#pragma omp atomic { single assignment statement }

\end{minted}

is  more efficient than 


\begin{minted}[fontsize=\fontsize{9pt}{9pt},linenos=false,mathescape,baselinestretch=1.0,fontfamily=tt,xleftmargin=7mm]{c++}
#pragma omp critical

\end{minted}
% --- end paragraph admon ---



\subsection*{Data scope}

% --- begin paragraph admon ---
\paragraph{}
\begin{itemize}
\item OpenMP data scope attribute clauses:
\begin{itemize}

 \item \textbf{shared}

 \item \textbf{private}

 \item \textbf{firstprivate}

 \item \textbf{lastprivate}

 \item \textbf{reduction}
\end{itemize}

\noindent
\end{itemize}

\noindent
What are the purposes of these attributes
\begin{itemize}
\item define how and which variables are transferred to a parallel region (and back)

\item define which variables are visible to all threads in a parallel region, and which variables are privately allocated to each thread
\end{itemize}

\noindent
% --- end paragraph admon ---



\subsection*{Some remarks}

% --- begin paragraph admon ---
\paragraph{}

\begin{itemize}
\item When entering a parallel region, the \textbf{private} clause ensures each thread having its own new variable instances. The new variables are assumed to be uninitialized.

\item A shared variable exists in only one memory location and all threads can read and write to that address. It is the programmer's responsibility to ensure that multiple threads properly access a shared variable.

\item The \textbf{firstprivate} clause combines the behavior of the private clause with automatic initialization.

\item The \textbf{lastprivate} clause combines the behavior of the private clause with a copy back (from the last loop iteration or section) to the original variable outside the parallel region.
\end{itemize}

\noindent
% --- end paragraph admon ---



\subsection*{Parallelizing nested for-loops}

% --- begin paragraph admon ---
\paragraph{}

\begin{itemize}
 \item Serial code
\end{itemize}

\noindent






\begin{minted}[fontsize=\fontsize{9pt}{9pt},linenos=false,mathescape,baselinestretch=1.0,fontfamily=tt,xleftmargin=7mm]{c++}
for (i=0; i<100; i++)
    for (j=0; j<100; j++)
        a[i][j] = b[i][j] + c[i][j];
    }
}

\end{minted}


\begin{itemize}
\item Parallelization
\end{itemize}

\noindent







\begin{minted}[fontsize=\fontsize{9pt}{9pt},linenos=false,mathescape,baselinestretch=1.0,fontfamily=tt,xleftmargin=7mm]{c++}
#pragma omp parallel for private(j)
for (i=0; i<100; i++)
    for (j=0; j<100; j++)
       a[i][j] = b[i][j] + c[i][j];
    }
}

\end{minted}


\begin{itemize}
\item Why not parallelize the inner loop? to save overhead of repeated thread forks-joins

\item Why must \textbf{j} be private? To avoid race condition among the threads
\end{itemize}

\noindent
% --- end paragraph admon ---



\subsection*{Nested parallelism}

% --- begin paragraph admon ---
\paragraph{}
When a thread in a parallel region encounters another parallel construct, it
may create a new team of threads and become the master of the new
team.









\begin{minted}[fontsize=\fontsize{9pt}{9pt},linenos=false,mathescape,baselinestretch=1.0,fontfamily=tt,xleftmargin=7mm]{c++}
#pragma omp parallel num_threads(4)
{
/* .... */
#pragma omp parallel num_threads(2)
{
//  
}
}

\end{minted}
% --- end paragraph admon ---



\subsection*{Parallel tasks}

% --- begin paragraph admon ---
\paragraph{}











\begin{minted}[fontsize=\fontsize{9pt}{9pt},linenos=false,mathescape,baselinestretch=1.0,fontfamily=tt,xleftmargin=7mm]{c++}
#pragma omp task 
#pragma omp parallel shared(p_vec) private(i)
{
#pragma omp single
{
for (i=0; i<N; i++) {
  double r = random_number();
  if (p_vec[i] > r) {
#pragma omp task
   do_work (p_vec[i]);

\end{minted}
% --- end paragraph admon ---



\subsection*{Common mistakes}

% --- begin paragraph admon ---
\paragraph{}
Race condition






\begin{minted}[fontsize=\fontsize{9pt}{9pt},linenos=false,mathescape,baselinestretch=1.0,fontfamily=tt,xleftmargin=7mm]{c++}
int nthreads;
#pragma omp parallel shared(nthreads)
{
nthreads = omp_get_num_threads();
}

\end{minted}

Deadlock










\begin{minted}[fontsize=\fontsize{9pt}{9pt},linenos=false,mathescape,baselinestretch=1.0,fontfamily=tt,xleftmargin=7mm]{c++}
#pragma omp parallel
{
...
#pragma omp critical
{
...
#pragma omp barrier
}
}

\end{minted}
% --- end paragraph admon ---



\subsection*{Not all computations are simple}

% --- begin paragraph admon ---
\paragraph{}
Not all computations are simple loops where the data can be evenly 
divided among threads without any dependencies between threads

An example is finding the location and value of the largest element in an array







\begin{minted}[fontsize=\fontsize{9pt}{9pt},linenos=false,mathescape,baselinestretch=1.0,fontfamily=tt,xleftmargin=7mm]{c++}
for (i=0; i<n; i++) { 
   if (x[i] > maxval) {
      maxval = x[i];
      maxloc = i; 
   }
}

\end{minted}
% --- end paragraph admon ---



\subsection*{Not all computations are simple, competing threads}

% --- begin paragraph admon ---
\paragraph{}
All threads are potentially accessing and changing the same values, \textbf{maxloc} and \textbf{maxval}.
\begin{enumerate}
\item OpenMP provides several ways to coordinate access to shared values
\end{enumerate}

\noindent


\begin{minted}[fontsize=\fontsize{9pt}{9pt},linenos=false,mathescape,baselinestretch=1.0,fontfamily=tt,xleftmargin=7mm]{c++}
#pragma omp atomic

\end{minted}

\begin{enumerate}
\item Only one thread at a time can execute the following statement (not block). We can use the critical option
\end{enumerate}

\noindent


\begin{minted}[fontsize=\fontsize{9pt}{9pt},linenos=false,mathescape,baselinestretch=1.0,fontfamily=tt,xleftmargin=7mm]{c++}
#pragma omp critical

\end{minted}

\begin{enumerate}
\item Only one thread at a time can execute the following block
\end{enumerate}

\noindent
Atomic may be faster than critical but depends on hardware
% --- end paragraph admon ---



\subsection*{How to find the max value using OpenMP}

% --- begin paragraph admon ---
\paragraph{}
Write down the simplest algorithm and look carefully for race conditions. How would you handle them? 
The first step would be to parallelize as 








\begin{minted}[fontsize=\fontsize{9pt}{9pt},linenos=false,mathescape,baselinestretch=1.0,fontfamily=tt,xleftmargin=7mm]{c++}
#pragma omp parallel for
 for (i=0; i<n; i++) {
    if (x[i] > maxval) {
      maxval = x[i];
      maxloc = i; 
    }
}

\end{minted}
% --- end paragraph admon ---



\subsection*{Then deal with the race conditions}

% --- begin paragraph admon ---
\paragraph{}
Write down the simplest algorithm and look carefully for race conditions. How would you handle them? 
The first step would be to parallelize as 











\begin{minted}[fontsize=\fontsize{9pt}{9pt},linenos=false,mathescape,baselinestretch=1.0,fontfamily=tt,xleftmargin=7mm]{c++}
#pragma omp parallel for
 for (i=0; i<n; i++) {
#pragma omp critical
  {
     if (x[i] > maxval) {
       maxval = x[i];
       maxloc = i; 
     }
  }
} 

\end{minted}


Exercise: write a code which implements this and give an estimate on performance. Perform several runs,
with a serial code only with and without vectorization and compare the serial code with the one that  uses OpenMP. Run on different archictectures if you can.
% --- end paragraph admon ---



\subsection*{What can slow down OpenMP performance?}
Give it a thought!

\subsection*{What can slow down OpenMP performance?}

% --- begin paragraph admon ---
\paragraph{}
Performance poor because we insisted on keeping track of the maxval and location during the execution of the loop.
\begin{itemize}
 \item We do not care about the value during the execution of the loop, just the value at the end.
\end{itemize}

\noindent
This is a common source of performance issues, namely the description of the method used to compute a value imposes additional, unnecessary requirements or properties

\textbf{Idea: Have each thread find the maxloc in its own data, then combine and use temporary arrays indexed by thread number to hold the values found by each thread}
% --- end paragraph admon ---



\subsection*{Find the max location for each thread}

% --- begin paragraph admon ---
\paragraph{}















\begin{minted}[fontsize=\fontsize{9pt}{9pt},linenos=false,mathescape,baselinestretch=1.0,fontfamily=tt,xleftmargin=7mm]{c++}
int maxloc[MAX_THREADS], mloc;
double maxval[MAX_THREADS], mval; 
#pragma omp parallel shared(maxval,maxloc)
{
  int id = omp_get_thread_num(); 
  maxval[id] = -1.0e30;
#pragma omp for
   for (int i=0; i<n; i++) {
       if (x[i] > maxval[id]) { 
           maxloc[id] = i;
           maxval[id] = x[i]; 
       }
    }
}

\end{minted}
% --- end paragraph admon ---



\subsection*{Combine the values from each thread}

% --- begin paragraph admon ---
\paragraph{}














\begin{minted}[fontsize=\fontsize{9pt}{9pt},linenos=false,mathescape,baselinestretch=1.0,fontfamily=tt,xleftmargin=7mm]{c++}
#pragma omp flush (maxloc,maxval)
#pragma omp master
  {
    int nt = omp_get_num_threads(); 
    mloc = maxloc[0]; 
    mval = maxval[0]; 
    for (int i=1; i<nt; i++) {
        if (maxval[i] > mval) { 
           mval = maxval[i]; 
           mloc = maxloc[i];
        } 
     }
   }

\end{minted}

Note that we let the master process perform the last operation.
% --- end paragraph admon ---



\subsection*{\href{{https://github.com/CompPhysics/ComputationalPhysicsMSU/blob/master/doc/Programs/ParallelizationOpenMP/OpenMPvectornorm.cpp}}{Matrix-matrix multiplication}}
This code computes the norm of a vector using OpenMp


























































\begin{minted}[fontsize=\fontsize{9pt}{9pt},linenos=false,mathescape,baselinestretch=1.0,fontfamily=tt,xleftmargin=7mm]{text}
//  OpenMP program to compute vector norm by adding two other vectors
#include <cstdlib>
#include <iostream>
#include <cmath>
#include <iomanip>
#include  <omp.h>
# include <ctime>

using namespace std; // note use of namespace
int main (int argc, char* argv[])
{
  // read in dimension of vector
  int n = atoi(argv[1]);
  double *a, *b, *c;
  int i;
  int thread_num;
  double wtime, Norm2, s, angle;
  cout << "  Perform addition of two vectors and compute the norm-2." << endl;
  omp_set_num_threads(4);
  thread_num = omp_get_max_threads ();
  cout << "  The number of processors available = " << omp_get_num_procs () << endl ;
  cout << "  The number of threads available    = " << thread_num <<  endl;
  cout << "  The matrix order n                 = " << n << endl;

  s = 1.0/sqrt( (double) n);
  wtime = omp_get_wtime ( );
  // Allocate space for the vectors to be used
  a = new double [n]; b = new double [n]; c = new double [n];
  // Define parallel region
# pragma omp parallel for default(shared) private (angle, i) reduction(+:Norm2)
  // Set up values for vectors  a and b
  for (i = 0; i < n; i++){
      angle = 2.0*M_PI*i/ (( double ) n);
      a[i] = s*(sin(angle) + cos(angle));
      b[i] =  s*sin(2.0*angle);
      c[i] = 0.0;
  }
  // Then perform the vector addition
  for (i = 0; i < n; i++){
     c[i] += a[i]+b[i];
  }
  // Compute now the norm-2
  Norm2 = 0.0;
  for (i = 0; i < n; i++){
     Norm2  += c[i]*c[i];
  }
// end parallel region
  wtime = omp_get_wtime ( ) - wtime;
  cout << setiosflags(ios::showpoint | ios::uppercase);
  cout << setprecision(10) << setw(20) << "Time used  for norm-2 computation=" << wtime  << endl;
  cout << " Norm-2  = " << Norm2 << endl;
  // Free up space
  delete[] a;
  delete[] b;
  delete[] c;
  return 0;
}

\end{minted}


\subsection*{\href{{https://github.com/CompPhysics/ComputationalPhysicsMSU/blob/master/doc/Programs/ParallelizationOpenMP/OpenMPmatrixmatrixmult.cpp}}{Matrix-matrix multiplication}}
This the matrix-matrix multiplication code with plain c++ memory allocation using OpenMP
















































































\begin{minted}[fontsize=\fontsize{9pt}{9pt},linenos=false,mathescape,baselinestretch=1.0,fontfamily=tt,xleftmargin=7mm]{text}
//  Matrix-matrix multiplication and Frobenius norm of a matrix with OpenMP
#include <cstdlib>
#include <iostream>
#include <cmath>
#include <iomanip>
#include  <omp.h>
# include <ctime>

using namespace std; // note use of namespace
int main (int argc, char* argv[])
{
  // read in dimension of square matrix
  int n = atoi(argv[1]);
  double **A, **B, **C;
  int i, j, k;
  int thread_num;
  double wtime, Fsum, s, angle;
  cout << "  Compute matrix product C = A * B and Frobenius norm." << endl;
  omp_set_num_threads(4);
  thread_num = omp_get_max_threads ();
  cout << "  The number of processors available = " << omp_get_num_procs () << endl ;
  cout << "  The number of threads available    = " << thread_num <<  endl;
  cout << "  The matrix order n                 = " << n << endl;

  s = 1.0/sqrt( (double) n);
  wtime = omp_get_wtime ( );
  // Allocate space for the two matrices
  A = new double*[n]; B = new double*[n]; C = new double*[n];
  for (i = 0; i < n; i++){
    A[i] = new double[n];
    B[i] = new double[n];
    C[i] = new double[n];
  }
  // Define parallel region
# pragma omp parallel for default(shared) private (angle, i, j, k) reduction(+:Fsum)
  // Set up values for matrix A and B and zero matrix C
  for (i = 0; i < n; i++){
    for (j = 0; j < n; j++) {
      angle = 2.0*M_PI*i*j/ (( double ) n);
      A[i][j] = s * ( sin ( angle ) + cos ( angle ) );
      B[j][i] =  A[i][j];
    }
  }
  // Then perform the matrix-matrix multiplication
  for (i = 0; i < n; i++){
    for (j = 0; j < n; j++) {
       C[i][j] =  0.0;    
       for (k = 0; k < n; k++) {
            C[i][j] += A[i][k]*B[k][j];
       }
    }
  }
  // Compute now the Frobenius norm
  Fsum = 0.0;
  for (i = 0; i < n; i++){
    for (j = 0; j < n; j++) {
      Fsum += C[i][j]*C[i][j];
    }
  }
  Fsum = sqrt(Fsum);
// end parallel region and letting only one thread perform I/O
  wtime = omp_get_wtime ( ) - wtime;
  cout << setiosflags(ios::showpoint | ios::uppercase);
  cout << setprecision(10) << setw(20) << "Time used  for matrix-matrix multiplication=" << wtime  << endl;
  cout << "  Frobenius norm  = " << Fsum << endl;
  // Free up space
  for (int i = 0; i < n; i++){
    delete[] A[i];
    delete[] B[i];
    delete[] C[i];
  }
  delete[] A;
  delete[] B;
  delete[] C;
  return 0;
}



\end{minted}

