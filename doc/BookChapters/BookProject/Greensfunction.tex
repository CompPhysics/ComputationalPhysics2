
% 
\chapter{Green's function theory}

\subsection{Single-particle Green's functions}

We consider first a particle in free space described by a single particle Hamiltonian $\hat{h}$. Its eigenstates and eigenenergies are

$$
\hat{h}\left|\phi_{n}\right\rangle=\varepsilon_{n}\left|\phi_{n}\right\rangle
$$

In general, if we put the particle in one of its
$\left|\phi_{n}\right\rangle$ orbits, it will remain in the same state
forever.

We prepare now the system in a generic state
$\left|\psi_{\mathrm{T}}\right\rangle$ ($\mathrm{T}$ stands for trial) and then
follow its time evolution. If the trial state is created at time
$t=0$, the wavefunction at a later time $t$ is given by

$$
\begin{aligned}
|\psi(t)\rangle & =e^{-i h_{1} t / \hbar}\left|\psi_{t r}\right\rangle \\
& =\sum_{n}\left|\phi_{n}\right\rangle e^{-i \varepsilon_{n} t / \hbar}\left\langle\phi_{n} \mid \psi_{t r}\right\rangle
\end{aligned}
$$


The above result shows that if one knows the eigentstates
$\left|\phi_{n}\right\rangle$, it is easy to compute the
time evolution. We expand $\left|\psi_{\mathrm{T}}\right\rangle$ in this
basis and let every component propagate independently. Eventually, at
time $t$, we want to know the probability amplitude for a  measurement
where  the particle is at a position $\mathbf{r}$,

$$
\begin{aligned}
\langle\mathbf{r} \mid \psi(t)\rangle & =\left\langle\mathbf{r}\left|e^{-i \hat{h} t / \hbar}\right| \psi_{\mathrm{T}}\right\rangle \\
& =\int d \mathbf{r}^{\prime}\left\langle\mathbf{r}\left|e^{-i \hat{h} t / \hbar}\right| \mathbf{r}^{\prime}\right\rangle\left\langle\mathbf{r}^{\prime} \mid \psi_{\mathrm{T}}\right\rangle
\end{aligned}
$$

$$
\begin{aligned}
& =\int d \mathbf{r}^{\prime} \sum_{n}\left\langle\mathbf{r} \mid \phi_{n}\right\rangle e^{-i \varepsilon_{n} t / \hbar}\left\langle\phi_{n} \mid \mathbf{r}^{\prime}\right\rangle\left\langle\mathbf{r}^{\prime} \mid \psi_{\mathrm{T}}\right\rangle \\
& \equiv \int d \mathbf{r}^{\prime} G\left(\mathbf{r}, \mathbf{r}^{\prime} ; t\right) \psi_{\mathrm{T}}\left(\mathbf{r}^{\prime}\right),
\end{aligned}
$$

which defines the propagator $G$.

 Once $G\left(\mathbf{r}, \mathbf{r}^{\prime} ; t\right)$
 is known it can be used to calculate the evolution of any initial
 state. However,there is more information included in the
 propagator. This is apparent from the expansion in the third line of the last equation.
 First, the braket $\left\langle\phi_{n} \mid
 \mathbf{r}\right\rangle=\left\langle\phi_{n}\left|\psi^{\dagger}(\mathbf{r})\right|
 0\right\rangle$ gives us the probability that putting a particle at
 position $\mathbf{r}$ and mesuring its energy right away, would make
 the system to collapse into the eigenstate
 $\left|\phi_{n}\right\rangle$.

 Second, the time evolution is a
 superposition of waves propagating with different energies and could
 be inverted to find the eigenspectrum. We could think of  an experiment in
 which the particle is put at position $\mathbf{r}$ and picked up at
 $\mathbf{r}^{\prime}$ after some time $t$. If one can do this for
 different positions and elapsed times and with good resolution, then a
 Fourier transform would simply give back the full eigenvalue
 spectrum. This  gives us the complete information about our particle.

We now want to apply the above ideas to see what we can learn by
adding and removing a particle in an environment when many other particles are
present. This can cause the particle to behave in an unxepected way,
induce collective excitations of the full systems, and so
on. Moreover, the role played by the physical vacuum in the above
example, is now taken by a many-body state (usually its ground
state). Thus, it is also possible to probe the system by removing
particles.


In the following we consider the Heisenberg description of the field operators (see discussions earlier on different pictures)),

$$
\psi_{s}^{\dagger}(\mathbf{r}, t)=e^{i H t / \hbar} \psi_{s}^{\dagger}(\mathbf{r}) e^{-i H t / \hbar}
$$

where the subscript $s$ serves to indicate possible internal degrees of freedom (spin, etc...). We omit the superscrips $\mathrm{H}$ (Heisenberg) and $S$ (Schr\"odinger) from the operators since the two pictures distinguish themselves by  the presence of the time variable, which appears only in the first case. Similarly,

$$
\psi_{s}(\mathbf{r}, t)=e^{i H t / \hbar} \psi_{s}(\mathbf{r}) e^{-i H t / \hbar},
$$


For the case of a general single-particle basis
$\left\{u_{\alpha}(\mathbf{r})\right\}$ one uses the following
creation and annihilation operators

$$
\begin{aligned}
& a_{\alpha}^{\dagger}(t)=e^{i H t / \hbar} a_{\alpha}^{\dagger} e^{-i H t / \hbar}, \\
& a_{\alpha}(t)=e^{i H t / \hbar} a_{\alpha} e^{-i H t / \hbar}
\end{aligned}
$$

which are related to $\psi_{s}^{\dagger}(\mathbf{r}, t)$ and
$\psi_{s}(\mathbf{r}, t)$.

{
\subsection{Eigenstates}


  In most applications the Hamiltonian is split in an unperturbed part $\hat{H}_{0}$ and a residual interaction $\hat{H}_I$

$$
\hat{H}=\hat{H}_{0}+\hat{H}_I .
$$

The $N$-body eigenstates  of the full Hamiltonian are $\left|\Psi_{n}^{N}\right\rangle$, while $\left|\Phi_{n}^{N}\right\rangle$ are the corresponding unperturbed ones

$$
\begin{aligned}
H\left|\Psi_{n}^{N}\right\rangle & =E_{n}^{N}\left|\Psi_{n}^{N}\right\rangle \\
H_{0}\left|\Phi_{n}^{N}\right\rangle & =E_{n}^{(0), N}\left|\Phi_{n}^{N}\right\rangle
\end{aligned}
$$

The definitions given in the following are general and do not depend on the type of interaction being used. Thus, most properties of Green's functions result from genaral principles of quantum mechanics and are valid for any system.

The two-points Green's function describes the propagation of one particle or one hole on top of the ground state $\left|\Psi_{0}^{N}\right\rangle$. This is defined by

$$
g_{s s^{\prime}}\left(\mathbf{r}, t ; \mathbf{r}^{\prime}, t^{\prime}\right)=-\frac{i}{\hbar}\left\langle\Psi_{0}^{N}\left|T\left[\psi_{s}(\mathbf{r}, t) \psi_{s^{\prime}}^{\dagger}\left(\mathbf{r}^{\prime}, t^{\prime}\right)\right]\right| \Psi_{0}^{N}\right\rangle
$$

where $T[\cdots]$ is the time ordering operator that imposes a change of sign for each exchange of two fermion operators

$$
T\left[\psi_{s}(\mathbf{r}, t) \psi_{s^{\prime}}^{\dagger}\left(\mathbf{r}^{\prime}, t^{\prime}\right)\right]= \begin{cases}\psi_{s}(\mathbf{r}, t) \psi_{s^{\prime}}^{\dagger}\left(\mathbf{r}^{\prime}, t^{\prime}\right), & t>t^{\prime} \\ \pm \psi_{s^{\prime}}^{\dagger}\left(\mathbf{r}^{\prime}, t^{\prime}\right) \psi_{s}(\mathbf{r}, t), & t^{\prime}>t\end{cases}
$$

where the upper (lower) sign is for bosons (fermions). A similar definition can be given for the non interacting state $\left|\Phi_{0}^{N}\right\rangle$, in this case the Heisenberg operators discussed above must evolve only according to $H_{0}$ and the notation $g^{(0)}$ is used.

\subsection{Fourier transform}

If the Hamiltonian does not depend on time, the propagator defined above
depends only on the difference $t-t^{\prime}$

$$
\begin{aligned}
g_{s s^{\prime}}\left(\mathbf{r}, \mathbf{r}^{\prime} ; t-t^{\prime}\right)= & -\frac{i}{\hbar} \theta\left(t-t^{\prime}\right)\left\langle\Psi_{0}^{N}\left|\psi_{s}(\mathbf{r}) e^{-i\left(H-E_{0}^{N}\right)\left(t-t^{\prime}\right) / \hbar} \psi_{s^{\prime}}^{\dagger}\left(\mathbf{r}^{\prime}\right)\right| \Psi_{0}^{N}\right\rangle \\
& \mp \frac{i}{\hbar} \theta\left(t^{\prime}-t\right)\left\langle\Psi_{0}^{N}\left|\psi_{s^{\prime}}^{\dagger}\left(\mathbf{r}^{\prime}\right) e^{i\left(H-E_{0}^{N}\right)\left(t-t^{\prime}\right) / \hbar} \psi_{s}(\mathbf{r})\right| \Psi_{0}^{N}\right\rangle .
\end{aligned}
$$

In this case it is useful to Fourier transform with respect to time and define

$$
g_{s s^{\prime}}\left(\mathbf{r}, \mathbf{r}^{\prime} ; \omega\right)=\int d \tau e^{i \omega \tau} g_{s s^{\prime}}\left(\mathbf{r}, \mathbf{r}^{\prime} ; \tau\right)
$$


\subsection{Rewrite of propagator}

Using the relation

$$
\theta( \pm \tau)=\mp \lim _{\eta \rightarrow 0^{+}} \frac{1}{2 \pi i} \int_{-\infty}^{+\infty} d \omega \frac{e^{-i \omega \tau}}{\omega \pm i \eta}
$$

we obtain

$$
\begin{aligned}
g_{s s^{\prime}}\left(\mathbf{r}, \mathbf{r}^{\prime} ; \omega\right)= & g_{s s^{\prime}}^{p}\left(\mathbf{r}, \mathbf{r}^{\prime} ; \omega\right)+g_{s s^{\prime}}^{h}\left(\mathbf{r}, \mathbf{r}^{\prime} ; \omega\right) \\
= & \left\langle\Psi_{0}^{N}\left|\psi_{s}(\mathbf{r}) \frac{1}{\hbar \omega-\left(H-E_{0}^{N}\right)+i \eta} \psi_{s^{\prime}}^{\dagger}\left(\mathbf{r}^{\prime}\right)\right| \Psi_{0}^{N}\right\rangle \\
& \mp\left\langle\Psi_{0}^{N}\left|\psi_{s^{\prime}}^{\dagger}\left(\mathbf{r}^{\prime}\right) \frac{1}{\hbar \omega+\left(H-E_{0}^{N}\right)-i \eta} \psi_{s}(\mathbf{r})\right| \Psi_{0}^{N}\right\rangle,
\end{aligned}
$$

In the last equation, $g^{p}$ propagates a particle from
$\mathbf{r}^{\prime}$ to $\mathbf{r}$, while $g^{h}$ propagates a hole
from $\mathbf{r}$ to $\mathbf{r}^{\prime}$. Note that the
interpretation is that a particle is added at $\mathbf{r}^{\prime}$,
and later on some (indistiguishable) particle is removed from
$\mathbf{r}^{\prime}$ (and similarly for holes). In the meantime, it
is the fully correlated $(N \pm 1)$ body system that propagates.

In many cases,  in particular in
the vicinity of the Fermi surface, this motion mantains many
characteristics that are typical of a particle moving in free space,
even if the motion itself could actually be a collective excitation of
many constituents. But since it looks like a single particle state we
may still refer to it as quasiparticle.

{
\subsection{Orthonormal basis set definitions}

The same definitions can be made for any orthonormal basis $\{\alpha\}$, leading to the realtions

$$
g_{\alpha \beta}\left(t, t^{\prime}\right)=-\frac{i}{\hbar}\left\langle\Psi_{0}^{N}\left|T\left[a_{\alpha}(t) a_{\beta}^{\dagger}\left(t^{\prime}\right)\right]\right| \Psi_{0}^{N}\right\rangle
$$

where

$$
g_{s s^{\prime}}\left(\mathbf{r}, t ; \mathbf{r}^{\prime}, t^{\prime}\right)=\sum_{\alpha \beta} u_{\alpha}(\mathbf{r}, s) g_{\alpha \beta}\left(t, t^{\prime}\right) u_{\beta}^{*}\left(\mathbf{r}^{\prime}, s^{\prime}\right)
$$

and

$$
\begin{aligned}
g_{\alpha \beta}(\omega)= & \left\langle\Psi_{0}^{N}\left|a_{\alpha} \frac{1}{\hbar \omega-\left(H-E_{0}^{N}\right)+i \eta} a_{\beta}^{\dagger}\right| \Psi_{0}^{N}\right\rangle \\
& \mp\left\langle\Psi_{0}^{N}\left|a_{\beta}^{\dagger} \frac{1}{\hbar \omega+\left(H-E_{0}^{N}\right)-i \eta} a_{\alpha}\right| \Psi_{0}^{N}\right\rangle .
\end{aligned}
$$

\subsection{Lehmann representation}

As discussed above for the one particle case, the information
contained in the propagators becomes more clear if one Fourier
transforms the time variable and inserts a completness for the
intermediate states. This is so because it makes the spectrum and the
transition amplitudes to apper explicitely. Using the completeness
relations for the $(N \pm 1)$-body systems in the last equation, one has

$$
\begin{aligned}
g_{\alpha \beta}(\omega)= & \sum_{n} \frac{\left\langle\Psi_{0}^{N}\left|a_{\alpha}\right| \Psi_{n}^{N+1}\right\rangle\left\langle\Psi_{n}^{N+1}\left|a_{\beta}^{\dagger}\right| \Psi_{0}^{N}\right\rangle}{\hbar \omega-\left(E_{n}^{N+1}-E_{0}^{N}\right)+i \eta} \\
& \mp \sum_{k} \frac{\left\langle\Psi_{0}^{N}\left|a_{\beta}^{\dagger}\right| \Psi_{k}^{N-1}\right\rangle\left\langle\Psi_{k}^{N-1}\left|a_{\alpha}\right| \Psi_{0}^{N}\right\rangle}{\hbar \omega-\left(E_{0}^{N}-E_{k}^{N-1}\right)-i \eta} .
\end{aligned}
$$

which is known as the Lehmann repressentation of a many-body Green's
function. Here, the first and second terms on the left hand
side describe the propagation of a (quasi)particle and a (quasi)hole
excitations.


The poles in last equation are the energies relatives to the
$\left|\Psi_{0}^{N}\right\rangle$ ground state. Hence they give the
energies actually relased in a capture reaction experiment to a bound
state of $\left|\Psi_{n}^{N+1}\right\rangle$. The residues are
transition amplitudes for the addition of a particle and take the name
of spectroscopic amplitudes. They play the same role as the
$\left\langle\phi_{n} \mid \mathbf{r}\right\rangle$ wave function.
In fact these energies and amplitudes are solutions of a
Schr\"odinger-like equation: the Dyson equation. The hole part of the
propagator gives instead information on the process of particle
emission, the poles being the exact energy absorbed in the
process. For example, in the single particle Green's function of an atome,
the quasiparticle and quasihole poles are respectively the
electron affinities an ionization energies.

We will look at the physical significance of spectroscopic amplitudes
below and derive the Dyson equation (which is the
fundamental equation in many-body Green's function theory) below.

\subsection{Spectral function}

As a last definition, we rewrite the above Lehmann representation  in a form
that can compared more easily to experiments. By using the relation

$$
\frac{1}{x \pm i \eta}=\mathcal{P} \frac{1}{x} \mp i \pi \delta(x)
$$
}
we can extract the one-body spectral function

$$
S_{\alpha \beta}(\omega)=S_{\alpha \beta}^{p}(\omega)+S_{\alpha \beta}^{h}(\omega),
$$

where the particle and hole components are

$$
\begin{aligned}
S_{\alpha \beta}^{p}(\omega) & =-\frac{1}{\pi} \operatorname{Im} g_{\alpha \beta}^{p}(\omega) \\
& =\sum_{n}\left\langle\Psi_{0}^{N}\left|a_{\alpha}\right| \Psi_{n}^{N+1}\right\rangle\left\langle\Psi_{n}^{N+1}\left|a_{\beta}^{\dagger}\right| \Psi_{0}^{N}\right\rangle \delta\left(\hbar \omega-\left(E_{n}^{N+1}-E_{0}^{N}\right)\right) \\
S_{\alpha \beta}^{h}(\omega) & =\frac{1}{\pi} \operatorname{Im} g_{\alpha \beta}^{h}(\omega) \\
& =\mp \sum_{k}\left\langle\Psi_{0}^{N}\left|a_{\beta}^{\dagger}\right| \Psi_{k}^{N-1}\right\rangle\left\langle\Psi_{k}^{N-1}\left|a_{\alpha}\right| \Psi_{0}^{N}\right\rangle \delta\left(\hbar \omega-\left(E_{0}^{N}-E_{k}^{N-1}\right)\right) .
\end{aligned}
$$


\subsection{Dispersion relation}

The diagonal part of the spectral function is interpreted as the
probability of adding $\left[S_{\alpha \alpha}^{p}(\omega)\right]$ or
removing $\left[S_{\alpha \alpha}^{h}(\omega)\right]$ one particle in
the state $\alpha$ leaving the residual system in a state of energy
$\omega$.

By comparing the last two equations to the Lehmann representation, we
see that the propagator is completely constrained by its imaginary
part. We have

$$
g_{\alpha \beta}(\omega)=\int d \omega^{\prime} \frac{S_{\alpha \beta}^{p}\left(\omega^{\prime}\right)}{\omega-\omega^{\prime}+i \eta}+\int d \omega^{\prime} \frac{S_{\alpha \beta}^{h}\left(\omega^{\prime}\right)}{\omega-\omega^{\prime}-i \eta} .
$$

In general the single particle propagator of a finite system has
isolated poles in correspondence to the bound eigenstates of the
$(N+1)$-body system. For larger enegies, where
$\left|\Psi_{n}^{N+1}\right\rangle$ are states in the continuum, it
develops a branch cut. The particle propagator $g^{p}(\omega)$ is
analytic in the upper half of the complex plane, and so is the full
propagator for $\omega \geq
E_{0}^{N+1}-E_{0}^{N}$. Analogously, the hole propagator has poles for
$\omega \leq E_{0}^{N}-E_{0}^{N-1}$ and is analytic in the lower
complex plane. Note that high excitation energies in the (N-1)body
system correspond to negative values of the poles
$E_{0}^{N}-E_{k}^{N-1}$, so $g^{h}(\omega)$ develops a branch cut for
large negative energies.


\subsection{Expectation values}

The one-body density matrix can be obtained from the one-body
propagator. One simply chooses the appropriate time ordering

$$
\rho_{\alpha \beta}=\left\langle\Psi_{0}^{N}\left|a_{\beta}^{\dagger} a_{\alpha}\right| \Psi_{0}^{N}\right\rangle= \pm i \hbar \lim _{t^{\prime} \rightarrow t^{+}} g_{\alpha \beta}\left(t, t^{\prime}\right)
$$

(where the upper sign is for bosons and the lower one is for fermions). Alternatively, the hole spectral function can be used

$$
\rho_{\alpha \beta}=\mp \int d \omega S_{\alpha \beta}^{h}(\omega) \text {. }
$$


\subsection{Single-particle Green's functions}

Thus, the expectation value of a one-body operator for the
ground states $\left|\Psi_{0}^{N}\right\rangle$ is usually as

$$
\begin{aligned}
\left\langle\Psi_{0}^{N}|O| \Psi_{0}^{N}\right\rangle & =\mp \sum_{\alpha \beta} \int d \omega o_{\alpha \beta} S_{\beta \alpha}^{h}(\omega) \\
& = \pm i \hbar \lim _{t^{\prime} \rightarrow t^{+}} \sum_{\alpha \beta} o_{\alpha \beta} g_{\beta \alpha}\left(t, t^{\prime}\right)
\end{aligned}
$$

and both terms are equivalent.


From the particle spectral function, one can extract the quantity

$$
d_{\alpha \beta}=\left\langle\Psi_{0}^{N}\left|a_{\alpha} a_{\beta}^{\dagger}\right| \Psi_{0}^{N}\right\rangle=\int d \omega S_{\beta \alpha}^{p}(\omega)
$$

which leads to the following sum rule

$$
\int d \omega S_{\alpha \beta}(\omega)=d_{\alpha \beta} \mp \rho_{\alpha \beta}=\left\langle\Psi_{0}^{N}\left|\left[a_{\alpha}, a_{\beta}^{\dagger}\right]_{\mp}\right| \Psi_{0}^{N}\right\rangle=\delta_{\alpha \beta} .
$$


For the case of an Hamiltonian containing up to two-body interactions only we have

$$
\begin{aligned}
H & =U+V \\
& =\sum_{\alpha \beta} t_{\alpha \beta} a_{\alpha}^{\dagger} a_{\beta}+\frac{1}{4} \sum_{\alpha \beta \gamma \delta} v_{\alpha \beta, \gamma \delta} a_{\alpha}^{\dagger} a_{\beta}^{\dagger} a_{\delta} a_{\gamma},
\end{aligned}
$$

and we can derive an important sum rule that relates the total energy
of the state $\left|\Psi_{0}^{N}\right\rangle$ to its one-body Green's
function. To derive this, one makes use of the equation of motion for
Heisenberg operators, which gives

$$
i \hbar \frac{d}{d t} a_{\alpha}(t)=e^{i H t / \hbar}\left[a_{\alpha}, H\right] e^{-i H t / \hbar},
$$

with

$$
\left[a_{\alpha}, H\right]=\sum_{\beta} t_{\alpha \beta} a_{\beta}+\frac{1}{2} \sum_{\beta \gamma \delta} v_{\alpha \beta \gamma \delta} a_{\beta}^{\dagger} a_{\delta} a_{\gamma}
$$

which is valid for both fermions and bosons.



If one uses the last equation we can derive the propagator as function of time through

\[
\begin{aligned}
i \hbar \frac{\partial}{\partial t} g_{\alpha \beta}\left(t-t^{\prime}\right)= & \delta\left(t-t^{\prime}\right) \delta_{\alpha \beta}+\sum_{\gamma} t_{\alpha \gamma} g_{\gamma \beta}\left(t-t^{\prime}\right) \\
& -\frac{i}{\hbar} \sum_{\eta \gamma \zeta} \frac{1}{2} v_{\alpha \eta, \gamma \zeta}\left\langle\Psi_{0}^{N}\left|T\left[a_{\eta}^{\dagger}(t) a_{\zeta}(t) a_{\gamma}(t) a_{\beta}^{\dagger}\left(t^{\prime}\right)\right]\right| \Psi_{0}^{N}\right\rangle
\end{aligned}
\]


The braket in the last line contains the four-point Green's function to be discussed below.
The four-point Green's function can describe the simultaneous propagation of
two particles. Thus, one sees that applying the equation of motion to
a propagator leads to relations which contain Green's functions of
higher order. This result is particularly important because it shows
there exist a hierarchy between propagators, so that the exact
equations that determine the one-body function will depend on the
two-body one, the two-body function will contain contributions from
three-body propagators, and so on.

For the moment we just want to select a particular order of the
operators in order to extract the one- and two-body
density matrices. To do this, we chose $t^{\prime}$ to be a later time
than $t$ and take its limit to the latter from above. This yields

$$
\pm i \hbar \lim _{t^{\prime} \rightarrow t^{+}} \sum_{\alpha} \frac{\partial}{\partial t} g_{\alpha \alpha}\left(t-t^{\prime}\right)=\langle T\rangle+2\langle V\rangle
$$

(note that for $t \neq t^{\prime}$, the term
$\delta\left(t-t^{\prime}\right)=0$ and it does not contribute to the
limit). This result can also be expressed in energy representation by
inverting the Fourier transformation. We have then

$$
\lim _{\tau \rightarrow 0^{-}} \frac{\partial}{\partial \tau} g_{\alpha \beta}(\tau)=-\int d \omega \omega S_{\alpha \beta}^{h}(\omega)
$$


\subsection{Expectation values of the energy}

By combining the above results we arrive at

$$
\begin{aligned}
\langle H\rangle=\langle U\rangle+\langle V\rangle & = \pm i \hbar \frac{1}{2} \lim _{t^{\prime} \rightarrow t^{+}} \sum_{\alpha \beta}\left\{\delta_{\alpha \beta} \frac{\partial}{\partial t}+t_{\alpha \beta}\right\} g_{\beta \alpha}\left(t-t^{\prime}\right) \\
& =\mp \frac{1}{2} \sum_{\alpha \beta} \int d \omega\left\{\delta_{\alpha \beta} \omega+t_{\alpha \beta}\right\} S_{\beta \alpha}^{h}(\omega)
\end{aligned}
$$

We used here the relation $[A, B C]_{-}=[A, B] C-B[C, A]=\{A, B\}
C-B\{C, A\}$ which is valid for both commutators and anticommutators


Surprisingly, for an Hamiltonian containing only two-body forces it is
possible to extract the ground state energy by knowing only the
one-body propagator. 

When interactions among three or more particles are present, this
relation has to be augmented to include additional terms. In these
cases higher order Green's functions will appear explicitly.


\subsection{Higher order  Green's functions}

The definition of the one-body Green's function  can be extended to Green's functions for the
propagation of more than one particle. In general, for each additional
particle it will be necessary to introduce one additional creation and
one annihilation operator. Thus a $2 n$-points Green's function will
propagate a maximum of $n$ quasiparticles. The explicit definition of
the four-point propagator is

$$
g_{\alpha \beta, \gamma \delta}^{4-p t}\left(t_{1}, t_{2} ; t_{1}^{\prime}, t_{2}^{\prime}\right)=-\frac{i}{\hbar}\left\langle\Psi_{0}^{N}\left|T\left[a_{\beta}\left(t_{2}\right) a_{\alpha}\left(t_{1}\right) a_{\gamma}^{\dagger}\left(t_{1}^{\prime}\right) a_{\delta}^{\dagger}\left(t_{2}^{\prime}\right)\right]\right| \Psi_{0}^{N}\right\rangle
$$

while the six-point case is

$$
\begin{aligned}
& g_{\alpha \beta \gamma, \mu \nu \lambda}^{6-p t}\left(t_{1}, t_{2}, t_{3} ; t_{1}^{\prime}, t_{2}^{\prime}, t_{3}^{\prime}\right)= \\
& \quad-\frac{i}{\hbar}\left\langle\Psi_{0}^{N}\left|T\left[a_{\gamma}\left(t_{3}\right) a_{\beta}\left(t_{2}\right) a_{\alpha}\left(t_{1}\right) a_{\mu}^{\dagger}\left(t_{1}^{\prime}\right) a_{\nu}^{\dagger}\left(t_{2}^{\prime}\right) a_{\lambda}^{\dagger}\left(t_{3}^{\prime}\right)\right]\right| \Psi_{0}^{N}\right\rangle,
\end{aligned}
$$


\subsection{Interpretations}

It should be noted that the actual number of particles that are
propagated by these objects depends on the ordering of the time
variables. Therefore the information on transitions between
eigenstates of the systems with $N, N \pm 1$ and $N \pm 2$ bodies are
all encoded in the four-point propagator, while additional states of $N \pm 3$-body
states are included in the six-point propagator. Obviously, the presence of so many
time variables makes the use of these functions extremely difficult
(and even impossible, in many cases). However, it is still useful to
consider only certain time orderings which allow to extract the
information not included in the two-point propagator.


\subsection{Two-particle-two-hole propagator}

The two-particle-two-hole propagator is a two-times Green's function
defined as

$$
g_{\alpha \beta, \gamma \delta}^{I I}\left(t, t^{\prime}\right)=-\frac{i}{\hbar}\left\langle\Psi_{0}^{N}\left|T\left[a_{\beta}(t) a_{\alpha}(t) a_{\gamma}^{\dagger}\left(t^{\prime}\right) a_{\delta}^{\dagger}\left(t^{\prime}\right)\right]\right| \Psi_{0}^{N}\right\rangle
$$

which corresponds to the limit $t_{1}^{\prime}=t_{2}^{\prime+}$ and $t_{2}=t_{1}^{+}$of $g^{4-p t}$.


As for the case of $g_{\alpha \beta}\left(t, t^{\prime}\right)$, if
the Hamiltonian is time-independent, the last equation is a function of the
time difference only. Therefore it has a Lehmann representation
containing the exact spectrum of the $(N \pm 2)$-body systems

$$
\begin{aligned}
g_{\alpha \beta, \gamma \delta}^{I I}(\omega) & =\sum_{n} \frac{\left\langle\Psi_{0}^{N}\left|a_{\beta} a_{\alpha}\right| \Psi_{n}^{N+2}\right\rangle\left\langle\Psi^{N+2}\left|a_{\gamma}^{\dagger} a_{\delta}^{\dagger}\right| \Psi_{0}^{N}\right\rangle}{\omega-\left(E_{n}^{N+2}-E_{0}^{N}\right)+i \eta} \\
& -\sum_{k} \frac{\left\langle\Psi_{0}^{N}\left|a_{\gamma}^{\dagger} a_{\delta}^{\dagger}\right| \Psi_{k}^{N-2}\right\rangle\left\langle\Psi_{k}^{N-2}\left|a_{\beta} a_{\alpha}\right| \Psi_{0}^{N}\right\rangle}{\omega-\left(E_{0}^{N}-E_{k}^{N-2}\right)-i \eta}
\end{aligned}
$$

\subsection{Spectral functions}

Similarly one defines the two-particle and two-hole spectral functions

$$
S_{\alpha \beta}^{I I}, \gamma \delta(\omega)=S_{\alpha \beta, \gamma \delta}^{p p}(\omega)+S_{\alpha \beta, \gamma \delta}^{h h}(\omega)
$$

and

$$
\begin{aligned}
S_{\alpha \beta, \gamma \delta}^{p p}(\omega) & =-\frac{1}{\pi} \operatorname{Im} g_{\alpha \beta, \gamma \delta}^{p p}(\omega) \\
& =\sum_{n}\left\langle\Psi_{0}^{N}\left|a_{\beta} a_{\alpha}\right| \Psi_{n}^{N+2}\right\rangle\left\langle\Psi_{n}^{N+2}\left|a_{\gamma}^{\dagger} a_{\delta}^{\dagger}\right| \Psi_{0}^{N}\right\rangle \delta\left(\hbar \omega-\left(E_{n}^{N+2}-E_{0}^{N}\right)\right), \\
S_{\alpha \beta, \gamma \delta}^{h h}(\omega) & =\frac{1}{\pi} \operatorname{Im} g_{\alpha \beta, \gamma \delta}^{h h}(\omega) \\
& =-\sum_{k}\left\langle\Psi_{0}^{N}\left|a_{\gamma}^{\dagger} a_{\delta}^{\dagger}\right| \Psi_{k}^{N-2}\right\rangle\left\langle\Psi_{k}^{N-2}\left|a_{\beta} a_{\alpha}\right| \Psi_{0}^{N}\right\rangle \delta\left(\hbar \omega-\left(E_{0}^{N}-E_{k}^{N-2}\right)\right) .
\end{aligned}
$$

Based on these equations it is easy to obtain relations for the two-body density matrix

$$
\Gamma_{\alpha \beta, \gamma \delta}=\left\langle\Psi^{N}\left|a_{\gamma}^{\dagger} a_{\delta}^{\dagger} a_{\beta} a_{\alpha}\right| \Psi^{N}\right\rangle=-\int d \omega S_{\alpha \beta, \gamma \delta}^{h h}(\omega)
$$

and, hence, for the expectation value of any two-body operator

$$
\begin{aligned}
\left\langle\Psi_{0}^{N}|V| \Psi_{0}^{N}\right\rangle & =-\sum_{\alpha \beta \gamma \delta} \int d \omega v_{\alpha \beta, \gamma \delta} S_{\gamma \delta, \alpha \beta}^{h}(\omega) \\
& =+i \hbar \lim _{t^{\prime} \rightarrow t^{+}} \frac{1}{4} \sum_{\alpha \beta \gamma \delta} v_{\alpha \beta, \gamma \delta} g_{\gamma \delta, \alpha \beta}^{I I}\left(t, t^{\prime}\right) .
\end{aligned}
$$

\subsection{Polarization propagator}

The polarization propagator $\Pi_{\alpha \beta, \gamma \delta}$
corresponds to the time ordering of $g^{4-p t}$ in which a
particle-hole excitation is created at one single time. Therefore, no
process involving particle transfer in included. However it describes
transition to the excitations of the system, as long as they can be
reached with a one-body operator. For example, this includes
collective modes of a nucleus. This is defined as

$$
\begin{aligned}
\Pi_{\alpha \beta, \gamma \delta}\left(t, t^{\prime}\right)=- & \frac{i}{\hbar}\left\langle\Psi_{0}^{N}\left|T\left[a_{\beta}^{\dagger}(t) a_{\alpha}(t) a_{\gamma}^{\dagger}\left(t^{\prime}\right) a_{\delta}\left(t^{\prime}\right)\right]\right| \Psi_{0}^{N}\right\rangle \\
& +\frac{i}{\hbar}\left\langle\Psi_{0}^{N}\left|a_{\beta}^{\dagger} a_{\alpha}\right| \Psi_{0}^{N}\right\rangle\left\langle\Psi_{0}^{N}\left|a_{\gamma}^{\dagger} a_{\delta}\right| \Psi_{0}^{N}\right\rangle
\end{aligned}
$$


  After including a completeness of $\left|\Psi_{n}^{N}\right\rangle$
states, the contribution of to the ground states (at zero
energy) is cancelled by the last term in the equation. Thus one can
Fourier transform to the Lehmann representation

$$
\begin{aligned}
\Pi_{\alpha \beta, \gamma \delta}(\omega) & =\sum_{n \neq 0} \frac{\left\langle\Psi_{0}^{N}\left|a_{\beta}^{\dagger} a_{\alpha}\right| \Psi_{n}^{N}\right\rangle\left\langle\Psi_{n}^{N}\left|a_{\gamma}^{\dagger} a_{\delta}\right| \Psi_{0}^{N}\right\rangle}{\omega-\left(E_{n}^{N}-E_{0}^{N}\right)+i \eta} \\
& -\sum_{n \neq 0} \frac{\left\langle\Psi_{0}^{N}\left|a_{\gamma}^{\dagger} a_{\delta}\right| \Psi_{n}^{N}\right\rangle\left\langle\Psi_{n}^{N}\left|a_{\beta}^{\dagger} a_{\alpha}\right| \Psi_{0}^{N}\right\rangle}{\omega+\left(E_{n}^{N}-E_{0}^{N}\right)-i \eta}
\end{aligned}
$$


\subsection{Transition matrix elements}

Note that $\Pi_{\alpha \beta, \gamma \delta}(\omega)=\Pi_{\delta
  \gamma, \beta \alpha}(-\omega)$ due to time reversal symmetry. Also
the forward and backward parts carry the same information.

Once again, the residues of the propagator can be used to
calculate expectation values. In this case, given a one-body operator
we obtain the transition matrix elements to any excited state

$$
\left\langle\Psi_{n}^{N}|O| \Psi_{0}^{N}\right\rangle=\sum_{\alpha \beta} o_{\beta \alpha}\left\langle\Psi_{n}^{N}\left|a_{\beta}^{\dagger} a_{\alpha}\right| \Psi_{0}^{N}\right\rangle
$$


\subsection{Coupling to experiment}

Here we explore the connection between the information contained in
various propagators and experimental data. The focus is on the
experimental properties that are probed by the removal of
particles. Also, from now on, we will only consider fermionic systems.

An important case is when the spectrum for the $N \pm 1$-particle
system near the Fermi energy involves discrete bound states. This
happens in finite system like nuclei or molecules. In these cases the
main quantity of interest is the overlap wave function

$$
\begin{aligned}
\psi_{k}^{\text {overlap }}(\mathbf{r})= & \left\langle\Psi_{k}^{N-1}\left|\psi_{s}(\mathbf{r})\right| \Psi_{0}^{N}\right\rangle \\
= & \sqrt{N} \int d \mathbf{r}_{2} \int d \mathbf{r}_{3} \cdots \int d \mathbf{r}_{N} \\
& \quad \times\left[\Psi_{k}^{N-1}\left(\mathbf{r}_{2}, \mathbf{r}_{3}, \ldots \mathbf{r}_{N}\right)\right]^{*} \Psi_{0}^{N}\left(\mathbf{r}, \mathbf{r}_{2}, \mathbf{r}_{3}, \ldots \mathbf{r}_{N}\right) .
\end{aligned}
$$


This integral comes out in the description of most
particle knock-out processes because it represents the matrix element
between the initial and final states, in the case when the emitted
particle is ejected with energy large enough the it interacts only
weakly with the residual system. The quantity of interest here is the
so called spectroscopic factor to the final state $k$,

$$
S_{k}=\int d \mathbf{r}\left|\psi_{k}^{\text {overlap }}(\mathbf{r})\right|^{2}
$$

When the system is made of completely non interacting particles,
$S_{k}$ is unity. In real cases however, correlations among the
constituents reduce this value. The possibility of extracting this
quantity from experimental data gives us information on the spectral
function and therefore on the structure of the correlated system.


\subsection{Spectroscopic strength from particle emission}

In order to make the connection with experimental data obtained from
knockout reactions, it is useful to consider the response of a system
to a weak probe. The hole spectral function
can be substantially "observed" these reactions. The general idea is
to transfer a large amount of momentum and energy to a particle of a
bound system in the ground state. This is then ejected from the
system, and one ends up with a fast-moving particle and a bound
$(N-1)$-particle system. By observing the momentum of the ejected
particle it is then possible the reconstruct the spectral function of
the system, provided that the interaction between the ejected particle
and the remainder is sufficiently weak or treated in a controlled
fashion, for example by constraining this treatment with information from
other experimental data.

We assume that the $N$-particle system is initially in its ground state,

$$
\left|\Psi_{i}\right\rangle=\left|\Psi_{0}^{N}\right\rangle
$$

and makes a transition to a final $N$-particle eigenstate

$$
\left|\Psi_{f}\right\rangle=a_{p}^{\dagger}\left|\Psi_{n}^{N-1}\right\rangle
$$

composed of a bound $(N-1)$-particle eigenstate, $\left|\Psi_{n}^{N-1}\right\rangle$, and a particle with momentum $\boldsymbol{p}$.


For simplicity we consider the transition matrix elements for a scalar external probe

$$
\rho(\boldsymbol{q})=\sum_{j=1}^{N} \exp \left(i \boldsymbol{q} \cdot \boldsymbol{r}_{j}\right)
$$

which transfers momentum $\boldsymbol{q}$ to a particle. Suppressing other possible sp quantum numbers, like e.g. spin, the second-quantized form of this operator is given by

$$
\hat{\rho}(\boldsymbol{q})=\sum_{\boldsymbol{p}, \boldsymbol{p}^{\prime}}\left\langle\boldsymbol{p}|\exp (i \boldsymbol{q} \cdot \boldsymbol{r})| \boldsymbol{p}^{\prime}\right\rangle a_{\boldsymbol{p}}^{\dagger} a_{\boldsymbol{p}^{\prime}}=\sum_{\boldsymbol{p}} a_{\boldsymbol{p}}^{\dagger} a_{\boldsymbol{p}-\boldsymbol{q}}
$$


\subsection{Transition matrix element}

The transition matrix element now becomes

$$
\begin{aligned}
\left\langle\Psi_{f}|\hat{\rho}(\boldsymbol{q})| \Psi_{i}\right\rangle & =\sum_{\boldsymbol{p}^{\prime}}\left\langle\Psi_{n}^{N-1}\left|a_{\boldsymbol{p}} a_{\boldsymbol{p}^{\prime}}^{\dagger} a_{\boldsymbol{p}^{\prime}-\boldsymbol{q}}\right| \Psi_{0}^{N}\right\rangle \\
& =\sum_{\boldsymbol{p}^{\prime}}\left\langle\Psi_{n}^{N-1}\left|\delta_{\boldsymbol{p}^{\prime}, \boldsymbol{p}} a_{\boldsymbol{p}^{\prime}-\boldsymbol{q}}+a_{\boldsymbol{p}^{\prime}}^{\dagger} a_{\boldsymbol{p}^{\prime}-\boldsymbol{q}} a_{\boldsymbol{p}}\right| \Psi_{0}^{N}\right\rangle \\
& \approx\left\langle\Psi_{n}^{N-1}\left|a_{\boldsymbol{p}-\boldsymbol{q}}\right| \Psi_{0}^{N}\right\rangle .
\end{aligned}
$$

    
The last line is obtained in the so-called Impulse Approximation (or
Sudden Approximation), where it is assumed that the ejected particle
is the one that has absorbed the momentum from the external
field. This is a very good approximation whenever the momentum
$\boldsymbol{p}$ of the ejectile is much larger than typical momenta
for the particles in the bound states; the neglected term 
is then very small, as it involves the removal of a particle with
momentum $\boldsymbol{p}$ from $\left|\Psi_{0}^{N}\right\rangle$.

There is one other assumption in the derivation: the fact that the
final eigenstate of the $N$-particle system was written in the form of
a plane-wave state for the ejectile on top of an $(N-1)$-particle
eigenstate. This is again a good approximation if the ejectile
momentum is large enough, as can be understood by rewriting the
Hamiltonian in the $N$-particle system as

$$
H_{N}=\sum_{i=1}^{N} \frac{\boldsymbol{p}_{i}^{2}}{2 m}+\sum_{i<j=1}^{N} V(i, j)=H_{N-1}+\frac{\boldsymbol{p}_{N}^{2}}{2 m}+\sum_{i=1}^{N-1} V(i, N)
$$


The last term in the above equation represents the Final State Interaction, or
the interaction between the ejected particle $N$ and the other
particles $1 \ldots N-1$. If the relative momentum between particle
$N$ and the others is large enough their mutual interaction can be
neglected, and $H_{N} \approx H_{N-1}+\boldsymbol{p}_{N}^{2} / 2
m$. The results above are called the Plane Wave Impulse
Approximation or PWIA knock-out amplitude, for obvious reasons, and is
precisely a removal amplitude (in the momentum representation)
appearing in the Lehmann representation of the sp propagator.

The cross section of the knock-out reaction, where the momentum and
energy of the ejected particle and the probe are either measured or
known, is according to Fermi's golden rule proportional to

$$
d \sigma \sim \sum_{n} \delta\left(\omega+E_{i}-E_{f}\right)\left|\left\langle\Psi_{f}|\hat{\rho}(\boldsymbol{q})| \Psi_{i}\right\rangle\right|^{2}
$$

where the energy-conserving $\delta$-function contains the energy
transfer $\omega$ of the probe, and the initial and final energies of
the system are $E_{i}=E_{0}^{N}$ and
$E_{f}=E_{n}^{N-1}+\boldsymbol{p}^{2} / 2 m$, respectively. Note that
the internal state of the residual $N-1$ system is not measured, hence
the summation over $n$. Defining the missing momentum
$\boldsymbol{p}_{\text {miss }}$ and missing energy $E_{\text {miss
}}$ of the knock-out reaction as ${ }^{1}$

$$
\boldsymbol{p}_{m i s s}=\boldsymbol{p}-\boldsymbol{q}
$$

and


\[
E_{\text {miss }}=\boldsymbol{p}^{2} / 2 m-\omega=E_{0}^{N}-E_{n}^{N-1}.
\]
We  neglect here the recoil of the residual $N-1$ system, i.e. we assume the mass of the $N$ and $N-1$ system to be much
heavier than the mass $m$ of the ejected particle.

The PWIA knock-out cross section can be rewritten as

\[
\begin{aligned}
d \sigma & \sim \sum_{n} \delta\left(E_{\text {miss }}-E_{0}^{N}+E_{n}^{N-1}\right)\left|\left\langle\Psi_{n}^{N-1}\left|a_{\boldsymbol{p}_{\text {miss }}}\right| \Psi_{0}^{N}\right\rangle\right|^{2} \\
& =S^{h}\left(\boldsymbol{p}_{\text {miss }}, E_{\text {miss }}\right) .
\end{aligned}
\]

The PWIA cross section is therefore exactly proportional to the
diagonal part of the hole spectral function. This is of course only
true in the PWIA, but when the deviations of the impulse approximation
and the effects of the final state interaction are under control, it
is possible to obtain precise experimental information on the hole
spectral function of the system under study.
