
\chapter{Many-body perturbation theory}

We assume here that we are only interested in the ground state of the system and 
expand the exact wave function in term of a series of Slater determinants
\[
\vert \Psi_0\rangle = \vert \Phi_0\rangle + \sum_{m=1}^{\infty}C_m\vert \Phi_m\rangle,
\]
where we have assumed that the true ground state is dominated by the 
solution of the unperturbed problem, that is
\[
\hat{H}_0\vert \Phi_0\rangle= W_0\vert \Phi_0\rangle.
\]
The state $\vert \Psi_0\rangle$ is not normalized, rather we have used an intermediate 
normalization $\langle \Phi_0 \vert \Psi_0\rangle=1$ since we have $\langle \Phi_0\vert \Phi_0\rangle=1$. 

The Schroedinger equation is
\[
\hat{H}\vert \Psi_0\rangle = E\vert \Psi_0\rangle,
\]
and multiplying the latter from the left with $\langle \Phi_0\vert $ gives
\[
\langle \Phi_0\vert \hat{H}\vert \Psi_0\rangle = E\langle \Phi_0\vert \Psi_0\rangle=E,
\]
and subtracting from this equation
\[
\langle \Psi_0\vert \hat{H}_0\vert \Phi_0\rangle= W_0\langle \Psi_0\vert \Phi_0\rangle=W_0,
\]
and using the fact that the both operators $\hat{H}$ and $\hat{H}_0$ are hermitian 
results in
\[
\Delta E=E-W_0=\langle \Phi_0\vert \hat{H}_I\vert \Psi_0\rangle,
\]
which is an exact result. We call this quantity the correlation energy.

This equation forms the starting point for all perturbative derivations. However,
as it stands it represents nothing but a mere formal rewriting of Schroedinger's equation and is not of much practical use. The exact wave function $\vert \Psi_0\rangle$ is unknown. In order to obtain a perturbative expansion, we need to expand the exact wave function in terms of the interaction $\hat{H}_I$. 

Here we have assumed that our model space defined by the operator $\hat{P}$ is one-dimensional, meaning that
\[
\hat{P}= \vert \Phi_0\rangle \langle \Phi_0\vert ,
\]
and
\[
\hat{Q}=\sum_{m=1}^{\infty}\vert \Phi_m\rangle \langle \Phi_m\vert .
\]

We can thus rewrite the exact wave function as
\[
\vert \Psi_0\rangle= (\hat{P}+\hat{Q})\vert \Psi_0\rangle=\vert \Phi_0\rangle+\hat{Q}\vert \Psi_0\rangle.
\]
Going back to the Schr\"odinger equation, we can rewrite it as, adding and a subtracting a term $\omega \vert \Psi_0\rangle$ as
\[
\left(\omega-\hat{H}_0\right)\vert \Psi_0\rangle=\left(\omega-E+\hat{H}_I\right)\vert \Psi_0\rangle,
\]
where $\omega$ is an energy variable to be specified later. 

We assume also that the resolvent of $\left(\omega-\hat{H}_0\right)$ exits, that is
it has an inverse which defined the unperturbed Green's function as
\[
\left(\omega-\hat{H}_0\right)^{-1}=\frac{1}{\left(\omega-\hat{H}_0\right)}.
\]

We can rewrite Schroedinger's equation as
\[
\vert \Psi_0\rangle=\frac{1}{\omega-\hat{H}_0}\left(\omega-E+\hat{H}_I\right)\vert \Psi_0\rangle,
\]
and multiplying from the left with $\hat{Q}$ results in
\[
\hat{Q}\vert \Psi_0\rangle=\frac{\hat{Q}}{\omega-\hat{H}_0}\left(\omega-E+\hat{H}_I\right)\vert \Psi_0\rangle,
\]
which is possible since we have defined the operator $\hat{Q}$ in terms of the eigenfunctions of $\hat{H}$.

These operators commute meaning that
\[
\hat{Q}\frac{1}{\left(\omega-\hat{H}_0\right)}\hat{Q}=\hat{Q}\frac{1}{\left(\omega-\hat{H}_0\right)}=\frac{\hat{Q}}{\left(\omega-\hat{H}_0\right)}.
\]
With these definitions we can in turn define the wave function as 
\[
\vert \Psi_0\rangle=\vert \Phi_0\rangle+\frac{\hat{Q}}{\omega-\hat{H}_0}\left(\omega-E+\hat{H}_I\right)\vert \Psi_0\rangle.
\]
This equation is again nothing but a formal rewrite of Schr\"odinger's equation
and does not represent a practical calculational scheme.  
It is a non-linear equation in two unknown quantities, the energy $E$ and the exact
wave function $\vert \Psi_0\rangle$. We can however start with a guess for $\vert \Psi_0\rangle$ on the right hand side of the last equation.

 The most common choice is to start with the function which is expected to exhibit the largest overlap with the wave function we are searching after, namely $\vert \Phi_0\rangle$. This can again be inserted in the solution for $\vert \Psi_0\rangle$ in an iterative fashion and if we continue along these lines we end up with
\[
\vert \Psi_0\rangle=\sum_{i=0}^{\infty}\left\{\frac{\hat{Q}}{\omega-\hat{H}_0}\left(\omega-E+\hat{H}_I\right)\right\}^i\vert \Phi_0\rangle, 
\]
for the wave function and
\[
\Delta E=\sum_{i=0}^{\infty}\langle \Phi_0\vert \hat{H}_I\left\{\frac{\hat{Q}}{\omega-\hat{H}_0}\left(\omega-E+\hat{H}_I\right)\right\}^i\vert \Phi_0\rangle, 
\]
which is now  a perturbative expansion of the exact energy in terms of the interaction
$\hat{H}_I$ and the unperturbed wave function $\vert \Psi_0\rangle$.

In our equations for $\vert \Psi_0\rangle$ and $\Delta E$ in terms of the unperturbed
solutions $\vert \Phi_i\rangle$  we have still an undetermined parameter $\omega$
and a dependecy on the exact energy $E$. Not much has been gained thus from a practical computational point of view. 

In Brilluoin-Wigner perturbation theory it is customary to set $\omega=E$. This results in the following perturbative expansion for the energy $\Delta E$
\[
\Delta E=\sum_{i=0}^{\infty}\langle \Phi_0\vert \hat{H}_I\left\{\frac{\hat{Q}}{\omega-\hat{H}_0}\left(\omega-E+\hat{H}_I\right)\right\}^i\vert \Phi_0\rangle=
\]
\[
\langle \Phi_0\vert \left(\hat{H}_I+\hat{H}_I\frac{\hat{Q}}{E-\hat{H}_0}\hat{H}_I+
\hat{H}_I\frac{\hat{Q}}{E-\hat{H}_0}\hat{H}_I\frac{\hat{Q}}{E-\hat{H}_0}\hat{H}_I+\dots\right)\vert \Phi_0\rangle. 
\]

\[
\Delta E=\sum_{i=0}^{\infty}\langle \Phi_0\vert \hat{H}_I\left\{\frac{\hat{Q}}{\omega-\hat{H}_0}\left(\omega-E+\hat{H}_I\right)\right\}^i\vert \Phi_0\rangle=\]
\[
\langle \Phi_0\vert \left(\hat{H}_I+\hat{H}_I\frac{\hat{Q}}{E-\hat{H}_0}\hat{H}_I+
\hat{H}_I\frac{\hat{Q}}{E-\hat{H}_0}\hat{H}_I\frac{\hat{Q}}{E-\hat{H}_0}\hat{H}_I+\dots\right)\vert \Phi_0\rangle. 
\]
This expression depends however on the exact energy $E$ and is again not very convenient from a practical point of view. It can obviously be solved iteratively, by starting with a guess for  $E$ and then solve till some kind of self-consistency criterion has been reached. 

Actually, the above expression is nothing but a rewrite again of the full Schr\"odinger equation. 

Defining $e=E-\hat{H}_0$ and recalling that $\hat{H}_0$ commutes with 
$\hat{Q}$ by construction and that $\hat{Q}$ is an idempotent operator
$\hat{Q}^2=\hat{Q}$. 
Using this equation in the above expansion for $\Delta E$ we can write the denominator 
\[
\hat{Q}\frac{1}{\hat{e}-\hat{Q}\hat{H}_I\hat{Q}}=
\]
\[
\hat{Q}\left[\frac{1}{\hat{e}}+\frac{1}{\hat{e}}\hat{Q}\hat{H}_I\hat{Q}
\frac{1}{\hat{e}}+\frac{1}{\hat{e}}\hat{Q}\hat{H}_I\hat{Q}
\frac{1}{\hat{e}}\hat{Q}\hat{H}_I\hat{Q}\frac{1}{\hat{e}}+\dots\right]\hat{Q}.
\]

Inserted in the expression for $\Delta E$ leads to 
\[
\Delta E=
\langle \Phi_0\vert \hat{H}_I+\hat{H}_I\hat{Q}\frac{1}{E-\hat{H}_0-\hat{Q}\hat{H}_I\hat{Q}}\hat{Q}\hat{H}_I\vert \Phi_0\rangle. 
\]
In RS perturbation theory we set $\omega = W_0$ and obtain the following expression for the energy difference
\[
\Delta E=\sum_{i=0}^{\infty}\langle \Phi_0\vert \hat{H}_I\left\{\frac{\hat{Q}}{W_0-\hat{H}_0}\left(\hat{H}_I-\Delta E\right)\right\}^i\vert \Phi_0\rangle=
\]
\[
\langle \Phi_0\vert \left(\hat{H}_I+\hat{H}_I\frac{\hat{Q}}{W_0-\hat{H}_0}(\hat{H}_I-\Delta E)+
\hat{H}_I\frac{\hat{Q}}{W_0-\hat{H}_0}(\hat{H}_I-\Delta E)\frac{\hat{Q}}{W_0-\hat{H}_0}(\hat{H}_I-\Delta E)+\dots\right)\vert \Phi_0\rangle.
\]

Recalling that $\hat{Q}$ commutes with $\hat{H_0}$ and since $\Delta E$ is a constant we obtain that
\[
\hat{Q}\Delta E\vert \Phi_0\rangle = \hat{Q}\Delta E\vert \hat{Q}\Phi_0\rangle = 0.
\]
Inserting this results in the expression for the energy results in
\[
\Delta E=\langle \Phi_0\vert \left(\hat{H}_I+\hat{H}_I\frac{\hat{Q}}{W_0-\hat{H}_0}\hat{H}_I+
\hat{H}_I\frac{\hat{Q}}{W_0-\hat{H}_0}(\hat{H}_I-\Delta E)\frac{\hat{Q}}{W_0-\hat{H}_0}\hat{H}_I+\dots\right)\vert \Phi_0\rangle.
\]

We can now this expression in terms of a perturbative expression in terms
of $\hat{H}_I$ where we iterate the last expression in terms of $\Delta E$
\[
\Delta E=\sum_{i=1}^{\infty}\Delta E^{(i)}.
\]
We get the following expression for $\Delta E^{(i)}$
\[
\Delta E^{(1)}=\langle \Phi_0\vert \hat{H}_I\vert \Phi_0\rangle,
\] 
which is just the contribution to first order in perturbation theory,
\[
\Delta E^{(2)}=\langle\Phi_0\vert \hat{H}_I\frac{\hat{Q}}{W_0-\hat{H}_0}\hat{H}_I\vert \Phi_0\rangle, 
\]
which is the contribution to second order.

\[
\Delta E^{(3)}=\langle \Phi_0\vert \hat{H}_I\frac{\hat{Q}}{W_0-\hat{H}_0}\hat{H}_I\frac{\hat{Q}}{W_0-\hat{H}_0}\hat{H}_I\Phi_0\rangle-
\langle\Phi_0\vert \hat{H}_I\frac{\hat{Q}}{W_0-\hat{H}_0}\langle \Phi_0\vert \hat{H}_I\vert \Phi_0\rangle\frac{\hat{Q}}{W_0-\hat{H}_0}\hat{H}_I\vert \Phi_0\rangle,
\]
being the third-order contribution. 

\subsection*{Interpreting the correlation energy and the wave operator}

In the shell-model lectures we showed that we could rewrite the exact state function for say the ground state, as a linear expansion in terms of all possible Slater determinants. That is, we 
define the ansatz for the ground state as 
\[
|\Phi_0\rangle = \left(\prod_{i\le F}\hat{a}_{i}^{\dagger}\right)|0\rangle,
\]
where the index $i$ defines different single-particle states up to the Fermi level. We have assumed that we have $N$ fermions. 
A given one-particle-one-hole ($1p1h$) state can be written as
\[
|\Phi_i^a\rangle = \hat{a}_{a}^{\dagger}\hat{a}_i|\Phi_0\rangle,
\]
while a $2p2h$ state can be written as
\[
|\Phi_{ij}^{ab}\rangle = \hat{a}_{a}^{\dagger}\hat{a}_{b}^{\dagger}\hat{a}_j\hat{a}_i|\Phi_0\rangle,
\]
and a general $ApAh$ state as 
\[
|\Phi_{ijk\dots}^{abc\dots}\rangle = \hat{a}_{a}^{\dagger}\hat{a}_{b}^{\dagger}\hat{a}_{c}^{\dagger}\dots\hat{a}_k\hat{a}_j\hat{a}_i|\Phi_0\rangle.
\]

We use letters $ijkl\dots$ for states below the Fermi level and $abcd\dots$ for states above the Fermi level. A general single-particle state is given by letters $pqrs\dots$.

We can then expand our exact state function for the ground state 
as
\[
|\Psi_0\rangle=C_0|\Phi_0\rangle+\sum_{ai}C_i^a|\Phi_i^a\rangle+\sum_{abij}C_{ij}^{ab}|\Phi_{ij}^{ab}\rangle+\dots
=(C_0+\hat{C})|\Phi_0\rangle,
\]
where we have introduced the so-called correlation operator 
\[
\hat{C}=\sum_{ai}C_i^a\hat{a}_{a}^{\dagger}\hat{a}_i  +\sum_{abij}C_{ij}^{ab}\hat{a}_{a}^{\dagger}\hat{a}_{b}^{\dagger}\hat{a}_j\hat{a}_i+\dots
\]
Since the normalization of $\Psi_0$ is at our disposal and since $C_0$ is by hypothesis non-zero, we may arbitrarily set $C_0=1$ with 
corresponding proportional changes in all other coefficients. Using this so-called intermediate normalization we have
\[
\langle \Psi_0 | \Phi_0 \rangle = \langle \Phi_0 | \Phi_0 \rangle = 1, 
\]
resulting in 
\[
|\Psi_0\rangle=(1+\hat{C})|\Phi_0\rangle.
\]

In a shell-model calculation, the unknown coefficients in $\hat{C}$ are the 
eigenvectors which result from the diagonalization of the Hamiltonian matrix.

How can we use perturbation theory to determine the same coefficients? Let us study the contributions to second order in the interaction, namely
\[
\Delta E^{(2)}=\langle\Phi_0\vert \hat{H}_I\frac{\hat{Q}}{W_0-\hat{H}_0}\hat{H}_I\vert \Phi_0\rangle.
\]

The intermediate states given by $\hat{Q}$ can at most be of a $2p-2h$ nature if we have a two-body Hamiltonian. This means that second order in the perturbation theory can have $1p-1h$ and $2p-2h$ at most as intermediate states. When we diagonalize, these contributions are included to infinite order. This means that higher-orders in perturbation theory bring in more complicated correlations. 

If we limit the attention to a Hartree-Fock basis, then we have that
$\langle\Phi_0\vert \hat{H}_I \vert 2p-2h\rangle$ is the only contribution and the contribution to the energy reduces to
\[
\Delta E^{(2)}=\frac{1}{4}\sum_{abij}\langle ij\vert \hat{v}\vert ab\rangle \frac{\langle ab\vert \hat{v}\vert ij\rangle}{\epsilon_i+\epsilon_j-\epsilon_a-\epsilon_b}.
\]

If we compare this to the correlation energy obtained from full configuration interaction theory with a Hartree-Fock basis, we found that
\[
E-E_0 =\Delta E=
\sum_{abij}\langle ij | \hat{v}| ab \rangle C_{ij}^{ab},
\]
where the energy $E_0$ is the reference energy and $\Delta E$ defines the so-called correlation energy.

We see that if we set
\[
C_{ij}^{ab} =\frac{1}{4}\frac{\langle ab \vert \hat{v} \vert ij \rangle}{\epsilon_i+\epsilon_j-\epsilon_a-\epsilon_b},
\]
we have a perfect agreement between FCI and MBPT. However, FCI includes such $2p-2h$ correlations to infinite order. In order to make a meaningful comparison we would at least need to sum such correlations to infinite order in perturbation theory. 

Summing up, we can see that
\begin{itemize}
\item MBPT introduces order-by-order specific correlations and we make comparisons with exact calculations like FCI

\item At every order, we can calculate all contributions since they are well-known and either tabulated or calculated on the fly.

\item MBPT is a non-variational theory and there is no guarantee that higher orders will improve the convergence. 

\item However, since FCI calculations are limited by the size of the Hamiltonian matrices to diagonalize (today's most efficient codes can attach dimensionalities of ten billion basis states, MBPT can function as an approximative method which gives a straightforward (but tedious) calculation recipe. 

\item MBPT has been widely used to compute effective interactions for the nuclear shell-model.

\item But there are better methods which sum to infinite order important correlations. Coupled cluster theory is one of these methods. 
\end{itemize}


\section{Time-dependent perturbation theory}


The time-dependent Schr\"odinger equation (or equation of motion) reads
\[
\imath \hbar\frac{\partial }{\partial t}|\Psi_S(t)\rangle = \hat{H}\Psi_S(t)\rangle,
\]
where the subscript $S$ stands for Schr\"odinger here.
A formal solution is given by 
\[
|\Psi_S(t)\rangle = \exp{(-\imath\hat{H}(t-t_0)/\hbar)}|\Psi_S(t_0)\rangle.
\]
The Hamiltonian $\hat{H}$ is hermitian and the exponent represents a unitary 
operator with an operation carried ut on the wave function at a time $t_0$.

Our Hamiltonian is normally written out as the sum of an unperturbed part $\hat{H}_0$ and an interaction part $\hat{H}_I$, that is
\[
\hat{H}=\hat{H}_0+\hat{H}_I.
\]
In general we have $[\hat{H}_0,\hat{H}_I]\ne 0$ since $[\hat{T},\hat{V}]\ne 0$.
We wish now to define a unitary transformation in terms of $\hat{H}_0$ by defining
\[
|\Psi_I(t)\rangle = \exp{(\imath\hat{H}_0t/\hbar)}|\Psi_S(t)\rangle,
\]
which is again a unitary transformation carried out now at the time $t$ on the 
wave function in the Schr\"odinger picture. 

We can easily find the equation of motion by taking the time derivative
\[
\imath \hbar\frac{\partial }{\partial t}|\Psi_I(t)\rangle = -\hat{H}_0\exp{(\imath\hat{H}_0t/\hbar)}\Psi_S(t)\rangle+\exp{(\imath\hat{H}_0t/\hbar)}
\imath \hbar\frac{\partial }{\partial t}\Psi_S(t)\rangle.
\]

Using the definition of the Schr\"odinger equation, we can rewrite the last equation as 
\[
\imath \hbar\frac{\partial }{\partial t}|\Psi_I(t)\rangle = \exp{(\imath\hat{H}_0t/\hbar)}\left[-\hat{H}_0+\hat{H}_0+\hat{H}_I\right]\exp{(-\imath\hat{H}_0t/\hbar)}\Psi_I(t)\rangle,
\]
which gives us
\[
\imath \hbar\frac{\partial }{\partial t}|\Psi_I(t)\rangle = \hat{H}_I(t)\Psi_I(t)\rangle,
\]
 with 
\[
\hat{H}_I(t)=
\exp{(\imath\hat{H}_0t/\hbar)}\hat{H}_I\exp{(-\imath\hat{H}_0t/\hbar)}.
\]

The order of the operators is important since $\hat{H}_0$ and $\hat{H}_I$ do generally not commute.
The expectation value of
an arbitrary operator in the interaction picture can now be written as
\[
\langle \Psi'_S(t)|\hat{O}_S|\Psi_S(t)\rangle = 
\langle \Psi'_I(t) |\exp{(\imath\hat{H}_0t/\hbar)}\hat{O}_I
\exp{(-\imath\hat{H}_0t/\hbar)}|\Psi_I(t)\rangle,
\]
and using the definition
\[
\hat{O}_I(t)=
\exp{(\imath\hat{H}_0t/\hbar)}\hat{O}_I\exp{(-\imath\hat{H}_0t/\hbar)},
\]
we obtain
\[
\langle \Psi'_S(t)|\hat{O}_S|\Psi_S(t)\rangle = 
\langle \Psi'_I(t) |\hat{O}_I(t)|\Psi_I(t)\rangle,
\]
stating that a unitary transformation does not change expectation values!

If the take the time derivative of the operator in the interaction picture we arrive at the following equation of motion
\[
\imath \hbar\frac{\partial }{\partial t}\hat{O}_I(t) = \exp{(\imath\hat{H}_0t/\hbar)}\left[\hat{O}_S\hat{H}_0-\hat{H}_0\hat{O}_S\right]\exp{(-\imath\hat{H}_0t/\hbar)}=\left[\hat{O}_I(t),\hat{H}_0\right].
\]
Here we have used the time-independence of the Schr\"odinger equation
together with the observation that any function of an operator commutes with the operator itself. 

In order to solve the equation of motion equation in the interaction picture, we define a unitary operator
time-development operator $\hat{U}(t,t')$. Later we will derive its
connection with the linked-diagram theorem, which yields a
linked expression for the actual operator. 
The action of the operator on the wave function is
\[
|\Psi_I(t) \rangle = \hat{U}(t,t_0)|\Psi_I(t_0)\rangle,
\]
with the obvious value $\hat{U}(t_0,t_0)=1$.

The time-development operator $U$ has the
properties that
\[
     \hat{U}^{\dagger}(t,t')\hat{U}(t,t')=\hat{U}(t,t')\hat{U}^{\dagger}(t,t')=1,
\]
which implies that $U$ is unitary
\[
     \hat{U}^{\dagger}(t,t')=\hat{U}^{-1}(t,t').
\]
Further,
\[
    \hat{U}(t,t')\hat{U}(t't'')=\hat{U}(t,t'')
\]
and
\[
    \hat{U}(t,t')\hat{U}(t',t)=1,
\]
which leads to
\[
    \hat{U}(t,t')=\hat{U}^{\dagger}(t',t).
\]

Using our definition of Schr\"odinger's equation in the interaction picture, we can then construct the operator $\hat{U}$. We have defined
\[
|\Psi_I(t)\rangle = \exp{(\imath\hat{H}_0t/\hbar)}|\Psi_S(t)\rangle,
\]
which can be rewritten as 
\[
|\Psi_I(t)\rangle = \exp{(\imath\hat{H}_0t/\hbar)}\exp{(-\imath\hat{H}(t-t_0)/\hbar)}|\Psi_S(t_0)\rangle,
\]
or
\[
|\Psi_I(t)\rangle = \exp{(\imath\hat{H}_0t/\hbar)}\exp{(-\imath\hat{H}(t-t_0)/\hbar)}\exp{(-\imath\hat{H}_0t_0/\hbar)}|\Psi_I(t_0)\rangle.
\]

From the last expression we can define
\[
\hat{U}(t,t_0)=\exp{(\imath\hat{H}_0t/\hbar)}\exp{(-\imath\hat{H}(t-t_0)/\hbar)}\exp{(-\imath\hat{H}_0t_0/\hbar)}.
\]
It is then easy to convince oneself that the properties defined above are satisfied by the definition of $\hat{U}$. 

We derive the equation of motion for $\hat{U}$ using the above definition.
This results in
\[
\imath \hbar\frac{\partial }{\partial t}\hat{U}(t,t_0) = \hat{H}_I(t)\hat{U}(t,t_0),
\]
which we integrate from $t_0$ to a time $t$ resulting in
\[
\hat{U}(t,t_0)-\hat{U}(t_0,t_0)=\hat{U}(t,t_0)-1=-\frac{\imath}{\hbar}\int_{t_0}^t dt' \hat{H}_I(t')\hat{U}(t',t_0),
\]
which can be rewritten as
\[
\hat{U}(t,t_0)=1-\frac{\imath}{\hbar}\int_{t_0}^t dt' \hat{H}_I(t')\hat{U}(t',t_0).
\]

We can solve this equation iteratively keeping in mind the time-ordering of the of the operators
\[
\hat{U}(t,t_0)=1-\frac{\imath}{\hbar}\int_{t_0}^t dt' \hat{H}_I(t')+\left(\frac{-\imath}{\hbar}\right)^2\int_{t_0}^t dt'\int_{t_0}^{t'} dt'' \hat{H}_I(t')\hat{H}_I(t'')+\dots
\]
The third term can be written as 
\[
\int_{t_0}^t dt'\int_{t_0}^{t'} dt'' \hat{H}_I(t')\hat{H}_I(t'')=
\frac{1}{2}\int_{t_0}^t dt'\int_{t_0}^{t'} dt'' \hat{H}_I(t')\hat{H}_I(t'')
+\frac{1}{2}\int_{t_0}^t dt''\int_{t''}^{t} dt' \hat{H}_I(t')\hat{H}_I(t'').
\]

We obtain this expression by changing the integration order in the second term
via a change of the integration variables $t'$ and $t''$  in 
\[
\frac{1}{2}\int_{t_0}^t dt'\int_{t_0}^{t'} dt'' \hat{H}_I(t')\hat{H}_I(t'').
\]
We can rewrite the terms which contain the double integral as
\[
\int_{t_0}^t dt'\int_{t_0}^{t'} dt'' \hat{H}_I(t')\hat{H}_I(t'')=\]
\[
\frac{1}{2}\int_{t_0}^t dt'\int_{t_0}^{t'} dt''\left[\hat{H}_I(t')\hat{H}_I(t'')\Theta(t'-t'')
+\hat{H}_I(t')\hat{H}_I(t'')\Theta(t''-t')\right],
\]
with $\Theta(t''-t')$ being the standard Heavyside or step function. The step function allows us to give a specific time-ordering to the above expression.

With the $\Theta$-function we can rewrite the last expression as 
\[
\int_{t_0}^t dt'\int_{t_0}^{t'} dt'' \hat{H}_I(t')\hat{H}_I(t'')=
\frac{1}{2}\int_{t_0}^t dt'\int_{t_0}^{t'} dt''\hat{T}\left[\hat{H}_I(t')\hat{H}_I(t'')\right],
\]
where $\Hat{T}$ is the so-called time-ordering operator. 

With this definition, we can rewrite the expression for $\hat{U}$ as 
\[
\hat{U}(t,t_0)=\sum_{n=0}^{\infty}\left(\frac{-\imath}{\hbar}\right)^n\frac{1}{n!}
\int_{t_0}^t dt_1\dots \int_{t_0}^t dt_N \hat{T}\left[\hat{H}_I(t_1)\dots\hat{H}_I(t_n)\right]=\hat{T}\exp{\left[\frac{-\imath}{\hbar}
\int_{t_0}^t dt' \hat{H}_I(t')\right]}.
\]
The above time-evolution operator in the interaction picture will be used
to derive various contributions to many-body perturbation theory. See also exercise 26 for a discussion of the various time orderings.

We wish now to define a unitary transformation in terms of $\hat{H}$ by defining
\[
|\Psi_H(t)\rangle = \exp{(\imath\hat{H}t/\hbar)}|\Psi_S(t)\rangle,
\]
which is again a unitary transformation carried out now at the time $t$ on the 
wave function in the Schr\"odinger picture. If we combine this equation with 
Schr\"odinger's equation we obtain the following equation of motion
\[
\imath \hbar\frac{\partial }{\partial t}|\Psi_H(t)\rangle = 0,
\]
meaning that $|\Psi_H(t)\rangle$ is time independent. An operator in this picture is defined as
\[
\hat{O}_H(t)=
\exp{(\imath\hat{H}t/\hbar)}\hat{O}_S\exp{(-\imath\hat{H}t/\hbar)}.
\]

The time dependence is then in the operator itself, and this yields in turn the
following equation of motion
\[
\imath \hbar\frac{\partial }{\partial t}\hat{O}_H(t) = \exp{(\imath\hat{H}t/\hbar)}\left[\hat{O}_H\hat{H}-\hat{H}\hat{O}_H\right]\exp{(-\imath\hat{H}t/\hbar)}=\left[\hat{O}_H(t),\hat{H}\right].
\]
We note that an operator in the Heisenberg picture can be related to the corresponding
operator in the interaction picture as 
\[
\hat{O}_H(t)=
\exp{(\imath\hat{H}t/\hbar)}\hat{O}_S\exp{(-\imath\hat{H}t/\hbar)}=\]
\[
\exp{(\imath\hat{H}_It/\hbar)}\exp{(-\imath\hat{H}_0t/\hbar)}\hat{O}_I\exp{(\imath\hat{H}_0t/\hbar)}\exp{(-\imath\hat{H}_It/\hbar)}.
\]

With our definition of the time evolution operator we see that
\[
\hat{O}_H(t)=\hat{U}(0,t)\hat{O}_I\hat{U}(t,0),
\]
which in turn implies that $\hat{O}_S=\hat{O}_I(0)=\hat{O}_H(0)$, all operators are equal at $t=0$. The wave function in the Heisenberg formalism is 
related to the other pictures as 
\[
|\Psi_H\rangle=|\Psi_S(0)\rangle=|\Psi_I(0)\rangle,
\]
since the wave function in the Heisenberg picture is time independent. 
We can relate this wave function to that a given time $t$ via the time evolution operator as
\[
|\Psi_H\rangle=\hat{U}(0,t)|\Psi_I(t)\rangle.
\]

We assume that the interaction term is switched on gradually. Our wave function at time $t=-\infty$ and $t=\infty$ is supposed to represent a non-interacting system
given by the solution to the unperturbed part of our Hamiltonian $\hat{H}_0$.
We assume the ground state is given by $|\Phi_0\rangle$, which could be a Slater determinant.
We define our Hamiltonian as
\[
\hat{H}=\hat{H}_0+\exp{(-\varepsilon t/\hbar)}\hat{H}_I,
\]
where $\varepsilon$ is a small number. The way we write the Hamiltonian 
and its interaction term is meant to simulate the switching of the interaction.

\subsection{Adiabatic hypothesis}
The time evolution of the wave function in the interaction picture is then
\[
|\Psi_I(t) \rangle = \hat{U}_{\varepsilon}(t,t_0)|\Psi_I(t_0)\rangle,
\]
with 
\[
\hat{U}_{\varepsilon}(t,t_0)=\sum_{n=0}^{\infty}\left(\frac{-\imath}{\hbar}\right)^n\frac{1}{n!}
\int_{t_0}^t dt_1\dots \int_{t_0}^t dt_N \exp{(-\varepsilon(t_1+\dots+t_n)/\hbar)}\hat{T}\left[\hat{H}_I(t_1)\dots\hat{H}_I(t_n)\right]
\]

In the limit $t_0\rightarrow -\infty$, the solution ot Schr\"odinger's equation is
$|\Phi_0\rangle$, and the eigenenergies are given by 
\[
\hat{H}_0|\Phi_0\rangle=W_0|\Phi_0\rangle,
\]
meaning that 
\[
|\Psi_S(t_0)\rangle = \exp{(-\imath W_0t_0/\hbar)}|\Phi_0\rangle,
\]
with the corresponding interaction picture wave function given by
\[
|\Psi_I(t_0)\rangle = \exp{(\imath \hat{H}_0t_0/\hbar)}|\Psi_S(t_0)\rangle=|\Phi_0\rangle.
\]

The solution becomes time independent in the limit $t_0\rightarrow -\infty$.
The same conclusion can be reached by looking at 
\[
\imath \hbar\frac{\partial }{\partial t}|\Psi_I(t)\rangle =
\exp{(\varepsilon |t|/\hbar)}\hat{H}_I|\Psi_I(t)\rangle 
\]
and taking the limit $t\rightarrow -\infty$.
We can rewrite the equation for the wave function at a time $t=0$ as
\[
|\Psi_I(0) \rangle = \hat{U}_{\varepsilon}(0,-\infty)|\Phi_0\rangle.
\]


\subsection{Goldstone's theorem and Gell-Mann and Low theorem on the ground state}

Our wave function for ground state (after Gell-Mann and Low, see Phys.~Rev.~{\bf 84}, 350 (1951)) is then
\[
        \frac{\ket{\Psi_0}}{\left\langle\Phi_0 | \Psi_0 \right\rangle}=
    \lim_{\epsilon \rightarrow 0}
   \lim_{t'\rightarrow -\infty}
   \frac{U(0,-\infty )\ket{\Phi_0} }
   { \bra{\Phi_0} U(0,-\infty )\ket{\Phi_0} },
\]
and we ask whether this quantity exists to all orders in perturbation theory.
Goldstone's theorem states that only linked diagrams enter the expression for the final binding energy. It means that energy difference reads now
\[
\Delta E=\sum_{i=0}^{\infty}\langle \Phi_0|\hat{H}_I\left\{\frac{\hat{Q}}{W_0-\hat{H}_0}\hat{H}_I\right\}^i|\Phi_0\rangle_L,
\]
where the subscript $L$ indicates that only linked diagrams are included. In our Rayleigh-Schr\"odinger expansion, the energy difference included also unlinked diagrams. 

If it does, Gell-Mann and Low showed that it is an eigenstate of $\hat{H}$ with eigenvalue
\[
 \hat{H}\frac{\ket{\Psi_0}}{\left\langle\Phi_0 | \Psi_0 \right\rangle}= E\frac{\ket{\Psi_0}}{\left\langle\Phi_0 | \Psi_0 \right\rangle}
\]
and multiplying from the left with $\langle \Phi_0|$ we can rewrite the last equation
as
\[
E-W_0=\frac{\langle \Phi_0|\hat{H}_I\ket{\Psi_0}}{\left\langle\Phi_0 | \Psi_0 \right\rangle},
\]
since $\hat{H}_0|\Phi_0\rangle = W_0|\Phi_0\rangle$. The numerator and the denominators of the last equation do not exist separately. The theorem of Gell-Mann and Low asserts that this ratio exists. 

We wish to link the above expression with the corresponding expression from time-dependent perturbation theory. We write our expression as
\[
E-W_0=\Delta E= \lim_{\epsilon \rightarrow 0^+}
   \frac{\bra{\Phi_0(0)}\hat{H}_IU_{\epsilon }(0,-\infty )\ket{\Phi_0(-\infty)} }
   { \bra{\Phi_0(0)} U_{\epsilon}(0,-\infty )\ket{\Phi_0(-\infty)} },
\]
with a numerator 
\[
N=\lim_{\epsilon \rightarrow 0^+}\bra{\Phi_0(0)}\hat{H}_IU_{\epsilon}(0,-\infty )\ket{\Phi_0(-\infty)}, 
\]
which we rewrite as
\[
 N=\lim_{\epsilon \rightarrow 0^+}\bra{\Phi_0(0)}\hat{H}_I(t=0)\displaystyle\sum_{n=0}^{\infty}\frac{(-i)^n}{n!}
   \int_{-\infty}^{0}dt_1  \int_{-\infty}^{0}dt_2\dots  \int_{-\infty}^{0}dt_n}\exp{(\epsilon/\hbar(t_1+\dots+t_n))} \hat{T}\left[H_1(t_1)H_1(t_2)\dots H_1(t_n)\right] \ket{\Phi_0(-\infty)}. 
\]

A linked diagram (or connected diagram) is a diagram which is linked to the last interaction vertex at $t=0$.
We divide the diagrams into linked and unlinked. 
In general, the way we can distribute $\mu$ unlinked diagrams among the total of $n$ diagrams is given by the combinatorial factor
\[
\left(\begin{array}{c} n \\ \mu \end{array}\right) = \frac{n!}{\mu!\nu!},
\]
and using the following relation
\[
\sum_{n=0}^{\infty}\frac{1}{n!}\sum_{\mu+\nu=n}^{\infty}\frac{n!}{\mu!\nu!}=\sum_{\mu=0}^{\infty}\frac{1}{\mu!}\sum_{\nu}^{\infty}\frac{1}{\nu!},
\]
we can rewrite the numerator $N$ as 
\[
N=\bra{\Phi_0(0)}\hat{H}_IU_{\epsilon}(0,-\infty )\ket{\Phi_0(-\infty)}_L\bra{\Phi_0(0)}U_{\epsilon}(0,-\infty )\ket{\Phi_0(-\infty)},  
=N_LD,
\]
with $N_L$ now containing only linked terms
\[
 N_L=\lim_{\epsilon \rightarrow 0^+}\bra{\Phi_0(0)}\hat{H}_I(t=0)\displaystyle\sum_{\nu=0}^{\infty}\frac{(-i)^{\nu}}{\nu!}
   \int_{-\infty}^{0}dt_1  \int_{-\infty}^{0}dt_2\dots  \int_{-\infty}^{0}dt_n}\exp{(\epsilon/\hbar(t_1+\dots+t_n))} \hat{T}\left[H_1(t_1)H_1(t_2)\dots H_1(t_n)\right] \ket{\Phi_0(-\infty)}_L, 
\]
with the subscript $L$ indicating that only linked diagrams appear, that is those diagrams which are linked to the last interaction vertex.

A linked diagram (or connected diagram) is a diagram which is linked to the last interaction vertex at $t=0$.
We divide the diagrams into linked and unlinked. 
In general, the way we can distribute $\mu$ unlinked diagrams among the total of $n$ diagrams is given by the combinatorial factor
\[
\left(\begin{array}{c} n \\ \mu \end{array}\right) = \frac{n!}{\mu!\nu!},
\]
and using the following relation
\[
\sum_{n=0}^{\infty}\frac{1}{n!}\sum_{\mu+\nu=n}^{\infty}\frac{n!}{\mu!\nu!}=\sum_{\mu=0}^{\infty}\frac{1}{\mu!}\sum_{\nu}^{\infty}\frac{1}{\nu!},
\]
we can rewrite the numerator $N$ as 
\[
N=\bra{\Phi_0(0)}\hat{H}_IU_{\epsilon}(0,-\infty )\ket{\Phi_0(-\infty)}_L\bra{\Phi_0(0)}U_{\epsilon}(0,-\infty )\ket{\Phi_0(-\infty)},  
=N_LD,
\]
with $N_L$ now containing only linked terms
\[
 N_L=\lim_{\epsilon \rightarrow 0^+}\bra{\Phi_0(0)}\hat{H}_I(t=0)\displaystyle\sum_{\nu=0}^{\infty}\frac{(-i)^{\nu}}{\nu!}
   \int_{-\infty}^{0}dt_1  \int_{-\infty}^{0}dt_2\dots  \int_{-\infty}^{0}dt_n}\exp{(\epsilon/\hbar(t_1+\dots+t_n))} \hat{T}\left[H_1(t_1)H_1(t_2)\dots H_1(t_n)\right] \ket{\Phi_0(-\infty)}_L, 
\]
with the subscript $L$ indicating that only linked diagrams appear, that is those diagrams which are linked to the last interaction vertex.

We note that also that the term $D$ is nothing but the denominator of the equation for the energy. We obtain then the following expression for the energy
\[
E-W_0=\Delta E=N_L= \bra{\Phi_0(0)}\hat{H}_IU_{\epsilon}(0,-\infty )\ket{\Phi_0(-\infty)}_L,
\]
and Goldstone's theorem is then proved. 
The corresponding expression from Rayleigh-Schr\"odinger perturbation theory is given by
\[
\Delta E=\langle \Phi_0|\left(\hat{H}_I+\hat{H}_I\frac{\hat{Q}}{W_0-\hat{H}_0}\hat{H}_I+
\hat{H}_I\frac{\hat{Q}}{W_0-\hat{H}_0}\hat{H}_I\frac{\hat{Q}}{W_0-\hat{H}_0}\hat{H}_I+\dots\right)|\Phi_0\rangle_C.
\]

An important point in the derivation of the Gell-Mann and Low theorem
\[
E-W_0=\frac{\langle \Phi_0|\hat{H}_I\ket{\Psi_0}}{\left\langle\Phi_0 | \Psi_0 \right\rangle},
\]
is that the numerator and the denominators of the last equation do not exist separately. The theorem of Gell-Mann and Low asserts that this ratio exists. To prove it we proceed as follows. Consider the expression
\[
(\hat{H}_0-E)U_{\epsilon }(0,-\infty )\ket{\Phi_0}=\left[\hat{H}_0,U_{\epsilon }(0,-\infty )\right]\ket{\Phi_0}.
\]

To evaluate the commutator 
\[
(\hat{H}_0-E)U_{\epsilon }(0,-\infty )\ket{\Phi_0}=\left[\hat{H}_0,U_{\epsilon }(0,-\infty )\right]\ket{\Phi_0}.
\]
we write the associate commutator as
\[
\left[\hat{H}_0,\hat{H}_I(t_1)\hat{H}_I(t_2)\dots \hat{H}_I(t_n)\right]=
\left[\hat{H}_0,\hat{H}_I(t_1)\right]\hat{H}_I(t_2)\dots \hat{H}_I(t_n)+
\]
\[
\dots+\hat{H}_I(t_1)\left[\hat{H}_0,\hat{H}_I(t_2)\right]\hat{H}_I(t_3)\dots \hat{H}_I(t_n)+\dots
\]
Using the equation of motion for an operator in the interaction picture we have
\[
\imath \hbar\frac{\partial }{\partial t}\hat{H}_I(t) = \left[\hat{H}_I(t),\hat{H}_0\right].
\]
Each of the above commutators yield then a time derivative!

We have then
\[
\left[\hat{H}_0,\hat{H}_I(t_1)\hat{H}_I(t_2)\dots \hat{H}_I(t_n)\right]=\imath \hbar\left(\frac{\partial }{\partial t_n}+\frac{\partial }{\partial t_1}+\dots+\frac{\partial }{\partial t_n}\right) \hat{H}_I(t_1)\hat{H}_I(t_2)\dots\hat{H}_I(t_n),
\]
meaning that we can rewrite
\[
(\hat{H}_0-E)U_{\epsilon }(0,-\infty )\ket{\Phi_0}=\left[\hat{H}_0,U_{\epsilon }(0,-\infty )\right]\ket{\Phi_0},
\]
as
\[
(\hat{H}_0-E)U_{\epsilon }(0,-\infty )\ket{\Phi_0}=-\sum_{n=1}^{\infty}\left(\frac{-\imath}{\hbar}\right)^{n-1}\frac{1}{n!}
\int_{t_0}^t dt_1\dots \int_{t_0}^t dt_N \exp{(-\varepsilon(t_1+\dots+t_n)/\hbar)}
\]
\[
\times\sum_{i=1}^n(\frac{\partial }{\partial t_i} )\hat{T}\left[\hat{H}_I(t_1)\dots\hat{H}_I(t_n)\right].
\]

All the time derivatives in this equation 
\[
(\hat{H}_0-E)U_{\epsilon }(0,-\infty )\ket{\Phi_0}=-\sum_{n=1}^{\infty}\left(\frac{-\imath}{\hbar}\right)^{n-1}\frac{1}{n!}
\int_{t_0}^t dt_1\dots \int_{t_0}^t dt_N \exp{(-\varepsilon(t_1+\dots+t_n)/\hbar)}
\]
\[
\times\sum_{i=1}^n(\frac{\partial }{\partial t_i} )\hat{T}\left[\hat{H}_I(t_1)\dots\hat{H}_I(t_n)\right],
\]
make the same contribution, as can be seen by changing dummy variables. We can therefore retain just one time derivative $\partial/\partial t$ and multiply with $n$. Integrating by parts wrt $t_1$  we obtain two terms. 

Integrating by parts wrt $t_1$  one can finally show that
\[
        \frac{\ket{\Psi_0}}{\left\langle\Phi_0 | \Psi_0 \right\rangle}=
    \lim_{\epsilon \rightarrow 0}
   \lim_{t'\rightarrow -\infty}
   \frac{U(0,-\infty )\ket{\Phi_0} }
   { \bra{\Phi_0} U(0,-\infty )\ket{\Phi_0} },
\]
For more details about the derivation, see Gell-Mann and Low, Phys.~Rev.~{\bf 84}, 350  (1951). See also chapter 6.2 of Raimes or Fetter and Walecka, chapter 3.6.


In the present discussion of the time-dependent theory we will make
use of the so-called complex-time approach to describe the time
evolution operator $U$.
This means that we
allow the time $t$ to be rotated by a small angle $\epsilon$
relative to the real time axis. The complex time $t$ is then
related to the real time $\tilde{t}$ by
\[
t=\tilde{t}(1-i\epsilon ).
\]
Let us first study the true eigenvector $\Psi_{\alpha}$ which evolves
from the unperturbed eigenvectors $\Phi_{\alpha}$ through the action of the
time development operator
\[
   U_{\varepsilon}(t,t')=\lim_{\epsilon \rightarrow 0}
   \lim_{t'\rightarrow -\infty}
   {\displaystyle\sum_{n=0}^{\infty}\frac{(-i)^n}{n!}
   \int_{t'}^{t}dt_1  \int_{t'}^{t}dt_2\dots  \int_{t'}^{t}dt_n}
\]
\[
	      \times T\left[H_1(t_1)H_1(t_2)\dots H_1(t_n)\right],
\]
where $T$ stands for the correct time-ordering.

In time-dependent
perturbation theory we let $\Psi_{\alpha}$ develop from $\Phi_{\alpha}$ in the
remote past to a given time $t$
\[
    \frac{\ket{\Psi_{\alpha}}}
    {\left\langle\psi_{\alpha} | \Psi_{\alpha} \right\rangle}=
    \lim_{\epsilon \rightarrow 0}
   \lim_{t'\rightarrow -\infty}
   \frac{U_{\varepsilon}(t,t' )\ket{\psi_{\alpha}} }
   { \bra{\psi_{\alpha}} U(t,t' )\ket{\Phi_{\alpha}} },
\]
and similarly, we let
$\Psi_{\beta}$ develop from $\Phi_{\beta}$ in the remote future
\[
    \frac{\bra{\Psi_{\beta}}}{\left\langle
    \psi_{\beta} | \Psi_{\beta} \right\rangle}=
    \lim_{\epsilon \rightarrow 0}
    \lim_{t'\rightarrow \infty}
    \frac{\bra{\psi_{\beta}}U_{\varepsilon}(t' ,t) }
    { \bra{\psi_{\beta}} U_{\varepsilon}(t' ,t)\ket{\Phi_{\beta}} }.
\]

Here we are interested in the expectation value of a given
operator ${\cal O}$ acting at a time $t=0$. This can be achieved
from the two previous equations defining
\[
     \ket{\Psi_{\alpha ,\beta}'}=
     \frac{\ket{\Psi_{\alpha ,\beta}}}
     {\left\langle\Phi_{\alpha ,\beta} | \Psi_{\alpha ,\beta} \right\rangle}
\]
we have
\[
   {\cal O}_{\alpha\beta}
  =\frac{N_{\beta\alpha}}{D_{\beta}D_{\alpha}},
\]
where we have introduced
\[
   N_{\beta\alpha}=
   \bra{\Phi_{\beta}}U_{\varepsilon}(\infty ,0){\cal O}U_{\varepsilon}(0,-\infty )\ket{\Phi_{\alpha}} ,
\]
and 
\[
   D_{\alpha ,\beta}=
   \sqrt{\bra{\psi_{\alpha ,\beta}}
   U_{\varepsilon}(\infty ,0)U_{\varepsilon}(0,-\infty )\ket{\Phi_{\alpha ,\beta}}}. 
\]

If the operator ${\cal O}$ stands for the hamiltonian $H$ we obtain
\[
    {\displaystyle  \frac{\bra{\Psi_{\lambda}'}H\ket{\Psi_{\lambda}'} }
   { \left\langle\Psi_{\lambda}' | \Psi_{\lambda}' \right\rangle} }
\]
At this stage, {\em it is important to observe} that our 
expression for the expectation value of a given operator ${\cal O}$
{\em is hermitian} insofar ${\cal O}^{\dagger}={\cal O}$. This is readily 
demonstrated. The above equation is of the general form
\[
U(t,t_0){\cal O}U(t_0,-t),
\]
and noting that 
\[
   U^{\dagger}(t,t_0)=
   \left({\displaystyle e^{iH_0t}e^{-iH(t-t_0)}e^{-iH_0t}}\right)^{\dagger}
   =U(t_0,-t),
\]
since $H^{\dagger}=H$ and $H_0^{\dagger}=H_0$, we have that
\[
    \left(U(t,t_0){\cal O}U(t_0,-t)\right)^{\dagger}
    =U(t,t_0){\cal O}U(t_0,-t).
\]
The question we pose now is what happens in the limit $\varepsilon\rightarrow 0$?
Do we get results which are meaningful?

Our wave function for ground state is then
\[
        \frac{\ket{\Psi_0}}{\left\langle\Phi_0 | \Psi_0 \right\rangle}=
    \lim_{\epsilon \rightarrow 0}
   \lim_{t'\rightarrow -\infty}
   \frac{U(0,-\infty )\ket{\Phi_0} }
   { \bra{\Phi_0} U(0,-\infty )\ket{\Phi_0} },
\]
meaning that the energy difference is given by
\[
E_0-W_0=\Delta E_0= \lim_{\epsilon \rightarrow 0}
   \lim_{t'\rightarrow -\infty}
   \frac{\bra{\Phi_0}\hat{H}_IU_{\varepsilon}(0,-\infty )\ket{\Phi_0} }
   { \bra{\Phi_0} U_{\varepsilon}(0,-\infty )\ket{\Phi_0} },
\]
and we ask whether this quantity exists to all orders in perturbation theory.

If it does, Gell-Mann and Low showed that it is an eigenstate of $\hat{H}$ with eigenvalue
\[
 \hat{H}\frac{\ket{\Psi_0}}{\left\langle\Phi_0 | \Psi_0 \right\rangle}= E_0\frac{\ket{\Psi_0}}{\left\langle\Phi_0 | \Psi_0 \right\rangle}
\]
and multiplying from the left with $\langle \Phi_0|$ we can rewrite the last equation
as
\[
E_0-W_0=\frac{\langle \Phi_0|\hat{H}_I\ket{\Psi_0}}{\left\langle\Phi_0 | \Psi_0 \right\rangle},
\]
since $\hat{H}_0|\Phi_0\rangle = W_0|\Phi_0\rangle$. The numerator and the denominators of the last equation do not exist separately. The theorem of Gell-Mann and Low asserts that this ratio exists. 

Goldstone's theorem states that only linked diagrams enter the expression for the final binding energy. It means that energy difference reads now
\[
\Delta E_0=\sum_{i=0}^{\infty}\langle \Phi_0|\hat{H}_I\left\{\frac{\hat{Q}}{W_0-\hat{H}_0}\hat{H}_I\right\}^i|\Phi_0\rangle_L,
\]
where the subscript $L$ indicates that only linked diagrams are included. In our Rayleigh-Schr\"odinger expansion, the energy difference included also unlinked diagrams. 

We wish to link the above expression with the corresponding expression from time-dependent perturbation theory. We write our expression as
\[
E_0-W_0=\Delta E_0= \lim_{\epsilon \rightarrow 0^+}
   \frac{\bra{\Phi_0(0)}\hat{H}_IU_{\epsilon }(0,-\infty )\ket{\Phi_0(-\infty)} }
   { \bra{\Phi_0(0)} U_{\epsilon}(0,-\infty )\ket{\Phi_0(-\infty)} },
\]
with a numerator 
\[
N=\lim_{\epsilon \rightarrow 0^+}\bra{\Phi_0(0)}\hat{H}_IU_{\epsilon}(0,-\infty )\ket{\Phi_0(-\infty)}, 
\]
which we rewrite as
\[
 N=\lim_{\epsilon \rightarrow 0^+}\bra{\Phi_0(0)}\hat{H}_I(t=0)\displaystyle\sum_{n=0}^{\infty}\frac{(-i)^n}{n!}
   \int_{-\infty}^{0}dt_1  \int_{-\infty}^{0}dt_2\dots  \int_{-\infty}^{0}dt_n}\exp{(\epsilon/\hbar(t_1+\dots+t_n))} \hat{T}\left[H_1(t_1)H_1(t_2)\dots H_1(t_n)\right] \ket{\Phi_0(-\infty)}. 
\]

From this term we can obtain both linked and unlinked contributions. Goldstone's theorem states that only linked diagrams enter the expression for the final binding energy. 
A linked diagram (or connected diagram) is a diagram which is linked to the last interaction vertex at $t=0$.
We label the number of linked diagrams with the variable $\nu$ and the number of unlinked with $\mu$  with $n=\nu+\mu$.  The number of unlinked diagrams is then $\mu=n-\nu$. 

In general, the way we can distribute $\mu$ unlinked diagrams among the total of $n$ diagrams is given by the combinatorial factor
\[
\left(\begin{array}{c} n \\ \mu \end{array}\right) = \frac{n!}{\mu!\nu!},
\]
and using the following relation
\[
\sum_{n=0}^{\infty}\frac{1}{n!}\sum_{\mu+\nu=n}^{\infty}\frac{n!}{\mu!\nu!}=\sum_{\mu=0}^{\infty}\frac{1}{\mu!}\sum_{\nu}^{\infty}\frac{1}{\nu!},
\]
we can rewrite the numerator $N$ as 
\[
N=\bra{\Phi_0(0)}\hat{H}_IU_{\epsilon}(0,-\infty )\ket{\Phi_0(-\infty)}_L\bra{\Phi_0(0)}U_{\epsilon}(0,-\infty )\ket{\Phi_0(-\infty)}=N_LD.
\]

We define  $N_L$ to contain only linked terms
%\[
% N_L=\lim_{\epsilon \rightarrow 0^+}\bra{\Phi_0(0)}\hat{H}_I(t=0)\sum_{\nu=0}^{\infty}\frac{(-i)^{\nu}}{\nu!}\int_{-\infty}^{0}dt_1  \int_{-\infty}^{0}dt_2\dots  \int_{-\infty}^{0}dt_n}\exp{(\epsilon/\hbar(t_1+\dots+t_n))} \hat{T}\left[H_1(t_1)H_1(t_2)\dots H_1(t_n)\right] \ket{\Phi_0(-\infty)}_L, 
%\]
with the subscript $L$ indicating that only linked diagrams appear, that is those diagrams which are linked to the last interaction vertex.

We note that also that the term $D$ is nothing but the denominator of the equation for the energy. We obtain then the following expression for the energy
\[
E_0-W_0=\Delta E_0=N_L= \bra{\Phi_0(0)}\hat{H}_IU_{\epsilon}(0,-\infty )\ket{\Phi_0(-\infty)}_L,
\]
and Goldstone's theorem is then proved. 
The corresponding expression from Rayleigh-Schr\"odinger perturbation theory is given by
\[
\Delta E_0=\langle \Phi_0|\left(\hat{H}_I+\hat{H}_I\frac{\hat{Q}}{W_0-\hat{H}_0}\hat{H}_I+
\hat{H}_I\frac{\hat{Q}}{W_0-\hat{H}_0}\hat{H}_I\frac{\hat{Q}}{W_0-\hat{H}_0}\hat{H}_I+\dots\right)|\Phi_0\rangle_C.
\]


